%-----------------------------------------
% Note: Use pdflatex to process this file.
%-----------------------------------------

%\documentclass{article}
\documentclass{hitec}
\usepackage{color,soul}

\usepackage{setspace}
\usepackage{graphicx}
\usepackage{moreverb}    % Defines {listing} environment.
\usepackage{amsmath, amsthm, amssymb, amsbsy, mathtools}
\usepackage{alltt}
\usepackage{rotating}
\usepackage{subcaption}
\usepackage{toc-bmad}
\usepackage{xspace}
\usepackage[section]{placeins}   % For preventing floats from floating to end of chapter.
\usepackage{longtable}   % For splitting long vertical tables into pieces
\usepackage{index}
\usepackage{multirow}
\usepackage{booktabs}    % For table layouts
\usepackage{yhmath}      % For widehat
\usepackage{xcolor}      % Needed for listings package.
\usepackage{listings}
\usepackage[T1]{fontenc}   % so _, <, and > print correctly in text.
\usepackage[strings]{underscore}    % to use "_" in text
\usepackage[pdftex,colorlinks=true]{hyperref}   % Must be last package!

\definecolor{light-gray}{gray}{0.95}
\lstset{backgroundcolor=\color{light-gray}}
\lstset{xleftmargin=0cm}
\lstset{framexleftmargin=0.3em}
%\lstset{basicstyle = \ttfamily\fontsize{11}{11}\selectfont} 
\lstset{basicstyle = \small}
\lstnewenvironment{code}{}{}

%---------------------------------------------------------------------------------

\definecolor{lightestgray}{gray}{0.99}
\sethlcolor{lightestgray}
\soulregister{\texttt}{1}
\newcommand\dottcmd[1]{\hl{\em#1}\endgroup}
%\newcommand\dottcmd[1]{{#1}\endgroup}
\newcommand{\vn}{\begingroup\catcode`\_=11 \catcode`\%=11 \dottcmd}
\newcommand{\da}{\vn{dynamic_aperture}\xspace}
\newcommand{\Newline}{\hfil \\}
\newcommand{\sref}[1]{$\S$\ref{#1}}
\newcommand{\Th}{$^{th}$\xspace}

%---------------------------------------------------------------------------------

\renewcommand{\textfraction}{0.1}
\renewcommand{\topfraction}{1.0}
\renewcommand{\bottomfraction}{1.0}

\settextfraction{0.9}  % Width of text
\setlength{\parindent}{0pt}
\setlength{\parskip}{1ex}
%\setlength{\textwidth}{6in}
\newcommand{\Section}[1]{\section{#1}\vspace*{-1ex}}

\newenvironment{display}
  {\vspace*{-1.5ex} \begin{alltt}}
  {\end{alltt} \vspace*{-1.0ex}}

%---------------------------------------------------------------------------------

\title{Dynamic_Aperture Program}
\author{}
\date{David Sagan \\ April 22, 2022}
\begin{document}
\maketitle

\tableofcontents

%---------------------------------------------------------------------------------
\Section{Introduction} 
\label{s:intro}

The \da program is for measuring the dynamic aperture. The concept of \vn{dynamic aperture} is that
a particle reaching a certain amplitude will quickly be resonantly driven to large amplitude where
it is lost. This amplitude where the particle becomes unstable is the dynamic aperture. This is to
be contrasted by the \vn{physical aperture} which is the aperture where a particle stricks the wall
of the beam chamber. The general idea in designing lattices is to make sure that the dynamic
aperture is large enough so that, in the normal course of events, particles in the beam have a very
small probability of getting lost due to their amplitude exceeding the dynamic aperture. For long
term stability, a common rule of thumb is to design lattices such that the dynamic aperture is 10 times
the beam sigma.\footnote
  {
While this might seem excessive, this rule of thumb gives some safety margin which is desireable
since designs are never exact.
  }
For injection studies, the minimum dynamic aperture will be determined in part by the size of the
injected beam. In any case, if the dynamic aperture is larger than the physical aperture, increasing
the dynamic aperture further will not help beam stability. 

If there are no apertures set in lattice used by the \da program, the calculated aperture will be the
dynamic aperture. If apertures are set in the lattice, the calculated aperture will be the minimum of
the dynamic and physical apertures.

The \da program is built atop the Bmad software toolkit \cite{b:bmad}. The Bmad toolkit is a
library, developed at Cornell, for the modeling relativistic charged particles in storage rings and
Linacs, as well as modeling photons in x-ray beam lines.

The \da program comes with the ``Bmad Distribution'' which is a package which contains Bmad along with
a number of Bmad based programs. See the Bmad web site for more details.

If the Bmad Distribution is compiled with \vn{OpenMP} enabled (see the documentation on the Bmad
Distribution ``Off-Site'' setup for more details), the \da program can be run parallel. With OpenMP
the computation load is distributed over a number of cores on the machine you are using. To set the
number of cores set the \vn{OMP_NUM_THREADS} environment variable. Example:
\begin{code}
export OMP_NUM_THREADS=8
\end{code}
And run the program as normal.

%------------------------------------------------------------------
\Section{Running the Dynamic Aperture Program} 
\label{s:run}

See the documentation for setting up the Bmad environment variables at
\begin{code}
  https://wiki.classe.cornell.edu/ACC/ACL/RunningPrograms
\end{code}

Once the Bmad environment variables have been set, the syntax for invoking the single threaded
version of the \da program is:
\begin{code}
  dynamic_aperture {<master_input_file_name>}
\end{code}
Example:
\begin{code}
  dynamic_aperture my_input_file.init
\end{code}
The \vn{<master_input_file_name>} optional argument is used to set the master input file name. The
default value is ``\vn{dynamic_aperture.init}''. The syntax of the master input file is explained
in \sref{s:input}.

Example input files are in the directory (relative to the root of a Distribution):
\begin{code}
  bsim/dynamic_aperture/example
\end{code}

%------------------------------------------------------------------
\Section{Time Ramping --- Time Varying Element Parameters}
\label{s:ramp}

%------------------------------------------------------------------
\Section{Time Ramping --- Time Varying Element Parameters}
\label{s:ramp}

``\vn{Ramping}'' is the situation where lattice parameters are changing as a function of
time. Ramping examples include changing magnet and RF strengths to ramp the beam energy or changing
magnet strengths to squeeze beta at the interaction pont of a colliding beam machine.

Ramping is accomplished by defining \vn{ramper} elements in the lattice file and setting
\vn{ramping_on} to True in the master input file (\sref{s:input}). Ramper elements will be applied
to each lattice element in turn before particles are tracked through them. Restriction: Only those
ramper elements that have \vn{time} as the first variable will be used. See the Bmad manual for
documentation on \vn{ramper} syntax.

Example:
\begin{code}
  ramp_e: ramper = \{*[e_tot]:\{4e+08, 4.00532e+08, 4.01982e+08, ...\}\},
                var = \{time\}, x_knot = \{0, 0.001, 0.002, ...\}

  amp = 1e9;  omega = 0.167;  t0 = 0.053
  ramp_rf: ramper = \{rfcavity::*[voltage]:amp*sin(omega *(time + t0)),
        rfcavity::*[phi0]:0.00158*time^2 + 2*q \}, var = \{time, q\}
\end{code}
The ``\vn{*[e_tot]}'' construct in the definition of \vn{ramp_e} means that the ramper will be
applied all elements (since the wild card character ``\vn{*}'' will match to any element name), and
it is the element's \vn{e_tot} attribute (the element's reference energy) that will be varied.

In the above example, the \vn{ramp_rf} ramper will be applied to all \vn{rfcavity} elements with
the cavity voltage and phase (\vn{phi0}) being varied.

Important! Only rampers that use \vn{time} as the variable name will be directly varied in the
tracking.

In the case where the reference energy \vn{e_tot} or reference momentum \vn{p0c} is being varied, the
effect on an element will depend upon the setting of the element's \vn{field_master} parameter. For
example:
\begin{code}
  q1: quadrupole, k1 = 0.3
  q2: quadrupole, k1 = 0.3, field_master = T
\end{code}
In this example, \vn{q1} will have its \vn{field_master} parameter set to \vn{False} since the
quadrupole strength was specified using the normalized strength \vn{k1}. With \vn{q1}, since
\vn{field_master} is False, varying the reference energy or momentum will result in the normalized
strength \vn{k1} remaining fixed and the unnormalized strength \vn{B1_gradient} varying in
proportion to the reference momentum. With \vn{q2}, since \vn{field_master} is True, the
unnormalized strength \vn{B1_gradient} will remain fixed and normalized \vn{k1} will vary
inversely with the reference momentum.

Before a simulation, individual ramper elements may be toggled on or off by setting the element's
\vn{is_on} attribute in the lattice file: 
\begin{code}
  ramp_rf: ramper = ...  ! Ramper element defined.
  ramp_rf[is_on] = F     ! After being defined, ramper may be turned off.
\end{code}

%------------------------------------------------------------------
\Section{Fortran Namelist Input}
\label{s:namelist}

Fortran namelist syntax is used for parameter input in the master input file. The general form of a namelist is
\begin{code}
&<namelist_name>
  <var1> = ...
  <var2> = ...
  ...
/
\end{code}
The tag \vn{"\&<namelist_name>"} starts the namelist where \vn{<namelist_name>} is the name of the
namelist. The namelist ends with the slash \vn{"/"} tag. Anything outside of this is ignored. Within
the namelist, anything after an exclamation mark \vn{"!"} is ignored including the exclamation
mark. \vn{<var1>}, \vn{<var2>}, etc. are variable names. Example:
\begin{code}
&place 
  section = 0.0, "arc_std", "elliptical", 0.045, 0.025 
/
\end{code}
here \vn{place} is the namelist name and \vn{section} is a variable name.  Notice that here
\vn{section} is a ``structure'' which has five components -- a real number, followed by two strings,
followed by two real numbers.

Everything is case insensitive except for quoted strings.

Logical values are specified by \vn{True} or \vn{False} or can be abbreviated \vn{T} or
\vn{F}. Avoid using the dots (periods) that one needs in Fortran code.

%------------------------------------------------------------------
\Section{Master Input File}
\label{s:input}

The \vn{master input file} holds the parameters needed for running the \da program. The master input
file must contain a single namelist (\sref{s:namelist}) named \vn{params}.  Example:
\begin{code}
&params
  lat_file   =  "lat.bmad"     ! Bmad lattice file
  ramping_on = False
  ramping_start_time = 0
  dat_file = "da.dat"
  set_rf_off = False
  dpz = 0.000, 0.005, 0.010
  bmad_com
  da_param%min_angle = 0
  da_param%max_angle = 3.1415926
  da_param%n_angle = 0
  da_param%n_turn = 2000
  da_param%x_init = 1e-3
  da_param%y_init = 1e-3
  da_param%rel_accuracy = 1e-2
  da_param%abs_accuracy = 1e-5
  da_param%start_ele = ''
/
\end{code}

Parameters in the master input file that are:
\begin{description}
\item[lat_file] \Newline
Name of the Bmad lattice file to use. This name is required.
%
\item[bmad_com\%...] \Newline
The \vn{bmad_com} structure contains various parameters that affect tracking. For example, whether
radiation damping and fluctuations are included in tracking. A full list of \vn{bmad_com} parameters
is detailed in the Bmad reference manual. Note: \vn{bmad_com} parameters can be set in the Bmad
lattice file as well. \vn{Bmad_com} parameter set in the master input file will take precedence over
parameters set in the lattice file.
%
\item[ele_start] \Newline
Name or element index of the element to start the tracking. Examples:
\begin{code}
  ele_start = "Q3##2"   ! 2nd element named Q3 in the lattice.
  ele_start = 37        ! 37th element in the lattice.
\end{code}
The default is to start at the beginning of the lattice. Notice that the tracking starts at the
downstream end of the element so the first element tracked through is the element after the chosen
one. Also see \vn{ele_stop}.
%
\item[n_turns] \Newline
Number of turns to track. See Section~\sref{s:track.methods}.
%
\item[ramping_on] \Newline
If set to True, \vn{ramper} control elements will be use to modify the lattice during tracking
(\sref{s:ramp}). Default is False.
%
\item[ramping_start_time] \Newline
The starting (offset) time used to set \vn{ramper} elements. This enables simulations to start in the middle
of a ramp cycle. Default is 0.
%
\item[rfcavity_on] \Newline
If set to \vn{False}, the voltage on all RF cavity elements will be turned off. Default is \vn{True}.
%
\end{description}

%------------------------------------------------------------------
\Section{Data Output}
\label{s:part.out}

The following describes the particle data output format. Particle data is outputted when the
\vn{simulation_mode} is set to \vn{"BEAM"}.

Particle data is always recorded when the beam is at the \vn{ele_start} position independent of
where the lattice begins and ends.

An output data file will be produced every \vn{particle_output_every_n_turns} turns. Each line
in this file records a particle's orbital and spin position. The name of the data file is derived
from the \vn{particle_output_file} string.  If \vn{particle_output_file} contains a hash
character ''\#'', the data file name is formed by substituting the turn number for the hash
token. If there is no hash character, the data file name is formed by appending the turn number to
the \vn{particle_output_file} string. If there are multiple bunches, instead of using the turn
number, the substituted string will be of the form
\begin{code}
  {bunch_index}-{turn_number}
\end{code}

Nominally particle data is recorded every \vn{particle_output_every_n_turns} number of
turns. However, if \vn{particle_output_every_n_turns} is set to 0, particle data is recorded
only at the start and end of the tracking. If \vn{particle_output_every_n_turns} is set to -1,
particle data is only recorded at the end of tracking.

%------------------------------------------------------------------
\Section{Plotting}
\label{s:plot}

%------------------------------------------------------------------
\begin{thebibliography}{9}

\bibitem{b:bmad}
D. Sagan,
``Bmad: A Relativistic Charged Particle Simulation Library''
Nuc.\ Instrum.\ \& Methods Phys.\ Res.\ A, {\bf 558}, pp 356-59 (2006).

\end{thebibliography}

\end{document}

