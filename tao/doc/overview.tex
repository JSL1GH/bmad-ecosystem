\chapter{Overview}
\label{c:overview}

%------------------------------------------------------------------------
\section{In the beginning...}
\label{s:overview}


\tao stands for ``Tool for Accelerator Optics''. \tao is a general purpose program for simulating
high energy particle beams in accelerators and storage rings. This manual assumes you are already
familiar with the basics of particle beam dynamics and its formalism. There are several books that
introduce the topics very well. A good place to start is, for example, \textit{The Physics of
Particle Accelerators} by Klaus Wille\cite{b:wille}.

\index{bmad}
The simulation engine that \tao uses for doing such things as particle tracking is the \bmad
software library\cite{b:bmad}. \bmad was developed as an object-oriented library so that common
tasks, such as reading in a lattice file and particle tracking, did not have to be coded from
scratch every time someone wanted to develop a program to calculate this, that or whatever.  An
understanding of the nitty-gritty details of the routines that comprise \bmad is not necessary,
however, one should be familiar with the conventions that \bmad uses and this is covered in the
\bmad manual.

\bmad was developed before \tao. As \bmad was being developed, it became apparent that many
simulation programs had common needs: For example, plotting data, viewing machine parameters,
etc. Because of this commonality, the \tao program was developed to reduce the time needed to
develop a working programs without sacrificing flexibility. That is, while the ``vanilla'' version
of the \tao program is quite a powerful simulation tool, \tao has been designed to be easily
customizable so that extending \tao to solve new and different problems is relatively straight
forward.

So, what is \tao good for? A large variety of applications: Single and multiparticle tracking,
lattice simulation and analysis, lattice design, machine commissioning and correction,
etc. Furthermore, it is designed to be extensible using interface ``hooks'' built into the program.
This versatility has been used, for example, to enable \tao to directly read in measurement data
from Cornell's Cesr storage ring and Jefferson Lab's FEL. Think of \tao as an accelerator design and
analysis environment. But even without any customizations, \tao will do much analysis.

More information, including the most up--to--date version of this manual, can be found at the \bmad
web site at:
\begin{example}
  classe.cornell.edu/bmad
\end{example}

Errors and omissions are a fact of life for any reference work and comments from you, dear reader,
are therefore most welcome. Please send any missives (or chocolates, or any other kind of
sustenance) to:
\begin{example}
  David Sagan <dcs16@cornell.edu>
\end{example}

It is my pleasure to express appreciation to people who have contributed to this effort. To Scott
Berg, Michael Ehrlichman, Chris Mayes, and Jeff Smith for bug reports, suggestions, code
improvements, Etc. To John Mastroberti and Kevin Kowalski for their work on a graphics user
interface and associated plotting, And last but not least thanks also must go to Dave Rubin and
Georg Hoffstaetter for their help, support, and patience.

%------------------------------------------------------------------------
\section{Bmad and Tao Tutorial}
\label{s:tutorial}

\vspace{0.1in} The \tao manual is organized as reference guide and so does not do a good job of
instructing the beginner as to how to use \tao. For this there is an introduction and tutorial on
\bmad and \tao concepts that can be downloaded from the \bmad web page. Go to either the \bmad
manual or the \tao manual page and there will be a link for the tutorial.

%------------------------------------------------------------------------
\section{Tao Examples}
\label{s:examples}

Example input files for running a number of different simulations can be found with the \bmad and \tao
Tutorial (\sref{s:tutorial}). Additionally, there are a number of examples in the directory
\begin{example}
  \$ACC_ROOT_DIR/bmad-doc/tao_examples/
\end{example}
where \vn{\$ACC_ROOT_DIR} is the base directory of your local \bmad Distribution or Release (a full
description of \bmad Distributions and Releases is given in the \bmad and \tao Tutorial).

%------------------------------------------------------------------------
\section{Manual Organization}

This manual is divided into two parts. Part I is the reference section which defines the terms used
by \tao, discusses \tao commands, etc. Part II is a programmer's guide which shows how to extend
\tao's capabilities including interfacing to Python and how to incorporate custom calculations into
\tao.

%------------------------------------------------------------------------
\section{Other Bmad based programs}
\label{s:other}

\tao is not the only program based upon \bmad. There are other programs which are specialized for
certain computations such as long-term tracking, beam break-up instability, multi-objective
optimization, etc. The discussion of these programs is beyond the scope of this manual. More
information along with instructions for obtaining the programs can be obtained from the \bmad web site (\sref{s:overview}).

