\chapter{Introduction}
\label{c:introduction}

%----------------------------------------------------------------
\section{Obtaining Tao}
\index{tao!Obtaining}
\label{s:obtaining}

A \vn{Distribution} is a set of files, including \bmad and \tao source files, which are used to
build the \bmad, the \tao program, and various other simulation programs. A \vn{Release} is like a
\vn{Distribution} except that it is created on the Linux computer system at CLASSE (Cornell's
Laboratory for Accelerator-based Sciences and Education). More information can be obtained from the
\bmad web site. 

If there is no local \bmad Guru to guide you, download and setup instructions for downloading a
Distribution, environment variable setup, and building \tao is contained on the \bmad web
site and will not be covered here.

%----------------------------------------------------------------
\section{Starting and Initializing Tao}
\index{initializing!files}
\label{s:initializing}

The syntax for starting \tao is given in \Sref{s:command.line}.

Initialization occurs when \tao is started. Initialization information is stored in one or more
files as discussed in Chapter \sref{c:init}.

%%----------------------------------------------------------------
\section{Running Tao with OpenMP}
\index{openMP}
\label{s:openmp}

\vn{OpenMP} is a standard that enables programs to run calculations with multiple threads which will
reduce computation time. Certain calculations done by \tao, including beam tracking and dynamic
aperture calculations, can be run multithreaded via OpenMP if the \tao executable file has been
properly compiled.  Interested users should consult their local \bmad Guru for guidance. Note:
\vn{OpenMP} multithreading involves using multiple cores of a single machine (unlike \vn{Open MPI}
which involves multiple machines). Therefore, it is not necessary to have a cluster of machines to
use \vn{OpenMP}.

To set the number of threads when running a program compiled with \vn{OpenMP}, set the environment variable
\vn{OMP_NUM_THREADS}. Example:
\begin{example}
  export OMP_NUM_THREADS=8
\end{example}

To the local \bmad Guru: Compiling and linking of \tao with \vn{OpenMP} is documented on the \bmad
web site. By default, \vn{OpenMP} is not enabled. Essentially, OpenMP is enabled by modifying
the \vn{dist_prefs} file before compiling and linking.

%----------------------------------------------------------------
\section{Command Line Mode and Single Mode}
\label{s:modes}

After \tao is initialized, \tao interacts with the user though the command line. \tao has two modes
for this. In \vn{command line} mode, which is the default mode, \tao waits until the the \vn{return}
key is depressed to execute a command. Command line mode is described in Chapter~\sref{c:command}. 

In \vn{single} mode, single keystrokes are interpreted as commands. \tao can be set up so that in
\vn{single mode} the pressing of certain keys increase or decrease variables. While the same effect
can be achieved in the standard \vn{line mode}, \vn{single mode} allows for quick adjustments of
variables. See Chapter~\sref{c:single} for more details.

%-----------------------------------------------------------------
\section{Lattice Calculations}
\index{lattice calculaitons}
\label{s:lat.calc.overview} 

By default \tao recalculates lattice parameters and does tracking of particles after each command.
The exception is for commands that do not change any parameter that would affect such calculations
such as the \vn{show} command. See \sref{s:lat.calc} for more details. If the recalculation takes a
significant amount of time, the recalculation may be suppressed using the \vn{set global
lattice_calc_on} command (\sref{s:set.global}) or the \vn{set universe} command
(\sref{s:set.universe}).

%-----------------------------------------------------------------
\section{Command Files and Aliases}
\index{command files}
\label{s:command.files} 

Typing repetitive commands in command line mode can become tedious. \tao has two constructs to
mitigate this: Aliases and Command Files. 

Aliases are just like aliases in Unix. See Section~\sref{s:alias} for more details.

Command files are like Unix shell scripts. A series of commands are
put in a file and then that file can be called using the \vn{call}
command (\sref{s:call}).

\tao will call a command file at startup. The default name of this startup file is \vn{tao.startup}
but this name can be changed (\sref{s:format}).

Do loops (\sref{s:do}) are allowed with the following syntax:
\begin{example}
  do <var> = <begin>, <end> \{, <step>\} 
    ...
    tao command [[<var>]]
    ...
  enddo
\end{example}
The \vn{<var>} can be used as a variable in the loop body but must be
bracketed ``[[<var>]]''.  The step size can be any integer positive or
negative but not zero.  Nested loops are allowed and command files can
be called within do loops.

\begin{example}
  do i = 1, 100
    call set_quad_misalignment [[i]] ! command file to misalign quadrupoles
    zero_quad 1e-5*2^([[i]]-1) ! Some user supplied command to zero quad number [[i]]
  enddo
\end{example}

To reduce unnecessary calculations, the logicals \vn{global%lattice_calc_on}
and \vn{global%plot_on} can be toggled from within the command file. Also 
setting \vn{global%quiet} can turn off verbose output to the terminal. Example
\begin{example}
  set global quiet = all          ! Turn off verbose output to the terminal.
  set global lattice_calc_on = F  ! Turn off lattice calculations
  set global plot_on = F          ! Turn off plot calculations
  ... do some stuff ...
  set global plot_on = T          ! Turn back on 
  set global lattice_calc_on = T  ! Turn back on
  set global quiet = off         
\end{example}
See \sref{s:globals} for more details.

A \vn{end_file} command (\sref{s:end.file}) can be used to signal the
end of the command file.

The \vn{pause} command (\sref{s:pause}) can be used to temporarily
pause the command file.

