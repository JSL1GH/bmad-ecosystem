\documentclass[11pt]{article}
\usepackage{tocloft}
\usepackage{geometry}            % See geometry.pdf to learn the layout options. There are lots.
\usepackage{xspace}
\geometry{letterpaper}           % ... or a4paper or a5paper or ... 
%\usepackage[parfill]{parskip}   % To begin paragraphs with an empty line rather than an indent
\usepackage{graphicx}
\usepackage{amssymb}
\usepackage{amsmath}
\usepackage{alltt}
\usepackage[T1]{fontenc}   % so _, <, and > print correctly in text.
\usepackage[strings]{underscore}    % to use "_" in text
\usepackage[pdftex,colorlinks=true]{hyperref}

%---------------------------------------------------------------------------------

\newcommand{\sref}[1]{$\S$\ref{#1}}
\newcommand{\srthree}{\texttt{Synrad3D}\xspace}
\newcommand\ttcmd{\begingroup\catcode`\_=11 \catcode`\%=11 \dottcmd}
\newcommand\dottcmd[1]{\texttt{#1}\endgroup}
\newcommand{\Begineq}{\begin{equation}}
\newcommand{\Endeq}{\end{equation}}
\newcommand{\fig}[1]{Figure~\ref{#1}}
\newcommand{\vn}{\ttcmd}           
\newcommand{\Th}{$^{th}$\xspace}
\newcommand{\Newline}{\hfil \\}


\newcommand{\bearray}{\begin{eqnarray}}
\newcommand{\eearray}{\end{eqnarray}}
\newcommand{\be}{\begin{equation}}
\newcommand{\ee}{\end{equation}}
\newcommand{\bearraynn}{\begin{eqnarray*}}
\newcommand{\eearraynn}{\end{eqnarray*}}
\newcommand{\benn}{\begin{displaymath}}
\newcommand{\eenn}{\end{displaymath}}
\newcommand{\eq}[1]{{Eq.~(\ref{#1})}}
\newcommand{\eqs}[2]{{Eqs.~(\ref{#1}--\ref{#2})}}

\newlength{\dPar}
\newlength{\ExBeg}
\newlength{\ExEnd}
\setlength{\dPar}{1.5ex}
\setlength{\ExBeg}{-\dPar}
\addtolength{\ExBeg}{-0.5ex}
\setlength{\ExEnd}{-\dPar}
\addtolength{\ExEnd}{-0.0ex}

\newenvironment{example}
  {\vspace{\ExBeg} \begin{alltt}}
  {\end{alltt} \vspace{\ExEnd}}

%---------------------------------------------------------------------------------

\setlength{\textwidth}{6.25in}
\setlength{\hoffset}{0.0in}
\setlength{\oddsidemargin}{0.25in}
\setlength{\evensidemargin}{0.0in}
\setlength{\textheight}{8.5in}
\setlength{\topmargin}{0in}

\setlength{\parskip}{\dPar}
\setlength{\parindent}{0ex}

\setlength\cftparskip{0pt}
\setlength\cftbeforesecskip{3pt}
\setlength\cftaftertoctitleskip{15pt}

%---------------------------------------------------------------------------------

\title{ibs_ring Simulation Program}
\author{Michael Ehrlichman, D. Sagan}
\date{February 28, 2014}

%---------------------------------------------------------------------------------

\begin{document}
\maketitle

\pdfbookmark[1]{Contents}{Contents}
\tableofcontents

%------------------------------------------------------------------
\section{Introduction} 
\label{s:intro}

\vn{ibs_ring} is a program for simulating intra-beam scattering (IBS) in a storage ring.
The source code for this program lives in the \vn{bsim} directory in
the standard Bmad\cite{b:bmad} distribution.

Many formulas for calculating IBS rates are implemented.  They are:
\begin{enumerate}
\item Modified Piwinski with Zotter's integral.
\item Modified Piwinski with constant Coulomb Log.
\item Completely Integrated Modified Piwinski (CIMP).
\item Bjorken \& Mtingwa's formula.
\item Bane's approximation of Bjorken \& Mtigwa's formula.
\item Kubo \& Oide's generalization of Bjorken \& Mtingwa's formula.
\end{enumerate}
These formula's are described in the following section.

\section{Methods for calculating IBS growth rates}

\subsection{Modified Piwinski with Zotter's integral}
This formula is selected by setting {\tt ibs_formula=`mpzt'}.

Piwinski's original formula for calculating IBS rates is Ref.~\cite{b:pw}.  The
original formula contained a numerically difficult triple integral.  In
Ref.~\cite{b:zotter}, the triple integral is reduced to a single integral that is 
much easier to evaluate.  No 
approximations are applied to obtain the single integral, it is exact.

The original formula included Twiss $\beta$ and dispersion $\eta$, but neglicted
the derivatives of the lattice functions Twiss $\alpha$ and $\eta'$.
In Ref.~\cite{b:martini}, the derivatives of the lattice functions are included.
This form of Piwinski's original formula with derivatives of the lattice
functions is usually called ``Modified Piwinski''.

\subsection{Modified Piwinski wiht constant Coulomb Log}
This formula is selected by setting {\tt ibs_formula=`mpxx'}.

Piwinski's origional IBS formula typically gives a considerablly different growth rate than
Bjorken \& Mtingwa's, Kubo \& Oide's, and approximations of Piwinski.  This is because
Piwinski's formula is unique in how it treats the Coulomb Logarithm.

In the {\tt `mpxx'} IBS formula, Piwinski's formula has been rederived and the
log term pulled out of the integral.  This allows the Coulomb Logarithm to be treated the
same as it is in other IBS formulas.  This derivation is available in Ref.~\cite{b:ehr-thesis}.

\subsection{Completely Integrated Modified Piwinski (CIMP)}
This formula is selected by setting {\tt ibs_formula=`cimp'}.

In Ref.~\cite{b:wolski}, a high energy approximation of the Modified Piwinski formula is 
obtained.  This formula contains one integral which can be easily easily and quickly tabulated.
This formula is very fast and, for ILC and CesrTA, returns growth rates similar to those obtained
from the more general IBS formulas.

\subsection{Bjorken \& Mtingwa's formula}
This formula is selected by setting {\tt ibs_formula=`cimp'}.

In Ref.~\cite{b:bjmt}, the authors take a distinct approach to calculating IBS growth rates.
In Ref.~\cite{b:bane-comp}, Bjorken \& Mtingwa's formula is compared to
the Modified Piwinski formula.  It is found that, for high energy beams, after some
modifications to Piwinski's formula, the two formulas are algebraically similar
and give similar results.

\subsection{Bane's approximation of Bjorken \& Mtingwa's formula}
This formula is selected by setting {\tt ibs_formula=`bane'}.

In Ref.~\cite{b:bane-approx}, a high energy approximation of Bjorken \& Mtingwa's formula
is obtained.  The formula is simpler and numerically easier to evaluate, but does
not give sensible results when vertical dispersion is zero.

\subsection{Kubo \& Oide's generalization of Bjorken \& Mtingwa's formula}
This formula is selected by setting {\tt ibs_formula=`kubo'}.

In Ref.~\cite{b:kubo}, a generalization of Bjorken \& Mtingwa's formula
is derived.  This formula is unique in that it is based on the $6\times6$
matrix of the second order moments of the beam distribution (the beam sigma matrix),
rather than on Twiss parameters.

This formula should be able to handle arbitrary coupling conditions, though
that has not been tested in experiment.

\section{Methods for determining equilibrium emittances}

\subsection{Derivatives method}
This method is selected by setting {\tt eqb_method=`der'}.

This method finds the equilibrium beam size using differential equations to 
evolve the emittance through time.  The time step is hard coded to $\frac{\tau}{10}$,
where $\tau$ is the horizontal damping time.

The differential equations are,
\begin{align}
\frac{d\epsilon_a}{dt}&= -\left(\epsilon_a-\epsilon_{a0}\right)\frac{2}{\tau_a}+\epsilon_a\frac{2}{T_a}\label{e:da}\\
\frac{d\epsilon_b}{dt}&= -\left(\epsilon_b-\epsilon_{b0}\right)\frac{2}{\tau_b}+\epsilon_b\frac{2}{T_b}\label{e:db}\\
\frac{d \sigma_p}{dt}&= -\left(\sigma_p-\sigma_{p0}\right)\frac{1}{\tau_z}+\sigma_p\frac{1}{T_z}\label{e:dp},
\end{align}
where $\epsilon_{a0}$, $\epsilon_{b0}$, and $\sigma_{p0}$ are the zero-current emittances and energy spread,
$\tau_a$, $\tau_b$, and $\tau_z$ are the damping times, and $T_a$, $T_b$, and $T_z$ are the IBS
growth rates given the IBS formulas.  The factors of $2$ come about because $\tau_a$, $\tau_b$,
$T_a$, and $T_b$ are betatron growth rates.  i.e.  they are the rates of change the beam sizes,
rather than the emittances.

\subsection{Relaxation of equilibrium equations}
This method is selected by setting {\tt eqb_method=`rlx'}.

The solutions to differential equations (\ref{e:da}), (\ref{e:db}), and (\ref{e:dp}) are,
\begin{align}
\epsilon_a&=\frac{1}{1-\frac{\tau_a}{T_a}}\epsilon_{a0}\label{e:eea}\\
\epsilon_b&=\frac{1}{1-\frac{\tau_b}{T_b}}\epsilon_{b0}\label{e:eeb}\\
\sigma_p&=\frac{1}{1-\frac{\tau_z}{T_z}}\sigma_{p0}\label{e:eep}.
\end{align}
Note that $T_a$, $T_b$, and $T_z$ are functions of $\epsilon_a$, $\epsilon_b$, and $\sigma_p$.

In Ref.~\cite{b:wolski}, a method for approximating the effect of coupling replaces Eqn.~\ref{e:eeb}
with,
\begin{equation}
\epsilon_b=\left(\left(1-r_\epsilon\right)\frac{1}{1-\frac{\tau_b}{T_b}}+
r_\epsilon\frac{1}{1-\frac{\tau_a}{T_a}}\right)\epsilon_{b0},
\end{equation}
where $r_\epsilon$ describes the amount of vertical emittance that is due to transverse mode coupling.
$0<r_\epsilon<1$.
$r_\epsilon=0$ describes a situation where there is no mode coupling and $\epsilon_{b0}$ is
determined entirely by physics in the vertical plane.  $r_\epsilon=1$ describes a situation
where $\epsilon_{b0}$ is determined entirely by coupling from the horizontal plane.

In the {\tt ibs\_ring} simulation $r_\epsilon$ is set by {\tt ratio}.


%------------------------------------------------------------------
\section{Running ibs_ring} 
\label{s:run}

The input for the \vn{ibs_ring} program uses Fortran90 namelist
syntax: The data begins with the string \vn{\&parameters} and ends
with a slash \vn{/}. Everything outside this is ignored.
The input parameters are:
\begin{example}
\&parameters
  lat_file = <lattice-file-name>
  ibs_formula = <type>  ! 'cimp', 'bjmt', 'bane', 'mpzt', 'mpxx', or 'kubo'
  eqb_method = <type>   ! 'der' or 'rlx'
  clog_to_use = <int>   ! 1=no tail cut, 2=raubenheimer, 3=kubo, 4=kubo w/vertical
  inductance = <real>
  resistance = <real>
  set_dispersion = <logical> ! applied only for kubo method.
  eta_set = <real>
  etap_set = <real>
  b_emit = <real>
  a_emit = <real>
  energy_spread = <real>
  ratio = <real>
  granularity = <real>   ! -1 for element-by-element.
  x_view = <int>  ! element lattice index 
  y_view = <int>  
  z_view = <int>  
  mA_per_bunch = <real>  ! highest current
  stop_mA = <real>       ! lowest current
  delta_mA = <real>      ! step size
/
\end{example}

  \begin{description}
  \item[\vn{lat_file}] \Newline
Bmad lattice file describing the ring.

  %------------------------------
  \item[\vn{ibs_formula}] \Newline
Name of the algorithm to use for the calculation. Possibilities are:
\begin{example}
  'cimp'
  'bjmt'
  'bane'
  'mpzt'
  'mpxx'
  'kubo'
\end{example}

  %------------------------------
  \item[\vn{eqb_method}] \Newline
Method used for finding the equilibrium solution. Possibilities are:
\begin{example}
  'der'
  'rlx'
\end{example}
'der' finds the equilibrium emittances using differential equations.
The differential is with respect to time.

'rlx finds the equilibrium emittances by iterating to find the
solution to an analytic form for the equilibrium emittance.  The 'rlx'
method includes the controversial 'coupling parameter' which says that
the vertical emittance has contributions from both the vertical and
horizontal IBS rates and damping rates.

In the limit that the coupling parameter is zero, 'der' and 'rlx' are
equivalent.  In fact, the equations used for 'rlx' are the equilibrium
solution to the 'der' method's differential equations.

'der' is much faster and robust.  However, it does not allow for the
'coupling parameter', which has its uses.  Also, the two different
methods can be useful in diagnosing the code.

  %------------------------------
  \item[\vn{clog_to_use}] \Newline
Logarithmic cutoff to use. Possibilities are:
\begin{example}
  1  ! Classic, no tail cut.  
  2  ! Raubenheimer.  
  3  ! Oide
  4  ! Bane.  
\end{example} 

  %------------------------------
  \item[\vn{inductance}] \Newline
Longitudinal inductance for PWD calc.
Effects bunch length vs. current.

  %------------------------------
  \item[\vn{resistance}] \Newline
Resistive inductance for PWD calc.
Currently not used. 

  %------------------------------
  \item[\vn{set_dispersion}] \Newline
If true, then apply eta_set and etap_set.
If false, then do not.

  %------------------------------
  \item[\vn{eta_set}] \Newline
Used only if ibs_formula set to 'kubo'.
Applies x-pz coupling to each
element of lattice when calculating IBS rates.

  %------------------------------
  \item[\vn{etap_set}] \Newline
Used only if ibs_formula set to 'kubo'. Applies px-pz coupling to
each element of lattice when calculating IBS rates.

  %------------------------------
  \item[\vn{a_emit}] \Newline
Zero current horizontal emittance.
Set to -1 for rad int calc.

  %------------------------------
  \item[\vn{b_emit}] \Newline
Zero current vertical emittance.
Set to -1 for rad int calc.

  %------------------------------
  \item[\vn{energy_spread}] \Newline
Zero current energy spread.
Set to -1 for rad int calc.

  %------------------------------
  \item[\vn{ratio}] \Newline
"Coupling parameter r" hack for including coupling.

  %------------------------------
  \item[\vn{granularity}] \Newline
Step size along lattice in meters.
Set to -1 for element-by-element.

  %------------------------------
  \item[\vn{x_view}] \Newline
Index of element where projection is taken for horizontal beam size
calculation.

  %------------------------------
  \item[\vn{y_view}] \Newline
Index of element where projection is taken for vertical beam size
calculation.

  %------------------------------
  \item[\vn{z_view}] \Newline
Index of element where projection is taken for longitudinal beam size
calculation.

  %------------------------------
  \item[\vn{mA_per_bunch}] \Newline
Largest current per bunch in mA.

  %------------------------------
  \item[\vn{stop_mA}] \Newline
Smallest current per bunch in mA.

  %------------------------------
  \item[\vn{delta_mA}] \Newline
mA step size.

  \end{description}

%------------------------------------------------------------------
\section{How to Run Multi-Threaded} 
\label{s:multi}


%------------------------------------------------------------------
\section{Output} 
\label{s:output}


%------------------------------------------------------------------
\begin{thebibliography}{9}

\bibitem{b:bmad}
D. Sagan,
"Bmad: A Relativistic Charged Particle Simulation Library"
Nuc.\ Instrum.\ \& Methods Phys.\ Res.\ A, {\bf 558}, pp 356-59 (2006).
The Bmad web site:
\hfill\break
\hspace*{0.3in} \url{http://www.lepp.cornell.edu/~dcs/bmad}

\bibitem{b:pw}
Piwinski, A., ``Intra-Beam Scattering''.
Proceedings of the 9th International Conference on High Energy Accelerators,
Stanford, CA, 1974. p. 405.

\bibitem{b:martini}
Martini, M., CERN PS/84-9 (AA) (1984).

\bibitem{b:zotter}
Evans, L. and Zotter, B., ``Intrabeam Scattering in the SPS''.
CERN-SPS-80-15, CERN, 1980.

\bibitem{b:wolski}
Kubo, K., Mtingwa, S. K. and Wolski, A., ``Intrabeam Scattering Formulas for 
High Energy Beams,'' Phys. Rev. ST Accel. Beams, 8, 2005.

\bibitem{b:ehr-thesis}
Ehrlichman, M., ``Thesis: Normal Mode Analysis of Single Bunch, Charge Density
Dependent Behavior in Electron/Positron Beams''.  Cornell University (2013).

\bibitem{b:bjmt}
Bjorken, J., Mtingwa, S., ``Intrabeam Scattering'', Particle Accelerators. 13. 
pp. 115-143. (1983).

\bibitem{b:bane-comp}
Bane, K., ``An Accurate, Simplified Model of Intrabeam Scattering'', SLAC-AP-141,
arXiv:physics/0205058. (2002).

\bibitem{b:bane-approx}
Bane, K. in {\it Proceedings of the 8th European Particle Accelerator Conference, Paris,
France, 2002}, p. 1443, (2002).

\bibitem{b:kubo}
Kubo, K., Katsunobu, O., ``Intrabeam Scattering in Electron Storage Rings'',
Phys. Rev. ST Accel. Beams 4, 124401 (2001).

\end{thebibliography}
\end{document}  
