\documentclass[11pt]{article}
%%\documentclass{book}
\usepackage{geometry}            % See geometry.pdf to learn the layout options. There are lots.
\usepackage{xspace}
\geometry{letterpaper}           % ... or a4paper or a5paper or ... 
%\geometry{landscape}            % Activate for for rotated page geometry
%\usepackage[parfill]{parskip}   % To begin paragraphs with an empty line rather than an indent
\usepackage{graphicx}
\usepackage{amssymb}
\usepackage{alltt}
%\usepackage{epstopdf}
%\DeclareGraphicsRule{.tif}{png}{.png}{`convert #1 `dirname #1`/`basename #1 .tif`.png}
\usepackage[T1]{fontenc}   % so _, <, and > print correctly in text.
\usepackage[strings]{underscore}    % to use "_" in text

\usepackage{hyperref}

%---------------------------------------------------------------------------------

\newcommand{\bbup}{\texttt{BBU_PROGRAM}\xspace}
\newcommand\ttcmd{\begingroup\catcode`\_=11 \catcode`\%=11 \dottcmd}
\newcommand\dottcmd[1]{\texttt{#1}\endgroup}
\newcommand{\Begineq}{\begin{equation}}
\newcommand{\Endeq}{\end{equation}}
\newcommand{\fig}[1]{Figure~\ref{#1}}
\newcommand{\vn}{\ttcmd}           
\newcommand{\Th}{$^{th}$\xspace}
\newcommand{\Newline}{\hfil \\}

\newlength{\dPar}
\newlength{\ExBeg}
\newlength{\ExEnd}
\setlength{\dPar}{1.5ex}
\setlength{\ExBeg}{-\dPar}
\addtolength{\ExBeg}{-0.5ex}
\setlength{\ExEnd}{-\dPar}
\addtolength{\ExEnd}{-0.0ex}

\newenvironment{example}
  {\vspace{\ExBeg} \begin{alltt}}
  {\end{alltt} \vspace{\ExEnd}}

%---------------------------------------------------------------------------------

\setlength{\textwidth}{6.25in}
\setlength{\hoffset}{0.0in}
\setlength{\oddsidemargin}{0.25in}
\setlength{\evensidemargin}{0.0in}
\setlength{\textheight}{8.5in}
\setlength{\topmargin}{0in}

\setlength{\parskip}{\dPar}
\setlength{\parindent}{0ex}

%---------------------------------------------------------------------------------

\title{ {\bbup}: BMAD Implementation of Beam Breakup Phenomena}
\author{J.A. Crittenden, D. Sagan}
\date{September 20, 2010}

\begin{document}
\maketitle

%------------------------------------------------------------------
\section{Introduction} 

\bbup is a program which calculates a variety of beam breakup phenomena
given a lattice model. It is implemented in the context of the BMAD library~\cite{ref:bmad}.

%------------------------------------------------------------------------
\section{Simulation technique}
%------------------------------------------------------------------
\subsection{Main input file} 

The main input file can be specified on the command line invoking {\bbup}.
If not given, the default name for the main input file is ``\vn{bbu.init}''.
Example main input file:
\begin{example}
&bbu_params
  bbu_param%lat_file_name = 'erl.lat'         ! Lattice file name
  bbu_param%bunch_freq = 1.3e9                ! Freq in Hz.
  bbu_param%init_particle_offset = 1e-8       ! Initial particle offset for particles born 
                                              ! in the first turn period.
  bbu_param%limit_factor = 2                  ! Init_hom_amp * limit_factor = simulation unstable limit
                                              ! Must be greater than 2
  bbu_param%simulation_turns_max = 11         ! Must be greater than 10
  bbu_param%hybridize = .true.                ! Combine non-hom elements to speed up simulation?
  bbu_param%keep_overlays_and_groups = .false. ! Keep when hybridizing?
  bbu_param%keep_all_lcavities = .false.      ! Keep when hybridizing?
  bbu_param%current = 20e-3                   ! Starting current (amps)
  bbu_param%rel_tol = 1e-2                    ! Final threshold current accuracy.
  bbu_param%write_hom_info = .true.  
  bbu_param%drscan = .false.                  ! If true, do DR scan as in PRSTAB 7 (2004) Fig. 3.
  bbu_param%elname = 'T1'                     ! Element to step length for DRSCAN
  bbu_param%nstep = 50                        ! Number of steps for DRSCAN
  bbu_param%begdr = 5.234                     ! Beginning DR value for DRSCAN
  bbu_param%enddr = 6.135                     ! End DR value for DRSCAN
  bbu_param%use_interpolated_threshold = .true.
  bbu_param%nrep = 10                         ! Number of times to repeat threshold or stable orbit calculation
  bbu_param%ran_seed = 0                      ! Set specific seed if desired (0 uses system clock)
  bbu_param%ran_gauss_sigma_cut = 3           ! If positive, limit ran_gauss values to within N sigma
  bbu_param%stable_orbit_anal = .false.       ! Write out cavity orbit data for SIMULATION_TURNS_MAX turns
                                              ! No thresholds are calculated
                                              ! Repeated NREP times with HOM frequencies re-randomized
                                              ! DRSCAN disabled
/
\end{example}
Fortran namelist input is used.
The namelist begins on the line starting with \vn{"\&bbu_params"}
and ends with the line containing the slash \vn{"/"}. Anything outside
of this is ignored. Within the namelist, anything after an exclamation
mark \vn{"!"} is ignored including the exclamation mark. 

  \begin{description}
  \item[\vn{lat_file_name}] \Newline
The \vn{lat_file_name} parameter specifies the lattice file to be used.
  \item[\vn{write_hom_info}] \Newline
If this parameter is set true, the HOM parameters are written to the main output file.
  \item[\vn{drscan}] \Newline
If enabled, the threshold current is calculated for over the range in return times
corresponding to the range from \vn{begdr} to \vn{enddr} in  \vn{nstep} steps,
lengthening the \vn{elname} element in the lattice accordingly. If the toy lattice
of Ref.~\cite{ref:Hoffstaetter04} is used, then its Fig.~3 is reproduced.
  \item[\vn{nrep}] \Newline
The BBU threshold current calculations are repeated \vn{nrep} times. This is useful
to estimate the effect of randomizing the HOM frequencies, since they are re-randomized for
each calculation.
  \item[\vn{ran_seed}] \Newline
Random number see used in by the random number generator. If set to 0, the system clock
will be used. That is, if set to 0, the output results will vary from run to run. 
  \item[\vn{ran_gauss_sigma_cut}] \Newline
Any randomized values in the lattice, such as HOM frequency spread or position jitter
are limited to a maximum deviation of \vn{ran_gauss_sigma_cut} rms deviations. 
  \item[\vn{stable_orbit_anal}] \Newline
If this parameter is set true, then the output files \tt{stable_orbit.out} and \tt{hom_voltage.out}
are written for each bunch train passage and each repetition, allowing the orbit deviations
and HOM loading to be analyzed as the equilibrium level is reached.

\end{description}
%------------------------------------------------------------------
\section{Output files} 

%------------------------------------------------------------------
\subsection{Main output file}

%------------------------------------------------------------------
\begin{thebibliography}{9}


\bibitem{ref:bmad}
D. Sagan, {\em Bmad: A Relativistic Charged Particle Simulation Library},
Nuc. Instrum. \& Methods {\bf A558}, 356 (2006).
The Bmad manual can be obtained at {\tt http://www.lepp.cornell.edu/{$\sim$}dcs}

\bibitem{ref:Hoffstaetter04} G.H.~Hoffstaetter, I.V.~Bazarov, \emph{
Beam-Breakup Instability Theory for Energy Recovery Linacs},
Phys.~Rev.~ST-AB {\bf 7}, 054401 (2004)

\end{thebibliography}
\end{document}  
