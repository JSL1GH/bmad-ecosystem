* Integer part of tune set by current tune.


%-----------------------------------------
% Note: Use pdflatex to process this file.
%-----------------------------------------

%\documentclass{article}
\documentclass{hitec}
\usepackage{color,soul}

\usepackage{setspace}
\usepackage{graphicx}
\usepackage{moreverb}    % Defines {listing} environment.
\usepackage{amsmath, amsthm, amssymb, amsbsy, mathtools}
\usepackage{alltt}
\usepackage{rotating}
\usepackage{subcaption}
\usepackage{toc-bmad}
\usepackage{xspace}
\usepackage[section]{placeins}   % For preventing floats from floating to end of chapter.
\usepackage{longtable}   % For splitting long vertical tables into pieces
\usepackage{index}
\usepackage{multirow}
\usepackage{booktabs}    % For table layouts
\usepackage{yhmath}      % For widehat
\usepackage{xcolor}      % Needed for listings package.
\usepackage{listings}
\usepackage[T1]{fontenc}   % so _, <, and > print correctly in text.
\usepackage[strings]{underscore}    % to use "_" in text
\usepackage[pdftex,colorlinks=true]{hyperref}   % Must be last package!

\definecolor{light-gray}{gray}{0.95}
\lstset{backgroundcolor=\color{light-gray}}
\lstset{xleftmargin=0cm}
\lstset{framexleftmargin=0.3em}
%\lstset{basicstyle = \ttfamily\fontsize{11}{11}\selectfont} 
\lstset{basicstyle = \small}
\lstnewenvironment{code}{}{}

%---------------------------------------------------------------------------------

\definecolor{lightestgray}{gray}{0.99}
\sethlcolor{lightestgray}
\soulregister{\texttt}{1}
\newcommand\dottcmd[1]{\hl{\em#1}\endgroup}
%\newcommand\dottcmd[1]{{#1}\endgroup}
\newcommand{\vn}{\begingroup\catcode`\_=11 \catcode`\%=11 \dottcmd}
\newcommand{\ts}{\vn{tune_scan}\xspace}
\newcommand{\Newline}{\hfil \\}
\newcommand{\sref}[1]{$\S$\ref{#1}}
\newcommand{\Th}{$^{th}$\xspace}

%---------------------------------------------------------------------------------

\renewcommand{\textfraction}{0.1}
\renewcommand{\topfraction}{1.0}
\renewcommand{\bottomfraction}{1.0}

\settextfraction{0.9}  % Width of text
\setlength{\parindent}{0pt}
\setlength{\parskip}{1ex}
%\setlength{\textwidth}{6in}
\newcommand{\Section}[1]{\section{#1}\vspace*{-1ex}}

\newenvironment{display}
  {\vspace*{-1.5ex} \begin{alltt}}
  {\end{alltt} \vspace*{-1.0ex}}

%---------------------------------------------------------------------------------

\title{Tune Scan Program}
\author{}
\date{David Sagan \\ November 7, 2021}
\begin{document}
\maketitle

\tableofcontents

%---------------------------------------------------------------------------------

\Section{Introduction} 
\label{s:intro}

A \vn{tune scan} is a way of examining resonances which involves varying the beam $Q_a$ and $Q_b$
tunes and observing a resonance related beam parameter such as the beam size or the beam lifetime.
A scan can be done in an actual machine or in simulation which is what \ts program discussed does. 

The \ts program is built atop the Bmad software toolkit \cite{b:bmad}. The Bmad toolkit is a
library, developed at Cornell, for the modeling relativistic charged particles in storage rings and
Linacs, as well as modeling photons in x-ray beam lines.

The \ts program comes with the ``Bmad Distribution'' which is a package which contains Bmad along with
a number of Bmad based programs. See the Bmad web site for more details.

The \ts program comes in two versions: 
\begin{code}
tune_scan      ! Single threaded version
tune_scan_mpi  ! Multi-threaded version
\end{code}
The first is a single threaded version and the second is a multi-threaded version using MPI.

[Note to Distribution maintainers: The single threaded version will be built when compiling a Bmad
Distribution. If MPI is not enabled (which is the default setting), the MPI version will not be built. 
The MPI version can be built by setting:
\begin{code}
  export ACC_ENABLE_MPI=Y
\end{code}
and then using the ``\vn{mk}'' command in the \vn{bsim} directory to build the executable. See the
documentation on the Bmad web site for more details.]

%------------------------------------------------------------------
\Section{Running the Long Term Tracking Program} 
\label{s:run}

See the documentation for setting up the Bmad environmental variables at
\begin{code}
  https://wiki.classe.cornell.edu/ACC/ACL/RunningPrograms
\end{code}

Once the Bmad environmental variables have been set, the syntax for invoking the single threaded
version of the \ts program is:
\begin{code}
  tune_scan {<master_input_file_name>}
\end{code}
Example:
\begin{code}
  tune_scan my_input_file.init
\end{code}
The \vn{<master_input_file_name>} optional argument is used to set the master input file name. The
default value is ``\vn{tune_scan.init}''. The syntax of the master input file is explained
in \sref{s:input}.

Example input files are in the directory (relative to the root of a Distribution):
\begin{code}
  bsim/long_turn_tracking/example
\end{code}

When running the MPI version threaded over multiple computers, the details of how to start the process will
vary depending upon the installation. See your local Guru for details. When running on a single machine, 
the typical command will look like
\begin{example}
  mpiexec -n <num_processes> tune_scan_mpi {<master_input_file_name>}
\end{example}
where \vn{<num_processes>} is the number of processes. 

%------------------------------------------------------------------
\Section{Fortran Namelist Input}
\label{s:namelist}

Fortran namelist syntax is used for parameter input in the master input file. The general form of a namelist is
\begin{code}
&<namelist_name>
  <var1> = ...
  <var2> = ...
  ...
/
\end{code}
The tag \vn{"\&<namelist_name>"} starts the namelist where
\vn{<namelist_name>} is the name of the namelist. The namelist ends
with the slash \vn{"/"} tag. Anything outside of this is
ignored. Within the namelist, anything after an exclamation mark
\vn{"!"} is ignored including the exclamation mark. \vn{<var1>},
\vn{<var2>}, etc. are variable names. Example:
\begin{code}
&place 
  section = 0.0, "arc_std", "elliptical", 0.045, 0.025 
/
\end{code}
here \vn{place} is the namelist name and \vn{section} is a
variable name.  Notice that here \vn{section} is a ``structure'' which
has five components -- a real number, followed by two strings,
followed by two real numbers.

Everything is case insensitive except for quoted strings.

Logical values are specified by \vn{True} or \vn{False} or can be
abbreviated \vn{T} or \vn{F}. Avoid using the dots (periods) that one
needs in Fortran code.

\newpage

%------------------------------------------------------------------
\Section{Master Input File}
\label{s:input}

The \vn{master input file} holds the parameters needed for running the long term tracking
program. The master input file must contain a single namelist (\sref{s:namelist}) named \vn{params}.
Example:
\begin{code}
&params
/
\end{code}

%------------------------------------------------------------------
\begin{thebibliography}{9}

\bibitem{b:bmad}
D. Sagan,
``Bmad: A Relativistic Charged Particle Simulation Library''
Nuc.\ Instrum.\ \& Methods Phys.\ Res.\ A, {\bf 558}, pp 356-59 (2006).

\bibitem{b:ptc}
É. Forest, Y. Nogiwa, and F. Schmidt. The FPP and PTC libraries. In Int.\ Conf.\ Accel.\
Phys pp 17–21, (2006).

\end{thebibliography}

\end{document}

