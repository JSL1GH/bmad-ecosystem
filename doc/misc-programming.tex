\chapter{Miscellaneous Programming}

%-----------------------------------------------------------------------------
\section{Custom Calculations}
\label{s:custom.ele}
\index{custom}

Normally a ``custom'' calculation is a calculation, 
done by \bmad, that is instead handled by having the appropriate \bmad
code call the appropriate custom code. Implementing custom
calculations involve writing custom code and linking this code into a
program. There are essentially two ways to do custom calculations. One
way involves using a \vn{custom} element (\sref{s:custom}). The
other way involves setting the appropriate \vn{method} component of an element
to \vn{custom}. An appropriate method component is one of
\begin{example}
  tracking_method       \sref{s:tkm}
  mat6_calc_method      \sref{s:xfer}
  field_calc            \sref{s:integ}
  aperture_type         \sref{s:limit}
\end{example}

There are eight routines that implement custom calculations:
\begin{example}
  \Hyperref{r:check.aperture.limit.custom}{check_aperture_limit_custom}
  \Hyperref{r:em.field.custom}{em_field_custom}
  \Hyperref{r:init.custom}{init_custom}
  \Hyperref{r:make.mat6.custom}{make_mat6_custom}
  \Hyperref{r:radiation.integrals.custom}{radiation_integrals_custom}
  \Hyperref{r:track1.custom}{track1_custom}
  \Hyperref{r:track1.spin.custom}{track1_spin_custom}
  \Hyperref{r:wall.hit.handler.custom}{wall_hit_handler_custom}
\end{example}
[Use \vn{getf} for more details about the argument lists for these
routines.]  The \bmad library has ``dummy'' routines of the same name to
keep the linker happy when custom routines are not implemented. These
dummy routines, if called, will print an error message and stop the
program. The exception is the dummy \vn{init_custom} routine which will
simply do nothing when called.

If a particular custom routine is not called in a program, then, obviously, a non-dummy
version does not have to be implemented. To link in a non-dummy version of a custom
routine, put the code file in the same directory as the program code and
relink. Important: 
\begin{itemize}
\item
Do not put the non-dummy version of a custom routine into a library that is linked to the
program. This will generally result in the non-dummy version {\em not} being linked to.
\item
Do not modify any subfile of the \vn{bmad} directory that holds the \bmad code. This is
unnecessary and can make updating the \bmad code cumbersome.
\end{itemize}

While coding a custom routine, it is important to remember that it is
{\em not} permissible to modify any routine argument that does not
appear in the list of output arguments shown in the comment section at
the top of the file.

\index{descrip}
The \Hyperref{r:init.custom}{init_custom} routine is called by
\Hyperref{r:bmad.parser}{bmad_parser} at the end of parsing for any
lattice element that is a \vn{custom} element or has set any one of
the element components as listed above to \vn{custom}. The
\vn{init_custom} routine can be used to initialize the internals of
the element. For example, consider a \vn{custom} element defined in a
lattice file by
\begin{example}
  my_element: custom, val1 = 1.37, descrip = "field.dat", mat6_calc_method = tracking
\end{example}
In this example, the \vn{descrip} (\sref{s:alias}) component is used
to specify the name of a file that contains parameters for this
element. When \vn{init_custom} is called for this element (see below),
the file can be read and the parameters stored in the element
structure. Besides the \vn{ele%value} array, parameters may be stored
in the general use components given in \sref{s:ele.gen}.

The \Hyperref{r:make.mat6.custom}{make_mat6_custom} routine is called by the
\Hyperref{r:track1}{track1} routine when calculating the transfer
matrix through an element. 

The \Hyperref{r:track1.custom}{track1_custom} routine is called by the
\Hyperref{r:track1}{track1} routine when the \vn{tracking_method} for
the element is set to \vn{custom}. Further customization can be set by
the routines \Hyperref{r:track1.preprocess}{track1_preprocess} and \Hyperref{r:track1.postprocess}{track1_postprocess}. See
Section~\sref{s:hook} for more details.

A potential problem with \vn{track1_custom} is that the calling routine, that is
\vn{track1}, does some work like checking aperture, etc. (see the \vn{track1} code for
more details). If this is not desired, the \vn{track1_preprocess} routine (\sref{s:hook})
can be used to do custom tracking and to make sure that \vn{track1} does not do any extra
calculations. This is accomplished by putting the custom tracking code in
\vn{track1_preprocess} and by setting the \vn{finished} argument of \vn{track1_preprocess}
to True.

The
\Hyperref{r:check.aperture.limit.custom}{check_aperture_limit_custom}
routine is used to check if a particle has hit an aperture in
tracking. It is called by the standard \bmad routine
\Hyperref{r:check.aperture.limit}{check_aperture_limit} when
\vn{ele%aperture_type} is set to \vn{custom\$}. A \vn{custom} element
has the standard limit attributes (\sref{s:limit}) so a \vn{custom}
element does not have to implement custom aperture checking code.

The \Hyperref{r:em.field.custom}{em_field_custom} routine is called by
the electro-magnetic field calculating routine
\Hyperref{r:em.field.calc}{em_field_calc} when \vn{ele%field_calc} is
set to \vn{custom\$}. As an alternative to \vn{em_field_custom}, a
\vn{custom} element can use the \vn{field} attribute
(\sref{s:em.fields}) to characterize the element's electromagnetic
fields.

Note: When tracking through a \vn{patch} element, the first step is to
transform the particle's coordinates from the entrance frame to the
exit frame. This is done since it simplifies the tracking. [The
criterion for stopping the propagation of a particle through a
\vn{patch} is that the particle has reached the exit face and the
calculation to determine if a particle has reached the exit face is
simplified if the particle's coordinates are expressed in the
coordinate frame of the exit face.] Thus for \vn{patch} elements,
unlike all other elements, the particle coordinates passed to
\Hyperref{r:em.field.custom}{em_field_custom} are the coordinates with
respect to the exit coordinate frame and not the entrance coordinate
frame. If field must be calculated in the entrance coordinate frame, 
a transformation between entrance and exit frames must be done:
\begin{example}
  subroutine em_field_custom (ele, param, s_rel, time, orb, &
                                  local_ref_frame, field, calc_dfield, err_flag)
  use lat_geometry_mod
  ...
  real(rp) w_mat(3,3), w_mat_inv(3,3), r_vec(3), r0_vec(3)
  real(rp), pointer :: v(:)
  ...
  ! Convert particle coordinates from exit to entrance frame.
  v => ele%value   ! v helps makes code compact
  call floor_angles_to_w_mat (v(x_pitch\$), v(y_pitch\$), v(tilt\$), w_mat, w_mat_inv)
  r0_vec = [v(x_offset\$), v(y_offset\$), v(z_offset\$)]
  r_vec = [orb%vec(1), orb%vec(3), s_rel]  ! coords in exit frame
  r_vec = matmul(w_mat, r_vec) + r0_vec      ! coords in entrance frame

  ! Calculate field and possibly field derivative
  ...

  ! Convert field from entrance to exit frame
  field%E = matmul(w_mat_inv, field%E)
  field%B = matmul(w_mat_inv, field%B)
  if (logic_option(.false., calc_dfield)) then
    field%dE = matmul(w_mat_inv, matmul(field%dE, w_mat))
    field%dB = matmul(w_mat_inv, matmul(field%dB, w_mat))
  endif
\end{example}

The \Hyperref{r:wall.hit.handler.custom}{wall_hit_handler_custom} routine is called when the
Runge-Kutta tracking code \Hyperref{r:odeint.bmad}{odeint_bmad} detects that a particle
has hit a wall (\sref{s:wall}). [This is separate from hitting an
aperture that is only defined at the beginning or end of an lattice
element.] The dummy \vn{wall_hit_handler_custom} routine does nothing.
To keep tracking, the particle must be marked as alive
\begin{example}
  subroutine wall_hit_handler_custom (orb, ele, s, t)
    ...
    orb%state = alive\$   ! To keep on truckin'
    ...
\end{example}
Note: \vn{odeint_bmad} normally does not check for wall collisions.
To change the default behavior, the \vn{runge_kutta_com} common block
must modified. This structure is defined in \vn{runge_kutta_mod.f90}:
\begin{example}
  type runge_kutta_common_struct
    logical :: check_wall_aperture = .false.
    integer :: hit_when = outside_wall\$   ! or wall_transition\$
  end type

  type (runge_kutta_common_struct), save :: runge_kutta_com
\end{example}
To check for wall collisions, the \vn{%check_wall_aperture} component
must be set to true. The \vn{%hit_when} components determines what
constitutes a collision. If this is set to \vn{outside_wall\$} (the
default), then any particle that is outside the wall is considered to
have hit the wall. If \vn{%hit_when} is set to \vn{wall_transition\$},
a collision occurs when the particle crosses the wall boundary. The
distinction between \vn{outside_wall\$} and \vn{wall_transition\$} is
important if particles are to be allowed to travel outside the wall.

%-----------------------------------------------------------------------------
\section{Hook Routines}
\label{s:hook}

\index{track1_preprocess}\index{track1_postprocess}
\index{apply_element_edge_kick_hook}
\index{ele_geometry_hook}
A \vn{hook} routine is like a \vn{custom} routine in that a \vn{hook} routine can be used
for custimizing a \bmad calculation by replacing the \vn{dummy} version of a \vn{hook}
routine with customized code. The difference is that the \vn{hook} routine is always
called at the appropriate time without regard to the type of lattice element under
consideration or what tracking method is being used.
There are three \vn{hook} routines that are available:
\begin{example}
  \Hyperref{r:apply.element.edge.kick.hook}{apply_element_edge_kick_hook}
  \Hyperref{r:ele.geometry.hook}{ele_geometry_hook}
  \Hyperref{r:track1.preprocess}{track1_preprocess}
  \Hyperref{r:track1.postprocess}{track1_postprocess}
\end{example}

The \vn{apply_element_edge_kick_hook} routine can be used for custom tracking through a fringe field.
See the documentation in the file \vn{apply_element_edge_kick_hook.f90} for more details.

The \vn{ele_geometry_hook} routine can be used for custom caculations of the global
geometry of an element. This is useful, for example, for a support table on a kinematic
mount since \bmad does not have the knowledge to calcuate the table orientation from the
position of the mount points. See the documentation in the file \vn{ele_geometry_hook.f90}
for more details.

The last two routines are called by the \Hyperref{r:track1}{track1} routine (along with
\vn{track1_custom} if the element being tracked through has its tracking method set to
\vn{custom}).  The \vn{track1_preprocess} and \vn{track1_postprocess}
routines are useful for a number of things. For example, if the effect of an electron
cloud is to be modeled, these two routines can be used to put in half the electron cloud
kick at the beginning of an element and half the kick at the end.

The routine \vn{track1_preprocess} has an additional feature in that
it has an argument \vn{radiation_included} that can be set to \vn{True}
if the routine \vn{track1_custom} will be called and \vn{track1_custom} will
be handling radiation damping and excitation effects.

%-----------------------------------------------------------------------------
\section{Physical and Mathematical Constants}
\label{s:physical.constants}

\index{constants}
Common physical and mathematical constants that can be used in any expression
are defined in the file:
\begin{example}
 sim_utils/interfaces/physical_constants.f90
\end{example}

The following constants are defined
\begin{example}
  pi = 3.14159265358979d0
  twopi = 2 * pi
  fourpi = 4 * pi
  sqrt_2 = 1.41421356237310d0
  sqrt_3 = 1.73205080757d0
  complex: i_imaginary = (0.0d0, 1.0d0)

  e_mass = 0.51099906d-3   ! DO NOT USE! In GeV
  p_mass   = 0.938271998d0   ! DO NOT USE! In GeV

  m_electron = 0.51099906d6  ! Mass in eV
  m_proton   = 0.938271998d9 ! Mass in eV

  c_light = 2.99792458d8             ! speed of light
  r_e = 2.8179380d-15                ! classical electron radius
  r_p = r_e * m_electron / m_proton  ! proton radius
  e_charge = 1.6021892d-19           ! electron charge

  h_planck = 4.13566733d-15          ! eV*sec Planck's constant
  h_bar_planck = 6.58211899d-16      ! eV*sec h_planck/twopi

  mu_0_vac = fourpi * 1e-7                   ! Permeability of free space
  eps_0_vac = 1 / (c_light**2 * mu_0_vac)    ! Permittivity of free space

  classical_radius_factor = r_e * m_electron ! Radiation constant

  g_factor_electron = 0.001159652193    ! Anomalous gyro-magnetic moment
  g_factor_proton   = 1.79285           ! Anomalous gyro-magnetic moment
\end{example}

%-----------------------------------------------------------------------------
\section{Global Coordinates and S-positions}
\label{s:global.coords}

\index{global coordinates}
\index{s-positions}
The routine \Hyperref{r:lat.geometry}{lat_geometry} will compute the
global floor coordinates at the end of every element in a lattice.
\vn{lat_geometry} works by repeated calls to \Hyperref{r:ele.geometry}{ele_geometry} which
takes the floor coordinates at the end of one element and calculates
the coordinates at the end of the next. For conversion between
orientation matrix $\Bf W$ (\sref{s:global}) and the orientation
angles $\theta, \phi, \psi$, the routines \Hyperref{r:floor.angles.to.w.mat}{floor_angles_to_w_mat}
and \Hyperref{r:floor.w.mat.to.angles}{floor_w_mat_to_angles} can be used.

The routine \Hyperref{r:s.calc}{s_calc} calculates the longitudinal $s$ positions for
the elements in a lattice.

%-----------------------------------------------------------------------------
\section{Reference Energy and Time}
\label{s:ref.energy.prog}

\index {reference energy}
\index{lcavity!reference energy}\index{custom!reference energy}\index{hybrid!reference energy}
The reference energy and time for the elements in a lattice is calculated by
\Hyperref{r:lat.compute.ref.energy.and.time}{lat_compute_ref_energy_and_time}.
The reference energy associated with a lattice element is stored in
\begin{example}
  ele%value(E_tot_start\$)   ! Total energy at upstream end of element (eV)
  ele%value(p0c_start\$)     ! Momentum * c_light at upstream end of element (eV)
  ele%value(E_tot\$)         ! Total energy at downstream end (eV)
  ele%value(p0c\$)           ! Momentum * c_light at downstream end(eV)
\end{example}
Generally, the reference energy is constant throughout an
element so that \vn{%value(E_tot_start\$} = \vn{%value(E_tot\$} and
\vn{%value(p0c_start\$} = \vn{%value(p0c\$}. Exceptions are elements of type:
\begin{example}
  custom,
  em_field,
  hybrid, or
  lcavity
\end{example}
In any case, the starting \vn{%value(E_tot_start\$} and
\vn{%value(p0c_start\$} values of a given element will be the same as
the ending \vn{%value(E_tot\$} and \vn{%value(p0c\$} energies
of the previous element in the lattice.

The reference time and reference transit time is stored in
\begin{example}
  ele%ref_time                ! Ref time at downstream end
  ele%value(delta_ref_time\$)
\end{example}

The reference orbit for computing the reference energy and time is
stored in
\begin{example}
  ele%time_ref_orb_in        ! Reference orbit at upstream end
  ele%time_ref_orb_out       ! Reference orbit at downstream end
\end{example}
Generally \vn{ele%time_ref_orb_in} is the zero orbit. The exception
comes when an element is a \vn{super_slave}. In this case, the
reference orbit through the super_slaves of a given \vn{super_lord} is
constructed to be continuous. This is done for consistency sake. For
example, to ensure that when a marker is superimposed on top of a
wiggler the reference orbit, and hence the reference time, is not altered.

\index{group!reference energy}\index{overlay!reference energy}
\index{superposition!reference energy}
\vn{group} (\sref{s:group}), \vn{overlay} (\sref{s:overlay}), and
\vn{super_lord} elements inherit the reference from the last slave in
their slave list (\sref{s:lat.control}). For \vn{super_lord} elements
this corresponds to inheriting the reference energy of the slave at
the downstream end of the \vn{super_lord}. For \vn{group} and \vn{overlay}
elements a reference energy only makes sense if all the elements under
control have the same reference energy.

Additionally, photonic elements like \vn{crystal}, \vn{capillary},
\vn{mirror} and \vn{multilayer_mirror} elements have an associated photon reference wavelength
\begin{example}
  ele%value(ref_wavelength\$)      ! Meters.
\end{example}

%-----------------------------------------------------------------------------
\section{Global Common Structures}
\label{s:com.struct}

\index{bmad_com}
\index{global_com}
There are two common variables used by Bmad for communication between
routines. These are \vn{bmad_com}, which is a \vn{bmad_common_struct}
structure, and \vn{global_com} which is a \vn{global_common_struct}
structure. The \vn{bmad_com} structure is documented in
Section~\sref{s:bmad.params}. 

\index{global_common_struct|hyperbf}
The \vn{global_common_struct} is meant to hold common parameters that should
not be modified by the user. 
\begin{example}
  type global_common_struct
    logical be_thread_safe = F    ! Avoid thread unsafe practices?
    logical exit_on_error  = T    ! Exit program on error?
  end type
\end{example}
A global variable \vn{global_com} is defined in the \vn{sim_utils} library:
\begin{example}
  type (global_common_struct), save :: global_com
\end{example}
And various routines use the settings in \vn{global_com}.

\begin{description}
\item[\vn{\%be_thread_safe}] \Newline
Toggle to prevent non thread safe calculational optimizations from
being done. See Sec.~\sref{s:parallel.proc} for more details.
\item[\vn{\%exit_on_error}] \Newline
The \vn{%exit_on_error} component tell a routine if it is OK to stop a
program on a severe error. Stopping is generally the right thing when
a program is simply doing a calculation and getting a wrong answer is
not productive. In control system programs and in interactive programs
like \vn{Tao}, it is generally better not to stop on an error.
\end{description}

%-----------------------------------------------------------------------------
\section{Parallel Processing}
\label{s:parallel.proc}

\index{global_common_struct!parallel processing}
\index{global_com!parallel processing}
\bmad was initially developed without regard to parallel
processing. When a demand for multithreading capability arose, \bmad
was modified to meet the need. In order to retain some of the speedups
that can be achieved with saved variables, a switch was introduced in
the \vn{global_common_struct} (\sref{s:com.struct}) called
\vn{%be_thread_safe}. The default is False. This should be set True in
a a multithreaded program:
\begin{example}
  global_com%be_thread_safe = .true.
\end{example}

\index{lat_struct!parallel processing}
One rule to follow in multithreaded programs: \vn{lat_struct}
instances must be local.

Currently, converting \bmad to be thread safe is an active
project. Please contact the \bmad maintainers for more details.

