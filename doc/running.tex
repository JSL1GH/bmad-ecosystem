\chapter{Running Tao}
\label{c:running}
\index{running tao}

%-----------------------------------------------------------------
\section{Initialization from the Command Line}
\index{command line}
\label{s:command.line} 

The syntax of the command line is:
\begin{example}
  EXE-DIRECTORY/tao \{OPTIONS\}
\end{example}
where \vn{EXE-DIRECTORY} is the directory where the tao executable lives.
The optional arguments are:
  \begin{description}
  \item[\vn{-beam <beam_file>}] \Newline
Overrides the \vn{beam_file} (\sref{s:init.global}) specified in the
\tao initialization file.
  \item[\vn{-beam_all <all_beam_file>}] \Newline
Overrides the \vn{beam_all_file} (\sref{s:beam.init}) specified in the
\vn{tao_beam_init} namelist.
  \item[\vn{-beam0 <beam0_file>}] \Newline
Overrides the \vn{beam0_file} (\sref{s:beam.init}) specified in the
\vn{tao_beam_init} namelist.
  \item[\vn{-building_wall <wall_file>}] \Newline
Overrides the \vn{building_wall_file} (\sref{s:init.global}) 
specified in the \tao initialization file.
  \item[\vn{-data <data_file>}] \Newline
Overrides the \vn{data_file} (\sref{s:init.global}) specified in the
\tao initialization file.
  \item[\vn{-geometry <width>x<height>}] \Newline
Overrides the plot window geometry. \vn{<width>} and \vn{<height>}
are in Points. This is equivalent to setting \vn{plot_page%size}
in the \vn{tao_plot_page} namelist \sref{s:init.plot}.
  \item[\vn{-init <tao_init_file>}] \Newline
replaces the default \tao initialization file name
(\vn{tao.init}). Note: A \tao initialization file is actually not
needed. If no \tao initialization file is used, the use of the
\vn{-lat} switch is mandatory and \tao will use a set of default plot
templates for plotting.
  \item[\vn{-lat <bmad_or_xsif_lattice_file>}] \Newline
Overrides the \vn{design_lattice}
lattice file specified in the \tao initialization file
(\sref{s:init.lat}). Example:
\begin{example}
  \$ACC_EXE/tao -init my.init -lat slac.xsif
\end{example}
If there is more than one universe and the universes have different
lattices, separate the different lattice names using a "|" character.
Do not put any spaces in between. Example:
\begin{example}
  \$ACC_EXE/tao -lat xsif::slac.lat|cesr.bmad
\end{example}
  \item[\vn{-log_startup}]
If there is a problem with \tao is started, \vn{-log_startup} can be used
to create a log file of the initialization process.
  \item[\vn{-noinit}] \Newline
Suppresses use of a \tao initialization file. In this case the use of
the \vn{-lat} switch is mandatory and \tao will use a set of default
plot templates for plotting.
  \item[\vn{-noplot}] \Newline
Suppresses the opening of the plot window.
  \item[\vn{-plot <plot_file>}] \Newline
Overrides the \vn{plot_file} (\sref{s:init.global}) specified in the
\tao initialization file.
  \item[\vn{-rf_on}]
Leaves \vn{rfcavity} elements on. Normally \tao turns off these elements
since Twiss and dispersion calculations do not make sense with them on.
  \item[\vn{-startup <startup_command_file>}]
Overrides the \vn{startup_file} (\sref{s:init.global}) specified in the
\tao initialization file.
  \item[\vn{-var <var_file>}] \Newline
Overrides the \vn{var_file} (\sref{s:init.global}) specified in the
\tao initialization file.

\end{description}

%-----------------------------------------------------------------
\section{Aliases}
\index{aliases}
\label{s:aliasing} 

Typing repetitive commands can become tedious. \tao has two constructs
to mitigate this: Aliases and Command Files. Aliases are just like
aliases in Unix. See Section~\sref{s:alias} for more details.

%-----------------------------------------------------------------
\section{Command Files}
\index{command files}
\label{s:command.files} 

Command files are like Unix shell scripts. A series of commands are
put in a file and then that file can be called using the \vn{call}
command (\sref{s:call}).

Do loops are allowed with the following syntax:
\begin{example}
  do <var> = <begin>, <end> \{, <step>\} 
    ...
    tao command [[<var>]]
    ...
  enddo
\end{example}
The \vn{<var>} can be used as a variable in the loop body but must be
bracketed ``[[<var>]]''.  The step size can be any integer positive or
negative but not zero.  Nested loops are allowed and command files can
be called within do loops.

\begin{example}
  do i = 1, 100
    call set_quad_misalignment [[i]] ! command file to misalign quadrupoles
    zero_quad 1e-5*2^([[i]]-1) ! Some user supplied command to zero quad number [[i]]
  enddo
\end{example}

To reduce unnecessary calculations, the logicals \vn{global%lattice_calc_on}
and \vn{global%plot_on} can be toggled from within the command file. Example
\begin{example}
  set global lattice_calc_on = F  ! Turn off lattice calculations
  set global plot_on = F          ! Turn off plot calculations
  ... do some stuff ...
  set global plot_on = T          ! Turn back on 
  set global lattice_calc_on = T  ! Turn back on
\end{example}
Additionally, the \vn{global%command_file_print_on} switch controls
whether printing is suppressed when a command file is called.

A \vn{end_file} command (\sref{s:end.file}) can be used to signal the
end of the command file.

The \vn{pause} command (\sref{s:pause}) can be used to temporarily
pause the command file.

