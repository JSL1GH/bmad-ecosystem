\chapter{Single Mode}\index{Single mode}
\label{c:single}

In \vn{single mode} each key stroke is interpreted by \tao without the user having to press 
the carriage control key (there are a few exceptions which are noted below). From \vn{line mode}
use the \vn{single_mode} command to get into \vn{single_mode}. To go back to \vn{line mode} type
"\vn{Z}".

Key bindings are established via the \vn{key_bindings} initialization namelist 
(See Section~\ref{s:init_single}). 
The variables are divided into banks of 10. The 0\Th bank uses variables 1
through 10 the 1\St bank uses variables 11 through 20, etc. 
At any one time only one bank is active. The relationship between
the keys and a change in a variable is:
\begin{example}
                 Change by factor of:          
     Variable    -10  -1    1   10
   ----------    ---  ---  ---  ---
    1 + 10*ib     Q    q    1    !   
    2 + 10*ib     W    w    2    @    
    3 + 10*ib     E    e    3    \#   
    4 + 10*ib     R    r    4    \$   
    5 + 10*ib     T    t    5    %   
    6 + 10*ib     Y    y    6    ^   
    7 + 10*ib     U    u    7    \&
    8 + 10*ib     I    i    8    *   
    9 + 10*ib     O    o    9    (   
   10 + 10*ib     P    p    0    )   
\end{example}
In the above table ib is the bank number (0 for the 0\Th bank, etc.),
and the change is in multiples of the \vn{delta} for a variable as
specified in the \vn{key_bindings} namelist. Initially the 0Th bank is
active. The left arrow and right arrow are used to decrease or
increase the bank number.  Additionally the "\vn{<}" and "$>$" keys
can be used to change the deltas for the variables.

%% keys ------------------------------------------------------------------------
\section{List of Key Strokes}\index{Single Mode!List of Key Strokes}
\label{s:keys}

\begin{description}
\item[?]
Type a short help message.

\item[z] 
Go back to \vn{line mode}

\item[Z] 
Exit \tao
                                        
\item[g]
Go run the default optimizer. The optimizer will run until you type a
'.' (a period).  Periodically during the optimization the variable
values will be written to files, one for each universe, whose name is
\vn{tao_opt_vars\#.dat}. where \vn{\#} is the universe number.

\item[s]  
Show various parameters.

\item[v $<$digit$>$ $<$value$>$]
Set variable value. \vn{<digit>} is between 0 and 9 corresponding to a
variable of the current bank. \vn{<value>} is the value to set the
variable to.

\item[$<$]
Reduce the deltas (the amount that a variable is changed when you use
the keys 0 through 9) of all the variables by a factor of 2.

\item[$>$]
Increase the deltas (the amount that a variable is changed when you
use the keys 0 through 9) of all the variables by a factor of 2.

\item[left\_arrow]
Shift the active key bank down by 1: ib -$>$ ib - 1

\item[right\_arrow]
Shift the active key bank up by 1: ib -$>$ ib + 1

\item[up\_arrow]
Scale plots by a factor of 0.5.

\item[down\_arrow]
Scale plots by a factor of 2.

\item[$<$CR$>$]
Do nothing but replot.

\item[/e $<$Index or Name$>$]
Prints info on a lattice element. If there are two lattices being used
and only the information of an element from one particular lattice is
wanted then after the name or index put ";n" where n is the lattice
index.

\item[/l]
Print a list of the lattice elements with Twiss parameters.

\item[/v $<$Universe Index$>$]
Switch the viewed universe.

\item[/w]
Write variable values to the default output file.
The default output file name is set by \vn{global%var_out}. This file is
in BMAD format.

\item[/x $<$min$>$ $<$max$>$]
Set the horizontal scale (longitudinal position) min and max values
for all the plots. This is the same as setting p%x%min and p%x%max in
the TOAD input file. The effective min and max can never be larger
than the bounds of the lattice itself. Thus to see the entire lattice
set min = 0 and max = something large. Typing a "$<$cr$>$" after
typing "/x" will set min = 0 and max = 1e20.

\item[/right\_arrow]
Set saved ("value0") values to variable values to saved values. The
saved values (the value0 column in the display) are initially set to
the initial value on startup. There are saved values for both the
manual and automatic variables. Note that reading in a TOAD input file
will reset the saved values. If you want to save the values of the
variables in this case use "/w" to save to a file. Use the
"\vn{/left_arrow}" command to go in the reverse direction.

\item[/left\_arrow]
Paste saved ("value0") values back to the variable values.  The saved
values (the value0 column in the display) are initially set to the
initial value on startup. Use the "\vn{/right_arrow}" command to go in
the reverse direction.

\item[/up\_arrow]
Increase all key deltas by a factor of 10.

\item[/down\_arrow]
Decrease all key deltas by a factor of 10.

\item[-p]
Toggle plotting. Whether to plot or not to plot is initially
determined by plot%enable.

\end{description}
