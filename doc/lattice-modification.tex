\chapter{Modifying a Lattice}
\label{c:lattice.mod}

To use \tao with this chapter, please starup \tao as explained in \sref{s:obtaining} in 
the \vn{introduction_to_tao} directory that you have copied to your local area.

%----------------------------------------------------------------
\section{Changing a Variable}
\label{s:change.variable}

Let's change a variable and see what happens to the lattice. We are going to
change a quadrupole strength so we should plot the change in beta and phase.
Type the following (everything after the `!' are just comments and can be
omitted):
\begin{example}
  x-axis * index        ! let the data index be the x-axis
  place top beta          ! plot Beta data on top plot
  place bottom phase      ! plot phase data on bottom plot
  plot * model - design ! plot the difference between model and design data
  scale                   ! scale all plots
\end{example}

The k1 value can be increased by 0.01 units for quadrupole Q05W by typing
\index{commands!change}
\begin{example}
  change var quad\_k1[5] 0.01
  scale
\end{example}
Note the information returned on the command line after the command
and the relative changes in beta and phase in the plot window. This is
a vertically focusing quadrupole so the vertical beta and phase are
affected more than the horizontal. The \cmd{0.01} at the end of the
command tells \tao to change this variable by 0.01 units. If you want
to set a variable to a particular value then use a ``@'' before the
value. So, to change this quadrupole k1 to -0.348 type
\begin{example}
  change var quad\_k1[5] @-0.348
\end{example}

%----------------------------------------------------------------
\section{Putting things back where you found them}
\label{s:put_it_back}

Let's put this quadrupole back where we found it. We can also modify
the quadrupole by modifying the lattice element directly by typing
\begin{example}
  change ele Q05W k1 d0.0
\end{example}
By modifying the element directly with the \cmd{change ele} command
you can modify almost any attribute of the element listed in the
output of \cmd{show ele Q05W}.  The ``d'' before the value is used to
set the variable relative to the design value.
\index{commands!change}

If you've changed the lattice around a lot using variables, a great
way to set all variables back to their design values is to type
\index{commands!set}
\begin{example}
  set var *|model = *|design
\end{example}
This only works if you just changed variables. If you changed any
elements directly with the \cmd{change ele} command then this will not
work. To set every attribute of every element back to the design type
\begin{example}
  set lattice model = design
\end{example}
Note that this will also recalculate the data and variable values
associated with the the model lattice to reflect the change so all the
bookkeeping is done for you.
