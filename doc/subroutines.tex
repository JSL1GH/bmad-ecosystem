\chapter{Bmad Library Subroutine List}

Below are a list of \bmad and dcslib routines sorted by their
functionality.  Use the \vn{getf} and \vn{listf} (\sref{s:getf}) 
scripts for more information on individual routines.
This list includes low level routines that are not generally used in
writing code for a program but may be useful in certain unique
situations.  Excluded from the list are very low level routines that are
solely meant for \bmad internal use.


\toffset
\begin{center}
\begin{tabular}{|l|l|} \hline
{\em Routine Type} & {\em Section} \\ \hline
  Beam: Long Range Beam-Beam Interaction      & \ref{r:lrbbi}          \\ \hline
  Beam: Low Level Routines                    & \ref{r:low.beam}       \\ \hline
  Beam: Tracking and Manipulation             & \ref{r:beam}           \\ \hline
  CESR Specific                               & \ref{r:cesr}           \\ \hline
  \cpp Interface                              & \ref{r:cpp}            \\ \hline
  Coherent Synchrotron Radiation (CSR)        & \ref{r:csr}            \\ \hline
  Collective Effects                          & \ref{r:collective}     \\ \hline
  Electro-Magnetic Fields                     & \ref{r:em.fields}      \\ \hline
  Inter-Beam Scattering (IBS)                 & \ref{r:ibs}            \\ \hline
  Lattice: Choosing                           & \ref{r:lat}            \\ \hline
  Lattice: Element Informational              & \ref{r:info}           \\ \hline
  Lattice: Element Manipulation               & \ref{r:elem}           \\ \hline
  Lattice: Geometry                           & \ref{r:geom}           \\ \hline
  Lattice: Low Level Stuff                    & \ref{r:low.help}       \\ \hline
  Lattice: Manipulation                       & \ref{r:trans}          \\ \hline
  Lattice: Miscellaneous                      & \ref{r:misc.help}      \\ \hline
  Lattice: Reading and Writing Files          & \ref{r:read}           \\ \hline
  Matrices                                    & \ref{r:mat}            \\ \hline
  Matrix: Low Level Routines                  & \ref{r:low.mat}        \\ \hline
  Measurement Simulation Routines             & \ref{r:meas}           \\ \hline
  Multipoles                                  & \ref{r:multi}          \\ \hline
  DCSLIB and CESR_UTILS                       & \ref{r:dcs.misc}       \\ \hline
  Optimizers (Nonlinear)                      & \ref{r:opti}           \\ \hline
  Overload Equal Sign                         & \ref{r:equal}          \\ \hline
  Particle Coordinate Stuff                   & \ref{r:coord}          \\ \hline
  PTC Interface                               & \ref{r:ptc}            \\ \hline
  Quick Plot                                  & \ref{r:qp}             \\ \hline
  Spin                                        & \ref{r:spin}           \\ \hline
  Transfer Maps: Routines Called by MAKE_MAT6 & \ref{r:mat6}           \\ \hline
  Transfer Maps: Taylor Maps                  & \ref{r:taylor}         \\ \hline
  Tracking and Closed Orbit                   & \ref{r:track}          \\ \hline
  Tracking: Low Level Routines                & \ref{r:low.track}      \\ \hline
  Tracking: Macroparticle                     & \ref{r:macro}          \\ \hline
  Tracking: Mad Routines                      & \ref{r:mad}            \\ \hline
  Tracking: Routines Called by TRACK1         & \ref{r:track1}         \\ \hline
  Twiss and Other Calculations                & \ref{r:twiss}          \\ \hline
  Twiss: 6-Dimensional                        & \ref{r:twiss6}         \\ \hline
  Wake Fields                                 & \ref{r:wake}           \\ \hline
\end{tabular}
\end{center}
\toffset

%------------------------------------------------------------------------
\section{Beam: Long Range Beam-Beam Interaction}
\label{r:lrbbi}    

\begin{description}

\index{Routine!init_lrbb}
\item[init_lrbbi(lat, oppos_lat, lrbbi_ele, ix_lrbbi, ix_oppos)] \Newline 
     Subroutine to calculate the basic parameters of a beambeam element. 
     Initializes the element and establishes the following values.

\index{Routine!insert_lrbbi}
\item[insert_lrbbi (lat, lat_oppos, cross_positions, ix_lrbbi)] \Newline
Subroutine to create and insert beambeam elements at each parasitic crossing
point as specified form a list of crossing points.

\index{Routine!lrbbi_crossings}
\item[lrbbi_crossings (n_bucket, oppos_buckets, cross_positions)] \Newline
Subroutine to calculate the location of the parasitic crossings points 
given a bunch and an array of positions of the bunches it will cross. 

\index{Routine!make_lrbb}
\item[make_lrbbi(master_lat, master_lat_oppos, lat, ix_lrbbi, master_ix_lrbbi)] \Newline
Subroutine to turn elements marking parasitic crossings into beam-beam elements. 

\index{Routine!mark_lrbb}
\item[mark_lrbbi(master_lat, master_lat_oppos, lat, crossings)] \Newline
Subroutine to insert named markers into the lat structure at the positions of parasitic crossings. 

\end{description}

%------------------------------------------------------------------------
\section{Beam: Low Level Routines}
\label{r:low.beam}

The following helper routines are generally not useful for general use.

\begin{description}

\index{Routine!beam_equal_beam}
\item[beam_equal_beam (beam1, beam2)] \Newline 
Subroutine to set one particle beam equal to another taking care of
pointers so that they don't all point to the same place.

\index{Routine!order_particles_in_z}
\item[order_particles_in_z (bunch)] \Newline 
Subroutine to order the particles longitudinally 
The ordering uses the centroid of the particles:

\index{Routine!track1_beam_lat}
\item[track1_beam_lat (beam_start, lat, ix_ele, beam_end)] \Newline 
Subroutine to track a beam of particles through a single element.
Overloaded by \vn{track1_beam}.

\index{Routine!track1_beam_ele}
\item[track1_beam_ele (beam_start, ele, param, beam_end)] \Newline 
Subroutine to track a beam of particles through a single element.
Overloaded by \vn{track1_beam}.

\index{Routine!track1_bunch_lat}
\item[track1_bunch_lat (bunch_start, lat, ix_ele, bunch_end)] \Newline 
Subroutine to track a bunch of particles through an element.
Overloaded by \vn{track1_bunch}.

\index{Routine!track1_bunch_ele}
\item[track1_bunch_ele (bunch_start, ele, param, bunch_end)] \Newline 
Subroutine to track a bunch of particles through an element.
Overloaded by \vn{track1_bunch}.

\index{Routine!track1_sr_wake}
\item[track1_sr_wake (bunch, ele)] \Newline 
Subroutine to apply the short range wake fields to a bunch. 

\index{Routine!track1_lr_wake}
\item[track1_lr_wake (bunch, ele)] \Newline 
Subroutine to put in the long-range wakes for particle tracking.

\index{Routine!track1_particle}
\item[track1_particle (start, ele, param, end)] \Newline 
Subroutine to track a particle through an element.

\index{Routine!add_sr_long_wake}
\item[add_sr_long_wake (ele, bunch, num_in_front, follower)] \Newline 
Adds the longitudinal wake for all particles in front of the follower.

\index{Routine!init_spin_distribution}
\item[init_spin_distribution (beam_init, bunch)] \Newline 
Initializes a spin distribution according to init_beam\%spin

\index{Routine!calc_bunch_params_slice}
\item[calc_bunch_params_slice (bunch, ele, params, plane, slice_center, slice_spread)] \Newline 
Finds all bunch parameters for a slice through the beam distribution.

\index{Routine!find_bunch_sigma_matrix}
\item[find_bunch_sigma_matrix (particle, ave, sigma)] \Newline 
Routine to find the sigma matrix elements of a particle distribution.

\end{description}

%------------------------------------------------------------------------
\section{Beam: Tracking and Manipulation}
\label{r:beam}    
\index{Beam Tracking!list of routines}

See \sref{s:part.track} for a discussion of using a collection of particles to simulate
a bunch.

\begin{description}

\index{Routine!angle_to_canonical_coords}
\item[angle_to_canonical_coords (particle, energy0)] \Newline 
Subroutine to convert particle coords from 
    (x, x', y, y', z, E)

\index{Routine!calc_bunch_params}
\item[calc_bunch_params (bunch, ele, params)] \Newline 
Finds all bunch parameters defined in bunch_params_struct, both normal-mode
and projected

\index{Routine!calc_bunch_params_slice}
\item[calc_bunch_params (bunch, ele, params, plane, slice_center, slice_spread)] \Newline 
Finds all bunch parameters for a slice through the beam distribution.

\index{Routine!canonical_to_angle_coords}
\item[canonical_to_angle_coords (particle, energy0)] \Newline 
Subroutine to convert particleparticle coords from 
    (x, px, y, py, z, pz)

\index{Routine!init_beam_distribution}
\item[init_beam_distribution (ele, beam_init, beam)] \Newline 
Subroutine to initialize a distribution of particles matched to
the Twiss parameters, centroid position, and Energy - z correlation

\index{Routine!reallocate_beam}
\item[reallocate_beam (beam, n_bunch, n_particle)] \Newline 
Subroutine to reallocate memory within a beam_struct.

\index{Routine!track1_bunch_custom}
\item[track1_bunch_custom (bunch_start, lat, ix_ele, bunch_end)] \Newline 
Dummy routine for custom bunch tracking. 

\index{Routine!track_beam}
\item[track_beam (lat, beam, ix1, ix2)] \Newline 
     Subroutine to track a beam of macroparticles from the end of
     lat\%ele(ix1) Through to the end of lat\%ele(ix2).

\end{description}

%------------------------------------------------------------------------
\section{CESR Specific}
\label{r:cesr}

These routines are specific to the Cornell CESR storage ring and
outside of Cornell are not of general interest.

\begin{description}

\index{Routine!bmad_to_cesr}
\item[bmad_to_cesr (lat, cesr)] \Newline
Subroutine to transfer information from the lat structure returned from 
\vn{Bmad_Parser} to a structure for the CESR lat.

\index{Routine!bmad_to_db}
\item[bmad_to_db (lat, db)] \Newline
Subroutine to return information on the database that pertains to CESR elements. 

\index{Routine!cesr_all_data_struct_init}
\item[cesr_all_data_struct_init (data)] \Newline 
Subroutine to initialize a cesr_all_data_struct variable to zero/False.

\index{Routine!cesr_crossings}
\item[\protect\parbox{6in}{cesr_crossings (i_train, j_car, species, n_trains_tot, \\
  \hspace*{2in} n_cars, cross_positions, n_car_spacing, train_spacing)}] \Newline
Subroutine to calculate all parasitic crossing points for a bunch at a
given location with oppositely circulating bunches of known spacing.

\index{Routine!cesr_loc_decod}
\item[cesr_loc_decode(string, array, num)] \Newline 
Subroutine to decode a list of locations in CESR.

\index{Routine!cesr_loc_encod}
\item[cesr_loc_encode(list, ew_encode, sense, string)] \Newline 
Subroutine to encode a list of locations into an output string.

\index{Routine!cesr_locator}
\item[cesr_locator (str_in, prefix, ix_pre, loc, err_flag)] \Newline 
Subroutine to parse a character string for a location in the CESR ring.

\index{Routine!cesr_elements_get}
\item[cesr_elements_get (name, n_found, ele)] \Newline 
Subroutine to find the location of elements from [CESR.SURVEY]LAT_MASTER.DAT

\index{Routine!choose_cesr_lattice}
\item[choose_cesr_lattice (lattice, lat_file, current_lat, lat)] \Newline
Subroutine to let the user choose a lattice. The subroutine will present a list to choose from. 

\index{Routine!create_vsp_volt_elements}
\item[create_vsp_volt_elements (lat, ele_type)] \Newline
Subroutine to create elements corresponding to the 6 database elements in CSR VSP VOLT. 

\index{Routine!db_group_to_bmad}
\item[\protect\parbox{6in}{db_group_to_bmad (ing_name, ing_num, \\
  \hspace*{2in} biggrp_set, lat, con_, n_con, ok, type_err)}] \Newline
Subroutine to take a database group element and find the elements 
controlled along with the coefficients. 

\index{Routine!db_group_to_bmad_group}
\item[\protect\parbox{6in}{db_group_to_bmad_group (group_name, group_num, \\
  \hspace*{3in} i_biggrp, lat, ix_ele, ok, type_err)}] \Newline
Subroutine to set up a database group knob in a Bmad lat structure. 

\index{Routine!db_struct_init_cu_now}
\item[db_struct_init_cu_now (db)] \Newline 
Subroutine to initialize a db_struct variable to zero.

\index{Routine!identify_db_node}
\item[identify_db_node (db_name, db, dp_ptr, ok, type_err)] \Newline
Subroutine to find which array in DB is associated with DB_NAME. 

\index{Routine!k_to_quad_calibrations}
\item[\protect\parbox{6in}{k_to_quad_calibrations (k_theory,  
\\ \hspace*{2in} energy, cu_theory, k_base, dk_gev_dcu, cu_per_k_gev)}] \Newline
Single/double precision wrapper routine for k_to_quad_calib. 
See k_to_quad_calib for more details.

\index{Routine!lattice_to_bmad_file_name}
\item[lattice_to_bmad_file_name (lattice, bmad_file_name)] \Newline
Subroutine to convert a lattice name to the appropriate Bmad file name. 

\index{Routine!location_decode}
\item[location_decode (string, array, ix_min, num)] \Newline 
Subroutine to set a list of locations in a logical array to True.

\index{Routine!location_encode}
\item[location_encode (string, loc, exists, ix_min)] \Newline 
Subroutine to encode a list of locations.

\index{Routine!number_to_cesr_data_file}
\item[number_to_cesr_data_file (number, who, file_name, err, print_err)] \Newline 
Routine to form the file name for a phase, orbit, eta, or ac_dispersion
data file. 

\index{Routine!number_to_data_file_number}
\item[number_to_data_file_number (num_in, who, num_out, err, print_err)] \Newline 
Routine to form the file name for a phase, orbit, eta, or ac_dispersion
data file. 

\index{Routine!quad_calib}
\item[\protect\parbox{6in}{quad_calib (lattice, k_theory, k_base, len_quad, 
\\ \hspace*{2in} cu_per_k_gev, quad_rot, dk_gev_dcu, cu_theory)}] \Newline
Subroutine to return the calibration constants for the CESR quads. 

\index{Routine!read_butns_file}
\item[read_butns_file (butns_num, butns, db, ok)] \Newline
Subroutine to read in the information in a BUTNS.nnnnn file. 

\index{Routine!read_cesr_cooked_data}
\item[read_cesr_cooked_data (file_name, all_dat, err)] \Newline 
Routine to read in the orbit, phase, cbar, or eta cooked data from a file. 

\index{Routine!read_cesr_cooked_data_parameters}
\item[read_cesr_cooked_data_parameters (file_name, param, err)] \Newline 
Routine to read in the header information from a data file.

\index{Routine!read_cesr_dispersion_data}
\item[read_cesr_dispersion_data (file_name, all_dat, err)] \Newline 
Routine to read the data in a raw dispersion file.
Note: You can use number_to_cesr_data_file to convert from an orbit number

\index{Routine!read_cesr_orbit_data}
\item[read_cesr_orbit_data (file_name, all_dat, err)] \Newline 
Routine to read the data in a raw orbit file.
Note: You can use number_to_cesr_data_file to convert from an orbit number

\index{Routine!read_cesr_phase_data}
\item[read_cesr_phase_data (file_name, all_dat, err)] \Newline 
Routine to read the data in a raw phase/coupling file.
Note: You can use number_to_cesr_data_file to convert from an orbit number

\index{Routine!lat_to_quad_calib}
\item[\protect\parbox{6in}{lat_to_quad_calib (lat, cesr, k_theory, k_base, 
\\ \hspace*{2in} len_quad, cu_per_k_gev, quad_rot, dk_gev_dcu, cu_theory)}] \Newline
Subroutine to return the calibration constants for the CESR quads. 

\end{description}

%------------------------------------------------------------------------
\section{C++ Interface}
\label{r:cpp}      
\index{C++ interface!list of routines}

\begin{description}

\index{Routine!amode_to_c}
\item[amode_to_c (f_amode, c_amode)] \Newline 
Subroutine to convert a Bmad amode_struct to a C++ C_amode.

\index{Routine!arr2mat}
\item[arr2mat (arr, n1, n2) result (mat)] \Newline 
Function to take a an array and turn it into a matrix.

\index{Routine!bmad_com_to_c}
\item[bmad_com_to_c (c_bmad_com)] \Newline 
Subroutine to convert the Bmad bmad_com_struct common block to 
a C++ C_bmad_com.

\index{Routine!c_logic}
\item[c_logic (logic) result (c_log)] \Newline 
Function to convert from a fortran logical to a C logical.

\index{Routine!c_str}
\item[c_str (str) result (c_string)] \Newline 
Function to append a null (0) character at the end of a string (trimmed
of trailing blanks) so it will look like a C character array. 

\index{Routine!control_to_c}
\item[control_to_c (f_control, c_control)] \Newline 
Subroutine to convert a Bmad control_struct to a C++ C_control.

\index{Routine!coord_to_c}
\item[coord_to_c (f_coord, c_coord)] \Newline 
Subroutine to convert a Bmad coord_struct to a C++ C_coord.

\index{Routine!ele_to_c}
\item[ele_to_c (f_ele, c_ele)] \Newline 
Subroutine to convert a Bmad ele_struct to a C++ C_ele.

\index{Routine!em_field_to_c}
\item[em_field_to_c (f_em_field, c_em_field)] \Newline 
Subroutine to convert a Bmad em_field_struct to a C++ C_em_field.

\index{Routine!f_logic}
\item[f_logic (logic) result (f_log)] \Newline 
Function to convert from a fortran logical to a C logical.

\index{Routine!floor_position_to_c}
\item[floor_position_to_c (f_floor_position, c_floor_position)] \Newline 
Subroutine to convert a Bmad floor_position_struct to a C++ C_floor_position.

\index{Routine!linac_mode_to_c}
\item[linac_mode_to_c (f_linac_mode, c_linac_mode)] \Newline 
Subroutine to convert a Bmad linac_mode_struct to a C++ C_linac_mode.

\index{Routine!lr_wake_to_c}
\item[lr_wake_to_c (f_lr_wake, c_lr_wake)] \Newline 
Subroutine to convert a Bmad lr_wake_struct to a C++ C_lr_wake.

\index{Routine!mat2arr}
\item[mat2arr (mat) result (arr)] \Newline 
Function to take a matrix and turn it into an array.

\index{Routine!modes_to_c}
\item[modes_to_c (f_modes, c_modes)] \Newline 
Subroutine to convert a Bmad modes_struct to a C++ C_modes.

\index{Routine!mode_info_to_c}
\item[mode_info_to_c (f_mode_info, c_mode_info)] \Newline 
Subroutine to convert a Bmad mode_info_struct to a C++ C_mode_info.

\index{Routine!param_to_c}
\item[param_to_c (f_param, c_param)] \Newline 
Subroutine to convert a Bmad param_struct to a C++ C_param.

\index{Routine!lat_to_c}
\item[lat_to_c (f_lat, c_lat)] \Newline 
Subroutine to convert a Bmad lat_struct to a C++ C_lat.

\index{Routine!sr_table_wake_to_c}
\item[sr_table_wake_to_c (f_sr_table_wake, c_sr_wake)] \Newline 
Subroutine to convert a Bmad sr_table_wake_struct to a C++ C_sr_table_wake.

\index{Routine!sr_mode_wake_to_c}
\item[sr_mode_wake_to_c (f_sr_mode_wake, c_sr_wake)] \Newline 
Subroutine to convert a Bmad sr_mode_wake_struct to a C++ C_sr_mode_wake.

\index{Routine!twiss_to_c}
\item[twiss_to_c (f_twiss, c_twiss)] \Newline 
Subroutine to convert a Bmad twiss_struct to a C++ C_twiss.

\index{Routine!taylor_term_to_c}
\item[taylor_term_to_c (f_taylor_term, c_taylor_term)] \Newline 
Subroutine to convert a Bmad taylor_term_struct to a C++ C_taylor_term.

\index{Routine!taylor_to_c}
\item[taylor_to_c (f_taylor, c_taylor)] \Newline 
Subroutine to convert a Bmad taylor_struct to a C++ C_taylor.

\index{Routine!wake_to_c}
\item[wake_to_c (f_wake, c_wake)] \Newline 
Subroutine to convert a Bmad wake_struct to a C++ C_wake.

\index{Routine!wig_term_to_c}
\item[wig_term_to_c (f_wig_term, c_wig_term)] \Newline 
Subroutine to convert a Bmad wig_term_struct to a C++ C_wig_term.

\index{Routine!xy_disp_to_c}
\item[xy_disp_to_c (f_xy_disp, c_xy_disp)] \Newline
Subroutine to convert a Bmad xy_disp_struct to a C++ C_xy_disp.

\end{description}

%------------------------------------------------------------------------
\section{Coherent Synchrotron Radiation (CSR)}
\label{r:csr}

\begin{description}

\index{Routine!csr_bin_particles}
\item[csr_bin_particles (particle, bin)] \Newline 
Routine to bin the particles longitudinally in s. 

\index{Routine!csr_bin_kicks}
\item[csr_bin_kicks (lat, ix_ele, s_travel, bin)] \Newline 
Routine to cache intermediate values needed for the csr calculations.

\index{Routine!i_csr}
\item[i_csr (z, d, val, bin) result (i_this)] \Newline 
Routine to calculate the CSR kick integral.

\index{Routine!z_calc_csr}
\item[z_calc_csr (d, val, bin, dz_dd) result (z_this)] \Newline 
Routine to calculate the distance between the source particle and the
kicked particle.

\index{Routine!d_calc_csr}
\item[d_calc_csr (dz_particles, val, bin) result (d_this)] \Newline 
Routine to calculate the distance between source and kick points.

\end{description}

%------------------------------------------------------------------------
\section{Collective Effects}
\label{r:collective}

\begin{description}

\index{Routine!setup_trans_space_charge_calc}
\item[setup_trans_space_charge_calc (calc_on, lattice, mode, closed_orb)] \Newline 
Subroutine to initialize constants needed by the transverse space charge 
tracking routine track1_space_charge. This routine must be called if 

\index{Routine!touschek_lifetime}
\item[touschek_lifetime (mode, lifetime, lat, orb)] \Newline
Subroutine to calculate the Touschek lifetime for a lat.

\index{Routine!ibs_rates}
\item[ibs_rates (lat, mode, rates, formula)] \Newline
Subroutine to calculate the IBS rates for a lat.

\index{Routine!ibs_equilibrium}
\item[ibs_equilibrium(lat, inmode, ibsmode, formula, coupling)] \Newline
Subroutine to calculate the equilibrium mode of a lat due to IBS effects
by iterating over derivatives of the equilibrium equations.

\index{Routine!ibsequilibrium2}
\item[ibsequilibrium2(lat, inmode, ibsmode, formula, ratio, initial_blow_up)] \Newline
Subroutine to calculate the equilibrium mode of a lat due to IBS effects
by iterating over the equilibrium equations.

\end{description}

%------------------------------------------------------------------------
\section{Electro-Magnetic Fields}
\label{r:em.fields}     

\begin{description}

\index{Routine!em_field_calc}
\item[em_field_calc (ele, param, s_pos, here, local_ref_frame, field, calc_dfield)] \Newline 
Subroutine to calculate the E and B fields for an element.

\index{Routine!em_field_custom}
\item[em_field_custom] \Newline
Custom routine for calculating fields.

\index{Routine!em_field_kick}
\item[em_field_kick (ele, param, s, r, local_ref_frame, dr_ds, dkick)] \Newline 
Subroutine to essentially calculate the kick felt by a particle in a
element. 

\end{description}

%------------------------------------------------------------------------
\section{Inter-Beam Scattering (IBS)}
\label{r:ibs}

\begin{description}

\index{Routine!ibs_lifetime}
\item[ibs_lifetime(lat, mode, lifetime, formula)] \Newline 
 This module computes the beam lifetime due to
 the diffusion process according to equation 12

\index{Routine!bjmt}
\item[bjmt(lat, mode, rates)] \Newline 
 This is a private subroutine.  To access this subroutine, call
 ibs_rates.

\index{Routine!bane}
\item[bane(lat, mode, rates)] \Newline 
 This is a private subroutine. To access this subroutine, call
 ibs_rates.

\index{Routine!cimp}
\item[cimp(lat, mode, rates)] \Newline 
 This is a private subroutine. To access this subroutine, call
 ibs_rates.

\index{Routine!g}
\item[g(u)] \Newline 
 This is an 13-degree piecewise polynomial interpolation of the
 integral for the CIMP ibs formulation (equation 34 in "Intrabeam 

\index{Routine!mtto}
\item[mtto(lat, mode, rates)] \Newline 
 NOTE:  The Mtingwa-Tollerstrup formula gives different from the other
 formulations in this module.

\end{description}

%------------------------------------------------------------------------
\section{Lattice: Choosing}
\label{r:lat}

\begin{description}

\index{Routine!choose_cesr_lattice}
\item[choose_cesr_lattice (lattice, lat_file, current_lat, lat)] \Newline
Subroutine to let the user choose a lattice. The subroutine will present a list to choose from. 

\index{Routine!get_lattice_list}
\item[get_lattice_list (lat_list, num_lats, directory)] \Newline
Subroutine to get the names of the lattices of the form: directory/bmad_*.lat 

\end{description}

%------------------------------------------------------------------------
\section{Lattice: Informational}
\label{r:info}     

\begin{description}

\index{Routine!attribute_index}
\item[attribute_index (key, name)] \Newline
Function to return the index of an attribute for a given element 
type and the name of the attribute 

\index{Routine!attribute_name}
\item[attribute_name (key, index)] \Newline
Function to return the name of an attribute for a particular type of element. 

\index{Routine!check_lat_controls}
\item[check_lat_controls (lat, exit_on_error)] \Newline
Subroutine to check if the control links in a lat structure are valid. 

\index{Routine!attribute_free}
\item[attribute_free (ele, ix_attrib, lat, err_print_flag) result (free)] \Newline
Function to check if an attribute is free to vary.

\index{Routine!elements_locator}
\item[elements_locator (key, lat, indx)] \Newline
Subroutine to locate all the elements of a certain kind in a lat. 

\index{Routine!element_locator}
\item[element_locator (ele_name, lat, ix_ele)] \Newline
Subroutine to locate an element in a lat. 

\index{Routine!ele_at_s}
\item[ele_at_s (lat, s, ix_ele)] \Newline 
Subroutine to return the index of the element at position s.

\index{Routine!equivalent_eles}
\item[equivalent_eles (ele1, ele2) result (equiv)] \Newline 
Subroutine to see if two elements are equivalent in terms of their attributes so
that their Taylor Maps, if they existed, would be the same.

\index{Routine!find_element_ends}
\item[find_element_ends (lat, ix_ele, ix_start, ix_end)] \Newline
Subroutine to find the end points of an element. 

\index{Routine!get_element_lord_list}
\item[get_element_lord_list (lat, ix_slave, lord_list, n_lord)] \Newline 
Subroutine to get the list of lords for an element.

\index{Routine!get_element_slave_list}
\item[get_element_slave_list (lat, ix_lord, slave_list, n_slave)] \Newline 
Subroutine to get the list of slaves for an element.

\index{Routine!key_name_to_key_index}
\item[key_name_to_key_index (key_str, abbrev_allowed) result (key_index)] \Newline 
Function to convert a character string  (eg: "drift") to an index (eg: drift\$).

\index{Routine!type_ele}
\item[\protect\parbox{6in}{type_ele (ele, type_zero_attrib, type_mat6, \\ 
\hspace*{1in} type_twiss, type_control, type_wake, type_floor_coords)}] \Newline
Subroutine to print the contents of an element at the terminal. 

\index{Routine!type2_ele}
\item[\protect\parbox{6in}{type2_ele (ele, lines, n_lines, type_zero_attrib, type_mat6, \\
\hspace*{1in} type_twiss, type_control, type_wake, type_floor_coords)}] \Newline
Like \vn{type_ele} but the output is stored in a string array. 

\index{Routine!type_twiss}
\item[type_twiss (ele, frequency_units)] \Newline
Subroutine to type out the Twiss parameters from an element. 

\index{Routine!type2_twiss}
\item[type2_twiss (ele, frequency_units, lines, n_lines)] \Newline
Like \vn{type_twiss} but the output is stored in a string array. 

\end{description}

%------------------------------------------------------------------------
\section{Lattice: Element Manipulation}
\label{r:elem}     

These routine are for adding elements, moving elements, etc.

\begin{description}

\index{Routine!add_superimpose}
\item[add_superimpose (lat, super_ele, ix_super)] \Newline
Subroutine to make a superimposed element. 

\index{Routine!attribute_bookkeeper}
\item[attribute_bookkeeper (ele, param)] \Newline
Subroutine to make sure the attributes of an element are self-consistent. 

\index{Routine!changed_attribute_bookkeeper}
\item[changed_attribute_bookkeeper (lat, a_ptr)] \Newline 
Subroutine to do bookkeeping when a particular attribute has been altered.

\index{Routine!create_group}
\item[create_group (lat, ix_ele, contrl)] \Newline
Subroutine to create a group control element. 

\index{Routine!create_girder}
\item[create_girder (lat, ix_girder, ix_slave)] \Newline 
     Subroutine to add the controller information to slave elements of
     an girder_lord.

\index{Routine!create_overlay}
\item[create_overlay (lat, ix_overlay, attrib_name, , contl)] \Newline
Subroutine to add the controller information to slave elements of an 
overlay_lord. 

\index{Routine!compress_lat}
\item[compress_lat (lat, ok)] \Newline
Subroutine to compress the ele(*) and control(*) arrays to remove
elements no longer used.

\index{Routine!insert_element}
\item[insert_element (lat, insert_ele, insert_index)] \Newline
Subroutine to Insert a new element into the tracking part of the 
lat structure. 

\index{Routine!make_hybrid_lat}
\item[make_hybrid_lat (lat_in, use_ele, remove_markers, lat_out, ix_out)] \Newline
Subroutine to concatenate together elements to make a hybrid lat 

\index{Routine!new_control}
\item[new_control (lat, ix_ele)] \Newline
Subroutine to create a new control element. 

\index{Routine!pointer_to_attribute}
\item[\protect\parbox{6in}{pointer_to_attribute (ele, attrib_name, do_allocation, 
\\ \hspace*{2in} ptr_attrib, ix_attrib, err_flag, err_print_flag)}] \Newline
Returns a pointer to an attribute of an element with name attrib_name. 

\index{Routine!pointers_to_attribute}
\item[pointers_to_attribute (lat, ele_name, attrib_name, do_allocation,] \Newline 
                    ptr_array, err_flag, err_print_flag, ix_eles, ix_attrib)
Returns an array of pointers to an attribute with name attrib_name within 
elements with name ele_name.

\index{Routine!remove_ele_from_lat}
\item[remove_ele_from_lat (lat, ix_ele)] \Newline 
Subroutine to remove an element from the lattice.
This subroutine assumes that the element has no controllers.

\index{Routine!split_lat}
\item[split_lat (lat, s_split, ix_split, split_done)] \Newline
Subroutine to split a lat at a point.

\index{Routine!update_hybrid_list}
\item[update_hybrid_list (lat, n_in, use_ele)] \Newline
Subroutine used to specify a list of element that should not be
hyberdized by \vn{make_hybrid_lat}.

\end{description}

%------------------------------------------------------------------------
\section{Lattice: Geometry}
\label{r:geom}     
\index{Coordinates!global!list of routines}

\begin{description}

\index{Routine!lat_geometry}
\item[lat_geometry (lat)] \Newline
Subroutine to calculate the physical placement of all the elements in a lat. 
That is, the layout on the floor. 

\index{Routine!s_calc}
\item[s_calc (lat)] \Newline
Subroutine to calculate the longitudinal distance S for the elements in a lat. 

\end{description}

%------------------------------------------------------------------------
\section{Lattice: Low Level Stuff}
\label{r:low.help} 

\begin{description}

\index{Routine!adjust_super_lord_s_position}
\item[adjust_super_lord_s_position (lat, ix_lord)] \Newline
Subroutine to adjust the positions of the slaves of a 
super_lord due to changes in the lord's s_offset. 

\index{Routine!bracket_index}
\item[bracket_index (s_, s, ix)] \Newline
Subroutine to find the index ix so that s(ix) $\le$ s $<$ s(ix+1). 
If s $<$ s(1) then ix = 0 

\index{Routine!deallocate_ele_pointers}
\item[deallocate_ele_pointers (ele)] \Newline
Subroutine to deallocate the pointers in an element. 

\index{Routine!dispersion_to_orbit}
\item[dispersion_to_orbit (ele, disp_orb)] \Newline
Subroutine to make an orbit vector proportional to the dispersion. 

\index{Routine!makeup_group_slaves}
\item[makeup_group_slaves (lat, ix_slave)] \Newline
Subroutine to calculate the attributes of group slave elements.

\index{Routine!makeup_super_slave}
\item[makeup_super_slave (lat, ix_slave)] \Newline
Subroutine to calculate the attributes of overlay slave elements. 

\index{Routine!orbit_to_dispersion}
\item[orbit_to_dispersion (orb_diff, ele)] \Newline
Subroutine to take an orbit vector difference and calculate the dispersion. 

\index{Routine!twiss1_propagate}
\item[twiss1_propagate (twiss1, mat2, length, twiss2)] \Newline 
Subroutine to propagate the twiss parameters of a single mode.

\end{description}

%------------------------------------------------------------------------
\section{Lattice: Manipulation}
\label{r:trans}    

\begin{description}

\index{Routine!control_bookkeeper}
\item[control_bookkeeper (lat, ix_ele)] \Newline
Subroutine to calculate the combined strength of the attributes for
controlled elements.

\index{Routine!deallocate_lat_pointers}
\item[deallocate_lat_pointers (lat)] \Newline 
Subroutine to deallocate the pointers in a lat.

\index{Routine!init_ele}
\item[init_ele (ele)] \Newline
Subroutine to initialize an element. 

\index{Routine!init_lat}
\item[init_lat (lat, n)] \Newline 
Subroutine to initialize a Bmad lat.

\index{Routine!lattice_bookkeeper}
\item[lattice_bookkeeper (lat)] \Newline 
Subroutine to do bookkeeping for the entire lattice.

\index{Routine!reallocate_coord}
\item[reallocate_coord (coord_, n_coord)] \Newline 
Subroutine to reallocate an allocatable  coord_struct array to at least:
coord(0:n_coord).

\index{Routine!reverse_ele}
\item[reverse_ele (ele)] \Newline
Subroutine to "reverse" an element for backward tracking. 

\index{Routine!lat_reverse}
\item[lat_reverse (lat_in, lat_rev)] \Newline
Subroutine to construct a lat structure with the elements in reversed 
order. This may be used for backward tracking through the lat. 

\index{Routine!set_design_linear}
\item[set_design_linear (lat)] \Newline
Subroutine to set only those elements on that constitute the "design" 
lattice. That is, only quadrupoles, bends and wigglers will be set on. 

\index{Routine!set_on_off}
\item[set_on_off (key, lat, switch, orb)] \Newline
Subroutine to turn on or off a set of elements (quadrupoles,
rfcavities, etc.) in a lat.

\index{Routine!transfer_ele}
\item[transfer_ele (ele1, ele2)] \Newline 
     Subroutine to set ele2 = ele1. 
     This is a plain transfer of information not using the overloaded equal.

\index{Routine!transfer_eles}
\item[transfer_eles (ele1, ele2)] \Newline 
     Subroutine to set ele2(:) = ele1(:). 
     This is a plain transfer of information not using the overloaded equal.

\index{Routine!transfer_ele_taylor}
\item[transfer_ele_taylor (ele_in, ele_out, taylor_order)] \Newline 
     Subroutine to transfer a Taylor map from one element to another.

\index{Routine!transfer_lat}
\item[transfer_lat (lat1, lat2)] \Newline 
     Subroutine to set lat2 = lat1. 
     This is a plain transfer of information not using the overloaded equal.

\index{Routine!transfer_lat_parameters}
\item[transfer_lat_parameters (lat_in, lat_out)] \Newline
Subroutine to transfer the lat parameters (such as lat\%name, 
lat\%param, etc.) from one lat to another. 

\index{Routine!transfer_lat_taylors}
\item[transfer_lat_taylors (lat_in, lat_out, 
                        type_out, transfered_all) ] \Newline 
Subroutine to transfer the taylor maps from the elements of one lat to
the elements of another. 

\index{Routine!zero_ele_offsets}
\item[zero_ele_offsets (ele)] \Newline 
Subroutine to zero the offsets, pitches and tilt of an element.

\end{description}

%------------------------------------------------------------------------
\section{Lattice: Miscellaneous}
\label{r:misc.help}

\begin{description}

\index{Routine!cross_product}
\item[cross_product (vec1, vec2)] \Newline 
Returns the cross product of vec1 x vec2

\index{Routine!c_multi}
\item[c_multi (n, m)] \Newline
Subroutine to compute multipole factors: 
c_multi(n, m) = +/- ("n choose m")/n! 

\index{Routine!compute_reference_energy}
\item[compute_reference_energy (lat)] \Newline
Subroutine to compute the reference energy for each element in a lattice. 

\index{Routine!custom_radiation_integrals}
\item[custom_radiation_integrals (lat, ir, orb)] \Newline
Dummy routine for the radiation_integrals calculation for CUSTOM elements. 

\index{Routine!convert_total_energy_to}
\item[convert_total_energy_to (E_tot, particle, gamma, kinetic, beta, pc, brho)] \Newline
Subroutine to calculate the momentum, etc. from a particle's toal energy. 

\index{Routine!convert_pc_to}
\item[convert_pc_to (pc, particle, E_tot, gamma, kinetic, beta, brho)] \Newline
Subroutine to calculate the energy, etc. from a particle's momentum. 

\index{Routine!field_interpolate_3d}
\item[field_interpolate_3d (position, field_mesh, deltas)] \Newline
Function to interpolate a 3d field. 

\index{Routine!name_to_list}
\item[name_to_list (lat, ele_names, use_ele)] \Newline
Subroutine to make a list of the elements in a lat 
whose name matches the names in the ele_names list. 

\index{Routine!order_super_lord_slaves}
\item[order_super_lord_slaves (lat, ix_lord)] \Newline
Subroutine to make the slave elements of a super_lord in order. 

\index{Routine!release_rad_int_cache}
\item[release_rad_int_cache (ix_cache)] \Newline 
     Subroutine to release the memory associated with caching wiggler values.

\index{Routine!wiggler_vec_potential}
\item[wiggler_vec_potential (ele, energy, here, vec_pot)] \Newline
Subroutine to calculate the normalized vector potential at a point for a wiggler.

\end{description}

%------------------------------------------------------------------------
\section{Reading and Writing Lattice Files} 
\label{r:read}
\index{Lattice files!reading and writing routines}

\begin{description}

\index{Routine!aml_parser}
\item[aml_parser (lat_file, lat, make_mats6, digested_read_ok, use_line)] \Newline 
Subroutine to parse an AML input file and put the information in a lat_struct.

\index{Routine!bmad_parser}
\item[bmad_parser (in_file, lat, make_mats6, digested_read_ok, use_line)] \Newline
Subroutine to parse (read in) a Bmad input file. 

\index{Routine!bmad_parser2}
\item[bmad_parser2 (in_file, lat, orbit, make_mats6)] \Newline
Subroutine to parse (read in) a Bmad input file to modify an existing lattice. 

\index{Routine!bmad_to_mad}
\item[bmad_to_mad (mad_file, lat, ix_start, ix_end)] \Newline 
Subroutine to write a mad lattice file using the information in
a lat_struct. 

\index{Routine!bmad_to_xsif}
\item[bmad_to_xsif (xsif_file, lat, ix_start, ix_end)] \Newline 
Subroutine to write a xsif lattice file using the information in
a lat_struct. Optionally only part of the lattice can be generated.

\index{Routine!combine_consecutive_elements}
\item[combine_consecutive_elements (lat)] \Newline 
Routine to combine consecutive elements in the lattice that have the same name.
This allows simplification, for example, of lattices where elements have been split 
to compute the beta function at the center.

\index{Routine!create_unique_ele_names}
\item[create_unique_ele_names (lat, key, suffix)] \Newline 
Routine to give elements in a lattice unique names.

\index{Routine!read_digested_bmad_file}
\item[read_digested_bmad_file (in_file_name, lat, version)] \Newline
Subroutine to read in a digested file. 

\index{Routine!write_bmad_lattice_file}
\item[write_bmad_lattice_file (lattice_name, lat)] \Newline 
Subroutine to write a Bmad lattice file using the information in
a lat_struct.

\index{Routine!write_digested_bmad_file}
\item[write_digested_bmad_file (digested_name, lat, n_files, file_names)] \Newline
Subroutine to write a digested file. 

\index{Routine!xsif_parser}
\item[xsif_parser (xsif_file, lat, make_mats6, use_line)] \Newline 
     Subroutine to parse an XSIF (extended standard input format) lattice file.

\end{description}

%------------------------------------------------------------------------
\section{Matrices}
\label{r:mat}
\index{Matrix!list of routines}

\begin{description}

\index{Routine!c_to_cbar}
\item[c_to_cbar (ele, cbar_mat)] \Newline
Subroutine to compute Cbar from the C matrix and the Twiss parameters. 

\index{Routine!cbar_to_c}
\item[cbar_to_c (cbar_mat, ele)] \Newline
Subroutine to compute C coupling matrix from the Cbar matrix and the Twiss parameters. 

\index{Routine!clear_lat_1turn_mats}
\item[clear_lat_1turn_mats (lat)] \Newline
Clear the 1-turn matrices in the lat structure. 

\index{Routine!do_mode_flip}
\item[do_mode_flip (ele, ele_flip)] \Newline
Subroutine to mode flip the Twiss parameters of an element 

\index{Routine!make_g2_mats}
\item[make_g2_mats (twiss, g_mat, g_inv_mat)] \Newline
Subroutine to make the matrices needed to go from normal mode coords to 
coordinates with the beta function removed. 

\index{Routine!make_g_mats}
\item[make_g_mats (ele, g_mat, g_inv_mat)] \Newline
Subroutine to make the matrices needed to go from normal mode coords to 
coordinates with the beta function removed. 

\index{Routine!make_mat6}
\item[make_mat6 (ele, param, c0, c1)] \Newline
Subroutine to make the 6x6 transfer matrix for an element. 

\index{Routine!make_v_mats}
\item[make_v_mats (ele, v_mat, v_inv_mat)] \Newline
Subroutine to make the matrices needed to go from normal mode coords to X-Y 
coords and vice versa. 

\index{Routine!mat6_to_taylor}
\item[mat6_to_taylor (mat6, vec0, bmad_taylor)] \Newline
Subroutine to form a first order Taylor map from the 6x6 transfer matrix 
and the 0th order transfer vector. 

\index{Routine!mat_det}
\item[mat_det (mat, det)] \Newline 
     Subroutine to take the determinant of a square matrix

\index{Routine!mat_eigen}
\item[mat_eigen (mat, eval_r, eval_i, evec_r, evec_i, error)] \Newline 
Routine for determining the eigen vectors and eigen values of a matrix.

\index{Routine!mat_inverse}
\item[mat_inverse (mat, mat_inv)] \Newline
Subroutine to take the inverse of a square matrix. 

\index{Routine!mat_make_unit}
\item[mat_make_unit (mat)] \Newline 
     routine to create a unit matrix.

\index{Routine!mat_rotation}
\item[mat_rotation (mat, angle, bet_1, bet_2, alph_1, alph_2)] \Newline 
     Subroutine to construct a 2x2 rotation matrix for translation from
     point 1 to point 2.

\index{Routine!mat_symplectify}
\item[mat_symplectify (mat_in, mat_symp)] \Newline
Subroutine to form a symplectic matrix that is approximately equal to the input matrix. 

\index{Routine!mat_symp_check}
\item[mat_symp_check (mat, error)] \Newline
Subroutine to check the symplecticity of a square matrix 

\index{Routine!mat_symp_conj}
\item[mat_symp_conj (mat1, mat2)] \Newline 
Subroutine to take the symplectic conjugate of a square matrix.

\index{Routine!mat_symp_decouple}
\item[mat_symp_decouple (t0, tol, stat, u, v, ubar, vbar, g, twiss1, twiss2, type_out)] \Newline
Subroutine to find the symplectic eigen--modes of the one turn 4x4 coupled 
transfer matrix T0. 

\index{Routine!mat_type}
\item[mat_type (mat, nunit, header)] \Newline 
     Subroutine to output matrices to the terminal or to a file

\index{Routine!match_ele_to_mat6}
\item[match_ele_to_mat6 (ele, mat6, vec0)] \Newline 
Subroutine to make the 6 x 6 transfer matrix from the twiss parameters.

\index{Routine!multi_turn_tracking_to_mat}
\item[multi_turn_tracking_to_mat (track, i_dim, mat1, track0, chi)] \Newline
Subroutine to analyze 1-turn tracking data to find the 1-turn transfer matrix 
and the closed orbit offset.

\index{Routine!transfer_matrix_calc}
\item[transfer_matrix_calc (lat, rf_on, mat6, ix1, ix2)] \Newline
Subroutine to calculate the transfer matrix between two elements. If
ix1 and ix2 are not present the full 1--turn matrix is calculated.

\index{Routine!one_turn_mat_at_ele}
\item[one_turn_mat_at_ele (ele, phi_a, phi_b, mat4)] \Newline
Subroutine to form the 4x4 1-turn coupled matrix with the reference point 
at the end of an element. 

\index{Routine!lat_make_mat6}
\item[lat_make_mat6 (lat, ix_ele, coord)] \Newline
Subroutine to make the 6x6 linear transfer matrix for an element 

\index{Routine!taylor_to_mat6}
\item[taylor_to_mat6 (a_taylor, c0, mat6, c1)] \Newline
Subroutine to calculate the linear (Jacobian) matrix about some trajectory from a Taylor map. 

\index{Routine!transfer_mat2_from_twiss}
\item[transfer_mat2_from_twiss (twiss1, twiss2, mat)] \Newline
Subroutine to make a 2 x 2 transfer matrix from the Twiss parameters at the end points. 

\index{Routine!transfer_mat_from_twiss}
\item[transfer_mat_from_twiss (ele1, ele2, m)] \Newline 
Subroutine to make a 6 x 6 transfer matrix from the twiss parameters
at the beginning and end of the element.

\index{Routine!twiss_from_mat2}
\item[twiss_from_mat2 (mat, det, twiss, stat, tol, type_out)] \Newline
Subroutine to extract the Twiss parameters from the one-turn 2x2 matrix 

\index{Routine!twiss_from_mat6}
\item[twiss_from_mat6 (mat6, ele, stable, growth_rate)] \Newline
Subroutine to extract the Twiss parameters from the one-turn 6x6 matrix 

\index{Routine!twiss_to_1_turn_mat}
\item[twiss_to_1_turn_mat (twiss, phi, mat2)] \Newline
Subroutine to form the 2x2 1-turn transfer matrix from the Twiss parameters. 

\end{description}

%------------------------------------------------------------------------
\section{Matrix: Low Level Routines}
\label{r:low.mat}  

Listed below are helper routines that are not meant for general use.

\begin{description}

\index{Routine!drift_mat6_calc}
\item[drift_mat6_calc (mat6, length, start, end)] \Newline
Subroutine to calculate a drift transfer matrix with a possible kick. 

\index{Routine!mat6_dispersion}
\item[mat6_dispersion (mat6, e_vec)] \Newline
Subroutine to put the dispersion into ele\%mat6 given the dispersion vector E_VEC 

\index{Routine!sol_quad_mat6_calc}
\item[sol_quad_mat6_calc (ks, k1, length, mat6, orb)] \Newline
Subroutine to calculate the transfer matrix for a combination solenoid/quadrupole element. 

\index{Routine!tilt_mat6}
\item[tilt_mat6 (mat6, tilt)] \Newline
Subroutine to transform a 6x6 transfer matrix to a new reference frame that is 
tilted in (x, Px, y, Py) with respect to the old reference frame. 

\end{description}

%------------------------------------------------------------------------
\section{Measurement Simulation Routines}
\label{r:meas}  
\index{Measurement Simulations!list of routines}

Routines to simulate errors in orbit, dispersion, betatron phase, and
coupling measurements

\begin{description}

\index{Routine!check_if_ele_is_monitor}
\item[check_if_ele_is_monitor (ele, err)] \Newline
Routine to check that the element is either an instrument, monitor, or marker.
This routine is private and not meant for general use.

\index{Routine!compute_bpm_transformation_numbers}
\item[compute_bpm_transformation_numbers (ele)] \Newline
Routine to compute the numbers associated with the transformation between
the actual orbit, phase, eta, and coupling, and what the measured values.

\index{Routine!to_eta_reading}
\item[to_eta_reading (eta, ele, axis, reading, err)] \Newline
Compute the measured dispersion reading given the true dispersion and the
monitor offsets, noise, etc.

\index{Routine!to_orbit_reading}
\item[to_orbit_reading (orb, ele, axis, reading, err)] \Newline
Calculate the measured reading on a bpm given the actual orbit and the
BPM's offsets, noise, etc.

\index{Routine!to_phase_and_coupling_reading}
\item[to_phase_and_coupling_reading (ele, mon, err)] \Newline
Find the measured coupling values given the actual ones


\end{description}

%------------------------------------------------------------------------
\section{Lattice: Multipoles}
\label{r:multi}    
\index{Multipole!list of routines}

\begin{description}

\index{Routine!ab_multipole_kick}
\item[ab_multipole_kick (a, b, n, coord, kx, ky)] \Newline 
Subroutine to put in the kick due to an ab_multipole.

\index{Routine!mexp}
\item[mexp (x, m) result (this_exp)] \Newline 
Returns x**m with 0**0 = 0.

\index{Routine!multipole_ab_to_kt}
\item[multipole_ab_to_kt (an, bn, knl, tn)] \Newline
Subroutine to convert ab type multipoles to kt (MAD standard) multipoles. 

\index{Routine!multipole_ele_to_ab}
\item[multipole_ele_to_ab (ele, particle, a, b, use_ele_tilt)] \Newline
Subroutine to put the scaled element multipole components (normal and skew) into 2 vectors. 

\index{Routine!multipole_ele_to_kt}
\item[multipole_ele_to_kt (ele, particle, knl, tilt, use_ele_tilt)] \Newline
Subroutine to put the scaled element multipole components (strength and tilt) 
into 2 vectors. 

\index{Routine!multipole_init}
\item[multipole_init] \Newline
Subroutine to initialize the multipole arrays within an element.

\index{Routine!multipole_kick}
\item[multipole_kick (knl, tilt, n, coord)] \Newline
Subroutine to put in the kick due to a multipole. 

\index{Routine!multipole_kt_to_ab}
\item[multipole_kt_to_ab (knl, tn, an, bn)] \Newline
Subroutine to convert kt (MAD standard) multipoles to ab type multipoles. 

\end{description}

%------------------------------------------------------------------------
\section{Miscellaneous DCSLIB and CESR_UTILS Routines}
\label{r:dcs.misc}      

\begin{description}

\index{Routine!abs_sort}
\item[abs_sort (array, index, n)] \Newline 
  Subroutine to sort by absolute value.

\index{Routine!bbi_kick}
\item[bbi_kick (x, y, r, kx, ky)] \Newline 
Subroutine to compute the normalized kick due to the beam-beam
interaction using the normalized position for input.

\index{Routine!cesr_iargc}
\item[cesr_iargc ()] \Newline 
Platform independent function to return the number of command
line arguments. Use this with cesr_getarg.

\index{Routine!cesr_getarg}
\item[cesr_getarg (i_arg, arg)] \Newline 
Platform independent function to return the i'th command
line argument. Use this with cesr_iargc.

\index{Routine!complex_error_function}
\item[complex_error_function (wr, wi, zr, zi)] \Newline 
This routine evaluates the function w(z) in the first quadrant of
the complex plane. 

\index{Routine!date_and_time_stamp}
\item[date_and_time_stamp (string, numeric_month)] \Newline 
Subroutine to return the current date and time in a character string.

\index{Routine!downcase_string}
\item[downcase_string (string)] \Newline 
Routine to convert a string to lowercase:

\index{Routine!err_exit}
\item[err_exit] \Newline 
Subroutine to first show the stack call list before exiting.
This routine is typically used when a program detects an error condition.

\index{Routine!get_tty_char}
\item[get_tty_char (this_char, wait, flush)] \Newline 
Subroutine for getting a single character from the terminal.
Also see: get_a_char

\index{Routine!get_a_char}
\item[get_a_char (this_char, wait, ignore_this)] \Newline 
Subroutine for getting a single character from the terminal.
Also see: get_tty_char

\index{Routine!indexx_char}
\item[indexx_char (arr, index)] \Newline 
Subroutine to sort a character array.
This subroutine is used to overload the generic name indexx.

\index{Routine!index_nocase}
\item[index_nocase (string, match_str) result (indx)] \Newline 
Function to look for a sub-string of string that matches match_str.
This routine is similar to the fortran INDEX function

\index{Routine!integer_option}
\item[integer_option (integer_default, opt_integer)] \Newline 
Function to retrun True or False dependending upon the state of an 
optional integer.

\index{Routine!is_integer}
\item[is_integer (string)] \Newline 
Function to tell if the first word in a string is a valid integer.

\index{Routine!is_real}
\item[is_real (string, ignore) result (good)] \Newline 
Function to test if a string represents a real number.

\index{Routine!linear_fit}
\item[linear_fit (x, y, n_data, a, b, sig_a, sig_b)] \Newline 
Subroutine to fit to y = A + B x

\index{Routine!logic_option}
\item[logic_option (logic_default, opt_logic)] \Newline 
Function to retrun True or False dependending upon the state of an 
optional logical.

\index{Routine!lunget}
\item[lunget()] \Newline 
Function to return a free file unit number to be used with an open statement.

\index{Routine!match_reg}
\item[match_reg (str, pat)] \Newline 
Function for matching with regular expressions.
Note: strings are trimmed before comparison.

\index{Routine!match_wild}
\item[match_wild (string, template) result (this_match)] \Newline 
Function to do wild card matches. Note: trailing blanks will be discarded
before any matching is done.

\index{Routine!modulo2}
\item[modulo2 (x, amp)] \Newline 
Function to return y = x + 2 * n * amp, n is an integer, such that y is 
in the interval [-amp, amp].

\index{Routine!out_io}
\item[out_io] \Newline 
Subroutine to print to the terminal for command line type programs.
The idea is that for programs with a gui this routine can be easily
replaced with another routine.

\index{Routine!ran_engine}
\item[ran_engine (set, get)] \Newline 
Subroutine to set what random number generator algorithm is used.
If this routine is never called then pseudo_random\$ is used.

\index{Routine!ran_gauss}
\item[ran_gauss (harvest)] \Newline 
Subroutine to return a Gaussian distributed random number with unit sigma.

\index{Routine!ran_gauss_converter}
\item[ran_gauss_converter (set, get, sigma_cut)] \Newline 
Subroutine to set what conversion routine is used for converting
uniformly distributed random numbers to Gaussian distributed random numbers.

\index{Routine!ran_seed_put}
\item[ran_seed_put (seed)] \Newline 
Subroutine to seed the random number generator. 

\index{Routine!ran_seed_get}
\item[ran_seed_get (seed)] \Newline 
Subroutine to return the seed used for the random number generator.

\index{Routine!ran_uniform}
\item[ran_uniform (harvest)] \Newline 
Subroutine to return a random number uniformly distributed in the 
interval [0, 1]. This routine uses the same algorithm as ran from

\index{Routine!re_allocate}
\item[re_allocate (ptr_to_array, n)] \Newline 
Function to reallocate a pointer to an array of strings, integers, reals, or logicals.

\index{Routine!re_associate}
\item[re_associate (array, n)] \Newline 
Function to reassociate an allocatable array of strings, integers, reals, or logicals.

\index{Routine!real_option}
\item[real_option (real_default, opt_real)] \Newline 
Function to retrun True or False dependending upon the state of an 
optional real.

\index{Routine!skip_header}
\item[skip_header (unit_, error_flag)] \Newline 
Subroutine to find the first line of data in a file. 

\index{Routine!splitfilename}
\item[splitfilename(filename, path, basename, is_relative) result (ix_char)] \Newline 
Routine to take filename and splits it into its constituent parts, 
the directory path and the base file name.  

\index{Routine!str_match_wild}
\item[str_match_wild(str, pat) result (a_match)] \Newline 
Function to match a character string against a regular expression pattern.
This is a replacement for the VMS function str\$match_wild.

\index{Routine!string_to_int}
\item[string_to_int (line, default, value, err_flag)] \Newline 
Subroutine to convert a string to an integer.

\index{Routine!string_trim}
\item[string_trim(in_string, out_string, word_len)] \Newline 
Subroutine to trim a string of leading blanks and/or tabs and also to return the
length of the first word.

\index{Routine!string_trim2}
\item[string_trim2 (in_str, delimitors, out_str,] \Newline 
                                     ix_word, delim, ix_next)
Subroutine to trim a string of leading delimitors and also to return the
length of the first word.

\index{Routine!spline_akima}
\item[spline_akima (spline, stat)] \Newline 
Given a set of (x,y) points we want to interpolate between the points.
This subroutine computes the semi-hermite cubic spline developed by akima

\index{Routine!spline_evaluate}
\item[spline_evaluate (spline, x, ok, y, dy)] \Newline 
Subroutine to evaluate a spline at a set of points.

\index{Routine!type_this_file}
\item[type_this_file (filename)] \Newline 
Subroutine to type out a file to the screen.

\index{Routine!upcase_string}
\item[upcase_string (string)] \Newline 
Routine to convert a string to uppercase:

\end{description}

%------------------------------------------------------------------------
\section{Nonlinear Optimizers}
\label{r:opti}      

\begin{description}


\index{Routine!amoeba}
\item[amoeba(p, y, ftol, func, iter)] \Newline
Downhill simplex method of Neider and Mead. Numerical Recipes routine.

\index{Routine!frprmn}
\item[frprmn (p, ftol, iter, fret)] \Newline
Fletcher--Reeves minimization. Numerical Recipes routine.

\index{Routine!mrqmin}
\item[mrqmin (x, y, sig, a, maska, covar, alpha, chisq, funcs, alamda)] \Newline
Levenburg--Marquardt minimization. Numerical Recipes routine.

\index{Routine!opti_lmdif}
\item[opti_lmdif (vec, n, merit, eps) result(this_opti)] \Newline 
Function which tries to get the merit function(s) as close to zero as possible
by changing the values in vec. Multiple merit functions can be used.

\index{Routine!initial_lmdif}  
\item[initial_lmdif] \Newline 
Subroutine that clears out previous saved values of the optimizer.

\index{Routine!suggest_lmdif}
\item[suggest_lmdif (xv,fv,eps,itermx,iend,reset_flag)] \Newline 
Reverse communication subroutine. 

\index{Routine!opti_de}
\item[opti_de (v_best, generations, population, merit_func, v0, v_del)] \Newline 
Differential Evolution for Optimal Control Problems.
This optimizer is based upon the work of Storn and Price. 

\end{description}

%------------------------------------------------------------------------
\section{Overloading the equal sign}
\label{r:equal}    

These routines are overloaded by the equal sign so should not be called explicitly.

\begin{description}

\index{Routine!beam_equal_beam}
\item[mp_beam_equal_mp_beam (beam1, beam2)] \Newline
Subroutine that is used to set one macroparticle beam to another. This routine
takes care of the pointers in beam1.

\index{Routine!branch_equal_branch}
\item[branch_equal_branch (branch1, branch2)] \Newline 
Subroutine that is used to set one branch equal to another. 

\index{Routine!bunch_equal_bunch}
\item[bunch_equal_bunch (bunch1, bunch2)] \Newline
Subroutine that is used to set one macroparticle bunch to another. This routine
takes care of the pointers in bunch1.

\index{Routine!coord_equal_coord}
\item[coord_equal_coord (coord1, coord2)] \Newline
Subroutine that is used to set one coord_struct equal to another. 

\index{Routine!ele_equal_ele}
\item[ele_equal_ele (ele1, ele2)] \Newline
Subroutine that is used to set one element equal to another. 
This routine takes care of the pointers in ele1. 

\index{Routine!ele_vec_equal_ele_vec}
\item[ele_vec_equal_ele_vec (ele1, ele2)] \Newline
Subroutine that is used to set one element vector equal to another. 
This routine takes care of the pointers in ele1. 

\index{Routine!real_8_equal_taylor}
\item[real_8_equal_taylor (y8, bmad_taylor)] \Newline
Subroutine to overload "=" in expressions real_8 (PTC) = bmad_taylor.

\index{Routine!lat_equal_lat}
\item[lat_equal_lat (lat1, lat2)] \Newline
Subroutine that is used to set one lat equal to another. 
This routine takes care of the pointers in lat1. 

\index{Routine!lat_vec_equal_lat_vec}
\item[lat_vec_equal_lat_vec (lat1, lat2)] \Newline
Subroutine that is used to set one lat array equal to another. 
This routine takes care of the pointers in lat1(:). 

\index{Routine!taylor_equal_real_8}
\item[taylor_equal_real_8 (bmad_taylor, y8)] \Newline
Subroutine to overload "=" in expressions bmad_taylor = real_8 (PTC) 

\index{Routine!universal_equal_universal}
\item[universal_equal_universal (universal1, universal2)] \Newline
Subroutine that is used to set one PTC universal_taylor 
structure equal to another. 

\end{description}

%------------------------------------------------------------------------
\section{Particle Coordinate Stuff}
\label{r:coord}    
\index{Coordinates!list of routines}

\begin{description}

\index{Routine!convert_coords}
\item[convert_coords (in_type_str, coord_in, ele, out_type_str, coord_out)] \Newline
Subroutine to convert between lab frame, normal mode, normalized normal mode, 
and action-angle coordinates. 

\index{Routine!type_coord}
\item[type_coord (coord)] \Newline
Subroutine to type out a coordinate. 

\end{description}

%------------------------------------------------------------------------
\section{Interface to PTC}
\label{r:ptc}      
\index{PTC/FPP!list of routines}

\begin{description}

\index{Routine!concat_real_8}
\item[concat_real_8 (y1, y2, y3)] \Newline
Subroutine to concatenate two real_8 taylor series. 

\index{Routine!ele_to_fibre}
\item[ele_to_fibre (ele, fiber, param, integ_order, steps)] \Newline
Subroutine to convert a Bmad element to a PTC fibre element. 

\index{Routine!map_coef}
\item[map_coef (y, i, j, k, l, style)] \Newline
Function to return the coefficient of the map y(:) up to 3rd order. 

\index{Routine!kill_gen_field}
\item[kill_gen_field (gen_field)] \Newline
Subroutine to kill a gen_field. 

\index{Routine!kind_name}
\item[kind_name (this_kind)] \Newline
Function to return the name of a PTC kind. 

\index{Routine!real_8_equal_taylor}
\item[real_8_equal_taylor (y8, bmad_taylor)] \Newline
Subroutine to overload "=" in expressions real_8 = bmad_taylor 

\index{Routine!real_8_to_taylor}
\item[real_8_to_taylor (y8, bmad_taylor, switch_z)] \Newline
Subroutine to convert from a real_8 taylor map in Etienne's PTC to a taylor map in Bmad. 

\index{Routine!real_8_init}
\item[real_8_init (y, set_taylor)] \Newline
Subroutine to allocate a PTC real_8 variable. 

\index{Routine!remove_constant_taylor}
\item[remove_constant_taylor (taylor_in, taylor_out, c0, remove_higher_order_terms)] \Newline
Subroutine to remove the constant part of a taylor series. 

\index{Routine!lat_to_layout}
\item[lat_to_layout (lat, ptc_layout)] \Newline
Subroutine to create a PTC layout from a Bmad lat. 

\index{Routine!set_ptc}
\item[set_ptc (param, taylor_order, integ_order, n_step, no_cavity, exact_calc)] \Newline
Subroutine to initialize PTC. 

\index{Routine!set_taylor_order}
\item[set_taylor_order (order, override_flag)] \Newline
Subroutine to set the taylor order. 

\index{Routine!sort_universal_terms}
\item[sort_universal_terms (ut_in, ut_sorted)] \Newline
Subroutine to sort the taylor terms from "lowest" to "highest". 

\index{Routine!taylor_equal_real_8}
\item[taylor_equal_real_8 (bmad_taylor, y8)] \Newline
Subroutine to overload "=" in expressions bmad_taylor = y8 

\index{Routine!taylor_to_real_8}
\item[taylor_to_real_8 (bmad_taylor, y8, switch_z)] \Newline
Subroutine to convert from a taylor map in Bmad to a real_8 taylor map in Etienne's PTC. 

\index{Routine!type_layout}
\item[type_layout (lay)] \Newline
Subroutine to print the global information in a PTC layout.

\index{Routine!type_map1}
\item[type_map1 (y, type0, n_dim, style)] \Newline
Subroutine to type the transfer map up to first order. 

\index{Routine!type_fibre}
\item[type_fibre (fib)] \Newline
Subroutine to print the global information in a fibre.

\index{Routine!type_map}
\item[type_map (y)] \Newline
Subroutine to type the transfer maps of a real_8 array. 

\index{Routine!type_real_8_taylors}
\item[type_real_8_taylors (y, switch_z)] \Newline
Subroutine to type out the taylor series from a real_8 array. 

\index{Routine!taylor_to_genfield}
\item[taylor_to_genfield (bmad_taylor, gen_field, c0)] \Newline
Subroutine to construct a genfield (partially inverted map) from a taylor map. 

\index{Routine!universal_to_bmad_taylor}
\item[universal_to_bmad_taylor (u_taylor, bmad_taylor, switch_z)] \Newline
Subroutine to convert from a universal_taylor map in Etienne's PTC to a taylor map in Bmad. 

\index{Routine!vec_bmad_to_ptc}
\item[vec_bmad_to_ptc (vec_bmad, vec_ptc)] \Newline
Subroutine to convert from Bmad to PTC coordinates. 

\index{Routine!vec_ptc_to_bmad}
\item[vec_ptc_to_bmad (vec_ptc, vec_bmad)] \Newline
Subroutine to convert from PTC to Bmad coordinates. 

\end{description}

%------------------------------------------------------------------------
\section{Quick Plot Routines}
\label{r:qp}      
\index{Quick Plot!list of routines}

%--------------------------------------
\subsection{Page Routines}

\begin{description}

\index{Routine!qp_open_page}
\item[qp_open_page (page_type, i_chan, x_len, y_len, units)] \Newline 
     Subroutine to Initialize a page (window) for plotting.

\index{Routine!qp_select_page}
\item[qp_select_page (iw)] \Newline 
     Subroutine to switch to a particular page for drawing graphics.

\index{Routine!qp_close_page}
\item[qp_close_page] \Newline 
     Subroutine to finish plotting on a page.

\end{description}

%--------------------------------------
\subsection{Calculational Routines}

\begin{description}

\index{Routine!qp_axis_niceness}
\item[qp_axis_niceness (imin, imax, divisions) result (score)] \Newline 
Routine to calculate how ``nicely'' an axis will look.
The higher the score the nicer.

\index{Routine!qp_calc_and_set_axis}
\item[qp_calc_and_set_axis (axis, data_min, data_max, ... ] \Newline
     Subroutine to calculate a "nice" plot scale given the minimum and maximum
     of the data. 

\index{Routine!qp_calc_axis_params}
\item[\protect\parbox{6in}{qp_calc_axis_params (data_min, data_max, div_min, 
\\ \hspace*{2in} div_max, how, places, axis_min, axis_max, divisions)}] \Newline 
     Subroutine to calculate a "nice" plot scale given the minimum and maximum
     of the data. This is similar to calc_axis_scale.

\index{Routine!qp_calc_axis_divisions}
\item[qp_calc_axis_divisions (axis_min, axis_max, div_min, div_max, divisions)] \Newline 
Routine to calculate the best (gives the nicest looking drawing) number 
of major divisions for fixed axis minimum and maximum.

\index{Routine!qp_calc_axis_places}
\item[qp_calc_axis_places (axis_min, axis_max, divisions, places)] \Newline 
     Subroutine to calculate the number of decimal places needed to display the
     axis numbers.

\index{Routine!qp_calc_axis_scale}
\item[\protect\parbox{6in}{qp_calc_axis_scale (data_min, data_max, divisions, how,
\\ \hspace*{2in} places, axis_min, axis_max, niceness_score)}] \Newline 
     Subroutine to calculate a "nice" plot scale given the minimum and maximum
     of the data. 

\index{Routine!qp_calc_minor_div}
\item[qp_calc_minor_div (delta, div_max, divisions)] \Newline 
     Subroutine to calculate the number of minor divisions an axis should have.

\index{Routine!qp_convert_rectangle_rel}
\item[qp_convert_rectangle_rel (rect1, rect2)] \Newline 
     Subroutine to convert a "rectangle" (structure of 4 points) from
     one set of relative units to another

\end{description}

%--------------------------------------
\subsection{Drawing Routines}

\begin{description}

\index{Routine!qp_clear_page}
\item[qp_clear_page] \Newline 
     Subroutine to clear all drawing from the page.

\index{Routine!qp_draw_circle}
\item[qp_draw_circle (x0, y0, r, angle0, del_angle, ] \Newline 
                                   units, width, color, style, clip)
Subroutine to plot a section of an ellipse.

\index{Routine!qp_draw_ellipse}
\item[\protect\parbox{6in}{qp_draw_ellipse (x0, y0, r_x, r_y, theta_xy, \\
  \hspace*{2in} angle1, angle2, units, width, color, style, clip) }] \Newline 
     Subroutine to plot a section of an ellipse.

\index{Routine!qp_draw_axes}
\item[qp_draw_axes] \Newline 
     Subroutine to plot the axes, title, etc. of a plot.

\index{Routine!qp_draw_data}
\item[qp_draw_data (x, y, draw_line, symbol_every, clip)] \Newline
     Subroutine to plot data, axes with labels, a grid, and a title.

\index{Routine!qp_draw_graph}
\item[\protect\parbox{6in}{qp_draw_graph (x, y, x_lab, y_lab, title, \\
  \hspace*{2in} draw_line, draw_symbol, clip, symbol_every) }] \Newline 
     Subroutine to plot data, axes with labels, a grid, and a title.

\index{Routine!qp_draw_graph_title}
\item[qp_draw_graph_title (title)] \Newline 
     Subroutine to draw the title for a graph.

\index{Routine!qp_draw_grid}
\item[qp_draw_grid] \Newline 
     Subroutine to draw a grid on the current graph.

\index{Routine!qp_draw_histogram}
\item[qp_draw_histogram (x_dat, y_dat, x_lab, y_lab, title, draw_axes)] \Newline 
     Subroutine to plot data, axes with labels, a grid, and a title.

\index{Routine!qp_draw_legend}
\item[qp_draw_legend (lines, x, y, units)] \Newline 
     Subroutine to draw a legend.

\index{Routine!qp_draw_main_title}
\item[qp_draw_main_title (lines, justify)] \Newline 
     Subroutine to plot the main title at the top of the page.

\index{Routine!qp_draw_polyline}
\item[qp_draw_polyline (x, y, units, width, color, style, clip)] \Newline 
     Subroutine to draw a polyline.

\index{Routine!qp_draw_polyline_no_set}
\item[qp_draw_polyline_no_set (x, y, units)] \Newline 
Subroutine to draw a polyline.
This is similar to qp_draw_polyline except qp_set_line_attrib is not called.

\index{Routine!qp_draw_polyline_basic}
\item[qp_draw_polyline_basic (x, y, units) ] \Newline 
     Subroutine to draw a polyline. See also qp_draw_polyline

\index{Routine!qp_draw_line}
\item[qp_draw_line (x1, x2, y1, y2, units, width, color, style, clip)] \Newline 
     Subroutine to draw a line.

\index{Routine!qp_draw_rectangle}
\item[qp_draw_rectangle (x1, x2, y1, y2, units, color, width,
                                             style, clip) ] \Newline 
     Subroutine to draw a rectangular box.

\index{Routine!qp_draw_symbol}
\item[qp_draw_symbol (x, y, units, type, height, color, fill, line_width, clip)] \Newline 
     Draws a symbol at (x, y) 

\index{Routine!qp_draw_symbols}
\item[\protect\parbox{6in}{qp_draw_symbols (x, y, units, type, height, color, \\
  \hspace*{2in} fill, line_width, clip, symbol_every)} ] \Newline 
     Draws a symbol at the (x, y) points. 

\index{Routine!qp_draw_text}
\item[qp_draw_text (text, x, y, units, justify, height
        color, angle, ...) ] \Newline 
     Subroutine to draw text.

\index{Routine!qp_draw_text_no_set}
\item[qp_draw_text_no_set (text, x, y, units, justify, angle)] \Newline 
Subroutine to display on a plot a character string.
See also: qp_draw_text.

\index{Routine!qp_draw_text_basic}
\item[qp_draw_text_basic (text, x, y, units, justify, angle)] \Newline 
     Subroutine to display on a plot a character string.
     See also: qp_draw_text.

\index{Routine!qp_draw_x_axis}
\item[qp_draw_x_axis (who, y_pos)] \Newline 
     Subroutine to draw a horizontal axis.

\index{Routine!qp_draw_y_axis}
\item[qp_draw_y_axis (who, x_pos)] \Newline 
     Subroutine to draw a horizontal axis.

\index{Routine!qp_paint_rectangle}
\item[qp_paint_rectangle (x1, x2, y1, y2, units, color)] \Newline 
Subroutine to paint a rectangular region a specified color.
The default color is the background color (white\$).

\index{Routine!qp_to_axis_number_text}
\item[qp_to_axis_number_text (axis, ix_n, text)] \Newline 
     Subroutine to form the text string for an axis number.

\end{description}

%--------------------------------------
\subsection{Set Routines}

\begin{description}

\index{Routine!qp_calc_and_set_axis}
\item[qp_calc_and_set_axis (axis, data_min, data_max, ... ] \Newline
     Subroutine to calculate a "nice" plot scale given the minimum and maximum
     of the data. 

\index{Routine!qp_eliminate_xy_distortion}
\item[qp_eliminate_xy_distortion] \Newline 
This subroutine will increase the x or y margins so that the conversion
between data units and page units is the same for the x and y axes.

\index{Routine!qp_set_axis}
\item[qp_set_axis (axis, a_min, a_max, ...)] \Newline
    Subroutine to set (but not plot) the min, max and divisions for the axes of the graph.

\index{Routine!qp_set_box}
\item[qp_set_box (ix, iy, ix_tot, iy_tot) ] \Newline 
     Subroutine to set the box on the physical page.
     This routine divides the page into a grid of boxes. 

\index{Routine!qp_set_graph}
\item[qp_set_graph (title)] \Newline 
     Subroutine to set certain graph attributes.

\index{Routine!qp_set_graph_limits}
\item[qp_set_graph_limits] \Newline 
     Subroutine to calculate the offsets for the graph.
     This subroutine also sets the PGPLOT window size equal to the graph size.

\index{Routine!qp_set_graph_placement}
\item[qp_set_graph_placement (x1_marg, x_graph_len, y1_marg, ] \Newline 
                                                       y_graph_len, units)
Subroutine to set the placement of the current graph inside the box. 
This routine can be used in place of QP_SET_MARGIN.

\index{Routine!qp_set_layout}
\item[qp_set_layout (x_axis, y_axis, x2_axis, y2_axis, ...] \Newline 
     Subroutine to set various attributes. This routine can be used
     in place of other qp_set_* routines.

\index{Routine!qp_set_line}
\item[qp_set_line (who, line)] \Newline 
     Subroutine to set the default line attributes.

\index{Routine!qp_set_margin}
\item[qp_set_margin (x1_marg, x2_marg, y1_marg, y2_marg, units)] \Newline 
Subroutine to set up the margins from the sides of the box (see QP_SET_BOX)
to the edges of the actual graph.

\index{Routine!qp_set_page_border}
\item[qp_set_page_border (x1_b, x2_b, y1_b, y2_b, units)] \Newline 
     Subroutine to set the border around the physical page.

\index{Routine!qp_set_page_border_to_box}
\item[qp_set_page_border_to_box ()] \Newline 
Subroutine to set the page border to correspond to the region of the
current box. This allows qp_set_box to subdivide the current box.

\index{Routine!qp_set_clip}
\item[qp_set_clip (clip)] \Newline 
     Subroutine to set the default clipping state.

\index{Routine!qp_set_parameters}
\item[qp_set_parameters (text_scale)] \Newline 
Subroutine to set various quick_plot parameters.

\index{Routine!qp_subset_box}
\item[qp_subset_box (ix, iy, ix_tot, iy_tot, x_marg, y_marg)] \Newline 
     Subroutine to set the box for a graph. This is the same as
     qp_set_box but the boundaries of the page are taken to be the box boundaries.

\index{Routine!qp_set_symbol}
\item[qp_set_symbol (symbol)] \Newline 
     Subroutine to set the type and size of the symbols used in plotting data.
     See the pgplot documentation for more details.

\index{Routine!qp_set_symbol_attrib}
\item[qp_set_symbol_attrib (type, height, color, fill, line_width, clip)] \Newline 
     Subroutine to set the type and size of the symbols used in plotting data.

\index{Routine!qp_set_line_attrib}
\item[qp_set_line_attrib (who, width, color, style, clip)] \Newline 
     Subroutine to set the default line attributes.

\index{Routine!qp_set_graph_attrib}
\item[qp_set_graph_attrib (draw_grid, draw_title)] \Newline 
     Subroutine to set attributes of the current graph.

\index{Routine!qp_set_text_attrib}
\item[\protect\parbox{6in}{qp_set_text_attrib (who, height, color, \\
  \hspace*{2in} background, uniform_spacing, spacing_factor)} ] \Newline 
     Subroutine to set the default text attributes.

\index{Routine!qp_use_axis}
\item[qp_use_axis (x, y)] \Newline 
Subroutine to set what axis to use: X or X2, Y or Y2.

\end{description}

%--------------------------------------
\subsection{Informational Routines}

\begin{description}

\index{Routine!qp_get_axis}
\item[qp_get_axis (axis, a_min, a_max, div, ... ) ] \Newline
     Subroutine to get the min, max, divisions etc. for the X and Y axes.

\index{Routine!qp_get_layout_attrib}
\item[qp_get_layout_attrib (who, x1, x2, y1, y2, units)] \Newline 
     Subroutine to get the attributes of the layout.

\index{Routine!qp_get_parameters}
\item[qp_get_parameters (text_scale)] \Newline 
Subroutine to get various quick_plot parameters.

\index{Routine!qp_text_len}
\item[qp_text_len (text)] \Newline 
     Function to find the length of a text string.

\end{description}

%--------------------------------------
\subsection{Conversion Routines}

\begin{description}

\index{Routine!qp_from_inch_rel}
\item[qp_from_inch_rel (x_inch, y_inch, x, y, units)] \Newline 
     Subroutine to convert from a relative position (an offset) in inches
     to other units.

\index{Routine!qp_from_inch_abs}
\item[qp_from_inch_abs (x_inch, y_inch, x, y, units)] \Newline 
     Subroutine to convert to absolute position (x, y) from inches referenced
     to the Left Bottom corner of the page

\index{Routine!qp_text_height_to_inches}
\item[qp_text_height_to_inches(height_pt) result (height_inch)] \Newline 
Function to convert from a text height in points to a text height in
inches taking into account the text_scale.

\index{Routine!qp_to_inch_rel}
\item[qp_to_inch_rel (x, y, x_inch, y_inch, units)] \Newline 
Subroutine to convert a relative (x, y) into inches.

\index{Routine!qp_to_inch_abs}
\item[qp_to_inch_abs (x, y, x_inch, y_inch, units)] \Newline 
Subroutine to convert an absolute position (x, y) into inches referenced
to the Left Bottom corner of the page.

\index{Routine!qp_to_inches_rel}
\item[qp_to_inches_rel (x, y, x_inch, y_inch, units)] \Newline 
     Subroutine to convert a relative (x, y) into inches.

\index{Routine!qp_to_inches_abs}
\item[qp_to_inches_abs (x, y, x_inch, y_inch, units)] \Newline 
     Subroutine to convert an absolute position (x, y) into inches referenced
     to the left bottom corner of the page.

\end{description}

%--------------------------------------
\subsection{Miscellaneous Routines}

\begin{description}

\index{Routine!qp_read_data}
\item[qp_read_data (iu, err_flag, x, ix_col, y, iy_col, z, iz_col, 
                                                               t, it_col) ] \Newline 
     Subroutine to read columns of data.

\end{description}

%--------------------------------------
\subsection{Low Level Routines}

\begin{description}

\index{Routine!qp_clear_box_basic}
\item[qp_clear_box_basic (x1, x2, y1, y2, page_type)] \Newline 
Subroutine to clear all drawing from a box.
That is, white out the box region.

\index{Routine!qp_clear_page_basic}
\item[qp_clear_page_basic] \Newline 
Subroutine to clear all drawing from the page.

\index{Routine!qp_close_page_basic}
\item[qp_close_page_basic] \Newline 
Subroutine to finish plotting on a page.
For X this closes the window.

\item[qp_convert_point_rel (x_in, y_in, units_in, x_out, y_out, units_out)] \Newline 
Subroutine to convert a (x, y) point from from
one set of relative units to another.

\index{Routine!qp_convert_point_abs}
\item[qp_convert_point_abs (x_in, y_in, units_in, x_out, y_out, units_out)] \Newline 
Subroutine to convert a (x, y) point from from
one set of absolute units to another.

\index{Routine!qp_draw_symbol_basic}
\item[qp_draw_symbol_basic (x, y, symbol)] \Newline 
Subroutine to draw a symbol.

\index{Routine!qp_init_com_struct}
\item[qp_init_com_struct ] \Newline 
Subroutine to initialize the common block qp_state_struct.
This subroutine is not for general use.

\index{Routine!qp_join_units_string}
\item[qp_join_units_string (u_type, region, corner, units)] \Newline 
Subroutine to form a units from its components.

\index{Routine!qp_justify}
\item[qp_justify (justify)] \Newline 
     Function to convert a justify character string to a real value
     representing the horizontal justification. 

\index{Routine!qp_open_page_basic}
\item[qp_open_page_basic (page_type, x_len, y_len, plot_file] \Newline 
      x_page, y_page, i_chan)
Subroutine to Initialize a page (window) for plotting.

\index{Routine!qp_paint_rectangle_basic}
\item[qp_paint_rectangle_basic (x1, x2, y1, y2, color, page_type)] \Newline 
Subroutine to fill a rectangle with a given color. 
A color of white essentially eraces the rectangle.

\index{Routine!qp_pointer_to_axis}
\item[qp_pointer_to_axis (axis, axis_ptr)] \Newline 
Subroutine to return a pointer to an common block axis.

\index{Routine!qp_restore_state}
\item[qp_restore_state] \Newline 
     Subroutine to restore saved attributes. 
     Use qp_save_state to restore the saved state.

\index{Routine!qp_restore_state_basic}
\item[qp_restore_state_basic ()] \Newline 
Subroutine to restore the print state.

\index{Routine!qp_save_state}
\item[qp_save_state (buffer)] \Newline 
     Subroutine to save the current attributes. 
     Use qp_restore_state to restore the saved state.

\index{Routine!qp_save_state_basic }
\item[qp_save_state_basic ] \Newline 
Subroutine to save the print state.

\index{Routine!qp_select_page_basic}
\item[qp_select_page_basic (iw)] \Newline 
Subroutine to switch to a particular page for drawing graphics.

\index{Routine!qp_set_char_size_basic}
\item[qp_set_char_size_basic (height)] \Newline 
Subroutine to set the character size.

\index{Routine!qp_set_clip_basic}
\item[qp_set_clip_basic (clip)] \Newline 
Subroutine to set the clipping state.
Note: This affects both lines and symbols.

\index{Routine!qp_set_color_basic}
\item[qp_set_color_basic (ix_color, page_type)  ] \Newline 
Subroutine to set the color taking into accout that GIF
inverts the black for white.

\index{Routine!qp_set_graph_position_basic}
\item[qp_set_graph_position_basic (x1, x2, y1, y2)] \Newline 
Subroutine to set the position of a graph.
Units are inches from lower left of page.

\index{Routine!qp_set_line_width_basic}
\item[qp_set_line_width_basic (line_width)] \Newline 
Subroutine to set the line width.

\index{Routine!qp_set_line_style_basic}
\item[qp_set_line_style_basic (style)] \Newline 
Subroutine to set the line style.

\index{Routine!qp_set_symbol_fill_basic}
\item[qp_set_symbol_fill_basic (fill)] \Newline 
Subroutine to set the symbol fill style.

\index{Routine!qp_set_symbol_size_basic}
\item[qp_set_symbol_size_basic (height, symbol_type, page_type, uniform_size)] \Newline 
Subroutine to set the symbol_size

\index{Routine!qp_set_text_background_color_basic}
\item[qp_set_text_background_color_basic (color)] \Newline 
Subroutine to set the character text background color.

\index{Routine!qp_split_units_string}
\item[qp_split_units_string (u_type, region, corner, units)] \Newline 
     Subroutine to split a units string into its components.

\index{Routine!qp_text_len_basic}
\item[qp_text_len_basic (text, len_text)] \Newline 
Function to find the length of a text string.

\index{Routine!qp_translate_to_color_index}
\item[qp_translate_to_color_index (name, index)] \Newline 
     Subroutine to translate from a string to a color index.

\end{description}

%------------------------------------------------------------------------
\section{Spin Tracking}
\label{r:spin}    
\index{Spin tracking!list of routines}

\begin{description}

\index{Routine!spinor_to_polar}
\item[spinor_to_polar (coord, polar)] \Newline 
Subroutine to convert a spinor into polar coordinates.

\index{Routine!polar_to_vec}
\item[polar_to_vec (polar, vec)] \Newline
Subroutine to convert a spin vector from polar coordinates to cartesian coordinates.

\index{Routine!polar_to_spinor}
\item[polar_to_spinor (polar, coord)] \Newline
Subroutine to convert a spin vector in polar coordinates to a spinor.

\index{Routine!vec_to_polar}
\item[vec_to_polar (vec, polar, phase)] \Newline
Subroutine to convert a spin vector from cartesian coordinates to polar coordinates 
preserving the complex phase.

\index{Routine!spinor_to_vec}
\item[spinor_to_vec (coord, vec)] \Newline
Subroutine to convert a spinor to a spin vector in cartesian coordinates.

\index{Routine!vec_to_spinor}
\item[vec_to_spinor (vec, coord, phase)] \Newline
Subroutine to convert a spin vector in cartesian coordinates to a spinor using
the specified complex phase.

\index{Routine!angle_between_polars}
\item[angle_between_polars (polar1, polar2)] \Newline
Function to return the angle between two spin vectors in polar coordinates.

\index{Routine!quaternion_track}
\item[quaternion_track (a, start, end)] \Newline
Subrotuine to track the spin with the Euler four-vector quaternion a.

\index{Routine!track1_spin}
\item[track1_spin (start, ele, param, end)] \Newline
Subroutine to track the particle spin through one element.

\end{description}

%------------------------------------------------------------------------
\section{Transfer Maps: Routines Called by make_mat6}
\label{r:mat6}
 
\vn{Make_mat6} is the routine for calculating the transfer matrix (Jacobin)
through an element. The routines listed below are used by \vn{make_mat6}.
In general a program should call \vn{make_mat6} rather than using these
routines directly.

\begin{description}

\index{Routine!make_mat6_bmad}
\item[make_mat6_bmad (ele, param, c0, c1)] \Newline
Subroutine to make the 6x6 transfer matrix for an element
using closed formulas.

\index{Routine!make_mat6_custom}
\item[make_mat6_custom (ele, param, c0, c1)] \Newline
Dummy routine for making the 6x6 transfer matrices.

\index{Routine!make_mat6_symp_lie_ptc}
\item[make_mat6_symp_lie_ptc (ele, param, c0, c1)] \Newline
Subroutine to make the 6x6 transfer matrix for an element using
the PTC symplectic integrator.

\index{Routine!make_mat6_taylor}
\item[make_mat6_taylor (ele, param, c0, c1)] \Newline
Subroutine to make the 6x6 transfer matrix for an element
from a Taylor map.

\index{Routine!make_mat6_tracking}
\item[make_mat6_tracking (ele, param, c0, c1)] \Newline
Subroutine to make the 6x6 transfer matrix for an element by 
tracking 7 particle with different starting conditions.

\end{description}

%------------------------------------------------------------------------
\section{Transfer Maps: Taylor Maps}
\label{r:taylor}   
\index{Taylor Map!list of routines}

\begin{description}

\index{Routine!concat_taylor}
\item[concat_taylor (taylor1, taylor2, taylor3)] \Newline
Subroutine to concatenate two taylor series: taylor3(x) = taylor2(taylor1(x)) 

\index{Routine!ele_to_taylor}
\item[ele_to_taylor (ele, param, orb0)] \Newline
Subroutine to make a Taylor map for an element. The order of the map is set by set_ptc.

\index{Routine!equivalent_eles}
\item[equivalent_eles (ele1, ele2) result (equiv)] \Newline 
Subroutine to see if to elements are equivalent in terms of attributes so
that their Taylor Maps would be the same. 

\index{Routine!init_taylor}
\item[init_taylor_series (bmad_taylor, n_term)] \Newline
Subroutine to initialize a Bmad Taylor series. 

\index{Routine!kill_taylor}
\item[kill_taylor (bmad_taylor)] \Newline
Subroutine to deallocate a Bmad Taylor map. 

\index{Routine!mat6_to_taylor}
\item[mat6_to_taylor (mat6, vec0, bmad_taylor)] \Newline
Subroutine to form a first order Taylor map from the 6x6 transfer matrix 
and the 0th order transfer vector. 

\index{Routine!set_taylor_order}
\item[set_taylor_order (order, override_flag)] \Newline
Subroutine to set the taylor order. 

\index{Routine!sort_taylor_terms}
\item[sort_taylor_terms (taylor_in, taylor_sorted)] \Newline
Subroutine to sort the taylor terms from "lowest" to "highest" of a
Taylor series.

\index{Routine!taylor_coef}
\item[taylor_coef (bmad_taylor, exp)] \Newline 
Function to return the coefficient for a particular taylor term from a
Taylor Series.

\index{Routine!taylor_equal_taylor}
\item[taylor_equal_taylor (taylor1, taylor2)] \Newline
Subroutine to transfer the values from one taylor map to another:
Taylor1 $\le$ Taylor2

\index{Routine!taylors_equal_taylors}
\item[taylors_equal_taylors (taylor1, taylor2)] \Newline 
Subroutine to transfer the values from one taylor map to another.

\index{Routine!taylor_make_unit}
\item[taylor_make_unit (bmad_taylor)] \Newline
Subroutine to make the unit Taylor map

\index{Routine!taylor_to_mat6}
\item[taylor_to_mat6 (a_taylor, c0, mat6, c1)] \Newline
Subroutine to calculate the linear (Jacobian) matrix about some
trajectory from a Taylor map.

\index{Routine!taylor_inverse}
\item[taylor_inverse (taylor_in, taylor_inv)] \Newline
Subroutine to invert a taylor map. 

\index{Routine!taylor_propagate1}
\item[taylor_propagate1 (tlr, ele, param)] \Newline
Subroutine to track a real_8 taylor map through an element. 
The alternative routine, if ele has a taylor series, is concat_taylor. 

\index{Routine!track_taylor}
\item[track_taylor (start, bmad_taylor, end)] \Newline
Subroutine to track using a Taylor map. 

\index{Routine!transfer_ele_taylor}
\item[transfer_ele_taylor (ele_in, ele_out, taylor_order)] \Newline 
Subroutine to transfer a Taylor map from one element to another.

\index{Routine!transfer_lat_taylors}
\item[transfer_lat_taylors (lat_in, lat_out, 
                                             type_out, transfered_all) ] \Newline 
Subroutine to transfer the taylor maps from the elements of one lat to
the elements of another. 

\index{Routine!type_taylors}
\item[type_taylors (bmad_taylor)] \Newline
Subroutine to print in a nice format a Bmad taylor map at the terminal. 

\index{Routine!type2_taylors}
\item[type2_taylors (bmad_taylor, lines, n_lines)] \Newline
Subroutine to write a Bmad taylor map in a nice format to a character array. 

\end{description}

%------------------------------------------------------------------------
\section{Tracking and Closed Orbit}
\label{r:track}    
\index{Tracking!list of routines}

The following routines perform tracking and closed orbit calculations.

\begin{description}

\index{Routine!check_aperture_limit}
\item[check_aperture_limit (orb, ele, param)] \Newline
Subroutine to check if an orbit is outside the aperture. 

\index{Routine!closed_orbit_calc}
\item[closed_orbit_calc (lat, closed_orb, i_dim, direction)] \Newline 
Subroutine to calculate the closed orbit at the beginning of the lat.

\index{Routine!closed_orbit_from_tracking}
\item[closed_orbit_from_tracking (lat, closed_orb_, i_dim, 
eps_rel, eps_abs, init_guess)] \Newline
Subroutine to find the closed orbit via tracking. 

\index{Routine!compute_even_steps}
\item[compute_even_steps (ds_in, length, ds_default, ds_out, n_step)] \Newline 
Subroutine to compute a step size ds_out, close to ds_in, so that an 
integer number of steps spans the length.

\index{Routine!dynamic_aperture}
\item[dynamic_aperture (lat, track_input, aperture)] \Newline
Subroutine to determine the dynamic aperture of a lattice via tracking. 

\index{Routine!multi_turn_tracking_analysis}
\item[multi_turn_tracking_analysis (track, i_dim, track0, ele, 
stable, growth_rate, chi)] \Newline
Subroutine to analyze multi-turn tracking data to get the Twiss
parameters etc.

\index{Routine!multi_turn_tracking_to_mat}
\item[multi_turn_tracking_to_mat (track, i_dim, 
mat1, track0, chi)] \Newline
Subroutine to analyze 1-turn tracking data to find the 1-turn transfer
matrix and the closed orbit offset.

\index{Routine!offset_particle}
\item[\protect\parbox{6in}{offset_particle (ele, param, coord, set, set_canonical, \\
\hspace*{2in} set_tilt, set_multipoles, set_hvkicks, s_pos)}] \Newline
Subroutine to effectively offset an element by instead offsetting 
the particle position to correspond to the local element coordinates. 

\index{Routine!orbit_amplitude_calc}
\item[orbit_amplitude_calc (ele, orb, amp_a, amp_b, amp_na, amp_nb, particle)] \Newline
Routine to calculate the "invariant" amplitude of a particle at a 
particular point in its orbit. 

\index{Routine!setup_radiation_tracking}
\item[setup_radiation_tracking (lat, closed_orb, fluctuations_on, damping_on)] \Newline
Subroutine to compute synchrotron radiation parameters prior to tracking. 

\index{Routine!tilt_coords}
\item[tilt_coords (tilt_val coord, set)] \Newline
Subroutine to effectively tilt (rotate in the x-y plane) an element by 
instead rotating the particle position with negative the angle. 

\index{Routine!track1}
\item[track1 (start, ele, param, end)] \Newline
Subroutine to track through a single element. 

\index{Routine!track1_bunch_csr}
\item[track1_bunch_csr (bunch_start, lat, ix_ele, bunch_end)] \Newline 
Routine to track a bunch of particles through the element lat\%ele(ix_ele)
with csr radiation effects.

\index{Routine!track_all}
\item[track_all (lat, orbit)] \Newline
Subroutine to track through the lat. 

\index{Routine!track_many}
\item[track_many (lat, orbit_, ix_start, ix_end, direction)] \Newline
Subroutine to track from one element in the lat to another. 

\index{Routine!twiss_and_track}
\item[twiss_and_track (lat, orb)] \Newline
See the Twiss section for more details. 

\index{Routine!twiss_and_track_at_s}
\item[twiss_and_track_at_s (lat, s, ele, orb_, here)] \Newline
Subroutine to calculate the Twiss parameters and orbit at a particular longitudinal position. 

\index{Routine!twiss_and_track_partial}
\item[twiss_and_track_partial (ele1, ele2, param, del_s, ele3, start, end)] \Newline
Subroutine to calculate the Twiss parameters and orbit at a particular position inside an element. 

\index{Routine!twiss_from_tracking}
\item[twiss_from_tracking (lat, closed_orb_, d_orb, error)] \Newline
Subroutine to compute from tracking the Twiss parameters and the transfer matrices 
for every element in the lat. 

\end{description}

%------------------------------------------------------------------------
\section{Tracking: Low Level Routines}
\label{r:low.track}

\begin{description}

\index{Routine!odeint_bmad}
\item[odeint_bmad (start, ele, param, end, s1, s2, rel_tol, abs_tol, h1, hmin)] \Newline
Subroutine to do Runge Kutta tracking. 

\index{Routine!track1_boris_partial}
\item[track1_boris_partial (start, ele, param, s, ds, end)] \Newline
Subroutine to track 1 step using boris tracking. 
This subroutine is used by track1_boris and track1_adaptive_boris. 

\index{Routine!track_a_drift}
\item[track_a_drift (orb, length)] \Newline
Subroutine to track through a drift. 

\index{Routine!track_a_bend}
\item[track_a_bend (start, ele, param, end)] \Newline
Particle tracking through a bend element. 

\end{description}

%------------------------------------------------------------------------
\section{Tracking: Macroparticle}
\label{r:macro}    
\index{Macroparticle!list of routines}

See sections \sref{s:macro} and \sref{s:macro.track} for a discussion of macroparticles
and macroparticle tracking.

\begin{description}

\index{Routine!calc_macro_bunch_params}
\item[calc_macro_bunch_params (bunch, ele, params)] \Newline
Subroutine to calculate various beam characteristics from a bunch.

\index{Routine!init_macro_distribution}
\item[init_macro_distribution (beam, init, canonical_out)] \Newline 
Subroutine to initialize a macroparticle distribution.
This routine uses the LIAR algorithm. See the Bmad manual for more details.

\index{Routine!mat_to_mp_sigma}
\item[mat_to_mp_sigma (mat, sigma)] \Newline 
Subroutine to convert a sigma matrix. to a linear array of 
macroparticle sigmas.

\index{Routine!mp_sigma_to_mat}
\item[mp_sigma_to_mat (sigma, mat)] \Newline 
Subroutine to convert a linear array of macroparticle sigmas to a 
sigma matrix. 

\index{Routine!mp_to_angle_coords}
\item[mp_to_angle_coords (mp, energy0)] \Newline 
Subroutine to convert macroparticle coords from 
(x, px, y, py, z, pz) to (x, x', y, y', z, E).

\index{Routine!mp_to_canonical_coords}
\item[mp_to_canonical_coords (mp, energy0)] \Newline 
Subroutine to convert macroparticle coords from 
(x, x', y, y', z, E) to (x, px, y, py, z, pz).

\index{Routine!reallocate_macro_beam}
\item[reallocate_macro_beam (beam, n_bunch, n_slice, n_macro)] \Newline 
Subroutine to reallocate memory within a beam_struct.

\index{Routine!track1_macro_beam}
\item[track1_macro_beam (start, ele, param, end] \Newline
Subroutine to track a beam of macroparticles through an element.

\index{Routine!track_macro_beam}
\item[track_macro_beam (lat, beam, ix1, ix2)] \Newline 
Subroutine to track a beam of macroparticles from the end of
lat\%ele(ix1) Through to the end of lat\%ele(ix2).

\index{Routine!track1_macroparticle}
\item[track1_macroparticle (start, ele, param, end)] \Newline 
Subroutine to track a macroparticle through an element.

\end{description}

%------------------------------------------------------------------------
\section{Tracking: Mad Routines}
\label{r:mad}      

\begin{description}

\index{Routine!make_mat6_mad}
\item[make_mat6_mad (ele, param, map, c0, c1)] \Newline 
     Subroutine to make the 6x6 transfer matrix for an element from the 
     2nd order MAD transport map. The map is stored in ele\%taylor.

\index{Routine!make_mad_map}
\item[make_mad_map (ele, particle, map)] \Newline 
     Subroutine to make a 2nd order transport map a la MAD.

\index{Routine!mad_add_offsets_and_multipoles}
\item[mad_add_offsets_and_multipoles (ele, energy, map)] \Newline 
     Subroutine to add in the effect of element offsets and/or multipoles
     on the 2nd order transport map for the element.

\index{Routine!mad_drift}
\item[mad_drift (ele, energy, map)] \Newline 
     Subroutine to make a transport map for a drift space.
     The equivalent MAD-8 routine is: TMDRF

\index{Routine!mad_elsep}
\item[mad_elsep (ele, energy, map)] \Newline 
     Subroutine to make a transport map for an electric separator. 
     The equivalent MAD-8 routine is: TMSEP

\index{Routine!mad_sextupole}
\item[mad_sextupole (ele, energy, map)] \Newline 
     Subroutine to make a transport map for an sextupole.
     The equivalent MAD-8 routine is: TMSEXT

\index{Routine!mad_sbend}
\item[mad_sbend (ele, energy, map)] \Newline 
     Subroutine to make a transport map for a sector bend element.
     The equivalent MAD-8 routine is: TMBEND

\index{Routine!mad_sbend_fringe}
\item[mad_sbend_fringe (ele, energy, into, map)] \Newline 
     Subroutine to make a transport map for the fringe field of a dipole.
     The equivalent MAD-8 routine is: TMFRNG

\index{Routine!mad_sbend_body}
\item[mad_sbend_body (ele, energy, map)] \Newline 
     Subroutine to make a transport map for the body of a sector dipole.
     The equivalent MAD-8 routine is: TMSECT

\index{Routine!mad_tmfoc}
\item[mad_tmfoc (el, sk1, c, s, d, f)] \Newline 
     Subroutine to compute the linear focussing functions.  
     The equivalent MAD-8 routine is: TMFOC

\index{Routine!mad_quadrupole}
\item[mad_quadrupole (ele, energy, map)] \Newline 
     Subroutine to make a transport map for an quadrupole element.
     The equivalent MAD-8 routine is: TMSEXT

\index{Routine!mad_rfcavity}
\item[mad_rfcavity (ele, energy, map)] \Newline 
     Subroutine to make a transport map for an rfcavity element.
     The equivalent MAD-8 routine is: TMRF

\index{Routine!mad_solenoid}
\item[mad_solenoid (ele, energy, map)] \Newline 
     Subroutine to make a transport map for an solenoid.
     The equivalent MAD-8 routine is: TMSEXT

\index{Routine!mad_sol_quad}
\item[mad_sol_quad (ele, energy, map)] \Newline 
     Subroutine to make a transport map for a combination solenoid/quadrupole.
     Note: There is no equivalent MAD-8 routine.

\index{Routine!mad_tmsymm}
\item[mad_tmsymm (te)] \Newline 
     subroutine to symmetrize the 2nd order map t.
     The equivalent MAD-8 routine is: tmsymm

\index{Routine!mad_tmtilt}
\item[mad_tmtilt (map, tilt)] \Newline 
     Subroutine to apply a tilt to a transport map.
     The equivalent MAD-8 routine is: TMTILT

\index{Routine!mad_concat_map2}
\item[mad_concat_map2 (map1, map2, map3)] \Newline 
     Subroutine to concatenate two 2nd order transport maps.
         map3 = map2(map1)

\index{Routine!mad_track1}
\item[mad_track1 (c0, map, c1)] \Newline 
     Subroutine to track through a 2nd order transfer map.
     The equivalent MAD-8 routine is: TMTRAK

\index{Routine!track1_mad}
\item[track1_mad (start, ele, param, end)] \Newline 
     Subroutine to track through an element using a 2nd order transfer map.
     Note: If map does not exist then one will be created. 

\index{Routine!mad_map_to_taylor}
\item[mad_map_to_taylor (map, taylor)] \Newline 
     Subroutine to convert a mad order 2 map to a taylor map.

\index{Routine!taylor_to_mad_map}
\item[taylor_to_mad_map (taylor, map)] \Newline 
     Subroutine to convert a Taylor map to a mad order 2 map.
     If any of the Taylor terms have order greater than 2 they are ignored.

\index{Routine!make_unit_mad_map}
\item[make_unit_mad_map (map)] \Newline 
     Subroutine to initialize a 2nd order transport map to unity.


\end{description}

%------------------------------------------------------------------------
\section{Tracking: Routines called by TRACK1}
\label{r:track1}   

Note: Generally you don't call these routines directly.

\begin{description}

\index{Routine!symp_lie_bmad}
\item[symp_lie_bmad (ele, param, start, end, calc_mat6)] \Newline
Symplectic integration through an element to 0th or 1st order.

\index{Routine!track1_adaptive_boris}
\item[track1_adaptive_boris (start, ele, param, end, s_start, s_end)] \Newline
Subroutine to do Boris tracking with adaptive step size control. 

\index{Routine!track1_boris}
\item[track1_boris (start, ele, param, end, s_start, s_end)] \Newline
Subroutine to do Boris tracking.  

\index{Routine!track1_bmad}
\item[track1_bmad (start, ele, param, end)] \Newline
Particle tracking through a single element BMAD_standard style. 

\index{Routine!track1_custom}
\item[track1_custom (start, ele, param, end)] \Newline
Dummy routine for custom_tracking.

\index{Routine!track1_linear}
\item[track1_linear (start, ele, param, end)] \Newline
Particle tracking through a single element using the transfer matrix.. 

\index{Routine!track1_radiation}
\item[track1_radiation (start, ele, param, end, edge)] \Newline
Subroutine to put in radiation damping and/or fluctuations. 

\index{Routine!track1_runge_kutta}
\item[track1_runge_kutta (start, ele, param, end)] \Newline
Subroutine to do tracking using Runge-Kutta integration. 

\index{Routine!track1_symp_lie_ptc}
\item[track1_symp_lie_ptc (start, ele, param, end)] \Newline
Particle tracking through a single element using a Hamiltonian and a 
symplectic integrator. 

\index{Routine!track1_symp_map}
\item[track1_symp_map (start, ele, param, end)] \Newline
Particle tracking through a single element using a partially inverted 
taylor map (In PTC/FPP this is called a genfield). 

\index{Routine!track1_taylor}
\item[track1_taylor (start, ele, param, end)] \Newline
Subroutine to track through an element using the elements taylor series. 

\index{Routine!track1_wiedemann_wiggler}
\item[track1_wiedemann_wiggler (start, ele, param, end)] \Newline
Subroutine to track through the body of a wiggler. 

\end{description}

%------------------------------------------------------------------------
\section{Twiss and Other Calculations}
\label{r:twiss}
\index{Twiss!list of routines}

\begin{description}

\index{Routine!calc_z_tune}
\item[calc_z_tune (lat)] \Newline
Subroutine to calculate the synchrotron tune from the full 6X6 1 turn matrix. 

\index{Routine!chrom_calc}
\item[chrom_calc (lat, delta_e, chrom_x, chrom_y)] \Newline
Subroutine to calculate the chromaticities by computing the tune 
change when then energy is changed. 

\index{Routine!chrom_tune}
\item[chrom_tune (lat, delta_e, target_x, target_y, err_flag)] \Newline
Subroutine to set the sextupole strengths so that the lat 
has the desired chromaticities. 

\index{Routine!quad_beta_ave}
\item[quad_beta_ave (lat, ix_ele, beta_x_ave, beta_y_ave)] \Newline
Subroutine to compute the average betas in a quad.

\index{Routine!radiation_integrals}
\item[radiation_integrals (lat, orb_, mode)] \Newline
Subroutine to calculate the synchrotron radiation integrals, the emittance, and energy spread. 

\index{Routine!relative_mode_flip}
\item[relative_mode_flip (ele1, ele2)] \Newline
Function to see if the modes of ELE1 are flipped relative to ELE2. 

\index{Routine!set_tune}
\item[set_tune (phi_x_set, phi_y_set, dk1, lat, orb_, ok)] \Newline
Subroutine to Q_tune a lat. This routine will set the tunes to within 0.001 radian (0.06 deg). 

\index{Routine!set_z_tune}
\item[set_z_tune (lat)] \Newline
Subroutine to set the longitudinal tune by setting the RF voltages in the RF cavities. 

\index{Routine!twiss_and_track}
\item[twiss_and_track (lat, orb)] \Newline
Subroutine to calculate the Twiss and orbit parameters. 
This is not necessarily the fastest routine. 

\index{Routine!twiss_and_track_partial}
\item[twiss_and_track_partial (ele1, ele2, param, del_s, ele3, start, end)] \Newline
Subroutine to propagate partially through ELE2 the Twiss parameters and the orbit. 

\index{Routine!twiss_at_element}
\item[twiss_at_element (lat, ix_ele, start, end, average)] \Newline
Subroutine to return the Twiss parameters at the beginning, end, or the average of an element. 

\index{Routine!twiss_and_track_at_s}
\item[twiss_and_track_at_s (lat, s, ele, orb_, here)] \Newline
Subroutine to calculate the Twiss parameters and orbit at a particular longitudinal position. 

\index{Routine!twiss_at_start}
\item[twiss_at_start (lat)] \Newline
Subroutine to calculate the Twiss parameters at the start of the lat. 

\index{Routine!twiss_from_tracking}
\item[twiss_from_tracking (lat, closed_orb_, d_orb, error)] \Newline
Subroutine to compute from tracking, for every element in the lat, 
the Twiss parameters and the transfer matrices. 

\index{Routine!twiss_propagate1}
\item[twiss_propagate1 (ele1, ele2)] \Newline
Subroutine to propagate the Twiss parameters from the end of ELE1 to the end of ELE2. 

\index{Routine!twiss_propagate_all}
\item[twiss_propagate_all (lat, set_match)] \Newline
Subroutine to propagate the Twiss parameters from the start to the end. 

\index{Routine!twiss_propagate_many}
\item[twiss_propagate_many (lat, ix_start, ix_end, direction)] \Newline
Subroutine to propagate the Twiss parameters from one element in the lat to another. 

\index{Routine!twiss_to_1_turn_mat}
\item[twiss_to_1_turn_mat (twiss, phi, mat2)] \Newline
Subroutine to form the 2x2 1-turn transfer matrix from the Twiss parameters. 

\end{description}

%------------------------------------------------------------------------
\section{Twiss: 6 Dimensional}
\label{r:twiss6}    
\index{Twiss!list of routines}

\begin{description}

\index{Routine!normal_mode3_calc}
\item[normal_mode3_calc (mat, tune, g, v, synchrotron_motion)] \Newline 
Decompose a 2n x 2n sympectic matrix into normal modes.
For more details see:

\index{Routine!twiss3_propagate_all}
\item[twiss3_propagate_all (lat)] \Newline 
Subroutine to propagate the twiss parameters using all three normal modes.

\index{Routine!twiss3_propagate1}
\item[twiss3_propagate1 (ele1, ele2)] \Newline 
Subroutine to propagate the twiss parameters using all three normal modes.

\index{Routine!twiss3_at_start}
\item[twiss3_at_start (lat)] \Newline 
Subroutine to propagate the twiss parameters using all three normal modes.


\end{description}

%------------------------------------------------------------------------
\section{Wake Fields}
\label{r:wake}    
\index{Wake fields!list of routines}

\begin{description}

\index{Routine!init_wake}
\item[init_wake (wake, n_sr_table, n_sr_mode_long, n_sr_mode_trans, n_lr)] \Newline 
Subroutine to initialize a wake struct.

\index{Routine!lr_wake_apply_kick}
\item[lr_wake_apply_kick (ele, s_ref, orbit)] \Newline 
Subroutine to apply the long-range wake kick to a particle.

\index{Routine!sr_table_apply_trans_kick}
\item[sr_table_apply_trans_kick (ele, leader, charge, follower)] \Newline 
Subroutine to put in the kick for the short-range wakes.

\index{Routine!sr_mode_long_wake_add_to}
\item[sr_mode_long_wake_add_to (ele, orbit, charge)] \Newline 
Subroutine to add to the existing short-range wake the contribution from
a passing (macro)particle.

\index{Routine!sr_mode_long_wake_apply_kick}
\item[sr_mode_long_wake_apply_kick (ele, orbit)] \Newline 
Subroutine to put in the kick for the short-range wakes.

\index{Routine!sr_mode_long_self_wake_apply_kick}
\item[sr_mode_long_self_wake_apply_kick (ele, charge, orbit)] \Newline 
Subroutine to put in the kick for the short-range wakes

\index{Routine!sr_mode_trans_wake_add_to}
\item[sr_mode_trans_wake_add_to (ele, orbit, charge)] \Newline 
Subroutine to add to the existing short-range wake the contribution from
a passing (macro)particle.

\index{Routine!sr_mode_trans_wake_apply_kick}
\item[sr_mode_trans_wake_apply_kick (ele, orbit)] \Newline 
Subroutine to put in the kick for the short-range wakes

\index{Routine!track1_sr_wake}
\item[track1_sr_wake (bunch, ele)] \Newline 
Subroutine to apply the short range wake fields to a bunch. 

\index{Routine!track1_lr_wake}
\item[track1_lr_wake (bunch, ele)] \Newline 
Subroutine to put in the long-range wakes for particle tracking.

\end{description}

