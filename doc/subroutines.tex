\chapter{Bmad Library Routine List}

Below are a list of \bmad and sim_utils routines sorted by their
functionality.  Use the \vn{getf} and \vn{listf} (\sref{s:getf})
scripts for more information on individual routines.
This list includes low level routines that are not generally used in
writing code for a program but may be useful in certain unique
situations.  Excluded from the list are very low level routines that are
solely meant for \bmad internal use.

\toffset
\begin{center}
\begin{tabular}{ll} \toprule
{\em Routine Type} & {\em Section} \\ \midrule
  Beam: Low Level Routines                    & \ref{r:low.beam}       \\
  Beam: Tracking and Manipulation             & \ref{r:beam}           \\
  Branch Handling                             & \ref{r:branch}         \\
  Coherent Synchrotron Radiation (CSR)        & \ref{r:csr}            \\
  Collective Effects                          & \ref{r:collective}     \\
  Custom and Hook Routines                    & \ref{r:custom}         \\
  Electro-Magnetic Fields                     & \ref{r:em.fields}      \\
  Helper Routines: File, System, and IO       & \ref{r:helper.file}    \\
  Helper Routines: Math (Except Matrix)       & \ref{r:helper.math}    \\
  Helper Routines: Matrix                     & \ref{r:helper.matrix}  \\
  Helper Routines: Miscellaneous              & \ref{r:helper.misc}    \\
  Helper Routines: String Manipulation        & \ref{r:helper.string}  \\
  Helper Routines: Switch to Name             & \ref{r:switch}         \\
  Inter-Beam Scattering (IBS)                 & \ref{r:ibs}            \\
  Lattice: Informational                      & \ref{r:info}           \\
  Lattice: Element Manipulation               & \ref{r:elem}           \\
  Lattice: Geometry                           & \ref{r:geom}           \\
  Lattice: Low Level Stuff                    & \ref{r:lat.low}        \\
  Lattice: Manipulation                       & \ref{r:trans}          \\
  Lattice: Miscellaneous                      & \ref{r:lat.misc}       \\
  Lattice: Reading and Writing Files          & \ref{r:read}           \\
  Matrices                                    & \ref{r:mat}            \\
  Matrix: Low Level Routines                  & \ref{r:low.mat}        \\
  Measurement Simulation Routines             & \ref{r:meas}           \\
  Multipass                                   & \ref{r:multipass}      \\
  Multipoles                                  & \ref{r:multipoles}     \\
  Optimizers (Nonlinear)                      & \ref{r:opti}           \\
  Overload Equal Sign                         & \ref{r:equal}          \\
  Particle Coordinate Stuff                   & \ref{r:coord}          \\
  Photon Routines                             & \ref{r:photon}         \\
  PTC Interface                               & \ref{r:ptc}            \\
  Quick Plot                                  & \ref{r:qp}             \\
  Spin                                        & \ref{r:spin}           \\
  Transfer Maps: Routines Called by make_mat6 & \ref{r:mat6}           \\
  Transfer Maps: Complex Taylor Maps          & \ref{r:ctaylor}        \\
  Transfer Maps: Taylor Maps                  & \ref{r:taylor}         \\
  Tracking: Tracking and Closed Orbit         & \ref{r:track}          \\
  Tracking: Low Level Routines                & \ref{r:low.track}      \\
  Tracking: Mad Routines                      & \ref{r:mad}            \\
  Tracking: Routines Called by track1         & \ref{r:track1}         \\
  Twiss and Other Calculations                & \ref{r:twiss}          \\
  Twiss: 6-Dimensional                        & \ref{r:twiss6}         \\
  Wake Fields                                 & \ref{r:wake}           \\
  C/C++ Interface                             & \ref{r:c.interface}    \\
  % Deprecated                                & \ref{r:deprecated}     \\ \bottomrule
\end{tabular}
\end{center}
\toffset

%------------------------------------------------------------------------
\section{Beam: Low Level Routines}
\label{r:low.beam}

The following helper routines are generally not useful for general use.

\begin{description}

\index[routine]{bend_edge_kick}
\label{r:bend.edge.kick}
\item[bend_edge_kick (ele, param, particle_at, orb, mat6, make_matrix, track_spin)] \Newline 
Subroutine to track through the edge field of an sbend.
Reverse tracking starts with the particle just outside the bend and

\index[routine]{find_bunch_sigma_matrix}
\label{r:find.bunch.sigma.matrix}
\item[find_bunch_sigma_matrix (particle, charge, bunch_params, sigma_s)] \Newline 
Routine to find the sigma matrix elements of a particle distribution.

\index[routine]{init_spin_distribution}
\label{r:init.spin.distribution}
\item[init_spin_distribution (beam_init, bunch)] \Newline 
Initializes a spin distribution according to init_beam\%spin

\index[routine]{order_particles_in_z}
\label{r:order.particles.in.z}
\item[order_particles_in_z (bunch)] \Newline 
Routine to order the particles longitudinally in terms of decreasing \%vec(5).
That is from large z (head of bunch) to small z.

\index[routine]{track1_beam}
\label{r:track1.beam}
\item[track1_beam (beam_start, lat, ele, beam_end, err, centroid, direction)] \Newline 
Routine to track a beam of particles through a single element.
Overloaded by \vn{track1_beam}.

\index[routine]{track1_bunch}
\label{r:track1.bunch}
\item[track1_bunch (bunch_start, lat, ele, bunch_end, err, centroid, direction)] \Newline 
Routine to track a bunch of particles through an element.

\index[routine]{track1_bunch_hom}
\label{r:track1.bunch.hom}
\item[track1_bunch_hom (bunch_start, ele, param, bunch_end, direction)] \Newline 
Routine to track a bunch of particles through an element.

\end{description}

%------------------------------------------------------------------------
\section{Beam: Tracking and Manipulation}
\label{r:beam}    
\index{beam tracking!list of routines}

See \sref{s:part.track} for a discussion of using a collection of particles to simulate
a bunch.

\begin{description}

\index[routine]{bbi_kick}
\label{r:bbi.kick}
\item[bbi_kick (x_norm, y_norm, r, kx, ky)] \Newline 
Routine to compute the normalized kick due to the beam-beam
interaction using the normalized position for input.

\index[routine]{calc_bunch_params}
\label{r:calc.bunch.params}
\item[calc_bunch_params (bunch, bunch_params, err, print_err)] \Newline 
Finds all bunch parameters defined in bunch_params_struct, both normal-mode
and projected

\index[routine]{calc_bunch_params_slice}
\label{r:calc.bunch.params.slice}
\item[\protect\parbox{6in}{
    calc_bunch_params (bunch, bunch_params, plane, slice_center, \\
    \hspace*{1in} slice_spread, err, print_err) }] \Newline 
Finds all bunch parameters for a slice through the beam distribution.

\index[routine]{init_beam_distribution}
\label{r:init.beam.distribution}
\item[init_beam_distribution (ele, param, beam_init, beam, err_flag)] \Newline 
Routine to initialize a distribution of particles matched to
the Twiss parameters, centroid position, and Energy - z correlation

\index[routine]{init_bunch_distribution}
\label{r:init.bunch.distribution}
\item[init_bunch_distribution (ele, param, beam_init, ix_bunch, bunch, err_flag)] \Newline 
Routine to initialize either a random or tail-weighted distribution of particles.  

\index[routine]{reallocate_beam}
\label{r:reallocate.beam}
\item[reallocate_beam (beam, n_bunch, n_particle)] \Newline 
Routine to reallocate memory within a beam_struct.

\index[routine]{reallocate_bunch}
\label{r:reallocate.bunch}
\item[reallocate_bunch (bunch, n_particle)] \Newline 
Subroutine to reallocate particles within a bunch_struct.

\index[routine]{track_beam}
\label{r:track.beam}
\item[track_beam (lat, beam, ele1, ele2, err, centroid, direction)] \Newline 
     Routine to track a beam of particles from the end of
     lat\%ele(ix1) Through to the end of lat\%ele(ix2).

\end{description}

%------------------------------------------------------------------------
\section{Branch Handling Routines}
\label{r:branch}

\begin{description}

\index[routine]{allocate_branch_array}
\label{r:allocate.branch.array}
\item[allocate_branch_array (lat, upper_bound)] \Newline 
Routine to allocate or re-allocate an branch array.
The old information is saved.

\index[routine]{transfer_branch}
\label{r:transfer.branch}
\item[transfer_branch (branch1, branch2)] \Newline 
Routine to set branch2 = branch1. 
This is a plain transfer of information not using the overloaded equal.

\index[routine]{transfer_branches}
\label{r:transfer.branches}
\item[transfer_branches (branch1, branch2)] \Newline 
Routine to set branch2 = branch1. 
This is a plain transfer of information not using the overloaded equal.

\end{description}

%------------------------------------------------------------------------
\section{Coherent Synchrotron Radiation (CSR)}
\label{r:csr}

\begin{description}

\index[routine]{csr_bin_particles}
\label{r:csr.bin.particles}
\item[csr_bin_particles (particle, csr)] \Newline 
Routine to bin the particles longitudinally in s. 

\index[routine]{csr_bin_kicks}
\label{r:csr.bin.kicks}
\item[csr_bin_kicks (ds_kick_pt, csr, err_flag)] \Newline 
Routine to cache intermediate values needed for the csr calculations.

\index[routine]{csr_kick_calc}
\label{r:csr.kick.calc}
\item[csr_kick_calc (csr, particle)] \Newline 
Routine to calculate the longitudinal coherent synchrotron radiation kick.

\index[routine]{i_csr}
\label{r:i.csr}
\item[i_csr (kick1, i_bin, csr) result (i_this)] \Newline 
Routine to calculate the CSR kick integral.

\index[routine]{z_calc_csr}
\label{r:z.calc.csr}
\item[z_calc_csr (d, k_factor, bin, small_angle_approx, dz_dd) result (z_this)] \Newline 
Routine to calculate the distance between the source particle and the
kicked particle.

\index[routine]{d_calc_csr}
\label{r:d.calc.csr}
\item[d_calc_csr (dz_particles, k_factor, bin, small_angle_approx) result (d_this)] \Newline 
Routine to calculate the distance between source and kick points.

\end{description}

%------------------------------------------------------------------------
\section{Collective Effects}
\label{r:collective}

\begin{description}

\index[routine]{setup_ultra_rel_space_charge_calc}
\label{r:setup.trans.space.charge.calc}
\item[setup_ultra_rel_space_charge_calc (calc_on, lattice, n_part, mode, closed_orb)] \Newline 
Routine to initialize constants needed by the transverse space charge 
tracking routine track1_space_charge.  

\index[routine]{touschek_lifetime}
\label{r:touschek.lifetime}
\item[touschek_lifetime (mode, Tl, lat)] \Newline
Routine to calculate the Touschek lifetime for a lat.

\end{description}

%------------------------------------------------------------------------
\section{Custom Routines}
\label{r:custom}

\begin{description}

\index[routine]{apply_element_edge_kick_hook}
\label{r:apply.element.edge.kick.hook}
\item[\protect\parbox{6in}{
  apply_element_edge_kick_hook (orb, fringe_info, track_ele, param, \\
  \hspace*{1in} finished, mat6, make_matrix, rf_time) }] \Newline 

Routine that can be customized to track through the edge field of an element.
This routine is always called by apply_element_edge_kick.

\index[routine]{check_aperture_limit_custom}
\label{r:check.aperture.limit.custom}
\item[check_aperture_limit_custom (orb, ele, particle_at, param, err_flag)] \Newline
Routine to check if an orbit is outside an element's aperture.
Used when \vn{ele%aperture_type} is set to \vn{custom\$} 

\index[routine]{ele_geometry_hook}
\label{r:ele.geometry.hook}
\item[ele_geometry_hook (floor0, ele, floor, finished, len_scale)] \Newline 
Routine that can be customized to calculate the floor position of an element.

\index[routine]{ele_to_fibre_hook}
\label{r:ele.to.fibre.hook}
\item[ele_to_fibre_hook (ele, ptc_fibre, param)] \Newline 
Routine that can be customized for creating a PTC fibre from a Bmad element.
This routine is always called by ele_to_fibre.

\index[routine]{em_field_custom}
\label{r:em.field.custom}
\item[\protect\parbox{6in}{
  em_field_custom(ele, param, s_rel, orb, local_ref_frame, field, \\
  \hspace*{1in} calc_dfield, err_flag) }] \Newline
Custom routine for calculating fields.

\index[routine]{init_custom}
\label{r:init.custom}
\item[init_custom (ele, err_flag)] \Newline
Routine for initializing custom elements or elements that do custom
calculations.

\index[routine]{make_mat6_custom}
\label{r:make.mat6.custom}
\item[make_mat6_custom (ele, param, start_orb, end_orb, err_flag)] \Newline
Routine for custom calculations of the 6x6 transfer matrices.

\index[routine]{radiation_integrals_custom}
\label{r:radiation.integrals.custom}
\item[radiation_integrals_custom (lat, ir, orb, err_flag)] \Newline
User supplied routine to calculate the synchrotron radiation integrals for
a custom element.

\index[routine]{time_runge_kutta_periodic_kick_hook}
\label{r:time.runge.kutta.periodic.kick.hook}
\item[time_runge_kutta_periodic_kick_hook (orbit, ele, param, stop_time, init_needed)] \Newline 
Custom routine to add a kick to a particle at periodic times.

\index[routine]{track1_beam_hook}
\label{r:track1.beam.hook}
\item[track1_beam_hook (beam_start, lat, ele, beam_end, err, centroid, direction, finished)] \Newline 
Routine that can be customized for tracking a beam through a single element.

\index[routine]{track1_custom}
\label{r:track1.custom}
    \item[track1_custom (start_orb, ele, param, end_orb, err_flag, finished, track)] \Newline
Dummy routine for custom tracking.

\index[routine]{track1_postprocess}
\label{r:track1.postprocess}
\item[track1_postprocess (start_orb, ele, param, end_orb)] \Newline 
Dummy routine for post processing after the track1 routine is done.

\index[routine]{track1_preprocess}
\label{r:track1.preprocess}
\item[track1_preprocess (start_orb, ele, param, err_flag, finished, radiation_included, track)] \Newline 
Dummy routine for pre processing at the start of the track1 routine.

\index[routine]{track1_spin_custom}
\label{r:track1.spin.custom}
\item[track1_spin_custom (start, ele, param, end, err_flag, track)] \Newline 
Dummy routine for custom spin tracking. 
This routine needs to be replaced for a custom calculation.

\index[routine]{track1_wake_hook}
\label{r:track1.wake.hook}
\item[track1_wake_hook (bunch, ele, finished)] \Newline 
Routine that can be customized for tracking through a wake.

\index[routine]{wall_hit_handler_custom}
\label{r:wall.hit.handler.custom}
\item[wall_hit_handler_custom (orb, ele, s)] \Newline 
This routine is called by the Runge-Kutta integrator odeint_bmad when a particle hits a wall.

\end{description}

%------------------------------------------------------------------------
\section{Electro-Magnetic Fields}
\label{r:em.fields}     

\begin{description}

\index[routine]{em_field_calc}
\label{r:em.field.calc}
\item[\protect\parbox{6in}{
    em_field_calc (ele, param, s_pos, orbit, local_ref_frame, field, calc_dfield, err_flag, \\
    \hspace*{0.5in} potential, use_overlap, grid_allow_s_out_of_bounds, rf_time, used_eles) }] \Newline 
Routine to calculate the E and B fields for an element.

\index[routine]{em_field_custom}
\item[\protect\parbox{6in}{
  em_field_custom(ele, param, s_rel, time, orb, local_ref_frame, field, \\
  \hspace*{1in} calc_dfield, err_flag) }] \Newline
Custom routine for calculating fields.

\end{description}

%------------------------------------------------------------------------
\section{Helper Routines: File, System, and IO}
\label{r:helper.file}

\begin{description}

\index[routine]{append_subdirectory}
\label{r:append.subdirectory}
\item[append_subdirectory (dir, sub_dir, dir_out, err)] \Newline 
Routine to combine a directory specification with a 
subdirectory specification to form a complete directory

\index[routine]{cesr_iargc}
\label{r:cesr.iargc}
\item[cesr_iargc ()] \Newline 
Platform independent function to return the number of command
line arguments. Use this with cesr_getarg.

\index[routine]{cesr_getarg}
\label{r:cesr.getarg}
\item[cesr_getarg (i_arg, arg)] \Newline 
Platform independent function to return the i'th command
line argument. Use this with cesr_iargc.

\index[routine]{dir_close}
\label{r:dir.close}
\item[dir_close () ] \Newline 
Routine to close a directory that was opened with dir_open.
Also see dir_read.

\index[routine]{dir_open}
\label{r:dir.open}
\item[dir_open (dir_name) result (opened)] \Newline 
Routine to open a directory to obtain a list of its files.
Use this routine with dir_read and dir_close.

\index[routine]{dir_read}
\label{r:dir.read}
\item[dir_read (file_name) result (valid)] \Newline 
Routine to get the names of the files in a directory.
Use this routine with dir_open and dir_close.

\index[routine]{file_suffixer}
\label{r:file.suffixer}
\item[file_suffixer (in_file_name, out_file_name, suffix, add_switch)] \Newline 
Routine to add/replace a suffix to a file name.

\index[routine]{get_tty_char}
\label{r:get.tty.char}
\item[get_tty_char (this_char, wait, flush)] \Newline 
Routine for getting a single character from the terminal.
Also see: get_a_char

\index[routine]{get_a_char}
\label{r:get.a.char}
\item[get_a_char (this_char, wait, ignore_this)] \Newline 
Routine for getting a single character from the terminal.
Also see: get_tty_char

\index[routine]{get_file_time_stamp}
\label{r:get.file.time.stamp}
\item[get_file_time_stamp (file, time_stamp)] \Newline 
Routine to get the "last modified" time stamp for a file.

\index[routine]{lunget}
\label{r:lunget}
\item[lunget()] \Newline 
Function to return a free file unit number to be used with an open statement.

\index[routine]{milli_sleep}
\label{r:milli.sleep}
\item[milli_sleep (milli_sec)] \Newline 
Routine to pause the program for a given number of milli-seconds.

\index[routine]{out_io}
\label{r:out.io}
\item[out_io (...)] \Newline 
Routine to print to the terminal for command line type programs.
The idea is that for programs with a gui this routine can be easily
replaced with another routine.

\index[routine]{out_io_called}
\label{r:out.io.called}
\item[out_io_called (level, routine_name)] \Newline 
Dummy routine for linker.
See out_io for more details.

\index[routine]{out_io_end}
\label{r:out.io.end}
\item[out_io_end ()] \Newline 
Dummy routine for linker.
See out_io for more details.

\index[routine]{out_io_line}
\label{r:out.io.line}
\item[out_io_line (line)] \Newline 
Dummy routine for linker.
See out_io for more details.

\index[routine]{output_direct}
\label{r:output.direct}
\item[output_direct (file_unit, do_print, post_process, min_level, max_level)] \Newline 
Routine to set where the output goes when out_io is called.
Output may be sent to the terminal screen, written to a file, or both.
Also can be used to restrict output verbosity.

\index[routine]{read_a_line}
\label{r:read.a.line}
\item[read_a_line (prompt, line_out, trim_prompt, prompt_color, prompt_bold)] \Newline 
Routine to read a line of input from the terminal.
The line is also add to the history buffer so that the up-arrow

\index[routine]{skip_header}
\label{r:skip.header}
\item[skip_header (ix_unit, error_flag)] \Newline 
Routine to find the first line of data in a file. 

\index[routine]{splitfilename}
\label{r:splitfilename}
\item[splitfilename(filename, path, basename, is_relative) result (ix_char)] \Newline 
Routine to take filename and splits it into its constituent parts, 
the directory path and the base file name.  

\index[routine]{system_command}
\label{r:system.command}
\item[system_command (line)] \Newline 
Routine to execute an operating system command from within the program.

\index[routine]{type_this_file}
\label{r:type.this.file}
\item[type_this_file (filename)] \Newline 
Routine to type out a file to the screen.

\end{description}

%------------------------------------------------------------------------
\section{Helper Routines: Math (Except Matrix)}
\label{r:helper.math}

\begin{description}

\index[routine]{complex_error_function}
\label{r:complex.error.function}
\item[complex_error_function (wr, wi, zr, zi)] \Newline 
This routine evaluates the function w(z) in the first quadrant of
the complex plane. 

\index[routine]{cross_product}
\label{r:cross.product}
\item[cross_product (a, b)] \Newline 
Returns the cross product of a x b

\index[routine]{linear_fit}
\label{r:linear.fit}
\item[linear_fit (x, y, n_data, a, b, sig_a, sig_b)] \Newline 
Routine to fit to y = A + B x

\index[routine]{modulo2}
\label{r:modulo2}
\item[modulo2 (x, amp)] \Newline 
Function to return y = x + 2 * n * amp, n is an integer, such that y is 
in the interval [-amp, amp].

\index[routine]{ran_engine}
\label{r:ran.engine}
\item[ran_engine (set, get, ran_state)] \Newline 
Routine to set what random number generator algorithm is used.
If this routine is never called then pseudo_random\$ is used.

\index[routine]{ran_gauss}
\label{r:ran.gauss}
\item[ran_gauss (harvest)] \Newline 
Routine to return a Gaussian distributed random number with unit sigma.

\index[routine]{ran_gauss_converter}
\label{r:ran.gauss.converter}
\item[ran_gauss_converter (set, set_sigma_cut, get, get_sigma_cut, ran_state)] \Newline 
Routine to set what conversion routine is used for converting
uniformly distributed random numbers to Gaussian distributed random numbers.

\index[routine]{ran_seed_put}
\label{r:ran.seed.put}
\item[ran_seed_put (seed, ran_state)] \Newline 
Routine to seed the random number generator. 

\index[routine]{ran_seed_get}
\label{r:ran.seed.get}
\item[ran_seed_get (seed, ran_state)] \Newline 
Routine to return the seed used for the random number generator.

\index[routine]{ran_uniform}
\label{r:ran.uniform}
\item[ran_uniform (harvest)] \Newline 
Routine to return a random number uniformly distributed in the 
interval [0, 1]. This routine uses the same algorithm as ran from

\index[routine]{spline_akima}
\label{r:spline.akima}
\item[spline_akima (spline, ok)] \Newline 
Given a set of (x,y) points we want to interpolate between the points.
This routine computes the semi-hermite cubic spline developed by akima

\index[routine]{spline_evaluate}
\label{r:spline.evaluate}
\item[spline_evaluate (spline, x, ok, y, dy)] \Newline 
Routine to evaluate a spline at a set of points.

\index[routine]{super_ludcmp}
\label{r:super.ludcmp}
\item[super_ludcmp (a,indx,d, err)] \Newline 
This routine is essentially ludcmp from Numerical Recipes with the added feature
that an error flag is set instead of bombing the program when there is a problem.

\end{description}

%------------------------------------------------------------------------
\section{Helper Routines: Matrix}
\label{r:helper.matrix}

\begin{description}

\index[routine]{mat_eigen}
\label{r:mat.eigen}
\item[mat_eigen (mat, eigen_val, eigen_vec, error, print_err)] \Newline 
Routine for determining the eigen vectors and eigen values of a matrix.

\index[routine]{mat_inverse}
\label{r:mat.inverse}
\item[mat_inverse (mat, mat_inv, ok, print_err)] \Newline
Routine to take the inverse of a square matrix. 

\index[routine]{mat_make_unit}
\label{r:mat.make.unit}
\item[mat_make_unit (mat)] \Newline 
     routine to create a unit matrix.

\index[routine]{mat_rotation}
\label{r:mat.rotation}
\item[mat_rotation (mat, angle, bet_1, bet_2, alph_1, alph_2)] \Newline 
     Routine to construct a 2x2 rotation matrix for translation from
     point 1 to point 2.

\index[routine]{mat_symplectify}
\label{r:mat.symplectify}
\item[mat_symplectify (mat_in, mat_symp, p0_ratio, r_root)] \Newline
Routine to form a symplectic matrix that is approximately equal to the input matrix. 

\index[routine]{mat_symp_error}
\label{r:mat.symp.error}
\item[mat_symp_error (mat, p0_ratio, err_mat) result (error)] \Newline
Routine to check the symplecticity of a square matrix 

\index[routine]{mat_symp_conj}
\label{r:mat.symp.conj}
\item[mat_symp_conj (mat) result (mat_conj)] \Newline 
Routine to take the symplectic conjugate of a square matrix.

\index[routine]{mat_symp_decouple}
\label{r:mat.symp.decouple}
\item[\protect\parbox{6in}{
    mat_symp_decouple (t0, stat, u, v, \\
    \hspace*{1in} ubar, vbar, g, twiss1, twiss2, gamma, type_out)} ] \Newline
Routine to find the symplectic eigen--modes of the one turn 4x4 coupled 
transfer matrix T0. 

\index[routine]{mat_type}
\label{r:mat.type}
\item[mat_type (mat, nunit, header, num_form)] \Newline 
     Routine to output matrices to the terminal or to a file

\end{description}

%------------------------------------------------------------------------
\section{Helper Routines: Miscellaneous}
\label{r:helper.misc}

\begin{description}

\index[routine]{date_and_time_stamp}
\label{r:date.and.time.stamp}
\item[date_and_time_stamp (string, numeric_month)] \Newline 
Routine to return the current date and time in a character string.

\index[routine]{err_exit}
\label{r:err.exit}
\item[err_exit()] \Newline 
Routine to first show the stack call list before exiting.
This routine is typically used when a program detects an error condition.

\index[routine]{integer_option}
\label{r:integer.option}
\item[integer_option (integer_default, opt_integer)] \Newline 
Function to return True or False depending upon the state of an 
optional integer.

\index[routine]{is_false}
\label{r:is.false}
\item[is_false (param) result (this_false)] \Newline 
Routine to translate from a real number to a boolian True or False.
Translation: 0 = False, nonzero = True.

\index[routine]{is_true}
\label{r:is.true}
\item[is_true (param) result (this_true)] \Newline 
Routine to translate from a real number to a boolian True or False.
Translation: 0 = False, nonzero = True.

\index[routine]{logic_option}
\label{r:logic.option}
\item[logic_option (logic_default, opt_logic)] \Newline 
Function to return True or False depending upon the state of an 
optional logical.

\index[routine]{re_allocate}
\label{r:re.allocate}
\item[re_allocate (ptr_to_array, n, exact)] \Newline 
Function to reallocate a pointer to an array of strings, integers, reals, or logicals.

\index[routine]{re_associate}
\label{r:re.associate}
\item[re_associate (array, n)] \Newline 
Function to reassociate an allocatable array of strings, integers, reals, or logicals.

\index[routine]{real_option}
\label{r:real.option}
\item[real_option (real_default, opt_real)] \Newline 
Function to return True or False depending upon the state of an 
optional real.

\index[routine]{string_option}
\label{r:string.option}
\item[string_option (string_out, string_default, opt_string)] \Newline 
Routine to return True or False depending upon the state of an 
optional string.

\end{description}

%------------------------------------------------------------------------
\section{Helper Routines: String Manipulation}
\label{r:helper.string}

\begin{description}

\index[routine]{downcase_string}
\label{r:downcase.string}
\item[downcase_string (string)] \Newline 
Routine to convert a string to lowercase:

\index[routine]{indexx_char}
\label{r:indexx.char}
\item[indexx_char (arr,index)] \Newline 
Routine to sort a character array.
This routine is used to overload the generic name indexx.

\index[routine]{index_nocase}
\label{r:index.nocase}
\item[index_nocase (string, match_str) result (indx)] \Newline 
Function to look for a sub-string of string that matches match_str.
This routine is similar to the fortran INDEX function

\index[routine]{is_integer}
\label{r:is.integer}
\item[is_integer (string)] \Newline 
Function to tell if the first word in a string is a valid integer.

\index[routine]{is_logical}
\label{r:is.logical}
\item[is_logical (string, ignore) result (good)] \Newline 
Function to test if a string represents a logical.
Accepted possibilities are (individual characters can be either case):

\index[routine]{is_real}
\label{r:is.real}
\item[is_real (string, ignore) result (good)] \Newline 
Function to test if a string represents a real number.

\index[routine]{match_reg}
\label{r:match.reg}
\item[match_reg (str, pat)] \Newline 
Function for matching with regular expressions.
Note: strings are trimmed before comparison.

\index[routine]{match_wild}
\label{r:match.wild}
\item[match_wild (string, template) result (this_match)] \Newline 
Function to do wild card matches. Note: trailing blanks will be discarded
before any matching is done.

\index[routine]{match_word}
\label{r:match.word}
\item[match_word (string, names, ix, exact_case, can_abbreviate, matched_name)] \Newline 
Routine to match the first word in a string against a list of names.
Abbreviations are accepted.  

\index[routine]{on_off_logic}
\label{r:on.off.logic}
\item[on_off_logic (logic) result (name)] \Newline 
Function to return the string "ON" or "OFF".

\index[routine]{str_match_wild}
\label{r:str.match.wild}
\item[str_match_wild(str, pat) result (a_match)] \Newline 
Function to match a character string against a regular expression pattern.

\index[routine]{string_to_int}
\label{r:string.to.int}
\item[string_to_int (line, default, value, err_flag)] \Newline 
Routine to convert a string to an integer.


\index[routine]{string_to_real}
\label{r:string.to.real}
\item[string_to_real (line, default, value, err_flag)] \Newline 
Routine to convert a string to an real.

\index[routine]{string_trim}
\label{r:string.trim}
\item[string_trim(in_string, out_string, word_len)] \Newline 
Routine to trim a string of leading blanks and/or tabs and also to return the
length of the first word.

\index[routine]{string_trim2}
\label{r:string.trim2}
\item[string_trim2 (in_str, delimitors, out_str, ix_word, delim, ix_next)] \Newline 
Routine to trim a string of leading delimiters and also to return the
length of the first word.

\index[routine]{str_downcase}
\label{r:str.downcase}
\item[str_downcase (dst, src)] \Newline 
Routine to convert a string to down case.

\index[routine]{str_substitute}
\label{r:str.substitute}
\item[str_substitute (string, str_match, str_replace, do_trim)] \Newline 
Routine to substitute all instances of one sub-string for another in a string

\index[routine]{upcase}
\label{r:upcase}
\item[upcase (str_in) result (str_out)] \Newline 
Routine to convert a string to upper case.

\index[routine]{upcase_string}
\label{r:upcase.string}
\item[upcase_string (string)] \Newline 
Routine to convert a string to uppercase:

\end{description}

%------------------------------------------------------------------------
\section{Helper Routines: Switch to Name}
\label{r:switch}

\begin{description}

\index[routine]{coord_state_name}
\label{r:coord.state.name}
\item[coord_state_name (coord_state) result (state_str)] \Newline 
Routine to return the string representation of a coord\%state state.

\end{description}

%------------------------------------------------------------------------
\section{Inter-Beam Scattering (IBS)}
\label{r:ibs}

\begin{description}

\index[routine]{ibs_lifetime}
\label{r:ibs.lifetime}
\item[ibs_lifetime(lat,ibs_sim_params,maxratio,lifetime,granularity)] \Newline 
 This module computes the beam lifetime due to
 the diffusion process according to equation 12

\end{description}

%------------------------------------------------------------------------
\section{Lattice: Element Manipulation}
\label{r:elem}     

These routine are for adding elements, moving elements, etc.

\begin{description}

\index[routine]{add_lattice_control_structs}
\label{r:add.lattice.control.structs}
\item[\protect\parbox{6in}{
  add_lattice_control_structs (ele, n_add_slave, n_add_lord, n_add_slave_field, \\
  \hspace*{1in} n_add_lord_field, add_at_end) }] \Newline 
Routine to adjust the control structure of a lat so that extra control elements can be added.

\index[routine]{add_superimpose}
\label{r:add.superimpose}
\item[\protect\parbox{6in}{
    add_superimpose (lat, super_ele_in, ix_branch, err_flag, super_ele_out, \\
    \hspace*{1in} save_null_drift, create_jumbo_slave, ix_insert)} ] \Newline
Routine to make a superimposed element. 

\index[routine]{attribute_bookkeeper}
\label{r:attribute.bookkeeper}
\item[attribute_bookkeeper (ele, param, force_bookkeeping)] \Newline
Routine to make sure the attributes of an element are self-consistent. 

\index[routine]{autoscale_phase_and_amp}
\label{r:autoscale.phase.and.amp}
\item[\protect\parbox{6in}{
      autoscale_phase_and_amp(ele, param, err_flag, scale_phase, \\
      \hspace*{1in} scale_amp, call_bookkeeper)} ] \Newline 
Routine to set the phase offset and amplitude scale of the accelerating field. 
This routine works on lcavity, rfcavity and e_gun elements.

\index[routine]{create_element_slice}
\label{r:create.element.slice}
\item[\protect\parbox{6in}{
    create_element_slice (sliced_ele, ele_in, l_slice, offset, param, \\
    \hspace*{0.5in} include_upstream_end, include_downstream_end, err_flag, old_slice)} ] \Newline 
Routine to transfer the \%value, \%wig_term, and \%wake\%lr information from a 
superposition lord to a slave when the slave has only one lord.

\index[routine]{create_field_overlap}
\label{r:create.field.overlap}
\item[create_field_overlap (lat, lord_name, slave_name, err_flag)] \Newline 
Subroutine to add the bookkeeping information to a lattice for an element's field
overlapping another element.

\index[routine]{create_group}
\label{r:create.group}
\item[create_group (lord, contrl, err, err_print_flag)] \Newline
Routine to create a group control element. 

\index[routine]{create_girder}
\label{r:create.girder}
\item[create_girder (lat, ix_girder, contrl, girder_info, err_flag)] \Newline 
     Routine to add the controller information to slave elements of
     an girder_lord.

\index[routine]{create_overlay}
\label{r:create.overlay}
\item[create_overlay (lord, contrl, err, err_print_flag)] \Newline
Routine to add the controller information to slave elements of an 
overlay_lord. 

\index[routine]{create_wiggler_model}
\label{r:create.wiggler.model}
\item[create_wiggler_model (wiggler_in, lat)] \Newline 
Routine to create series of bend and drift elements to serve as a model for a wiggler.
This routine uses the mrqmin nonlinear optimizer to vary the parameters in the wiggler 

\index[routine]{insert_element}
\label{r:insert.element}
\item[insert_element (lat, insert_ele, insert_index, ix_branch, orbit)] \Newline
Routine to Insert a new element into the tracking part of the 
lat structure. 

\index[routine]{make_hybrid_lat}
\label{r:make.hybrid.lat}
\item[make_hybrid_lat (lat_in, lat_out, use_taylor, orb0_arr)] \Newline
Routine to concatenate together elements to make a hybrid lat 

\index[routine]{new_control}
\label{r:new.control}
\item[new_control (lat, ix_ele)] \Newline
Routine to create a new control element. 

\index[routine]{pointer_to_attribute}
\label{r:pointer.to.attribute}
\item[\protect\parbox{6in}{
  pointer_to_attribute (ele, attrib_name, do_allocation, \\
  \hspace*{1in} a_ptr, err_flag, err_print_flag, ix_attrib)}] \Newline
Returns a pointer to an attribute of an element with name attrib_name. 

\index[routine]{pointers_to_attribute}
\label{r:pointers.to.attribute}
\item[\protect\parbox{6in}{
    pointers_to_attribute (lat, ele_name, attrib_name, do_allocation, \\
    \hspace*{1in} ptr_array, err_flag, err_print_flag, eles, ix_attrib)} ] \Newline 
Returns an array of pointers to an attribute with name attrib_name within 
elements with name ele_name.

\index[routine]{pointer_to_branch}
\label{r:pointer.to.branch}
\item[pointer_to_branch] \Newline 
Routine to return a pointer to a lattice branch.

\index[routine]{pointer_to_next_ele}
\label{r:pointer.to.next.ele}
\item[pointer_to_next_ele (this_ele, offset, skip_beginning, follow_fork) result (next_ele)] \Newline 
Function to return a pointer to the N\^th element relative to this_ele
in the array of elements in a lattice branch.

\index[routine]{pointer_to_ele}
\label{r:pointer.to.ele}
\item[\protect\parbox{6in}{
  pointer_to_ele (lat, ix_ele, ix_branch) result (ele_ptr) \\
  pointer_to_ele (lat, ele_loc_id) result (ele_ptr)
  }] \Newline 
Routine to point to a given element.

\index[routine]{pointer_to_element_at_s}
\label{r:pointer.to.element.at.s}
\item[pointer_to_element_at_s (branch, s, choose_max, err_flag, s_eff, position) result (ele)] \Newline 
Function to return a pointer to the element at position s.

\index[routine]{remove_eles_from_lat}
\label{r:remove.eles.from.lat}
\item[remove_eles_from_lat (lat, check_sanity)] \Newline 
Routine to remove an elements from the lattice.

\index[routine]{set_ele_attribute}
\label{r:set.ele.attribute}
\item[set_ele_attribute (ele, set_string, lat, err_flag, err_print_flag)] \Newline 
Routine to set an element's attribute.

\index[routine]{set_ele_status_stale}
\label{r:set.ele.status.stale}
\item[set_ele_status_stale (ele, status_group, set_slaves)] \Newline 
Routine to set a status flags to stale in an element and the corresponding 
ones for any slaves the element has.

\index[routine]{set_status_flags}
\label{r:set.status.flags}
\item[set_status_flags (bookkeeping_state, stat)] \Newline 
Routine to set the bookkeeping status block.

\index[routine]{split_lat}
\label{r:split.lat}
\item[\protect\parbox{6in}{
    split_lat (lat, s_split, ix_branch, ix_split, split_done, \\
    \hspace*{1in} add_suffix, check_sanity, save_null_drift, err_flag)} ] \Newline
Routine to split a lat at a point.

\index[routine]{value_of_attribute}
\label{r:value.of.attribute}
\item[value_of_attribute (ele, attrib_name, err_flag, err_print_flag) result (value)] \Newline 
Returns the value of an element attribute.

\end{description}

%------------------------------------------------------------------------
\section{Lattice: Geometry}
\label{r:geom}     
\index{global coordinates!list of routines}

\begin{description}

\index[routine]{ele_geometry}
\label{r:ele.geometry}
\item[ele_geometry (floor0, ele, floor, len_scale, set_ok)] \Newline 
Routine to calculate the physical (floor) placement of an element given the
placement of the preceding element. This is the same as the MAD convention.

\index[routine]{floor_angles_to_w_mat}
\label{r:floor.angles.to.w.mat}
\item[floor_angles_to_w_mat (theta, phi, psi, w_mat, w_mat_inv)] \Newline 
Routine to construct the W matrix that specifies the orientation of an element
in the global "floor" coordinates. See the Bmad manual for more details.

\index[routine]{coords_floor_to_relative}
\label{r:floor.to.local}
\item[\protect\parbox{6in}{
    coords_floor_to_relative (floor0, global_position, \\
    \hspace*{1in} calculate_angles, is_delta_position) result (local_position)} ] \Newline 
Returns local floor position relative to floor0 given a global floor position.
This is an essentially an inverse of routine coords_relative_to_floor.

\index[routine]{floor_w_mat_to_angles}
\label{r:floor.w.mat.to.angles}
\item[floor_w_mat_to_angles (w_mat, theta, phi, psi, floor0)] \Newline 
Routine to construct the angles that define the orientation of an element
in the global "floor" coordinates from the W matrix. See the Bmad manual for more details.

\index[routine]{lat_geometry}
\label{r:lat.geometry}
\item[lat_geometry (lat)] \Newline
Routine to calculate the physical placement of all the elements in a lattice. 
That is, the physical machine layout on the floor. 

\index[routine]{coords_relative_to_floor}
\label{r:local.to.floor}
\item[coords_relative_to_floor (floor0, dr, theta, phi, psi) result (floor1)] \Newline 
Starting from a given reference frame and given a shift in position, return
the resulting reference frame.

\index[routine]{patch_flips_propagation_direction}
\label{r:patch.flips.propagation.direction}
\item[patch_flips_propagation_direction (x_pitch, y_pitch) result (is_flip)] \Newline 
Routine to determine if the propagation direction is flipped in a patch.
This is true if the tranformation matrix element S(3,3) = cos(x_pitch) * cos(y_pitch) 

\index[routine]{coords_local_curvilinear_to_floor}
\label{r:position.in.global.frame}
\item[\protect\parbox{6in}{
  coords_local_curvilinear_to_floor (local_position, ele, in_ele_frame, \\
  \hspace*{1in} w_mat, calculate_angles) result (global_position) }] \Newline 
Given a position local to ele, return global floor coordinates.

\index[routine]{coords_floor_to_local_curvilinear}
\label{r:position.in.local.frame}
\item[\protect\parbox{6in}{
      coords_floor_to_local_curvilinear  (global_position, ele, status, w_mat) \\
      \hspace*{1in} \hfill result(local_position)} ] \Newline 
Given a position in global coordinates, return local curvilinear coordinates in ele
  relative to floor0

\index[routine]{s_calc}
\label{r:s.calc}
\item[s_calc (lat)] \Newline
Routine to calculate the longitudinal distance S for the elements in a lat. 

\index[routine]{w_mat_for_x_pitch}
\label{r:w.mat.for.x.pitch}
\item[w_mat_for_x_pitch (x_pitch, return_inverse)] \Newline 
Routine to return the transformation matrix for an x_pitch.

\index[routine]{w_mat_for_y_pitch}
\label{r:w.mat.for.y.pitch}
\item[w_mat_for_y_pitch (y_pitch, return_inverse)] \Newline 
Routine to return the transformation matrix for an y_pitch.

\index[routine]{w_mat_for_tilt}
\label{r:w.mat.for.tilt}
\item[w_mat_for_tilt (tilt, return_inverse)] \Newline 
Routine to return the transformation matrix for an tilt.

\end{description}

%------------------------------------------------------------------------
\section{Lattice: Informational}
\label{r:info}     

\begin{description}

\index[routine]{attribute_free}
\label{r:attribute.free}
\item[\protect\parbox{6in}{
  attribute_free (ix_ele, attrib_name, lat, err_print_flag, except_overlay) result (free) \\
  attribute_free (ele, attrib_name, lat, err_print_flag, except_overlay) result (free) \\
  attribute_free (ix_ele, ix_branch, attrib_name, lat, err_print_flag, except_overlay) result (free)
  }] \Newline
Overloaded function to check if an attribute is free to vary.

\index[routine]{attribute_index}
\label{r:attribute.index}
\item[attribute_index (ele, name, full_name)] \Newline
Function to return the index of an attribute for a given element 
type and the name of the attribute 

\index[routine]{attribute_name}
\label{r:attribute.name}
\item[attribute_name (ele, ix_att)] \Newline
Function to return the name of an attribute for a particular type of element. 

\index[routine]{attribute_type}
\label{r:attribute.type}
\item[attribute_type (attrib_name) result (attrib_type)] \Newline 
Routine to return the type (logical, integer, real, or named) of an attribute.

\index[routine]{branch_name}
\label{r:branch.name}
\item[branch_name(branch) result (name)] \Newline 
Routine to return a string with the lattice branch name encoded.
This routine is useful for error messages.

\index[routine]{check_if_s_in_bounds}
\label{r:check.if.s.in.bounds}
\item[check_if_s_in_bounds (branch, s, err_flag, translated_s)] \Newline 
Routine to check if a given longitudinal position s is within the bounds of a given branch of a lattice.

\index[routine]{lat_sanity_check}
\label{r:lat.sanity.check}
\item[lat_sanity_check (lat, err_flag)] \Newline
Routine to check the control links in a lat structure, etc.

\index[routine]{element_at_s}
\label{r:element.at.s}
\item[element_at_s (lat, s, choose_max, ix_branch, err_flag, s_eff, position) result (ix_ele)] \Newline 
Routine to return the index of the element at position s.

\index[routine]{ele_has_offset}
\label{r:ele.has.offset}
\item[ele_has_offset (ele) result (has_offset)] \Newline 
Function to tell if an element has a non-zero offset, pitch or tilt.

\index[routine]{ele_loc_to_string}
\label{r:ele.loc.to.string}
\item[ele_loc_to_string (ele, show_branch0) result (str)] \Newline 
Routine to encode an element's location into a string.

\index[routine]{ele_to_lat_loc}
\label{r:ele.to.lat.loc}
\item[ele_to_lat_loc (ele) result (ele_loc)] \Newline 
Function to return an lat_ele_loc_struct identifying where an element is in the lattice.

\index[routine]{equivalent_taylor_attributes}
\item[equivalent_taylor_attributes (ele_taylor, ele2) result (equiv)] \Newline 
Routine to see if two elements are equivalent in terms of their attributes so
that their Taylor Maps, if they existed, would be the same.

\index[routine]{find_element_ends}
\label{r:find.element.ends}
\item[find_element_ends (ele, ele1, ele2, ix_multipass)] \Newline
Routine to find the end points of an element. 

\index[routine]{get_slave_list}
\label{r:get.slave.list}
\item[get_slave_list (lord, slaves, n_slave)] \Newline 
Routine to get the list of slaves for an element.

\index[routine]{key_name}
\label{r:key.name}
\item[{key_name (key_index)}] \Newline
Translate an element key index (EG: quadrupole\$, etc.) to a character string.

\index[routine]{key_name_to_key_index}
\label{r:key.name.to.key.index}
\item[key_name_to_key_index (key_str, abbrev_allowed) result (key_index)] \Newline 
Function to convert a character string  (eg: "drift") to an index (eg: drift\$).

\index[routine]{lat_ele_locator}
\label{r:lat.ele.locator}
\item[lat_ele_locator (loc_str, lat, eles, n_loc, err, 
            above_ubound_is_err, ix_dflt_branch)] \Newline 
Routine to locate all the elements in a lattice that corresponds to loc_str. 

\index[routine]{n_attrib_string_max_len}
\label{r:n.attrib.string.max.len}
\item[n_attrib_string_max_len () result (max_len)] \Newline 
Routine to return the the maximum number of characters in any attribute
name known to bmad.

\index[routine]{name_to_list}
\label{r:name.to.list}
\item[name_to_list (lat, ele_names)] \Newline
Routine to make a list of the elements in a lat 
whose name matches the names in the ele_names list. 

\item[num_lords (slave, lord_type) result (num)] \Newline 
Routine to return the number of lords of a lattice element of a certain type.

\index[routine]{pointer_to_indexed_attribute}
\label{r:pointer.to.indexed.attribute}
\item[\protect\parbox{6in}{
  pointer_to_indexed_attribute (ele, ix_attrib, do_allocation, \\
  \hspace*{1in} a_ptr, err_flag, err_print_flag)} ] \Newline 
Returns a pointer to an attribute of an element ele with attribute index ix_attrib.

\index[routine]{pointer_to_lord}
\label{r:pointer.to.lord}
\item[pointer_to_lord (slave, ix_lord, control, ix_slave, field_overlap_ptr) result (lord_ptr)] \Newline 
Function to point to a lord of a slave.

\index[routine]{pointer_to_multipass_lord}
\label{r:pointer.to.multipass.lord}
\item[pointer_to_multipass_lord (ele, ix_pass, super_lord) result (multi_lord)] \Newline 
Routine to find the multipass lord of a lattice element.
A multi_lord will be found for:

\index[routine]{pointer_to_slave}
\label{r:pointer.to.slave}
\item[pointer_to_slave (lord, ix_slave, control, field_overlap_ptr) result (slave_ptr)] \Newline 
Function to point to a slave of a lord.

\index[routine]{rf_is_on}
\label{r:rf.is.on}
\item[rf_is_on (branch) result (is_on)] \Newline 
Routine to check if any rfcavity is powered in a branch.

\index[routine]{switch_attrib_value_name}
\label{r:switch.attrib.value.name}
\hspace*{1in} 
\item[\protect\parbox{6in}{
      switch_attrib_value_name (attrib_name, attrib_value, ele, is_default, name_list) \\
      \hfill result (val_name)} ] \Newline 
Routine to return the name corresponding to the value of a given switch attribute.

\index[routine]{type_ele}
\label{r:type.ele}
\item[\protect\parbox{6in}{type_ele (ele, type_zero_attrib, type_mat6, type_taylor, \\
\hspace*{1in} twiss_out, type_control, type_wake, type_floor_coords, \\
\hspace*{1in} type_field, type_wall, lines, n_lines)}] \Newline
Subroutine to print or put in a string array information on a lattice element.

\index[routine]{type_spin_taylors}
\label{r:type.spin.taylors}
\item[type_spin_taylors (spin_taylor, max_order, lines, n_lines, file_id)] \Newline 
Subroutine to print or put in a string array a Bmad spin taylor map.

\index[routine]{type_twiss}
\label{r:type.twiss}
\item[type_twiss (ele, frequency_units, compact_format, lines, n_lines)] \Newline
Subroutine to print or put in a string array Twiss information from an element.

\index[routine]{valid_tracking_method}
\label{r:valid.tracking.method}
\item[valid_tracking_method (ele, species, tracking_method, num_valid) result (is_valid)] \Newline 
Routine to return whether a given tracking method is valid for a given element.

\index[routine]{valid_mat6_calc_method}
\label{r:valid.mat6.calc.method}
\item[\protect\parbox{6in}{
    valid_mat6_calc_method (ele, species, mat6_calc_method, \\
    \hspace*{1in} num_valid) result (is_valid)} ] \Newline 
Routine to return whether a given mat6_calc method is valid for a given element.

\end{description}

%------------------------------------------------------------------------
\section{Lattice: Low Level Stuff}
\label{r:lat.low} 

\begin{description}

\index[routine]{bracket_index}
\label{r:bracket.index}
\item[bracket_index (s_arr, i_min, i_max, s, ix)] \Newline
Routine to find the index ix so that s(ix) $\le$ s $<$ s(ix+1). 
If s $<$ s(1) then ix = 0 

\index[routine]{check_controller_controls}
\label{r:check.controller.controls}
\item[check_controller_controls (contrl, name, err)] \Newline 
Routine to check for problems when setting up group or overlay controllers.

\index[routine]{deallocate_ele_pointers}
\label{r:deallocate.ele.pointers}
\item[deallocate_ele_pointers (ele, nullify_only, nullify_branch, dealloc_poles)] \Newline
Routine to deallocate the pointers in an element. 

\index[routine]{re_allocate_eles}
\label{r:re.allocate.eles}
\item[re_allocate_eles (eles, n, save_old, exact)] \Newline 
Routine to allocate an array of ele_pointer_structs.

\index[routine]{twiss1_propagate}
\label{r:twiss1.propagate}
\item[twiss1_propagate (twiss1, mat2, ele_key, length, twiss2, err)] \Newline 
Routine to propagate the twiss parameters of a single mode.

\end{description}

%------------------------------------------------------------------------
\section{Lattice: Manipulation}
\label{r:trans}    

\begin{description}

\index[routine]{allocate_element_array}
\label{r:allocate.element.array}
\item[allocate_element_array (ele, upper_bound, init_ele0)] \Newline 
Routine to allocate or re-allocate an element array.

\index[routine]{allocate_lat_ele_array}
\label{r:allocate.lat.ele.array}
\item[allocate_lat_ele_array (lat, upper_bound, ix_branch)] \Newline 
Routine to allocate or re-allocate an element array.

\index[routine]{control_bookkeeper}
\label{r:control.bookkeeper}
\item[control_bookkeeper (lat, ele, dummy, err_flag)] \Newline
Routine to calculate the combined strength of the attributes for
controlled elements.

\index[routine]{deallocate_ele_array_pointers}
\label{r:deallocate.ele.array.pointers}
\item[deallocate_ele_array_pointers (eles)] \Newline 
Routine to deallocate the pointers of all the elements in an 
element array and the array itself.

\index[routine]{deallocate_lat_pointers}
\label{r:deallocate.lat.pointers}
\item[deallocate_lat_pointers (lat)] \Newline 
Routine to deallocate the pointers in a lat.

\index[routine]{init_ele}
\label{r:init.ele}
\item[init_ele (ele, key, sub_key, ix_ele, branch)] \Newline
Routine to initialize an element. 

\index[routine]{init_lat}
\label{r:init.lat}
\item[init_lat (lat, n)] \Newline 
Routine to initialize a Bmad lat.

\index[routine]{lattice_bookkeeper}
\label{r:lattice.bookkeeper}
\item[lattice_bookkeeper (lat, err_flag)] \Newline 
Routine to do bookkeeping for the entire lattice.

\index[routine]{reallocate_coord}
\label{r:reallocate.coord}
\item[reallocate_coord (coord, n_coord)] \Newline 
Routine to reallocate an allocatable  coord_struct array to at least:
coord(0:n_coord).

\index[routine]{reallocate_coord_array}
\label{r:reallocate.coord.array}
\item[reallocate_coord_array (coord_array, lat)] \Newline 
Routine to allocate an allocatable coord_array_struct array to
the proper size for a lattice.

\index[routine]{set_attribute_alias}
\label{r:set.attribute.alias}
\item[set_attribute_alias (attrib_name, alias_name, err_flag, lat)] \Newline 
Routine to setup an alias for element attributes like custom_attribute1\$, etc. in
the attribute name table.

\index[routine]{set_ele_defaults}
\label{r:set.ele.defaults}
\item[set_ele_defaults (ele, do_allocate)] \Newline 
Subroutine to set the defaults for an element of a given type.

\index[routine]{set_on_off}
\label{r:set.on.off}
\item[set_on_off (key, lat, switch, orb, use_ref_orb, ix_branch, saved_values, ix_attrib)] \Newline
Routine to turn on or off a set of elements (quadrupoles,
RF cavities, etc.) in a lat.

\index[routine]{transfer_ele}
\label{r:transfer.ele}
\item[transfer_ele (ele1, ele2, nullify_pointers)] \Newline 
     Routine to set ele2 = ele1. 
     This is a plain transfer of information not using the overloaded equal.

\index[routine]{transfer_eles}
\label{r:transfer.eles}
\item[transfer_eles (ele1, ele2)] \Newline 
     Routine to set ele2(:) = ele1(:). 
     This is a plain transfer of information not using the overloaded equal.

\index[routine]{transfer_ele_taylor}
\item[transfer_ele_taylor (ele_in, ele_out, taylor_order)] \Newline 
     Routine to transfer a Taylor map from one element to another.

\index[routine]{transfer_lat}
\label{r:transfer.lat}
\item[transfer_lat (lat1, lat2)] \Newline 
     Routine to set lat2 = lat1. 
     This is a plain transfer of information not using the overloaded equal.

\index[routine]{transfer_lat_parameters}
\label{r:transfer.lat.parameters}
\item[transfer_lat_parameters (lat_in, lat_out)] \Newline
Routine to transfer the lat parameters (such as lat\%name, 
lat\%param, etc.) from one lat to another. 


\index[routine]{zero_ele_kicks}
\label{r:zero.ele.kicks}
\item[zero_ele_kicks (ele)] \Newline 
Subroutine to zero any kick attributes like hkick$, bl_vkick$, etc.
See also: ele_has_kick, ele_has_offset, zero_ele_offsets.

\index[routine]{zero_ele_offsets}
\label{r:zero.ele.offsets}
\item[zero_ele_offsets (ele)] \Newline 
Routine to zero the offsets, pitches and tilt of an element.

\end{description}

%------------------------------------------------------------------------
\section{Lattice: Miscellaneous}
\label{r:lat.misc}

\begin{description}

\index[routine]{c_multi}
\label{r:c.multi}
\item[c_multi (n, m, no_n_fact, c_full)] \Newline
Routine to compute multipole factors: 
c_multi(n, m) = +/- ("n choose m")/n! 

\index[routine]{ele_compute_ref_energy_and_time}
\label{r:ele.compute.ref.energy.and.time}
\item[ele_compute_ref_energy_and_time (ele0, ele, param, err_flag)] \Newline 
Routine to compute the reference energy and reference time at the end
of an element given the reference enegy and reference time at the
start of the element.

\index[routine]{lat_compute_ref_energy_and_time}
\label{r:lat.compute.ref.energy.and.time}
\item[lat_compute_ref_energy_and_time (lat, err_flag)] \Newline
Routine to compute the reference energy for each element in a lattice. 

\index[routine]{field_interpolate_3d}
\label{r:field.interpolate.3d}
\item[field_interpolate_3d (position, field_mesh, deltas, position0)] \Newline
Function to interpolate a 3d field. 

\index[routine]{order_super_lord_slaves}
\label{r:order.super.lord.slaves}
\item[order_super_lord_slaves (lat, ix_lord)] \Newline
Routine to make the slave elements of a super_lord in order. 

\index[routine]{release_rad_int_cache}
\label{r:release.rad.int.cache}
\item[release_rad_int_cache (ix_cache)] \Newline 
     Routine to release the memory associated with caching wiggler values.

\index[routine]{set_flags_for_changed_attribute}
\label{r:set.flags.for.changed.attribute}
\item[set_flags_for_changed_attribute (ele, attrib)] \Newline 
Routine to mark an element as modified for use with "intelligent" bookkeeping.

\end{description}

%------------------------------------------------------------------------
\section{Lattice: Reading and Writing Files} 
\label{r:read}
\index{lattice files!reading and writing routines}

\begin{description}

\index[routine]{aml_parser}
\label{r:aml.parser}
\item[aml_parser (lat_file, lat, make_mats6, digested_read_ok, use_line, err_flag)] \Newline 
Routine to parse an AML input file and put the information in a lat_struct.

\index[routine]{bmad_and_xsif_parser}
\label{r:bmad.and.xsif.parser}
\item[\protect\parbox{6in}{
    bmad_and_xsif_parser (lat_file, lat, make_mats6, digested_read_ok, \\
    \hspace*{1in} use_line, err_flag)} ] \Newline 
Subroutine to parse either a Bmad or XSIF (extended standard input format) lattice files.

\index[routine]{bmad_parser}
\label{r:bmad.parser}
\item[bmad_parser (lat_file, lat, make_mats6, digested_read_ok, use_line, err_flag)] \Newline
Routine to parse (read in) a Bmad input file. 

\index[routine]{bmad_parser2}
\label{r:bmad.parser2}
\item[bmad_parser2 (lat_file, lat, orbit, make_mats6, err_flag)] \Newline
Routine to parse (read in) a Bmad input file to modify an existing lattice. 

\index[routine]{write_lattice_in_foreign_format}
\label{r:write.lattice.in.foreign.format}
\item[\protect\parbox{6in}{write_lattice_in_foreign_format (out_type, out_file_name, lat, ref_orbit, \\
  \hspace*{1in} use_matrix_model, include_apertures, dr12_drift_max, \\
  \hspace*{1in} ix_start, ix_end, ix_branch, converted_lat, err)}] \Newline 
Routine to write a mad or xsif lattice file using the information in
a lat_struct. 

\index[routine]{combine_consecutive_elements}
\label{r:combine.consecutive.elements}
\item[combine_consecutive_elements (lat)] \Newline 
Routine to combine consecutive elements in the lattice that have the same name.
This allows simplification, for example, of lattices where elements have been split 
to compute the beta function at the center.

\index[routine]{create_sol_quad_model}
\label{r:create.sol.quad.model}
\item[create_sol_quad_model (sol_quad, lat)] \Newline 
Routine to create series of solenoid and quadrupole elements to serve as a replacement
model for a sol_quad element.

\index[routine]{create_unique_ele_names}
\label{r:create.unique.ele.names}
\item[create_unique_ele_names (lat, key, suffix)] \Newline 
Routine to give elements in a lattice unique names.

\index[routine]{read_digested_bmad_file}
\label{r:read.digested.bmad.file}
\item[read_digested_bmad_file (digested_file, lat, inc_version, err_flag)] \Newline
Routine to read in a digested file. 

\index[routine]{write_bmad_lattice_file}
\label{r:write.bmad.lattice.file}
\item[write_bmad_lattice_file (bmad_file, lat, err)] \Newline 
Routine to write a Bmad lattice file using the information in
a lat_struct.

\index[routine]{write_digested_bmad_file}
\label{r:write.digested.bmad.file}
\item[write_digested_bmad_file (digested_name, lat, n_files, file_names, extra, err_flag)] \Newline
Routine to write a digested file. 

\index[routine]{xsif_parser}
\label{r:xsif.parser}
\item[xsif_parser (xsif_file, lat, make_mats6, digested_read_ok, use_line, err_flag)] \Newline 
     Routine to parse an XSIF (extended standard input format) lattice file.

\end{description}

%------------------------------------------------------------------------
\section{Matrices}
\label{r:mat}
\index{matrix!list of routines}

\begin{description}

\index[routine]{c_to_cbar}
\label{r:c.to.cbar}
\item[c_to_cbar (ele, cbar_mat)] \Newline
Routine to compute Cbar from the C matrix and the Twiss parameters. 

\index[routine]{cbar_to_c}
\label{r:cbar.to.c}
\item[cbar_to_c (cbar_mat, a, b, c_mat)] \Newline
Routine to compute C coupling matrix from the Cbar matrix and the Twiss parameters. 

\index[routine]{clear_lat_1turn_mats}
\label{r:clear.lat.1turn.mats}
\item[clear_lat_1turn_mats (lat)] \Newline
Clear the 1-turn matrices in the lat structure. 

\index[routine]{concat_transfer_mat}
\label{r:concat.transfer.mat}
\item[concat_transfer_mat (mat_1, vec_1, mat_0, vec_0, mat_out, vec_out)] \Newline 
Routine to concatinate two linear maps

\index[routine]{determinant}
\label{r:determinant}
\item[determinant (mat) result (det)] \Newline 
Routine to take the determinant of a square matrix
This routine is adapted from Numerical Recipes.

\index[routine]{do_mode_flip}
\label{r:do.mode.flip}
\item[do_mode_flip (ele, err_flag)] \Newline
Routine to mode flip the Twiss parameters of an element 

\index[routine]{make_g2_mats}
\label{r:make.g2.mats}
\item[make_g2_mats (twiss, g2_mat, g2_inv_mat)] \Newline
Routine to make the matrices needed to go from normal mode coords to 
coordinates with the beta function removed. 

\index[routine]{make_g_mats}
\label{r:make.g.mats}
\item[make_g_mats (ele, g_mat, g_inv_mat)] \Newline
Routine to make the matrices needed to go from normal mode coords to 
coordinates with the beta function removed. 

\index[routine]{make_mat6}
\label{r:make.mat6}
\item[make_mat6 (ele, param, start_orb, end_orb, err_flag)] \Newline
Routine to make the 6x6 transfer matrix for an element. 

\index[routine]{make_v_mats}
\label{r:make.v.mats}
\item[make_v_mats (ele, v_mat, v_inv_mat)] \Newline
Routine to make the matrices needed to go from normal mode coords to X-Y 
coords and vice versa. 

\index[routine]{mat6_from_s_to_s}
\label{r:mat6.from.s.to.s}
\item[\protect\parbox{6in}{
    mat6_from_s_to_s (lat, mat6, vec0, s1, s2, orbit, ix_branch, one_turn, \\
    \hspace*{1in} unit_start, err_flag, ele_save)} ] \Newline 
Subroutine to calculate the transfer map between longitudinal positions
s1 to s2.

\index[routine]{mat6_to_taylor}
\item[mat6_to_taylor (vec0, mat6, bmad_taylor)] \Newline
Routine to form a first order Taylor map from the 6x6 transfer matrix 
and the 0th order transfer vector. 

\index[routine]{match_ele_to_mat6}
\label{r:match.ele.to.mat6}
\item[match_ele_to_mat6 (ele, err_flag, twiss_ele)] \Newline 
Routine to make the 6 x 6 transfer matrix from the twiss parameters.

\index[routine]{multi_turn_tracking_to_mat}
\item[multi_turn_tracking_to_mat (track, i_dim, map1, map0, track0, chi)] \Newline
Routine to analyze 1-turn tracking data to find the 1-turn transfer matrix 
and the closed orbit offset.

\index[routine]{transfer_matrix_calc}
\label{r:transfer.matrix.calc}
\item[transfer_matrix_calc (lat, xfer_mat, xfer_vec, ix1, ix2, ix_branch, one_turn)] \Newline
Routine to calculate the transfer matrix between two elements. If
ix1 and ix2 are not present the full 1--turn matrix is calculated.

\index[routine]{one_turn_mat_at_ele}
\label{r:one.turn.mat.at.ele}
\item[one_turn_mat_at_ele (ele, phi_a, phi_b, mat4)] \Newline
Routine to form the 4x4 1-turn coupled matrix with the reference point 
at the end of an element. 

\index[routine]{lat_make_mat6}
\label{r:lat.make.mat6}
\item[lat_make_mat6 (lat, ix_ele, ref_orb, ix_branch, err_flag)] \Newline
Routine to make the 6x6 linear transfer matrix for an element 

\index[routine]{taylor_to_mat6}
\item[taylor_to_mat6 (a_taylor, r_in, vec0, mat6, r_out)] \Newline
Routine to calculate the linear (Jacobian) matrix about some trajectory from a Taylor map. 

\index[routine]{transfer_mat2_from_twiss}
\label{r:transfer.mat2.from.twiss}
\item[transfer_mat2_from_twiss (twiss1, twiss2, mat)] \Newline
Routine to make a 2 x 2 transfer matrix from the Twiss parameters at the end points. 

\index[routine]{transfer_mat_from_twiss}
\label{r:transfer.mat.from.twiss}
\item[transfer_mat_from_twiss (ele1, ele2, orb1, orb2, m)] \Newline 
Routine to make a 6 x 6 transfer matrix from the twiss parameters
at the beginning and end of the element.

\index[routine]{twiss_from_mat2}
\label{r:twiss.from.mat2}
\item[twiss_from_mat2 (mat_in, twiss, stat, type_out)] \Newline
Routine to extract the Twiss parameters from the one-turn 2x2 matrix 

\index[routine]{twiss_from_mat6}
\label{r:twiss.from.mat6}
\item[twiss_from_mat6 (mat6, orb0, ele, stable, growth_rate, status, type_out)] \Newline
Routine to extract the Twiss parameters from the one-turn 6x6 matrix 

\index[routine]{twiss_to_1_turn_mat}
\item[twiss_to_1_turn_mat (twiss, phi, mat2)] \Newline
Routine to form the 2x2 1-turn transfer matrix from the Twiss parameters. 

\end{description}

%------------------------------------------------------------------------
\section{Matrix: Low Level Routines}
\label{r:low.mat}  

Listed below are helper routines that are not meant for general use.

\begin{description}

\index[routine]{sol_quad_mat6_calc}
\label{r:sol.quad.mat6.calc}
\item[sol_quad_mat6_calc (ks_in, k1_in, length, ele, orbit, mat6, make_matrix)] \Newline
Routine to calculate the transfer matrix for a combination solenoid/quadrupole element. 

\index[routine]{tilt_mat6}
\label{r:tilt.mat6}
\item[tilt_mat6 (mat6, tilt)] \Newline
Routine to transform a 6x6 transfer matrix to a new reference frame that is 
tilted in (x, Px, y, Py) with respect to the old reference frame. 

\end{description}

%------------------------------------------------------------------------
\section{Measurement Simulation Routines}
\label{r:meas}  
\index{measurement simulations!list of routines}

Routines to simulate errors in orbit, dispersion, betatron phase, and
coupling measurements

\begin{description}

\index[routine]{check_if_ele_is_monitor}
\label{r:check.if.ele.is.monitor}
\item[check_if_ele_is_monitor (ele, err)] \Newline
Routine to check that the element is either an instrument, monitor, or marker.
This routine is private and not meant for general use.

\index[routine]{to_eta_reading}
\label{r:to.eta.reading}
\item[to_eta_reading (eta_actual, ele, axis, reading, err)] \Newline
Compute the measured dispersion reading given the true dispersion and the
monitor offsets, noise, etc.

\index[routine]{to_orbit_reading}
\label{r:to.orbit.reading}
\item[to_orbit_reading (orb, ele, axis, reading, err)] \Newline
Calculate the measured reading on a bpm given the actual orbit and the
BPM's offsets, noise, etc.

\index[routine]{to_phase_and_coupling_reading}
\label{r:to.phase.and.coupling.reading}
\item[to_phase_and_coupling_reading (ele, reading, err)] \Newline
Find the measured coupling values given the actual ones


\end{description}

%------------------------------------------------------------------------
\section{Multipass}
\label{r:multipass}
\index{multipass!list of routines}

\begin{description}

\index[routine]{multipass_all_info}
\label{r:multipass.all.info}
\item[multipass_all_info (lat, info)] \Newline 
Routine to put multipass to a multipass_all_info_struct structure.

\index[routine]{multipass_chain}
\label{r:multipass.chain}
\item[multipass_chain (ele, ix_pass, n_links, chain_ele)] \Newline 
Routine to return the chain of elements that represent the same physical element
when there is multipass.

\index[routine]{pointer_to_multipass_lord}
\item[pointer_to_multipass_lord (ele, lat, ix_pass, super_lord) result (multi_lord)] \Newline 
Routine to find the multipass lord of a lattice element.
A multi_lord will be found for:

\end{description}

%------------------------------------------------------------------------
\section{Multipoles}
\label{r:multipoles}
\index{multipole!list of routines}

\begin{description}

\index[routine]{ab_multipole_kick}
\label{r:ab.multipole.kick}
\item[ab_multipole_kick (a, b, n, ref_species, ele_orientation, coord, kx, ky, dk, pole_type, scale)] \Newline 
Routine to put in the kick due to an ab_multipole.

\index[routine]{multipole_kicks}
\label{r:multipole.kicks}
\item[multipole_kicks (knl, tilt, ref_species, ele, orbit, pole_type, ref_orb_offset)] \Newline 
Routine to put in the kick due to a multipole.

\index[routine]{mexp}
\label{r:mexp}
\item[mexp (x, m) result (this_exp)] \Newline 
Returns x**m with 0**0 = 0.

\index[routine]{multipole_ab_to_kt}
\label{r:multipole.ab.to.kt}
\item[multipole_ab_to_kt (an, bn, knl, tn)] \Newline
Routine to convert ab type multipoles to kt (MAD standard) multipoles. 

\index[routine]{multipole_ele_to_ab}
\label{r:multipole.ele.to.ab}
\item[multipole_ele_to_ab (ele, use_ele_tilt, ix_pole_max, a, b, pole_type, include_kicks)] \Newline
Routine to put the scaled element multipole components (normal and skew) into 2 vectors. 

\index[routine]{multipole_ele_to_kt}
\label{r:multipole.ele.to.kt}
\item[multipole_ele_to_kt (ele, use_ele_tilt, ix_pole_max, knl, tilt, pole_type)] \Newline
Routine to put the scaled element multipole components (strength and tilt) 
into 2 vectors. 

\index[routine]{multipole_init}
\label{r:multipole.init}
\item[multipole_init(ele, who, zero)] \Newline
Routine to initialize the multipole arrays within an element.

\index[routine]{multipole_kick}
\label{r:multipole.kick}
\item[multipole_kick (knl, tilt, n, ref_species, ele_orientation, coord, pole_type, ref_orb_offset)] \Newline
Routine to put in the kick due to a multipole. 

\index[routine]{multipole_kt_to_ab}
\label{r:multipole.kt.to.ab}
\item[multipole_kt_to_ab (knl, tn, an, bn)] \Newline
Routine to convert kt (MAD standard) multipoles to ab type multipoles. 

\end{description}

%------------------------------------------------------------------------
\section{Nonlinear Optimizers}
\label{r:opti}      

\begin{description}

\index[routine]{opti_lmdif}
\label{r:opti.lmdif}
\item[opti_lmdif (vec, n, merit, eps) result(this_opti)] \Newline 
Function which tries to get the merit function(s) as close to zero as possible
by changing the values in vec. Multiple merit functions can be used.

\index[routine]{initial_lmdif}  
\label{r:initial.lmdif}  
\item[initial_lmdif()] \Newline 
Routine that clears out previous saved values of the optimizer.

\index[routine]{suggest_lmdif}
\label{r:suggest.lmdif}
\item[suggest_lmdif (XV, FV, EPS, ITERMX, at_end, reset_flag)] \Newline 
Reverse communication routine. 

\index[routine]{super_mrqmin}
\label{r:super.mrqmin}
\item[\protect\parbox{6in}{super_mrqmin (y, weight, a, \\
  \hspace*{1in} chisq, funcs, storage, alamda, status, maska)}] \Newline 
Routine to do non-linear optimizations. 
This routine is essentially mrqmin from Numerical Recipes with some added features.

\index[routine]{opti_de}
\label{r:opti.de}
\item[opti_de (v_best, generations, population, merit_func, v_del, status)] \Newline 
Differential Evolution for Optimal Control Problems.
This optimizer is based upon the work of Storn and Price. 

\end{description}

%------------------------------------------------------------------------
\section{Overloading the equal sign}
\label{r:equal}    

These routines are overloaded by the equal sign so should not be called explicitly.

\begin{description}

\index[routine]{branch_equal_branch}
\label{r:branch.equal.branch}
\item[branch_equal_branch (branch1, branch2)] \Newline 
Routine that is used to set one branch equal to another. 

\index[routine]{bunch_equal_bunch}
\label{r:bunch.equal.bunch}
\item[bunch_equal_bunch (bunch1, bunch2)] \Newline
Routine that is used to set one macroparticle bunch to another. This routine
takes care of the pointers in bunch1.

\index[routine]{coord_equal_coord}
\label{r:coord.equal.coord}
\item[coord_equal_coord (coord1, coord2)] \Newline
Routine that is used to set one coord_struct equal to another. 

\index[routine]{ele_equal_ele}
\label{r:ele.equal.ele}
\item[ele_equal_ele (ele_out, ele_in)] \Newline
Routine that is used to set one element equal to another. 
This routine takes care of the pointers in ele1. 

\index[routine]{real_8_equal_taylor}
\item[real_8_equal_taylor (y8, bmad_taylor)] \Newline
Routine to overload "=" in expressions real_8 (PTC) = bmad_taylor.

\index[routine]{lat_equal_lat}
\label{r:lat.equal.lat}
\item[lat_equal_lat (lat_out, lat_in)] \Newline
Routine that is used to set one lat equal to another. 
This routine takes care of the pointers in lat1. 

\index[routine]{lat_vec_equal_lat_vec}
\label{r:lat.vec.equal.lat.vec}
\item[lat_vec_equal_lat_vec (lat1, lat2)] \Newline
Routine that is used to set one lat array equal to another. 
This routine takes care of the pointers in lat1(:). 

\index[routine]{taylor_equal_real_8}
\item[taylor_equal_real_8 (bmad_taylor, y8)] \Newline
Routine to overload "=" in expressions bmad_taylor = real_8 (PTC) 

\index[routine]{universal_equal_universal}
\label{r:universal.equal.universal}
\item[universal_equal_universal (ut1, ut2)] \Newline
Routine that is used to set one PTC universal_taylor 
structure equal to another. 

\end{description}

%------------------------------------------------------------------------
\section{Particle Coordinate Stuff}
\label{r:coord}    
\index{coordinates!list of routines}

\begin{description}

\index[routine]{convert_coords}
\label{r:convert.coords}
\item[convert_coords (in_type_str, coord_in, ele, out_type_str, coord_out, err_flag)] \Newline
Routine to convert between lab frame, normal mode, normalized normal mode, 
and action-angle coordinates. 

\index[routine]{convert_pc_to}
\label{r:convert.pc.to}
\item[convert_pc_to (pc, particle, E_tot, gamma, kinetic, beta, brho, beta1, err_flag)] \Newline
Routine to calculate the energy, etc. from a particle's momentum. 

\index[routine]{convert_total_energy_to}
\label{r:convert.total.energy.to}
\item[\protect\parbox{6in}{
    convert_total_energy_to (E_tot, particle, gamma, kinetic, beta, pc, brho, \\
    \hspace*{1in} beta1, err_flag)} ] \Newline
Routine to calculate the momentum, etc. from a particle's total energy. 

\index[routine]{init_coord}
\label{r:init.coord}
\item[\protect\parbox{6in}{
    init_coord (orb, vec, ele, element_end, particle, direction, E_photon, \\
    \hspace*{1in} t_ref_offset, shift_vec6, spin)} ] \Newline 
Routine to initialize a coord_struct.

\index[routine]{type_coord}
\label{r:type.coord}
\item[type_coord (coord)] \Newline
Routine to type out a coordinate. 

\end{description}

%------------------------------------------------------------------------
\section{Photon Routines}
\label{r:photon}
\index{photons!list of routines}

\begin{description}

\index[routine]{bend_photon_init}
\label{r:bend.photon.init}
\item[bend_photon_init (g_bend_x, g_bend_y, gamma, orbit, E_min, E_max, E_integ_prob)] \Newline 
Routine to initialize a photon

\index[routine]{bend_photon_vert_angle_init}
\label{r:bend.photon.vert.angle.init}
\item[bend_photon_vert_angle_init (E_rel, gamma_phi, r_in)] \Newline 
Routine to convert a "random" number in the interval [0,1] to a photon vertical emission 
angle for a simple bend.

\index[routine]{bend_photon_energy_init}
\label{r:bend.photon.energy.init}
\item[bend_photon_energy_init (e_rel, r_in)] \Newline 
Routine to convert a random number in the interval [0,1] to a photon energy.

\end{description}

%------------------------------------------------------------------------
\section{Interface to PTC}
\label{r:ptc}      
\index{PTC/FPP!list of routines}

\begin{description}

\index[routine]{concat_real_8}
\label{r:concat.real.8}
\item[concat_real_8 (y1, y2, y3, r2_ref, keep_y1_const_terms)] \Newline
Routine to concatenate two real_8 taylor series. 

\index[routine]{ele_to_fibre}
\label{r:ele.to.fibre}
\item[\protect\parbox{6in}{
    ele_to_fibre (ele, ptc_fibre, param, use_offsets, integ_order, steps, \\
    \hspace*{1in} for_layout, track_particle, use_hard_edge_drifts, kill_layout) }] \Newline
Routine to convert a Bmad element to a PTC fibre element. 

\index[routine]{map_coef}
\label{r:map.coef}
\item[map_coef (y, i, j, k, l)] \Newline
Function to return the coefficient of the map y(:) up to 3rd order. 

\index[routine]{kill_ptc_genfield}
\label{r:kill.ptc.genfield}
\item[kill_ptc_genfield (ptc_genfield)] \Newline 
Subroutine to kill a ptc_genfield.

\index[routine]{kill_ptc_layouts}
\label{r:kill.ptc.layouts}
\item[kill_ptc_layouts (lat)] \Newline 
Routine to kill the layouts associated with a Bmad lattice.

\index[routine]{kind_name}
\label{r:kind.name}
\item[kind_name (this_kind)] \Newline
Function to return the name of a PTC kind. 

\index[routine]{real_8_equal_taylor}
\label{r:real.8.equal.taylor}
\item[real_8_equal_taylor (y8, bmad_taylor)] \Newline
Routine to overload "=" in expressions real_8 = bmad_taylor 

\index[routine]{real_8_to_taylor}
\label{r:real.8.to.taylor}
\item[real_8_to_taylor (y8, beta0, beta1, bmad_taylor)] \Newline
Routine to convert from a real_8 taylor map in \'Etienne's PTC to a taylor map in Bmad. 

\index[routine]{real_8_init}
\label{r:real.8.init}
\item[real_8_init (y, set_taylor)] \Newline
Routine to allocate a PTC real_8 variable. 

\index[routine]{remove_constant_taylor}
\label{r:remove.constant.taylor}
\item[remove_constant_taylor (taylor_in, taylor_out, c0, remove_higher_order_terms)] \Newline
Routine to remove the constant part of a taylor series. 

\index[routine]{lat_to_ptc_layout}
\label{r:lat.to.ptc.layout}
\item[lat_to_ptc_layout (lat, use_hard_edge_drifts)] \Newline
Routine to create a PTC layout from a Bmad lat. 

\index[routine]{set_ptc}
\label{r:set.ptc}
\item[\protect\parbox{6in}{
    set_ptc (e_tot, particle, taylor_order, integ_order, n_step, no_cavity, \\
    \hspace*{1in} exact_modeling, exact_misalign, init_complex, force_init) }] \Newline
Routine to initialize PTC. 

\index[routine]{sort_universal_terms}
\label{r:sort.universal.terms}
\item[sort_universal_terms (ut_in, ut_sorted)] \Newline
Routine to sort the taylor terms from "lowest" to "highest". 

\index[routine]{taylor_equal_real_8}
\label{r:taylor.equal.real.8}
\item[taylor_equal_real_8 (bmad_taylor, y8)] \Newline
Routine to overload "=" in expressions bmad_taylor = y8 

\index[routine]{taylor_to_real_8}
\label{r:taylor.to.real.8}
\item[taylor_to_real_8 (bmad_taylor, beta0, beta1, y8, remove_constant)] \Newline
Routine to convert from a taylor map in Bmad to a real_8 taylor map in \'Etienne's PTC. 

\index[routine]{type_ptc_layout}
\label{r:type.layout}
\item[type_layout (lay)] \Newline
Routine to print the global information in a PTC layout.

\index[routine]{type_map1}
\label{r:type.map1}
\item[type_map1 (y, type0, n_dim)] \Newline
Routine to type the transfer map up to first order. 

\index[routine]{type_ptc_fibre}
\label{r:type.fibre}
\item[type_fibre (ptc_fibre, print_coords, lines, n_lines)] \Newline
Routine to print the global information in a fibre.

\index[routine]{type_map}
\label{r:type.map}
\item[type_map (y)] \Newline
Routine to type the transfer maps of a real_8 array. 

\index[routine]{type_real_8_taylors}
\label{r:type.real.8.taylors}
\item[type_real_8_taylors (y)] \Newline
Routine to type out the taylor series from a real_8 array. 

\index[routine]{taylor_to_genfield}
\label{r:taylor.to.genfield}
\item[taylor_to_genfield (bmad_taylor, ptc_genfield, c0)] \Newline
Routine to construct a genfield (partially inverted map) from a taylor map. 

\index[routine]{universal_to_bmad_taylor}
\label{r:universal.to.bmad.taylor}
\item[universal_to_bmad_taylor (u_taylor, bmad_taylor)] \Newline
Routine to convert from a universal_taylor map in \'Etienne's PTC to a taylor map in Bmad. 

\index[routine]{vec_bmad_to_ptc}
\label{r:vec.bmad.to.ptc}
\item[vec_bmad_to_ptc (vec_bmad, beta0, vec_ptc, conversion_mat)] \Newline
Routine to convert from Bmad to PTC coordinates. 

\index[routine]{vec_ptc_to_bmad}
\label{r:vec.ptc.to.bmad}
\item[vec_ptc_to_bmad (vec_ptc, beta0, vec_bmad, conversion_mat)] \Newline
Routine to convert from PTC to Bmad coordinates. 

\end{description}

%------------------------------------------------------------------------
\section{Quick Plot Routines}
\label{r:qp}      
\index{quick plot!list of routines}

%--------------------------------------
\subsection{Quick Plot Page Routines}

\begin{description}

\index[routine]{qp_open_page}
\label{r:qp.open.page}
\item[qp_open_page (page_type, i_chan, x_len, y_len, units, plot_file, scale)] \Newline 
     Routine to Initialize a page (window) for plotting.

\index[routine]{qp_select_page}
\label{r:qp.select.page}
\item[qp_select_page (iw)] \Newline 
     Routine to switch to a particular page for drawing graphics.

\index[routine]{qp_close_page}
\label{r:qp.close.page}
\item[qp_close_page()] \Newline 
     Routine to finish plotting on a page.

\end{description}

%--------------------------------------
\subsection{Quick Plot Calculational Routines}

\begin{description}

\index[routine]{qp_axis_niceness}
\label{r:qp.axis.niceness}
\item[qp_axis_niceness (imin, imax, divisions) result (score)] \Newline 
Routine to calculate how ``nicely'' an axis will look.
The higher the score the nicer.

\index[routine]{qp_calc_and_set_axis}
\item[\protect\parbox{6in}{
    qp_calc_and_set_axis (axis_str, data_min, data_max, div_min, div_max, \\
    \hspace*{1in} bounds, axis_type, slop_factor)}] \Newline
     Routine to calculate a "nice" plot scale given the minimum and maximum
     of the data. 

\index[routine]{qp_calc_axis_params}
\label{r:qp.calc.axis.params}
\item[\protect\parbox{6in}{qp_calc_axis_params (data_min, data_max, div_min, \\ 
      \hspace*{1in} div_max, axis, slop_factor)}] \Newline 
     Routine to calculate a "nice" plot scale given the minimum and maximum
     of the data. This is similar to calc_axis_scale.

\index[routine]{qp_calc_axis_divisions}
\label{r:qp.calc.axis.divisions}
\item[qp_calc_axis_divisions (axis_min, axis_max, div_min, div_max, divisions)] \Newline 
Routine to calculate the best (gives the nicest looking drawing) number 
of major divisions for fixed axis minimum and maximum.

\index[routine]{qp_calc_axis_places}
\label{r:qp.calc.axis.places}
\item[qp_calc_axis_places (axis)] \Newline 
     Routine to calculate the number of decimal places needed to display the
     axis numbers.

\index[routine]{qp_calc_axis_scale}
\label{r:qp.calc.axis.scale}
\item[qp_calc_axis_scale (data_min, data_max, axis, niceness_score, slop_factor)] \Newline 
     Routine to calculate a "nice" plot scale given the minimum and maximum
     of the data. 

\index[routine]{qp_calc_minor_div}
\label{r:qp.calc.minor.div}
\item[qp_calc_minor_div (delta, div_max, divisions)] \Newline 
     Routine to calculate the number of minor divisions an axis should have.

\index[routine]{qp_convert_rectangle_rel}
\label{r:qp.convert.rectangle.rel}
\item[qp_convert_rectangle_rel (rect1, rect2)] \Newline 
     Routine to convert a "rectangle" (structure of 4 points) from
     one set of relative units to another

\end{description}

%--------------------------------------
\subsection{Quick Plot Drawing Routines}

\begin{description}

\index[routine]{qp_clear_box}
\label{r:qp.clear.box}
\item[qp_clear_box()] \Newline 
Routine to clear the current box on the page.

\index[routine]{qp_clear_page}
\label{r:qp.clear.page}
\item[qp_clear_page()] \Newline 
Routine to clear all drawing from the page.

\index[routine]{qp_draw_circle}
\label{r:qp.draw.circle}
\item[\protect\parbox{6in}{
      qp_draw_circle (x0, y0, r, angle0, del_angle, \\
      \hspace*{1in} units, width, color, line_pattern, clip)}] \Newline 
Routine to plot a section of an ellipse.

\index[routine]{qp_draw_ellipse}
\label{r:qp.draw.ellipse}
\item[\protect\parbox{6in}{
    qp_draw_ellipse (x0, y0, r_x, r_y, theta_xy, \\
    \hspace*{1in} angle0, del_angle, units, width, color, line_pattern, clip) }] \Newline 
     Routine to plot a section of an ellipse.

\index[routine]{qp_draw_axes}
\label{r:qp.draw.axes}
\item[qp_draw_axes(x_lab, y_lab, title, draw_grid)] \Newline 
     Routine to plot the axes, title, etc. of a plot.

\index[routine]{qp_draw_data}
\label{r:qp.draw.data}
\item[qp_draw_data (x_dat, y_dat, draw_line, symbol_every, clip)] \Newline
     Routine to plot data, axes with labels, a grid, and a title.

\index[routine]{qp_draw_graph}
\label{r:qp.draw.graph}
\item[\protect\parbox{6in}{qp_draw_graph (x_dat, y_dat, x_lab, y_lab, title, \\
  \hspace*{1in} draw_line, symbol_every, clip) }] \Newline 
     Routine to plot data, axes with labels, a grid, and a title.

\index[routine]{qp_draw_graph_title}
\label{r:qp.draw.graph.title}
\item[qp_draw_graph_title (title)] \Newline 
     Routine to draw the title for a graph.

\index[routine]{qp_draw_grid}
\label{r:qp.draw.grid}
\item[qp_draw_grid()] \Newline 
     Routine to draw a grid on the current graph.

\index[routine]{qp_draw_histogram}
\label{r:qp.draw.histogram}
\item[qp_draw_histogram (x_dat, y_dat, fill_color, fill_pattern, line_color, clip)] \Newline 
     Routine to plot data, axes with labels, a grid, and a title.

\index[routine]{qp_draw_curve_legend}
\label{r:qp.draw.curve.legend}
\item[\protect\parbox{6in}{qp_draw_curve_legend (x_origin, y_origin, units, line, line_length, \\ 
\hspace*{1in}  symbol, text, text_offset, draw_line, draw_symbol, draw_text) }] \Newline
Routine to draw a legend with each line in the legend having
  a line, a symbol, some text.

\index[routine]{qp_draw_text_legend}
\label{r:qp.draw.text.legend}
\item[qp_draw_text_legend (text, x_origin, y_origin, units)] \Newline 
Routine to draw a legend of lines of text.

\index[routine]{qp_draw_main_title}
\label{r:qp.draw.main.title}
\item[qp_draw_main_title (lines, justify)] \Newline 
     Routine to plot the main title at the top of the page.

\index[routine]{qp_draw_polyline}
\label{r:qp.draw.polyline}
\item[qp_draw_polyline (x, y, units, width, color, line_pattern, clip, style)] \Newline 
     Routine to draw a polyline.

\index[routine]{qp_draw_polyline_no_set}
\label{r:qp.draw.polyline.no.set}
\item[qp_draw_polyline_no_set (x, y, units)] \Newline 
Routine to draw a polyline.
This is similar to qp_draw_polyline except qp_set_line_attrib is not called.

\index[routine]{qp_draw_polyline_basic}
\label{r:qp.draw.polyline.basic}
\item[qp_draw_polyline_basic (x, y)] \Newline 
     Routine to draw a polyline. See also qp_draw_polyline

\index[routine]{qp_draw_line}
\label{r:qp.draw.line}
\item[qp_draw_line (x1, x2, y1, y2, units, width, color, line_pattern, clip, style)] \Newline 
     Routine to draw a line.

\index[routine]{qp_draw_rectangle}
\label{r:qp.draw.rectangle}
\item[qp_draw_rectangle (x1, x2, y1, y2, units, color, width, line_pattern, clip, style)] \Newline 
     Routine to draw a rectangular box.

\index[routine]{qp_draw_symbol}
\label{r:qp.draw.symbol}
\item[qp_draw_symbol (x, y, units, type, height, color, fill_pattern, line_width, clip)] \Newline 
     Draws a symbol at (x, y) 

\index[routine]{qp_draw_symbols}
\label{r:qp.draw.symbols}
\item[\protect\parbox{6in}{qp_draw_symbols (x, y, units, type, height, color, \\
  \hspace*{1in} fill_pattern, line_width, clip, symbol_every)} ] \Newline 
     Draws a symbol at the (x, y) points. 

\index[routine]{qp_draw_text}
\label{r:qp.draw.text}
\item[\protect\parbox{6in}{
  qp_draw_text (text, x, y, units, justify, height, color, angle, \\
  \hspace*{1in} background, uniform_spacing, spacing_factor) }] \Newline 
     Routine to draw text.

\index[routine]{qp_draw_text_no_set}
\label{r:qp.draw.text.no.set}
\item[qp_draw_text_no_set (text, x, y, units, justify, angle)] \Newline 
Routine to display on a plot a character string.
See also: qp_draw_text.

\index[routine]{qp_draw_text_basic}
\label{r:qp.draw.text.basic}
\item[qp_draw_text_basic (text, len_text, x0, y0, angle, justify)] \Newline 
     Routine to display on a plot a character string.
     See also: qp_draw_text.

\index[routine]{qp_draw_x_axis}
\label{r:qp.draw.x.axis}
\item[qp_draw_x_axis (who, y_pos)] \Newline 
     Routine to draw a horizontal axis.

\index[routine]{qp_draw_y_axis}
\label{r:qp.draw.y.axis}
\item[qp_draw_y_axis (who, x_pos)] \Newline 
     Routine to draw a horizontal axis.

\index[routine]{qp_paint_rectangle}
\label{r:qp.paint.rectangle}
\item[qp_paint_rectangle (x1, x2, y1, y2, units, color, fill_pattern)] \Newline 
Routine to paint a rectangular region a specified color.
The default color is the background color (white\$).

\index[routine]{qp_to_axis_number_text}
\label{r:qp.to.axis.number.text}
\item[qp_to_axis_number_text (axis, ix_n, text)] \Newline 
     Routine to form the text string for an axis number.

\end{description}

%--------------------------------------
\subsection{Quick Plot Set Routines}

\begin{description}

\index[routine]{qp_calc_and_set_axis}
\label{r:qp.calc.and.set.axis}
\item[\protect\parbox{6in}{
  qp_calc_and_set_axis (axis, data_min, data_max, \\
  \hspace*{1in} div_min, div_max, bounds, axis_type, slop_factor)}] \Newline
     Routine to calculate a "nice" plot scale given the minimum and maximum
     of the data. 

\index[routine]{qp_eliminate_xy_distortion}
\label{r:qp.eliminate.xy.distortion}
\item[qp_eliminate_xy_distortion(axis_to_scale)] \Newline 
This routine will increase the x or y margins so that the conversion
between data units and page units is the same for the x and y axes.

\index[routine]{qp_set_axis}
\label{r:qp.set.axis}
\item[\protect\parbox{6in}{qp_set_axis (axis_str, a_min, a_max, div, places, label, draw_label, \\
\hspace*{1in} draw_numbers, minor_div, minor_div_max, mirror, number_offset, \\
\hspace*{1in} label_offset, major_tick_len, minor_tick_len, ax_type)}] \Newline
    Routine to set (but not plot) the min, max and divisions for the axes of the graph.

\index[routine]{qp_set_box}
\label{r:qp.set.box}
\item[qp_set_box (ix, iy, ix_tot, iy_tot)] \Newline 
     Routine to set the box on the physical page.
     This routine divides the page into a grid of boxes. 

\index[routine]{qp_set_graph}
\label{r:qp.set.graph}
\item[qp_set_graph (title)] \Newline 
     Routine to set certain graph attributes.

\index[routine]{qp_set_graph_limits}
\label{r:qp.set.graph.limits}
\item[qp_set_graph_limits()] \Newline 
     Routine to calculate the offsets for the graph.
     This routine also sets the PGPLOT window size equal to the graph size.

\index[routine]{qp_set_graph_placement}
\label{r:qp.set.graph.placement}
\item[qp_set_graph_placement (x1_marg, x_graph_len, y1_marg, y_graph_len, units)] \Newline 
Routine to set the placement of the current graph inside the box. 
This routine can be used in place of qp_set_margin.

\index[routine]{qp_set_layout}
\label{r:qp.set.layout}
\item[\protect\parbox{6in}{
  qp_set_layout (x_axis, y_axis, x2_axis, y2_axis, \\
  \hspace*{1in} x2_mirrors_x, y2_mirrors_y, box, margin, page_border) }] \Newline 
     Routine to set various attributes. This routine can be used
     in place of other qp_set_* routines.

\index[routine]{qp_set_line}
\label{r:qp.set.line}
\item[qp_set_line (who, line)] \Newline 
     Routine to set the default line attributes.

\index[routine]{qp_set_margin}
\label{r:qp.set.margin}
\item[qp_set_margin (x1_marg, x2_marg, y1_marg, y2_marg, units)] \Newline 
Routine to set up the margins from the sides of the box (see QP_SET_BOX)
to the edges of the actual graph.

\index[routine]{qp_set_page_border}
\label{r:qp.set.page.border}
\item[qp_set_page_border (x1_b, x2_b, y1_b, y2_b, units)] \Newline 
     Routine to set the border around the physical page.

\index[routine]{qp_set_page_border_to_box}
\label{r:qp.set.page.border.to.box}
\item[qp_set_page_border_to_box ()] \Newline 
Routine to set the page border to correspond to the region of the
current box. This allows qp_set_box to subdivide the current box.

\index[routine]{qp_set_clip}
\label{r:qp.set.clip}
\item[qp_set_clip (clip)] \Newline 
     Routine to set the default clipping state.

\index[routine]{qp_set_parameters}
\label{r:qp.set.parameters}
\item[\protect\parbox{6in}{
    qp_set_parameters (text_scale, default_draw_units, default_set_units, \\
    \hspace*{1in} default_axis_slop_factor) }] \Newline 
Routine to set various quick plot parameters.

\index[routine]{qp_subset_box}
\label{r:qp.subset.box}
\item[qp_subset_box (ix, iy, ix_tot, iy_tot, x_marg, y_marg)] \Newline 
     Routine to set the box for a graph. This is the same as
     qp_set_box but the boundaries of the page are taken to be the box boundaries.

\index[routine]{qp_set_symbol}
\label{r:qp.set.symbol}
\item[qp_set_symbol (symbol)] \Newline 
     Routine to set the type and size of the symbols used in plotting data.
     See the pgplot documentation for more details.

\index[routine]{qp_set_symbol_attrib}
\label{r:qp.set.symbol.attrib}
\item[qp_set_symbol_attrib (type, height, color, fill_pattern, line_width, clip)] \Newline 
     Routine to set the type and size of the symbols used in plotting data.

\index[routine]{qp_set_line_attrib}
\label{r:qp.set.line.attrib}
\item[qp_set_line_attrib (style, width, color, pattern, clip)] \Newline 
     Routine to set the default line attributes.

\index[routine]{qp_set_graph_attrib}
\label{r:qp.set.graph.attrib}
\item[qp_set_graph_attrib (draw_grid, draw_title)] \Newline 
     Routine to set attributes of the current graph.

\index[routine]{qp_set_text_attrib}
\label{r:qp.set.text.attrib}
\item[\protect\parbox{6in}{qp_set_text_attrib (who, height, color, \\
  \hspace*{1in} background, uniform_spacing, spacing_factor)} ] \Newline 
     Routine to set the default text attributes.

\index[routine]{qp_use_axis}
\label{r:qp.use.axis}
\item[qp_use_axis (x, y)] \Newline 
Routine to set what axis to use: X or X2, Y or Y2.

\end{description}

%--------------------------------------
\subsection{Informational Routines}

\begin{description}

\index[routine]{qp_get_axis_attrib}
\label{r:qp.get.axis}
\item[\protect\parbox{6in}{
  qp_get_axis_attrib (axis_str, a_min, a_max, div,  places, label, \\
  \hspace*{1in} draw_label, draw_numbers, minor_div, mirror, number_offset, \\
  \hspace*{1in} label_offset, major_tick_len, minor_tick_len, ax_type) }] \Newline
     Routine to get the min, max, divisions etc. for the X and Y axes.

\index[routine]{qp_get_layout_attrib}
\label{r:qp.get.layout.attrib}
\item[qp_get_layout_attrib (who, x1, x2, y1, y2, units)] \Newline 
     Routine to get the attributes of the layout.

\index[routine]{qp_get_line_attrib}
\label{r:qp.get.line}
\item[qp_get_line (style, line)] \Newline 
Routine to get the default line attributes.

\index[routine]{qp_get_parameters}
\label{r:qp.get.parameters}
\item[\protect\parbox{6in}{
    qp_get_parameters (text_scale, default_draw_units, default_set_units, \\
    \hspace*{1in} default_axis_slop_factor) }] \Newline 
Routine to get various quick_plot parameters.

\index[routine]{qp_get_symbol_attrib}
\label{r:qp.get.symbol}
\item[qp_get_symbol (symbol)] \Newline 
Routine to get the symbol parameters used in plotting data.
Use qp_set_symbol or qp_set_symbol_attrib to set symbol attributes.

\index[routine]{qp_text_len}
\label{r:qp.text.len}
\item[qp_text_len (text)] \Newline 
     Function to find the length of a text string.

\end{description}

%--------------------------------------
\subsection{Conversion Routines}

\begin{description}

\index[routine]{qp_from_inch_rel}
\label{r:qp.from.inch.rel}
\item[qp_from_inch_rel (x_inch, y_inch, x, y, units)] \Newline 
     Routine to convert from a relative position (an offset) in inches
     to other units.

\index[routine]{qp_from_inch_abs}
\label{r:qp.from.inch.abs}
\item[qp_from_inch_abs (x_inch, y_inch, x, y, units)] \Newline 
     Routine to convert to absolute position (x, y) from inches referenced
     to the Left Bottom corner of the page

\index[routine]{qp_text_height_to_inches}
\label{r:qp.text.height.to.inches}
\item[qp_text_height_to_inches(height_pt) result (height_inch)] \Newline 
Function to convert from a text height in points to a text height in
inches taking into account the text_scale.

\index[routine]{qp_to_inch_rel}
\label{r:qp.to.inch.rel}
\item[qp_to_inch_rel (x, y, x_inch, y_inch, units)] \Newline 
Routine to convert a relative (x, y) into inches.

\index[routine]{qp_to_inch_abs}
\label{r:qp.to.inch.abs}
\item[qp_to_inch_abs (x, y, x_inch, y_inch, units)] \Newline 
Routine to convert an absolute position (x, y) into inches referenced
to the Left Bottom corner of the page.

\index[routine]{qp_to_inches_rel}
\label{r:qp.to.inches.rel}
\item[qp_to_inches_rel (x, y, x_inch, y_inch, units)] \Newline 
     Routine to convert a relative (x, y) into inches.

\index[routine]{qp_to_inches_abs}
\label{r:qp.to.inches.abs}
\item[qp_to_inches_abs (x, y, x_inch, y_inch, units)] \Newline 
     Routine to convert an absolute position (x, y) into inches referenced
     to the left bottom corner of the page.

\end{description}

%--------------------------------------
\subsection{Miscellaneous Routines}

\begin{description}

\index[routine]{qp_read_data}
\label{r:qp.read.data}
\item[qp_read_data (iu, err_flag, x, ix_col, y, iy_col, z, iz_col, 
                                                               t, it_col)] \Newline 
     Routine to read columns of data.

\end{description}

%--------------------------------------
\subsection{Low Level Routines}

\begin{description}

\index[routine]{qp_clear_box_basic}
\label{r:qp.clear.box.basic}
\item[qp_clear_box_basic (x1, x2, y1, y2)] \Newline 
Routine to clear all drawing from a box.
That is, white out the box region.

\index[routine]{qp_clear_page_basic}
\label{r:qp.clear.page.basic}
\item[qp_clear_page_basic()] \Newline 
Routine to clear all drawing from the page.

\index[routine]{qp_close_page_basic}
\label{r:qp.close.page.basic}
\item[qp_close_page_basic()] \Newline 
Routine to finish plotting on a page.
For X this closes the window.

\index[routine]{qp_convert_point_rel}
\label{r:qp.convert.point.rel}
\item[qp_convert_point_rel (x_in, y_in, units_in, x_out, y_out, units_out)] \Newline 
Routine to convert a (x, y) point from from
one set of relative units to another.

\index[routine]{qp_convert_point_abs}
\label{r:qp.convert.point.abs}
\item[qp_convert_point_abs (x_in, y_in, units_in, x_out, y_out, units_out)] \Newline 
Routine to convert a (x, y) point from from
one set of absolute units to another.

\index[routine]{qp_draw_symbol_basic}
\label{r:qp.draw.symbol.basic}
\item[qp_draw_symbol_basic (x, y, symbol)] \Newline 
Routine to draw a symbol.

\index[routine]{qp_init_com_struct}
\label{r:qp.init.com.struct}
\item[qp_init_com_struct ()] \Newline 
Routine to initialize the common block qp_state_struct.
This routine is not for general use.

\index[routine]{qp_join_units_string}
\label{r:qp.join.units.string}
\item[qp_join_units_string (u_type, region, corner, units)] \Newline 
Routine to form a units from its components.

\index[routine]{qp_justify}
\label{r:qp.justify}
\item[qp_justify (justify)] \Newline 
     Function to convert a justify character string to a real value
     representing the horizontal justification. 

\index[routine]{qp_open_page_basic}
\label{r:qp.open.page.basic}
\item[\protect\parbox{6in}{
    qp_open_page_basic (page_type, x_len, y_len, plot_file, \\
    \hspace*{1in} x_page, y_page, i_chan, page_scale) }] \Newline 
Routine to Initialize a page (window) for plotting.

\index[routine]{qp_paint_rectangle_basic}
\label{r:qp.paint.rectangle.basic}
\item[qp_paint_rectangle_basic (x1, x2, y1, y2, color, fill_pattern)] \Newline 
Routine to fill a rectangle with a given color. 
A color of white essentially erases the rectangle.

\index[routine]{qp_pointer_to_axis}
\label{r:qp.pointer.to.axis}
\item[qp_pointer_to_axis (axis_str, axis_ptr)] \Newline 
Routine to return a pointer to an common block axis.

\index[routine]{qp_restore_state}
\label{r:qp.restore.state}
\item[qp_restore_state()] \Newline 
     Routine to restore saved attributes. 
     Use qp_save_state to restore the saved state.

\index[routine]{qp_restore_state_basic}
\label{r:qp.restore.state.basic}
\item[qp_restore_state_basic (buffer_basic)] \Newline 
Routine to restore the print state.

\index[routine]{qp_save_state}
\label{r:qp.save.state}
\item[qp_save_state (buffer_basic)] \Newline 
     Routine to save the current attributes. 
     Use qp_restore_state to restore the saved state.

\index[routine]{qp_save_state_basic}
\label{r:qp.save.state.basic}
\item[qp_save_state_basic ()] \Newline 
Routine to save the print state.

\index[routine]{qp_select_page_basic}
\label{r:qp.select.page.basic}
\item[qp_select_page_basic (iw)] \Newline 
Routine to switch to a particular page for drawing graphics.

\index[routine]{qp_set_char_size_basic}
\label{r:qp.set.char.size.basic}
\item[qp_set_char_size_basic (height)] \Newline 
Routine to set the character size.

\index[routine]{qp_set_clip_basic}
\label{r:qp.set.clip.basic}
\item[qp_set_clip_basic (clip)] \Newline 
Routine to set the clipping state.
Note: This affects both lines and symbols.

\index[routine]{qp_set_color_basic}
\label{r:qp.set.color.basic}
\item[qp_set_color_basic (ix_color)  ] \Newline 
Routine to set the color taking into account that GIF
inverts the black for white.

\index[routine]{qp_set_graph_position_basic}
\label{r:qp.set.graph.position.basic}
\item[qp_set_graph_position_basic (x1, x2, y1, y2)] \Newline 
Routine to set the position of a graph.
Units are inches from lower left of page.

\index[routine]{qp_set_line_width_basic}
\label{r:qp.set.line.width.basic}
\item[qp_set_line_width_basic (line_width)] \Newline 
Routine to set the line width.

\index[routine]{qp_set_symbol_fill_basic}
\label{r:qp.set.symbol.fill.basic}
\item[qp_set_symbol_fill_basic (fill)] \Newline 
Routine to set the symbol fill style.

\index[routine]{qp_set_symbol_size_basic}
\label{r:qp.set.symbol.size.basic}
\item[qp_set_symbol_size_basic (height, symbol_type, uniform_size)] \Newline 
Routine to set the symbol_size

\index[routine]{qp_set_text_background_color_basic}
\label{r:qp.set.text.background.color.basic}
\item[qp_set_text_background_color_basic (color)] \Newline 
Routine to set the character text background color.

\index[routine]{qp_split_units_string}
\label{r:qp.split.units.string}
\item[qp_split_units_string (u_type, region, corner, units)] \Newline 
     Routine to split a units string into its components.

\index[routine]{qp_text_len_basic}
\label{r:qp.text.len.basic}
\item[qp_text_len_basic (text)] \Newline 
Function to find the length of a text string.

\index[routine]{qp_translate_to_color_index}
\label{r:qp.translate.to.color.index}
\item[qp_translate_to_color_index (name)] \Newline 
     Routine to translate from a string to a color index.

\end{description}

%------------------------------------------------------------------------
\section{Spin Tracking}
\label{r:spin}    
\index{spin tracking!list of routines}

\begin{description}

\index[routine]{spinor_to_polar}
\label{r:spinor.to.polar}
\item[spinor_to_polar (spinor) result (polar)] \Newline 
Routine to convert a spinor into polar coordinates.

\index[routine]{polar_to_vec}
\label{r:polar.to.vec}
\item[polar_to_vec (polar) result (vec)] \Newline
Routine to convert a spin vector from polar coordinates to Cartesian coordinates.

\index[routine]{polar_to_spinor}
\label{r:polar.to.spinor}
\item[polar_to_spinor (polar) result (coord)] \Newline
Routine to convert a spin vector in polar coordinates to a spinor.

\index[routine]{vec_to_polar}
\label{r:vec.to.polar}
\item[vec_to_polar (vec, phase) result (polar)] \Newline
Routine to convert a spin vector from Cartesian coordinates to polar coordinates 
preserving the complex phase.

\index[routine]{spinor_to_vec}
\label{r:spinor.to.vec}
\item[spinor_to_vec (spinor) result (vec)] \Newline
Routine to convert a spinor to a spin vector in Cartesian coordinates.

\index[routine]{vec_to_spinor}
\label{r:vec.to.spinor}
\item[vec_to_spinor (vec, phase) result (coord)] \Newline
Routine to convert a spin vector in Cartesian coordinates to a spinor using
the specified complex phase.

\index[routine]{angle_between_polars}
\label{r:angle.between.polars}
\item[angle_between_polars (polar1, polar2)] \Newline
Function to return the angle between two spin vectors in polar coordinates.

\index[routine]{spin_omega}
\label{r:spin.omega}
\item[spin_omega (field, coord, sign_z_vel, phase_space_coords), result (omega)] \Newline 
Return the modified T-BMT spin omega vector.

\index[routine]{track1_spin}
\label{r:track1.spin}
\item[track1_spin (start_orb, ele, param, end_orb)] \Newline
Routine to track the particle spin through one element.

\end{description}

%------------------------------------------------------------------------
\section{Transfer Maps: Routines Called by make_mat6}
\label{r:mat6}
 
\vn{Make_mat6} is the routine for calculating the transfer matrix (Jacobin)
through an element. The routines listed below are used by \vn{make_mat6}.
In general a program should call \vn{make_mat6} rather than using these
routines directly.

\begin{description}

\index[routine]{make_mat6_bmad}
\label{r:make.mat6.bmad}
\item[make_mat6_bmad (ele, param, start_orb, end_orb, err)] \Newline
Routine to make the 6x6 transfer matrix for an element
using closed formulas.

\index[routine]{make_mat6_custom}
\item[make_mat6_custom (ele, param, c0, c1, err_flag)] \Newline
Routine for custom calculations of the 6x6 transfer matrices.

\index[routine]{make_mat6_symp_lie_ptc}
\label{r:make.mat6.symp.lie.ptc}
\item[make_mat6_symp_lie_ptc (ele, param, start_orb, end_orb)] \Newline
Routine to make the 6x6 transfer matrix for an element using
the PTC symplectic integrator.

\index[routine]{make_mat6_taylor}
\label{r:make.mat6.taylor}
\item[make_mat6_taylor (ele, param, start_orb, end_orb, err_flag)] \Newline
Routine to make the 6x6 transfer matrix for an element
from a Taylor map.

\index[routine]{make_mat6_tracking}
\label{r:make.mat6.tracking}
\item[make_mat6_tracking (ele, param, start_orb, end_orb)] \Newline
Routine to make the 6x6 transfer matrix for an element by 
tracking 7 particle with different starting conditions.

\end{description}

%------------------------------------------------------------------------
\section{Transfer Maps: Complex Taylor Maps}
\label{r:ctaylor}   
\index{complex taylor map!list of routines}

\begin{description}

\index[routine]{add_complex_taylor_term}
\label{r:add.complex.taylor.term}
\item[add_complex_taylor_term (bmad_complex_taylor, coef, exp)] \Newline 
Subroutine add_complex_taylor_term (bmad_complex_taylor, coef, i1, i2, i3, i4, i5, i6, i7, i8, i9)
Routine to add a complex_taylor term to a complex_taylor series.

\index[routine]{complex_taylor_coef}
\label{r:complex.taylor.coef}
\item[complex_taylor_coef (bmad_taylor, expn)] \Newline 
Function complex_taylor_coef (bmad_complex_taylor, i1, i2, i3, i4, i5, i6, i7, i8, i9)
Function to return the coefficient for a particular complex_taylor term
from a complex_taylor Series.

\index[routine]{complex_taylor_equal_complex_taylor}
\label{r:complex.taylor.equal.complex.taylor}
\item[complex_taylor_equal_complex_taylor (complex_taylor1, complex_taylor2)] \Newline 
Subroutine that is used to set one complex_taylor equal to another. 
This routine takes care of the pointers in complex_taylor1. 

\index[routine]{complex_taylor_make_unit}
\label{r:complex.taylor.make.unit}
\item[complex_taylor_make_unit (bmad_complex_taylor)] \Newline 
Subroutine to make the unit complex_taylor map:
      r(out) = Map * r(in) = r(in)

\index[routine]{complex_taylor_exponent_index}
\label{r:complex.taylor.exponent.index}
\item[complex_taylor_exponent_index(expn) result(index)] \Newline 
Function to associate a unique number with a complex_taylor exponent.

\index[routine]{complex_taylor_to_mat6}
\label{r:complex.taylor.to.mat6}
\item[complex_taylor_to_mat6 (a_complex_taylor, r_in, vec0, mat6, r_out)] \Newline 
Subroutine to calculate, from a complex_taylor map and about some trajectory:
  The 1st order (Jacobian) transfer matrix.

\index[routine]{complex_taylors_equal_complex_taylors}
\label{r:complex.taylors.equal.complex.taylors}
\item[complex_taylors_equal_complex_taylors (complex_taylor1, complex_taylor2)] \Newline 
Subroutine to transfer the values from one complex_taylor map to another:
    complex_taylor1 <= complex_taylor2

\index[routine]{init_complex_taylor_series}
\label{r:init.complex.taylor.series}
\item[init_complex_taylor_series (bmad_complex_taylor, n_term, save)] \Newline 
Subroutine to initialize a Bmad complex_taylor series (6 of these series make
a complex_taylor map). Note: This routine does not zero the structure. The calling

\index[routine]{kill_complex_taylor}
\label{r:kill.complex.taylor}
\item[kill_complex_taylor (bmad_complex_taylor)] \Newline 
Subroutine to deallocate a Bmad complex_taylor map.

\index[routine]{mat6_to_complex_taylor}
\label{r:mat6.to.complex.taylor}
\item[mat6_to_complex_taylor (vec0, mat6, bmad_complex_taylor)] \Newline 
Subroutine to form a first order complex_taylor map from the 6x6 transfer
matrix and the 0th order transfer vector.

\index[routine]{sort_complex_taylor_terms}
\label{r:sort.complex.taylor.terms}
\item[sort_complex_taylor_terms (complex_taylor_in, complex_taylor_sorted)] \Newline 
Subroutine to sort the complex_taylor terms from "lowest" to "highest" of
a complex_taylor series.

\index[routine]{track_complex_taylor}
\label{r:track.complex.taylor}
\item[track_complex_taylor (start_orb, bmad_complex_taylor, end_orb)] \Newline 
Subroutine to track using a complex_taylor map.

\index[routine]{truncate_complex_taylor_to_order}
\label{r:truncate.complex.taylor.to.order}
\item[truncate_complex_taylor_to_order (complex_taylor_in, order, complex_taylor_out)] \Newline 
Subroutine to throw out all terms in a complex_taylor map that are above a certain order.

\index[routine]{type_complex_taylors}
\label{r:type.complex.taylors}
\item[type_complex_taylors (bmad_complex_taylor, max_order, lines, n_lines)] \Newline 
Subroutine to output a Bmad complex_taylor map.

\end{description}

%------------------------------------------------------------------------
\section{Transfer Maps: Taylor Maps}
\label{r:taylor}   
\index{taylor map!list of routines}

\begin{description}

\index[routine]{add_taylor_term}
\label{r:add.taylor.term}
\item[\protect\parbox{6in}{
  add_taylor_term (bmad_taylor, coef, expn, replace) \\
  add_taylor_term (bmad_taylor, coef, i1, i2, i3, i4, i5, i6, i7, i8, i9, replace)}] \Newline 
Overloaded routine to add a Taylor term to a Taylor series.

\index[routine]{concat_ele_taylor}
\label{r:concat.ele.taylor}
\item[concat_ele_taylor (taylor1, ele, taylor3)] \Newline 
Routine to concatenate two taylor maps.

\index[routine]{concat_taylor}
\label{r:concat.taylor}
\item[concat_taylor (taylor1, taylor2, taylor3)] \Newline
Routine to concatenate two taylor series: taylor3(x) = taylor2(taylor1(x)) 

\index[routine]{ele_to_taylor}
\label{r:ele.to.taylor}
\item[ele_to_taylor (ele, param, bmad_taylor, orb0, taylor_map_includes_offsets)] \Newline
Routine to make a Taylor map for an element. The order of the map is set by set_ptc.

\index[routine]{equivalent_taylor_attributes}
\label{r:equivalent.taylor.attributes}
\item[equivalent_taylor_attributes (ele1, ele2) result (equiv)] \Newline 
Routine to see if to elements are equivalent in terms of attributes so
that their Taylor Maps would be the same. 

\index[routine]{init_taylor_series}
\label{r:init.taylor.series}
\item[init_taylor_series (bmad_taylor, n_term, save_old)] \Newline
Routine to initialize a Bmad Taylor series. 

\index[routine]{kill_taylor}
\label{r:kill.taylor}
\item[kill_taylor (bmad_taylor, spin_taylor)] \Newline
Routine to deallocate a Bmad Taylor map. 

\index[routine]{mat6_to_taylor}
\label{r:mat6.to.taylor}
\item[mat6_to_taylor (mat6, vec0, bmad_taylor)] \Newline
Routine to form a first order Taylor map from the 6x6 transfer matrix 
and the 0th order transfer vector. 

\index[routine]{sort_taylor_terms}
\label{r:sort.taylor.terms}
\item[sort_taylor_terms (taylor_in, taylor_sorted, min_val)] \Newline
Routine to sort the taylor terms from "lowest" to "highest" of a
Taylor series.

\index[routine]{taylor_coef}
\label{r:taylor.coef}
\item[taylor_coef (bmad_taylor, expn)] \Newline 
Function to return the coefficient for a particular taylor term from a
Taylor Series.

\index[routine]{taylor_equal_taylor}
\label{r:taylor.equal.taylor}
\item[taylor_equal_taylor (taylor1, taylor2)] \Newline
Routine to transfer the values from one taylor map to another:
Taylor1 $\le$ Taylor2

\index[routine]{transfer_map_calc}
\label{r:transfer.map.calc}
\item[transfer_map_calc (lat, t_map, err_flag, ix1, ix2, ref_orb, ix_branch, one_turn, unit_start) ] \Newline 
Routine to calculate the transfer map between two elements.

\index[routine]{transfer_map_from_s_to_s}
\label{r:transfer.map.from.s.to.s}
\item[\protect\parbox{6in}{
    transfer_map_from_s_to_s (lat, t_map, s1, s2, ref_orb, ix_branch, \\
    \hspace*{1in} one_turn, unit_start, err_flag) }] \Newline 
Subroutine to calculate the transfer map between longitudinal positions
s1 to s2.

\index[routine]{taylor_minus_taylor}
\label{r:taylor.minus.taylor}
\item[taylor_minus_taylor (taylor1, taylor2) result (taylor3)] \Newline 
Routine to add two taylor maps.

\index[routine]{taylor_plus_taylor}
\label{r:taylor.plus.taylor}
\item[taylor_plus_taylor (taylor1, taylor2) result (taylor3)] \Newline 
Routine to add two taylor maps.

\index[routine]{taylors_equal_taylors}
\label{r:taylors.equal.taylors}
\item[taylors_equal_taylors (taylor1, taylor2)] \Newline 
Routine to transfer the values from one taylor map to another.

\index[routine]{taylor_make_unit}
\label{r:taylor.make.unit}
\item[taylor_make_unit (bmad_taylor, ref_orbit)] \Newline
Routine to make the unit Taylor map

\index[routine]{taylor_to_mat6}
\label{r:taylor.to.mat6}
\item[taylor_to_mat6 (a_taylor, c0, mat6, c1)] \Newline
Routine to calculate the linear (Jacobian) matrix about some
trajectory from a Taylor map.

\index[routine]{taylor_inverse}
\label{r:taylor.inverse}
\item[taylor_inverse (taylor_in, taylor_inv, err)] \Newline
Routine to invert a taylor map. 

\index[routine]{taylor_propagate1}
\label{r:taylor.propagate1}
\item[taylor_propagate1 (bmad_taylor, ele, param, track_particle)] \Newline
Routine to track a real_8 taylor map through an element. 
The alternative routine, if ele has a taylor series, is concat_taylor. 

\index[routine]{track_taylor}
\label{r:track.taylor}
\item[track_taylor (start_orb, bmad_taylor, end_orb, ref_orb)] \Newline
Routine to track using a Taylor map. 

\index[routine]{transfer_ele_taylor}
\label{r:transfer.ele.taylor}
\item[transfer_ele_taylor (ele_in, ele_out, taylor_order)] \Newline 
Routine to transfer a Taylor map from one element to another.

Routine to transfer the taylor maps from the elements of one lat to
the elements of another. 

\index[routine]{truncate_taylor_to_order}
\label{r:truncate.taylor.to.order}
\item[truncate_taylor_to_order (taylor_in, order, taylor_out)] \Newline 
Routine to throw out all terms in a taylor map that are above a certain order.

\index[routine]{type_taylors}
\label{r:type.taylors}
\item[type_taylors (bmad_taylor, max_order, lines, n_lines, file_id)] \Newline
Routine to output a Bmad taylor map.

\end{description}

%------------------------------------------------------------------------
\section{Tracking and Closed Orbit}
\label{r:track}    
\index{tracking!list of routines}

The following routines perform tracking and closed orbit calculations.

\begin{description}

\index[routine]{check_aperture_limit}
\label{r:check.aperture.limit}
\item[check_aperture_limit (orb, ele, particle_at, param, old_orb, check_momentum)] \Newline
Routine to check if an orbit is outside an element's aperture. 

\index[routine]{check_aperture_limit_custom}
\item[check_aperture_limit_custom (orb, ele, particle_at, param, err_flag)] \Newline
Routine to check if an orbit is outside an element's aperture.
Used when \vn{ele%aperture_type} is set to \vn{custom\$} 

\index[routine]{closed_orbit_calc}
\label{r:closed.orbit.calc}
\item[closed_orbit_calc (lat, closed_orb, i_dim, direction, ix_branch, err_flag, print_err)] \Newline 
Routine to calculate the closed orbit at the beginning of the lat.

\index[routine]{closed_orbit_from_tracking}
\label{r:closed.orbit.from.tracking}
\item[\protect\parbox{6in}{
  closed_orbit_from_tracking (lat, closed_orb, i_dim, eps_rel, eps_abs, \\
  \hspace*{1in} init_guess, err_flag) }] \Newline
Routine to find the closed orbit via tracking. 

\index[routine]{compute_even_steps}
\label{r:compute.even.steps}
\item[compute_even_steps (ds_in, length, ds_default, ds_out, n_step)] \Newline 
Routine to compute a step size ds_out, close to ds_in, so that an 
integer number of steps spans the length.

\index[routine]{dynamic_aperture1}
\label{r:dynamic.aperture1}
\item[dynamic_aperture1 (lat, orb0, theta_xy, aperture_param, aperture, check_xy_init)] \Newline
Routine to determine the dynamic aperture of a lattice via tracking. 

\index[routine]{multi_turn_tracking_analysis}
\label{r:multi.turn.tracking.analysis}
\item[\protect\parbox{6in}{
    multi_turn_tracking_analysis (track, i_dim, track0, ele, stable, growth_rate, \\
    \hspace*{1in} chi, err_flag)} ] \Newline
Routine to analyze multi-turn tracking data to get the Twiss
parameters etc.

\index[routine]{multi_turn_tracking_to_mat}
\label{r:multi.turn.tracking.to.mat}
\item[multi_turn_tracking_to_mat (track, i_dim, 
mat1, track0, chi)] \Newline
Routine to analyze 1-turn tracking data to find the 1-turn transfer
matrix and the closed orbit offset.

\index[routine]{offset_particle}
\label{r:offset.particle}
\item[\protect\parbox{6in}{
    offset_particle (ele, param, set, orbit, set_tilt, \\
    \hspace*{1in} set_hvkicks, set_z_offset, ds_pos, set_spin, mat6, make_matrix)} ] \Newline
Routine to effectively offset an element by instead offsetting 
the particle position to correspond to the local element coordinates. 

\index[routine]{offset_photon}
\label{r:offset.photon}
\item[offset_photon (ele, orbit, set, offset_position_only, rot_mat)] \Newline 
Routine to effectively offset an element by instead offsetting
the photon position to correspond to the local crystal or mirror coordinates.

\index[routine]{orbit_amplitude_calc}
\label{r:orbit.amplitude.calc}
\item[orbit_amplitude_calc (ele, orb, amp_a, amp_b, amp_na, amp_nb)] \Newline
Routine to calculate the "invariant" amplitude of a particle at a 
particular point in its orbit. 

\index[routine]{particle_is_moving_backwards}
\label{r:particle.is.moving.backwards}
\item[particle_is_moving_backwards (orbit) result (is_moving_backward)] \Newline 
Routine to determine if a particle is moving in the backward -s direction.
If not moving backward it is dead or is moving backward.

\index[routine]{particle_is_moving_forward}
\label{r:particle.is.moving.forward}
\item[particle_is_moving_forward (orbit) result (is_moving_forward)] \Newline 
Routine to determine if a particle is moving in the forward +s direction.
If not moving forward it is dead or is moving backward.

\index[routine]{tilt_coords}
\label{r:tilt.coords}
\item[tilt_coords (tilt_val, coord, mat6, make_matrix)] \Newline
Routine to effectively tilt (rotate in the x-y plane) an element by 
instead rotating the particle position with negative the angle. 

\index[routine]{track1}
\label{r:track1}
\item[track1 (start_orb, ele, param, end_orb, track, err_flag, ignore_radiation, mat6, make_matrix)] \Newline
Routine to track through a single element. 

\index[routine]{track1_beam_simple}
\label{r:track1.beam.simple}
\item[track1_beam_simple (beam_start, ele, param, beam_end)] \Newline 
Routine to track a beam of particles through a single element.
This routine does *not* include multiparticle effects.

\index[routine]{track1_bunch_csr}
\label{r:track1.bunch.csr}
\item[track1_bunch_csr (bunch_start, ele, centroid, bunch_end, err, s_start, s_end)] \Newline 
Routine to track a bunch of particles through the element lat\%ele(ix_ele)
with csr radiation effects.

\index[routine]{track1_spin_custom}
\item[track1_spin_custom (start, ele, param, end, err_flag, track)] \Newline 
Dummy routine for custom spin tracking. 
This routine needs to be replaced for a custom calculation.

\index[routine]{track_all}
\label{r:track.all}
\item[track_all (lat, orbit, ix_branch, track_state, err_flag, orbit0)] \Newline
Routine to track through the lat. 

\index[routine]{track_from_s_to_s}
\label{r:track.from.s.to.s}
\item[\protect\parbox{6in}{
    track_from_s_to_s (lat, s_start, s_end, orbit_start, orbit_end, all_orb, \\
    \hspace*{1in} ix_branch, track_state)} ] \Newline
Routine to track a particle between two s-positions.

\index[routine]{track_many}
\label{r:track.many}
\item[track_many (lat, orbit, ix_start, ix_end, direction, ix_branch, track_state)] \Newline
Routine to track from one element in the lat to another. 

\index[routine]{twiss_and_track}
\label{r:twiss.and.track}
\item[\protect\parbox{6in}{
  twiss_and_track (lat, orb, ok) \\
  twiss_and_track (lat, orb_array, ok) 
}] \Newline
Routine to calculate the twiss parameters, transport matrices and orbit.

\index[routine]{twiss_and_track_at_s}
\label{r:twiss.and.track.at.s}
\item[\protect\parbox{6in}{
    twiss_and_track_at_s (lat, s, ele_at_s, orb, orb_at_s, ix_branch, err, \\
    \hspace*{1in} use_last, compute_floor_coords)} ] \Newline
Routine to calculate the Twiss parameters and orbit at a particular longitudinal position. 

\index[routine]{twiss_and_track_from_s_to_s}
\label{r:twiss.and.track.from.s.to.s}
\item[\protect\parbox{6in}{
    twiss_and_track_from_s_to_s (branch, orbit_start, s_end, orbit_end, \\
    \hspace*{1in} ele_start, ele_end, err, compute_floor_coords)} ] \Newline
Routine to track a particle from one location to another.

\index[routine]{twiss_and_track_intra_ele}
\label{r:twiss.and.track.intra.ele}
\item[\protect\parbox{6in}{
  twiss_and_track_intra_ele (ele, param, l_start, l_end, track_upstream_end, \\
  \hspace*{1in} track_downstream_end, orbit_start, orbit_end, ele_start, ele_end, err, compute_floor_coords) }] \Newline 
Routine to track a particle within an element.

\index[routine]{twiss_from_tracking}
\item[twiss_from_tracking (lat, ref_orb0, symp_err, err_flag, d_orb)] \Newline
Routine to compute from tracking the Twiss parameters and the transfer matrices 
for every element in the lat. 

\index[routine]{wall_hit_handler_custom}
\item[wall_hit_handler_custom (orb, ele, s, t)] \Newline 
This routine is called by the Runge-Kutta integrator odeint_bmad when a particle hits a wall.

\end{description}

%------------------------------------------------------------------------
\section{Tracking: Low Level Routines}
\label{r:low.track}

\begin{description}

\index[routine]{odeint_bmad}
\label{r:odeint.bmad}
\item[\protect\parbox{6in}{
    odeint_bmad (orbit, ele, param, s1, s2, local_ref_frame, err_flag, track) }] \Newline
Routine to do Runge Kutta tracking. 

\index[routine]{create_uniform_element_slice}
\label{r:create.uniform.element.slice}
\item[\protect\parbox{6in}{
    create_uniform_element_slice (ele, param, i_slice, \\
    \hspace*{1in} n_slice_tot, sliced_ele, s_start, s_end)} ] \Newline 
Routine to create an element that represents a slice of another element.
This routine can be used for detailed tracking through an element.

\index[routine]{track1_boris_partial}
\label{r:track1.boris.partial}
\item[track1_boris_partial (start, ele, param, s, ds, end)] \Newline
Routine to track 1 step using boris tracking. 

\index[routine]{track_a_drift}
\label{r:track.a.drift}
\item[track_a_drift (orb, length, mat6, make_matrix, include_ref_motion)] \Newline
Routine to track through a drift. 

\index[routine]{track_a_bend}
\label{r:track.a.bend}
\item[track_a_bend (orbit, ele, param, mat6, make_matrix)] \Newline
Particle tracking through a bend element. 

\end{description}

%------------------------------------------------------------------------
\section{Tracking: Mad Routines}
\label{r:mad}      

\begin{description}

\index[routine]{make_mat6_mad}
\label{r:make.mat6.mad}
\item[make_mat6_mad (ele, param, c0, c1)] \Newline 
     Routine to make the 6x6 transfer matrix for an element from the 
     2nd order MAD transport map. The map is stored in ele\%taylor.

\index[routine]{make_mad_map}
\label{r:make.mad.map}
\item[make_mad_map (ele, param, energy, map)] \Newline 
     Routine to make a 2nd order transport map a la MAD.

\index[routine]{mad_add_offsets_and_multipoles}
\label{r:mad.add.offsets.and.multipoles}
\item[mad_add_offsets_and_multipoles (ele, map)] \Newline 
     Routine to add in the effect of element offsets and/or multipoles
     on the 2nd order transport map for the element.

\index[routine]{mad_drift}
\label{r:mad.drift}
\item[mad_drift (ele, energy, map)] \Newline 
     Routine to make a transport map for a drift space.
     The equivalent MAD-8 routine is: TMDRF

\index[routine]{mad_elsep}
\label{r:mad.elsep}
\item[mad_elsep (ele, energy, map)] \Newline 
     Routine to make a transport map for an electric separator. 
     The equivalent MAD-8 routine is: TMSEP

\index[routine]{mad_sextupole}
\label{r:mad.sextupole}
\item[mad_sextupole (ele, energy, map)] \Newline 
     Routine to make a transport map for an sextupole.
     The equivalent MAD-8 routine is: TMSEXT

\index[routine]{mad_sbend}
\label{r:mad.sbend}
\item[mad_sbend (ele, energy, map)] \Newline 
     Routine to make a transport map for a sector bend element.
     The equivalent MAD-8 routine is: TMBEND

\index[routine]{mad_sbend_fringe}
\label{r:mad.sbend.fringe}
\item[mad_sbend_fringe (ele, energy, into, map)] \Newline 
     Routine to make a transport map for the fringe field of a dipole.
     The equivalent MAD-8 routine is: TMFRNG

\index[routine]{mad_sbend_body}
\label{r:mad.sbend.body}
\item[mad_sbend_body (ele, energy, map)] \Newline 
     Routine to make a transport map for the body of a sector dipole.
     The equivalent MAD-8 routine is: TMSECT

\index[routine]{mad_tmfoc}
\label{r:mad.tmfoc}
\item[mad_tmfoc (el, sk1, c, s, d, f)] \Newline 
     Routine to compute the linear focusing functions.  
     The equivalent MAD-8 routine is: TMFOC

\index[routine]{mad_quadrupole}
\label{r:mad.quadrupole}
\item[mad_quadrupole (ele, energy, map)] \Newline 
     Routine to make a transport map for an quadrupole element.
     The equivalent MAD-8 routine is: TMSEXT

\index[routine]{mad_rfcavity}
\label{r:mad.rfcavity}
\item[mad_rfcavity (ele, energy, map)] \Newline 
     Routine to make a transport map for an rfcavity element.
     The equivalent MAD-8 routine is: TMRF

\index[routine]{mad_solenoid}
\label{r:mad.solenoid}
\item[mad_solenoid (ele, energy, map)] \Newline 
     Routine to make a transport map for an solenoid.
     The equivalent MAD-8 routine is: TMSEXT

\index[routine]{mad_tmsymm}
\label{r:mad.tmsymm}
\item[mad_tmsymm (te)] \Newline 
     routine to symmetrize the 2nd order map t.
     The equivalent MAD-8 routine is: tmsymm

\index[routine]{mad_tmtilt}
\label{r:mad.tmtilt}
\item[mad_tmtilt (map, tilt)] \Newline 
     Routine to apply a tilt to a transport map.
     The equivalent MAD-8 routine is: TMTILT

\index[routine]{mad_concat_map2}
\label{r:mad.concat.map2}
\item[mad_concat_map2 (map1, map2, map3)] \Newline 
     Routine to concatenate two 2nd order transport maps.

\index[routine]{mad_track1}
\label{r:mad.track1}
\item[mad_track1 (c0, map, c1)] \Newline 
     Routine to track through a 2nd order transfer map.
     The equivalent MAD-8 routine is: TMTRAK

\index[routine]{track1_mad}
\label{r:track1.mad}
\item[track1_mad (start_orb, ele, param, end_orb)] \Newline 
     Routine to track through an element using a 2nd order transfer map.
     Note: If map does not exist then one will be created. 

\index[routine]{mad_map_to_taylor}
\label{r:mad.map.to.taylor}
\item[mad_map_to_taylor (map, energy, taylor)] \Newline 
     Routine to convert a mad order 2 map to a taylor map.

\index[routine]{taylor_to_mad_map}
\label{r:taylor.to.mad.map}
\item[taylor_to_mad_map (taylor, energy, map)] \Newline 
     Routine to convert a Taylor map to a mad order 2 map.
     If any of the Taylor terms have order greater than 2 they are ignored.

\index[routine]{make_unit_mad_map}
\label{r:make.unit.mad.map}
\item[make_unit_mad_map (map)] \Newline 
     Routine to initialize a 2nd order transport map to unity.


\end{description}

%------------------------------------------------------------------------
\section{Tracking: Routines called by track1}
\label{r:track1.call}   

Note: Unless you know what you are doing do not call these routines directly. Rather use
track1.

\begin{description}

\index[routine]{symp_lie_bmad}
\label{r:symp.lie.bmad}
\item[symp_lie_bmad (ele, param, start_orb, end_orb, make_matrix, track, offset_ele)] \Newline
Symplectic integration through an element to 0th or 1st order.

\index[routine]{track1_boris}
\label{r:track1.boris}
\item[track1_boris (orb_start, ele, param, orb_end, err_flag, track, s_start, s_end)] \Newline
Routine to do Boris tracking.  

\index[routine]{track1_bmad}
\label{r:track1.bmad}
\item[track1_bmad (start_orb, ele, param, end_orb, err_flag, mat6, make_matrix)] \Newline
Particle tracking through a single element BMAD_standard style. 

\index[routine]{track1_custom}
\item[track1_custom (start_orb, ele, param, end_orb, err_flag, finished, track)] \Newline
Dummy routine for custom tracking.

\index[routine]{track1_linear}
\label{r:track1.linear}
\item[track1_linear (start_orb, ele, param, end_orb)] \Newline
Particle tracking through a single element using the transfer matrix.. 

\index[routine]{track1_postprocess}
\item[track1_postprocess (start_orb, ele, param, end_orb)] \Newline 
Dummy routine for post processing after the track1 routine is done.

\index[routine]{track1_preprocess}
\item[track1_preprocess (start_orb, ele, param, err_flag, finished, radiation_included, track)] \Newline 
Dummy routine for pre processing at the start of the track1 routine.

\index[routine]{track1_radiation}
\label{r:track1.radiation}
\item[track1_radiation (orb_start, ele, param, orb_end, edge, z_start)] \Newline
Routine to put in radiation damping and/or fluctuations. 

\index[routine]{track1_runge_kutta}
\label{r:track1.runge.kutta}
\item[track1_runge_kutta (start_orb, ele, param, end_orb, err_flag, track)] \Newline
Routine to do tracking using Runge-Kutta integration. 

\index[routine]{track1_symp_lie_ptc}
\label{r:track1.symp.lie.ptc}
\item[track1_symp_lie_ptc (start_orb, ele, param, end_orb, track)] \Newline
Particle tracking through a single element using a Hamiltonian and a 
symplectic integrator. 

\index[routine]{track1_symp_map}
\label{r:track1.symp.map}
\item[track1_symp_map (start_orb, ele, param, end_orb)] \Newline
Particle tracking through a single element using a partially inverted 
taylor map (In PTC/FPP this is called a genfield). 

\index[routine]{track1_taylor}
\label{r:track1.taylor}
\item[track1_taylor (start_orb, ele, param, end_orb, taylor, mat6, make_matrix)] \Newline
Routine to track through an element using the elements taylor series. 

\index[routine]{track1_time_runge_kutta}
\label{r:track1.time.runge.kutta}
\item[track1_time_runge_kutta(start_orb, ele, param, end_orb, err_flag, track)] \Newline 
Routine to track a particle through an element using 
Runge-Kutta time-based tracking.

\end{description}

%------------------------------------------------------------------------
\section{Twiss and Other Calculations}
\label{r:twiss}
\index{twiss!list of routines}

\begin{description}

\index[routine]{calc_z_tune}
\label{r:calc.z.tune}
\item[calc_z_tune (lat, ix_branch)] \Newline
Routine to calculate the synchrotron tune from the full 6X6 1 turn matrix. 

\index[routine]{chrom_calc}
\label{r:chrom.calc}
\item[\protect\parbox{6in}{
    chrom_calc (lat, delta_e, chrom_x, chrom_y, err_flag, pz, \\
    \hspace*{1in} low_E_lat, high_E_lat, low_E_orb, high_E_orb, ix_branch)} ] \Newline
Routine to calculate the chromaticities by computing the tune 
change when then energy is changed. 

\index[routine]{chrom_tune}
\label{r:chrom.tune}
\item[chrom_tune (lat, delta_e, target_x, target_y, err_tol, err_flag)] \Newline
Routine to set the sextupole strengths so that the lat 
has the desired chromaticities. 

\index[routine]{quad_beta_ave}
\label{r:quad.beta.ave}
\item[quad_beta_ave (ele, beta_a_ave, beta_b_ave)] \Newline
Routine to compute the average betas in a quad.

\index[routine]{radiation_integrals}
\label{r:radiation.integrals}
\item[radiation_integrals (lat, orbit, mode, ix_cache, ix_branch, rad_int_by_ele)] \Newline
Routine to calculate the synchrotron radiation integrals, the emittance, and energy spread. 

\index[routine]{radiation_integrals_custom}
\item[radiation_integrals_custom (lat, ir, orb, err_flag)] \Newline
User supplied routine to calculate the synchrotron radiation integrals for
a custom element.

\index[routine]{relative_mode_flip}
\label{r:relative.mode.flip}
\item[relative_mode_flip (ele1, ele2)] \Newline
Function to see if the modes of ELE1 are flipped relative to ELE2. 

\index[routine]{set_tune}
\label{r:set.tune}
\item[set_tune (phi_a_set, phi_b_set, dk1, lat, orb, ok)] \Newline
Routine to Q_tune a lat. This routine will set the tunes to within 0.001 radian (0.06 deg). 

\index[routine]{set_z_tune}
\label{r:set.z.tune}
\item[set_z_tune (lat, z_tune, ok)] \Newline
Routine to set the longitudinal tune by setting the RF voltages in the RF cavities. 

\index[routine]{transfer_twiss}
\label{r:transfer.twiss}
\item[transfer_twiss (ele_in, ele_out)] \Newline 
Routine to transfer the twiss parameters from one element to another.

\index[routine]{twiss_and_track}
\item[twiss_and_track (lat, orb)] \Newline
Routine to calculate the Twiss and orbit parameters. 
This is not necessarily the fastest routine. 

\index[routine]{twiss_at_element}
\label{r:twiss.at.element}
\item[twiss_at_element (ele, start, end, average)] \Newline
Routine to return the Twiss parameters at the beginning, end, or the average of an element. 

\index[routine]{twiss_and_track_at_s}
\item[twiss_and_track_at_s (lat, s, ele, orb_, here)] \Newline
Routine to calculate the Twiss parameters and orbit at a particular longitudinal position. 

\index[routine]{twiss_at_start}
\label{r:twiss.at.start}
\item[twiss_at_start (lat, status, ix_branch)] \Newline
Routine to calculate the Twiss parameters at the start of the lat. 

\index[routine]{twiss_from_tracking}
\label{r:twiss.from.tracking}
\item[twiss_from_tracking (lat, closed_orb_, d_orb, error)] \Newline
Routine to compute from tracking, for every element in the lat, 
the Twiss parameters and the transfer matrices. 

\index[routine]{twiss_propagate1}
\label{r:twiss.propagate1}
\item[twiss_propagate1 (ele1, ele2, err_flag)] \Newline
Routine to propagate the Twiss parameters from the end of ELE1 to the end of ELE2. 

\index[routine]{twiss_propagate_all}
\label{r:twiss.propagate.all}
\item[twiss_propagate_all (lat, ix_branch, err_flag, ie_start, ie_end, zero_uncalculated)] \Newline
Routine to propagate the Twiss parameters from the start to the end. 

\index[routine]{twiss_to_1_turn_mat}
\label{r:twiss.to.1.turn.mat}
\item[twiss_to_1_turn_mat (twiss, phi, mat2)] \Newline
Routine to form the 2x2 1-turn transfer matrix from the Twiss parameters. 

\end{description}

%------------------------------------------------------------------------
\section{Twiss: 6 Dimensional}
\label{r:twiss6}    
\index{twiss!list of routines}

\begin{description}

\index[routine]{normal_mode3_calc}
\label{r:normal.mode3.calc}
\item[normal_mode3_calc (t6, tunes, B, HV, above_transition)] \Newline 
Decompose a 2n x 2n symplectic matrix into normal modes.
For more details see:

\index[routine]{twiss3_propagate_all}
\label{r:twiss3.propagate.all}
\item[twiss3_propagate_all (lat, ix_branch)] \Newline 
Routine to propagate the twiss parameters using all three normal modes.

\index[routine]{twiss3_propagate1}
\label{r:twiss3.propagate1}
\item[twiss3_propagate1 (ele1, ele2, err_flag)] \Newline 
Routine to propagate the twiss parameters using all three normal modes.

\index[routine]{twiss3_at_start}
\label{r:twiss3.at.start}
\item[twiss3_at_start (lat, err_flag, ix_branch, tune3)] \Newline 
Routine to propagate the twiss parameters using all three normal modes.

\end{description}

%------------------------------------------------------------------------
\section{Wake Fields}
\label{r:wake}    
\index{wake fields!list of routines}

\begin{description}

\index[routine]{init_wake}
\label{r:init.wake}
\item[init_wake (wake, n_sr_long, n_sr_trans, n_lr_mode, n_lr_spline, always_allocate)] \Newline 
Routine to initialize a wake struct.

\index[routine]{randomize_lr_wake_frequencies}
\label{r:randomize.lr.wake.frequencies}
\item[randomize_lr_wake_frequencies (ele, set_done)] \Newline 
Routine to randomize the frequencies of the lr wake HOMs.

\index[routine]{sr_long_wake_particle}
\label{r:sr.long.wake.particle}
\item[sr_long_wake_particle (ele, orbit)] \Newline 
Subroutine to apply the short-range wake kick to a particle and then add 
to the existing short-range wake the contribution from the particle.

\index[routine]{sr_trans_wake_particle}
\label{r:sr.trans.wake.particle}
\item[sr_trans_wake_particle (ele, orbit)] \Newline 
Subroutine to apply the short-range wake kick to a particle and then add 
to the existing short-range wake the contribution from the particle.

\index[routine]{track1_sr_wake}
\label{r:track1.sr.wake}
\item[track1_sr_wake (bunch, ele)] \Newline 
Routine to apply the short range wake fields to a bunch. 

\index[routine]{track1_lr_wake}
\label{r:track1.lr.wake}
\item[track1_lr_wake (bunch, ele)] \Newline 
Routine to put in the long-range wakes for particle tracking.

\index[routine]{zero_lr_wakes_in_lat}
\label{r:zero.lr.wakes.in.lat}
\item[zero_lr_wakes_in_lat (lat)] \Newline 
Routine to zero the long range wake amplitudes for the elements that have
long range wakes in a lattice.

\end{description}

%------------------------------------------------------------------------
\section{C/C++ Interface}
\label{r:c.interface}  
\index{C/C++ interface!list of routines}

\begin{description}

\index[routine]{fscalar2scalar}
\label{r:fscalar2scalar}
\item[fscalar2scalar (f_scalar, n) result (c_scalar)] \Newline 
Function to translate a scalar from Fortran form to C form.

\index[routine]{fvec2vec}
\label{r:fvec2vec}
\item[fvec2vec (f_vec, n) result (c_vec)] \Newline 
Function to translate a vector from Fortran form to C form.

\index[routine]{mat2vec}
\label{r:mat2vec}
\item[mat2vec (mat, n) result (vec)] \Newline 
Function to take a matrix and turn it into an array in C standard row-major order.

\index[routine]{tensor2vec}
\label{r:tensor2vec}
\item[tensor2vec (tensor, n) result (vec)] \Newline 
Function to take a tensor and turn it into an array in 
C standard row-major order::

\index[routine]{vec2mat}
\label{r:vec2mat}
\item[vec2mat (vec, mat)] \Newline 
Routine to take a an array in C standard row-major 
order and turn it into a matrix.

\index[routine]{vec2tensor}
\label{r:vec2tensor}
\item[vec2tensor (vec, tensor)] \Newline 
Routine to take a an array in C standard row-major 
order and turn it into a tensor.

\index[routine]{remove_null_in_string}
\label{r:remove.null.in.string}
\item[remove_null_in_string (str_in, str_out] \Newline 
Routine to convert a null character in a string to a blank.

\index[routine]{f_logic}
\label{r:f.logic}
\item[f_logic (logic) result (f_log)] \Newline 
Function to convert from a C logical to a Fortran logical.

\index[routine]{c_logic1}
\label{r:c.logic1}
\item[c_logic1 (logic) result (c_log)] \Newline 
Function to convert from a fortran logical to a C logical.
See c_logic for more details.

\index[routine]{c_logic_vec}
\label{r:c.logic.vec}
\item[c_logic_vec (logic) result (c_log)] \Newline 
Function to convert from a fortran logical to a C logical.
See c_logic for more details.

\index[routine]{f_logic_int}
\label{r:f.logic.int}
\item[f_logic_int (logic) result (f_log)] \Newline 
Function to convert from a C logical to a Fortran logical.
This function is overloaded by f_logic.

\index[routine]{f_logic_bool}
\label{r:f.logic.bool}
\item[f_logic_bool (logic) result (f_log)] \Newline 
Function to convert from a C logical to a Fortran logical.
This function is overloaded by f_logic.

\index[routine]{remove_null_in_string_arr}
\label{r:remove.null.in.string.arr}
\item[remove_null_in_string_arr (str_in, str_out)] \Newline 
This routine overloaded by:
       remove_null_in_string

\index[routine]{remove_null_in_string_char}
\label{r:remove.null.in.string.char}
\item[remove_null_in_string_char (str_in, str_out)] \Newline 
This routine overloaded by:
       remove_null_in_string

\index[routine]{to_c_str}
\label{r:to.c.str}
\item[to_c_str (f_string, c_string)] \Newline 
Subroutine to append a null (0) character at the end of a string (trimmed
of trailing blanks) so it will look like a C character array. 

\index[routine]{c_string}
\label{r:c.string}
\item[c_string(f_string) ] \Newline 
Functional form of subroutine to_c_str (f_string, c_string)

\index[routine]{to_f_str}
\label{r:to.f.str}
\item[to_f_str (c_string, f_string)] \Newline 
Subroutine to append a null (0) character at the end of a string (trimmed
of trailing blanks) so it will look like a C character array. 

\end{description}

% %------------------------------------------------------------------------
% \section{Deprecated}
% \label{r:deprecated}
% \index{deprecated routines}
%
% \begin{description}
%
% \end{description}
