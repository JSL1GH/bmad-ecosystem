%---------------------------------------------------------------------------------------------------------------
%---------------------------------------------------------------------------------------------------------------
% WARNING! If you are modifying this file, be aware that the online help system depends upon the lines starting 
% with "%%" to match blocks of text in this file with a given command. 
%
% The online help system also makes some further assumptions about how this file is formatted.
% Please test any modifications by running Tao and using the appropriate help command to see how
% your modifications translate. 
% Note: The translation code is at: 
%     tao/code/tao_help.f90
%---------------------------------------------------------------------------------------------------------------
%---------------------------------------------------------------------------------------------------------------

\chapter{Commands}
\label{c:command}
\index{commands!Command list} 

\tao has two \vn{modes} for entering commands. In \vn{line mode}'', described in this chapter, \tao
waits until the \vn{return} key is depressed to execute a command. That is, a command consists of a
single line of input. Conversely, \vn{Single Mode}, which is described in \vn{Single Mode} chapter
(\sref{c:single}), interprets each keystroke as a command. Single Mode is useful for quickly varying
parameters to see how they affect a lattice but the number of commands in Single Mode is limited. To
put \tao into \vn{single mode} use the \vn{single_mode} command (\sref{s:sing}).

The syntax for \vn{line mode} commands is discussed in Section~\sref{s:com.syntax}. The list of
commands is shown in Table~\ref{t:commands}.

This chapter uses the following special characters to define the command line syntax:
\begin{example}
  \{\}        ! Identifies an optional argument.
            !   Arguments now enclosed in brackets are required
  <>        ! Indicates a non-literal argument.
\end{example}

Example:
\begin{example}
  change \{-silent\} variable <name>[<locations>] <number>
\end{example}
Here the \vn{-silent} argument is optional while the \vn{variable} argument is mandatory.
Appropriate values for \vn{<name>}, \vn{<locations>}, and \vn{<number>} must be substituted. A
possible

\begin{example}
  change var steering[34:36] @1e-3  ! set the steering strength #34-36 to 0.001
\end{example}

%% command_table -----------------------------------------------------

\begin{table}[h]
\centering {\tt
\begin{tabular}{ll|ll} \toprule
  {\it Command} & {\it Section}     & {\it Command} & {\it Section}     \\ \midrule
  alias         & \sref{s:alias}    & re_execute    & \sref{s:re.exe}   \\
  call          & \sref{s:call}     & read          & \sref{s:read}     \\
  change        & \sref{s:change}   & reinitialize  & \sref{s:reinit}   \\ 
  clear         & \sref{s:clear}    & restore       & \sref{s:restore}  \\ 
  clip          & \sref{s:clip}     & run_optimizer & \sref{s:run}      \\ 
  cut_ring      & \sref{s:cut.ring} & scale         & \sref{s:scale}    \\ 
  continue      & \sref{s:continue} & set           & \sref{s:set}      \\  
  derivative    & \sref{s:deriv}    & show          & \sref{s:show}     \\ 
  do, enddo     & \sref{s:do}       & single_mode   & \sref{s:sing}     \\ 
  end_file      & \sref{s:end.file} & spawn         & \sref{s:spawn}    \\ 
  exit          & \sref{s:exit}     & taper         & \sref{s:taper}    \\
  flatten       & \sref{s:flatten}  & timer         & \sref{s:timer}    \\
  help          & \sref{s:help}     & use           & \sref{s:use}      \\ 
  ls            & \sref{s:ls}       & veto          & \sref{s:veto}     \\ 
  misalign      & \sref{s:misalign} & wave          & \sref{s:wave}     \\
  pause         & \sref{s:pause}    & write         & \sref{s:write}    \\ 
  place         & \sref{s:place}    & x_axis        & \sref{s:x.axis}   \\ 
  ptc           & \sref{s:ptc}      & x_scale       & \sref{s:x.scale}  \\ 
  python        & \sref{s:python}   & xy_scale      & \sref{s:xy.scale} \\
  quit          & \sref{s:quit}     &               &                   \\
\bottomrule 
\end{tabular}}
\caption{Table of \tao commands.}
\label{t:commands}
\end{table}

When running \tao, use the \vn{help} (\sref{s:help}) command to show documentation on any command.
For example, \vn{help plot} will show documentation on the \vn{plot} command.

%% Marker: "help" will not display anything after this  -----------

\vfil
\break

%% alias --------------------------------------------------------------
\section{alias}\index{commands!alias}
\label{s:alias}

The \vn{alias} command defines command shortcuts. Format:
\begin{example}
  alias \{<alias_name> <string>\}
\end{example}

\vskip 7pt 

\vn{Alias} is like Unix aliases. Using the \vn{alias} command without any arguments
results in a printout of the aliases that have been defined. When using an alias up to 9 arguments
may be substituted in the \vn{<string>}. The i\Th argument is substituted in place of the sub-string
``[[i]]'' or ``[<i>]''.  Arguments that do not have a corresponding ``[[i]]'' or ``[<i>]'' are
placed at the end of \vn{<string>}. The difference between ``[[i]]'' and ``[<i>]'' is that ``[[i]]''
is a required argument while ``[<i>]'' defines an optional argument. For example
\begin{example}
  alias aaa show element [[1]] [[2]]
  alias zzz show element [[1]] [<2>]
\end{example}
This defines ``\vn{aaa}'' as an alias for the \vn{show element} command with two required arguments
while ``\vn{zzz}'' has only one requred argument.

Aliases can be set up for multiple commands using semicolons.

Examples:
\begin{example}
  alias xyzzy plot [[1]] model  ! Define xyzzy
  alias                         ! Show all aliases
  xyzzy top                     ! Use an alias
  plot top model                ! Equivalent to "xyzzy top"
  xyzzy top abc                 ! Equivalent to "plot top model abc"
  alias foo  show uni; show top ! "foo" equivalent to "show uni; show top"
\end{example}
In the above example ``xyzzy'' is the alias for the string ``plot [[1]] model''.  When the
command xyzzy is used ``top'' is substituted for ``[[1]]'' in the string.

%% call --------------------------------------------------------------
\section{call}\index{commands!call}
\label{s:call}

The \vn{call} command opens a command file (\sref{s:command.files}) and executes the commands in
it. Format: 
\vskip 1pt
\begin{example}
  call <filename> \{<arg_list>\}
  call -no_calc \{<arg_list>\}
  call -ptc <filename>
\end{example}

\vskip 1pt 
The \vn{call} command without \vn{-ptc} is for running a set of \tao commands.  Up to 9
arguments may be passed to the command file. The i\Th argument is substituted in place of the string
``[[i]]'' in the file. Nesting of command files (command files calling other command files) is
allowed. There is no limit to the number of nested files.  See the \vn{Command Files and Aliases}
section (\sref{s:command.files}) for more details.

The \vn{call -ptc} command passes the command file to PTC for processing. Previous to such a call,
the command \vn{ptc init} must be issued. This is for PTC wizards only.

If a command file calls another command file, and the name of the second command file has a relative
(as opposed to absolute) path name, \tao will look for the second command file relative to the
directory of the first command file. To have \tao look relative to your current working directory
(where you started \tao), use the prefix \vn{\$PWD/}. For example, to call a command file that is
one level up from your current working directory use
\begin{example}
  call \$PWD/../second.cmd
\end{example}

Command loops can be implemented in a command file. See the documentation on \vn{do/enddo}
(\sref{s:do}) for more details.

The \vn{-no_calc} option is equivalent to putting the following at the beginning of the
command file to speed up execution time:
\begin{example}
  set global lattice_calc_on = F   ! Stop lattice calculations (\sref{s:lat.calc}).
  set global plot_on = F           ! Halt replotting 
or
  set calc -on                     ! Same as the two lines above.
\end{example}
When using the \vn{-no_calc} option, at the end of the command file the \vn{lattice_calc_on} and
\vn{plot_on} logicals will be toggled back to their initial values.

To suppress the output when running a command file use the command:
\begin{example}
  set global quiet = all
\end{example}
Note: quiet mode is automatically turned off at the end when a command file exits back to the
terminal level. See the \vn{Tao_global_struct Structure} section (\sref{s:tao.global.struct}) for
more details.

Examples:
\begin{example}
    call -no my_cmd_file abc def 
\end{example}
In the above example the argument ``abc'' is substituted for any ``[[1]]'' appearing the
file and ``def'' is substituted for any ``[[2]]''.  \Newline

%% change --------------------------------------------------------------
\section{change}\index{commands!change}
\label{s:change}

The \vn{change} command changes element attribute values or variable values in the \vn{model}
lattice. Format:
\begin{example}
  change \{-update\} element <element_list> <attribute> \{prefix>\} <number>
  change \{-silent\} variable <name>[<locations>] \{<prefix>\} <number>
  change  \{n@\}particle_start <coordinate> \{prefix>\} <number>
\end{example}

\vskip 10pt 
The \vn{change} is used for changing real (as opposed to integer or logical) parameters. Also
consider using the \vn{set} command (\sref{s:set}) which is more general.

If \vn{<prefix>} is not present, \vn{<number>} is added to the existing value
of the attribute or variable. That is:
\begin{example}
  final_model_value = initial_model_value + <number>
\end{example}
If \vn{<prefix>} is present, it may be one of
\begin{example}
  @       final_model_value = <number>
  d       final_model_value = design_value + <number>
  \%       final_model_value = initial_model_value * (1 + <number> / 100)
\end{example}

Element list format (\sref{s:ele.list.format}), without any embedded blanks, is used for
the \vn{<element_list>} argument.

For \vn{change particle_start}, The optional \vn{n@} universe specification (\sref{s:universe}) may
be used to specify the universe or universes to apply the change command to.

For lattices with an open geometry, \vn{change particle_start <coordinate> <number>} can be used to
vary the starting coordinates for single particle tracking. If the \vn{use_particle_start} of the
\vn{beam_init} structure (\sref{s:beam.init}) is set to True, \vn{particle_start} will also vary the
beam centroid and the beam particle spin for tracking. Here \vn{<coordinate>} is one of:
\begin{example}
  x, px, y, py, z, pz, t
\end{example}
For photons, \vn{<coordinate>} may also be:
\begin{example}
  field_x, field_y, phase_x, phase_y
\end{example}
For closed lattices only the \vn{pz} component is applicable. For lattices that have an \vn{e_gun}
(which necessarily implies that the lattice has an open geometry), the time \vn{t} coordinate must
be varied instead of \vn{pz}.

For open lattices, \vn{change element beginning <twiss>} can be used to vary the starting Twiss
parameters where \vn{<twiss>} is one of:
\begin{example}
  beta_a, beta_b, alpha_a, alpha_b 
  eta_a, eta_b,etap_a, etap_b    
\end{example}

The \vn{-silent} switch, if present, suppresses the printing of what variables are changed.

The \vn{-update} switch, if present, suppresses \tao from printing error messages if a ``variable
slave value mismatch'' is detected (\sref{s:var.mismatch}). Independent of whether \vn{-update} is
present or not, \tao will fix the mismatch using the changed value to set all of the slave values.

Note: The \vn{change element} command can be used with \vn{ramper} type elements.

Examples:
\begin{example}
  change ele 3@124 x_offset 0.1        ! Offset element #124 in universe 3 by 0.1
  change ele 1,3:5 x_offset 0.1        ! Offset elements 1, 3, 4, and 5 by 0.1
  change ele q* k1 d 1.2e-2            ! Set the k1 strength of all elements starting with
                                       !   the letter "q" relative to the design
  change ele quadrupole::* k1 d 1.2e-2 ! Set the k1 strength of all quadrupole elements.
  change var steering[34:36] @1e-3     ! set the steering strength #34-36 to 0.001
  change var steering[*] \%10           ! vary all steering strengths by 10\%
  change 2@particle_start x @0.001     ! set beginning x position in universe 2 to 1 mm.
\end{example}


%% clear --------------------------------------------------------------
\section{clear}\index{commands!clear}
\label{s:clear}

The \vn{clear} command clears stored spin and orbital Taylor maps from all elements in a lattice
with the exception of \vn{Taylor} elements (which are specified in the lattice file as opposed to
being calculated by \bmad). Format:
\begin{example}
  clear maps
\end{example}

Clearing the Taylor maps may be needed if the maps are in use (for example, with a spin polarization
calculation) and orbit excursions place the calculated orbit outside of the range of validity of the
maps.

%% clip --------------------------------------------------------------
\section{clip}\index{commands!clip}
\label{s:clip}

The \vn{clip} command vetoes data points for plotting and optimizing. That is, the \vn{good_user}
logical of the data associated with the out-of-bound plotted points are set to False.  Format:
\begin{example}
  clip \{-gang\} \{<where> \{<limit1> \{<limit2>\}\}\}
\end{example}

\vskip 10pt Which graphs are clipped is determined by the \vn{<where>} switch. If \vn{<where>} is
not present, all graphs are clipped. If \vn{where} is a plot name, then all the graphs of that plot
are clipped. If \vn{where} is the name of a \vn{d2_data} (for example, \vn{orbit}) or a \vn{d1_data}
(for example, \vn{orbit.x}) structure, then those graphs that display this data are clipped.

The points that are clipped those points whose $y$ values are outside a certain range defined by
\vn{<limit1>} and \vn{<limit2>}. If neither \vn{<limit1>} nor \vn{<limit2>} are present, the clip
range is taken to be outside the graph minimum and maximum $y$--axis values. If only \vn{<limit1>}
is present then the clip range is outside the region from -\vn{<limit1>} to +\vn{<limit1>}. If both
are present than the range is from \vn{<limit1>} to \vn{<limit2>}.

The \vn{-gang} switch is apply a clip to corresponding data in a \vn{d2_data} structure. For example
\begin{example}
  clip -g orbit.x   ! Clips both orbit.x and orbit.y 
\end{example}
Here the \vn{orbit.x} data is clipped and the corresponding data in \vn{orbit.y} is also vetoed. For
example, if datum number 23 in \vn{orbit.x} is clipped, datum number 23 in \vn{orbit.y} will be
vetoed.

Examples:
\begin{example}
  clip top.x -3  7  ! Clip the curves in the x graph in the region named "top".
  clip bottom       ! Clip the graphs in the "bottom" region
  clip -g orbit.x   ! Clip the orbit.x graph and also veto corresponding points
                    ! in other graphs of the orbit plot.
\end{example}

%% cut_ring --------------------------------------------------------------
\section{cut_ring}\index{commands!cut_ring}
\label{s:cut.ring}

Format:
\begin{example}
  cut_rung
\end{example}

The \vn{cut_ring} command is used to switch the geometry of the viewed \vn{model} lattice from
\vn{closed} to \vn{open} and vice versa.

%% continue --------------------------------------------------------------
\section{continue}\index{commands!continue}
\label{s:continue}

The \vn{continue} command is used to continue reading of a suspended command file
(\sref{s:command.files}) after a \vn{pause} command (\vn{s:pause}). Format:
\begin{example}
  continue
\end{example}

%% do --------------------------------------------------------------
\section{do/enddo command file looping}\index{commands!do}
\label{s:do}

Command loops can be implemented in a command file files. Format:
\begin{example}
  do <var> = <l_bound>, <u_bound> \{, <incr>\}
    ...   ! use the syntax ``[[<var>]]'' to refer to a variable.
  enddo
\end{example}
Note: ``\vn{enddo}'' is one word and my not be split into two words. Loops can be nested and the
number of levels is not unlimited.

A loop will execute the code in between the \vn{do} and \vn{enddo} lines a certain number of
times. Each time trough the the the integer variable \vn{<var>} will be incremented by \vn{<incr>},
starting at \vn{<l_bound>} and stopping before \vn{<var>} is greater than \vn{<u_bound>}. If
\vn{<incr>} is not present, the increment will be 1. Note: \vn{<l_bound>}, \vn{<u_bound>}, and
\vn{<incr>} must all be integers.

Example:
\begin{example}
  do j = 0, 10, 2
    set particle_start pz = 1e-3 * [[j]]
    ...
  enddo
\end{example}
As shown in the above example, to refer to a loop variable in a command, use the syntax ``[[<var>]]''.

%% end_file --------------------------------------------------------------
\section{end_file} \label{s:end.file}
\index{commands!end_file}

The \vn{end_file} command is used in command files (\sref{s:command.files}) to signal the end of the
file. Everything after an \vn{end_file} command is ignored. An \vn{end_file} command entered at the
command line will simply generate an error message.  Format:
\begin{example}
  end_file
\end{example}

%% exit --------------------------------------------------------------
\section{exit}\index{commands!exit}
\label{s:exit}

The \vn{exit} command exits the program. Same as \vn{Quit}.  Format:
\begin{example}
  exit
\end{example}

%% derivative --------------------------------------------------------------
\section{derivative}\index{commands!derivative}
\label{s:deriv}

The \vn{derivative} command calculates the \vn{dModel_Data/dVar} derivative matrix needed for the
\vn{lm} optimizer.  Format:
\begin{example}
  derivative
\end{example}

%% flatten --------------------------------------------------------------
\section{flatten}\index{commands!flatten}
\label{s:flatten}

The \vn{Flatten} command runs the optimizer to minimize the merit function. This is the same as the
\vn{run_optimizer} command.  See the \vn{run_optimizer} command for more details. Format:
\begin{example}
  flatten \{<optimizer>\}
\end{example}

\vskip 10pt

%% help --------------------------------------------------------------
\section{help}\index{commands!help}
\label{s:help}

The \vn{help} command gives help on \tao commands. Format:
\begin{example}
  help \{<command> \{<subcommand>\}\}
\end{example}

\vskip 10pt
The \vn{help} command without any arguments gives a list of all commands.  Some commands, like
\vn{show}, are so large that help on these commands is divided up by their subcommand.

Examples:
\begin{example}
  help            ! Gives list of commands.
  help run        ! Gives help on the run_optimizer command.
  help show       ! Help on the show command.
  help show alias ! Help on the show alias command.
\end{example}

The \vn{help} command works by parsing the file \vn{\$TAO_DIR/doc/command-list.tex} which is the
LaTeX file for the \vn{Tao Commands} chapter of the \tao manual. For the \vn{help} command to work
properly, the environmental variable \vn{TAO_DIR} must be appropriately defined. Generally,
\vn{TAO_DIR} will be defined if the appropriate \bmad setup script has been run. For
``Distributions'', this is the same setup script used to setup a distribution. See your local \bmad
guru for details.

When the \vn{help} command parses the \vn{\$TAO_DIR/doc/command-list.tex} file, LaTeX syntax will be
modified to produce a reasonable looking output on the terminal. This translation is not perfect so
reference should be made to the \tao manual if there is a problem in the translation.

%% misalign --------------------------------------------------------------
\section{misalign}\index{command!misalign}
\label{s:misalign}

%% ls --------------------------------------------------------------
\section{ls}\index{command!ls}
\label{s:ls}

The \vn{ls} command is the same as the standard UNIX \vn{ls} command to display a list of files and
directories. The standard ls switches are accepted.

This is equivalent to the \vn{spawn ls} command.

%% pause --------------------------------------------------------------
\section{pause}\index{commands!pause}
\label{s:pause}

The \vn{pause} command is used to pause \tao when executing a command file
(\sref{s:command.files}). Format:
\begin{example}
  pause \{<time>\} ! Pause time in seconds.
\end{example}

\vskip 10pt
If \vn{<time>} is not present or zero, \tao will pause until the \vn{CR} key is pressed. Once the
\vn{CR} key is pressed, the command file will be resumed. If \vn{<time>} is negative, \tao will
suspend the command file. Commands can now be issued from the keyboard and the command file will be
resumed when a \vn{continue} command (\sref{s:continue}) is issued. Multiple command files can be
simultaneously suspended.  Thus, while one command file is suspended, a second command file can be
run and this command file too can be suspended. A \vn{continue} command will resume the second
command file and when that command file ends, another \vn{continue} command will be needed to
complete the first suspended command file. Use the \vn{show global} command to see the number of
suspended command files.

Example:
\begin{example}
  pause 1.5    ! Pause for 1.5 seconds.
  pause -1     ! Suspend the command file until a \vn{continue} 
               !   command is issued.
\end{example}

%% place --------------------------------------------------------------
\section{place}\index{commands!place}
\label{s:place}

The \vn{place} command is used to associate a \vn{<template>} plot with a \vn{<region>} and thus
create a visible plot in that region. Format:
\begin{example}
  place \{-no_buffer\} <region> <template>
  place <region> none
  place * none
\end{example}

\vskip 10pt 
If \vn{<region>} is set to ``\vn{*}'' then all regions are selected.

If \vn{<template>} is set to ``\vn{none}'' all selected regions are cleared of plots.

The \vn{-no_buffer} optional switch is used when external plotting is being done (EG with a GUI) and
is not of interest otherwise.

Notice that by using multiple \vn{place} commands a \vn{template} can be associated with more than
one region. For example, if multiple orbit plots are desired.

Examples:
\begin{example}
  place * none     ! Erase all plots.
  place top orbit  ! Place the orbit template in the top region
  place top none   ! Erase any plots in the top region
\end{example}

%% ptc -----------------------------------------------------------
\section{ptc}\index{commands!ptc}
\label{s:ptc}

The \vn{ptc} command is used manipulating PTC layouts associated with Bmad
lattices. Format:
\begin{example}
  ptc init            ! Init associated PTC layout.
\end{example}

\vskip 10pt 
The \vn{ptc init} command must be run before running any other \vn{ptc} command is used.

Also see:
\begin{example}
  call -ptc <file>         ! Run a PTC script
  read ptc                 ! Read a PTC lattice
  write ptc                ! Write a PTC lattice
\end{example}

Examples:
\begin{example}
  ptc init
\end{example}

%% python -----------------------------------------------------------
\section{python}\index{commands!python}
\label{s:python}

The \vn{python} command is like the \vn{show} command in that the \vn{python} command prints
information to the terminal. The difference is that the output from the \vn{show} command is meant
for viewing by the user while the output of the \vn{python} command is meant for easy
parsing. Format:
\begin{example}
  python \{-append <file_name>\} \{-noprint\} <subcommand> <arguments>
  python \{-write <file_name>\} \{-noprint\} <subcommand> <arguments>
\end{example}

The \vn{python} command has \vn{-append} and \vn{-write} optional arguments which can be used to
write the results to a file. The \vn{python -append} command will appended to the output file. The
\vn{python -write} command will first erase the contents of the output file. Example:
\begin{example}
  python -write d2.dat data_d2    ! Write to file "d2.dat"
\end{example}

The \vn{-noprint} option suppresses printing and is useful when writing large amounts of data to a
file.  The \vn{python} command can be used to pass information to a parent process when \tao is run
as a subprocess.  The parent process may be any scripting program like Python, Perl, Tcl, etc.  In
particular, see the \vn{Python/GUI Interface} chapter (\sref{c:python}) for details on how to run
\tao as a Python subprocess.

In terms of long term maintainability, the advantage of using the \vn{python} command in the scripts
over the \vn{show} command comes from the fact that the output syntax of \vn{show} commands can (and
does) change.

For further documentation on the python command and interfacing to python is in the \vn{Python/GUI
Interface} chapter (\sref{c:python}).

Documentation on interfacing Python scripts to \tao's python command is given in the \vn{Tao Python
Command} section (\sref{s:python.python}).

List of possible \vn{<what_to_print>} choices:
\begin{example}
  beam, beam_init, branch1, bunch_params, bunch1, bmad_com, building_wall_list
  building_wall_graph, building_wall_point, building_wall_section constraints,
  da_params, da_aperture data, data_d2_create, data_d2_destroy, data_d_array,
  data_d1_array data_d2, data_d2_array, data_set_design_value, data_parameter
  datum_create, datum_has_ele, derivative, ele:ac_kicker, ele:cartesian_map,
  ele:chamber_wall, ele:control_var, ele:cylindrical_map, ele:elec_multipoles,
  ele:floor, ele:grid_field, ele:gen_attribs, ele:head, ele:lord_slave, ele:mat6,
  ele:methods, ele:multipoles, ele:orbit, ele:param, ele:photon, ele:spin_taylor,
  ele:taylor, ele:taylor_field, ele:twiss, ele:wake, ele:wall3d, em_field, enum,
  evaluate, floor_plan, floor_orbit, global, help, inum lat_branch_list,
  lat_calc_done, lat_ele_list, lat_list, lat_param_units matrix, merit, orbit_at_s
  place_buffer, plot_curve, plot_graph, plot_histogram, plot_lat_layout, plot_line
  plot_plot_manage, plot_graph_manage, plot_curve_manage plot_list, plot_symbol,
  plot_transfer, plot1, ptc_com, ring_general, shape_list shape_manage,
  shape_pattern_list, shape_pattern_manage, shape_pattern_point_manage, shape_set
  show, species_to_int, species_to_str, spin_polarization, spin_resonance,
  super_universe twiss_at_s, universe var_v1_create, var_v1_destroy, var_create,
  var_general, var_v1_array, var_v_array, var wave
\end{example}

%% quit --------------------------------------------------------------
\section{quit}\index{commands!quit}
\label{s:quit}

\vn{Quit} exits the program. Same as \vn{exit}.
Format:
\begin{example}
  quit
\end{example}

%% re_execute --------------------------------------------------------------
\section{re_execute}
\index{commands!re_execute}
\label{s:re.exe}

The \vn{re_execute} command reruns prior commands.  Format:
\begin{example}
  re_execute <index>   ! Re-execute a command with the given index number.
  re_execute <string>  ! Re-execute last command that begins with <string>.
\end{example}

\vskip 10pt 

Every \tao command entered is recorded in a ``history stack''. These commands can be viewed using
the \vn{show history} command. The \vn{show history} command will also display the index number
associated with each command.

Note: The up and down arrow keys on the keyboard can be used to scroll through the command history
stack.

Examples
\begin{example}
  re_exe 34   ! Re-execute command number 34.
  re_exe set  ! Re-execute last ``set'' command.  
\end{example}

%% read --------------------------------------------------------------
\section{read}\index{commands!read}
\label{s:read}

The \vn{read} command is used to modify the (\bmad) \vn{model} lattice or the associated \vn{PTC}
lattice. Format:
\begin{example}
  read lattice \{-silent\} \{-universes <universe-list>\} <file_name>
  read ptc <file_name>
\end{example}

\vskip 10pt 

With the \vn{read lattice} command, the \vn{model} lattices contained in the universes specified by
\vn{<universe-list>} are modified using a ``secondary lattice'' file.  [See the \bmad manual for the
definition of ``secondary lattice''.] For example, with the appropriate file, the \vn{read} command
can be used to misalign the lattice elements. For the \vn{read lattice} command, the input file must
be in Bmad standard lattice format.

If \vn{-universes} is not present, only the \vn{model} lattice
in the default universe is modified.

If, after the lattice file has been read in, a given \tao variable has slave parameters that have
different values there is a problem. For example, if a \tao variable controls the \vn{k2} value of
sextupoles elements \vn{S1} and \vn{S2}, and if \vn{S1} is set to a different value than \vn{S2},
there is an inconsistency which needs to be corrected. This can be done in a number of ways. For
example, by using the \vn{set ele -update} command or using a further \vn{read lattice} command with
a lattice that corrects the problem.

If desired, the \vn{-silent} switch can be used to suppress error messages about differing \tao
variable slave parameter values.

Note: Due to bookkeeping complications, the number of lattice elements may not be modified. If it is
desired to initiate \tao using both ``primary'' and secondary lattice files, this can be done as
illustrated in \sref{s:init.lat}.

The \vn{read ptc} command reads in a PTC lattice. WARNING: This command is untested. Please contact
David Sagan if you want to use it.

Examples:
\begin{example}
  read lat -uni * lat.bmad   ! Modify model lattice of all universes.
  read lat -uni 2,3 lat.bmad ! Modify model lattice universes 2 and 3.
\end{example}

%% reinitialize -------------------------------------------------------
\section{reinitialize}\index{commands!reinitialize}
\label{s:reinit}

The \vn{reinitialize} command reinitializes various things. Format:
\begin{example}
  reinitialize beam
  reinitialize data
  reinitialize tao \{-clear\} \{command line optional arguments\}
\end{example}

\vskip 10pt 

The \vn{reinitialize beam} command reinitializes the beam at the start of the lattice. That is, a
new random distribution is generated.  Note: This also reinitializes the model data.

\vn{reinitialize data} forces a recalculation of the model data.  Normally, a recalculation is done
automatically when any lattice parameter is changed so this command is generally only useful for
debugging purposes.

\vn{reinitializes tao} reinitializes \tao. This can be useful to reset everything to initial
conditions or to perform analysis with more than one initialization file. See the Command Line
Initialization section (\sref{s:command.line}) for a list of the optional arguments. If an argument
is not set, the \vn{reinitialize} command uses the same argument value that were used in the last
\vn{reinitialize} command, or, if this is the first reinitialization, what was used to start \tao.
Exception: If the \vn{-clear} switch is present, all initialization parameters are set to their
default state before the command line arguments specified in the \vn{reinitialize} command are
parsed. The \vn{-clear} switch, if used, should come before any command line arguments since if
there are command line arguments before the \vn{-clear} switch, these arguments will be cleared.

Examples:
\begin{example}
  reinit tao                         ! Reinit using previous arguments
  reinit tao -init special.init      ! Reinitializes \tao with the initialization file 
                                     !   special.init.
  reinit -clear -start my_start      ! Use default init values except for the start file.                    
\end{example}

%% restore --------------------------------------------------------------
\section{restore}\index{commands!restore}
\label{s:restore}

The \vn{restore} command cancels data or variable vetoes. Format:
\begin{example}
  restore data  <data_name> <locations>
  restore var <var_name> <locations>
\end{example}

\vskip 10pt 

See also the \vn{use} and \vn{veto} commands.

Examples:
\begin{example}
  restore data orbit.x[23,34:56]   ! un-veto orbit.x 23 and 34 through 56.
  restore data orbit.x[23,34:56:2] ! un-veto orbit.x 23 and even data between 34 
                                   !                                          and 56
  restore data *@orbit[34]         ! un-veto orbit data in all universes.
  restore var quad_k1[67]          ! un-veto variable
\end{example}

%% run --------------------------------------------------------------
\section{run_optimizer}\index{commands!run}
\label{s:run}

The \vn{run_optimizer} command runs an optimizer. Format:
\begin{example}
  run_optimizer \{<optimizer>\}
\end{example}

\vskip 10pt 

\index{de!optimizer}\index{lm!optimizer}
If \vn{<optimizer>} is not given then the default optimizer is used.  Use the \vn{show optimizer}
(\sref{s:show.optimizer}) command to see optimizer parameters.  To stop the optimizer before it is
finished press the period ``.''  key. If you want the optimizer to run forever run the optimizer in
\vn{single mode}. Valid optimizers are:
\begin{example}
  custom        ! Used when a custom optimizer has been implemented (\sref{c:custom.tao}).
  de            ! Differential Evolution (good for global optimizations).
  geodesic_lm   ! ``Geodesic'' Levenburg-Marquardt (good for local optimizations).
  lm            ! Levenburg-Marquardt (good for local optimizations).
  lmdif         ! Levenburg-Marquardt (alternative version) (good for local optimizations).
  svd           ! svd optimizer (good for local optimizations).
\end{example}

See the optimization chapter (\sref{c:opti}) for details on how \tao structures optimization and for
more details on the different optimizers.

Examples:
\begin{example}
  run         ! Run the default optimizer
  run de      ! Run the de optimizer
\end{example}

%% scale --------------------------------------------------------------
\section{scale}\index{commands!scale}
\label{s:scale}

The \vn{scale} command scales the vertical axis of a graph or set of graphs.  Format:
\begin{example}
  scale \{-exact\} \{-gang\} \{-include_wall\} \{-nogang\} 
             \{-y\} \{-y2\} \{<where> \{<value1> \{<value2>\}\}\}
\end{example}

Which graphs are scaled is determined by the \vn{<where>} switch. If \vn{<where>} is not present or
\vn{<where>} is \vn{all} then all graphs are scaled. \vn{<where>} can be a plot name or the name of
an individual graph withing a plot.

\vn{scale} adjusts the vertical scale of graphs. If neither \vn{<value1>} nor \vn{<value2>} is
present then an \vn{autoscale} is performed and the scale is adjusted so that all the data points
are within the graph region. If an autoscale is performed upon an entire plot, and if
\vn{plot%autoscale_gang_y} (\sref{s:template}) is True, then the chosen scales will be the same for
all graphs. That is, a single scale is calculated so that all the data of all the graphs is within
the plot region. The affect of \vn{plot%autoscale_gang_y} can be overridden by using the \vn{-gang}
or \vn{-nogang} switches.

If only \vn{<value1>} is present then the scale is taken to be from -\vn{<value1>} to +\vn{<value1>}.
If both are present than the scale is from \vn{<value1>} to \vn{<value2>}.

A graph can have a \vn{y2} (left) axis scale that is separate from the \vn{y} (right) axis.
Normally, the \vn{scale} command will scale both axes.  Scaling of just one of these axes can be
achieved by using the \vn{-y} or \vn{-y2} switches.

How a graph is scaled is determined in part by the setting of the \vn{bounds} parameter in the
\vn{y} and \vn{y2} components of the graph. See \vn{s:quick.plot} for more details. The \vn{-exact}
switch, if present, will set \vn{bounds} to \vn{"EXACT"} which means that \tao will use the min and
max bounds as given by \vn{<value1>} and \vn{<value2>} and not try to find ``nice'' values near the
given ones. If \vn{<value1>} and \vn{<value2>} are not given, and if \vn{bounds} is set to
\vn{"EXACT"}, \tao will set \vn{bounds} to \vn{"GENERAL"}. Note: To set the axis \vn{bounds}
directly, use the \vn{set graph} command.

For scaling \vn{floor_plan} plots where there is a building wall to be drawn, if \vn{-include_wall}
is present and autoscaling is being done, then the plot bounds are extended to include the extent of
the building wall.

Examples:
\begin{example}
  scale top.x -3  7  ! Scale the x graph in the top region
  scale -y2 top.x    ! Scale only the y2 axis of the top.x graph.
  scale bottom       ! Autoscale the graphs of the plot in the bottom region
  scale -include     ! Scale everything and include the extent of any 
                     !   building walls in the calculation of the plot bounds.
\end{example}


%% set --------------------------------------------------------------
\section{set}\index{commands!set}
\label{s:set}

The \vn{set} command is used to set values for data, variables, etc. Subcommands are:
\begin{example}
  set beam \{n@\}<parameter> = <value>                        ! \sref{s:set.beam}
  set beam_init \{n@\}<parameter> = <value>                   ! \sref{s:set.beam.init}
  set bmad_com <parameter> = <value>                        ! \sref{s:set.bmad.com}
  set branch <branch> <parameter> = <value>                 ! \sref{s:set.branch}
  set calculate <on/off>                                    ! \sref{s:set.calc}
  set curve <curve> <parameter> = <value>                   ! \sref{s:set.curve}
  set data <data_name>|<parameter> = <value>                ! \sref{s:set.data}
  set default <parameter> = <value>                         ! \sref{s:set.default}
  set dynamic_aperture \{n@\}<parameter = <value>             ! \sref{s:set.da}
  set element <element_list> <attribute> = <value>          ! \sref{s:set.element}
  set floor_plan <parameter> = <value>                      ! \sref{s:set.floor.plan}
  set geodesic_lm <parameter> = <value>                     ! \sref{s:set.geodesic.lm}
  set global <parameter> = <value>                          ! \sref{s:set.global}
  set graph <graph> <parameter> = <value>                   ! \sref{s:set.graph}
  set key <key> = <command>                                 ! \sref{s:set.key}
  set lat_layout <parameter> = <value>                      ! \sref{s:set.lat.layout}
  set lattice \{n@\}<destination_lat> = <source_lat>          ! \sref{s:set.lattice}
  set opti_de_param <parameter> = <value>                   ! \sref{s:set.opti.de.param}
  set particle_start \{n@\}<coordinate> = <value>             ! \sref{s:set.particle.start}
  set plot <plot> <parameter> = <value>                     ! \sref{s:set.plot}
  set plot_page <parameter> = <value1> \{<value2>\}           ! \sref{s:set.plot.page}
  set ran_state = <random_number_generator_state>           ! \sref{s:set.ran.state}
  set region <region> <parameter> = <value>                 ! \sref{s:set.region}
  set space_charge_com <parameter> = <value>                ! \sref{s:set.sc.com}
  set symbolic_number <name> = <value>                      ! \sref{s:set.symbolic}
  set universe <what_universe> <on/off>                     ! \sref{s:set.universe}
  set universe <what_universe> <calc_name> <on/off>         ! \sref{s:set.universe}
  set variable <var_name>|<parameter> = <value>             ! \sref{s:set.variable}
  set wave <parameter> = <value>                            ! \sref{s:set.wave}
\end{example}

\vskip 10pt 

When running \tao, to see documentation on any of the subcommands, use the \vn{help set
<subcommand>} command. For example, \vn{help set element} will show information on the \vn{set
element} subcommand.

Also see the \vn{change} command (\sref{s:change}). The \vn{change} command is specialized for
varying real parameters while the \vn{set} command is more general.

Note: The \vn{show} command (\sref{s:show}) is able to display the settings of many variables that
can be set by the \vn{set} command.

To apply a set to all data or variable classes use ``*'' in place of \vn{<data_name>} or \vn{var_name}.

To set the prompt color, use the command
\begin{example}
  set global prompt_color = <value>
\end{example}
Where \vn{<value>} may be one of:
\begin{example}
  'BLACK'
  'RED'
  'GREEN'
  'YELLOW'
  'BLUE'
  'MAGENTA'
  'CYAN'
  'GRAY'
  'DEFAULT'       ! Default foreground color
\end{example}

% Use the command:
%   help set <subcommand>
% to obtain more information on a particular set subcommand. Example:
%   help set plot

%% set beam --------------------------------------------------------------

\subsection{set beam}
\label{s:set.beam}

Format:
\begin{example}
  set beam \{n@\}<parameter> = <value>
  set beam \{n@\}beginning = <ele-name>
  set beam \{n@\}add_saved_at = <ele-list>
  set beam \{n@\}subtract_saved_at = <ele-list>
\end{example}

The \vn{set beam} command sets beam parameters such as the initial and final tracking positions.
Use the \vn{show beam} command (\sref{s:show}) to see the current values.

For the \vn{set beam beginning <ele-name>} command, the element specified by \vn{<ele-name>} must be
an element where particle positions of the tracked beam have been stored. With this command, the
initial distribution of the beam at the beginning of the lattice will be set to the distribution at
the indicated element. This is useful to track the beam over many turns.

The \vn{set beam \{n@\}add_saved_at} command adds to the list of elements where the beam
distribution is saved at.

The \vn{set beam \{n@\}subtract_saved_at} command subtracts from the list of elements where the beam
distribution is saved at.

The optional \vn{n@} allows the specification of the universe or universes the set is applied to.
The current default universe (\sref{s:universe}) will be used if no universe is given.

Also see the commands: \vn{set beam_init} and \vn{set particle_start}.

Examples:
\begin{example}
  set beam 2@track_start = q10w  ! Set the tracking start at element Q10W in universe 2.
  set beam saved_at = "Q*, B*"   ! Save beam parameters (sigma matrix, etc.) at elements
                                 !  whose names begin with "Q" or "B".
  set beam add_saved_at = S10    ! Save beam parameters at element "S10" as well.
  set beam beginning = end       ! Set the initial beam distribution equal to the distribution at
                                 !  the lattice element named "end".
\end{example}

%% set beam_init --------------------------------------------------------------

\subsection{set beam_init}
\label{s:set.beam.init}

Format:
\begin{example}
  set beam_init \{n@\}<parameter> = <value>
\end{example}

The \vn{set beam_init} command sets parameters of the \vn{beam_init} structure (\sref{s:beam.init}).
Additionally, the \vn{set beam_init} command can set the parameters (\sref{s:beam.init})
\begin{example}
  track_start  and
  track_end
\end{example}

The optional \vn{n@} allows the specification of the universe or universes the set is applied to.
The current default universe (\sref{s:universe}) will be used if no universe is given.

Use the \vn{show beam} command (\sref{s:show}) to see the current values.

Also see the commands: \vn{set beam} and \vn{set particle_start}.

Examples:
\begin{example}
  set beam_init 3@center(2) = 0.004   ! Set px center of beam for universe 3.
  set beam_init [1,2]@sig_e = 0.02    ! Set sig_e for universes 1 and 2.
  set beam_init track_end = q10w      ! Set track_end parameter.
\end{example}

%% set bmad_com --------------------------------------------------------------

\subsection{set bmad_com}
\label{s:set.bmad.com}

Format:
\begin{example}
  set bmad_com <parameter> = <value>
\end{example}

Sets global \bmad parameters. Use the \vn{show global -bmad_com} command to see a list of
\vn{<parameter>}s. See the \bmad manual for information on this structure.

Example:
\begin{example}
  set bmad_com radiation_fluctuations_on = T ! Turn on synchrotron radiation fluctuations.
\end{example}

%% set branch --------------------------------------------------------------

\subsection{set branch}
\label{s:set.branch}

Format:
\begin{example}
  set branch <branch-id> <parameter> = <value>
\end{example}

Sets parameters associated with a lattice branch. The parameters that can be set are:
\begin{example}
  particle                  = <species>   ! Reference particle
  default_tracking_species  = <species>   ! Particle that is tracked.
  geometry                  = open or closed
  live_branch               = T or F
\end{example}
Use the \vn{show branch} command to see lattice branch information. \vn{<branch-id>} may be the
branch index or branch name. \vn{<branch-id>} may also contain an optional \vn{n@} prefix to
specify a particular universe to apply the set to. The default is to only set the current viewed
universe.

Note: When toggling a branch from closed to open the beginning orbit and Twiss parameters will not
change. On the other hand, when toggling a branch from open to closed, the orbit and Twiss
parameters will, in general, shift.

Examples:
\begin{example}
  set branch 2@0 live_branch = F     ! Suppress calculations for branch \# 0 of universe 2.
  set branch a_line geometry = open  ! Open geometry for branch named a_line.
  set branch default_tracking_species = positron
                                     ! Set the tracking species to positron.
\end{example}

%% set calculate --------------------------------------------------------------

\subsection{set calculate}
\label{s:set.calc}

Format:
\begin{example}
  set calculate \{<on/off>\}
\end{example}

Toggles the following on (True) or off (False):
\begin{example}
  global%lattice_calc_on
  global%plot_on
\end{example}

Examples:
\begin{example}
  set calc on    ! Sets lattice calc and plot_on to True
  set calc off   ! Sets lattice calc and plot_on to False
  set call       ! Toggles lattice_calc and sets plot_on to
                 !  the same value as lattice_calc.
\end{example}

%% set curve --------------------------------------------------------------

\subsection{set curve}
\label{s:set.curve}

Format:
\begin{example}
  set curve <curve> <parameter> = <value>
\end{example}

For \vn{set curve}, the \vn{<parameter>}s that can be set are:
\begin{example}
  ele_ref_name        = <string>  ! Name or index of the reference element
  component           = <string>  ! \sref{s:curve.comp}
  ix_branch           = <number>  ! Branch index.
  ix_bunch            = <number>  ! Bunch index.
  ix_universe         = <number>  ! Universe index.
  symbol_every        = <number>  ! Symbol skip number.
  y_axis_scale_factor = <number>  ! Scaling of y axis
  draw_line           = <logical> 
  draw_symbols        = <logical> 
  draw_symbol_index   = <logical> 
\end{example}
See the \vn{Plot Templates} section (\sref{s:template}) for a description of these attributes.  Use
the \vn{show curve} (\sref{s:show}) to see the settings of the attributes.

If there are visible plots with the same name as the plot parameter of \vn{<curve>}, a template plot
of the same name is ignored. To set template plot curve(s) in this case, add a ``\vn{T::}'' prefix.

Examples:
\begin{example}
  set curve top.x.c1 ix_universe = 2       ! Set universe number for curve.
  set curve T::orbit.x.c1 ix_universe = 2  ! Set curve in template plot.
\end{example}

%% set data --------------------------------------------------------------

\subsection{set data}
\label{s:set.data}

Format:
\begin{example}
  set data <data_name>|<component> = <value>
\end{example}

For \vn{set data}, the \vn{<component>}s that can be set are:
\begin{example}
  base        ! Base model value
  design      ! Design model value
  meas        ! Measured data value.
  ref         ! Reference data value.
  weight      ! Weight for the merit function.
  exists      ! Valid datum for computations?
  good_meas   ! A valid measurement has been taken?
  good_ref    ! A valid reference measurement has been taken?
  good_opt    ! Good for using in the merit function for optimization?
  good_plot   ! Good for using in a plot?
  good_user   ! This is what is set by the use, veto, and restore commands.
  merit_type  ! How merit contribution is calculated.
\end{example}
Besides a numeric value \vn{<value>} can be any of the above along with:
\begin{example}
  meas        ! Measured data value.
\end{example}

Examples:
\begin{example}
  set data *|ref = *|meas            ! Set ref data = measured in current universe.
  set data 2@orbit.x|base = 2@orbit.x|model 
                                     ! Set the base orbit.x in universe 2 to model
  set data beta.x[10]|weight = 1e-5  ! Set weight of datum.
\end{example}

%% set default --------------------------------------------------------------

\subsection{set default}
\label{s:set.default}

Format:
\begin{example}
  set default <parameter> = <value>
\end{example}

The parameters that can be set are:
\begin{example}
  branch            ! See: Lattices section (\sref{s:lattice})
  universe          ! See: Universe section (\sref{s:universe})
\end{example}

Use the \vn{show global} (\sref{s:show}) command to see the current
default values.

Example:
\begin{example}
  set default universe = 3
\end{example}

%% set dynamic_aperture --------------------------------------------------------------

\subsection{set dynamic_aperture}
\label{s:set.da}

Format:
\begin{example}
  set dynamic_aperture \{n@\}<parameter> = <value>
\end{example}

The \vn{set dynamic_aperture} command sets parameters for dynamic aperture simulations
(\sref{s:da.calc}) Also see the \vn{set universe dynamic_aperture} (\sref{s:set.universe}) and
\vn{show dynamic_aperture} (\sref{s:show.da}).

To set the particle energy for the <n>\Th scan use \vn{pz(<n>)}. Use a value less than -1 to remove
the scan.

The optional \vn{n@} prefix allows the specification of the universe or universes the set is applied
to. The current default universe (\sref{s:universe}) will be used if no universe is given.

Examples:
\begin{example}
  set dy 2@n_angle = 20   ! Set number of scan points for universe 2.
  set dy accuracy = 1e-5  ! Set scan scan accuracy
  set dy pz(3) = -0.05    ! Set particle energy for the 3rd scan.
\end{example}

%% set element --------------------------------------------------------------

\subsection{set element}
\label{s:set.element}

Format:
\begin{example}
  set \{-update\} \{-lord_no_set\} element <element_list> <attribute> = <value>
\end{example}

The \vn{set element} command sets the attributes of an element. Use the \vn{show element}
command to view the attributes of an element.

The \vn{-lord_no_set} switch, if present, will prevent the set of the corresponding attribute in a
\vn{super_lord} or \vn{multipass_lord} of a slave element that appears in \vn{element_list}. For
example:
\begin{example}
  set ele -lord quad::A:B field_calc = True
\end{example}
In this example the element list is all quadrupole in the region between elements \vn{A} and \vn{B}
in the lattice. The presence of the \vn{-lord_no_set} switch means that any \vn{super_slave} or
\vn{multipass_slave} quadrupole in that region will be ignored.

The \vn{-update} switch, if present, suppresses \tao from printing error messages if a ``variable
slave value mismatch'' is detected (\sref{s:var.mismatch}). Independent of whether \vn{-update} is
present or not, \tao will fix the mismatch using the changed value to set all of the slave values.

Note: \vn{set element} can be used to set \vn{ramper} type elements.

Note: If an element in the \vn{<element_list>} does not specify a universe (or universes),
only the element in the viewed universe is used. See the examples below.

Note: It is also possible to use the \vn{change element} command to change real (as opposed to
logical or integer) attributes.

Examples:
\begin{example}
  set ele rfcav::* is_on = F        ! Turn off all rfcavity elements the viewed universe.
  set ele *@rfcav::* is_on = F      ! Turn off all rfcavity elements in all universes.
  set ele A:B track_method = linear ! Set tracking_method for all elements between 
                                    !   elements A and B
  set ele q10w k1 = 0.7             ! Set element q10w k1 of the viewed universe.
\end{example}

%% set floor_plan --------------------------------------------------------------

\subsection{set floor_plan}
\label{s:set.floor.plan}

Format:
\begin{example}
  set floor_plan <parameter> = <value>
\end{example}


Sets parameters for \vn{floor_plan} plots (\sref{s:shapes}).  Possible \vn{<parameters>} are:
\begin{example}
  <shape_name>%<shape_parameter>
  draw_beam_chamber_wall
  beam_chamber_wall_scale
\end{example}
Where \vn{<ele_shape_name>} is of the form ``\vn{ele_shape(<n>)}'' where \vn{<n>} is the index of
the \vn{ele_shape} in the \vn{floor_plan_drawing} namelist.  Use ``\vn{show plot -floor_plan}'' to
see the current state of the \vn{floor_plan} parameters

Example:
\begin{example}
  set floor_plan ele_shape(2)%draw = F  ! Veto drawing of ele_shape(2)
  set floor_plan beam_chamber_scale = 0.5
\end{example}

%% set geodesic_lm --------------------------------------------------------------

\subsection{set geodesic_lm}
\label{s:set.geodesic.lm}

Format:
\begin{example}
  set geodesic_lm <parameter> = <value>
\end{example}

For \vn{set geodesic_lm}: The \vn{show optimizer geodesic_lm} command will give a list of
\vn{<parameter>}s.

Example:
\begin{example}
  set geodesic_lm method = 10
\end{example}

%% set global --------------------------------------------------------------

\subsection{set global}
\label{s:set.global}

Format:
\begin{example}
  set global <parameter> = <value>
\end{example}

The \vn{set global} command sets global parameters of \tao. The \vn{show global} command will give
a list of global parameters.

Example:
\begin{example}
  set global n_opti_loops = 30  ! Set number of optimization cycles
  set global rf_on = T          ! Turn on the RF cavities.
\end{example}

%% set graph --------------------------------------------------------------

\subsection{set graph}
\label{s:set.graph}

Format:
\begin{example}
  set graph <graph> <parameter> = <value>
\end{example}

The \vn{set graph} command is used to set parameters of a graph structure (\sref{s:template}).

If the \vn{<graph>} name corresponds to a plot, the set is applied to all the graphs associated with
the plot. If there are visible plots with the same name as the plot parameter of \vn{<graph>}, a
template plot of the same name is ignored. To set template plot graphs(s) in this case, add a
``\vn{T::}'' prefix.

For setting the \vn{parameter} attribute see also the commands:
\begin{example}
  set plot parameter      ! \sref{s:set.plot}
  set curve parameter     ! \sref{s:set.curve}
\end{example}

Example:
\begin{example}
  set graph orbit.x component = model - design  ! Plot orbit (model - design).
  set graph orbit component = model - design    ! Applies to all graphs of orbit plot.
  set graph T::orbit.x component = design       ! Set template plot
  set graph r11 floor_plan%orbit_scale = 100    ! To display an orbit.
  set graph beta y%bounds = "zero_at_end"       ! \sref{s:quick.plot}.
\end{example}

%% set key --------------------------------------------------------------

\subsection{set key}
\label{s:set.key}

Format:
\begin{example}
  set key <key> = <command>
\end{example}

Binds a custom command to a key for use in single mode (\sref{c:single}).  This will override the
default behavior (if there is one) of the key.  The command \vn{default} will reset the key to its
default usage.

Example:
\begin{example}
  set key h = veto var *
  set key j = default
\end{example}


%% set lat_layout --------------------------------------------------------------

\subsection{set lat_layout}
\label{s:set.lat.layout}

Format:
\begin{example}
  set lat_layout <parameter> = <value>
\end{example}

Sets parameters for \vn{lat_layout} plots (\sref{s:shapes}).  Syntax for ``\vn{set lat_layout}'' is
identical to syntax of ``\vn{set floor_plan}''.  See ``\vn{set floor_plan}'' for more details.

Use ``\vn{show plot -lat_layout}'' to see a listing of all shapes. 

Example:
\begin{example}
  set lat_layout ele_shape(2)%draw = F  ! Veto drawing of shape \#2
\end{example}

%% set lattice --------------------------------------------------------------

\subsection{set lattice}
\label{s:set.lattice}

Format:
\begin{example}
  set lattice \{n@\}<destination_lat> = <source_lat>
\end{example}

The \vn{set lattice} command transfers lattice parameters (element strengths, etc., etc.)  from one
lattice (the \vn{source} lattice) to another (the \vn{destination} lattice). Both lattices are
restricted to be from the same universe. The optional \vn{n@} prefix (\sref{s:universe}) of the
destination lattice can be used to specify which universe the lattices are in. If multiple universes
are specified, the corresponding destination lattice will be set to the corresponding source lattice
in each universe. Note: At this time, it is not permitted to transfer parameters between lattices in
different universes.

The destination lattices that can be set are:
\begin{example}
  model      ! Model lattice.
  base       ! Base lattice
\end{example}
The source lattice can be:
\begin{example}
  model       ! model lattice.
  base        ! base lattice.
  design      ! design lattice
\end{example}

Note: \tao variables that control parameters in multiple universes can complicate things. If, for
example, there are two universes, and a \tao variable controls, say, the quadrupole strength of
quadrupoles in both universes, then a ``set lat 2@model = design'' will result in the quadrupole
strengths of those quadrupoles controlled by the variable in universe 1 being changed.

Example:
\begin{example}
  set lattice *@model = design  ! Set the model lattice to the design in 
                                !   all universes.
  set lattice base = model      ! Set the base lattice to the model lattice in 
                                !   the default universe.
\end{example}

%% set opti_de_param --------------------------------------------------------------

\subsection{set opti_de_param}
\label{s:set.opti.de.param}

Format:
\begin{example}
  set opti_de_param <parameter> = <value>
\end{example}

For \vn{set opti_de_param}: The \vn{show global} command will give a list of \vn{<parameter>}s.

Example:
\begin{example}
  set opti_de_param binomial_cross = T  ! Use binomial crossovers 
\end{example}

%% set particle_start --------------------------------------------------------------

\subsection{set particle_start}
\label{s:set.particle.start}

Format:
\begin{example}
  set particle_start \{n@\}<coordinate> = <value>
\end{example}
The \vn{set particle_start} command sets the starting coordinates for single particle tracking for
lattices with an open geometry. If the \vn{use_particle_start} of the \vn{beam_init}
structure (\sref{s:beam.init}) is set to True, \vn{particle_start} will also vary the beam centroid
and beam particle spin for beam tracking.

The optional \vn{n@} universe specification (\sref{s:universe}) may be used to specify the universe
or universes to apply the set command to.

\vn{<coordinate>} is one of:
\begin{example}
  x, px, y, py, z, pz, t
  spin_x, spin_y, spin_z
\end{example}
For photons, \vn{<coordinate>} may also be:
\begin{example}
  field_x, field_y, phase_x, phase_y
\end{example}
The \vn{*} coordinate denotes the phase space vector $(x, p_x, y, p_y, z, p_z)$.  For closed
lattices only the \vn{pz} parameter is applicable. For lattices that have an \vn{e_gun} (which
necessarily implies that the lattice has an open geometry), the time \vn{t} coordinate must be
varied instead of \vn{pz}.

To see the values for \vn{particle_start} use the command \vn{show element 0}.

Also see the commands: \vn{set beam} (\sref{s:set.beam}), \vn{set beam_init} (\sref{s:set.beam.init}), 
and \vn{change particle_start} (\sref{s:change}).

Examples:
\begin{example}
  set particle_start 2@x = 0.001         ! Set beginning x position in universe 2 to 1 mm.
  set particle_start field_x = 1         ! Set photon field
  set particle_start spin_y = 0.37       ! Set spin parameter.
\end{example}

%% set plot --------------------------------------------------------------

\subsection{set plot}
\label{s:set.plot}

The \vn{set plot} command set various parameters of a plot. Format:
\begin{example}
  set plot <plot_or_region> <parameter> = <value>
\end{example}

The \vn{<parameters>}s that can be set are:
\begin{example}
  autoscale_x        = <logical>
  autoscale_y        = <logical>
  visible            = <logical>
  component          = <string>    ! Sets curve component \sref{s:curve.comp}
  x%<axis_parameter> = <value>
  n_curve_pts        = <integer>
\end{example}
Use the \vn{show plot <plot_name>} to see the settings of various parameters. See the \vn{Plot
Templates} section (\sref{s:template}) for information on the plotting parameters.

The \vn{visible} parameter hides a plot but keeps the plot associated with the associate region. If
the plot window is not enabled (\vn{-noplot} option used at startup), the \vn{visible} parameter is
used by \tao to decide whether to calculate the points needed for plotting curves (saves time if the
computation is not needed). This is relevant when \tao is interfaced to a \vn{GUI}
(\sref{s:gui.plot}).

The \vn{n_curve_pts} parameters sets the number of points to use for drawing ``smooth'' curves. This
overrides the setting of \vn{plot_page%n_plot_pts} (\sref{s:init.plot}). Warning: \tao will cache
intermediate calculations used to compute a smooth curve to use in the computation of other smooth
curves. \tao will only do this for curves that have \vn{plot_page%n_curve_pts} number of
points. Depending upon the circumstances, setting \vn{plot%n_curve_pts} for individual plots may
slow down plotting calculations significantly.

Note: If the \vn{component} parameter is set, the \vn{<value>} is stored in each of the curves of
the plot since the \vn{component} attribute is associated with individual curves and not the plot as
a whole.

If \vn{<plot_or_region>} is a plot name, and there are visible plots of that name, any template plot
of the same name is ignored. To set a template plot in this case, add a ``\vn{T::}'' prefix.

Example:
\begin{example}
  set plot orbit visible = F           ! Hide orbit plot
  set plot beta component = design     ! Plot the design value.
  set plot T::beta component = design  ! Set the template plot instead of any displayed plots.
  set plot x%draw_label = False        ! Do not draw x-axis label.
\end{example}

%% set plot_page --------------------------------------------------------------

\subsection{set plot\_page}
\label{s:set.plot.page}

Format:
\begin{example}
  set plot_page <parameter> = <value1> \{<value2>\}
\end{example}

The \vn{set plot_page} command sets \vn{plot page} parameters (\sref{s:plot.page.def}).
use the \vn{show plot -page} command to see a list of plot page parameters.

Example:
\begin{example}
  set plot_page title = 'XYZ'  ! Set plot page title string
\end{example}

%% set ran_state --------------------------------------------------------------

\subsection{set ran\_state}
\label{s:set.ran.state}

Format:
\begin{example}
  set ran_state = <random_number_generator_state>
\end{example}

Sets the state of the random number generator to a specific state. Use \vn{show global -ran_state}
to show the random number generator state. Manipulating the state for generating random numbers is
generally only used for debugging purposes and is not of interest to the typical user.

%% set region --------------------------------------------------------------

\subsection{set region}
\label{s:set.region}

Format:
\begin{example}
  set region <parameter> = <value>
\end{example}

Sets a plot region parameter. Parameters are:
\begin{example}
  x1, x2, y1, y2    ! Region rectangle placement
  visible           ! Is plot in region visible?
\end{example}

Use the \vn{show plot} command to see a listing of region parameters.

Example:
\begin{example}
  set region r13 y2 = 0.3  ! Set y2 parameter of region r13
\end{example}

%% set space_charge_com --------------------------------------------------------------

\subsection{set space_charge_com}
\label{s:set.sc.com}

Format:
\begin{example}
  set space_charge_com <parameter> = <value>
\end{example}

Sets global space charge (including CSR) parameters. Use the \vn{show global -space_charge_com}
command to see a list of \vn{<parameter>}s. See the \bmad manual for information on this structure.

Example:
\begin{example}
  set space_charge_com n_bin = 30  ! Set number of bins used in the csr calc.
\end{example}

%% set symbolic_number --------------------------------------------------------------

\subsection{set symbolic_number}
\label{s:set.symbolic}

Format:
\begin{example}
  set symbolic_number <name> = <value>
\end{example}

Create a symbolic number that can be used in expressions. Use the \vn{show symbolic_number} command
to show a list of symbols that have been defined. Repeated \vn{set} commands may be used to modify
the value of a symbol if desired.

Example:
\begin{example}
  set sym aa = 23.4 * pi  ! Define the symbol "aa"
\end{example}

%% set universe --------------------------------------------------------------

\subsection{set universe}
\label{s:set.universe}

Format:
\begin{example}
  set universe <what_universe> <on/off>
  set universe <what_universe> recalculate
  set universe <what_universe> twiss_calc <on/off>
  set universe <what_universe> dynamic_aperture_calc <on/off>
  set universe <what_universe> one_turn_map_calc <on/off>
  set universe <what_universe> track_calc <on/off>
\end{example}

The \vn{<what_universe>} argument may be a list of universes enclosed in brackets ``[...]''. See
below for an example.

The \vn{set universe <what_universe> ...} command will turn on or off specified lattice/tracking
calculations for the specified universe(s). Turning specified calculations off for a universe is
useful to speed up lattice calculations when the calculation is not necessary. To specify the
currently default universe (\sref{s:universe}), you can use \vn{-1} as the \vn{<what_universe>}
index. To specify all universes, use \vn{*}. Use the \vn{show universe} command to see the state of
these switches are.

Note: The global logical \vn{lattice_calc_on} (\sref{s:globals}) is separate from the logicals set
by \vn{set universe}. That is, toggling the state of \vn{lattice_calc_on} will not affect the
settings of the logicals set by \vn{set universe}. If \vn{lattice_calc_on} is set to \vn{False} then
no calculations are done in any universe independent of the settings of the \vn{set universe}
logicals. That is, \vn{lattice_calc_on} acts as a master toggle that can be used to turn off all
lattice/tracking calculations.

If optimizing while one or more universes are turned off, the variables associated with that
universe will still be included in the merit function but not the data for that universe. The
variables will still vary in the turned off universe.

The \vn{set universe <what_universe> recalculate} command will recalculate the lattice parameters
for that universe.

The \vn{set universe <what_universe> dynamic_aperture_calc} command will enable the dynamic aperture
calculation for a ring. See the \vn{Initializing Dynamic Aperture} section (\sref{s:da.calc}) for
more details. To enable the dynamic aperture calculation at startup, set the
\vn{design_lattice(i)%dynamic_aperture} parameter (\sref{s:init.lat}).

The \vn{set universe <what_universe> one_turn_map_calc} command will enable a one-turn-map
calculation for a ring using PTC, and populate the normal form taylor maps. See
Eq.~\ref{normalform1} and Eq.~\ref{normalform2} in the \vn{normal.} data type. To enable the map
calculation at startup, set the \vn{design_lattice(i)%one_turn_map_calc} parameter
(\sref{s:init.lat}).

The commands
\begin{example}
  set universe <what_universe> twiss_calc  and
  set universe <what_universe> track_calc
\end{example}
will set whether the 6x6 transfer matrices and the central orbit (closed orbit for circular rings)
is calculated for a given universe. Turning this off is useful in speeding up calculations in the
case where the transfer matrices and/or orbit is not being used.

Example:
\begin{example}
  set universe [1,3] off ! Set universes 1 and 3 off.
  set universe -1 on     ! Set on currently default universe.
  set universe * recalc  ! Recalculate in all universes.
\end{example}

%% set variable --------------------------------------------------------------

\subsection{set variable}
\label{s:set.variable}

Format:
\begin{example}
  set variable <var_name>|<parameter> = <value>
\end{example}

For \vn{set var}, the \vn{<parameter>}s that can be set are:
\begin{example}
  model       ! Model lattice value.
  base        ! Base model value
  design      ! Design model value
  meas        ! Value at the time of a measurement.
  ref         ! Value at the time of a reference measurement.
  weight      ! Weight for the merit function.
  exists      ! Does this variable actually correspond to something?
  good_var    ! The optimizer can be allowed to vary it
  good_opt    ! Good for using in the merit function for optimization?
  good_plot   ! Good for using in a plot?
  good_user   ! This is what is set by the use, veto, and restore commands.
  step        ! Sets what a "small" variation of the variable is.
  merit_type  ! How merit contribution is calculated.
  key_bound   ! Model value can be modified using keyboard?
  key_delta   ! Change in model value when key is pressed.
\end{example}

Example:
\begin{example}
  set var quad_k1|weight = 0.1         ! Set quad_k1 weights. 
\end{example}

%% set wave --------------------------------------------------------------

\subsection{set wave}
\label{s:set.wave}

Format:
\begin{example}
  set wave <parameter> = <value>
\end{example}

The \vn{set wave} command sets the boundaries of the $A$ and $B$ regions for the wave analysis
(\sref{c:wave}). The parameters are
\begin{example}
  ix_a = <ix_a1> <ix_a2>  ! A-region left and right boundaries.
  ix_b = <ix_b1> <ix_b2>  ! B-region left and right boundaries.
\end{example}

Example:
\begin{example}
  set wave ix_a = 15 27    ! Set A-region to span from datum #15 to #27
\end{example}

Note: Use the \vn{wave} command (\sref{s:wave}) first to setup the display of the wave analysis.

%% show --------------------------------------------------------------

\section{show}\index{commands!show}
\label{s:show}

The \vn{show} command is used to display information.
Format:
\begin{example}
  show \{-append <file_name>\} \{-noprint\} \{-no_err_out\} <subcommand>
  show \{-write <file_name>\} \{-noprint\} \{-no_err_out\} <subcommand>
\end{example}

\vn{<subcommand>} subcommands may be one of:
\begin{example}
  show alias                   ! Show aliases \sref{s:show.alias}.
  show beam ...                ! Show beam info \sref{s:show.beam}.
  show branch ...              ! Show lattice branch info \sref{s:show.branch}.
  show building_wall           ! Show building wall info \sref{s:show.building}.
  show chromaticity ...        ! Show chromaticity, momentum compaction, phase slip \sref{s:show.chrom}.
  show constraints             ! Show optimization constraints \sref{s:show.constraints}.
  show control ...             ! Show lords and slaves of a given element \sref{s:show.control}.
  show curve ...               ! Show plot curve info \sref{s:show.curve}.
  show data ...                ! Show optimization data info \sref{s:show.data}.
  show derivative ...          ! Show d_data/d_var optimization info \sref{s:show.derivative}.
  show dynamic_aperture        ! Show DA info \sref{s:show.da}.
  show element ...             ! Show lattice element info \sref{s:show.element}.
  show emittance               ! Show normal mode emittances \sref{s:show.emit}.
  show field ...               ! Show EM field \sref{s:show.field}.
  show global ...              ! Show Tao global parameters \sref{s:show.global}.
  show graph ...               ! Show plot graph info \sref{s:show.graph}.
  show history ...             ! Show command history \sref{s:show.history}.
  show hom                     ! Show Higher Order Mode info \sref{s:show.hom}.
  show internal ...            ! Used for code debugging \sref{s:show.internal}.
  show key_bindings            ! Show single mode key bindings \sref{s:show.key}.
  show lattice ...             ! Lattice element-by-element table \sref{s:show.lattice}.
  show matrix ...              ! Show transport matrix \sref{s:show.matrix}.
  show merit ...               ! Show optimization merit function \sref{s:show.merit}.
  show optimizer ...           ! Show optimizer info \sref{s:show.optimizer}.
  show particle ...            ! Show tracked particle beam info \sref{s:show.particle}.
  show plot ...                ! Show plot info \sref{s:show.plot}.
  show ptc ...                 ! Show PTC calculated parameters \sref{s:show.ptc}.
  show radiation_integrals ... ! Show synchrotron radiation integrals \sref{s:show.rad.int}.
  show spin ...                ! Show information on spin simulations \sref{s:show.spin}.
  show string ...              ! Print a string \sref{s:show.string}.
  show symbolic_numbers ...    ! Show symbolic constants \sref{s:show.symbolic}.
  show taylor_map ...          ! Show transport Taylor map\sref{s:show.taylor}.
  show track ...               ! Show phase space coords, Twiss, EM field, 
                               !   and other info along the tracked orbit \sref{s:show.track}.
  show twiss_and_orbit ...     ! Show Twiss and orbit info at given position including
                               !   synchrotron radiation related parameters \sref{s:show.twiss}.
  show universe ...            ! Show universe info \sref{s:show.universe}.
  show use                     ! Show data and vars used in optimization \sref{s:show.use}.
  show value ...               ! Show value of an expression \sref{s:show.value}.
  show variables ...           ! Show optimization variable info \sref{s:show.variables}.
  show version                 ! Show Tao version.
  show wakes                   ! Show wake info \sref{s:show.wakes}.
  show wall ...                ! Show vacuum chamber wall info \sref{s:show.wall}.
  show wave                    ! Show wave analysis info \sref{s:show.wave}.
\end{example}

\vskip 10pt 

When running \tao, to see documentation on any of the subcommands, use the \vn{help show
<subcommand>} command. For example, \vn{help show element} will show information on the \vn{show
element} subcommand.

The \vn{show} command has \vn{-append} and \vn{-write} optional arguments which can be used to write
the results to a file.  The \vn{show -append} command will appended to the output file. The \vn{show
-write} command will first erase the contents of the output file. If \vn{global%write_file} has a
\vn{*} character in it, a three digit number is substituted for the \vn{*}. The value of the number
starts at \vn{001} and increases by 1 each time \vn{show -write} is used.  Example:
\begin{example}
  show -write floor.dat lat -floor  ! Write floor positions to the file "floor.dat".
\end{example}

The \vn{-noprint} option suppresses printing and is useful when writing large amounts of data to a
file.

When writing to a file, if there are any error messages (for example, that something could not be
computed), the error messages are reproduced in the file. If this behavior is not wanted, the
\vn{-no_err_out} switch may be used to block the error messages being written.

The \vn{-append}, \vn{-write}, \vn{-noprint}, and \vn{-no_err_out} switches must be placed before
\vn{<subcommand>}.

Note: When running \tao as a subprocess, use the \vn{python} command (\sref{s:python})
instead of the \vn{show} command for communicating with the parent process.

% Use the command:
%   help show <subcommand>
% to obtain more information on a particular show subcommand. Example:
%   help show plot

%% show alias --------------------------------------------------------------

\subsection{show alias}
\label{s:show.alias}

Syntax:
\begin{example}
  show alias
\end{example}

Shows a list of defined aliases. See the \vn{alias} command for more details.

%% show beam --------------------------------------------------------------

\subsection{show beam}
\label{s:show.beam}

Syntax:
\begin{example}
  show beam \{-universe <uni_index>\} \{-lattice\} \{<element_id>\}
\end{example}

If both \vn{<element_id>} and \vn{-lattice} are absent, \vn{show beam} shows parameters
used with beam tracking including the number of particles in a bunch, etc.

If no universe is given, the current default universe (\sref{s:universe}) is used.

If \vn{<element_id>} is present, and \vn{-lattice} is not, \vn{show beam} will show beam
parameters at the selected element. Also see \vn{show particle}.  Use the \vn{set beam_init} command
to set values of the \vn{beam_init} structure.

If \vn{-lattice} is present, \vn{show beam} will show the beam sigma matrix as calculated from the
lattice at the position given by \vn{<element_id>} (\sref{s:lat.sig.init}). If an
element is not specified, the beginning element (with index 0) will be used.

Examples:
\begin{example}
  show beam          ! Show beam initialization parameters.
  show beam -lat 37  ! Show sigma matrix calculated from lattice Twiss at element #37
\end{example}

%% show branch --------------------------------------------------------------

\subsection{show branch}
\label{s:show.branch}

Syntax:
\begin{example}
  show branch \{-universe <uni_index>\}
\end{example}

Lists the lattice branches of the lattice associated with the given universe along with information
on the fork elements connecting the branches.  If no universe is given, the current default universe
(\sref{s:universe}) is used.

Example:
\begin{example}
  show branch -u 2     ! Show info on lattice branches associated with universe 2
\end{example}

%% show building_wall --------------------------------------------------------------

\subsection{show building_wall}
\label{s:show.building}

Syntax:
\begin{example}
  show building_wall
\end{example}

List all building wall (\sref{s:building.wall}) sections along with the points that define
the sections.

For vacuum_chamber, capillary, and diffraction_plate walls use the ``show wall'' command.

%% show chromaticity --------------------------------------------------------------

\subsection{show chromaticity}
\label{s:show.chrom}

Syntax:
\begin{example}
  show chromaticity \{-taylor\} \{-universe <uni_index>\} 
\end{example}

Shows chromaticity and derivatives as calculated from PTC normal form analysis. Also shown
is momentum compaction and phase slip and derivatives.

If no universe is given, the current default universe (\sref{s:universe}) is used.

The \vn{-taylor} switch will show the Taylor series for the three normal mode tunes and spin tune
as functions of the phase space coordinates. The computation uses complex series. The imaginary part
should be zero (or very small). The spin Taylor series is only computed when spin tracking is on.


%% show constraints --------------------------------------------------------------

\subsection{show constraints}
\label{s:show.constraints}

Syntax:
\begin{example}
  show constraints
\end{example}

Lists data and variable constraints. Also see \vn{show merit}.

%% show constraints --------------------------------------------------------------

\subsection{show control}
\label{s:show.control}

Syntax:
\begin{example}
  show control {element-name-or-index}
\end{example}

This command compiles a list of all lords (and lords of lords, etc.) of the given element as well as
a list of all slaves (and slaves of slaves, etc.) of the given element. Then for each element in the
lists, the lords and slaves of that element are displayed. Example:
\begin{example}
  show control q1#2   ! Show lords/slaves of second instance of element named q1.
\end{example}

%% show curve --------------------------------------------------------------

\subsection{show curve}
\label{s:show.curve}

Syntax:
\begin{example}
  show curve \{-line\} \{-no_header\} \{-symbol\} <curve_name>
\end{example}

Show information on a particular curve of a particular plot. See \sref{c:plotting} for the syntax on
plot, graph, and curve names.  Use \vn{show plot} to get a list of plot names. The \vn{-symbol}
switch will additionally print the (x,y) points for the symbol placement and the \vn{-line} switch
will print the (x,y) points used to draw the ``smooth'' curve in between the symbols. The line or
symbol points from multiple curves can be printed by specifying multiple curves. Example:
\begin{example}
  show curve -sym orbit
\end{example}
This will produce a three column table assuming that the orbit plot has curves \vn{orbit.x.c1} and
\vn{orbit.y.c1}. When specifying multiple curves, each curve must have the same number of data
points and it will be assumed that the horizontal data values are the same for all curves so the
horizontal data values will be put in column 1.

The \vn{-no_header} switch is used with \vn{-line} and \vn{-symbol} to suppress the printing of
header lines. This is useful when the generated table is to be read in by another program.

If there are visible plots with the same name as the plot parameter of \vn{<curve>}, a template plot
of the same name is ignored. To show template plot curve(s) in this case, add a ``\vn{T::}'' prefix.

Also see: \vn{show plot} and \vn{show graph} commands.

Example:
\begin{example}
  show curve r2.g1.c3     ! Show the attributes of a curve named "c3" which is 
                          !   in the graph "g1" which is plotted in region "r2".
\end{example}

%% show data --------------------------------------------------------------

\subsection{show data}
\label{s:show.data}

Syntax:
\begin{example}
  show data \{<data_name>\}
\end{example}

Shows data information. If \vn{<data_name>} is not present then a list of all \vn{d2_data} names is
printed.

Examples:
\begin{example}
  show data                   ! Lists d2_data for all universes
  show data *@*               ! Same as above
  show data -1@*              ! Lists d2_data for the currently default universe.
  show data *                 ! Same as above.
  show data 2@*               ! Shows d2_data in universe 2.
  show data orbit             ! Show orbit data.
  show data orbit.x           ! list all orbit.x data elements.
  show data orbit.x[35]       ! Show details for orbit.x element 35
  show data orbit.x[35,86:95] ! list orbit.x elements 35 and 86 through 95
  show data orbit.x[1:99:5]   ! list every fifth orbit.x between 1 and 99  
\end{example}

%% show derivative --------------------------------------------------------------

\subsection{show derivative}
\label{s:show.derivative}

Syntax:
\begin{example}
  show derivative \{-derivative_recalc\} \{<data_name(s)>\} \{<var_name(s)>\}
\end{example}

\index{lm}\index{svd}
Shows the derivative dData_Model_Value/dVariable. This derivative is used by the optimizers \vn{lm}
and \vn{svd}. Note: Wild card characters can be used to show multiple derivatives.  Default values
for \vn{<data_name(s)>} and \vn{<var_name(s)>} is "*" (all data or variables).

The \vn{-derivative_recalc} forces a recalculation of the derivative matrix. This is exactly the
same as using \vn{derivative} command (\sref{s:deriv}) before the \vn{show derivative} command.

Note: Derivatives are only calculated for data and variables that are used in an optimization. That
is, derivatives are only calculated for data and variables whose \vn{useit_opt} parameter (see
\sref{s:data.anatomy} and \sref{c:var}) is True.

The output of this command is a number of lines that look like:
\begin{example}
  Data                Variable               Derivative   ix_dat  ix_var
  k.22a[98]           v_steer[92]           -7.63151E+01    1584     214
  k.22a[98]           v_steer[93]           -1.81810E+00    1584     215
\end{example}
The first and second columns are the datum and variable names, the third column is the derivative,
and the last two columns are the indexes of where the derivative is stored in \tao's internal
derivative matrix. These last two columns are for debugging purposes and can be ignored.

Example:
\begin{example}
  show deriv orbit.x[2] k1[3] ! Show dModel_Value/dVariable Derivative.
  show deriv                  ! Show all derivatives. Warning! The output may be large.

\end{example}

%% show dynamic_aperture --------------------------------------------------------------

\subsection{show dynamic_aperture}
\label{s:show.da}
\index{dynamic_aperture}

Syntax:
\begin{example}
  show dynamic_aperture
\end{example}

Shows parameters and results of the dynamic aperture calculation (\sref{s:da.calc}).
See also the commands \vn{set dynamic_aperture}, and \vn{set universe dynamic_aperture}.

%% show element --------------------------------------------------------------

\subsection{show element}
\label{s:show.element}

Syntax:
\begin{example}
  show element \{-attributes\} \{-base\} \{-data\} \{-design\} \{-all\} \{-field\}
      \{-floor_coords\} \{-no_slaves\} \{-no_super_slaves\} \{-ptc\} \{-taylor\} \{-wall\} 
      \{-xfer_mat\} <ele_name>
\end{example}

This shows information on lattice elements. The syntax for \vn{<ele_name>} is explained in section
\sref{s:ele.list.format}. If \vn{<ele_name>} contains a wild card or a class name then a list of
elements that match the name are shown. If no wild--card or class name is present then information
about the element whose name matches \vn{<ele_name>} is shown.

If the \vn{-ptc} switch is used, then the associated PTC fibre information will be displayed. If
there is not associated PTC fibre (which will be true if PTC has not been used for tracking with
this element), an associated PTC fibre will be created. In this case, only the PTC information will
be displayed and the other switches will be ignored.

If the \vn{-attributes} switch is present, then all of the element ``attributes'' will be
displayed. The default is is to display only those attributes with non-zero values. ``Attributes''
here does not include such things as the cross-section, Taylor map and wiggler element parameters.

By default, the appropriate element(s) within the \vn{model} lattice (\sref{s:universe}) are
used. This can be overridden by using the \vn{-base} or the \vn{-design} switches which switch the
lattice to the \vn{base} or \vn{design} lattices respectively.

If the \vn{-wall} switch is present, the wall information for the element, if it has been defined in
the lattice file, is displayed. For an x-ray \vn{capillary} element, the wall is the inner surface
of the capillary. For all other elements, the wall is the beam chamber wall.

If the \vn{-data} switch is present, information about the all the datums associated with the
element will be listed.

If the \vn{-floor_coords} switch is present, the global floor coordinates at the exit end of the
element will be printed. See the \bmad manual for an explanation of the floor coordinates.

When using wild cards in the element name, if the \vn{-no_super_slaves} switch is present then
\vn{super_slave} elements will not be included in the output. If the \vn{-no_slaves} switch is
present, both \vn{super_slave} and \vn{multipass_slave} elements will be ignored.

If the \vn{-taylor} switch is present, the Taylor map associated with an element, if there is one,
is also displayed. An element will have an associated Taylor map if tracking or transfer matrix
calculations for the element call for one. For example, if an elements \vn{tracking_method} is set
to \vn{Taylor}, it will have an associated Taylor map. To see the Taylor map for an element that
does not have an associated map, use the \vn{show taylor_map} command.

If the \vn{-field} switch is present, any associated Electro-magnetic field maps or grid data is
printed. For example, wiggler terms for a \vn{map_type} \vn{wiggler} element are printed.

If the \vn{-xfer_mat} switch is present, the 6x6 transfer matrix (the first order part of the
transfer map) along with the zeroth order part of the transfer map are printed.

The \vn{-all} switch is equivalent to using:
\begin{example}
  -attributes
  -floor_coords
  -taylor
  -wall
  -xfer_mat
\end{example}
If the element has a field map, the \vn{-all} switch will print map parameters (such as the spacing between points) but
not the entire field table itself. To print the field table as well, use the \vn{-field} switch.

Example:
\begin{example}
  show ele quad::z* -no_slaves  ! list all non-slave quadrupole elements with 
                                !   names beginning with "z".
  show ele q10w                 ! Show a particular lattice element.
  show ele -att 105             ! Show element #105 in the lattice.
\end{example}

%% show emittance --------------------------------------------------------------

\subsection{emittance}
\label{s:show.emit}

Syntax:
\begin{example}
  show emittance \{-element <ele_id>\} \{-sigma_matrix\} \{-universe <uni_index>\} \{-xmatrix\} 
\end{example}

The \vn{show emittance} command shows, for a given lattice branch, the three normal mode emittances
as calculated by PTC, a full non-PTC based 6D calculation, and via radiation integrals evaluation
(also non-PTC).

The \vn{-element} switch is used to select what lattice element is used as the end points of the
one-turn integrals that are used in the calculation. If the \vn{-element} switch is not present, the
beginning element of the default branch (set by \vn{set default branch}) is used. With radiation
damping, the emittance is not an exact invariant of the motion. Thus the calculated emittance will
vary depending upon what lattice element is used.

If the \vn{-sigma_matrix} switch is present, the sigma matrix at the given element will be displayed
along with the emittances.

If the \vn{-xmatrix} switch is present, instead of showing emittances, the damping and stochastic
kick transfer matrices used for particle tracking are displayed for the element given by the
\vn{-element} switch.

Examples:
\begin{example}
  show emit -ele 1>>q7  ! Use element q7 in branch \#1
\end{example}

%% show field --------------------------------------------------------------

\subsection{show field}
\label{s:show.field}

Syntax:
\begin{example}
  show field <ele> \{-derivatives\} \{-absolute_s\} \{-percent_len\} <x> <y> <s> \{<t-or-z>\}
\end{example}

The \vn{show field} command shows the electric and magnetic field at a point in space-time and, if
the \vn{-derivatives} switch is present, the field derivatives as well as curl and divergence are
also printed.

\vn{<ele>} is the lattice element whose fields are to be displayed. The syntax for \vn{<ele>} is
explained in section \sref{s:ele.list.format}. Wild card characters are permitted. If multiple
elements are matched, the field for each will be printed.

\vn{<x>}, and \vn{<y>} are the transverse coordinates and \vn{<s>} coordinate is the longitudinal
position with respect to the beginning of the element. 

The \vn{<t-or-z>} argument is optional and specifies the time if absolute time tracking is being
used or the phase space $z$ value if relative time tracking is being used (use the \vn{show
universe} command to see if absolute time tracking is used or not). The \vn{<t-or-z>} argument is
only useful for elements with RF fields. If not set, \vn{<t-or-z>} will default to zero.

Expressions can be used for all real quantities. An expression must be quoted if it contains any
blank spaces or, simpler, any blank spaces can be removed.

If the \vn{-absolute_s} switch is present, the \vn{<s>} value will be relative to the start
of the element's lattice branch instead of relative to the start of the element.

If the \vn{-percent_len} switch is present, the \vn{<s>} value will be taken as a percentage
of the element length with 0.0 representing the upstream end of the element and 1.0 representing
the downstream end.

\begin{example}
  show field q1 0.1  0.2, 0.5*ele::q1[L]  ! Show field at (x, y) = (0.1, 0.2) and 
                                          !    at s-center of q1 element.
  show field q1 -percent 0.1  0.2 0.5     ! Same as above.
  show field 2>>3 0 0 0 -deriv            ! Show field at start of 3rd element in branch 2.
\end{example}

%% show global --------------------------------------------------------------

\subsection{show global}
\label{s:show.global}

Syntax:
\begin{example}
  show global \{-bmad_com\} \{-space_charge_com\} \{-optimization\} \{-ptc\} \{-ran_state\} 
\end{example}

The \vn{show global} command prints lists of global parameters. 

Note: The state of the random number generator is only used for debugging purposes and is not of
interest to the typical user.

Specifically:
\begin{example}
  show global                   ! Displays \tao's global parameters.
  show global -bmad_com         ! Displays \vn{bmad_com} parameters (\sref{s:globals}).
  show global -space_charge_com ! Displays \vn{space_charge_com} parameters (\sref{s:globals}).
  show global -optimization     ! Displays optimization parameters.
  show global -ran_state        ! Displays the state of the random number generator.
\end{example}

%% show graph --------------------------------------------------------------

\subsection{show graph}
\label{s:show.graph}

Syntax:
\begin{example}
  show graph <graph_name>
\end{example}

Show information on a particular graph of a particular plot. See \sref{c:plotting} for the syntax on
plot, graph, and curve names.  Use \vn{show plot} to get a list of plot names.

If there are visible plots with the same name as the plot parameter of \vn{<graph>}, a template plot
of the same name is ignored. To show template plot graphs(s) in this case, add a ``\vn{T::}''
prefix.

Also see: \vn{show plot} and \vn{show curve} commands.

Example:
\begin{example}
  show graph r2.g1         ! Show the attributes of graph "g1" which is 
                           !   plotted in region "r2".
\end{example}

%% show history --------------------------------------------------------------

\subsection{show history}
\label{s:show.history}

Syntax:
\begin{example}
  show history \{-filed\} \{-no_num\} \{<num_to_display>\}
\end{example}

Shows the command history. Each command is given an index number starting from 1 for the first
command. This index is printed with the command unless the \vn{-no_num} switch is present. If the
\vn{-filed} switch is present, the numbering is shifted so that the current \vn{show history} command
has index zero.

The number of commands printed is, by default, the last 50. Setting the \vn{<num_to_display>} will
change this. Setting \vn{<num_to_display>} to \vn{all}  will cause all the commands to be printed.

Use the command \vn{re_execute} (\sref{s:re.exe}) to re-execute a command. Also the up and down
arrow keys on the keyboard can be used to scroll through the command history stack.

If a command file has been called, the commands within the command file will be displayed but will
be proceeded by an exclamation mark ``!'' to show that the command was not ``directly'' executed.

Commands from previous sessions of \tao are saved in the file \vn{~/.history_tao}. By default they
are not displayed. Use the \vn{-filed} switch to include commands from previous sessions.

Examples
\begin{example}
  show -write cmd_file hist all -no   ! Create a command history file
  show hist 30                        ! Show the last 30 commands.
\end{example}

%% show hom --------------------------------------------------------------

\subsection{show hom}
\label{s:show.hom}

Syntax:
\begin{example}
  show hom
\end{example}

Shows long--range higher order mode information for linac accelerating
cavities.

%% show hom --------------------------------------------------------------

\subsection{show internal}
\label{s:show.internal}

The \vn{show internal} command is for printing parameter values that are internal to \tao. This
command is used for code debugging and not useful (nor understandable) to non-programmers. Note to
programmers: Further information is contained in the code that executes the \vn{show internal}
command.

%% show key_bindings  --------------------------------------------------------------

\subsection{show key_bindings}
\label{s:show.key}

Syntax:
\begin{example}
  show key_bindings
\end{example}

Shows all key bindings (\sref{s:key.bind}).

%% show lattice --------------------------------------------------------------

\subsection{show lattice}
\label{s:show.lattice}

Syntax:
\begin{example}
  show lattice \{-0undef\} \{-all\} \{-attribute <attrib>\} \{-base\} \{-beginning\}
      \{-blank_replacement <string>\}  \{-branch <name_or_index>\}
      \{-custom <file_name>\} \{-design\} \{-floor_coords\} \{-lords\} \{-middle\}
      \{-no_label_lines\} \{-no_slaves\} \{-no_super_slaves\} \{-no_tail_lines\} \{-orbit\} 
      \{-python\} \{-radiation_integrals\} \{-remove_line_if_zero <column \#>\} 
      \{-rms\} \{-s <s1>:<s2>\} \{-spin\} \{-sum_radiation_integrals\} \{-tracking_elements\} 
      \{-undef0\} \{-universe <uni_index>\} \{<element_list>\} 
\end{example}

Show a table of Twiss and orbit data, etc. at the specified element locations. The default is to
show the parameters at the exit end of the elements. To show the parameters in the middle use the
\vn{-middle} switch.

By default, the appropriate element(s) within the \vn{model} lattice (\sref{s:universe}) are
used. This can be overridden by using the \vn{-base} or the \vn{-design} switches which switch the
lattice to the \vn{base} or \vn{design} lattices respectively.

\begin{description}
\item[-0undef] \Newline
See the \vn{-undef0} attribute for a description. Also see the \vn{blank_replacement} switch.
%
\item[-all] \Newline
For lattices with a large number of elements, the \vn{show lattice} command defaults to only showing
the first 200 elements or so to prevent the accidental generation of possibly tens of thousands of
lines. The \vn{-all} switch overrides this default and shows all tracking and lord elements. Also
see the \vn{-lords}, \vn{-no_slaves}, \vn{no_super_slaves}, \vn{-tracking_elements} switches.
%
\item[-attribute <attrib>] \Newline
Instead of defining a custom file, the \vn{-attribute <attrib>} switch can be used as a shortcut way
for customizing the output columns.  When using the \vn{-attribute} switch, the first five columns
are the the same default columns of \vn{index}, \vn{name}, \vn{element key}, \vn{s} and
\vn{length}. All additional columns are determined by the \vn{-attribute} switch. Multiple
\vn{-attribute} switches can be present and the number of additional columns will be equal to the
number of times \vn{-attribute} is used.  The \vn{<attrib>} parameter for each \vn{-attribute}
switch specifies what attribute will be printed.  The general form of \vn{<attrib>} is:
\begin{example}
  attribute-name         or
  attribute-name@format
\end{example}
where \vn{attribute-name} is the name of an attribute and \vn{format} specifies the Fortran style
edit descriptors to be used (\sref{s:edit.descrip}). The default format is \vn{es12.4}.  Example:
\begin{example}
  show lat -attrib is_on@l4 -attrib voltage rfcavity::*
\end{example}
In the above example, \vn{-attribute} appears twice and the total number of columns of output will
thus be 7 (= 5 + 2). The sixth column will have the \vn{is_on} element attribute and will be printed
using the \vn{l4} format (logical with a field width of 4 characters). The seventh column will show
the voltage attribute.

Note: Data can be used in custom output but data is evaluated independent of whether the
\vn{-middle} switch is used.

Also see the \vn{-0undef}, \vn{-undef0}, and \vn{-blank_replacement} switches.
%
\item[-base] \Newline
  Show values from the \vn{base} lattice instead of the \vn{model} lattice. Also see the \vn{-design} switch.
%
\item[-beginning] \Newline
Show value evaluated at the beginning of the lattice elements instead of the default exit end.  The
\vn{-beginning} switch is ignored when displaying ``Intrinsic'' element parameters such as the
element's length or an element's field strength (which can be displayed using the \vn{-attribute}
switch as discussed below). Also the \vn{-beginning} switch is ignored when displaying beam based
parameters. Also see \vn{-middle}.

%
\item[-blank_replacement <string>] \Newline
The \vn{-blank_replacement} switch specifies that whenever a blank string is encountered (for
example, the \vn{type} attribute for an element can be blank), \vn{<string>} should be substituted
in its place. \vn{<string>} may not contain any blank characters. Example:
\begin{example}
  show lat -cust custom.file -blank zz 1:100
\end{example}
This will replace any blank fields with ``zz''.
%
\item[-branch] \Newline
The \vn{-branch <name_or_index>} option can be used to specify the branch of the lattice.
\vn{<name_or_index>} can be the name or index of the branch.  The default is the main branch (\# 0).
%
\item[-custom <file_name>] \Newline
A table with customized columns may be constructed either by using the \vn{-custom} switch which
specifies a file containing a description of the custom columns or by using one or more
\vn{-attribute} switches. Example customization file:
\begin{example}
  &custom_show_list
    column(1)  = "#",                      "i6",     6 
    column(2)  = "x",                      "x"       1   ! blank space
    column(3)  = "ele::#[name]",           "a",      0
    column(4)  = "ele::#[key]",            "a16",   16
    column(5)  = "ele::#[s]",              "f10.3", 10
    column(6)  = "ele::#[l]",              "f10.3", 10
    column(7)  = "ele::#[beta_a]",         "f7.2",   7
    column(8)  = "1e3 * ele::#[orbit_x]",  "f8.3",   8, "Orbit_x| (mm)" 
    column(9)  = "lat::unstable.orbit[#]", "f8.3",   8 
    column(10) = "beam::n_particle_loss[#]", "i8",   8 
  /
\end{example}
each \vn{column(n)} line has four parameters. The first parameter is what is to be displayed in that
column. Algebraic expressions are permitted (\sref{s:arithmetic.exp}).

The second parameter is the Fortran edit descriptor. The third column is the total width of the
field. Notice that strings (like the element name) are left justified and numbers are right
justified. In the case of a number followed by a string, there will be no white space in
between. The use of an "x" column can solve this problem. A field width of 0, which can only be used
for an \vn{ele::\#[name]} column, indicates that the field width will be taken to be one greater
then the maximum characters of any element name.

The last parameter is column title name. This parameter is optional and if not present then \tao
will choose something appropriate. The column title can be split into two lines using \vn{"|"} as a
separator.  In the example above, The column title corresponding to \vn{"Orbit_x| (mm)"} will have
``Orbit_x'' printed in one row of the title and ``(mm)'' in the next row.

To encode the element index, use a \vn{\#} or \vn{\#index}. To encode the branch index, use
\vn{\#branch}. Any element attribute is permitted ("show ele" will show element attributes or see
the Bmad manual). Additionally, the following are recognized:
\begin{example}
  x                           ! Add spaces
  #                           ! Index number of element.
  ele::#[name]                ! Name of element.
  ele::#[key]                 ! Type of element (``quadrupole'', etc.)
  ele::#[slave_status]        ! Slave type (``super_slave'', etc.)
  ele::#[lord_status]         ! Slave type (``multipass_lord'', etc.)
  ele::#[type]                ! Element type string (see \bmad manual).
\end{example}

Note: Data can be used in custom output but data is evaluated independent of whether the
\vn{-middle} switch is used.

Also see the \vn{-0undef}, \vn{-undef0}, and \vn{-blank_replacement} switches.
%
\item[-design] \Newline
Show values from the \vn{design} lattice instead of the \vn{model} lattice. Also see the \vn{-base}
switch.
%
\item[<element_list>] \Newline
The locations to show are specified either by specifying an element list or by specifying a
longitudinal position range using the \vn{-s} switch The syntax used for specifying the element list
is given in the \vn{Lattice Element List Format} section (\sref{s:ele.list.format}). In this case
there should be no blank characters in the list.
%
\item[-floor_coords] \Newline
If present, the \vn{-floor_coords} switch will print the global floor (laboratory) coordinates for
each element.
%
\item[-lords] \Newline
If present, the \vn{-lords} switch will print a list of lord elements only. Also see the
\vn{-all}, \vn{-no_slaves}, \vn{no_super_slaves}, \vn{-tracking_elements} switches.
%
\item[-middle] \Newline
Show value evaluated at the middle of the lattice elements instead of the default exit end.
The \vn{-middle} switch is ignored when displaying ``Intrinsic'' element
parameters such as the element's length or an element's field strength (which can be displayed using
the \vn{-attribute} switch as discussed below). Also the \vn{-middle} switch is ignored when
displaying beam based parameters. Also see \vn{-beginning}.
%
\item[-no_label_lines] \Newline
If present, the \vn{-no_label_lines} switch will prevent the printing of the header (containing the
column labels) lines at the top and bottom of the table.  This is useful when the output needs to be
read in by another program. Also see the \vn{-no_tail_lines} switch.
%
\item[-no_slaves] \Newline
If the \vn{-no_slaves} switch is present, all \vn{super slave} and \vn{multipass slave} elements
will be ignored. Also see the \vn{-all}, \vn{-lords}, \vn{no_super_slaves}, \vn{-tracking_elements}
switches.
%
\item[-no_super_slaves] \Newline
If present, the \vn{-no_super_slaves} switch will veto from the list of elements to print all
\vn{super slave} elements. Also see the \vn{-all}, \vn{-lords}, \vn{-no_slaves}, \vn{no_super_slaves},
\vn{-tracking_elements} switches.
%
\item[-no_tail_lines] \Newline
The \vn{-no_tail_lines} just suppress the header lines at the bottom of the table. Also see the
\vn{-no_label_lines} switch.
%
\item[-orbit] \Newline
The \vn{-orbit} switch will show the particle's phase space orbit which is the closed orbit if the
lattice has a closed geometry and is the orbit beginning from the specified starting position for
lattices with an open geometry. Use \vn{set particle_start} to vary the starting position in this
case. If the \vn{-spin} switch is also present, the particle's spin will also be displayed.
%
\item[-python] \Newline
The \vn{-python} switch gives a comma delimited table as output. This switch is used with the
\vn{python} command (\sref{s:python}).
%
\item[-radiation_integrals] \Newline
The \vn{-radiation_integrals} switch, if present, will display the radiation integrals for each
lattice element instead of the standard Twiss and orbit data. See the \bmad manual for the
definitions of the radiation integrals. Also see \vn{-sum_radiation_integrals}.
%
\item[-remove_line_if_zero <column \#>] \Newline
If present, the \vn{-remove_line_if_zero} switch will suppress any lines where the value
in the column given by \vn{<column \#>} is zero or not defined. Notice that when specifying custom
columns using the \vn{-custom} switch, columns that only insert blank space are not counted. For
example:
\begin{example}
  show lat -custom cust.table -remove 5
\end{example}
Assuming that the file \vn{cust.table} contains the example customization given above, the fifth
visible column corresponds to \vn{column(6)} which prints the element length. The \vn{-remove 5}
will then remove all lines associated with elements whose length is zero. Multiple
\vn{-remove_line_if_zero} may be present. In this case the row will be suppressed if all designated
columns have a zero entry.
%
\item[-rms] \Newline
When the \vn{-rms} switch is present, five additional lines will be added to the output showing the
mean and RMS values for columns that had floating point numbers along with mean and RMS values
integrated over the longitudinal s-position. A simple trapezoid integration is used for the
integrated values. The fifth row shows the number of data points. Only valid values will be included
in the calculation.
%
\item[-s <s1>:<s2>] \Newline
The locations to show are specified either by specifying a longitudinal position range with \vn{-s},
or by specifying a list \vn{<element_list>} of elements. 
%
\item[-spin] \Newline
The \vn{-spin} switch will show the particle's spin which is the invariant spin if the lattice has a
closed geometry and is the spin beginning from the specified starting spin for lattices with an open
geometry. Use \vn{set particle_start} to vary the starting spin in this case. If the \vn{-orbit} switch
is also present, the particle's phase space orbit will also be displayed.
%
\item[-sum_radiation_integrals] \Newline
The \vn{-sum_radiation_integrals} switch, if present, will display the radiation integrals
integrated from the start of the lattice to for each lattice element. See the \bmad manual for the
definitions of the radiation integrals. Also see the \vn{-radiation_integrals} switch.
%
\item[-tracking_elements] \Newline
The \vn{-tracking_elements} switch can be used to show all the elements in the tracking part of the
lattice. Also see the \vn{-all}, \vn{-lords}, \vn{-no_slaves}, \vn{no_super_slaves} switches.
%
\item[-undef0] \Newline
If an attribute does not exist for a given element (for example, \vn{quadrupole}s do not have a
\vn{voltage}), a series of dashes, ``-{}-{}-{}-'', will be placed in the appropriate spot in the
table.  Additionally, an arithmetic expression that results in a divide by zero will result in
dashes being printed. This behavior is changed if the \vn{-0undef} or \vn{-undef0} switch is
present. In this case, a zero, ``0'', will be printed. The difference between \vn{-0undef} and
\vn{-undef0} is that with \vn{-undef0} the zero will be printed using the same format as the other
numbers in the column. With the \vn{-0undef} switch the zero will be printed as a right justified
``0'' which gives a visual clue to differentiate between a true zero value and a zero that
represents an undefined parameter.
%
\item[-universe <index>] \Newline
The \vn{-universe} switch specifies which universe is used. If not present, the current viewed universe is
used.
\end{description}

Examples:
\begin{example}
  show lattice 50:100         ! Show lattice elements with index 50 through 100
  show lat 45:76, 101, 106    ! Show element #45 through #76 and 101 and 106.
  show lat q34w:q45e          ! Show from element q34w through q45e.
  show lat q*                 ! Show elements whose name begins with "q"
  show lat marker::bpm*       ! Show marker elements whose name begins with "bpm"
  show lat -s 23.9:55.3       ! Show elements whose position is between 
                              !   23.9 meters and 55.3 meters.
  show lat -att x_offset -rms ! Table will have a column of lattice element x_offset
                              !   values. Additionally, mean and RMS will be shown.
\end{example}

%% show matrix --------------------------------------------------------------

\subsection{show matrix}
\label{s:show.matrix}

Syntax:
\begin{example}
  show matrix \{-eigen_modes\} \{-ptc\} \{-s\} \{-universe\} \{loc1 \{loc2\}\}
\end{example}

Shows the transfer matrix (Jacobian) for the \vn{model} lattice. The matrix is calculated with
respect to the particle orbit, not the zero orbit. This command is equivalent to \vn{show taylor_map
-order 1}. See the \vn{show taylor_map} documentation for more details.

Also see \vn{write matrix}.

%% show merit --------------------------------------------------------------

\subsection{show merit}
\label{s:show.merit}

Syntax:
\begin{example}
  show merit \{-derivative\} \{-merit_only\}
\end{example}

If the \vn{-derivative} switch is present, this command shows top dMerit/dVariable derivatives, and
Largest changes in variable value. If not present, this command shows top contributors to the merit
function.

Also see: \vn{show constraints}.

If the \vn{-merit_only} switch is present, only the value of the merit function is printed and
nothing else. That is, it makes the output compact if only the value of the merit function is
desired.

Note: To set the number of top contributors shown, use the command 
\begin{example}
  set global n_top10_merit = <number>
\end{example}
where \vn{<number>} is the desired number of top contributors to the merit function to be shown.

Note: The \vn{show merit} command was once called the \vn{show top10} command.

Example:
\begin{example}
  show merit -der     ! Show merit derivative info
\end{example}

%% show optimizer --------------------------------------------------------------

\subsection{show optimizer}
\label{s:show.optimizer}

Syntax:
\begin{example}
  show optimizer \{-geodesic_lm\}
\end{example}


Shows parameters pertinent to optimization: Data and variables used,
etc. 

If \vn{-geodesic_lm} option is present, parameters for the \vn{geodesic_lm} optimizer will
be shown. These parameters are shown in any case if the optimizer has been set to use
\vn{geodesic_lm}.

Also see:
\begin{example}
  show constraints
  show data
  show derivative
  show merit
  show variables
\end{example}

%% show particle --------------------------------------------------------------

\subsection{show particle}
\label{s:show.particle}

Syntax:
\begin{example}
    show particle \{-bunch <bunch_index>\} \{-particle <particle_index>
                  \{-element <element_id>\} \{-lost\} \{-all\}
\end{example}


Shows individual beam particle information except if the the \vn{-lost} or \vn{-all} options are used. 

The default for the optional \vn{-bunch} index is set by the global variable \vn{global%bunch_to_plot}.
The default \vn{-element} is \vn{init} which is the initial beam distribution.
The default \vn{-particle} to show is the particle with index 1.

The \vn{-lost} option shows which particles are lost during beam tracking. Note: Using the
\vn{-lost} option results in one line printed for each lost particle. It is thus meant for use with
bunches with a small number of particles.

The \vn{-all} option shows all particles at the given element.

The \vn{dtime} column shows the time relative to the reference time ($t - t_{ref}$). The conversion
between phase space $z$ and \vn{dtime} is $z = -\beta \, c \, (t - t_{ref})$ where $\beta = v/c$ is
the normalized velocity. Since \vn{dtime} is undefined if $\beta = 0$, zero will be displayed in
this case.

Also see \vn{show beam}.

Examples:
\begin{example}
  show part -bun 3 -part 47 -ele 8 ! Shows information on particle #47 of 
                                   !   bunch #3 at lattice element #8.
  show part -part 47 -ele 8        ! Same as above except the default bunch is used.
  show part -lost -bun 3           ! Show lost particle positions for bunch #3
\end{example}

%% show plot --------------------------------------------------------------

\subsection{show plot}
\label{s:show.plot}

Syntax:
\begin{example}
  show plot \{-floor_plan\} \{-page\} \{-lat_layout\} \{-regions\} 
            \{-templates\} \{<plot_or_template_name>\}
\end{example}

The \vn{show plot -floor_plan} and \vn{show plot -lat_layout} commands show the parameters
associated with the \vn{floor_plan} or \vn{lat_layout} plots (\sref{s:shapes}). Use the \vn{set
floor_plan} or \vn{set lat_layout} commands to set these parameters.

The \vn{show plot -page} command shows some plot page plotting parameters like the size of the plot
window.

The \vn{show plot -regions} command shows what plots are placed in which regions. Use the \vn{place}
command to change where plots are placed.

The \vn{show plot -templates} command displays what plot templates have been defined for plotting.
See \sref{s:init.plot} for information on setting up template plots.

The \vn{show plot <plot_or_region_name>} command will display information on a particular plot. If
there are visible plots with the same name, a template plot of the same name is ignored. To show a
template plot in this case, add a ``\vn{T::}'' prefix.

The various \vn{show plot} options are mutually exclusive and only the last option is used. That is,
a command like
\begin{example}
  show plot -lat_layout -regions
\end{example}
is equivalent to \vn{show plot -regions}.

Also see \vn{show graph} and \vn{show_curve}.

Examples:
\begin{example}
  show plot         ! Show plot region information by default.
  show plot r13     ! Show information on plot in region r13.
  show plot T::beta ! Show template beta plot.
\end{example}

%% show ptc --------------------------------------------------------------

\subsection{show ptc}
\label{s:show.ptc}

Syntax:
\begin{example}
  show ptc 
\end{example}

Show quantities as calculated by PTC. This command is under development. Currently emittances and
tunes are shown.

%% show radiation_integrals --------------------------------------------------------------

\subsection{show radiation_integrals}
\label{s:show.rad.int}

Syntax:
\begin{example}
  show  \{-branch <branch_name>\} \{universe_number\}
\end{example}

Show radiation integrals along with associated parameters computed from the integrals like the
emittance, damping decrement, etc. Values for the associated parameters like the emittance will vary
from the values shown by the \vn{show universe} command since the \vn{show universe} command uses a
computation that derives from the transport matrices with included radiation effects. Differences
between the two are due to differing approximations as explained in the \bmad manual (See the
chapter on \vn{Synchrotron Radiation}).

%% show spin --------------------------------------------------------------

\subsection{show spin}
\label{s:show.spin}

Syntax:
\begin{example}
  show spin \{-element \{<ref_ele_name\} <ele_name\} \{-g_map\}
                                    \{-n_axis <nx>, <ny>, <nz>\} \{-l_axis <lx>, <ly>, <lz>\}
\end{example}

Show spin related information.  Note: To see the closed orbit invariant spin at any element, make
sure spin tracking is on (if not, use: \vn{set bmad_com spin_tracking_on = T}), and then the
\vn{show element} command will display $n_0$.

If \vn{-element} is not present, \vn{show spin} will show various quantities including polarization
limits and polarization rates. See the \vn{Spin Dynamics} chapter of the \bmad manual for a
discussion of how these quantities are calculated.

If \vn{-element} is present, the output will be the first order spin transfer map from the
downstream end of the element given by \vn{<ref_ele_name>} to the downstream end of the element
given by \vn{<ele_name>}.  If \vn{<ref_ele_name>} is not given, the default is the element
given by \vn{<ele_name>}. This gives the 1-turn spin transport from the downstream end of
\vn{<ele_name>}. The spin map can be displayed in two different forms. If the \vn{-l_axis}
or \vn{-n_axis} is given, or if \vn{-g_map} is present, the G-matrix form of the spin map will be
printed.  The G-matrix is dependent upon the $(\Bf l, \Bf n, \Bf m)$ axes used to define the spin
vector (see the \vn{SLIM Formalism} section of the Spin chapter in the Bmad manual). The $\Bf n$ and
$\Bf l$ axes can be specified using the \vn{-n_axis} and \vn{-l_axis} switches. The $\Bf m$-axis is
calculated from knowledge of the other two axes. If not given, the $\Bf n$-axes will be set to the
reference orbit $\bfn_0$ value (the reference orbit is the closed orbit if the lattice geometry is
closed).  The $\Bf l$-axis, if not given, will be chosen to be perpendicular to $\Bf n$. Commas
between axis parameters are optional. If \vn{q_map} is present, or if the G-matrix is not being
printed, the quaternion form of the map will be printed.

To see the linear spin-orbit resonance strengths, use the \vn{-ele} switch with either
\vn{<ref_ele_name>} being blank or the same as \vn{<ele_name>}. The spin-orbit resonance strength
table has three rows for the three orbital modes $a$, $b$, and $c$. The columns of the table are:
\begin{Itemize}
  \item Col 1: Orbital tune.
  \item Col 2: Minimum of (|Q_orb-Qs|, |Q_orb+Qs|). That is, the minimum of the tune separation between the 
     "sum" and "difference" resonances. The resonance calculation will only be accurate near resonance.
  \item Cols 3\&4: The normalized resonance strengths. For computations use the largest value.
\end{Itemize}
\vskip -1.0ex
See the Bmad manual section on the linear spin/orbit resonance analysis for more details.  The
\vn{python spin_resonance} command can be used to extract resonance values when using running \tao
with a script.

Example:
\begin{example}
  show spin
  show spin -ele Q1-1 Q1 -n 0,1,0 -l 1,0,0  ! G-matrix for Q1
  show spin -ele S1     ! 1-turn analysis at downstream end of S1 
  show spin -ele 4 7    ! G-matrix from end of element 4 to end of element 7
\end{example}

%% show string --------------------------------------------------------------

\subsection{show string}
\label{s:show.string}

Syntax:
\begin{example}
  show string \{string-to-print\}
\end{example}

Print a string. This can be useful when creating a data file. Use ``\vn{\B{}n}'' to output multiple
lines. Anything within backticks, \`{}...\`{}, will be evaluated. If an evaluated quantity is an
array, the array is enclosed in brackets ``[...]''. Also if the evaluated quanty ends in
\vn{``@@N''} where \vn{N} is an integer, this is used to determine the accuracy of the printed
value(s). The default is 14. Also see \vn{show value}.

Examples:
\begin{example}
  show -append a.dat str 2 + 2 = \`{}2+2\`{} ! Writes the line "2 + 2 = 4" to a.dat
  show str \`{}1e3*lat::orbit.x[3:5]@@2\`{}  ! Prints something like "[3.4, 3.2, 2.7]"
\end{example}

%% show symbolic_numbers --------------------------------------------------------------

\subsection{show symbolic_numbers}
\label{s:show.symbolic}

Syntax:
\begin{example}
  show symbolic_numbers \{-physical_constants\} \{-lattice_constants\}
\end{example}

Show the symbolic constants created using the \vn{set symbolic_number} command. 

If the \vn{-physical_constants} switch is present, the predefined physical constants (like
\vn{c_light}) along with predefined mathematical constants (like \vn{pi}) are displayed instead
(Also see the \bmad manual for this list).

If the \vn{-lattice_constants} switch is present, constants defined in the lattice are displayed.
Note: To import these symbols into Tao, set \vn{global%symbol_import} to True (or use the
\vn{-symbol_import} switch on the startup command line). The default is to not import lattice
symbols.

Examples:
\begin{example}
  set sym aaa = 23  ! Set a symbol.
  show sym          ! Show all user defined symbols.
  show sym -phys    ! Show predefined physical and mathematical constants.
\end{example}

%% show taylor_map --------------------------------------------------------------

\subsection{show taylor_map}
\label{s:show.taylor}

Syntax:
\begin{example}
  show taylor_map \{-angle_coordinates\} \{-eigen_modes\} \{-lattice_format\} 
            \{-number_format <fmt>\} \{-order <n_order>\} \{-ptc\} \{-radiation\}
            \{-s\} \{-universe\} \{loc1 \{loc2\}\}
\end{example}

Shows the Taylor transfer map for the \vn{model} lattice of the default universe (set by
\vn{set default universe}).  See also \vn{show matrix}.

If neither \vn{loc1} nor \vn{loc2} are present, the transfer map is computed for the
entire lattice.

if \vn{loc1} and \vn{loc2} are the same, the 1-turn transfer map is computed. If the
s-position of \vn{loc1} is greater than the s-position of \vn{loc2}, the map from
\vn{loc1} to the end of the lattice with the map from the beginning to \vn{loc2} is
computed.

If the \vn{-s} switch is present, \vn{loc1} and \vn{loc2} will be interpreted as longitudinal
s-positions. In this case, if \vn{loc2} is not present, the map will be the 1-turn map if the
lattice is circular and the map from the beginning to \vn{loc1} if the map is not.

If the \vn{-s} switch is not present, \vn{loc1} and \vn{loc2} will be interpreted as element names
or indexes. The map will be from the exit end of the \vn{loc1} element to the exit end of the
\vn{loc2} element. In this case, if \vn{loc2} is not present, the map will be the for the element
given by \vn{loc1}

If the \vn{-eigen_modes} switch is present, the first order part of the map will be treated as a
1-turn matrix and the corresponding eigen values and eigen vectors will be printed.

The \vn{-number_format} switch, if present, overrides the default format for the displayed format.
Examples:
\begin{example}
  show taylor -num f12.6  ! Fixed format. 12 char width, 6 digits after decimal point.
  show taylor -num es12.4 ! Float format. 12 char width, 4 digits after decimal point.
\end{example}

The \vn{-order} switch, if present, gives the limiting order to display. In any case, the
maximum order of the map is limited to the order set by the lattice file.

The \vn{-ptc} switch is used with \vn{-order 1}. By default, order 1 maps (matrices) are calculated
using native Bmad code. If the \vn{-ptc} switch is present, the matrix is calculated using the
\vn{PTC} code (see the \bmad manual for details on PTC). Since PTC is always used to calculate maps
of order higher than 1, the \vn{-ptc} switch is ignored for higher orders. If used with the
\vn{-radiation} switch, the \vn{-ptc} switch 

The \vn{-lattice_format} switch, if present, causes the output to be displayed in a format suitable
for using in a \bmad lattice file.

If the the \vn{-radiation} switch is present, displayed will be the linear map with radiation
damping and excitation. See the \vn{Synchrotron Radiation} chapter in the \bmad manual for how the
matrices displayed here are defined. The RF cavities should, in general be powered since these
affect the reference orbit and transfer matrix. The damping matrix will be computed independent of
whether radiation damping is on or off for tracking. The difference is that the reference orbit is
affected by having damping on or off. Having radiation damping off in fact may be the preferable
since in an actual machine the ``sawtooth'' orbit which comes with having damping on will tend to be
compensated by tuning of the machine to meet the design conditions. If the \vn{-ptc} switch is used
with \vn{-radiation}, the PTC calculated closed orbit will be used as a reference instead of the
\bmad one.

To toggle radiation damping and RF use the commands:
\begin{example}
  set bmad_com radiation_damping_on = T
  set global rf_on = T
\end{example}
Note that the calculation does not depend upon the radiation excitation being turned on (since
radiation excitation will not affect the reference orbit or transfer matrix).

The \vn{-angle_coordinates} switch, if present, causes the output to be displayed using ``angle''
phase space coordinates $(x, x', y, y', z, p_z)$ in place of the standard \bmad canonical
coordinates $(x, p_x, y, p_y, z, p_z)$.\footnote
  {
The conversion between the two coordinate systems is given in the Bmad manual.
  }

Maps are computed about the particle orbit. Not the zero orbit. Also see \vn{write matrix}.

Expressions can be used for all real quantities (that is, \vn{loc1} and \vn{loc2} if \vn{-s} is
present). An expression must be quoted if it contains any blank spaces or, simpler, any blank spaces
can be removed.

Examples:
\begin{example}
  show taylor -order 1 q10w q12e  ! 0th and 1st order maps from q10w to q12e
  show taylor 45                  ! Transfer map of element #45
  show taylor -s 13 23            ! Transfer map from s = 13 meters to 23 meters. 
\end{example} 

\vfill\break

%% show track -----------------------------------------------------------------------

\subsection{show track}
\label{s:show.track}

Syntax:
\begin{example}
  show track \{-b_field \{<fmt>\}\} \{-base\} \{-branch <name_or_index>\} \{-design\} 
      \{-dispersion \{<fmt>\}\} \{-e_field \{<fmt>\}\} \{-element <ele_id>\} \{-momentum \{<fmt>\}\} 
      \{-no_label_lines\} \{-points <num>\} \{-position \{<fmt>\}\} \{-energy \{<fmt>\}\} 
      \{-range <s1> <s2>\} \{-s \{<fmt>\}\} \{-spin \{<fmt>\}\} \{-time \{<fmt>\}\} 
      \{-twiss \{<fmt>\}\} \{-universe <ix_uni>\} \{-velocity \{<fmt>\}\}
\end{example}

The \vn{show track} command shows a table of phase space coords, Twiss parameters, EM fields, and
other info at equally spaced points along the tracked orbit. Also see the \vn{show twiss_and_orbit}
command.

Command arguments that toggle whether a certain quantity is displayed have an optional \vn{<fmt>}
format specifier that can be used to set the format of the displayed quantities. The format uses
Fortran edit descriptor syntax (\sref{s:edit.descrip}). If ``\vn{no}'' is used as the format then
the associated quantity will not be displayed. If there is no format specified then \tao will use
a default format. Example:
\begin{example}
  show track -b_field      ! Display magnetic field parameters using the default format
  show track -position no  ! Do not display position information.
  show track -s 3pf12.1    ! Display S-position with decimal point shifted by 3 places.
                           !   That is, display the S-position in millimeters.
\end{example}
When the value of quantities are shifted, using the ``\vn{P}'' prefix, the header string for the
corresponding column(s) will be appropriately marked.

\begin{description}
\item[\vn{\{-b_field \{<fmt>\}\}}] \Newline
Set the format for the three parameters of the magnetic field (in Tesla). The default, if
\vn{-b_field} is not present, is not to print the field.
%
\item[\vn{\{-base\}}] \Newline
If present, use the \vn{base} lattice for evaluating quantities. The default is the \vn{model}
lattice.
%
\item[\vn{\{-branch <name_or_index>\}}] \Newline
Lattice branch to use. The default is the default branch (\sref{s:lattice})
%
\item[\vn{\{-design\}}] \Newline
If present, use the \vn{design} lattice for evaluating quantities. The default is the \vn{model}
lattice.
%
\item[\vn{\{-dispersion \{<fmt>\}\}}] \Newline
Set the format for the dispersion and dispersion derivative columns $(\eta_x, \eta'_x, \eta_y, \eta'_y)$. 
The default is not to print these columns.
%
\item[\vn{\{-e_field \{<fmt>\}\}}] \Newline
Set the format for the three parameters of the electric field (in V/m). The default is not to print
the field.
%
\item[\vn{\{-element <ele_id>\}}] \Newline
The \vn{-element} switch can be used instead of \vn{-range} to set the $s$-position. With
\vn{-element} the track range is the extent of the element given by \vn{<ele_id>}. Note: If the
element is a \vn{beambeam} element, the track will show the before and after particle positions at
each strong beam slice along with the initial position after tracking through the previous lattice
element along with the final position which is used to start tracking through the next element.
%
\item[\vn{\{-momentum \{<fmt>\}\}}] \Newline
Set the format for the three phase space momentum parameters $(p_x, p_y, p_z)$. Notice that these
are canonical momenta and are dimensionless as explained in the \bmad manual. In particular, $p_z$
is the momentum deviation from the reference momentum. The default is to print the momenta using the
default format.
%
\item[\vn{\{-no_label_lines\} }] \Newline
If present then suppress the output header lines.
%
\item[\vn{\{-points <num>\}}] \Newline
Set the number of evaluation points. That is, set the number of rows in the table.
%
\item[\vn{\{-position \{<fmt>\}\}}] \Newline
Set the format for the three phase space position parameters $(x, y, z)$. See the \bmad manual for 
details on phase space coordinates. The default is to print the position using the default format.
The default format is \vn{3PF14.6} so the output will be in mm.
%
\item[\vn{\{-energy \{<fmt>\}\}}] \Newline
Set the format for the column showing the total energy (in eV) of the particle. The default is not
to print this.
%
\item[\vn{\{-range <s1> <s2>\}}] \Newline
Set the S-position min/max bounds for the table. Default is beginning and ending $s$-positions of the
lattice.
%
\item[\vn{\{-s \{<fmt>\}\}}] \Newline
Set the format for the S-position column. The default, if \vn{-s} is not present, is to print the
column.
%
\item[\vn{\{-spin \{<fmt>\}\}}] \Newline
Set the format for the three parameters of the particle's spin. The default, if
\vn{-spin} is not present, is not to print the spin.
%
\item[\vn{\{-time \{<fmt>\}\}}] \Newline
Set the format for the time column. The default, if \vn{-s} is not present, is to not the column.
%
\item[\vn{\{-twiss \{<fmt>\}\}}] \Newline
Set the format for the Beta and Alpha functions of the two transverse normal modes. The default, if
\vn{-twiss} is not present, is not to print the Twiss parameters
%
\item[\vn{\{-universe <ix_uni>\}}] \Newline
Set the universe to use. The default is the default universe (\sref{s:universe}).
%
\item[\vn{\{-velocity \{<fmt>\}\}}] \Newline
Set the format for the three particle velocity parameters $(v_x/c, v_y/c, v_z/c)$ normalized by the
speed of light. The default is not to print the velocity.
%
\end{description}

%% show twiss_and_orbit --------------------------------------------------------------

\subsection{show twiss_and_orbit}
\label{s:show.twiss}

Syntax:
\begin{example}
    show twiss_and_orbit \{-base\} \{-branch <name_or_index>\} \{-design\}
    \{-universe <ix_uni>\} <s_position>
\end{example}

The \vn{show twiss_and_orbit} shows Twiss and orbit information at a given longitudinal position
\vn{<s_position>} including synchrotron radiation related parameters. Also see \vn{show track}.

The default universe to use is the current default universe. This can be changed using the
\vn{-universe} switch.

The default is to show the \vn{model} Twiss and orbit parameters. The use of \vn{-base} or
\vn{-design} switches can be used to show parameters for the \vn{base} or \vn{design} lattices.

The particular branch used in the analysis can be selected by the \vn{-branch} switch. The
default is the default branch (\sref{s:lattice}).

Examples:
\begin{example}
  show twiss -uni 2 23.7     ! Show parameters in universe 2 at s = 23.7 meters.
\end{example} 

%% show universe --------------------------------------------------------------

\subsection{show universe}
\label{s:show.universe}

Syntax:
\begin{example}
  show universe \{-branch <branch_name>\} \{universe_number\}
\end{example}

Shows various parameters associated with a given branch of a given universe. If no universe is
specified, the current default universe is used. If no branch is given, the current default branch
is used. Parameters displayed include tune, emittances, etc. 

Here quantities like the emittance or momentum compaction factor are calculated from the transport
matrices with included radiation effects. Previously (pre April 2022), such quantities where
calculated from evaluation of the synchrotron radiation integrals. To see the radiation integral
derived values use the command \vn{show} \vn{radiation_integrals}. Differences between the results
of the transport matrix and radiation integral treatments are due to differing approximations as
explained in the \bmad manual (See the chapter on \vn{Synchrotron Radiation}).

Example:
\begin{example}
  show universe -branch 1 3  ! Show info on branch 1 of universe 3.
\end{example}

%% show use --------------------------------------------------------------

\subsection{show use}
\label{s:show.use}

Syntax:
\begin{example}
  show use
\end{example}

Shows what data and variables are used in a format that, if saved to a file, can be read
in with a \vn{call} command.

%% show value --------------------------------------------------------------

\subsection{show value}
\label{s:show.value}

Syntax:
\begin{example}
  show value \{-format <format-string>\} <expression>
\end{example}

Shows the value of an expression. The \vn{-format} switch can be used to set the number format.
The default format is \vn{es25.17} which results in number printed in scientific notation format
with a field width of 25 and 17 digits displayed after the decimal place. Use ``\vn{f}'' for fixed
point numbers. For example, ``\vn{f10.3}'' will give display numbers in fixed point format with a
field width of 10 and 3 digits displayed after the decimal place.

Also see \vn{show string}. 

Examples:
\begin{example}
  show value sqrt(3@lat::orbit.x[34]|model) + sin(0.35)
  show value -form f10.4 ran_gauss()
\end{example}


%% show variables --------------------------------------------------------------

\subsection{show variables}
\label{s:show.variables}

Syntax:
\begin{example}
  show variables \{-no_label_lines\} \{-universe <universes>\}            
         \{-good_opt_only\} \{-bmad_format\} \{<var_name>\}
\end{example}


Shows variable information. If \vn{<var_name>} is not present, a list of all appropriate \vn{v1_var}
classes is printed.

The \vn{-universe} switch is used to select only variables what control parameters in a given
universe or universes. Use \vn{-universe @} to select the current viewed universe.

If the \vn{-bmad_format} switch is used then the Bmad lattice parameters that the \tao variables
control will be printed in Bmad lattice format. This is the same syntax used in generating the
variable files when an optimizer is run. If \vn{-good_opt_only} is used in conjunction with
\vn{-bmad_format} then the list of variables will be restricted to ones that are currently being
used in the optimization.

If present, the \vn{-no_label_lines} switch will prevent the printing of the header (containing the
column labels) lines. This switch is ignored if \vn{-bmad_format} is present.

Examples:
\begin{example}
  show var             ! List all v1 variables.
  show var quad_k1     ! List variables in the quad_k1[*] array.
  show var quad_k1[10] ! List detailed information on the variable quad_k1[10].
  show var -uni 2      ! List all variables that control attributes in universe 2.
  show var -bmad       ! List variables in Bmad Lattice format.
\end{example}

%% show version --------------------------------------------------------------

\subsection{show version}
\label{s:show.version}

Syntax:
\begin{example}
  show version 
\end{example}

The \vn{show version} command will show the ``version'' of \tao corresponding to \tao executable
being run. Since \bmad and \tao are under continuous development, the standard semantic versioning
scheme using major and minor numbers does not make sense. Instead, the version string associated
with \tao is a date encoded in the form
\begin{example}
  YYYY_MMDD
\end{example}
where \vn{YYYY} is the four digit year, \vn{MM} is the two digit month, and \vn{DD} is the two digit
day. The version string is stored in the file \vn{\$TAO_DIR/VERSION}.  For the \vn{show version}
command to work properly, the environmental variable \vn{TAO_DIR} must be appropriately
defined. Generally, \vn{TAO_DIR} will be defined if the appropriate \bmad setup script has been
run. For ``Bmad Distributions'', this is the same setup script used to setup a distribution. See
your local \bmad guru for details.\footnote
  {
For developers who use SVN to update their Distributions: The version string is not updated when
you update your Distribution (the version file is not in the SVN repository). 
  }

%% show wakes --------------------------------------------------------------

\subsection{show wakes}
\label{s:show.wakes}

Syntax:
\begin{example}
  show wakes 
\end{example}

The \vn{show wakes} command will list the lattice elements that have associated wake fields.  Use
the \vn{show ele} command to get more details on a given element. Note that wakes only affect
particle tracking when tracking with a beam of particles (not when tracking just a single particle
which is the default for \tao).

At this point in time, \tao is not setup to do multiturn tracking with bunches which means that if
simulations with wakefields is desired, a different program have to be used.

%% show wall --------------------------------------------------------------

\subsection{show wall}
\label{s:show.wall}

Syntax:
\begin{example}
  show wall \{-branch <name_or_index>\}\{-section <index>\} \{-angle <angle>\}
  \{-s <s1>:<s2>\} \{<n1>:<n2>\}
\end{example}

The \vn{show wall} command shows the vacuum chamber wall associated with a lattice branch.

For the building wall, use the ``show building_wall'' command.

For showing the wall associated with a given element, use the ``show ele -wall'' command.

The \vn{-branch} switch is used to select a particular branch.

The \vn{-section} switch is used to show information about a specific chamber wall cross-section. In
this case, all the other options are ignored except for \vn{-branch}.

If \vn{-section} is not present, a list of vacuum chamber wall sections is presented. In this case,
the range of wall sections shown is given by \vn{<n1>:<n2>} except if \vn{-s} is present in which
case all sections within a range of \vn{s} values is given within the range \vn{<s1>} to
\vn{<s2>}. With each section, a wall radius is given. The angle in the $(x,y)$ plane at which the
radius is computed is determined by the \vn{-angle} option. The default angle is 0 which corresponds
to the $+x$ direction.

Examples:
\begin{example}
  show wall 45:100       ! Show vacuum chamber wall sections 45 through 100.
  show wall -s 10.0:37.5 ! Show wall sections that have S-position between 10 and 37.5.
  show wall -section 49  ! Show chamber wall section 49.
\end{example}

%% show wave --------------------------------------------------------------

\subsection{show wave}
\label{s:show.wave}

Syntax:
\begin{example}
  show wave
\end{example}

The \vn{show wave} command shows the results of the current wave analysis (\sref{c:wave}).

%% single_mode --------------------------------------------------------------

\section{single_mode}\index{commands!single_mode}
\label{s:sing}

The \vn{single_mode} command puts \tao into \vn{single mode} (\sref{c:single}).  For on-line help
when running \tao go to \vn{single mode} and type ``?''.  To get out of single mode type ``Z''.

%% spawn --------------------------------------------------------------
\section{spawn}\index{commands!spawn}
\label{s:spawn}

The \vn{spawn} command is used to pass a command to the command shell. Format:
\begin{example}
  spawn <shell_command>
\end{example}

The users default shell is used. \vn{spawn} only works in Linux and Unix environments.

Examples:
\begin{example}
  spawn gv quick_plot.ps &      ! view a postscript file with ghostview
                                ! (and return to the TAO prompt)
  spawn tcsh                    ! launch a new tcsh shell 
                                ! (type 'exit' to return to TAO)
  spawn ls                      ! Get a directory listing.
\end{example}

%% taper --------------------------------------------------------------

\section{taper}
\index{commands!taper}
\label{s:taper}

The \vn{taper} command is used to vary magnet strengths to eliminate the transverse orbital and Twiss
changes due to the radiation damping induced ``sawtooth'' effect. Format:
\begin{example}
  taper \{-universe <ix_uni>\} \{-except <ele_list>\}
\end{example}

The sawtooth effect is the variation of the energy as a function of longitudinal position due to
radiation damping. This affects the transverse closed orbit and Twiss parameters. The \vn{taper}
command will adjust magnet strengths in the \vn{model} universe to counteract this making strengths
weaker or stronger in proportion to the local closed orbit momentum deviation.

Note: Another way of handling the sawtooth effect is to set \vn{bmad_com%radiation_zero_average} to
True. See the \bmad manual for more details.

The magnet strengths that are tapered are the ``multipole'' like parameters:
\begin{example}
  dg, k1, k2, k3, ks, hkick, vkick, kick, a1, a2, ..., b1, b2, ...
\end{example}
Notice that something like wiggler strengths are not included.

The \vn{-except} switch is used select lattice elements to not vary their magnetic strength. 

Note: Tapering must be done with Rf on.

Important! The scaling of the magnet strengths is done with respect to the \vn{base} lattice (which
is the same as the \vn{design} lattice if the \vn{set lattice base = ...} command has not been
issued). That is, if two \vn{taper} commands are issued one after another, the second \vn{taper}
command will not affect magnet strengths. The reason why this is done is so that successive
\vn{taper} commands do not cause the magnet strengths to ``walk''. That is, there is no unique
solution to the taper problem since a changing all magnet strengths along with a corresponding shift
in particle energy will not change the orbit.

The \vn{-universe <ix_uni>} switch can be used select the desired universe to taper. If not present,
the default universe will be used. Set <ix_uni> to ``*'' to choose all universes.

The taper command will vary magnet strengths independent of the element's name. That is, the magnet
strengths of elements which have the same name will be different. If a \bmad lattice file is created
after tapering using the \vn{write bmad} command, The created lattice file will have the proper
magnet strengths (the lattice file will use the \vn{\#\#N} construct to set strengths for individual
elements with the same same). However, creating a lattice file in a non-\bmad (MAD, etc.) format will
be problematical.

No tapering will be applied to strengths that are controlled by an \vn{overlay} controller element.

Examples:
\begin{example}
  taper -uni *            ! Taper all universes.
  taper -exc solenoid::*  ! Taper default uni except solenoid elements.
\end{example}

%% timer --------------------------------------------------------------

\section{timer}
\index{commands!timer}
\label{s:timer}

The \vn{timer} command is used to show computation time. Format:
\begin{example}
  timer start      ! Start (reset) the timer
  timer read       ! Display the time from the last \vn{timer start} command.    
  timer beam       ! Toggle beam timing mode on/off.
\end{example}
The timer has a \vn{beam timing} mode which can be toggled using the \vn{timer beam} command. The
initial state, when \tao is started, is for \vn{beam timing} to be off. With \vn{beam timing} mode
on, when \tao is tracking a particle beam through the lattice, \tao will print, about once a minute,
the element number and the elapsed time.

The \vn{timer start} and \vn{timer read} commands can be used to time execution
times. Example:
\begin{example}
  timer start ; call my_cmd_file ; timer read
\end{example}

Note: \vn{timer start} will toggle \vn{beam timing} off.

%% use --------------------------------------------------------------
\section{use}\index{commands!use}
\label{s:use}

The \vn{use} command un-vetoes data or variables and sets a veto for the rest of the
data. Format:
\begin{example}
  use data  <data_name>
  use var <var_name>
\end{example}

\vskip 7pt 

See also the \vn{restore} and \vn{veto} commands.

Examples:
\begin{example}
  use data orbit.x             ! use orbit.x data in the default universe.
  use data *@orbit[34]         ! use element 34 orbit data in all universes.
  use var quad_k1[67]          ! use variable.
  use var quad_k1[30:60:10]    ! use variables 30, 40, 50 and 60.
  use data *                   ! use all data in the default universe.
  use data *@*                 ! use all data in all universes.
\end{example}

%% veto --------------------------------------------------------------

\section{veto}\index{commands!veto}
\label{s:veto}

The \vn{veto} command vetoes data or variables. Format:
\begin{example}
  veto data <data_name> <locations>
  veto var <var_name> <locations>
\end{example}

\vskip 7pt 

See also the \vn{restore} and \vn{use} commands.

Examples:
\begin{example}
  veto data orbit.x[23,34:56]  ! veto orbit.x data.
  veto data *@orbit.*[34]      ! veto orbit data in all universes.
  veto var quad_k1[67]         ! veto variable
  veto var quad_k1[30:60:10]   ! veto variables 30, 40, 50 and 60
  veto data *                  ! veto all data
  veto data *[10:20]           ! veto all data from index 10 to 20 (see note)
\end{example}

Note: The command `\cmd{veto data *.*[10:20]}' will veto all \vn{d1_data} elements within the range
10:20 \textit{using the index convention for each \vn{d1_data} structure separately}. This may produce
curious results if the indexes for the \vn{d1_data} structures do not all point to the same lattice
elements.

%% wave --------------------------------------------------------------

\section{wave}\index{commands!wave}
\label{s:wave}

The \vn{wave} command sets what data is to be used for the wave analysis (\sref{c:wave}). 
Format:
\begin{example}
  wave <curve-or-data_type> \{<plot_location>\}
\end{example}

\vskip 7pt

The \vn{<curve-or-data-type>} argument specifies what plot \vn{curve} or \vn{data_type} is to be
used in the analysis. Possible \vn{<data_type>}s that can be analyzed are:
\begin{example}
  orbit.x, orbit.y
  beta.a,  beta.b
  phase.a, phase.b
  eta.x, eta.y
  cbar.11, cbar.12, cbar.22      ! Analysis not possible for cbar.21
  ping_a.amp_x, ping_a.phase_x
  ping_a.sin_y, ping_a.cos_y
  ping_a.amp_sin_y, ping_a.amp_cos_y
  ping_a.amp_sin_rel_y, ping_a.amp_cos_rel_y
  ping_b.amp_y, ping_b.phase_y
  ping_b.sin_x, ping_b.cos_x
  ping_b.amp_sin_x, ping_b.amp_cos_x
  ping_b.amp_sin_rel_x, ping_b.amp_cos_rel_x
\end{example}
If there is more than one displayed curve that has the \vn{data_type} to be analyzed, use the
\vn{curve} name instead (\sref{c:plotting}).

The \vn{<plot_location>} argument specifies the plot region where the results of the wave analysis
is to be plotted. If not present, the region defaults to the region of the plot containing the curve
used for the analysis.

Note: use the \vn{set wave} (\sref{s:set.wave}) command to set the boundaries of the fit regions.

Examples:
\begin{example}
  wave orbit.x      ! Use the orbit.x curve for the wave analysis.
  wave top.x bottom ! Use the curve in top.x and the results of the 
                    !  wave analysis are put in the bottom region.
\end{example}

%% write --------------------------------------------------------------

\section{write}\index{commands!write}
\label{s:write}

The \vn{write} command creates various files.
Format:
\begin{example}
  write beam ...                  ! \sref{s:write.beam}
  write blender ...               ! \sref{s:write.blender}
  write bmad ...                  ! \sref{s:write.bmad}
  write covariance_matrix ...     ! \sref{s:write.covar.matrix}
  write curve ...                 ! \sref{s:write.curve}
  write derivative_matrix ...     ! \sref{s:write.deriv.matrix}
  write digested ...              ! \sref{s:write.digested}
  write elegant ...               ! \sref{s:write.elegant}
  write gif ...                   ! \sref{s:write.gif}
  write hard                      ! \sref{s:write.hard}
  write mad8 ...                  ! \sref{s:write.mad8}
  write madx ...                  ! \sref{s:write.madx}
  write matrix ...                ! \sref{s:write.mat}
  write namelist ...              ! \sref{s:write.namelist}
  write opal ...                  ! \sref{s:write.opal}
  write ps ...                    ! \sref{s:write.ps}  
  write ptc ...                   ! \sref{s:write.ptc}
  write sad ...                   ! \sref{s:write.sad}
  write spin_mat8                 ! \sref{s:write.spin.mat8}
  write variable ...              ! \sref{s:write.variable}
  write xsif ...                  ! \sref{s:write.xsif}
\end{example}

% Use the command:
%   help write <subcommand>
% to obtain more information on a particular write subcommand. Example:
%   help write namelist

%% write beam --------------------------------------------------------------

\subsection{write beam}
\label{s:write.beam}

The \vn{write beam} command writes beam particle information to a file.
Syntax:
\begin{example}
    write beam \{-ascii\} \{-floor_position\} -at <element_list> \{<file_name>\} 
\end{example}

The \vn{write beam} command creates a file of beam particle positions at a given lattice element(s). The
\vn{-at} switch specifies at what elements the particle positions are written. Element list format
(\sref{s:ele.list.format}), without any embedded blanks, is used for the \vn{<element_list>}
argument to the \vn{-at} switch. 

The default, if \vn{-floor_position} is not present, is to write particle phase space positions.
If \vn{-floor_position} is present. The particle position in global coordinate space is written.
In this case, a ASCII file is always produced.

Note: Non-floor position beam files can be used to initialize \tao (\sref{s:command.line}).

If \vn{-floor_position} is not present, the default is to write a binary HDF5 file. See the \vn{Beam
Initialization} chapter in the \bmad manual for a discussion of the syntax. This default can be overridden
by using the \vn{-ascii} switch.

If \vn{-floor_position} is present, the default file name is \vn{beam_floor_\#.dat} where \vn{\#} is
replaced by the universe number. If \vn{-floor_position} is not present, the default ASCII file name
is \vn{beam_\#.dat} and the default HDF5 binary file name is \vn{beam_\#.hdf5}.

Examples:
\begin{example}
  write beam -at *        ! Output beam at every element.
  write beam -floor end   ! Output beam floor coords at element named ``end''.
\end{example}

%% write blender --------------------------------------------------------------

\subsection{write blender}
\label{s:write.blender}

The \vn{write blender} creates a script which can then be run by
the \vn{blender} program\cite{b:blender}. Syntax:
\begin{example}
    write blender \{<file_name>\}            ! Write a blender script (Same as 3d_model).
\end{example}

The default file name is \vn{blender_lat_\#.py} where \vn{\#} is replaced by the universe number. 

\vn{Blender} is a free, open source, program for creating, among other things, 3D images. This
script will create a 3D model of the lattice in the current default universe (\sref{s:universe}).
The suffix must by '.py' and if this suffix is not present it will be added. To run the script in
\vn{blender}, use the following on the operating system command line:
\begin{example}
  <path-to-blender-exe>/blender -P <script-file-from-tao>
\end{example}
To learn how to pan, zoom, etc. in \vn{blender}, consult any one of a number of online tutorials and
videos. A good place to start is:
\begin{example}
  www.blender.org/support/tutorials/  
\end{example}
Note: In order of the script to work, the script must be able to find the ``base'' file
\vn{blender_base.py}. This base file lives in the \vn{bmad/scripts} directory and the \vn{bmad}
directory is found using one of the following environmental variables:
\begin{example}
  BMAD_BASE_DIR
  DIST_BASE_DIR
  ACC_RELEASE_DIR
\end{example}
Generally, one of the latter two environmental variables will be defined.  If not, a copy of the
\bmad directory must be created and then \vn{BMAD_BASE_DIR} be appropriately defined.

%% write bmad --------------------------------------------------------------

\subsection{write bmad}
\label{s:write.bmad}

The \vn{write bmad} command will create a bmad lattice file. Syntax:
\begin{example}
    write bmad \{-format <type>\} \{<file_name>\} 
\end{example}

The default file name is \vn{lat_\#.bmad} where \vn{\#} is replaced by the universe number. 

The \vn{-format} switch is used set how field description parameters of a lattice element
are stored. The \vn{-format} switch can be set to one of:
\begin{example}
  one_file      ! One lattice file.
  ascii         ! Separate ASCII field files.
  binary        ! [Default] Separate ASCII field files with the exception that
                !   grid_field files use the HDF5 binary format.
\end{example}
Lattice elements may have associated field descriptions. There are four types as explained in the
\bmad manual:
\begin{example}
  cartesian_map
  cylindrical_map
  grid_field
  taylor_field
\end{example}
Since the data associated with these may be largish, there is the option of storing the data is
separate secondary lattice files. This is done by setting \vn{format} to either \vn{ascii} or
\vn{binary}. The difference between \vn{ascii} and \vn{binary} is that for \vn{grid_field}s, which
may have a huge amount of associated data, the \vn{binary} format stores \vn{grid_field}s using
HDF5. The other three field description types are always stored using ASCII files.

\begin{example}
  write bmad -format one lat.bmad  ! Single lattice file lat.bmad created.
\end{example}

%% write covariance_matrix --------------------------------------------------------------

\subsection{write covariance_matrix}
\label{s:write.covar.matrix}

Syntax:
\begin{example}
    write covariance_matrix \{file_name\}    ! Write the covariance and alpha matrices 
\end{example}

The default file name is \vn{covar.matrix}.

%% write curve  --------------------------------------------------------------

\subsection{write curve}
\label{s:write.curve}

The \vn{write curve} command writes plot curve data to a file.
Syntax:
\begin{example}
    write curve <curve_name> \{<file_name>\} ! Write the curve data
\end{example}

\vn{write curve} will produce two or three files:
\begin{example}
  <file_name>.symbol_dat    ! Symbol coordinates file
  <file_name>.line_dat      ! Curve coords.
  <file_name>.particle_dat  ! Particle data file
\end{example}
The particle data file is only produced if particle data is associated with the curve.  The curve
coordinates are the the set of points that are used to draw the (possibly smooth) curve through the
symbols.


%% write derivative_matrix --------------------------------------------------------------

\subsection{write derivative_matrix}
\label{s:write.deriv.matrix}

Syntax:
\begin{example}
    write derivative_matrix \{file_name\}    ! Write the \vn{dModel_Data/dVar} matrix.
\end{example}

The default file name is \vn{derivative_matrix.dat}. 

%% write digested --------------------------------------------------------------

\subsection{write digested}
\label{s:write.digested}

Syntax:
\begin{example}
    write digested \{<file_name>\}      ! Write a digested Bmad lattice file of the model.
\end{example}

The default file name is \vn{lat_\#.digested} where \vn{\#} is replaced by the universe number. 


%% write elegant --------------------------------------------------------------

\subsection{write elegant}
\label{s:write.elegant}

Syntax:
\begin{example}
    write elegant \{<file_name>\}           ! Elegant lattice file using the model lattice.
\end{example}

Write a lattice file in \vn{Elegant} format. The default file name is \vn{lat_\#.lte} where \vn{\#}
is replaced by the universe number.

%% write gif --------------------------------------------------------------

\subsection{write gif}
\label{s:write.gif}

Syntax:
\begin{example}
    write gif \{<file_name>\}           ! Create a gif file of the plot window.
\end{example}

Write a \vn{gif} file. The default file name is \vn{tao.gif}.

Note: PGPLOT, if being used, does a poor job producing gif files so consider making a
postscript file instead and using a ps to gif converter.

%% write hard --------------------------------------------------------------

\subsection{write hard}
\label{s:write.hard}

Syntax:
\begin{example}
    write hard                        ! Print the plot window to a printer.
\end{example}

%% write mad8 --------------------------------------------------------------

\subsection{write mad8}
\label{s:write.mad8}

Syntax:
\begin{example}
    write mad8 \{<file_name>\}  ! Write a MAD-8 lattice file of the model
\end{example}

The default file name is \vn{lat_\#.mad8} where \vn{\#} is replaced by the universe number. 

%% write madx --------------------------------------------------------------

\subsection{write madx}
\label{s:write.madx}

Syntax:
\begin{example}
    write madx \{<file_name>\}  ! Write a MAD-X lattice file of the model
\end{example}

The default file name is \vn{lat_\#.madx} where \vn{\#} is replaced by the universe number. 

%% write matrix --------------------------------------------------------------

\subsection{write matrix}
\label{s:write.mat}

Syntax:
\begin{example}
    write madx \{-branch <branch>\} \{-single\} \{-from_start\} \{-combined\}
               \{-universe <ix_uni>\} \{<file_name>\}  ! Write transport matrices
\end{example}

The \vn{write matrix} command writes transport matrices to a file for a particular lattice branch
determined by the \vn{-universe} and \vn{-branch} switches. The default is the current viewed
universe and lattice branch 0.

If \vn{<file_name>} is not present, the default file name to write to is \vn{matrix.dat}.

What is written is determined by the \vn{-single}, \vn{-from_start}, and \vn{-combined}
switches. If \vn{-single} is present, the transfer matrices through each lattice element is recorded.
If \vn{-from_start} is present, the transfer matrices from the start of the branch to each element
is written. If \vn{-combined} is present, both the element matrices and the matrices from the start
of the branch are written. The default is to write the element matrices.

Examples:
\begin{example}
  write mat -uni 3 -br 1  ! Write matrices from universe 3, branch 1.
  write mat -from m.dat   ! Write matrices in m.dat from branch beginning to elements
\end{example}

%% write namelist --------------------------------------------------------------

\subsection{write namelist}
\label{s:write.namelist}

Syntax:
\begin{example}
    write namelist \{-append\} \{-data\} \{-plot\} \{-variable\} \{file_name\}
\end{example}

The default file name is \vn{tao.namelist}.

%% write opal --------------------------------------------------------------

\subsection{write opal}
\label{s:write.opal}

Syntax:
\begin{example}
    write opal \{<file_name>\}  ! Write a OPAL lattice file of the model
\end{example}

The default file name is \vn{lat_\#.opal} where \vn{\#} is replaced by the universe number. 

%% write ps --------------------------------------------------------------

\subsection{write ps}
\label{s:write.ps}

The \vn{write ps} command produces postscript output. Syntax:
\begin{example}
  write ps \{-scale <scale>\} \{<file_name>\}
\end{example}

The default file name is \vn{tao.ps}. 

The optional \vn{-scale} switch sets the scale for the postscript file. A value of 1.0 (the
default) will result in no scaling, 2.0 will double the size, etc.

%% write ptc --------------------------------------------------------------

\subsection{write ptc}
\label{s:write.ptc}

Syntax:
\begin{example}
  write ptc \{-all\} \{-old\} \{-branch <name_or_index\} \{<file_name>\}
\end{example}

The default file name is \vn{ptc.flatfile}

The \vn{write ptc} command creates PTC lattice files (called ``flat'' files).  If the \vn{-all}
switch is present, there will be two main flat files generated. The \vn{-all} switch needs to be
used when there are multiple lattice branches that need to be translated to PTC. For example, in a
dual colliding ring machine with two storage rings. Both \vn{M_u} and \vn{M_t} mad_universe
structures will be generated. The two main files generated will have the suffixes \vn{.m_u} and
\vn{.m_t} appended to the file names. In this case, the setting of \vn{-branch} is ignored.

If \vn{-all} is not present, only one main flat file is generated. In this case, if \vn{-old} is
present, the flat file generated will be of the ``old'' syntax. Generally there is no reason to
generate old style flat files. When generating a single flat file (no \vn{-all} switch present), the
flat file will contain the information for a single lattice branch. The lattice branch used can be
specified by the \vn{-branch} switch. The default, if \vn{-branch} is not present, is to use lattice
branch \# 0. The \vn{-old} switch will generate an ``old style'' version.

In all cases, the \vn{write ptc} command can only be used after a \vn{ptc init} command has been
used to setup PTC.

%% write sad --------------------------------------------------------------

\subsection{write sad}
\label{s:write.sad}

Syntax:
\begin{example}
  write sad \{<file_name>\}  ! Write a SAD lattice file of the model
\end{example}

The default file name is \vn{lat_\#.sad} where \vn{\#} is replaced by the universe number. 

%% write spin_mat8 --------------------------------------------------------------

\subsection{write spin_mat8}
\label{s:write.spin.mat8}

The \vn{write spin_mat8} writes the 8x8 matrices spin/orbit transport matrices, element-by-element, for
a given branch of the model lattice of the default universe.
Syntax:
\begin{example}
  write spin_mat8 -l_axis <lx> <ly> <lz> \{-branch <name_or_index>\} \{<file_name>\}
\end{example}
See the \bmad manual for details on how the 8x8 spin/orbit matrices are defined. 

The default file name if \vn{<file_name>} is not present is \vn{spin_mat8.dat}.

The computation starts at the beginning of the lattice. The $n_0$-axis is computed by \bmad. The
$l_0$-axis must be given in the \vn{write spin_mat8} command.  The $m_0$-axis will be computed so
that $(l_0, n_0, m_0)$ form a right handed coordinate system.

Example:
\begin{example}
  write spin -l 1 0 0    ! l-axis is (1, 0, 0).
\end{example}

%% write variable --------------------------------------------------------------

\subsection{write variable}
\label{s:write.variable}

The \vn{write variable} command writes \tao variable values to a file or files.
Syntax:
\begin{example}
  write variable \{-good_var_only\} \{-tao_format\} \{<file_name>\} 
\end{example}
This is useful, for example, for recording changes when a lattice is optimized.

If the \vn{-tao_format} switch is absent, the output is a list of lines with each line of the form:
\begin{example}
  slave_ele[slave_param] = value
\end{example}
where \vn{slave_param} is a parameter of \bmad element \vn{slave_ele} that is controlled by a \tao
variable. And \vn{value} is the value of that parameter. For example, if \vn{quad_rot[14]} is a \tao
variable that controls the \vn{tilt} parameter of \bmad lattice element \vn{q_arc_12}, the output
will contain a line like:
\begin{example}
  q_arc12[tilt] = 0.23465e-4
\end{example}
This output can be used construct a lattice with optimized values. For example, if the output file
name of the \vn{write variable} command is called \vn{var1.out}, a three line lattice file with the
optimized values can be constructed which looks like:
\begin{example}
  call, file = original_lattice.bmad
  expand_lattice
  call, file = var1.out
\end{example}
where ``\vn{original_lattice.bmad}'' is the name of the original unoptimized lattice file. The
\vn{expand_lattice} command may not be needed if the controlled \bmad elements have unique names.

If the \vn{tao_format} switch is present, the output is a set of lines of the form:
\begin{example}
  set tao_var[index]|model = value
\end{example}
where \vn{tao_var[index]} designates a \tao variable which has the given \vn{model} value. For
example, if \vn{quad_rot[14]} is a \tao variable, the output
will contain a line like:
\begin{example}
  set quad_rot[14]|model = 0.23465e-4
\end{example}
With this format, the output of \vn{write variable} can be read back into \tao using a \vn{call}
command.

Note: after the \bmad or \tao compatible set lines, there will be an \vn{end_file} command (so \bmad
or \tao will ignore the rest of the file), and following this will be information on the
optimization state.

When there are multiple universes, and if \vn{-tao_format} is not present, the \vn{write variable}
command writes a number of files, one for each universe.

The default file name is \vn{var\#.out} where \vn{\#} is replaced by the universe number. 

If the optional \vn{-good_var_only} switch is present, only the information on variables that are
currently used in the optimization is written.

%% write madx --------------------------------------------------------------

\subsection{write xsif}
\label{s:write.xsif}

Syntax:
\begin{example}
    write xsif \{<file_name>\}  ! Write an XSIF lattice file of the model
\end{example}

The default file name is \vn{lat_\#.xsif} where \vn{\#} is replaced by the universe number. XSIF is
a version of MAD-8 customized by SLAC to handle \vn{lcavity} elements.

%% x_axis --------------------------------------------------------------

\section{x_axis}\index{commands!x_axis}
\label{s:x.axis}

The \vn{x_axis} command sets the data type used for the x-axis coordinate. Format:
\begin{example}
  x_axis <where> <axis_type>
\end{example}

\vskip 7pt 

The \vn{x_axis} command sets the \vn{plot%x_axis_type}. This determines what data is used
for the horizontal axis. Possibilities for \vn{<axis_type>} are:
\begin{example}
  index     -- Use data index
  ele_index -- Use data element index
  s         -- Use longitudinal position.
\end{example}
Note that \vn{index} only makes sense for data that has an index associated with it.

Examples:
\begin{example}
  x_axis * s
  x_axis top index
\end{example}

%% x_scale --------------------------------------------------------------
\section{x_scale}\index{commands!x_scale}
\label{s:x.scale}

The \vn{x_scale} command scales the horizontal axis of a graph or set of graphs. Format:
\begin{example}
  x_scale \{-exact\} \{-gang\} \{-nogang\} \{-include_wall\} \{<where> \{<value1> <value2>\}\}
\end{example}

Which graphs are scaled is determined by the \vn{<where>} switch. If \vn{<where>} is not present or
\vn{<where>} is \vn{*} then all graphs are scaled. \vn{<where>} can be a plot name or the name of an
individual graph withing a plot. If \vn{<where>} is \vn{s} then the scaling is done only for the
plots where the x-axis scale is the longitudinal s-position.

\vn{x_scale} sets the lower and upper bounds for the horizontal axis.  If \vn{<bound1>} and
\vn{<bound2>} are present, \vn{<bound1>} is taken to be the lower (left) bound and \vn{<bound2>} is
the upper (right) bound. If neither is present, an \vn{autoscale} will be invoked to give the
largest bounds commensurate with the data. If an autoscale is performed upon an entire plot. In the
case where there is an autoscale, if \vn{plot%autoscale_gang_x} (\sref{s:template}) is True, then
the chosen scales will be the same for all graphs. That is, a single scale is calculated so that all
the data of all the graphs is within the plot region. The affect of \vn{plot%autoscale_gang_x} can
be overridden by using the \vn{-gang} or \vn{-nogang} switches.

How a graph is scaled is determined in part by the setting of the \vn{bounds} parameter in the
\vn{x} parameter of the graph. See \vn{s:quick.plot} for more details. The \vn{-exact} switch, if
present, will set \vn{bounds} to \vn{"EXACT"}. which means that \tao will use the min and
max bounds as given by \vn{<value1>} and \vn{<value2>} and not try to find ``nice'' values near the
given ones. If \vn{<value1>} and \vn{<value2>} are not given, and if \vn{bounds} is set to
\vn{"EXACT"}, \tao will set \vn{bounds} to \vn{"GENERAL"}. Note: To set the axis \vn{bounds}
directly, use the \vn{set graph} command.

Note: The \vn{x_scale} command will vary the number of major divisions (set by
\vn{graph%x%major_divisions} (\sref{s:template})) to try to give a nice looking axis. The result
can be that if two plots have the same range of data but differing major division settings, the
\vn{x_scale} command can produce differing results.

For scaling \vn{floor_plan} plots where there is a building wall to be drawn, if \vn{-include_wall}
is present and autoscaling is being done, then the plot bounds are extended to include the extent of
the building wall.

Example:
\begin{example}
  x_scale -include      ! Autoscale all x-axes and include the extent of any 
                        !   building walls in the calculation of the plot bounds.
  x_scale * 0 100       ! Scale all x-axes to go from 0 to 100.
  x_scale orbit -10 10  ! This "wraps around" the beginning of the lattice.
  x_scale s             ! Scale all graphs using x_axis = "s".
\end{example}

%% xy_scale --------------------------------------------------------------

\section{xy_scale}\index{commands!xy_scale}
\label{s:xy.scale}

The \vn{xy_scale} command sets horizontal and vertical axis bounds. Format:
\begin{example}
  xy_scale \{-include_wall\} \{<where> \{<bound1> <bound2>\}\}\}
\end{example}

\vn{xy_scale} is equivalent to an \vn{x_scale} followed by a \vn{y-scale}.

Which graphs are scaled is determined by the \vn{<where>} switch. If \vn{<where>} is not present or
\vn{<where>} is \vn{*} then all graphs are scaled. \vn{<where>} can be a plot name or the name of an
individual graph withing a plot.

\vn{xy_scale} sets the lower and upper bounds for both the horizontal and vertical axes.  This is
just a shortcut for doing an \vn{x_scale} followed by a \vn{scale}.  If both \vn{<bound1>} and
\vn{<bound2>} are present then \vn{<bound1>} is taken to be the lower (left) bound and \vn{<bound2>}
is the upper (right) bound. If only \vn{<bound1>} is present then the bounds will be from
-\vn{<bound1>} to \vn{<bound1>}.

If neither \{<bound1>\} nor \{<bound2>\} is present then an \vn{autoscale} will be invoked to give
the largest bounds commensurate with the data.

For scaling \vn{floor_plan} plots where there is a building wall to be drawn, if \vn{-include_wall}
is present and autoscaling is being done, then the plot bounds are extended to include the extent of
the building wall.

Example:
\begin{example}
  xy_scale -include ! Autoscale all axes and include the extent of any 
                    !   building walls in the calculation of the plot bounds.
  xy_scale * -1 1   ! Scale all axes to go from -1 to 1.
\end{example}

