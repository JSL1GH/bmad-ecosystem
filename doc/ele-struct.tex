\chapter{The Ele_struct}
\label{c:ele.struct}
\index{ele_struct|hyperbf}

This chapter describes the \vn{ele_struct} which is the structure that
holds all the information about an individual element: quadrupoles,
separators, wigglers, etc. 
Part of the substructure of the \vn{ele_struct} is shown
in figure~\ref{f:ele.struct} (use \vn{getf} to see the entire structure 
definition). This structure is somewhat
complicated, however, in practice, a lot of the complexity is
generally hidden  by the \bmad bookkeeping routines.

\begin{figure}[htb]
\centering
\small
\begin{verbatim}
  type ele_struct
    character(16) name                ! name of element
    character(16) type                ! type name
    character(16) alias               ! Another name
    character(200), pointer :: descrip => null()     ! For general use
    type (twiss_struct)  a, b, z      ! Twiss parameters at end of element
    type (xy_disp_struct) x, y        ! Projected dispersion
    type (floor_position_struct) position
    real(rp) value(n_attrib_maxx)     ! attribute values
    real(rp) ref_orb_start(6)         ! Reference orbit for mat6 calc at entrance of element.
    real(rp) ref_orb_end(6)           ! Reference orbit at exit of element.
    real(rp) vec0(6)                  ! 0th order transport vector
    real(rp) mat6(6,6)                ! 1st order transport matrix
    real(rp) c_mat(2,2)               ! 2x2 C coupling matrix
    real(rp) gamma_c                  ! gamma associated with C matrix
    real(rp) s                        ! longitudinal position at the end
    real(rp), pointer :: a_pole(:) => null()  ! skew multipole component
    real(rp), pointer :: b_pole(:) => null()  ! normal multipole component
    type (wake_struct), pointer :: wake => null() ! Wake fields
    type (taylor_struct) :: taylor(6)             ! Taylor terms
    integer key                    ! key value
    integer sub_key                ! For wigglers: map_type$, periodic_type$
    integer control_type           ! SUPER_SLAVE$, OVERLAY_LORD$, etc.
    integer mat6_calc_method       ! bmad_standard$, taylor$, etc.
    integer tracking_method        ! bmad_standard$, taylor$, etc.
    integer field_calc             ! Used with integrators (Runge-Kutta et. al)
    integer num_steps              ! number of slices for DA_maps
    integer integrator_order       ! For Etiennes' PTC: 2, 4, or 6.
    logical symplectify            ! Symplectify mat6 matrices.
    logical exact_rad_int_calc     ! Exact radiation integral calculation?
    logical map_with_offsets       ! Taylor map calculated with element offsets?
    logical field_master           ! Calculate strength from the field value?
    logical is_on                  ! For turning element on/off.
    logical on_a_girder            ! Have an Girder lord?
    logical csr_calc_on            ! Coherent synchrotron radiation calculation
    logical offset_moves_aperture  ! element offsets affects aperture?
  end type
\end{verbatim}
\caption[The \vn{ele_struct}.]{The \vn{ele_struct}. Only part of the 
  substructure is shown.}
\label{f:ele.struct}
\end{figure}

%--------------------------------------------------------------------------
\section{String components}

\index{ele_struct!\%descrip}\index{ele_struct!\%alias}
\index{ele_struct!\%type}\index{ele_struct!\%name}
\vn{%name}, \vn{%type}, \vn{%alias}, and \vn{%descrip} are character arrays
within an \vn{ele_struct}. All of these have a direct correspondence with
the \vn{name}, \vn{type}, \vn{alias}, and \vn{descrip} element attributes in
an input lattice file. The \vnr{bmad_parser} routine will convert 
\vn{name}, \vn{type}, and \vn{alias} attributes to uppercase before loading them
into an \vn{ele_struct}. \bmad generally ignores these character arrays. 
Notice that \vn{%descrip} is a pointer so if you make use of it in a program 
make sure that it is allocated before hand.

%--------------------------------------------------------------------------
\section{Initialization and Element Key}
\index{ele_struct!initialization}

\index{ele_struct!\%key}
The \vn{%key} integer component gives the type of element
(\vn{Quadrupole}, \vn{RFcavity}, etc.). In general to get the
corresponding integer parameter for an element class just add a ``\$''
character to the type name. For example \vn{quadrupole\$} is the integer
parameter for \vn{quadrupole} elements. The \vn{key_name} array converts from
integer to the appropriate string. For example
\begin{example}
  type (ele_struct) ele
  call init_ele (ele)                  ! Initialize
  if (ele%key == wiggler\$) then       ! Test if element is a wiggler.
  print *, 'This element: ', key_name(ele%key) ! Prints, for example, 'WIGGLER'
\end{example}
The call to \vn{init_ele} is needed for any \vn{ele_struct} defined
outside of a \vn{lat_struct} structure.

\index{wiggler}
\index{rbend}
\index{sbend}\index{ele_struct!\%sub_key}
The \vn{%sub_key} component is only used for \vn{Wiggler} and \vn{Rbend} and
\vn{Sbend} elements. For \vn{Wiggler} elements, \vn{%sub_key} is either set to
\begin{example}
  map_type$ or
  periodic_type$
\end{example}
depending upon the type of wiggler. For bends the parser routines
turn all \vn{Sbend}s into \vn{Rbend}s. To keep track of what they
were in the input lattice file, the \vn{%sub_key} is set to
\begin{example}
  sbend$ or
  rbend$
\end{example}
whatever is appropriate.

%--------------------------------------------------------------------------
\section{Twiss Parameters, etc.}
\label{s:ele.twiss}
\index{ele_struct!Twiss}
\index{ele_struct!global coordinates}
\index{Coordinates!global!in ele_struct}

For variables like the Twiss parameters that vary along the length of an
element the rule is that the value stored in the \vn{ele_struct} is
the value at the exit end of the element. Such variables are:
\index{ele_struct!\%s}
\index{ele_struct!\%gamma_c}
\index{ele_struct!\%c_mat}
\index{ele_struct!\%a}
\begin{example}
  type (twiss_struct)  a, b, z          ! Twiss parameters
  type (xy_disp_struct) x, y            ! Projected dispersion
  type (floor_position_struct) position ! Floor position
  real(rp) c_mat(2,2)                   ! Coupling c matrix
  real(rp) gamma_c                      ! Coupling parameter
  real(rp) s                            ! Longitudinal position
\end{example}
To get the Twiss parameters, floor (global) position, etc.\ for the beginning
of the element you need to look at the preceding element in the
\vn{lat%ele(:)} array 
\begin{example}
  lat%ele(i-1)%a%beta  ! Beta_a at beginning of ith element.
\end{example}
To get the parameters at a position within an element you can use the
routines \vn{twiss_and_track_at_s} or \vn{twiss_and_track_partial}.

\index{ele_struct!\%z}\index{ele_struct!\%b}\index{ele_struct!\%a}
\index{normal mode!in Ele_struct}
The \vn{%a}, \vn{%b}, and \vn{%z} components are
\vn{twiss_struct} structures that hold the Twiss parameters for the
$a$, $b$ and $z$ modes respectively. The definition of
the \vn{twiss_struct} structure is
\index{ele_struct!twiss_struct substruct}
\begin{example}
  type twiss_struct
    real(rp) beta, alpha, gamma, phi, eta, etap
    real(rp) sigma, emit
  end type 
\end{example} 
\vn{eta} and \vn{etap} are also the normal mode dispersion and
dispersion derivative. The projected horizontal and vertical dispersions
are given in the \vn{xy_disp_struct}
\begin{example}
  type xy_disp_struct
    real(rp) eta, etap
  end type 
\end{example} 
The relationship between the projected and normal mode dispersions are
given in \Eq{avx}. When there is no coupling (\vn{%c_mat} = 0) the
normal mode dispersions are the same as the projected horizontal and
vertical dispersions.

\vn{%c_mat} is the coupling \vn{C} matrix given by \Eq{vgicc1} and
\vn{%gamma_c} is the $\gamma$ factor in \Eq{vgicc1}. There are several
routines to manipulate the coupling factors. For example:
\begin{example}
  C_to_Cbar(ele, cbar_mat)             ! Form Cbar(2,2) matrix
  Make_V_Mats(ele, v_mat, v_inv_mat)   ! Form V matrices.
\end{example}
See \sref{r:mat} for a complete listing of such routines.

If the Twiss parameters need to be transfered from 
one element to another The three components \vn{%a}, \vn{%b}, and \vn{%c}
need to be copied. Additionally, to transfer the dispersion, the two
components \vn{%x} and \vn{%y} need to be copied since the standard
dispersion propagation routine, \vn{twiss_propagate1}, computes normal mode
dispersions from the projected ones. Thus a full transfer looks like
\begin{example}
  ele2%a = ele1%a
  ele2%b = ele1%b
  ele2%x = ele1%x
  ele2%y = ele1%y
  ele2%z = ele1%z
\end{example}

%--------------------------------------------------------------------------
\section{Attribute Values: Dependent and Otherwise}
\label{s:ele.dep}
\index{ele_struct!attribute values}

\index{ele_struct!\%value(:)}
Most of the real valued attributes of an element are held in the
\vn{%value(:)} array. For example, the value of the \vn{k1} attribute
for a quadrupole element is stored in \vn{%value(k1\$)} where
\vn{k1\$} is an integer parameter. In general to get the correct index
in \vn{%value(:)} for a given attribute just add a ``\$" as a
suffix. To convert from an attribute name to its index in the
\vn{%value} array use the \vn{attribute_index} routine.  To go back
from an index in the \vn{%value} array to a name use the
\vn{attribute_name} routine. Example:
\begin{example}
  type (ele_struct) ele
  call init_ele (ele)    ! Initialize element
  ele%key = quadrupole$  ! Set element to be a quadrupole
  ele%value(k1$) = 0.3                                        ! Set K1 value
  print *, 'Index for Quad K1:  ', attribute_index(ele, 'K1') ! prints: `4' (= k1\$)
  print *, 'Name for Quad k1\$: ', attribute_name (ele, k1\$)    ! prints: `K1' 
\end{example}
The list of attributes for a given element
type is given in the writeup for the different element in
Chapter~\ref{c:elements}. 

There are also 5 slots in the \vn{%value(:)} array for general use.
they have indexes labeled \vn{general1\$} through \vn{general5\$}. These
slots are not used by \bmad so a program can take advantage of them.
The index names can be redefined to fit a particular need. For example, suppose
a program needs to store a time stamp number. The code to do this could look
like:
\begin{example}
  integer, parameter :: time_stamp\$ = general1\$
  ...
  lat%ele(i)%value(time_stamp$) = ...
\end{example}

\index{ele_struct!\%b_pole(:)}\index{ele_struct!\%a_pole(:)}
\index{ele_struct!\%value(:)}
Certain real valued attributes are {\em not}
found in the \vn{%value(:)} array. For example, multipole components
are stored in the \vn{%a_pole(:)} and \vn{%b_pole(:)} arrays. The full list
of real attributes not in \vn{%value(:)} is:
\begin{center}
\begin{tabular}{|l|l|} \hline
  {\em Attribute} & {\em Component in Ele_struct} \\ \hline
  a$n$, b$n$      & \%a_pole(:), \%b_pole(:) \\ \hline
  K$n$L, T$n$     & \%a_pole(:), \%b_pole(:) \\ \hline
  Taylor term     & \%taylor(:)              \\ \hline
  wakes           & \%wake                   \\ \hline
  wiggler terms   & \%wig_term(:)            \\ \hline
\end{tabular}
\end{center}

\vn{attribute_bookkeeper} is the routine that makes sure that the
dependent variables (See \sref{s:depend}) of an element are keep
up--to--date. \vn{attribute_bookkeeper} is called behind the scenes
when \vnf{make_mat6} is called to make transfer matrices and when
\vnf{track1} is called to track through an element. As a general rule,
to be safe, when element attributes are changed, call
\vnf{attribute_bookkeeper}. If the element is part of a
\vn{Lat_struct} (as it is in most cases), the routine
\vnf{control_bookkeeper} (which calls \vn{attribute_bookkeeper})
should be called instead.  The function \vnr{attribute_free} will
check if an attribute is free to vary.
\begin{example}
  type (lat_struct) lat
  integer ix_ele
  ...
  call element_locator ('Q10W', lat, ix_ele)   ! look for a quadrupole
  free = attribute_free (lat%ele(ix_ele), k1$, lat, .false.)
  if (.not. free) print *, 'Cannot vary: Q10W k1'
\end{example}
\vn{%field_master} is the logical within appropriate elements that
sets whether it is the normalized strength or field strength that is
the independent variable. See \sref{s:depend} for more details.

In designing a program that will let a user input what
attribute to vary, 

With user input the routine \vn{pointer_to_attribute} is a convenient
way to obtain from an input string a pointer that points to the
appropriate attribute. For example:
\begin{example}
  type (lat_struct) lat
  character(16) attrib_name, ele_name
  real(rp), pointer :: attrib_ptr
  real(rp) set_value
  logical err_flag
  integer ix_attrib, ie
  ...
  write (*, '(a)', advance = 'no') ' Name of element to vary: '
  accept '(a)', ele_name
  write (*, '(a)', advance = 'no') ' Name of attribute to vary: '
  accept '(a)', attrib_name
  write (*, '(a)', advance = 'no') ' Value to set attribute at: '
  accept *, set_value
  do ie = 1, lat%n_ele_max
    if (lat%ele(ie)%name == ele_name) then
      call pointer_to_attribute (lat%ele(ie), attrib_name, &
                            .false., attrib_ptr, ix_attrib, err_flag)
      if (err_flag) exit      ! Do nothing on an error
      attrib_ptr = set_value  ! Set the attribute
    endif
  enddo
\end{example}

%--------------------------------------------------------------------------
\section{Control_type}
\index{control_type}
\label{s:control.type}

In a \vn{lat_struct} (\sref{c:lat.struct}), elements in the lattice
can control the attributes of other elements. The \vn{%control_type}
component of an element identifies whether the element is a slave
element (it is being controlled or is a lord element (it controls
other elements). Possible values for \vn{%control_type} are:
\index{free}\index{super_lord}
\index{group_lord}\index{girder_lord}
\index{multipass_lord}\index{super_slave}
\index{overlay_slave}\index{overlay_lord}
\index{multipass_slave}
\begin{example}
  free\$
  super_lord\$
  group_lord\$
  girder_lord\$
  multipass_lord\$
  overlay_lord\$
  multipass_slave\$
  super_slave\$
  overlay_slave\$
\end{example}

\begin{description}
\item[free element]\Newline 
A \vn{free} element is just that, An element that neither
controls or is controlled. The exception is that attributes of this element
may be controlled by a \vn{group_lord}.

\item[super_lord and super_slave]\Newline 
\vn{super_lord} and \vn{super_slave} elements
come from superimposing elements on top of other elements (\sref{s:super}).

\item[overlay_lord, overlay_slave]\Newline 
An \vn{overlay_lord} is an \vn{overlay} element. If an element it
controls would otherwise be a \vn{free} element, that element becomes
and \vn{overlay_slave} element (\sref{s:overlay}). The
\vn{control_type} of a lord element, such as a \vn{super_lord} is not
affected by being controlled by an \vn{overlay} lord.

\item[group_lord]\Newline 
A \vn{group_lord} is a \vn{group} element (\sref{s:group}). Since a
\vn{group_lord} element only modifies the attributes of the elements
it is linked to when the \vn{command} attribute of the \vn{group_lord}
is modified, there is no corresponding \vn{group_slave} designation
for the corresponding ``slave'' elements.

\item[girder_lord]\Newline 
\index{on_a_girder}
A \vn{girder_lord} is a \vn{girder} element. In this case, the slaves
are marked by setting the \vn{%on_a_girder} component in the
\vn{ele_struct}.

\item[multipass_lord, multipass_slave]\Newline
\vn{multipass_lord}s and \vn{multipass_slave}s are created when
multipass lines are present (\sref{s:multipass}). In the example in
\sref{s:multipass}, Element \vn{A} would be a \vn{multipass_lord} and
\vn{A\B1} and \vn{A\B2} would be \vn{multipass_slave}s. When
superposition is combined with multipass, the elements in the tracking
part of the lattice will be \vn{super_slave}s.  These elements will be
controlled by \vn{super_lord}s and the \vn{super_lord}s will
be controlled by \vn{multipass_lord}s.

\end{description}

%--------------------------------------------------------------------------
\section{Positional Offsets}
\index{x_offset}\index{y_offset}\index{x_pitch}\index{y_pitch}
\index{tilt}\index{x_offset_tot}\index{y_offset_tot}\index{x_pitch_tot}
\index{y_pitch_tot}\index{tilt_tot}

The positional offsets for an element are stored in:
\begin{example}
  %value(x_offset\$)
  %value(y_offset\$)
  %value(x_pitch\$)
  %value(y_pitch\$)
  %value(tilt\$)
\end{example}
\index{girder}
If the element is supported by an \vn{Girder} then the \vn{Girder}
offsets are added to the element offsets 
and the total offset with respect to the
reference coordinate system is stored in:
\begin{example}
  %value(x_offset_tot\$)
  %value(y_offset_tot\$)
  %value(x_pitch_tot\$)
  %value(y_pitch_tot\$)
  %value(tilt_tot\$)
\end{example}
If there is no \vn{Girder}, the values for \vn{%value(x_offset_tot\$)}, etc.
are set to the corresponding values in \vn{%value(x_offset\$)}, etc.
Thus, to vary the position of an individual
element the values of \vn{%value(x_offset\$)}, etc. are changed and to
read the position of an element a program should look at
\vn{%value(x_offset_tot\$)}, etc.

%--------------------------------------------------------------------------
\section{Transfer Maps}
\index{ele_struct!transfer maps}
\index{transfer map!in ele_struct}
\index{mat6}
\index{vec0}
\index{ref_orb}

The routine \vn{make_mat6} computes the linear transfer matrix (Jacobian). 
This matrix is stored in \vn{%mat6(6,6)}. The reference orbit at
the entrance end of the element about
which the transfer matrix is computed is stored in \vn{%ref_orb_start}
and the the reference orbit at the exit end is stored in \vn{%ref_orb_end}.
Finally, \vn{make_mat6} stores in \vn{%vec0} a vector such that
\begin{example}
  ref_orb_end = %mat6 * ref_orbit_start + %vec0
\end{example}
Notice that \vn{%vec0} only corresponds to the zeroth
order part of the transport map when the transport map is linear.
The reason redundant information is stored in the element is to save
computation time.

To compute the transfer maps for an entire lattice use the routine \vn{lat_make_mat6}.

%--------------------------------------------------------------------------
\section{Taylor Maps}
\index{ele_struct!Taylor maps}
\index{taylor map!structure in ele_struct}

\vn{taylor_order} is the order of the Taylor map
(see~\sref{s:taylor.phys}). The map itself is stored in
\vn{%taylor(1:6)}. Each \vn{%taylor(i)} is a \vn{taylor_struct}
structure that defines a Taylor series. The structure is defined as
\begin{example}
  type taylor_struct
    real (rp) ref
    type (taylor_term_struct), pointer :: term(:) => null()
  end type
\end{example}
Each Taylor series is an array of \vn{taylor_term_struct} term defined as
\begin{example}
  type taylor_term_struct
    real(rp) :: coef
    integer :: exp(6)
  end type
\end{example}

To see if there is a Taylor map associated with an element the
association status of \vn{%taylor(1)%term} needs to be checked.
As an example the following finds the order of a Taylor map.
\begin{example}
  type (ele_struct) ele
  ...
  if (associated(ele%taylor(1)%term) then  ! Taylor map exists
    taylor_order = 0
    do i = 1, 6
      do j = 1, size(ele%taylor(i)%term)
        taylor_order = max(taylor_order, sum(ele%taylor(i)%term(j)%exp)
      enddo
    enddo
  else  ! Taylor map does not exist
    taylor_order = -1  ! flag non-existence
  endif
\end{example}

The Taylor map is made up around some reference phase space point
corresponding to the coordinates at the entrance of the element.
This reference point is saved in \vn{%taylor(1:6)%ref}.  Once a Taylor map is
made the reference point is not needed in subsequent
calculations. However, The Taylor map itself will depend upon what
reference point is chosen (\sref{s:taylor.phys}).

%--------------------------------------------------------------------------
\section {Wake fields}
\index{wake fields!in ele_struct}

\index{ele_struct!\%wake}
See \sref{s:wake fields} for the equations used in wake field
calculations.  Wake fields are stored in the \vn{%wake} struct:
\begin{example}
  type wake_struct
    character(200) :: sr_file = ' '
    character(200) :: lr_file = ' '
    type (sr_table_wake_struct), pointer :: sr_table(0:) => null()
    type (sr_mode_wake_struct), pointer :: sr_mode_long(:) => null()
    type (sr_mode_wake_struct), pointer :: sr_mode_trans(:) => null()
    type (lr_wake_struct), pointer :: lr(:) => null()
    real(rp) :: z_sr_mode_max = 0   
  end type
\end{example}
Since \vn{%wake} is a pointer its association status must be tested
before any of its sub--components are accessed.
\begin{example}
  type (ele_struct) ele
  ...
  if (associated(ele%wake)) then
    ...
\end{example}

\bmad observes the following rule: If \vn{%wake} is associated then it
is assumed that all the sub--components (\vn{%wake%sr_table}, etc.) are
associated. This simplifies programming in that you do not have to
test directly the association status of the sub--components.

The short--range wake can be parameterized in either of two ways.
One parameterization uses a table of wake verses z position. 
If this parameterization utilizes the \vn{%wake%sr_table(0:)} array
The structure of each element in this array is:
\begin{example}
  type sr_table_wake_struct  ! Tabular short-Range Wake struct
    real(rp) z               ! Distance behind the leading particle
    real(rp) long            ! Longitudinal wake in V/C/m
    real(rp) trans           ! Transverse wake in V/C/m^2
  end type
\end{example}
All \vn{%wake%sr_table(0:)%z} must be negative except \vn{%wake%sr_table(0)%z} 
= 0. Wake field kicks are applied using \Eqs{delvp} and \eq{pelqxw}.

The alternative short--range wake parameterization uses pseudo--modes (\Eq{wadzk}).
This parameterization utilizes the \vn{%wake%sr_mode_long}, and \vn{%wake%sr_mode_trans}
arrays for the longitudinal and transverse modes respectively.
The structure used for the elements of these arrays are:
\begin{example}
  type sr_mode_wake_struct  ! Pseudo-mode short-range wake struct 
    real(rp) amp        ! Amplitude
    real(rp) damp       ! Damping factor.
    real(rp) freq       ! Frequency in Hz
    real(rp) phi        ! Phase in radians/2pi
    real(rp) norm_sin   ! non-skew sin-like component of the wake
    real(rp) norm_cos   ! non-skew cos-like component of the wake
    real(rp) skew_sin   ! skew sin-like component of the wake
    real(rp) skew_cos   ! skew cos-like component of the wake
  end type
\end{example}
The wake field kick is calculated from \Eq{wadzk}. \vn{%amp},
\vn{%damp}, \vn{%freq}, and \vn{%phi} are the input parameters from
the lattice file. the last four components (\vn{%norm_sin}, etc.)
store the accumulated wake: Before the bunch passes through these are
set to zero and as each particle passes through the cavity the
contribution to the wake due to the particle is calculated and added
the components.

\vn{%wake%z_sr_mode_max} is the maximum $z$ value beyond which the pseudo
mode representation is not valid. This is set in the input lattice file.

The \vn{%wake%lr} array stores the long--range wake modes. The
structure definition is:
\begin{example}
  type lr_wake_struct   ! Long-Range Wake struct 
    real(rp) freq       ! Actual Frequency in Hz
    real(rp) freq_in    ! Input frequency in Hz
    real(rp) R_over_Q   ! Strength in V/C/m^2
    real(rp) Q          ! Quality factor
    real(rp) angle      ! polarization angle (radians/2pi).
    integer m           ! Order (1 = dipole, 2 = quad, etc.)
    real(rp) norm_sin   ! non-skew sin-like component of the wake
    real(rp) norm_cos   ! non-skew cos-like component of the wake
    real(rp) skew_sin   ! skew sin-like component of the wake
    real(rp) skew_cos   ! skew cos-like component of the wake
    logical polarized   ! Polarized mode?
  end type
\end{example}
\index{lR_freq_spread}
This is similar to the \vn{sr_mode_wake_struct}. \vn{%freq_in} is the
actual frequency in the input file. \vnr{bmad_parser} will set
\vn{%freq} to \vn{%freq_in} except when the \vn{lr_freq_spread} attribute
is non-zero in which case \vn{bmad_parser} will vary \vn{%freq} as
explained in \sref{s:lcav}. \vn{%polarized} is a logical that
indicates whether the the mode has a polarization angle. If so, then
\vn{%angle} is the polarization angle.

%--------------------------------------------------------------------------
\section {Wiggler Types}
\index{wiggler!types}

The \vn{%sub_key} component of the \vn{ele_struct} is used to distinguish
between \vn{map type} and \vn{periodic type} wigglers (\sref{s:wiggler}):
\begin{example}
  if (ele%key == wiggler\$ .and. ele%sub_key == map_type$) ...
  if (ele%key == wiggler\$ .and. ele%sub_key == periodic_type$) ...
\end{example}
For a \vn{map} type wiggler, the wiggler field terms
(\sref{s:wiggler.phys}) are stored in the \vn{%wig_term(:)} array of
the \vn{element_struct}. This is an array of \vn{wig_term_struct}
structure. A \vn{wig_term_struct} looks like:
\index{wig_term_struct}
\begin{example}
  type wig_term_struct
    real(rp) coef
    real(rp) kx, ky, kz
    real(rp) phi_z
    integer type      ! hyper_y\$, hyper_xy\$, or hyper_x\$
  end type
\end{example}
A \vn{periodic} wiggler will have a single \vn{%wig_term(:)} term that
can be used for tracking purposes, etc. The setting for this
\vn{wig_term} element is
\begin{example}
  ele%wig_term(1)%ky     = pi / ele%value(l_pole$)
  ele%wig_term(1)%coef   = ele%value(b_max$)
  ele%wig_term(1)%kx     = 0
  ele%wig_term(1)%kz     = ele%wig_term(1)%ky
  ele%wig_term(1)%phi_z  = (ele%value(l_pole$) - ele%value(l$)) / 2
  ele%wig_term(1)%type   = hyper_y$
\end{example}

%--------------------------------------------------------------------------
\section {Multipoles}
\index{ele_struct!multipoles}

\index{ele_struct!\%b(:)}\index{ele_struct!\%a(:)}
\index{multipole!an, bn!in ele_struct}
\index{multipole!KnL, Tn!in ele_struct}
The multipole components of an element (See \sref{s:fields}) are
stored in the pointers \vn{%a(:)} and \vn{%b(:)}. If \vn{%a} and
\vn{%b} are allocated they always have a range \vn{%a(0:n_pole_maxx)}
and \vn{%b(0:n_pole_maxx)}. Currently \vn{n_pole_maxx} = 20. For a
\vn{Multipole} element, the \vn{%a(n)} array stores the integrated
multipole strength \vn{KnL}, and the \vn{%b(n)} array stores the tilt
\vn{Tn}.

A list of \bmad routines for manipulating multipoles can be found in
\sref{r:multi}.

%--------------------------------------------------------------------------
\section{General Use Components}
\index{ele_struct!components not used by Bmad}

\index{ele_struct!\%r}
\index{ele_struct!\%ix_pointer}
\index{ele_struct!\%logic}
There are three components of an \vn{ele_struct} that are guaranteed
to never be used by any \bmad routine and so are available for use by
someone writing a program. These components are
\begin{example}
   real(rp), pointer :: r(:) => null()  ! For general use. Not used by \bmad. 
   integer ix_pointer                   ! For general use. Not used by \bmad.
   logical logic                        ! For general use. Not used by \bmad.
\end{example}

%--------------------------------------------------------------------------
\section{Initialization and Pointers}
\index{ele_struct!initialization}
\index{ele_struct!pointer components}

Generally most \vn{ele_struct} elements are part of a \vn{lat_struct}
variable and elements in a \vn{lat_struct} are handled by the
\vn{lat_struct} allocation/deallocation routines (see
\sref{s:lat:point}). In case you do have a local \vn{ele_struct}
variable within a subroutine or function then you either have to
deallocate the pointers within it with a call to
\vn{deallocate_ele_pointers} or you use the save attribute.
\begin{example}
  type (ele_struct), save :: ele     ! Either do this at the beginning ...
  ...
  call deallocate_ele_pointers (ele) ! ... Or do this at the end.
\end{example}

\noindent
The equal sign in the assignment
\begin{example}
  ele1 = ele2
\end{example}
is overloaded by the routine \vn{ele_equal_ele} to ensure that the
pointers of \vn{ele1} do not point to the same memory locations as the
pointers of \vn{ele2}.

