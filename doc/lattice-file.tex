\chapter{Lattice File Overview}
\label{c:lat.file}
\index{lattice files|hyperbf}

\index{Bmad!lattice file format}
A lattice (\sref{c:lat.concepts}) defines the sequence of elements
that a particle will travel through along with the attributes (length,
strength, orientation, etc.) of the elements.  A lattice file (or
files) is a file that is used to describe an accelerator or storage
ring. The \bmad software library comes with routines to parse lattice
files. There are two input formats that \bmad understands: The \bmad
standard format and XSIF\cite{b:xsif}. XSIF, developed at SLAC, has
its own documentation and will not be discussed in detail here except
for a few brief notes in Section~(\sref{s:xsif}).  Throughout this
manual, unless explicitly stated otherwise, the lattice format under
discussion will always pertain to the \bmad standard format.

While \bmad cannot directly read in \mad\cite{b:maduser} files, 
translation between \mad and \bmad lattice files is possible using
the \vn{Universal Accelerator Parser} as discussed in Section~\sref{s:aml}.

%---------------------------------------------------------------------------
\section{Lattice File Format}
\label{s:lattice.file.formats}

\index{MAD!lattice format}
The syntax that a \bmad standard lattice file must conform to is
modeled after the lattice input format of the \mad program.
Essentially, a \bmad lattice file is similar to a \mad lattice file
except that a \bmad file has no ``action'' commands (action commands
tell the program to calculate the Twiss parameters, do tracking,
etc.).  Since \bmad is a software library, interacting with the user to 
determine what actions a program
should take is left to the program and is not part of \bmad (although
the \bmad library provides the routines to perform many of the standard
calculations). A program is not required to use the \bmad parser
routines but, if it does, the following chapters describe how to
construct a valid lattice file.

%---------------------------------------------------------------------------
\section{File Example and Syntax}
\index{Bmad!statement syntax}

\index{parameter statement}
\index{use statement}
The following (rather silly) example shows some of the features of a
\bmad lattice file:
\begin{example}
  ! This is a comment
  parameter[E_TOT] = 5e9                   ! Parameter definition
  pa1 = sin(3.47 * pi / c_light)                 ! Constant definition
  bend1: sbend, type = "arc bend", l = 2.3,      ! An element definition
      g = 2*pa1, tracking_method = bmad_standard
  bend2: bend1, l = 3.4                          ! Another element def
  bend2[g] = 105 - exp(2.3) / 37.5               ! Redefining an attribute
  ln1: line = (ele1, ele2, ele3)                 ! A line definition
  ln2: line = (ln1, ele4, ele5)                  ! Lines can contain lines
  arg_ln(a, b) = (ele1, a, ele2, b)              ! A line with arguments.
  use, ln2                                       ! Which line to use for the lattice
\end{example}

\index{"! comment symbol}
\index{comment symbol ("!)}
A \bmad lattice file consists of a sequence of statements. An
exclamation mark (!) denotes a comment and the exclamation mark and
everything after the exclamation mark on a line are ignored.  \bmad is
generally case insensitive. All names are converted to uppercase
except strings (\sref{s:var.types}). Exceptions where a name is not
converted include file names and atomic formulas for materials used in
crystal diffraction.

\index{\& continuation symbol}
\index{continuation symbol (\&)}
Normally a statement occupies a single line in the file. Several
statements may be placed on the same line by inserting a semicolon
(``;'') between them. A long statement can occupy multiple lines by
putting an ampersand (``\&'') at the end of each line of the statement
except for the last line. Additionally, lines that end with an
``implicit continuation character''
are automatically continued to the next line. The implicit continuation 
characters are
\begin{example}
  ,   +   -   *   /   (   \{   [   =
\end{example}
Notice
that this is {\em not} like \vn{C/C++}. Thus the following is bad syntax
\begin{example}
  wall = \{
    section = \{s = 0.45     ! BAD SYNTAX. NO CONTINUATION CHARACTER HERE.
    \}                       ! BAD SYNTAX. NO CONTINUATION CHARACTER HERE.
  \}
\end{example}
Correct is:
\begin{example}
  wall = \{
    section = \{s = 0.45\} \}
\end{example}
or even:
\begin{example}
  wall = \{
    section = \{s = 0.45\} \&
  \}
\end{example}

\index{lattice files!name syntax}
Names of constants, elements, lines, etc. are limited to 40
characters. The first character must be a letter (\vn{A} --- \vn{Z}).
The other characters may be a letter, a digit (\vn{0} --- \vn{9}) or
an underscore (\vn{_}). Other characters may appear but should be avoided
since they are used by Bmad for various purposes. For example, the 
backslash (\vn{\B}) character is used to by Bmad when forming the names of
superposition slaves (\sref{s:super}) and dots (\vn{.}) are used by Bmad 
when creating names of \vn{tagged} elements (\sref{s:tag}). Also use of
special characters may make the lattice files less portable to non-Bmad programs.

\index{parameter statement}
\index{parameter statement}
\index{parameter statement}
\index{beginning statement}
The following example constructs a linear lattice with two elements: 
\begin{example}
  parameter[lattice_type] = LINEAR_LATTICE
  parameter[e_tot] =2.7389062E9
  parameter[particle] = POSITRON
  beginning[beta_a] = 14.5011548
  beginning[alpha_a] = -0.53828197
  beginning[beta_b] = 31.3178048
  beginning[alpha_b] = 0.25761815
  q: quadrupole, l = 0.6, b1_gradient = 9.011
  d: drift, l = 2.5
  t: line = (q, d)
  use, t 
\end{example}
here \vn{parameter[lattice_type]} (\sref{s:param}) is set to \vn{LINEAR_LATTICE}
which specifies that the lattice is not circular. In this case, the beginning 
Twiss parameters need to be specified and this is done by the \vn{beginning}
statements (\sref{s:beginning}). A quadrupole named \vn{q}
and a drift element named \vn{d} are specified
and the entire lattice consists of element \vn{q} followed by element \vn{d}.

%----------------------------------------------------------------------------
\section{Digested Files}
\label{s:digested}
\index{digested files}

Normally the \bmad parser routine will create what is called a
``digested file'' after it has parsed a lattice file so that when a
program is run and the same lattice file is to be read in again, to save
time, the digested file can be used to load in the lattice information.
This digested file is in binary format and is not human readable. The
digested file will contain the transfer maps for all the elements. 
Using a digested file can save considerable time if some of the
elements in the lattice need to have Taylor maps computed.
(this occurs typically with map--type wigglers).

\bmad creates the digested file in the same area as the lattice file.
If \bmad is not able to create a digested file (typically because it
does not have write permission in the directory), an error message will
be generated but otherwise program operation will be normal.

\index{ran}
\index{ran_gauss}
Digested files contain the names of the lattice files used to create
them. If a lattice file has been modified since the digested file has
been created then the lattice files will be reread and a new
digested file will be generated. 

Note: If any of the random number functions (\sref{s:functions}) are
used in the process of creating the lattice, the digested file will be
ignored. In this case, each time the lattice is read into a program,
different random numbers will be generated for expressions that use such
random numbers.

Digested files can also be used for easy transport of lattices between
programs or between sessions of a program. For example, using one
program you might read in a lattice, make some adjustments (say to model
shifts in magnet positions) and then write out a digested version of the
lattice. This adjusted lattice can now be read in by another program.

%---------------------------------------------------------------------------
\section{Element Sequence Definition}

\index{line}\index{use statement|hyperbf}
A \vn{line} defines a sequence of elements. \vn{lines} may contain
other \vn{lines} and so a hierarchy may be established. One line is
selected, via a \vn{use} statement, that defines the lattice. For
example:
\begin{example}
  l3: line = (l1, l2)   ! Concatenate two lines
  l1: line = (a, b, c)  ! Line with 3 elements
  l2: line = (a, z)     ! Another line 
  use, l3               ! Use l3 as the lattice definition.
\end{example}
In this case the lattice would be
\begin{example}
  (a, b, c, a, z)
\end{example}
\vn{Lines} can be defined in any order. See \cref{c:sequence} for more
details.

\index{superimpose}
The \vn{superimpose} construct allows elements to be placed in a
lattice at a definite longitudinal position. What happens is that
after a lattice is expanded, there is a reshuffling of the elements to
accommodate any new superimpose elements. See \sref{s:super} for more
details.

%---------------------------------------------------------------------------
\section{Lattice Elements}

The syntax for defining a lattice element roughly follows the
\mad\cite{b:maduser} program:
\begin{example}
  ele_name: keyword [, attributes]
\end{example}
where \vn{ele_name} is the element name, \vn{keyword} is the type of
element, and \vn{attributes} is a list of the elements
attributes. \cref{c:elements} gives a list of elements types with
their attributes.
\vn{Overlay} and \vn{group} type elements have a slightly different syntax:
\begin{example}
  ele_name: keyword = \{ list \}, master-attribute [= value] [, attributes]
\end{example}
and \vn{Girder} elements have the syntax
\begin{example}
  ele_name: keyword = \{ list \} [, attributes]
\end{example}  
For example:
\begin{example}
  q01w: quadrupole, type = "A String", l = 0.6, tilt = pi/2
  h10e: overlay = \{ b08e, b10e \}, hkick
\end{example}

%---------------------------------------------------------------------------
\section{Lattice Element Names}
\label{s:ele.names}
\index{element!names}

\index{element!name}\index{superimpose}\index{multipass}
A valid element name may be up to 40 characters in length. The first
character of the name must be a letter [A-Z]. After that, the rest of
the name can contain only letters, digits [0-9], underscore ``_'',
period ``.'', backslash ``\B'', or a hash mark ``\#''. It is best to
avoid these last three symbols since \bmad uses them to denote
``relationships''.  Periods are used for tagging (\sref{s:tag}), and
backslash and hash marks are used for to compose names for
superposition (\sref{s:super}) and multipass (\sref{s:multipass})
slave elements.

\index{reserved names}
\index{beam}\index{beam_start}\index{beginning}
\index{end}\index{parameter}\index{root}
There is a short list of names that cannot be used as an element name. 
These reserved names are:
\begin{example}
  beam
  beam_start
  beginning
  end
  parameter
  root
\end{example}

Where appropriate, for example when setting element attributes
(\sref{s:lat.attribs}), the wildcards \vn{``*''} and \vn{``%''} can be
used to select multiple elements.  The \vn{``*''} character will match
any number of characters (including zero) while \vn{``%''} maches to
any single character. Additionally, matching can be restricted to a
certain element class using the syntax:
\begin{example}
  class::element_name
\end{example}
where \vn{class} is a class name. For example:
\begin{example}
  m*              ! Match to all elements whose name begins with "m".
  a%c             ! Match to "abc" but not to "ac" or "azzc".
  quadrupole::*w  ! Match to all quadrupoles whose name ends in "w"
\end{example}

After lattice expansion (\sref{s:expand}), the general syntax to
specify a set of elements is:
\begin{example}
  \{class::\}\{branch_id>>\}element_id\{##N\}
\end{example}
where \vn{\{...\}} markes an optional component, \vn{class} is a class
name, \vn{branch_id} is a branch name or index (\sref{s:branching}),
\vn{element_id} is and element name or element index
(\sref{s:lines.wo.arg}), and \vn{\#\#N} indicates that the N\Th matching
element is to be used. Examples:
\begin{example}
  quad::x_br>>q*        ! All quadrupoles of branch "x_br" whose name begins with "q".
  2>>45                 ! Element 45 of branch 2.
  q01##3                ! The 3rd element in each branch named q01.
\end{example}

Multiple elements in a lattice may share the same name.  When
branching (\sref{s:branching}) is used, to differentiate elements that
appear in different branches, the ``branch qualified'' element name may be
used. The branch qualified element name is of the form
\begin{example}
  branch_name>>element_name
\end{example}
where \vn{branch_name} is the name of the branch and \vn{element_name}
is the ``regular'' name of the element. Example:
\begin{example}
  root>>q10w
  x_branch>>crystal3
\end{example}

For branch lines (\sref{s:branching}), the full ``branch
qualified'' name of an element is of the form
\begin{example}
  branch_name>>element_name
\end{example}
where \vn{branch_name} is the name of the branch and \vn{element_name} is the
``regular'' name of the element. Example:
\begin{example}
  root>>q10w
  xline>>cryst3
\end{example}
Using the full name is only needed to distinguish elements that have
the same regular name in separate branches.  When parsing a lattice
file, branches are not formed until the lattice is expanded
(\sref{s:expand}). Therefore an \vn{expand_lattice} statement is
required before full names can be used in statements.



%---------------------------------------------------------------------------
\section{Lattice Element Attributes}
\label{s:lat.attribs}
\index{element attributes|hyperbf}

Any lattice element has various attributes like its name, its length,
its strength, etc. The values of element attributes can be
specified when the element is defined. For example:
\begin{example}
  b01w: sbend, l = 6.0, rho = 89.0 ! Define an element with attributes.
\end{example}
After an element's definition, an individual attribute may be referred
to using the syntax
\begin{example}
  element-name[attribute-name]
\end{example}
Element attributes can be set or used in an algebraic expression:
\begin{example}
  bo1w[roll] = 6.5                  ! Set an attribute value.
  b01w[l] = 6.5                     ! Change an attribute value.
  b01w[l] = b01w[rho] / 12          ! OK to reset an attribute value.
  my_const = b01w[rho] / b01w[l]    ! Use of attribute values in an expression.
\end{example}
Notice that there can be no space between the element name and the
\vn{[} opening bracket.  

Chapter \cref{c:elements} lists the attributes appropriate for each
element class.

When setting an attribute value, if more than one element has the
\vn{element-name} then {\it all} such elements will be set.  When
setting an attribute value, if \vn{element-name} is the name of a type
of element, all elements of that type will be set. For example
\begin{example}
  q_arc[k1] = 0.234                      ! Set all elements named Q_ARC. 
  rfcavity::*[voltage] = 3.7             ! Set all RFcavity elements.
\end{example}

A prepended plus sign ``+'' indicates that if no elements are matched to, this
is not an error. For example
\begin{example}
  +sextupole::*[tracking_method] = taylor
\end{example}
In this case, it is not an error if there are no sextupoles in the lattice.

The wild cards \vn{``*''}, and \vn{``\%''} can be used to can be used
(\sref{s:ele.names}). Examples:
\begin{example}
  *[tracking_method] = bmad_standard  ! Matches all elements.
  quadrupole::Q*[k1] = 0.234    ! Matches all quadrupoles with names beginning with Q.
  Q%1[k1] = 0.234               ! Matches to "Q01" but not "Q001".
\end{example}
\index{beginning element}
Note: A name with wild cards will never match to the \vn{BEGINNING} element (\sref{s:use}).

After lattice expansion (\sref{s:expand}), the attributes of specific elements
may be set using the syntax as discussed in Section \sref{s:ele.names}. Example:
\begin{example}
  expand_lattice              ! Expand the lattice.
  97[x_offset] = 0.0023       ! Set x_offset attribute of 97th element
  b2>>si_cryst##2[tilt] = 0.1 ! Tilt the 2nd instance of "si_cryst" in branch "b2" 
\end{example}

%---------------------------------------------------------------------------
\section{Custom Element Attribute Names}
\label{s:custom.attrib}
\index{element attributes!defining custom attributes}

Custom element attributes may be defined for any class of element.
Custom element attributes are useful with programs that need to
associate ``extra'' information with particular lattice elements and
it is desired that this extra information be settable from within a
lattice file. For example, a program might need error tollerances for
the strength of quadrupoles.

Adding custom attributes will not disrupt programs that are not
designed to use the custom attributes. Currently, up to five custom
attributes may be defined for any given element type. The syntax for
defining custom attributes is:
\begin{example}
  parameter[custom_attributeN] = "\{class::\}attribute_name"
\end{example}
\vn{custom_attributeN} may be one of:
\begin{example}
  custom_attribute1
  ...
  custom_attribute5
\end{example}
and "\vn{attribute_name}" is the name of the attribute. To restrict the
custom attribute to a particular element class, the element class
can be prefixed to the attribute name. Examples:
\begin{example}
  parameter[custom_attribute1] = "quadrupole::error_k1"
  parameter[custom_attribute1] = "sextupole::error_k2"
  parameter[custom_attribute2] = "mag_id"
\end{example}
The first example assigns a custom attribute name of \vn{error} to all
quadrupoles.  The third example assigns a custom attributge name of
\vn{mag_id} to all elements. Once a particular \vn{custom_attributeN}
has been assigned to a particular element class, the particular
\vn{custom_attributeN} may not be reassigned for that element class
but may be used for classes that have not been set. Thus, in the above
example, the set of \vn{custom_attribute1} for quadrupoles and
sextupoles is valid.

Once a custom attribute has been defined it may be set for an element
of the correct type. Example:
\begin{example}
  parameter[custom_attribute2] = "lcavity::rms_phase_err"
  ...
  l2a: lcavity, rms_phase_err = 0.0034, ...
\end{example}

Note: If custom string information needs to be associated with an
element, the \vn{type}, \vn{alias} and \vn{descrip} element components
(\sref{s:alias} are available.

%---------------------------------------------------------------------------
\section{Variable Types}
\label{s:var.types}
\index{arithmetic expressions!variables}

\index{logicals|hyperbf}
There are five types of variables in \bmad: reals, integers, switches,
logicals (booleans), and strings. Acceptable logical values are
\begin{example}
   true    false
   t       f
\end{example}
For example
\begin{example}
  rf1[is_on] = False
\end{example}

\index{strings|hyperbf}
String literals can be quoted using double quotes (") or single quotes ('). 
If there are no
blanks or commas within a string, the quotes can be omitted. For example:
\begin{example}
  Q00W: Quad, type = "My Type", alias = Who_knows, &
                                  descrip = "Only the shadow knows"
\end{example}
Unlike most everything else, strings are not converted to uppercase.

\index{switches|hyperbf}
Switches are variables that take discrete values. For example:
\begin{example}
  parameter[particle] = positron          
  q01w: quad, tracking_method = bmad_standard 
\end{example}
The name ``switch'' can refer to the variable (for example,
\vn{tracking_method}) or to a value that it can take (for example,
\vn{bmad_standard}). The name ``method'' is used interchangeably with switch.

%-----------------------------------------------------------------
\section{Units and Constants}
\label{s:constants}
\index{constants|hyperbf}

\bmad uses SI (Syst\'eme International) units as shown in
Table~\ref{t:units}.

\bmad defines commonly used physical and mathematical constants
shown in Table~\ref{t:constants}.  All symbols use straight SI units
except for \vn{e_mass} and \vn{p_mass} which are provided for
compatibility with \mad.

\begin{table}
\centering
\begin{tabular}{|l|l|l|l|} \hline
  {\em Symbol}  & {\em Value}       & {\em Units} &  {\em Name}       \\ \hline
  pi            & 3.14159265359          &        &                   \\
  twopi         & 2 * pi                 &        &                   \\
  fourpi        & 4 * pi                 &        &                   \\
  e_log         & 2.718281828            &        &                   \\
  sqrt_2        & 1.4142135623731        &        &                   \\
  degrees       & 180 / pi               &        &                   \\
  m_electron    & $0.51099906 \pow{6}$   & eV     & Electron mass     \\
  m_muon        & $0.1056583893 \pow{6}$ & eV     & Muon mass         \\
  m_proton      & $0.938271998 \pow{9}$  & eV     & Proton mass       \\
  c_light       & $2.99792458 \pow{8}$   & m/sec  & Speed of light    \\
  r_e           & $2.8179380 \pow{-15}$  & m      & Electron radius   \\
  r_p           & $1.5346980 \pow{-18}$  & m      & Proton radius     \\
  e_charge      & $1.6021892 \pow{-19}$  & C      & Electron charge   \\
  h_planck      & $4.13566733 \pow{-15}$ & eV*sec & Planck's constant \\
  h_bar_planck  & $6.58211899 \pow{-16}$ & eV*sec & Planck / $2\pi$   \\
  e_mass        & $0.51099906 \pow{-3}$  & GeV    & Electron mass     \\
  p_mass        & $0.938271998$          & GeV    & Proton mass       \\ \hline
\end{tabular}
\caption{Physical and mathematical constants recognized by \bmad.}
\label{t:constants}
\end{table}

%---------------------------------------------------------------------------
\section{Arithmetic Expressions}
\index{arithmetic expressions} 

Arithmetic expressions can be used in a place where a real value is required.
The standard operators are defined: \hfil\break
\hspace*{0.15in}
\begin{tabular}{ll}
  $a + b$           & Addition        \\
  $a - b$           & Subtraction     \\
  $a \, \ast \, b$  & Multiplication  \\
  $a \; / \; b$     & Division        \\
  $a \, \land \, b$ & Exponentiation  \\
\end{tabular}
\hfil\break
\bmad also has a set of intrinsic functions. A list of these is given
in \sref{s:functions}.

\index{arithmetic expressions!constants}
Literal constants can be entered with or without a decimal point. An
exponent is marked with the letter E. For example
\begin{example}
  1, 10.35, 5E3, 314.159E-2
\end{example}
Symbolic constants can be defined using the syntax
\begin{example}
  parameter_name = expression
\end{example}
\index{MAD!syntax compatibility with BMAD}
Alternatively, to be compatible with \mad, using ``:='' instead of ``='' is accepted
\begin{example}
  parameter_name := expression
\end{example}
Examples:
\begin{example}
  my_const = sqrt(10.3) * pi^3
  abc     := my_const * 23
\end{example}
\index{MAD!delayed substitution}
Unlike \mad, \bmad uses immediate substitution so that all constants
in an expression must have been previously defined. For example, the
following is not valid:
\begin{example}
  abc      = my_const * 23      ! No: my_const needs to be defined first.
  my_const = sqrt(10.3) * pi^3
\end{example}
here the value of \vn{my_const} is not known when the line ``\vn{abc}
= $\ldots$'' is parsed. Once
defined, symbolic constants cannot be redefined. For example:
\begin{example}
  my_const = 1
  my_const = 2  ! No: my_const cannot be redefined.
\end{example}

Another potential pitfall with immediate substitution is when using
dependent element attributes (\sref{s:depend}). For example:
\begin{example}
  b01w: sbend, l = 0.5, angle = 0.02
  a_const = b01w[g]    ! No: bend g has not yet been computed!
\end{example}
Here the bend strength \vn{g} (\sref{s:bend}) will eventually be
computed to be 0.04 (= angle / l) but that computation does not happen
until lattice expansion (\sref{s:expand}). In this case, the value of
\vn{a_const} will be the default value of \vn{g} which is zero.  As a
rule of thumb, never rely on dependent attributes having their correct
value.

%---------------------------------------------------------------------------
\section{Intrinsic functions}
\label{s:functions}
\index{intrinsic functions}

\index{sqrt}\index{log}\index{exp}\index{sin}\index{cos}\index{tan}
\index{asin}\index{acos}\index{atan}\index{abs}\index{ran}\index{ran_gauss}
The following intrinsic functions are recognized by \bmad: \hfil\break
\hspace*{0.15in}
\begin{tabular}{ll}
  \vn{sqrt}(x)      & Square Root    \\
  \vn{log}(x)       & Logarithm      \\
  \vn{exp}(x)       & Exponential    \\
  \vn{sin}(x)       & Sine           \\
  \vn{cos}(x)       & Cosine         \\
  \vn{tan}(x)       & Tangent        \\
  \vn{asin}(x)      & Arc sine       \\
  \vn{acos}(x)      & Arc cosine     \\
  \vn{atan}(x)      & Arc Tangent    \\
  \vn{abs}(x)       & Absolute Value \\
  \vn{ran}()        & Random number between 0 and 1 \\
  \vn{ran_gauss}()  & Gaussian distributed random number \\
\end{tabular}

\index{ran_seed}
\vn{ran_gauss} is a Gaussian distributed random number with unit RMS. 
Both \vn{ran} and \vn{ran_gauss} use a seeded random number generator. 
To choose the seed set 
\begin{example}
  parameter[ran_seed] = <Integer>
\end{example}
A \vn{value} of zero will set the seed using the system clock so that
different sequences of random numbers will be generated each time a
program is run.  The default behavior if \vn{parameter[ran_seed]} is
not present is to use the system clock for the seed.

\index{expand_lattice}
If an element is used multiple times in a lattice, and if \vn{ran} or
\vn{gauss_ran} is used to set an attribute value of this element, then
to have all instances of the element have different attribute values
the setting of the attribute must be after the lattice has been
expanded (\sref{s:expand}). For example:
\begin{example}
  a: quad 
  a[x_offset] = 0.001*ran_gauss()
  my_line: line = (a, a)
  use, my_line
\end{example}
Here, because \bmad does immediate evaluation, the \vn{x_offset}
values for \vn{a} gets set in line 2 and so both copies of \vn{a} in
the lattice get the same value. This is probably not what is wanted.
On the other hand if the attribute is set after lattice expansion:
\begin{example}
  a: quad 
  my_line: line = (a, a)
  use, my_line
  expand_lattice
  a[x_offset] = 0.001*ran_gauss()
\end{example}
Here the two \vn{a} elements in the lattice get different values for
\vn{x_offset}.

%-----------------------------------------------------------------------------
\section{Print Statement}
\label{s:print}
\index{print statement|hyperbf}

The \vn{print} statement prints a message at the terminal when the 
lattice file is parsed by a program. Syntax:
\begin{example}
  print <String>
\end{example}
For example
\begin{example}
  print Remember! Q01 quad strength not yet optimized!
\end{example}
The \vn{print} statement is useful to remind someone using the lattice of important details.

%-----------------------------------------------------------------------------
\section{Title Statement}
\index{title statement|hyperbf}

The \vn{title} statement sets a title string which can be used by a program. 
For consistency with \mad there are two possible syntaxes
\begin{example}
  title, <String>
\end{example}
or the statement can be split into two lines
\begin{example}
  title
  <String>
\end{example}
For example
\begin{example}
  title
  "This is a title"
\end{example}

%--------------------------------------------------------------------------
\section{Call Statement}
\label{s:call}
\index{call statement|hyperbf}

It is frequently convenient to separate the lattice definition into
several files.  Typically there might be a file (or files) that define
the layout of the lattice (something that doesn't change often) and a
file (or files) that define magnet strengths (something that changes
more often).  The \vn{call} is used to read in separated lattice
files. The syntax is
\begin{example}
  call, filename = <String>
\end{example}
Example:
\begin{example}
  call, filename = "../layout/my_layout.bmad"      ! Relative pathname
  call, filename = "/nfs/cesr/lat/my_layout.bmad"  ! Absolute pathname
\end{example}
\bmad will read the called file until a \vn{return} or \vn{end_file}
statement is encountered or the end of the file is reached.

For filenames that are relative, the called file will be searched for in
two different locations:
\begin{example}
  1) Relative to the directory of the calling file.
  2) Relative to the current directory.
\end{example}
The first instance where a file is found is used.
Thus, in the above example, the first call will search for:
\begin{example}
  2) ../layout/my_layout.bmad  (relative to the calling file directory)
  1) ../layout/my_layout.bmad  (relative to the current directory)
\end{example}

An XSIF (\sref{s:lattice.file.formats}) lattice file may be called
from within an \bmad lattice file by prepending \vn{"xsif::"} to the
file name. Example:
\begin{example}
  call, filename = "xsif::my_lattice.xsif"
\end{example}
This statement must be the first statement in the \bmad lattice file
except for any comments or debugging statements (\sref{s:debug}. 
The XSIF lattice file must define a
complete lattice and cannot contain any \bmad specific statements. The
call to the XSIF file automatically expands the lattice
(\sref{s:expand}) and any additional statements in the \bmad lattice
file operate on the expanded lattice.

%--------------------------------------------------------------------------
\section{Inline Call}
\label{s:call.inline}
\index{call!inline}

Any lattice elements will have a set of attributes that need to be defined.
As a convenience, it is possible to segregate an element attribute or attributes
into a separate file and then ``call'' this file using an
``inline call''. The inline call has three forms. In an element definition,
the inline call has the form
\begin{example}
  <ele_name>: <ele_type>, ..., call::<file_name>, ...
\end{example}
or
\begin{example}
  <ele_name>: <ele_type>, ..., <attribute_name> = call::<file_name>, ...
\end{example}
where \vn{<attribute_name>} is the name of the attribute and
\vn{<file_name>} is the name of the where the attribute structure is
given.  The third form of the inline call occurs when an element
attribute is redefined and has the form
\begin{example}
  <ele_name>[<attribute_name>] = call::<file_name>
\end{example}


%Most element attributes are associated with a single real number. Some
%attributes are more complicated and are defined by a ``structure''
%which contain multiple numbers. The \vn{wall} attribute
%(\sref{s:wall}) is an example of an attribute that has a complicated
%structure. 

%--------------------------------------------------------------------------
\section{Return and End_File statements}
\index{return statement|hyperbf}
\index{end_file statement|hyperbf}

\vn{Return} and \vn{end_file} have identical effect and tell \bmad to
ignore anything beyond the \vn{return} or \vn{end_file} statement in
the file.

%----------------------------------------------------------------------------
\section{Expand_Lattice Statement}
\label{s:expand}
\index{lattice!expansion|hyperbf}
\index{expand_lattice|hyperbf}

At some point in parsing a lattice file, the ordered sequence of
elements that form a lattice must be constructed. This process is called
\vn{lattice expansion} since the element sequence can be built up from
sub--sequences (\sref{c:sequence}). Normally lattice expansion happens
automatically at the end of the parsing of the lattice file but an
explicit \vn{expand_lattice} statement in a lattice file will cause
immediate expansion. The reason why this can be important is that there
are restrictions, on some types of operations which must come either
before or after lattice expansion:
\begin{Itemize}
\item 
\index{ran}
\index{ran_gauss}
The \vn{ran} and \rn{ran_gauss} functions, when used with elements
that show up multiple times in a lattice, generally need to be used
after lattice expansion. See \sref{s:functions}.
\item 
Some dependent variables may be set as if they are independent
variables but only if done before lattice expansion. See \sref{s:depend}.
\index{multipass}
\item 
Setting the \vn{dphi0} attribute for an 
\vn{Lcavity} or \vn{RFcavity} multipass
slave may only be done after lattice expansion (\sref{s:multipass}).
\item
\index{tags for Lines and Lists}
Setting individual element attributes for tagged elements can only be done
after lattice expansion (\sref{s:tag}).
\end{Itemize}

Lattice expansion is only done once so it is an error if multiple
\vn{expand_lattice} statements are present.

%--------------------------------------------------------------------------
\section{Debugging Statements}
\label{s:debug}
\index{no_digested statement}
\index{no_superimpose statement}
\index{parser_debug statement}
\index{debug_marker statement}
\index{no_digested statement}
\index{lattice files!parser debugging}

There are a few statements
which can help in debugging the \bmad lattice parser
itself. That is, these statements are generally only used by programmers.
These statements are:
\begin{example}
  debug_marker
  no_digested
  no_superimpose
  parser_debug
\end{example}

The \vn{debug_marker} statement is used for marking a place in the lattice file
where program execution is to be halted. This only works when running
a program in conjunction with a program debugging tool. 

The \vn{no_digested} statement if present, will prevent \bmad from 
creating a digested file (\sref{s:digested}. That is, the lattice file will always
be parsed when a program is run.

The \vn{no_superimpose} statement is used to suppress superpositions
(\sref{s:super}). This is useful for debugging purposes.

The \vn{parser_debug} statement will cause information about the
lattice to be printed out at the terminal. It is recommended that this
statement be used with small test lattices since it can generate a lot
of output. The syntax is
\begin{example}
  parser_debug <switches>
\end{example}
Valid \vn{<switches>} are
\begin{example}
  beam_start          ! Print the beam_start information.
  ele <n1> <n2> ...   ! Print full info on selected elements.
  lattice             ! Print a list of lattice element information.
  lord                ! Print full information on all lord elements.
  seq                 ! Print sequence information.
  slave               ! Print full information on all slave elements.
  var                 ! Print variable information.
\end{example}
Here $<n1>$, $<n2>$, etc. are the index of the selected elements in
the lattice.  Example
\begin{example}
  parser_debug var lat ele 34 78
\end{example}

%---------------------------------------------------------------------------
\section{XSIF Lattice File Format}
\label{s:xsif}
\index{XSIF|hyperbf}
\index{parameter statement!lattice_type}

XSIF, developed at SLAC, stands for 
``Extended Standard Input Format.''
XSIF is essentially a subset of the \mad\cite{b:maduser} input format. 
XSIF has its own documentation which can be found at:
\begin{example}
  http://www-project.slac.stanford.edu/
              lc/ilc/TechNotes/LCCNotes/PDF/LCC-0060%20rev.1.pdf
\end{example}
The major reason for using XSIF in \bmad is when portability is needed
with a non-\bmad program.

An XSIF lattice file may be called from within a \bmad lattice
file. See Section~\sref{s:call} for details.

Since XSIF does not have a \vn{parameter[lattice_type]} statement
(\sref{s:param}), the type of the lattice (whether circular or linear)
is determined by the presence or absence of any \vn{lcavity} elements
in the XSIF file. This is independent of whether \vn{lcavity} elements
are actually used in the lattice.

Note: One point that is not covered in the XSIF
documentation is that for a \vn{MATRIX} element, unlike \mad, the
\vn{R$ii$} terms (the diagonal terms of the linear matrix) are not
unity by default. Thus
\index{MAD}
\begin{example}
  m: matrix
\end{example}
in an XSIF file will give a matrix with all elements being zero.

%----------------------------------------------------------------------------
\section{Translation between MAD, Bmad, and XSIF Using the Accelerator Markup Language}
\label{s:aml}
\index{Accelerator Markup Language (AML)}
\index{Universal Accelerator Parser (UAP)}

\index{MAD}
The \vn{Accelerator Markup Language} (\vn{AML}) / \vn{Universal
Accelerator Parser} (\vn{UAP}) project\cite{b:aml} is a collaboratory
effort with the aim of 1) creating a lattice format (the \vn{AML} part)
that can be used to fully describe accelerators and storage rings, and
2) producing software (the \vn{UAP} part) that can parse \vn{AML}
lattice files. A side benefit of this project is that the \vn{UAP} code
has been extended to be able to translate between \vn{AML}, \bmad, \mad-8,
\mad-X, and \vn{XSIF}. See the AML/UAP project home web page for more details.
