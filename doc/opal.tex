\chapter{OPAL}
\label{c:opal}
\index{OPAL}
%----------------------------------------------------------------------------

OPAL (Object Oriented Parallel Accelerator Library) is a tool for charged-particle optic
calculations in large accelerator structures and beam lines including 3D space charge. OPAL is built
from first principles as a parallel application, OPAL admits simulations of any scale: on the laptop
and up to the largest High Performance Computing (HPC) clusters available today. Simulations, in
particular HPC simulations, form the third pillar of science, complementing theory and experiment.

OPAL includes various beam line element descriptions and methods for single particle optics, namely
maps up to arbitrary order, symplectic integration schemes and lastly time integration. OPAL is
based on IPPL (Independent Parallel Particle Layer) which adds parallel capabilities. Main functions
inherited from IPPL are: structured rectangular grids, fields and parallel FFT and particles with
the respective interpolation operators. Other features are, expression templates and massive
parallelism (up to 8000 processors) which makes is possible to tackle the largest problems in the
field.

The  manual can be obtained at
\begin{example} 
  amas.web.psi.ch/docs/opal/   
\end{example}

%--------------------------------------------------------------------------
\section{Phase Space}
\label{s:opal.space}
\index{OPAL!phase space}

OPAL uses different longitudinal phase space coordinates compared to \bmad.  \bmad's phase space
coordinates are
\Begineq
  (x, p_x/p_0, y, p_y/p0, -\beta c (t - t_0), (p-p_0)/p_0)
\Endeq
OPAL uses
\Begineq
  (x, \gamma \beta_x,  y, \gamma \beta_y, z, \gamma \beta_z)
\Endeq
\vn{convert_particle_coordinates_s_to_t} and \vn{convert_particle_coordinates_s_to_t} are conversion routines \ldots

%%--------------------------------------------------------------------------
%\section{Initialization}
%\label{s:etienne.init}
%\index{PTC/FPP!initialization}
%
%One important parameter in PTC is the order of the Taylor maps.
%By default \bmad will set this to 3. The order can be set within
%a lattice file using the \vn{parameter[taylor_order]} attribute.
%In a program the order can be set using \vn{set_ptc}. In fact
%\vn{set_ptc} must be called by a program before PTC can be used.
%\vn{bmad_parser} will do this when reading in a lattice file.
%That is, if a program does not use \vn{bmad_parser} then to use PTC it
%must call \vn{set_ptc}. Note that resetting PTC to a different order
%reinitializes PTC's internal memory so one must be careful if one wants
%to change the order in mid program.
%
%%--------------------------------------------------------------------------
%\section{Taylor Maps}
%\label{s:etienne.taylor}
%\index{PTC/FPP!Taylor Maps}
%
%\index{PTC/FPP!real_8}\index{PTC/FPP!universal_taylor}
%FPP stores its \vn{real_8} Taylor maps in such a way that it is not
%easy to access them directly to look at the particular terms. To
%simplify life, \'Etienne has implemented the
%\vn{universal_taylor}structure:
%\begin{example}
%  type universal_taylor
%    integer, pointer  :: n       ! Number of coefficients
%    integer, pointer  :: nv      ! Number of variables
%    real(dp), pointer :: c(:)    ! Coefficients C(N)
%    integer, pointer  :: j(:,:)  ! Exponents of each coefficients J(N,NV)
%  end type
%\end{example}
%\bmad always sets \vn{nv} = 6. \bmad overloads the equal sign to call 
%routines to convert between \'Etienne's
%\vn{real_8} Taylor maps and \vn{universal_taylor}:
%\begin{example}
%  type (real_8) tlr(6)           ! Taylor map
%  type (universal_taylor) ut(6)  ! Taylor map
%  ...
%  tlr = ut                       ! Convert universal_taylor -> real_8
%  ut = tlr                       ! Convert real_8 -> universal_taylor
%\end{example}
