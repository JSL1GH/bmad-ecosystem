\chapter{Starting Tao}
\label{c:starting.tao}

%----------------------------------------------------------------
\section{Obtaining Tao}
\label{s:obtaining}

Instructions for setting up the appropriate environmental variables
and for obtaining the source files can be found at:
\begin{example}
  http://www.lepp.cornell.edu/~dcs/bmad/
\end{example}

Briefly, you should be able to run \tao using the command
\begin{example}
  tao \{-init <tao_input_file>\} \{-beam_all <beam_file>\} 
           \{-beam0 <beam_file>\} \{-lat <lattice_file>\}
\end{example}
\vn{\$ACC_EXE} is an environmental variable pointing to the directory
the \tao executable is in.  The root initialization file
\vn{<tao_input_file>} is the file that \tao reads to start \tao's
initialization process. If not present, \vn{<tao_input_file>} defaults
to \vn{tao.init}. The \vn{-beam_all} switch is for reading in data
generated from beam tracking (\sref{s:beam.init}). The \vn{-beam0}
switch is for specifying the initial beam distribution.  The
\vn{-lat} switch is used to override the lattice file specified in
the root initialization file. See section~\sref{s:command.line} for
more details. Example:
\begin{example}
  tao -init my.init -lat slac.xsif
\end{example}
An initialization file is actually not needed. In this case, a
\vn{-lat} switch is manditory and \tao will use a set of default plot
templates for plotting.

This tutorial uses the example set of input files that comes with the \tao library.
If you are using a computer on the Cornell CLASSE linux cluster, you can get the
files from: 
\begin{example}
  $ACC_RELEASE_DIR/tao/examples/introduction_to_tao
\end{example}
If you are not on the cluster the example is at:
\begin{example}
  $ACC_ROOT_DIR/tao/examples/introduction_to_tao
\end{example}
In either case, copy this directory to your local area for use with the following sections.

%----------------------------------------------------------------
\section{Initializing Tao}
\index{initializing!files}
\label{s:initializing}

Initialization occurs when \tao is started. The initialization information can reside in one file or
it can be split into a number of files as discussed in Section~\sref{s:init.global}. If no
initialization files are found. \tao uses a default initialization.

\tao is started with the command:
\begin{example}
  tao
\end{example}
Since no initialization file is specified on the command line, the default file \vn{tao.init} (if it
exists) is used.  Using the example files in the \vn{introduction_to_tao} directory \tao
(\sref{s:obtaining}), the \vn{tao.init} there has the following lines:
\begin{example}
  &tao_start
    plot_file = 'tao_plot.init'
  /
\end{example}
The plotting information will
come from the file \vn{tao_plot.init}. Since no other initialization
files are specified (\sref{s:init.global}), \tao will look for the
non-plotting information (except for the lattice file) in \vn{tao.init}.

The lattice file is specified in the \vn{tao_design_lattice} namelist
in \vn{tao.init}:
\begin{example}
  &tao_design_lattice
    n_universes = 1
    design_lattice(1) = "bmad_L9A18A000-_MOVEREC.lat"
  /
\end{example}
\tao will setup a single universe since \vn{n_universes = 1}.
By default, \tao assumes that this lattice uses the \bmad lattice
format.  With the above information, \tao has the information on what
files it needs to read to initialize itself.

%----------------------------------------------------------------
\section{Single Mode}\index{single Mode}
\label{s:single.mode}

\tao has a \vn{single mode} in which single keystrokes are interpreted
as commands. \tao can be set up so that in \vn{single mode} the
pressing of certain keys increase or decrease variables. While the
same effect can be achieved in the standard \vn{line mode}, \vn{single
mode} allows for quick adjustments of variables. See
Chapter~\sref{c:single} for more details.

