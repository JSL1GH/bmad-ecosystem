\chapter{Synchrotron Radiation}

%-----------------------------------------------------------------
\section{Radiation Damping and Excitation}
\label{s:radiation}
\index{radiation!damping and excitation|hyperbf}

Emission of synchrotron radiation by a particle can be decomposed into two parts. The deterministic
average energy emitted produces damping while the stochastic fluctuating part of the energy spectrum
produces excitation\cite{b:jowett}.

The treatment of radiation damping by \bmad essentially follows Jowett\cite{b:jowett}. The energy
loss at a given location is modeled via
\begin{equation}
  \frac{\Delta E}{E_0} = 
  -k_E \equiv -\left[ k_d \ave{g^2} \, L_p + \sqrt{k_\mk{f} \ave{g^3} L_p} \,\, \xi \right] \, (1 + p_z)^2
  \label{deekk}
\end{equation}
where $L_p$ is the actual path length, $g$ is the bending strength ($1/g$ is the bending radius),
and $\ave{\ldots}$ is an average over the actual path.  In the above equation $k_d$ gives the
deterministic part of the emission, $\xi$ is a Gaussian distributed random number with unit sigma
and zero mean, and $k_\mk{f}$ is the amplitude of the stochastic part of the emission. Values for $k_d$
and $k_\mk{f}$ are calculated via the equations
\begin{align}
  k_d &= \frac{2 \, \rc}{3} \, \gamma_0^3
    \label{k2r3g} \\
  k_\mk{f} &= \frac{55 \, \rc \, \hbar}{24 \, \sqrt{3} \, m \, c} \, \gamma_0^5
    \label{k55rh}
\end{align}
where $\gamma_0$ is the energy factor of an on-energy particle and $\rc$ is the particles
``classical radius'' given by
\begin{equation}
  \rc = \frac{q^2}{4 \, \pi \, \epsilon_0 \, m \, c^2} 
  \label{rq4pe}
\end{equation}
where $q$ is the particle's charge and $m$ is the particle's mass.

Ignoring the finite opening angle of the emission for now, the angular orientation of the particle
motion is invariant for forward directed emission which leads to the following equations for the
changes in momentum phase space coordinates
\begin{equation}
  \Delta p_x = -\frac{k_E}{1 + p_z} \, p_x , \qquad
  \Delta p_y = -\frac{k_E}{1 + p_z} \, p_y, \qquad
  \Delta p_z \approx \frac{\Delta E}{E_0} = -k_E 
  \label{pk1pp}
\end{equation}

Synchrotron radiation emission involves energy loss and this energy loss leads to what is known as
the energy ``sawtooth'' effect where the curve of particle energy on the closed orbit as a function
of longitudinal position has a sawtooth shape. A sawtooth pattern can also be generally seen in the
horizontal orbit. It is sometimes convenient in simulations to eliminate the sawtooth effect. This
can be done by shifting the photon emission spectrum at any given element to have zero average
energy loss along the closed orbit. For this calculation the closed orbit should be the closed orbit
as calculated without radiation damping (in other words the closed orbit without the sawtooth). In
this case, $k_E$ is calculated by
\begin{equation}
  k_E = \left[ k_d \, \ave{g^2} L_p + \sqrt{k_\mk{f} \ave{g^3} L_p} \,\, \xi \right] \, (1 + p_z)^2 - 
  k_d \, \ave{g_0^2} \, L_p
\end{equation}
where $g_0$ is $g$ evaluated along the closed orbit. In practice, for the calculation, \bmad
approximates the closed orbit as the zero orbit. 

The deterministic and stochastic parts of the emission can be included or excluded from a tracking
simulation by setting in a lattice file the \bmad global parameters (\sref{s:bmad.common})
\begin{example}
  bmad_com[radiation_damping_on]      = True or False  ! Deterministic part on/off.
  bmad_com[radiation_fluctuations_on] = True or False  ! Stochastic part on/off.
  bmad_com[radiation_zero_average]    = True or False  ! Make ave radiation kick zero.
\end{example}
The global parameter \vn{bmad_com[radiation_zero_average]} controls the shifting of the photon
spectrum to have zero average. Currently, the shifting of the spectrum only works for non PTC
dependent tracking. That is, the shifting is not applicable to tracking with Taylor maps and with
\vn{symp_lie_ptc} (\sref{s:tkm}) tracking.

The fact that an emitted photon is not exactly colinear with the particle direction (often called
the ``vertical opening angle'') can be modeled as a separate process from the energy loss. With this
approximation, the change $\Delta p_\perp$ in the momentum transverse to the bending plane is given
by
\begin{equation}
  \Delta p_\perp = \sqrt{k_v \ave{g^3} L_p} \,\, \xi
  \label{dpkgl}
\end{equation}
where the $\xi$ in \Eq{dpkgl} is independent of the $\xi$ in \Eq{deekk} and
\begin{equation}
 k_v = \frac{13 \, \rc \, \hbar}{24 \, \sqrt{3} \, m \, c} \, \gamma_0^3
\end{equation}

%-----------------------------------------------------------------
\section{Transport Map with Radiation Included}
\label{s:map.rad}
\index{map!with radiation included}

Transport maps which include radiation effects can be constructed\cite{b:ohmi}. The first step is to
calculate the reference orbit which is the closed orbit for lattices with a closed geometry and for
lattices with an open geometry the reference orbit is the orbit for some given initial
position. Orbits here are calculated with radiation damping but ignoring stochastic effects. The
transfer map from $s_1$ to $s_2$ will be of the form
\begin{equation}
  \delta\bfr_2 = \calM_{21}(\delta\bfr_1) + \Cal{S}_{21} \Bf\Xi
  \label{rmrsx}
\end{equation}
where $\delta\bfr_1$ and $\delta\bfr_2$ are the particle positions with respect to the reference
orbit at $s_1$ and $s_2$ respectively and $\calM_{21}$ is the transfer map with damping. The
stochastic radiation part is represented by a $6\times6$ matrix $\Cal{S}$ times a $6$-vector
\begin{equation}
  \Bf\Xi = (\xi_1, \xi_2, \xi_3, \xi_4, \xi_5, \xi_6)
\end{equation}
with each $\xi_i$ being an independent Gaussian distributed random number with unit sigma and zero
mean. The stochastic transport (second term in \Eq{rmrsx}) is treated here only in lowest
order. This is a good approximation as long as the radiation emitted is small enough in the region
between $s_1$ and $s_2$. This is true for nearly all practical cases. In the case where this
approximation fails, the equilibrium beam distribution would not be Gaussian and the standard
radiation integral treatment (\sref{s:synch.ints}), which relies on this approximation, would not be
valid.

The transfer map with damping $\calM$ is calculated by adding in the effect of the damping
(\Eqs{pk1pp}) when integrating the equations of motion to form the map. Through a given lattice
element, it is generally very safe to assume that the change in energy is small compared to the
energy of a particle. Thus the matrix $\bfM$ through an element, which is the first order part of
$\calM$, can is computed via first order perturbation theory to be
\begin{equation}
  \bfM = \bfT + \bfZ
  \label{mtz}
\end{equation}
where $\bfT$ is the transfer matrix without damping and $\bfZ$ is the change in $\bfT$ due
to damping computed via
\begin{equation}
  \bfZ = \int_{s_1}^{s_2} ds \, \bfT_{2,s} \, \bfd(s) \, \bfT_{s,1}
  \label{zstd}
\end{equation}
where $s_1$ and $s_2$ are the longitudinal positions at the ends of the element and
the local damping matrix $\bfd$ is computed from \Eqs{pk1pp}
\begin{equation}
  \bfd = -k_d \, \begin{pmatrix}
    0                           & 0           & 0                           & 0           & 0 & 0       \\
    \frac{dg^2}{dx} p_x (1+p_z) & g^2 (1+p_z) & \frac{dg^2}{dy} p_x (1+p_z) & 0           & 0 & g^2 p_x \\
    0                           & 0           & 0                           & 0           & 0 & 0       \\
    \frac{dg^2}{dx} p_y (1+p_z) & 0           & \frac{dg^2}{dy} p_y (1+p_z) & g^2 (1+p_z) & 0 & g^2 p_y \\
    0                           & 0           & 0                           & 0           & 0 & 0       \\
    \frac{dg^2}{dx} (1+p_z)^2   & 0           & \frac{dg^2}{dy} (1+p_z)^2   & 0           & 0 & 2 g^2 (1+p_z) \\
  \end{pmatrix}
\end{equation}
All quantities are evaluated on the closed orbit. Notice that since $\calM_{21}$ is computed with
respect to the beam centroid orbit, there is no constant part to the map. Since $\bfT_{21}$ is
invertible, \Eq{mtz} can be written in the form
\begin{equation}
  \bfM_{21} = \bigl( \boldsymbol{1} + \bfZ_{21} \, \bfT_{21}^{-1} \bigr) \, \bfT_{21} 
  \equiv \bfD_{21} \, \bfT_{21} 
\end{equation}
$\bfD$ is defined by this equation. The 1-turn damping decrement $\alpha$ for each mode $a$, $b$,
and $c$ of oscillation can be calculated from $\bfD$ using Eq.~(86) of Ohmi\cite{b:ohmi}.

The $\Cal{S}$ matrix (\Eq{rmrsx}) is calculated by first noting that, to linear order, the
distribution of $\delta\bfr_2$ due to stochastic radiation over some length $ds$ as some point $s$
is
\begin{equation}
  \delta\bfr_2 = \sqrt{ds} \, \bfM_{2,s} \, \left( \bfF_\mk{f}(s) \, \xi_1 + \bfF_\mk{v} \, \xi_2 \right)
\end{equation}
where $\bfM_{2,s}$ is the first order part (matrix) of the map $\calM_{2,s}$ from $s$ to $s_2$,
$\xi_1$ and $\xi_2$ are two independent Gaussian random numbers with unit sigma and zero mean, and
$\bfF_\mk{f}$ and $\bfF_\mk{v}$ are (see \sref{s:radiation})
\begin{align}
  \bfF_\mk{f} &= \sqrt{k_\mk{f} g_0^3} \, (0, p_x \, (1 + p_z), 0, p_y \, (1 + p_z), 0, (1 + p_z)^2) \\
  \bfF_\mk{v} &= \sqrt{k_v g_0} \, (0, -g_y, 0, g_x, 0, 0)
  \label{fk000}
\end{align}
where $k_\mk{f}$ $p_x$, $p_y$ and $p_z$ are to be evaluated on the reference orbit and $(g_x, g_y)$ is the
curvature vector which points away from the center of curvature of the particle's orbit. Notice that since
$\delta\bfr$ is, by definition, the deviation from the reference orbit, $p_x = r_2$ and $p_y = r_4$
will be zero on the reference orbit. The covariance matrix $\bfsig_\gamma$ is defined by
$\sigma_{\gamma ij} \equiv \langle r_i \, r_j \rangle_\gamma$ where $\langle \ldots \rangle_\gamma$
is an average over the photon emission spectrum. The contribution, $\bfsig_{\gamma21}$, to the
covariance matrix at $s_2$ due to the stochastic emission over the region between $s_1$ and $s_2$,
is
\begin{equation}
  \bfsig_{\gamma21} = \int_{s_1}^{s_2} ds \, 
    \bfM_{2,s} \, \big[ 
    \bfF_\mk{f}(s) \, \bfF_\mk{f}^t(s) + \bfF_\mk{v}(s) \, \bfF_\mk{v}^t(s) \big] \, \bfM_{2,s}^t
  \label{smvvm}
\end{equation}
where the $t$ superscript indicates transpose. $\bfsig_{\gamma21}$ is related to $\Cal{S}$ via
\begin{equation}
  \bfsig_{\gamma21} = \Cal{S}_{21} \, \Cal{S}_{21}^t
  \label{sxx}
\end{equation}
The calculation of $\Cal{S}_{21}$ involves calculating $\bfsig_{\gamma21}$ via \Eq{smvvm} and then
using \Eq{sxx} to solve for $\Cal{S}_{21}$ using, say, a Cholesky decomposition. Notice that while
\Eq{sxx} does not have a unique solution, what matters here is that $\Cal{S}_{21} \, \Bf\Xi$ (see
\Eq{rmrsx}) gives the correct distribution. The $\Cal{S}_{21}$ matrix may contain columns or rows
that are all zero. This can happen if there are vectors $\bfz$ where $\bfz^t \bfsig_{\gamma21} \bfz$
is zero. For example, in a planer ring where the vertical emittance is zero there will be rows that
are zero.

The covariance matrix $\bfsig_\gamma(s_2)$ at $s_2$ relative to the covariance matrix at $s_1$ is
\begin{equation}
  \bfsig_\gamma(s_2) = \bfsig_{\gamma21} + \bfM_{21} \, \bfsig_\gamma(s_1) \, \bfM_{21}^t
  \label{ssmsm}
\end{equation}
The beam size matrix $\bfsig$ is not the same as the covariance matrix since the beam size matrix is
an average over the particles of a beam and not an average over the photon emission
spectrum. However, in equilibrium, the two are the same. To calculate the equilibrium beam size
matrix, \Eq{ssmsm} is recast. For any symmetric $6\times6$ matrix $\bfA$, define the 21-vector
$\bfV(\bfA)$ by
\begin{equation}
  \bfV(\bfA) \equiv (A_{11}, A_{12}, \ldots, A_{16}, A_{22}, A_{23}, \ldots, A_{56}, A_{66})
\end{equation}
With $s_1 = s_2 = s$, and using \Eq{ssmsm}, the equilibrium beam size matrix can be calculated via
\begin{equation}
  \bfV(\bfsig(s)) = \bfV(\bfsig_{\gamma ss}) + \wt\bfM \, \bfV(\bfsig(s))
  \label{vsmv}
\end{equation}
where the $21\times21$ matrix $\wt\bfM$ is defined so that for any symmetric $\bfA$, $\wt\bfM \, \bfV(\bfA) =
\bfV(\bfM\bfA\bfM^t)$. That is
\begin{equation}
  \wt\bfM = \begin{pmatrix}
    M_{11}^2      & 2 M_{11} M_{12} & \cdots & 2 M_{15} M_{16} & M_{16} M_{16} \\
    M_{11} M_{21} & 2 M_{11} M_{22} & \cdots & 2 M_{15} M_{26} & M_{16} M_{26} \\
    \vdots        & \vdots          & \ddots & \vdots          & \vdots      \\
    M_{51} M_{61} & 2 M_{51} M_{62} & \cdots & 2 M_{55} M_{66} & M_{56} M_{66} \\
    M_{61} M_{61} & 2 M_{61} M_{62} & \cdots & 2 M_{65} M_{66} & M_{66} M_{66}
  \end{pmatrix}
\end{equation}
\Eq{vsmv} is linear in the unknown $\bfV(\bfsig)$ and is easily solved.

The emittances can be calculated from the eigenvalues of the matrix $\bfsig \, \bfS$
(Wolski\cite{b:wolski.coupling} Eq.~30) where $\bfS$ is given by \Eq{s010000}. Specifically, the
eigenvalues of $\bfsig \, \bfS$ are pure imaginary and, using the eigenvector ordering given by
\Eq{vsvi0} (which is opposite that of Wolski), the emittances are the imaginary part of the odd
eigenvalues $(\epsilon_a, \epsilon_b, \epsilon_c) = (\im{\lambda_1}, \im{\lambda_3}, \im{\lambda_5})$.

Unlike the case where radiation is ignored and the motion is symplectic, the calculated emittances
along with the beam size will vary from point to point in a manner similar to the variation of the average
beam energy (sawtooth effect) around the ring (\sref{s:radiation}).

The emittance calculation makes a number of approximations. One approximation is embodied in
\Eq{zstd} which assumes that the damping is weak enough so that second order and higher terms can be
neglected. Another approximation is that, within the extent of the beam, the damping as a function
of transverse position is linear. That is, the effect of the damping is well represented by the
matrix $\bfd$ in \Eq{zstd}. The third major assumption is that, within the extent of the beam, the
stochastic kick coefficient $\bfF_\mk{f}$ (\Eq{fk000}) is independent of the transverse coordinates.
Other approximations involve the assumption of linearity of the guide fields and the ignoring of any
resonance or wakefield effects. To the extent that these assumptions are violated, this will lead to
a non-Gaussian beam shape.

%-----------------------------------------------------------------
\section{Synchrotron Radiation Integrals}
\label{s:synch.ints}
\index{synchrotron radiation!integrals}

the synchrotron radiation integrals can be used to compute emittances, the energy spread,
etc. However, using the 6D damped and stochastic transport matrices (\sref{s:map.rad}) has
a number of advantages
\begin{itemize}
%
\item
Unlike the radiation integrals, the 6D calculation does not make the approximation that the
synchrotron frequency is negligible. Therefore, the 6D calculation will be more accurate.
%
\item
The 6D calculation is simpler: Not as many integrals needed (only 2) and the 6D calculation does not
depend upon calculation of the Twiss parameters.
%
\item
When doing any lattice design which involves constraining the emittances: Since the integrals of the
6D calculation are local (the integrations through any given lattice element are only dependent upon
the properties of that lattice element), by caching integrals element-by-element, the computation of
the emittances can be speeded up. That is, in a design problem, only the parameters of some subset
of all the lattice elements will be varied (for example, a design may only involve varying the
strength of quadrupoles), only this subset of elements needs to have their integrals recomputed.  On
the other hand, the radiation integrals are dependent on the Twiss, dispersion, and coupling
parameters which make the integrals nonlocal.
%
\item
The 6D formalism can be used to construct transport maps with radiation damping and excitation for
efficient particle tracking.
\end{itemize}

The standard radiation formulas assume no coupling between the horizontal and vertical
plains\cite{b:helm,b:jowett}. With coupling, the equations need to be generalized and this is
detailed below.

\index{dispersion}
In the general case, the curvature vector $\bfg = (g_x, g_y)$, which points away from the center of
curvature of the particle's orbit and has a magnitude of $|\bfg| = 1/\rho$, where $\rho$ is the
radius of curvature (see \fig{f:local.coords}), does not lie in the horizontal plane. Similarly, the
dispersion $\bfeta\two = (\eta_x, \eta_y)$ will not lie in the horizontal plane. With this notation,
the synchrotron integrals for coupled motion are:
  \begingroup
  \allowdisplaybreaks
  \begin{align}
    I_0 &= \oint ds \, \gamma_0 \, g \\
    I_1 &= \oint ds \, \bfg \dotproduct \bfeta 
         \equiv \oint ds \, (g_x \, \eta_x + g_y \, \eta_y) \\
    I_2 &= \oint ds \, g^2 \\
    I_3 &= \oint ds \, g^3 \\
    I_{4a} &= \oint ds \, \left[ g^2 \, \bfg \dotproduct \bfeta\two_a + 
         \nabla g^2 \dotproduct \bfeta\two_a \right] \\
    I_{4b} &= \oint ds \, \left[ g^2 \, \bfg \dotproduct \bfeta\two_b + 
         \nabla g^2 \dotproduct \bfeta\two_b \right] \\
    I_{4z} &= \oint ds \, \left[ g^2 \, \bfg \dotproduct \bfeta\two + 
         \nabla g^2 \dotproduct \bfeta\two \right] \\
    I_{5a} &= \oint ds \, g^3 \, \calH_a \\
    I_{5b} &= \oint ds \, g^3 \, \calH_b \\
    I_{6b} &= \oint ds \, g^3 \, \beta_b
  \end{align}
  \endgroup
where $\gamma_0$ is that usual relativistic factor and $\calH_a$ is 
  \begin{equation}
    \calH_a = \gamma_a \, \eta_a^2 + 2 \, \alpha_a \, \eta_a \, \eta_a' + 
      \beta_a \eta_a'^2 
  \end{equation}
with a similar equation for $\calH_b$. Here $\bfeta\two_a =
(\eta_{ax}, \eta_{ay})$, and $\bfeta\two_b = (\eta_{bx}, \eta_{by})$
are the dispersion vectors for the $a$ and $b$ modes respectively in
$x$--$y$ space (these 2--vectors are not to be confused with the
dispersion 4--vectors used in the previous section). The position
dependence of the curvature function is:
  \begin{align}
    g_x(x,y) = g_{x} + x \, k_1 + y \, s_1 \CRNO
    g_y(x,y) = g_{y} + x \, s_1 - y \, k_1 
  \end{align}
where $k_1$ is the quadrupole moment and $s_1$ is the skew--quadrupole moment.
Using this gives on--axis ($x = y = 0$)
  \begin{equation}
    \nabla g^2 = 2 \left( g_x k_1 + g_y s_1, \, g_x s_1 - g_y k_1 \right)
    \label{g2gkg}
  \end{equation}
Note: The above equations must be modified in places in the lattice where there are mode flips
(\sref{s:coupling} since an individual integral must be evaluated using the same physical mode
throughout the lattice.

$I_0$ is not a standard radiation integral. It is useful, though, in calculating the average number
of photons emitted. For electrons:
  \begin{equation}
    {\cal N} = \frac{5 \: \rc m \, c^2}{2 \sqrt{3} \, \hbar \, c} \, I_0 
  \end{equation}
where $\cal N$ is the average number of photons emitted by a particle over one turn, and $\rc$ is
the the particle's ``classical radius'' given by \Eq{rq4pe}.

In a dipole a non--zero $e_1$ or $e_2$ gives a contribution to $I_4$ via the $\nabla g^2 \dotproduct
\bfeta$ term. The edge field is modeled as a thin quadrupole of length $\delta$ and strength $k = -g
\, \tan(e) / \delta$. It is assumed that $\bfg$ rises linearly within the edge field from zero on
the outside edge of the edge field to its full value on the inside edge of the edge field. Using
this in \Eq{g2gkg} and integrating over the edge field gives the contribution to $I_4$ from a
non--zero $e_1$ as
  \begin{equation}
    I_{4z} = -\tan(e_1) \, g^2
    \left( \cos(\theta) \, \eta_x + \sin(\theta) \, \eta_y \right)
    \label{iegct}
  \end{equation}
With an analogous equation for a finite $e_2$. The extension to $I_{4a}$ and $I_{4b}$ involves using
$\bfeta\two_a$ and $\bfeta\two_b$ in place of $\bfeta\two$.  In \Eq{iegct} $\theta$ is the reference
\vn{tilt} angle which is non--zero if the bend is not in the horizontal plane. Here use of the fact
has been made that the $\bfg$ vector rotates as $\theta$ and the quadrupole and skew quadrupole
strengths rotate as $2\, \theta$.

The above integrals are invariant under rotation of the $(x,y)$ coordinate system and reduce to the
standard equations when $g_y = 0$ as they should.

There are various parameters that can be expressed in terms of these integrals. The $I_1$ integral
can be related to the momentum compaction $\alpha_p$ via
  \begin{equation}
    I_1 = L \, \frac{dL/L}{dp/p} = L \, \alpha_p
  \end{equation}
where $p$ is the momentum and $L$ is the ring circumference. The can be related to the time slip
factor $\eta_p$ by
\begin{equation}
  \eta_p = \frac{dt/t}{dp/p} = \alpha_p - \frac{1}{\gamma^2}
\end{equation}

The energy loss per turn is related to $I_2$ via
  \begin{equation}
    U_0 = \frac{2 \, \rc E_0^4}{3 \, (mc^2)^3} I_2
  \end{equation}
where $E_0$ is the nominal energy.

The damping partition numbers are related to the radiation integrals via
  \begin{equation}
    J_a = 1 - \frac{I_{4a}}{I_2} \comma \quad
    J_b = 1 - \frac{I_{4b}}{I_2} \comma \, \text{and} \quad \label{j1ii}
    J_z = 2 + \frac{I_{4z}}{I_2} \period
  \end{equation}
Since 
  \begin{equation}          
    \bfeta\two_{a} + \bfeta\two_{b} = \bfeta\two
    \comma \label{eee}
  \end{equation}
Robinson's theorem, $J_a + J_b + J_z = 4$, is satisfied.
Alternatively, the exponential damping coefficients per turn are
  \begin{equation}
    \alpha_a = \frac{U_0 \, J_a}{2 E_0} \comma \quad
    \alpha_b = \frac{U_0 \, J_b}{2 E_0} \comma \, \text{and} \quad
    \alpha_z = \frac{U_0 \, J_z}{2 E_0} \period
  \end{equation}
The energy spread is given by
  \begin{equation}
    \sigma_{pz}^2 = \left( \frac{\sigma_E}{E_0} \right)^2 = 
    C_q \gamma_0^2 \frac{I_3}{2I_2 + I_{4z}}
  \end{equation}
where $\gamma_0$ is the usual energy factor and 
  \begin{equation}
    C_q = \frac{55}{32 \, \sqrt{3}} \, \frac{\hbar}{mc} = 
    3.832 \times 10^{-13} \, \text{meter for electrons}
  \end{equation}
If the synchrotron frequency is not too large, the bunch length is given by
  \begin{equation}
    \sigma_z^2 = \frac{I_1}{M(6,5)} \, \sigma_{pz}^2
  \end{equation}
where $M(6,5)$ is the $(6,5)$ element for the 1--turn transfer matrix
of the storage ring. Finally, the emittances are given by
  \begin{align}
    \epsilon_a &= \frac{C_q}{I_2 - I_{4a}} 
      \, \gamma_0^2 \, I_{5a} \CRNO
    \epsilon_b &= \frac{C_q}{I_2 - I_{4b}} 
      \, \left( \gamma_0^2 \, I_{5b} + \frac{13}{55} \, I_{6b} \right)
  \end{align}
The $I_{6b}$ term come from the finite vertical opening angle of the
radiation\cite{b:tol}. Normally this term is very small compared to
the emittance due to coupling or vertical kicks due to magnet misalignment.

For a non-circular machine, radiation integrals are still of interest
if there are bends or steering elements. However, in this case, the
appropriate energy factors must be included to take account any
changes in energy due to any \vn{lcavity} elements.  For a
non-circular machine, the $I_1$ integral is not altered and the $I_4$
integrals are not relevant. The other integrals become
  \begin{align}
    L_2 &= \int ds \, g^2 \, \gamma_0^4 \\
    L_3 &= \int ds \, g^3 \, \gamma_0^7 \\
    L_{5a} &= \int ds \, g^3 \, \calH_a \, \gamma_0^6 \\
    L_{5b} &= \int ds \, g^3 \, \calH_b \, \gamma_0^6
  \end{align}
In terms of these integrals, the energy loss through the lattice is
  \begin{equation}
    U_0 = \frac{2 \, \rc \, mc^2}{3} L_2
  \end{equation}
The energy spread assuming $\sigma_E$ is zero at the start and neglecting
any damping is
  \begin{equation}
    \sigma_E^2 = \frac{4}{3} \, C_q \, \rc \, \left( m c^2 \right)^2 \, L_3
  \end{equation}
The above equation is appropriate for a linac. In a storage ring, where
there are energy oscillations, the growth of $\sigma_E^2$ due to
quantum excitation is half that. One way to explain this is that in a
storage ring, the longitudinal motion is ``shared'' between the $z$ and
$pz$ coordinates and, to preserve phase space volume, this reduces
$\sigma_E^2$ by a factor of 2.

Again neglecting any initial beam width, the transverse beam size
at the end of the lattice is
  \begin{align}
    \epsilon_a &= \frac{2}{3} \, C_q \, \rc \, 
    \frac{L_{5a}}{\gamma_f} \CRNO
    \epsilon_b &= \frac{2}{3} \, C_q \, \rc \, 
    \frac{L_{5b}}{\gamma_f} 
  \end{align}
Where $\gamma_f$ is the final gamma.
