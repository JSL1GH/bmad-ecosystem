\chapter{Synchrotron Radiation}

%-----------------------------------------------------------------
\section{Synchrotron Radiation Damping and Excitation}
\label{s:radiation}
\index{synchrotron radiation!damping and excitation|hyperbf}

Emission of synchrotron radiation by a particle can be decomposed into two parts. The deterministic
average energy emitted produces damping while the stochastic fluctuating part of the energy spectrum
produces excitation\cite{b:jowett}.

\index{MAD!radiation}
The treatment of radiation damping by \bmad essentially follows \mad. The energy loss through and element is
modeled via 
\begin{equation}
  \frac{\Delta E}{E_0} = -k_E \equiv -(k_d + k_f \, \xi) \, (1 + p_z)^2
\end{equation}
where $k_d$ gives the deterministic part of the emission, $\xi$ is a Gaussian distributed random
number with unit sigma and zero mean, and $k_f$ is the amplitude of the stochastic part of the
emission. The deterministic and stochastic parts of the emission can be included or excluded from a
tracking simulation by setting in a lattice file the \bmad global parameters (\sref{s:bmad.common})
\begin{example}
  bmad_com[radiation_damping_on]      = True or False  ! Deterministic part on/off.
  bmad_com[radiation_fluctuations_on] = True or False  ! Stochastic part on/off.
  bmad_com[radiation_zero_average]    = True or False  ! Make ave radiation kick zero.
\end{example}

Values for $k_d$ and $k_f$ are calculated via the equations
\begin{align}
  k_d &= \frac{2 \, r_e}{3} \, \gamma_0^3 \, \ave{g_0^2} \, L_p
    \label{k2r3g} \\
  k_f &= \left( \frac{55 \, r_e \, \hbar}{24 \, \sqrt{3} \, m_e \, c} \, 
    L_p \, \gamma_0^5 \ave{g_0^3} \right)^{1/2}
    \label{k55rh}
\end{align}
where $r_e$ is the classical electron radius, $L_p$ is the actual path length, $\gamma_0$ is the energy
factor of an on-energy particle, $1/g_0$ is the bending radius of an on--energy particle, and
$\ave{g_0^2}$ is an average of $g_0^2$ over the actual path.

Since the radiation is emitted in the forward direction the angular orientation of the particle
motion is invariant which leads to the following equations for the changes in momentum phase space
coordinates
\begin{align}
  \Delta p_x &= -\frac{k_E}{1 + p_z} \, p_x , \qquad  
    \Delta p_y = -\frac{k_E}{1 + p_z} \, p_y \CRNO
  \Delta p_z &\approx \frac{\Delta E}{E_0} = -k_E 
\end{align}

The above formalism does not take into account the fact that radiation is emitted with a $1/\gamma$
angular distribution. This means that the calculated vertical emittance for a lattice with bends
only in the horizontal plane and without any coupling elements such as skew quadrupoles will be
zero. However, in practice, the vertical emittance will be dominated by coupling so this
approximation is generally a good one.

Synchrotron radiation emission involves energy loss and this energy loss leads to what is known as
the energy ``sawtooth'' effect where the curve of particle energy on the closed orbit as a function
of longitudinal position has a sawtooth shape. A sawtooth pattern can also be generally seen in the
horizontal orbit. It is sometimes convenient in simulations to eliminate the sawtooth effect. This
can be done by shifting the photon emission spectrum at any given element to have zero average
energy loss along the closed orbit. For this calculation the closed orbit should be the closed orbit
as calculated without radiation damping (in other words the closed orbit without the sawtooth). In
this case, $k_E$ is calculated by
\begin{equation}
  k_E = (k_d + k_f \, \xi) \, (1 + p_z)^2 - k_{d0}
\end{equation}
where $k_{d0}$ is $k_d$ evaluated along the closed orbit. In practice, for the calculation, \bmad
approximates the closed orbit as the zero orbit. 

The global parameter \vn{bmad_com[radiation_zero_average]} controls the shifting of the photon
spectrum to have zero average. Currently, the shifting of the spectrum only works for non PTC
dependent tracking. That is, the shifting is not applicable to tracking with Taylor maps and with
\vn{symp_lie_ptc} (\sref{s:tkm}) tracking.

%-----------------------------------------------------------------
\section{Synchrotron Radiation Integrals}
\label{s:synch.ints}
\index{synchrotron radiation!integrals}

The synchrotron radiation integrals are used to compute emittances,
the energy spread, etc. The standard formulas assume no coupling
between the horizontal and vertical planes\cite{b:helm,b:jowett}. With
coupling, the equations need to be generalized and this is detailed
below.

\index{dispersion}
In the general case, the curvature vector $\bfg = (g_x, g_y)$, which
points away from the center of curvature of the particle's orbit and
has a magnitude of $|\bfg| = 1/\rho$, where $\rho$ is the radius of
curvature (see \fig{f:local.coords}), does not lie in the
horizontal plane. Similarly, the dispersion $\bfeta\two =
(\eta_x, \eta_y)$ will not lie in the horizontal plane. With this
notation, the synchrotron integrals for coupled motion are:
  \begingroup
  \allowdisplaybreaks
  \begin{align}
    I_0 &= \oint ds \, \gamma_0 \, g \\
    I_1 &= \oint ds \, \bfg \dotproduct \bfeta 
         \equiv \oint ds \, (g_x \, \eta_x + g_y \, \eta_y) \\
    I_2 &= \oint ds \, g^2 \\
    I_3 &= \oint ds \, g^3 \\
    I_{4a} &= \oint ds \, \left[ g^2 \, \bfg \dotproduct \bfeta\two_a + 
         \nabla g^2 \dotproduct \bfeta\two_a \right] \\
    I_{4b} &= \oint ds \, \left[ g^2 \, \bfg \dotproduct \bfeta\two_b + 
         \nabla g^2 \dotproduct \bfeta\two_b \right] \\
    I_{4z} &= \oint ds \, \left[ g^2 \, \bfg \dotproduct \bfeta\two + 
         \nabla g^2 \dotproduct \bfeta\two \right] \\
    I_{5a} &= \oint ds \, g^3 \, \calH_a \\
    I_{5b} &= \oint ds \, g^3 \, \calH_b \\
    I_{6b} &= \oint ds \, g^3 \, \beta_b
  \end{align}
  \endgroup
where $\gamma_0$ is that usual relativistic factor and $\calH_a$ is 
  \begin{equation}
    \calH_a = \gamma_a \, \eta_a^2 + 2 \, \alpha_a \, \eta_a \, \eta_a' + 
      \beta_a \eta_a'^2 
  \end{equation}
with a similar equation for $\calH_b$. Here $\bfeta\two_a =
(\eta_{ax}, \eta_{ay})$, and $\bfeta\two_b = (\eta_{bx}, \eta_{by})$
are the dispersion vectors for the $a$ and $b$ modes respectively in
$x$--$y$ space (these 2--vectors are not to be confused with the
dispersion 4--vectors used in the previous section). The position
dependence of the curvature function is:
  \begin{align}
    g_x(x,y) = g_{x} + x \, k_1 + y \, s_1 \CRNO
    g_y(x,y) = g_{y} + x \, s_1 - y \, k_1 
  \end{align}
where $k_1$ is the quadrupole moment and $s_1$ is the skew--quadrupole moment.
Using this gives on--axis ($x = y = 0$)
  \begin{equation}
    \nabla g^2 = 2 \left( g_x k_1 + g_y s_1, \, g_x s_1 - g_y k_1 \right)
    \label{g2gkg}
  \end{equation}

$I_0$ is not a standard radiation integral. It is useful,
though, in calculating the average number of photons emitted. For electrons:
  \begin{equation}
    {\cal N} = \frac{5 \: r_{\! f}}{2 \sqrt{3} \, \hbar \, c} \, I_0 
  \end{equation}
where $\cal N$ is the average number of photons emitted by a particle
over one turn, and the ``classical radius factor'' $r_{\! f}$ is 
  \begin{equation}
    r_{\! f} = \frac{e^2}{4 \, \pi \, \epsilon_0} 
  \end{equation}
$r_{\! f}$ has a value of $1.4399644 \cdot 10^{-9} \; \text{meters-eV}$
for all particles of charge $\pm 1$.

In a dipole a non--zero $e_1$ or $e_2$ gives a contribution to $I_4$
via the $\nabla g^2 \dotproduct \bfeta$ term. The edge field is modeled as a
thin quadrupole of length $\delta$ and strength $k = -\tan(e) /
\delta$. It is assumed that $\bfg$ rises linearly within the edge field
from zero on the outside edge of the edge field to its full value on the inside 
edge of the edge field. 
Using this in \Eq{g2gkg} and integrating over the edge field gives the contribution
to $I_4$ from a non--zero $e_1$ as
  \begin{equation}
    I_{4z} = -\tan(e_1) \, g^2
    \left( \cos(\theta) \, \eta_x + \sin(\theta) \, \eta_y \right)
    \label{iegct}
  \end{equation}
With an analogous equation for a finite $e_2$. The extension to
$I_{4a}$ and $I_{4b}$ involves using $\bfeta\two_a$ and $\bfeta\two_b$
in place of $\bfeta\two$.  In \Eq{iegct} $\theta$ is the \vn{tilt}
angle which is non--zero if the bend is not in the horizontal plane.

The above integrals are invariant under rotation of the $(x,y)$ coordinate
system and reduce to the standard equations when $g_y = 0$ as they should.

There are various parameters that can be expressed in terms of these
integrals.  The $I_1$ integral can be related to the momentum
compaction $\alpha_p$ via
  \begin{equation}
    I_1 = \alpha_p \, L
  \end{equation}
where $L$ is the storage ring circumference. The energy loss per turn is
  \begin{equation}
    U_0 = \frac{2 \, r_e E_0^4}{3 \, (mc^2)^3} I_2
  \end{equation}
where $E_0$ is the nominal energy and $r_e$ is the classical electron
radius (electrons are assumed here but the formulas are easily
generalized).

The damping partition numbers are
  \begin{equation}
    J_a = 1 - \frac{I_{4a}}{I_2} \comma \quad
    J_b = 1 - \frac{I_{4b}}{I_2} \comma \, \text{and} \quad \label{j1ii}
    J_z = 2 + \frac{I_{4z}}{I_2} \period
  \end{equation}
Since 
  \begin{equation}          
    \bfeta\two_{a} + \bfeta\two_{b} = \bfeta\two
    \comma \label{eee}
  \end{equation}
Robinson's theorem, $J_a + J_b + J_z = 4$, is satisfied.
Alternatively, the exponential damping coefficients per turn are
  \begin{equation}
    \alpha_a = \frac{U_0 \, J_a}{2 E_0} \comma \quad
    \alpha_b = \frac{U_0 \, J_b}{2 E_0} \comma \, \text{and} \quad
    \alpha_z = \frac{U_0 \, J_z}{2 E_0} \period
  \end{equation}
The energy spread is given by
  \begin{equation}
    \sigma_{pz}^2 = \left( \frac{\sigma_E}{E_0} \right)^2 = 
    C_q \gamma_0^2 \frac{I_3}{2I_2 + I_{4z}}
  \end{equation}
where $\gamma_0$ is the usual energy factor and 
  \begin{equation}
    C_q = \frac{55}{32 \, \sqrt{3}} \, \frac{\hbar}{mc} = 
    3.84 \times 10^{-13} \, \text{meter for electrons}
  \end{equation}
If the synchrotron frequency is not too large, the bunch length is given by
  \begin{equation}
    \sigma_z^2 = \frac{I_1}{M(6,5)} \, \sigma_{pz}^2
  \end{equation}
where $M(6,5)$ is the $(6,5)$ element for the 1--turn transfer matrix
of the storage ring. Finally, the emittances are given by
  \begin{align}
    \epsilon_a &= \frac{C_q}{I_2 - I_{4a}} 
      \, \gamma_0^2 \, I_{5a} \CRNO
    \epsilon_b &= \frac{C_q}{I_2 - I_{4b}} 
      \, \left( \gamma_0^2 \, I_{5b} + \frac{13}{55} \, I_{6b} \right)
  \end{align}
The $I_{6b}$ term come from the finite vertical opening angle of the
radiation\cite{b:tol}. Normally this term is very small compared to
the emittance due to coupling or vertical kicks due to magnet misalignment.

For a non-circular machine, radiation integrals are still of interest
if there are bends or steering elements. However, in this case, the
appropriate energy factors must be included to take account any
changes in energy due to any \vn{lcavity} elements.  For a
non-circular machine, the $I_1$ integral is not altered and the $I_4$
integrals are not relevant. The other integrals become
  \begin{align}
    L_2 &= \int ds \, g^2 \, \gamma_0^4 \\
    L_3 &= \int ds \, g^3 \, \gamma_0^7 \\
    L_{5a} &= \int ds \, g^3 \, \calH_a \, \gamma_0^6 \\
    L_{5b} &= \int ds \, g^3 \, \calH_b \, \gamma_0^6
  \end{align}
In terms of these integrals, the energy loss through the lattice is
  \begin{equation}
    U_0 = \frac{2 \, r_e \, mc^2}{3} L_2
  \end{equation}
The energy spread assuming $\sigma_E$ is zero at the start and neglecting
any damping is
  \begin{equation}
    \sigma_E^2 = \frac{4}{3} \, C_q \, r_e \, \left( m c^2 \right)^2 \, L_3
  \end{equation}
The above equation is appropriate for a linac. In a storage ring, where
there are energy oscillations, the growth of $\sigma_E^2$ due to
quantum excitation is half that. One way to explain this is that in a
storage ring, the longitudinal motion is ``shared'' between the $z$ and
$pz$ coordinates and, to preserve phase space volume, this reduces
$\sigma_E^2$ by a factor of 2.

Again neglecting any initial beam width, the transverse beam size
at the end of the lattice is
  \begin{align}
    \epsilon_a &= \frac{2}{3} \, C_q \, r_e \, 
    \frac{L_{5a}}{\gamma_f} \CRNO
    \epsilon_b &= \frac{2}{3} \, C_q \, r_e \, 
    \frac{L_{5b}}{\gamma_f} 
  \end{align}
Where $\gamma_f$ is the final gamma.
