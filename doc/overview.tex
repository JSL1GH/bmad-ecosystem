\section*{Overview}
\index{Bmad|hyperbf}
\pdfbookmark[1]{Overview}{Overview}

\bmad (Otherwise known as ``Baby MAD" or ``Better MAD" or just plain
``Be MAD!") is a subroutine library for relativistic
charged--particle and X-Ray simulations in high energy accelerators and
storage rings. \bmad has been developed at Cornell University's
Laboratory for Elementary Particle Physics and has been in use since
1996.

Prior to the development of \bmad, simulation programs at Cornell were
written almost from scratch to perform calculations that were beyond
the capability of existing, generally available software. This
practice was inefficient, leading to much duplication of effort.
Since the development of simulation programs was time consuming,
needed calculations where not being done.  
As a response, the \bmad subroutine library, using an
object oriented approach and written in Fortran 2008, were developed.
The aim of the \bmad project was to:
\begin{Itemize}
\item Cut down on the time needed to develop programs.
\item Cut down on programming errors.
\item Provide a simple mechanism for lattice function calculations
from within control system programs.
\item Provide a flexible and powerful lattice input format.
\item Standardize sharing of lattice information between 
programs.
\end{Itemize}

\bmad can be used to study both single and multi--particle beam dynamics as well as X-rays. 
Over the years, \bmad modules have been developed for simulating a wide variety of phenomena 
including intra beam scattering (IBS), coherent synchrotron radiation (CSR), Wakefields, 
Touschek scattering, higher order mode (HOM) resonances, etc., etc.
\bmad has various
tracking algorithms including Runge--Kutta and symplectic (Lie
algebraic) integration.  Wake fields, and radiation excitation and
damping can be simulated. \bmad has routines for calculating transfer
matrices, emittances, Twiss parameters, dispersion, coupling, etc. The
elements that \bmad knows about include quadrupoles, RF cavities (both
storage ring and LINAC accelerating types), solenoids, dipole bends,
Bragg crystals etc. 
In addition, elements can be defined to control the attributes of
other elements. This can be used to simulate the ``girder'' which
physically support components in the accelerator or to easily simulate
the action of control room ``knobs'' that gang together, say, the
current going through a set of quadrupoles.

One current area of development for \bmad is X-ray simulation. To that
end, new element classes have been defined including a \vn{mirror}
element and a \vn{crystal} element for simulations of crystal
diffraction. The ultimate aim is to develop a environment where \bmad
can be used for simulations starting from electron generation from a
cathode, to X-ray generation in Wigglers and other elements, to X-ray
tracking through to the experimental end stations.

To be able to extend \bmad easily, \bmad has been developed in a
modular, object oriented, fashion to maximize flexibility. As just one
example, each individual element can be assigned a particular tracking
method in order to maximize speed or accuracy and the tracking methods
can be assigned via the lattice file or at run time in a program.

\vfill
\break
