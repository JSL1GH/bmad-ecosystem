\chapter{Overview: Starting and Running Tao}
\label{c:overview.tao}

%----------------------------------------------------------------
\section{Tao Setup}
\label{s:obtaining}

Instructions for obtaining and for setting up \tao can be found at:
\begin{example}
  classe.cornell.edu/bmad/
\end{example}

%----------------------------------------------------------------
\section{Tao Tutorial}
\label{s:tutorial}

This manual is organized more as a reference guide than as a tutorial so for an introduction
to \tao (and \bmad) there is a link on the web page at:
\begin{example}
  classe.cornell.edu/bmad/tao.html
\end{example}

%-----------------------------------------------------------------
\section{Command Line Initialization}
\index{command line}
\label{s:command.line} 

The syntax of the command line for running \tao is:
\begin{example}
  EXE-DIRECTORY/tao \{OPTIONS\}
\end{example}
where \vn{EXE-DIRECTORY} is the directory where the tao executable lives. If this directory is
listed in your \vn{PATH} environmental variable then the directory specification may be omitted.
The optional arguments are:
%
\begin{description}
%
\item[\vn{-beam_file <file_name>}] \Newline
Sets the name of the file containing the \vn{tao_beam_init} namelist (\sref{s:beam.init}).
Overrides the setting of \vn{beam_file} (\sref{s:init.global}) specified in the \tao initialization
file.
%
\item[\vn{-beam_track_data_file <file_name>}] \Newline
Overrides the setting of \vn{beam_track_data_file} (\sref{s:beam.init}) specified in the \vn{tao_beam_init} namelist.
%
\item[\vn{-beam_init_position_file <file_name>}] \Newline
Specifies the file containing initial particle positions.  Overrides the setting of
\vn{beam_init%position_file} (\sref{s:beam.init}) specified in the \vn{tao_beam_init} namelist.
%
\item[\vn{-building_wall_file <file_name>}] \Newline
Overrides the \vn{building_wall_file} (\sref{s:init.global}) 
specified in the \tao initialization file.
%
\item[\vn{-data_file <file_name>}] \Newline
Overrides the \vn{data_file} (\sref{s:init.global}) specified in the
\tao initialization file.
%
\item[\vn{-disable_smooth_line_calc}] \Newline
Disable computation of the ``smooth curves'' used in plotting. 
This can be used to speed up \tao as discussed in \sref{s:plot.data}.
%
\item[\vn{-external_plotting}] \Newline
This tells \tao that plotting is done externally to \tao. This is done, for example, when using a
Graphics User Interface (GUI) (\sref{s:gui.plot}).
%
\item[\vn{-geometry <width>x<height>}] \Newline
Overrides the plot window geometry. \vn{<width>} and \vn{<height>}
are in Points. This is equivalent to setting \vn{plot_page%size}
in the \vn{tao_plot_page} namelist \sref{s:init.plot}.
%
\item[\vn{-hook_init_file}] \Newline
Specifies an input file for customized versions of Tao. Default file
name is \vn{tao_hook.init}.
%
\item[\vn{-init_file <file_name>}] \Newline
replaces the default \tao initialization file name
(\vn{tao.init}). Note: A \tao initialization file is actually not
needed. If no \tao initialization file is used, the use of the
\vn{-lattice_file} switch is mandatory and \tao will use a set of default plot
templates for plotting.
%
\item[\vn{-lattice_file <file_name>}] \Newline
Overrides the \vn{design_lattice}
lattice file specified in the \tao initialization file
(\sref{s:init.lat}). Example:
\begin{example}
  tao -init my.init -lat slac.bmad
\end{example}
If there is more than one universe and the universes have different
lattices, separate the different lattice names using a "|" character.
Do not put any spaces in between. Example:
\begin{example}
  tao -lat slac.bmad|cesr.bmad
\end{example}
%
\item[\vn{-log_startup}]
If there is a problem with \tao is started, \vn{-log_startup} can be used
to create a log file of the initialization process.
%
\item[\vn{-no_stopping}] \Newline
For debugging purposes. Prevents \tao from stopping where there is a fatal error.
%
\item[\vn{-noinit}] \Newline
Suppresses use of a \tao initialization file. In this case the use of
the \vn{-lattice_file} switch is mandatory and \tao will use a set of default
plot templates for plotting.
%
\item[\vn{-noplot}] \Newline
Suppresses the opening of the plot window.
%
\item[\vn{-no_rad_int}] \Newline
Suppresses the radiation integrals calculation. Radiation integrals are used to calculate such
things as emittances, etc. Generally the calculation is not a problem but in some special
circumstances the calculation can take appreciable time.
%
\item[\vn{-plot_file <file_name>}] \Newline
Overrides the \vn{plot_file} (\sref{s:init.global}) specified in the
\tao initialization file.
%
\item[\vn{-prompt_color}] \Newline
Sets the prompt string color to Blue. For different colors, use the
\vn{set global prompt_color} command (\sref{s:set}).
%
\item[\vn{-rf_on}]
Leaves \vn{rfcavity} elements on. Normally \tao turns off these elements since Twiss and dispersion
calculations do not make sense with them on.  Note: If you want to see orbit changes with RF
frequency changes then you will need to set \vn{parameter[absolute_time_tracking]} to True. See the
``Relative Versus Absolute Time Tracking'' section in the\bmad manual for more details.
%
\item[\vn{-slice_lattice <element_list>}]
If present, discard from the lattice all lattice elements that are not in the \vn{<element_list>}.
Overrides the setting of \vn{design_lattice(i)%slice_lattice}.
%
\item[\vn{-startup_file <file_name>}]
Overrides the \vn{startup_file} (\sref{s:init.global}) specified in the
\tao initialization file.
%
\item[\vn{-var_file <file_name>}] \Newline
Overrides the \vn{var_file} (\sref{s:init.global}) specified in the
\tao initialization file.

\end{description}

To negate an argument, use a two dash prefix instead of a single dash prefix. For example:
\begin{example}
  tao -noplot --noplot
\end{example}
The \vn{-noplot} argument turns off plotting and the following \vn{--noplot} argument negates the
effect of \vn{-noplot} and turns plotting back on. This is useful with the \vn{reinit tao} command
(\sref{s:reinit}) to negate saved command line argument settings.

%----------------------------------------------------------------
\section{Initializing Tao}
\index{initializing!files}
\label{s:initializing}

Initialization occurs when \tao is started. Initialization information is stored in one or more
files as discussed in Chapter \sref{c:init}. If no initialization files are found. \tao uses a
default initialization.

%----------------------------------------------------------------
\section{Command Line Mode and Single Mode}
\label{s:modes}

After \tao is initialized, \tao interacts with the user though the command line. \tao has two modes
for this. In \vn{command line} mode, which is the default mode, \tao waits until the the \vn{return}
key is depressed to execute a command. Command line mode is described in Chapter~\sref{c:command}. 

In \vn{single} mode, single keystrokes are interpreted as commands. \tao can be set up so that in
\vn{single mode} the pressing of certain keys increase or decrease variables. While the same effect
can be achieved in the standard \vn{line mode}, \vn{single mode} allows for quick adjustments of
variables. See Chapter~\sref{c:single} for more details.

%-----------------------------------------------------------------
\section{Lattice Calculations}
\index{lattice calculaitons}
\label{s:lat.calc.overview} 

By default \tao recalculates lattice parameters and does tracking of particles after each command.
The exception is for commands that do not change any parameter that would affect such calculations
such as the \vn{show} command. See \sref{s:lat.calc} for more details. If the recalculation takes a
significant amount of time, the recalculation may be suppressed using the \vn{set global
lattice_calc_on} command (\sref{s:set.global}) or the \vn{set universe} command
(\sref{s:set.universe}).

%-----------------------------------------------------------------
\section{Command Files and Aliases}
\index{command files}
\label{s:command.files} 

Typing repetitive commands in command line mode can become tedious. \tao has two constructs to
mitigate this: Aliases and Command Files. 

Aliases are just like aliases in Unix. See Section~\sref{s:alias} for more details.

Command files are like Unix shell scripts. A series of commands are
put in a file and then that file can be called using the \vn{call}
command (\sref{s:call}).

\tao will call a command file at startup. The default name of this startup file is \vn{tao.startup}
but this name can be changed (\sref{s:format}).

Do loops (\sref{s:do}) are allowed with the following syntax:
\begin{example}
  do <var> = <begin>, <end> \{, <step>\} 
    ...
    tao command [[<var>]]
    ...
  enddo
\end{example}
The \vn{<var>} can be used as a variable in the loop body but must be
bracketed ``[[<var>]]''.  The step size can be any integer positive or
negative but not zero.  Nested loops are allowed and command files can
be called within do loops.

\begin{example}
  do i = 1, 100
    call set_quad_misalignment [[i]] ! command file to misalign quadrupoles
    zero_quad 1e-5*2^([[i]]-1) ! Some user supplied command to zero quad number [[i]]
  enddo
\end{example}

To reduce unnecessary calculations, the logicals \vn{global%lattice_calc_on}
and \vn{global%plot_on} can be toggled from within the command file. Also 
setting \vn{global%quiet} can turn off verbose output to the terminal. Example
\begin{example}
  set global quiet = all          ! Turn off verbose output to the terminal.
  set global lattice_calc_on = F  ! Turn off lattice calculations
  set global plot_on = F          ! Turn off plot calculations
  ... do some stuff ...
  set global plot_on = T          ! Turn back on 
  set global lattice_calc_on = T  ! Turn back on
  set global quiet = off         
\end{example}
See \sref{s:globals} for more details.

A \vn{end_file} command (\sref{s:end.file}) can be used to signal the
end of the command file.

The \vn{pause} command (\sref{s:pause}) can be used to temporarily
pause the command file.

%----------------------------------------------------------------
\section{Customizing Tao}
\label{s:cust.tao}

Custom code can be linked with \tao to extend \tao's capabilities. For example, \tao can be extended to
be used as an online model in a control system. See Chapter~\sref{c:custom.tao} for more details.
