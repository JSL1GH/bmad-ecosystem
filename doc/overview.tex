\SECTION{Overview}
\index{Bmad|textbf}

\bmad\ (Otherwise known as ``Baby MAD" or ``Better MAD" or just plain
``Be MAD!") is a subroutine library for relativistic
charged--particle dynamics simulations in high energy accelerators and
storage ringss. \bmad has been developed at Cornell University's
Laboratory for Elementary Particle Physics and has been in use since
1996. 

Prior to the development of \bmad, simulation programs at Cornell were
written almost from scratch to perform calculations that were beyond
the capability of existing, generally available software. This
practice was inefficient, leading to much duplication of effort.
Since it was time consuming to write simulations, needed calculations
where not being done.  As a response, the \bmad subroutines, using an
object oriented approach and written in Fortran90, were developed to:
\begin{Itemize}
\item Cut down on the time needed to develop programs.
\item Minimize computation times.
\item Cut down on programming errors, 
\item Provide a simple mechanism for lattice function calculations
from within control system programs.
\item Provide a flexible and powerful lattice input format.
\item Standardize sharing of lattice information between 
programs.
\end{Itemize}

\bmad has a wide range of routines to do many things.  \bmad can be
used to study both single and multi--particle beam dynamics.  It has
routines to track both particles and macroparticles. \bmad has various
tracking algorithms including Runge--Kutta and symplectic (Lie
algebraic) integration.  Wake fields, and radiation excitation and
damping can be simulated. \bmad has routines for calculating transfer
matrices, emittances, Twiss parameters, dispersion, coupling, etc. The
elements that \bmad knows about include quadrupoles, RF cavities (both
storage ring and LINAC accelerating types), solenoids, dipole bends,
etc. In addition, elements can be defined to control the attributes of
other elements. This can be used to simulate the ``girder'' which
physically support components in the accelerator or to easily simulate
the action of control room ``knobs'' that gang together, say, the
current going through a set of quadrupoles.

To be able to extend \bmad easily, \bmad has been developed in a
modular, object oriented, fashion to maximize flexibility. As just one
example, each individual element can be assigned a particular tracking
method in order to maximize speed or accuracy and the tracking methods
can be assigned via the lattice file or at run time in a program.


\vfill
\break
