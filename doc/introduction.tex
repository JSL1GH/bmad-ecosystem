\section*{Introduction}

As a consequence of \bmad being a software library, this manual serves
two masters: The programmer who wants to develop applications and
needs to know about the inner workings of \bmad, and the user who
simply needs to know about the \bmad standard input format and about
the physics behind the various calculations that \bmad performs.

\index{MAD|hyperbf}
To this end, this manual is divided into three parts. The first two
parts are for both the user and programmer while the third part is
meant just for programmers. 
  \begin{description}
  \item[Part~I] \Newline
Part~I discusses the \bmad lattice input standard.  The \bmad lattice
input standard was developed using the \mad lattice input standard as
a starting point\cite{b:maduser,b:madphysics}. \mad (Methodical
Accelerator Design) is a widely used stand--alone program developed at
CERN by Christoph Iselin for charged--particle optics
calculations. Since it can be convenient to do simulations with both
\mad and \bmad, differences and similarities between the two input
formats are noted.
  \item[Part~II] \Newline
part~II gives the conventions used by
\bmad --- coordinate systems, magnetic field expansions, etc. ---
along with some of the physics behind the calculations. By necessity,
the physics documentation is brief and the reader is assumed to be familiar
with high energy accelerator physics formalism. 
  \item[Part~III] \Newline
Part~III gives the nitty--gritty details of the \bmad
subroutines and the structures upon which they are based.
\end{description}

\index{Bmad!information}
More information, including the most up--to--date version of this
manual, can be found at the \bmad web site\cite{b:bmad.web}.
Errors and omissions are a fact of life for any reference work and
comments from you, dear reader, are therefore most welcome. Please
send any missives (or chocolates, or any other kind of sustenance) to:
\begin{example}
  David Sagan <dcs16@cornell.edu>
\end{example}
\index{Bmad!error reporting}

It is my pleasure to express appreciation to people who have
contributed to this effort: To David Rubin for his support, to Etienne
Forest for use of his remarkable PTC/FPP library not to mention his
patience in explaining everything to me, to Mark Palmer for all his
work porting \bmad to different platforms, to Hans Grote for granting
the adaptation of figures in the \mad manual for use in this one, and
to Moritz Beckmann, Joel Brock, Sarah Buchan, Joseph Choi, Gerry
Dugan, Michael Ehrlichman, Ken Finkelstein, Mike Forster, Richard
Helms, Georg Hoffstaetter, Chris Mayes, Karthik Narayan, Tia Plautz,
Michael Saelim, Jeff Smith, Jeremy Urban, Mark Woodley, and Jing Yee
Chee for their help.

