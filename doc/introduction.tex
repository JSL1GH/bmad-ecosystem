\section*{Introduction}

\index{MAD|hyperbf}
The strength of \bmad is that, as a subroutine library, it provides a
flexible framework from which sophisticated simulation programs may
easily be developed.  The weakness of \bmad comes from its strength:
\bmad cannot be used straight out of the box. Someone must put the
pieces together into a program. This means that \bmad is
complementary to a general purpose program like
\mad\cite{b:maduser,b:madphysics}: If \mad can solve the problem at
hand you don't need \bmad. If there are no programs that do what you
want, or they are too slow, then \bmad is very useful.

As a consequence of \bmad being a software library, this manual serves
two masters: The programmer who wants to develop applications and
needs to know about the inner workings of \bmad, and the user who
simply needs to know about the \bmad standard input format and about
the physics behind the various calculations that \bmad performs.

To this end, this manual is divided into three parts. The first two
parts are for both the user and programmer while the third part is
meant just for programmers. Part~I gives the conventions used by
\bmad --- coordinate systems, magnetic field expansions, etc. ---
along with some of the physics behind the calculations. By necessity,
the physics documentation is brief and the reader is assumed to be familiar
with high energy accelerator physics formalism. Part~II discusses the
\bmad lattice input standard.  The \bmad lattice input standard was
developed using the \mad lattice input standard as a starting
point. \mad (Methodical Accelerator Design) is a widely used
stand--alone program developed at CERN by Christoph Iselin for
charged--particle optics calculations. Since it can be convenient
to do simulations with both \mad and \bmad, differences and
similarities between the two input formats are noted. 
Finally, Part~III gives the nitty--gritty details of the \bmad
subroutines and the structures upon which they are based.

More information, including the most up--to--date version of this
manual, can be found at the \bmad web site at
\begin{example}
  http://www.lepp.cornell.edu/~dcs/bmad
\end{example}
\index{Bmad!information}

Errors and omissions are a fact of life for any reference work and
comments from you, dear reader, are therefore most welcome. Please
send any missives (or chocolates, or any other kind of sustenance) to:
\begin{example}
  David Sagan <dcs16@cornell.edu>
\end{example}
\index{Bmad!error reporting}

It is my pleasure to express appreciation to people who have
contributed to this effort: To David Rubin for his support, to Etienne
Forest for use of his remarkable PTC/FPP library not to mention his
patience in explaining everything to me, to Mark Palmer for all his
work porting \bmad to different platforms, to Hans Grote for granting
the adaptation of figures in the \mad manual for use in this one, to
Sarah Buchan, Joseph Choi, Gerry Dugan, Michael Ehrlichman, Mike
Forster, Richard Helms, Chris Mayes, Jeff Smith, Jeremy Urban, and
Mark Woodley for their help.

