\section*{Introduction}

\tao stands for ``Tool for Accelerator Optics''. \tao is a general
purpose program for simulating high energy particle beams in
accelerators and storage rings. The simulation engine that \tao uses
is the \bmad software library\cite{b:bmad}. \bmad was developed as an
object-oriented library so that common tasks, such as reading in a
lattice file and particle tracking, did not have to be coded from
scratch every time someone wanted to develop a program to calculate
this, that or whatever.

After the development of \bmad, it became apparent that many simulation
programs had common needs: For example, plotting data, viewing machine
parameters, etc. Because of this commonality, the \tao program was
developed to reduce the time needed to develop a working programs
without sacrificing flexibility. That is, while the ``vanilla''
version of the \tao program is quite a powerful simulation tool, \tao
has been designed to be easily customizable so that extending \tao to
solve new and different problems is relatively straight forward.

This manual is divided into three parts. Part I gives an introduction
and tutorial to \tao with examples of how to set-up tracking, do
plotting, etc. Part II is the reference section which defines the
terms used by \tao and explains in detail the syntax of the
configuration files that \tao uses to make a connection with a
specific machine. Finally, Part III is a programmer's guide which
shows how to extend \tao's capabilities and incorporate custom
calculations.

More information, including the most up--to--date version of this
manual, can be found at the \bmad web site at
\begin{example}
  http://www.lepp.cornell.edu/~dcs/bmad
\end{example}

Errors and omissions are a fact of life for any reference work and
comments from you, dear reader, are therefore most welcome. Please
send any missives (or chocolates, or any other kind of sustenance) to:
\begin{example}
  David Sagan <dcs16@cornell.edu>
\end{example}

It is my pleasure to express appreciation to people who have
contributed to this effort. In particular Jeff Smith who greatly
contributed to this manual and Chris Mayes for his bug reports and
suggestions for improvements to the program. Thanks also must go to
Dave Rubin and Georg Hoffstaetter.
