\section*{Tao: The Tool for Accelerator Optics}

Many simulation problems fall into one of three categories: 

\begin{itemize}
\item 
You want to design a lattice subject to various constraints.
\item 
You have some measured data and you want to make a correction. For
example, you want to know what steering strength changes will make an orbit
flat.
\item
You want to simulate what happens to the orbit, beta function,
etc., when you change something in the machine.
\end{itemize}

Programs that are written to solve these types of problems have common
elements: You have variables you want to vary in your model of your
machine, you have "data" that you want to view, and, in the first two
categories above, you want to match the machine model to the data (in
designing a lattice the constraints correspond to the data).

Because of this commonality of design, the \tao program was developed
to reduce the time needed to develop working programs. \tao is a
machine independent program that implements the essential ingredients
needed to solve simulation problems. To make the
connection, \tao uses configuration input files that can be tailored to
specific machines. Additionally, \tao has been built
to be easily customizable so that extending \tao to solve new and
different problems is relatively straight forward.

More information, including the most up--to--date version of this
manual, can be found at the \bmad web site at
\begin{example}
  http://www.lepp.cornell.edu/~dcs/bmad
\end{example}

Errors and omissions are a fact of life for any reference work and
comments from you, dear reader, are therefore most welcome. Please
send any missives (or chocolates, or any other kind of sustenance) to:
\begin{example}
  David Sagan <dcs16@cornell.edu>
\end{example}