\chapter{Tracking and Transfer Maps}
\label{c:tracking}
\index{tracking}

\index{macroparticles!tracking}
\index{tracking!Macroparticles}
\bmad has routines for tracking two types of objects called
``\vn{particles}'' and ``\vn{macroparticles}''. \vn{Particles} are
characterized by a six-vector representing the particle's phase space
coordinates and a pair of complex numbers characterizing the
particle's spin.  A macroparticle is like a particle with the
addition of a $6\times 6$ ``sigma'' matrix characterizing the size of
the macroparticle.

Macroparticle tracking was implemented in \bmad in order to simulate
particle bunches.  The idea was that far fewer macroparticles than
particles would be needed to characterize a bunch. In practice, it was
found that the complexity of handling the macroparticle sigma matrix
more than offset the reduction in the number of particles
needed. Hence, while the basic macroparticle tracking routines still
exist, macroparticle tracking is not currently maintained and the use
of this code is discouraged. However macroparticle tracking could be
revived in the future if there is a demonstrated need for it.

\index{beam}\index{bunch}
Particle tracking can be divided into ``single particle'' tracking and
``beam'' tracking. Single particle tracking is simply tracking a
single particle. Beam tracking is tracking an ensemble of particles
divided up into a number of bunches that make up a ``beam''. Both
types particle tracking are covered in this chapter.

%----------------------------------------------------------------
\section{The coord_struct}
\label{s:coord.struct}
\index{coord_struct}

\index{spin}
\index{phase space coordinates}
The \vn{coord_struct} holds the coordinates of a particle at a given
longitudinal position in the lattice. The definition of the
\vn{coord_struct} is
\begin{example}
  type coord_struct
    real(rp) vec(6)     ! (x, px, y, py, z, pz)
    real(rp) s          ! Longitudinal position
    real(rp) t          ! Absolute time (not relative to reference).
    complex(rp) spin(2) ! Spin in spinor notation
    real(rp) field(2)   ! Photon (x, y) field intensity.
    real(rp) phase(2)   ! Photon (x, y) phase.
    real(rp) charge     ! charge in a particle (Coul).
    real(rp) path_len   ! path length (used by coherent photons).
    real(rp) p0c        ! For non-photons: Reference momentum. Negative -> going backwards.
                        !     For photons: Photon momentum (not reference).
    real(rp) beta       ! Velocity / c_light. 
    integer ix_ele      ! Index of element particle was tracked through.
                        !   May be -1 or -2 if element is not associated with a lattice.
    integer state       ! alive\$, lost\$, lost_neg_x_aperture\$, etc.
    integer direction   ! Longitudinal direction of motion = +/- 1
    integer species     ! Positron\$, proton\$, etc.
    integer location    ! upstream_end\$, inside\$, or downstream_end\$
end type
\end{example}

The \vn{%vec(:)} array defines the phase space coordinants
(\sref{s:phase.space}). Note that for photons, the definition of the
phase space coordinates (\sref{s:photon.phase.space}) is different
from that used for charged particles.

For charged-particles, the reference momentum in eV is stored in the
\vn{%p0c} component. For photons, \vn{%p0c} is the actual (not
reference) momentum. For charged-particles, \vn{%p0c} may be negative
if the particle is traveling backwards longitudinally. For photons,
\vn{%vec(6)} ($\beta_z$) will be negative if the photon is going
backward.

The normalized velocity $v/c$ is stored in \vn{%beta}.

The \vn{%spin} component gives a particles spin (\sref{s:spin.dyn}) in
spinor notation.

The \vn{%s} component gives the absolute s-position of the particle
and \vn{%t} gives the absolute time.

The \vn{%field_x} and \vn{%field_y} components are for photon
tracking and are in units of field/sqrt(cross-section-area). That is,
the square of these units is an intensity. It is up to individual
programs to define an overall scaling factor for the intensity if
desired.

The \vn{%ix_ele} component gives the index of the element in the
\vn{lat%branch(ib)%ele(:)} array that was tracked through. If the
element tracked through is not associated with a lattice, \vn{%ix_ele}
is set to -1. When initializing a \vn{coord_struct} (see below),
\vn{%ix_ele} will be initialized to \vn{not_set\$}.

The \vn{%state} component will be one of:
\begin{example}
  not_set\$
  alive\$
  lost\$
  lost_neg_x_aperture\$
  lost_pos_x_aperture\$
  lost_neg_y_aperture\$
  lost_pos_y_aperture\$
  lost_z_aperture\$
\end{example}
The \vn{not_set\$} setting indicates that the \vn{coord_struct} has
not yet been used in tracking. The \vn{alive\$} setting indicates that
the particle is alive. If a particle is ``dead'', \vn{%state}
component will be set to on of the other settings. The
\vn{lost_neg_x_aperture\$} setting indicates that the particle was
lost at an aperture on the $-x$ side of the element.  The
\vn{lost_z_aperture\$} setting is used to indicate an energy aperture
where a particle has been decelerated to the point where it is turning
around. The \vn{lost\$} setting is used when neither of the
\vn{lost_*_aperture\$} settings are not appropriate. For example,
\vn{lost\$} is used in Runge-Kutta tracking when the adaptive step
size becomes too small.

To convert the integer value of \vn{%state} to a string that can be 
printed, use the function \Hyperref{r:coord.state.name}{coord_state_name}
\begin{example}
  type (coord_struct) orbit
  print *, 'State of the orbit: ', coord_state_name(orbit%state)
\end{example}

The \vn{%location} component indicates where a particle is
longitudinally with respect to the element being
tracked. \vn{%location} will be on of:
\begin{example}
  entrance_end\$
  inside\$
  exit_end\$
\end{example}
\vn{entrance_end\$} indicates that the particle is at the element's
entrance ($-s$) end and \vn{exit_end\$} indicates that the particle is
at the element's exit ($+s$) end.  \vn{inside\$} indicates that the
particle is in between. If the element has edge fields (for example,
the \vn{e1} and \vn{e2} edge fields of a bend), a particle at the
\vn{entrance_end\$} or \vn{exit_end\$} is considered to be just
outside the element.

To initialize a \vn{coord_struct} so it can be used as the start of
tracking, the \Hyperref{r:init.coord}{init_coord} routine can be used:
\begin{example}
  type (coord_struct) start_orb
  real(rp) phase_space_start(6)
  ...
  phase_space_start = [...]
  call init_coord (start_orb, phase_space_start, lat%ele(i), lat%param%particle)
\end{example}
Here \vn{init_coord} will initialize \vn{start_orb} appropriately for 
tracking through element \vn{lat%ele(i)} with the particle species set to the 
species of the reference particle given by \vn{lat%param%particle}. 

%----------------------------------------------------------------
\section{Tracking Through a Single Element}
\label{s:track1}

\Hyperref{r:track1}{track1} is the routine used for tracking through a
single element
\begin{example}
  type (coord_struct), start_orb, end_orb
  type (ele_struct) ele
  real(rp) start_phase_space(6)
  logical err
  ...
  start_phase_space = [...]
  call init_coord (start_orb, start_phase_space, ele, photon\$) 
  call track1 (start_orb, ele, end_orb, err_flag = err)
  if (.not. particle_is_moving_forward(end_orb)) then
    print *, 'Particle is lost and gone forever...'
  endif
\end{example}
To check if a particle is still traveling in the forward direction,
the \Hyperref{r:particle.is.moving.forward}{particle_is_moving_forward} 
routine can be used as shown in the above example.

The ``virtual'' entrance and exit ends of a lattice element are, by
definition, where the physical ends of the element would be if there
were no offsets. In particular, if an element has a finite
\vn{z_offset} (\sref{s:ele.offset}), the physical ends will be
displaced from the virtual ends. The position \vn{ds} of a particle
with respect to the physical entrance end of the element is
\begin{example}
  ds = coord%s - (ele%s + ele%value(z_offset_tot\$) - ele%value(l\$))
\end{example}
When tracking through an element, the starting and ending positions
always correspond to the virtual ends. If there is a finite
\vn{z_offset}, the tracking of the element will involve tracking
through drifts just before and just after the tracking of the body of
the element so that the particle ends at the proper virtual exit end.

%----------------------------------------------------------------
\section{Tracking Through a Lattice Branch}

When tracking through a lattice branch, one often defines an array of
\vn{coord_struct}s -- one for each element of the lattice branch. In
this case, the $i$\Th \vn{coord_struct} corresponds to the particle
coordinates at the end of the $i$\Th element. Since the number of
elements in the lattice is not known in advance, the array must be
declared to be allocatable. The lower bound of the array must be set
to zero to match a \vn{lat%branch(i)%ele(:)} array.  The upper bound
should be the upper bound of the \vn{%branch(i)%ele(:)} array.  The
routine \Hyperref{r:reallocate.coord}{reallocate_coord} will allocate
an array of \vn{coord_struct}s:
\begin{example}
  type (coord_struct), allocatable :: orbit(:)
  type (lat_struct) lat
  ...
  call reallocate_coord (orbit, lat, ix_branch)
\end{example}
Alternatively, the \vn{save} attribute can be used so that the array
stays around until the next time the routine is called
\begin{example}
  type (coord_struct), allocatable, save :: orb(:) 
\end{example}
Saving the \vn{coord_stuct} is faster but leaves memory tied up. Note
that in the main program, the \vn{save} attribute is not permitted If
a \vn{coord_struct} array is passed to a routine, the routine must
explicitly set the lower bound to zero if the array is not declared as
allocatable:
\begin{example}
  subroutine my_routine (orbit1, orbit2)
    use bmad
    implicit none
    type (coord_struct), allocatable :: orbit1(:)  ! OK
    type (coord_struct) orbit2(0:)                 ! Also OK
    ...
\end{example}
Declaring the array allocatable is mandatory if the array is to be resized
or the array is passed to a routine that declares it allocatable.

\index{coord_array_struct}
For an entire lattice, the \vn{coord_array_struct} can be used to define an array
of \vn{coord_array} arrays:
\begin{example}
  type coord_array_struct
    type (coord_struct), allocatable :: orb(:)
  end type
\end{example}
The routine \Hyperref{r:reallocate.coord.array}{reallocate_coord_array} will allocate an
\vn{coord_array_struct} instance
\begin{example}
  type (coord_array_struct), allocatable :: all_orbit(:)
  type (lat_struct) lat
  ...
  call reallocate_coord_array (all_orbit, lat)
  ...
\end{example}

\index{lat_param_struct!ix_track}
Once an array of \vn{coord_struct} elements is defined, the \Hyperref{r:track.all}{track_all} 
routine can be used to track through a given lattice branch
\begin{example}
  type (coord_struct), allocatable :: orbit(:)
  integer ib, track_state
  ...
  ib = 1                      ! Branch to track through
  call init_coord(orbit(0), init_phase_space, lat%branch(ib)%ele(0), proton$) 
  call track_all (lat, orbit, ib, track_state, err_flag)
  if (track_state /= moving_forward\$) then
    print *, 'Particle lost at element:', track_state
    print *, 'Aperture lost at: ', coord_state_name(orbit(track_state)%state) 
\end{example}
After tracking, \vn{orbit(i)} will correspond to the particles orbit
at the end of \vn{lat%branch(ib)%ele(i)}.  

For routines like \vn{track_all} where an array of \vn{coord_struct}s
is used, an integer \vn{track_state} argument is provided that is set
to \vn{moving_forward\$} if the particle survives to the end, or is
set to the index of the element at which the particle either hit an
aperture or the particle's longitudinal velocity is reversed. 

The reason why the reversal of the particle's longitudinal velocity
stops tracking is due to the fact that the standard tracking routines,
which are $s$-based (that is, use longitudinal position $s$ as the
independent coordinate), are not designed to handle particles that
reverse direction. To properly handle this situation, time-based
tracking needs to be used (\sref{s:time.tracking}). Notice that this is
different from tracking a particle in the reversed ($-s$) direction
which can be handled by constructing a ``reversed'' lattice
(\sref{s:reverse.track}).

Alternatively to \vn{track_all}, the routine
\Hyperref{r:track.many}{track_many} can be used to track through a
selected number of elements or to track backwards (See
\sref{s:reverse.track}).

The \vn{track_all} routine serves as a good example of how tracking
works. A condensed version of the code is shown in
\fig{f:track.all}. The call to \vn{track1} (line~18) tracks
through one element from the exit end of the $n-1$\St\ element to the
exit end of the $n$\Th element.

\begin{figure}[h!]
\begin{centering}
\small
\begin{listing}{1}
  subroutine track_all (lat, orbit, ix_branch, track_state, err_flag)
    use bmad
    implicit none
    type (lat_struct), target :: lat
    type (branch_struct), pointer :: branch
    type (coord_struct), allocatable :: orbit(:)
    integer, optional :: ix_branch, track_state
    logical, optional :: err_flag
    logical err

    ! 

    branch => lat%param(integer_option(0, ix_branch))
    branch%param%ix_track = moving_forward
    if (present(track_state)) track_state = moving_forward\$

    do n = 1, branch%n_ele_track
      call track1 (orbit(n-1), branch%ele(n), branch%param, orbit(n), err_flag = err)
      if (.not. particle_is_moving_forward(orbit(n))) then
        if (present(track_state)) track_state = n
        orbit(n+1:)%status = not_set$
        return
      endif
    enddo
  end subroutine
\end{listing}
\label{f:track.all}
\caption{Condensed track_all code.}
\end{centering}
\end{figure}

%----------------------------------------------------------------
\section{Forking from Branch to Branch}

\index{fork}\index{photon_fork}
Tracking from a \vn{fork} or \vn{photon_fork} (\sref{s:fork}) element
to the target \vn{branch} is not ``automatic''. That is, since the
requirements of how to handle forking can vary greatly from one
situation to the next, \bmad does not try to track from one \vn{branch}
to the next in any one of its tracking routines. 

The discussion here is restriced to the case where the paticle being
tracked is simply transfered from the forking element to the target
branch. [Thus the subject of photon generation is not covered here.]

There are two cases discussed here. The first case is when a given
branch (called \vn{to_branch}) has an associated forking element in
the \vn{from_branch} that forks to the beginning of the
\vn{to_branch}. Appropriate code is:
\begin{example}
  type (lat_struct), target :: lat   ! Lattice 
  type (branch_struct) :: to_branch  ! Given target branch
  type (branch_struct), pointer  :: from_branch ! Base branch
  type (ele_struct), pointer :: fork_ele
  type (coord_struct), allocatable :: from_orbit(:), to_orbit(:)
  integer ib_from, ie_from

  ib_from  = to_branch%ix_from_branch

  if (ib_from < 0) then
    ! Not forked to ...

  else
    from_branch => lat%branch(ib_from)
    ie_from = to_branch%ix_from_ele
    fork_ele => from_branch%ele(ie_from)
    to_orbit(0) = from_orbit(ie_from)
    call transfer_twiss (fork_ele, to_branch%ele(0))
  endif
\end{example}
\vn{from_orbit(0:)} and \vn{to_orbit(0:)} are arrays holding the
orbits at the exit end of the elements for the \vn{from_branch} and
\vn{to_branch} respectively. The call to \Hyperref{r:transfer.twiss}{transfer_twiss}
transfers the Twiss values to the \vn{to_branch} which can then be
propagated through the \vn{to_branch} using \vn{twiss_propagate_all}.

The second case starts with the \vn{fork_ele} forking element.  This
is similar to the first case but is a bit more general since here the
element, called \vn{to_ele} in the \vn{to_branch} that is connected to
\vn{fork_ele} need not be the starting element of \vn{to_branch}.
\begin{example}
  type (lat_struct), target :: lat   ! Lattice 
  type (branch_struct), pointer :: to_branch  ! Target branch
  type (ele_struct), pointer :: to_ele
  type (coord_struct), allocatable :: from_orbit(:), to_orbit(:)
  integer ib_to, ie_to

  ib_to  = nint(fork_ele%value(ix_to_branch\$))
  ie_to  = nint(fork_ele%value(ix_to_element\$))

  to_branch => lat%branch(ib_to)
  to_ele => to_branch%ele(ie_to)
  to_orbit(to_ele%ix_ele) = from_orbit(fork_ele%ix_ele)
\end{example}  
Notice that, by convention, the transferred orbit is located at the
exit end of the \vn{to_ele}.

%----------------------------------------------------------------
\section{Multi-turn Tracking}

Multi-turn tracking over a branch is simply a matter of
setting the coordinates at the beginning zeroth element equal to the
last tracked element within a loop:
\begin{example}
  type (lat_struct) lat             ! lattice to track through
  type (coord_struct), allocatable :: orbit(:)
  ...
  call reallocate_coord (orbit, lat, ix_branch = 1)
  orbit(0)%vec = [0.01, 0.2, 0.3, 0.4, 0.0, 0.0] ! init
  do i = 1, n_turns
    call track_all (lat, orbit, 1)
    orbit(0) = orbit(lat%branch(1)%n_ele_track)
  end do
\end{example}
Often times it is only the root branch, \vn{branch(0)}, that is to be tracked.
In this case, the above reduces to
\begin{example}
  type (lat_struct) lat             ! lattice to track through
  type (coord_struct), allocatable :: orbit(:)
  ...
  call reallocate_coord (orbit, lat%n_ele_max)
  orbit(0)%vec = [0.01, 0.2, 0.3, 0.4, 0.0, 0.0] ! init
  do i = 1, n_turns
    call track_all (lat, orbit)
    orbit(0) = orbit(lat%n_ele_track)
  end do
\end{example}

%----------------------------------------------------------------
\section{Closed Orbit Calculation}

\index{closed orbit}
For a circular lattice the closed orbit may be calculated using
\vn{closed_orbit_calc}. By default this routine will track in the
forward direction which is acceptable unless the particle you are
trying to simulate is traveling in the reverse direction and there is
radiation damping on. In this case you must tell
\vn{closed_orbit_calc} to do backward tracking. This routine works by
iteratively converging on the closed orbit using the 1--turn matrix to
calculate the next guess. On rare occasions if the nonlinearities are
strong enough, this can fail to converge. An alternative routine is
\vn{closed_orbit_from_tracking} which tries to do things in a more
robust way but with a large speed penalty.

%----------------------------------------------------------------
\section{Partial Tracking through elements}
\label{s:tracking.partial}
\index{tracking!partial}

There are two routines for tracking partially through an element:
\begin{example}
  \Hyperref{r:twiss.and.track.at.s}{twiss_and_track_at_s} (lat, s, ele, orb, orb_at_s, ix_branch, err, use_saved_data)
  \Hyperref{r:twiss.and.track.intra.ele}{twiss_and_track_intra_ele}  (ele, param, l_start, l_end, track_entrance,
                   track_exit, orbit_start, orbit_end, ele_start, ele_end, err)

\end{example}
Both routines make use of element ``slices'' (\sref{s:ele.lat}) which
are elements that represent some sub-section of an element. The
routine \Hyperref{r:create.element.slice}{create_element_slice} can be used to create such slices.

%----------------------------------------------------------------
\section{Apertures}
\label{s:tracking.apertures}
\index{tracking!apertures}

\index{ele_struct!\%aperture_type}
The routine \Hyperref{r:check.aperture.limit}{check_aperture_limit}
checks the aperture at a given element. The \vn{ele%aperture_type}
component determines the type of aperture. Possible values for
\vn{ele%aperture_type} are
\begin{example}
  rectangular$
  elliptical$
  custom$
\end{example} %$
With \vn{custom\$}, a program needs to be linked with a custom version
of
\Hyperref{r:check.aperture.limit.custom}{check_aperture_limit_custom}.

\index{lat_param_struct!aperture_limit_on}
\index{bmad_common_struct!max_aperture_limit}
The logical \vn{lat%param%aperture_limit_on} determines if element
apertures (See \sref{s:limit}) are used to determine if a
particle has been lost in tracking.  The default
\vn{lat%param%aperture_limit_on} is True.  Even if this is False
there is a ``hard'' aperture limit set by
\vn{bmad_com%max_aperture_limit}. This hard limit is used to prevent
floating point overflows. The default hard aperture limit is 1000
meters. Additionally, even if a particle is within the hard limit,
some routines will mark a particle as lost if the tracking calculation
will result in an overflow.

\index{lat_param_struct!end_lost_at}
\index{lat_param_struct!lost}
\index{lat_param_struct!ix_lost}
\index{entrance_end}
\index{exit_end}
\vn{lat%param%lost} is the logical to check to see if a particle has
been lost. \vn{lat%param%ix_lost} is set by \vn{track_all} and gives
the index of the element at which a particle is lost.
\vn{%param%end_lost_at} gives which end the particle was lost at. 
The possible values for \vn{lat%param%end_lost_at} are:
\begin{example}
  entrance_end\$
  exit_end\$
\end{example}
When tracking forward, if a particle is lost at the exit end of an
element then the place where the orbit was outside the aperture is at
\vn{orbit(ix)} where \vn{ix} is the index of the element where the
particle is lost (given by \vn{lat%param%ix_lost}). If the
particle is lost at the entrance end then the appropriate index is one
less (remember that \vn{orbit(i)} is the orbit at the exit end of an
element). 

To tell how a particle is lost, check the \vn{lat%param%plane_lost_at}
parameter. Possible values for this are:
\begin{example}
  x_plane\$
  y_plane\$
  z_plane\$
\end{example}
\vn{x_plane\$} and \vn{y_plane\$} indicate that the particle was lost
either horizontally, or vertically. \vn{z_plane\$} indicates that the
particle was turned around in an \vn{lcavity} element. That is, the 
cavity was decelerating the particle and the particle did not not have
enough energy going into the cavity to make it to the exit.

%----------------------------------------------------------------
\section {Tracking Methods}

\index{ele_struct!\%tracking_method}
For each element the method of tracking may be set either via the
input lattice file (see \sref{s:tkm}) or directly in the
program by setting the \vn{%tracking_method} attribute of an element
\begin{example}
  type (ele_struct) ele
  ...
  ele%tracking_method = boris$  ! for boris tracking
  print *, 'Tracking_method: ', calc_method_name(ele%tracking_method)
\end{example}
To form the corresponding parameter to a given tracking method just
put ``\$'' after the name. For example, the \vn{bmad_standard}
tracking method is specified by the \vn{bmad_standard\$} parameter. To
convert the integer \vn{%tracking_method} value to a string suitable
for printing, use the \vn{tracking_method_name} array.

\index{ele_struct!\%mat6}\index{linear}
It should be noted that except for \vn{linear} tracking, none of the
\bmad tracking routines make use of the \vn{ele%mat6} transfer
matrix. The reverse, however, is not true.  The transfer matrix
routines (\vn{lat_make_mat6}, etc.)  will do tracking.

For determining what tracking methods are valid for a given element,
use \Hyperref{r:valid.tracking.method}{valid_tracking_method} and \Hyperref{r:valid.mat6.calc.method}{valid_mat6_calc_method}
functions
\begin{example}
  print *, 'Method is valid: ', valid_tracking_method(ele, boris\$)
\end{example}

\index{synchrotron radiation!calculating}
\bmad simulates radiation damping and excitation by applying a kick
just before and after each element. 

%----------------------------------------------------------------
\section{Time Tracking}
\label{s:time.tracking}

Time tracking uses time as the independent variable as opposed to the
standard $s$ based tracking. Time tracking is useful when a particle's
trajectory can reverse itself longitudinally. For example, low energy
particles generated when a relativistic particle hits the vacuum
chamber wall are good candidates for time tracking. 

Currently, the only \vn{ele%tracking_method} available for time
tracking is \vn{time_runge_kutta\$}. Time tracking needs extra
bookkeeping due to the fact that the particle may reverse directions.
See the \vn{dark_current_tracker} program as an example. 

%----------------------------------------------------------------
\section{Taylor Maps}
\label{s:taylor.track}
\index{taylor Map}

A list of routines for manipulating Taylor maps is given
in~\sref{r:taylor}. The order of the Taylor maps is set in the lattice
file using the \vn{parameter} statement (\sref{s:param}). In a program
this can be overridden using the routine
\Hyperref{r:set.ptc}{set_ptc}. The routine
\Hyperref{r:taylor.coef}{taylor_coef} can be used to get the
coefficient of any given term in a Taylor map.
\begin{example}
  type (taylor_struct) t_map(6)
  ...
  print *, 'out(4)=coef * in(1)^2:', taylor_coef(t_map(4), 1, 1)
  print *, 'out(4)=coef * in(1)^2:', taylor_coef(t_map(4), [2,0,0,0,0,0])
\end{example}

\index{symp_lie_Bmad}
\index{symp_lie_PTC}
\index{symp_map}
\index{taylor}
\index{taylor!deallocating}
Transfer Taylor maps for an element are generated as needed when the
\vn{ele%tracking_method} or \vn{ele%mat6_calc_method} is set to
\vn{Symp_Lie_Bmad}, \vn{Symp_Lie_PTC}, \vn{Symp_Map}, or
\vn{Taylor}. Since generating a map can take an appreciable time,
\bmad follows the rule that once generated, these maps are never
regenerated unless an element attribute is changed.  To generate a
Taylor map within an element irregardless of the
\vn{ele%tracking_method} or \vn{ele%mat6_calc_method} settings use the
routine \Hyperref{r:ele.to.taylor}{ele_to_taylor}. This routine will kill any old Taylor map
before making any new one. To kill a Taylor map (which frees up the
memory it takes up) use the routine \Hyperref{r:kill.taylor}{kill_taylor}.

To test whether a \vn{taylor_struct} variable has an associated Taylor
map. That is, to test whether memory has been allocated for the map,
use the Fortran associated function:
\begin{example}
  type (bmad_taylor) taylor(6)
  ...
  if (associated(taylor(1)%term)) then  ! If has a map ...
    ...
\end{example}

To concatenate the Taylor maps in a set of elements the routine
\Hyperref{r:concat.taylor}{concat_taylor} can be used
\begin{example}
  type (lat_struct) lat          ! lattice
  type (taylor_struct) taylor(6)  ! taylor map
  ...
  call taylor_make_unit (taylor)  ! Make a unit map
  do i = i1+1, i2
    call concat_taylor (taylor, lat%ele(i)%taylor, taylor)
  enddo
\end{example}
The above example forms the transfer Taylor map starting at the end of
element \vn{i1} to the end of element \vn{i2}. Note: This example
assumes that all the elements have a Taylor map. The problem with
concatenating maps is that if there is a constant term in the map
``feed down'' can make the result inaccurate (\sref{s:taylor.phys}. To
get around this one can ``track'' a taylor map through an element
using symplectic integration.
\begin{example}
  type (lat_struct) lat          ! lattice
  type (taylor_struct) taylor(6)  ! taylor map
  ...
  call taylor_make_unit (taylor)  ! Make a unit map
  do i = i1+1, i2
    call call taylor_propagate1 (taylor, lat%ele(i), lat%param)
  enddo
\end{example}
\index{ds_step}
\index{integrator_order}
Symplectic integration is typically much slower than concatenation.
The width of an integration step is given by \vn{%ele%value(ds_step\$}.
The attribute \vn{%ele%value(num_steps\$)}, which gives the number
of integration steps, is a dependent variable 
(\sref{s:depend}) and should not be set directly.
The order of the integrator (\sref{s:taylor.phys})
is given by \vn{%ele%integrator_order}. 
PTC (\sref{c:ptc}) currently implements integrators of order 2, 4, or 6.

%----------------------------------------------------------------
\section{Reverse Tracking}
\label{s:reverse.track}
\index{tracking!reverse}

Reverse tracking in when a particle goes in the direction of
decreasing \vn{s}. The \Hyperref{r:track.many}{track_many} routine can
be used for this. See the \vn{track_many} routine for more details.

Another way to track in the reverse direction is to create a
``reversed'' lattice with the elements in reverse order and with
each lattice element's orientaiton reversed. The routine for this
is \Hyperref{r:lat.reverse}{lat_reverse}.


%----------------------------------------------------------------
\section{Beam (Particle Distribution) Tracking}
\label{s:part.track}
\index{tracking!particle distributions}

Tracking with multiple particles is done with a \vn{beam_struct} instance:
\begin{example}
  type beam_struct
    type (bunch_struct), allocatable :: bunch(:)
  end type
\end{example}
A \vn{beam_struct} is composed of an array of bunches of type
\vn{bunch_struct}:
\begin{example}
  type bunch_struct
    type (coord_struct), allocatable :: particle(:)
    integer, allocatable :: ix_z(:)  ! bunch%ix_z(1) is index of head particle, etc.
    real(rp) charge_tot  ! Total charge in bunch (Coul).
    real(rp) charge_live ! Total charge of live particles in bunch (Coul).
    real(rp) z_center    ! Longitudinal center of bunch (m). Note: Generally, z_center of 
                         !   bunch #1 is 0 and z_center of the other bunches is negative.
    real(rp) t_center    ! Center of bunch creation time relative to head bunch.
    integer species      ! electron\$, proton\$, etc.
    integer ix_ele       ! Element this bunch is at.
    integer ix_bunch     ! Bunch index. Head bunch = 1, etc.
  end type
\end{example}
The \vn{bunch_struct} has an array of particles of type
\vn{coord_struct} (\sref{s:coord.struct}).

Initializing a \vn{beam_struct} to conform to some initial set of
Twiss parameters and emittances is done using the routine
\Hyperref{r:init.beam.distribution}{init_beam_distribution}: 
\begin{example}
  type (lat_struct) lat
  type (beam_init_struct) beam_init
  type (beam_struct) beam
  ...
  call init_beam_distribution (lat%ele(0), lat%param, beam_init, beam)
\end{example}
The \vn{lat%ele(0)} argument, which is of type \vn{ele_struct}, gives
the twiss parameters to initialize the beam to. In this case, we are
starting tracking from the beginning of the lattice. The
\vn{beam_init} argument which is of type \vn{beam_init} gives
additional information, like emittances, which is needed to initialize
the beam. See Section~\sref{s:beam.init} for more details.

Tracking a beam is done using the \Hyperref{r:track.beam}{track_beam} routine
\begin{example}
  type (lat_struct) lat
  type (beam_struct) beam
  ...
  call track_beam (lat, beam)
\end{example}
or, for tracking element by element, \Hyperref{r:track1.beam}{track1_beam} can be used.

For analyzing a bunch of particles, that is, for computing such things
as the sigma matrix from the particle distribution, the
\Hyperref{r:calc.bunch.params}{calc_bunch_params} routine can be used. 

%----------------------------------------------------------------
\section{Spin Tracking}
\label{s:spin.track}
\index{tracking!spin}

See Section~\sref{s:spin.methods} for a list of spin tracking methods
available. To turn spin tracking on, use the
\vn{bmad_com%spin_tracking_on} flag. \vn{ele%spin_tracking_method}
sets the method used for spin tracking. After properly initializing
the spin in the \vn{coord_struct}, calls to \vn{track1} will track
both the particle orbit and the spin.


%-------------------------------------------------------------------------
\section{X-ray Targeting}
\label{s:targeting.code}

X-rays can have a wide spread of trajectories resulting in many
``doomed'' photons that hit apertures or miss the detector with only a
small fraction of ``successful'' photons actually contributing to the
simulation results. The tracking of doomed photons can therefore
result in an appreciable lengthening of the simulation time. To get
around this, \bmad can be setup to use what is called ``targeting'' to
minimize the number of doomed photons generated. 

This is explained in detail in \sref{s:targeting}. The coordinates of
the four or eight corner points and the center target point are 
stored in:
\begin{example}
  gen_ele%photon%target%corner(:)%r(1:3)
  gen_ele%photon%target%center%r(1:3)
\end{example}
where \vn{gen_ele} is the 
generating element (not the element with the aperture).
