\chapter{Lattice File Global Parameters}

This chapter deals with statements that can be used to set ``global''
parameter values. That is, parameter values that are associated with
the lattice as a whole and not simply associated with a single element.

%-----------------------------------------------------------------------------
\section{Parameter Statement}
\label{s:param}
\index{parameter statement|hyperbf}


\index{lattice}\index{geometry}\index{photon_type}
\index{taylor_order}\index{e_tot}
\index{p0c}\index{ran_seed}\index{absolute_time_tracking}
\index{n_part}\index{no_end_marker}
\index{ptc_exact_model}\index{ptc_exact_misalign}
The \vn{parameter} statement is used to set the \vn{lattice} name and
other variables. If multiple branches are present (\sref{s:branch.def}), these
variables pertain to the \vn{root} branch. The variables that can be
set by \vn{parameter} are
\begin{example}
  parameter[absolute_time_tracking]   = <Logical>  ! Absolute time used for RF clock?
  parameter[custom_attributeN]        = <string>   ! Defining custom attributes (\sref{s:custom.attrib}).
  parameter[default_tracking_species] = <Switch>   ! Default type of tracked particle. 
                                                   !    Default is ref_particle.
  parameter[e_tot]                    = <Real>     ! Reference total Energy. 
                                                   !      Default: 1000 * rest_energy.
  parameter[electric_dipole_moment]   = <Real>     ! Electric dipole moment of tracked particles.
  parameter[geometry]                 = <Switch>   ! Open or closed
  parameter[lattice]                  = <String>   ! Lattice name 
  parameter[n_part]                   = <Real>     ! Number of particles in a bunch.
  parameter[no_end_marker]            = <Logical>  ! Default: False.
  parameter[p0c]                      = <Real>     ! Reference momentum.
  parameter[particle]                 = <Switch>   ! Reference species: positron, proton, etc.
  parameter[photon_type]              = <Switch>   ! Incoherent or coherent photons?
  parameter[ptc_exact_model]          = <Logical>  ! PTC to do "exact" tracking?
  parameter[ptc_exact_misalignment]   = <Logical>  ! PTC to "exactly" misalign elements?
  parameter[ptc_max_fringe_order]     = <Integer>  ! Max fringe order. 
                                                   !    Default: 2 => Quadrupole.
  parameter[ran_seed]                 = <Integer>  ! Random number generator init.
  parameter[taylor_order]             = <Integer>  ! Default: 3
  parameter[use_hard_edge_drifts]     = <Logical>
\end{example}

\noindent
Examples
\begin{example}
  parameter[lattice]      = "L9A19C501.FD93S_4S_15KG"
  parameter[geometry]     = closed
  parameter[taylor_order] = 5
  parameter[E_tot]        = 5.6e9    ! eV
\end{example}

  \begin{description}
  \index{absolute_time_tracking}\index{lcavity}\index{rfcavity}
  \item[{parameter[absolute_time_tracking]}] \Newline
The \vn{absolute_time_tracking} switch sets whether the clock for the \vn{lcavity} and
\vn{rfcavity} elements is tied to the reference particle or to uses the absolute time
(\sref{s:rf.time}). A value of \vn{False} (the default) mandates relative time and a value
of \vn{True} mandates absolute time. The exception is that for an \vn{e_gun} element
(\sref{s:e.gun}), absolute time tracking is always used in order to be able to avoid
problems with a zero reference momentum at the beginning of the element.
  \index{custom_attributeN}
  \item[{parameter[custom_attributeN]}] \Newline
For more information on defining custom attributes, see \sref{s:custom.attrib}.
  \index{default_tracking_species}
  \item[{parameter[default_tracking_species]}] \Newline
The \vn{parameter[default_tracking_species]} switch establishes the
default type of particles to be tracked. Possible setting include
all the settings of \vn{parameter[particle]}. In addition, 
this switch can be set to:
\begin{example}
  ref_particle     ! default
  anti_ref_particle
\end{example}
By default, \vn{default_tracking_species} is set to \vn{ref_particle}
so that the particle being tracked is the same as the reference
particle set by \vn{param[particle]}. In the case, for example,
where there are particles going one way and antiparticles going another,
\vn{default_tracking_species} can be used to switch between
tracking the particles or antiparticles.
  \index{e_tot}\index{p0c}
  \index{lcavity}\index{patch}
  \item[{parameter[e_tot], parameter[p0c]}] \Newline
The \vn{parameter[e_tot]} and \vn{parameter[p0c]} are the reference
total energy and momentum at the start of the lattice. Each element
in a lattice has an individual reference \vn{e_tot} and \vn{p0c} attributes
which are dependent parameters. The reference energy and momentum will only
change between \vn{LCavity} or \vn{Patch} elements. The starting
reference energy, if not set, will be set to 1000 time the particle
rest energy.  Note: \vn{beginning[e_tot]} and \vn{beginning[p0c]} (\sref{s:beginning}) are
equivalent to \vn{parameter[e_tot]} and \vn{parameter[p0c]}.
  \index{electric_dipole_moment}
  \item[{parameter[electric_dipole_moment]}] \Newline
The \vn{electric_dipole_moment} sets the electric dipole moment value $\eta$ for use
when tracking with spin (\sref{s:spin.dyn}). 
  \index{geometry}\index{closed}\index{open}
  \index{lcavity!and geometry}
  \item[{parameter[geometry]}] \Newline
Valid \vn{geometry} settings are
\begin{example}
  closed  ! Default w/o LCavity element present.
  open    ! Default if LCavity elements present.
\end{example}
A machine with a \vn{closed} geometry is something like a storage ring
where the particle beam recirculates through the machine.  A machine
with an \vn{open} geometry is something like a linac.  In this case,
the initial Twiss parameters need to be specified in the lattice
file. If the \vn{geometry} is not specified, \vn{closed} is the
default. The exception is that if there is an \vn{Lcavity} element
present or the reference particle is a photon, \vn{open} will be the
default.

Notice that by specifying a \vn{closed} geometry it {\em does} not
mean that the downstream end of the last element of the lattice has
the same global coordinates (\sref{s:global}) as the global
coordinates at the beginning. Setting the geometry to \vn{closed}
simply signals to a program to compute closed orbits and periodic
Twiss parameters as opposed to calculating orbits and Twiss parameters
based upon initial orbit and Twiss parameters at the beginning of the
lattice. And indeed, it is sometimes convenient to treat lattices as
closed even though there is no closure in the global coordinate sense.

Note: \vn{geometry} used to be called \vn{lattice_type}, \vn{closed}
used to be called \vn{circular_lattice} and \vn{open} used to be
called \vn{linear_lattice}.
  \index{lattice}
  \item[{parameter[lattice]}] \Newline
Used to set the lattice name. The \vn{lattice} name is stored by \bmad
for use by a program but it does not otherwise effect any \bmad
routines.
  \index{n_part}\index{beambeam}\index{lcavity}
  \item[{parameter[n_part]}] \Newline
The \vn{parameter[n_part]} is the number of particle in a bunch.
it is used with \vn{BeamBeam} elements and is used to calculate the
change in energy through an \vn{Lcavity}. See~\sref{s:lcav} for more
details.
  \index{no_end_marker}\index{end element}
  \item[{parameter[no_end_marker]}] \Newline
The \vn{parameter[no_end_marker]} is use to suppress the automatic inclusion
of a marker named \vn{END} at the end of the lattice (\sref{s:branch.construct}). 
  \item[{parameter[p0c]}] \Newline
See \vn{parameter[e_tot]}.
  \index{particle}
  \index{positron}\index{electron}\index{proton}\index{antiproton}\index{photon}
  \index{muon}\index{antimuon}\index{pion0}\index{pion-}\index{pion+}
  \item[{parameter[particle]}] \Newline
The \vn{parameter[particle]} switch sets the reference species. The possible settings for
this attribute can be divided into four groups. One group are are fundamental
particles. These are:
\begin{example}
  electron,  positron,
  muon,      antimuon,
  proton,    antiproton,
  photon, 
  pion+,      pion0,      pion-
  deuteron
\end{example}
Names for the fundamental particles are {\em not} case sensitive.

Another group are atoms. The general syntax for atoms is:
\begin{example}
  \{\#xxx\}AA\{@Mmmmm\}\{ccc\}
\end{example}
The curly brackets \{...\} denote optional prefixes and suffixes. \vn{AA} here is the
atomic symbol, \vn{\#xxx} is the number of nucleons, \vn{@Mmmmm} is the mass in AMU, and \vn{ccc}
is the charge. Examples:
\begin{example}
  parameter[particle] = \#12C+3       ! Triply charged carbon-12
  parameter[particle] = Hg@M200.6--  ! Doubly charged mercury with mass = 200.6
\end{example}
The mass may to specified to hundridths of an AMU. If the mass is not given, but the
number of nucleons is, the appropriate weight for that isotope is used. If both the mass
and the number of nucleons is not present, the mass is an average weighted by the isotopic
abondances of the element. The charge may be given by using the appropriate number of plus
(+) or minus (-) signs or by using a plus or minus sign followed by a number. Thus
``\vn{-{-}-}'' is equivalent to ``-3''. Names here are case sensitive including using ``@M''
but not ``@m'' for specifying the mass.

Another group of particles are the ``known'' molecules. The syntax for these are:
\begin{example}
  BBB\{@Mmmmm\}\{ccc\}
\end{example}
\vn{@Mmmmm} and \vn{ccc} are the same as for atoms.
\vn{BBB} is the molecular formulas. The known molecules are:
\begin{example}
CO       CO2      
D2       D2O      
OH       O2      
H2       H2O      HF
N2       NH2      NH3      
CH2      CH3      CH4      
C2H3     C2H4     C2H5
\end{example}
Again, if the mass is not specified, the average mass is used. Examples:
\begin{example}
  C2H3@M28.4+     ! Singly charged C2H3 with mass of 28.4
  CH2             ! Neutral CH2
\end{example}
Like the atomic formulas, molecular formulas are case sensitive.

The last group of particle are particles where only the mass and charge are specified.
The syntax for these are:
\begin{example}
  @Mmmmm\{ccc\}
\end{example}

The setting of the reference particle is
used, for example, to determine the direction of the field in a magnet
and given the normalized field strength (EG: \vn{k1} for a
quadrupole).  Generally, the particles that by default are tracked
through a lattice are the same as the reference particle. This default
behavior can be altered by setting
\vn{parameter[default_tracking_species]}.

  \index{photon_type}
  \item[{parameter[photon_type]}] \Newline
The \vn{photon_type} switch is used to set the type of photons that
are used in tracking. Possible settings are:
\begin{example}
  incoherent    ! Default
  coherent 
\end{example}
The general rule is use incoherent tracking except when there is a
\vn{diffraction_plate} element in the lattice. 

  \index{ptc_exact_model}\index{ptc_exact_misalign}
  \item[{parameter[ptc_exact_model]}] \Newline
The \vn{ptc_exact_model} and \vn{ptc_exact_misalign} switches affect
tracking using the \vn{PTC} library. See \sref{s:integ} for more
details.

  \index{ptc_max_fringe_order}
  \item[{parameter[ptc_max_fringe_order]}] \Newline
When using PTC tracking (\sref{s:ptc.intro}), the
\vn{parameter[ptc_max_fringe_order]} determines the maximum order of
the calculated fringe fields. The default is 2 which means that fringe
fields due to a quadrupolar field. These fields are 3\Rd order in the
transverse coordinates.

  \index{ran_seed}
  \item[{parameter[ran_seed]}] \Newline
For more information on \vn{parameter[ran_seed]} see \sref{s:functions}.

  \index{taylor_order}
  \item[{parameter[taylor_order]}] \Newline
The Taylor order (\sref{s:taylor.phys}) is set by
\vn{parameter[taylor_order]} and is the maximum order for a Taylor map.

  \index{use_hard_edge_drifts}
  \item[{parameter[use_hard_edge_drifts]}] \Newline
The \vn{use_hard_edge_drifts} switch determines if a ``hard edge''
model of certain elements is used. For example, if \vn{runge_kutta}
tracking is used for an \vn{rfcavity} or \vn{lcavity} using a standing
wave model then the cavity length should be a multiple of the RF
wavelength/2. To achieve this, during tracking, the cavity length is
appropriately modified and drifts are inserted at either ends of the
cavity to keep the total length constant. Default is \vn{True}.

  \end{description}

%-----------------------------------------------------------------------------
\section{Beam_start Statement} \label{s:beam.start}
\index{beam_start statement|hyperbf}

\index{e_gun}
\index{x}\index{px}\index{y}\index{py}\index{z}\index{pz}
\index{emittance_a}\index{emittance_b}\index{emittance_z}
\index{field_x}\index{field_y}\index{phase_x}\index{phase_y}
\index{spin_x}\index{spin_y}\index{spin_z}\index{spinor_theta}
\index{spinor_phi}\index{spinor_xi}\index{spinor_polarization}
The \vn{beam_start} statement is used to set the starting coordinates
for particle tracking. If multiple branches are present
(\sref{s:branch.def}), these variables pertain to the \vn{root}
branch.
\begin{example}
  beam_start[x]                   = <Real>   ! Horizontal position.
  beam_start[px]                  = <Real>   ! Horizontal momentum.
  beam_start[y]                   = <Real>   ! Vertical position.
  beam_start[py]                  = <Real>   ! Vertical momentum.
  beam_start[z]                   = <Real>   ! Longitudinal position.
  beam_start[pz]                  = <Real>   ! Momentum deviation. Only used for non-photons
  beam_start[E_photon]            = <Real>   ! Energy (eV). Only used for photons.
  beam_start[emittance_a]         = <Real>   ! A-mode emittance
  beam_start[emittance_b]         = <Real>   ! B-mode emittance
  beam_start[emittance_z]         = <Real>   ! Z-mode emittance
  beam_start[sig_z]               = <Real>   ! Beam sigma in z-direction
  beam_start[sig_e]               = <Real>   ! Relative beam energy sigma (Actually: Sigma_E/E)
  beam_start[field_x]             = <Real>   ! Photon beam field along x-axis
  beam_start[field_y]             = <Real>   ! Photon beam field along y-axis
  beam_start[phase_x]             = <Real>   ! Photon beam phase along x-axis
  beam_start[phase_y]             = <Real>   ! Photon beam phase along y-axis
  beam_start[t]                   = <Real>   ! Absolute time
  beam_start[spin_x]              = <Real>   ! Spin polarization x-coordinate
  beam_start[spin_y]              = <Real>   ! Spin polarization y-coordinate
  beam_start[spin_z]              = <Real>   ! Spin polarization z-coordinate
  beam_start[spinor_polarization] = <Real>   ! Spin polarization
  beam_start[spinor_theta]        = <Real>   ! Spin angle
  beam_start[spinor_phi]          = <Real>   ! Spin angle
  beam_start[spinor_xi]           = <Real>   ! Spin angle
\end{example}
Normally the absolute time, set by \vn{beam_start[t]}, is a dependent
parameter set by solving \Eq{zbctt} for $t$. The exception is when the
initial velocity is zero. (This can happen if there is an \vn{e_gun}
(\sref{s:e.gun}) element in the lattice). In this case, $z$ must be
zero and $t$ is an independent parameter that can be set.

For particles with spin, the spin can be specified in one of two ways:
One way is to specify the $(x, y, z)$ spin components using
\vn{spin_x}, \vn{spin_y}, and \vn{spin_z}. The second way is by
specifying the polar angles and magnitude with \vn{spinor_theta},
\vn{spinor_phi}, \vn{spinor_xi}, and \vn{spinor_polarization}. The
relationship between the two spin representations is given by
\Eq{pp1p2}. To prevent ambiguity, only one representation can can be
used in lattice. The default is
\begin{example}
  spinor_polarization = 1
  spinor_theta        = 0
  spinor_phi          = 0
  spinor_xi           = 0

  spin_x              = 0
  spin_y              = 0
  spin_z              = 1
\end{example}

For photons, \vn{px}, \vn{py}, and \vn{pz} are the normalized velocity components
(Cf.~\Eq{xbybzb}). For photons \vn{pz} is a dependent parameter which will be set
so that \Eq{bbb1} is obayed.


Example
\begin{example}
  beam_start[y] = 2 * beam_start[x]
\end{example}

%-----------------------------------------------------------------------------
\section{Beam Statement}
\index{beam statement|hyperbf}

\index{energy}
\index{particle}
\index{n_part}
\index{MAD!beam statement}
The \vn{beam} statement is provided for compatibility with \mad. The syntax is
\begin{example}
  beam, energy = GeV, pc = GeV, particle = <Switch>, n_part = <Real>
\end{example}
For example
\index{MAD}
\begin{example}
  beam, energy = 5.6  ! Note: GeV to be compatible with \mad
  beam, particle = electron, n_part = 1.6e10
\end{example}
Setting the reference energy using the \vn{energy} attribute is the
same as using \vn{parameter[e_tot]}. Similarly, setting \vn{pc} is
equivalent to setting \vn{parameter[p0c]}. Valid \vn{particle} switches
are the same as \vn{parameter[particle]}.

%--------------------------------------------------------------------------
\section{Beginning and Line Parameter Statements}
\label{s:beginning}
\index{beginning statement|hyperbf}

\index{beta_a}\index{alpha_a}
\index{phi_a}\index{eta_x}
\index{etap_x}\index{beta_b}
\index{alpha_b}\index{phi_b}
\index{eta_y}\index{etap_y}
\index{cmat_ij}\index{e_tot}
\index{p0c}\index{ref_time}
For non--circular lattices, the \vn{beginning} statement can be used to
set the Twiss parameters and beam energy at the beginning of the first lattice branch.
\begin{example}
  beginning[alpha_a]  = <Real>  ! "a" mode alpha
  beginning[alpha_b]  = <Real>  ! "b" mode alpha
  beginning[beta_a]   = <Real>  ! "a" mode beta
  beginning[beta_b]   = <Real>  ! "b" mode beta
  beginning[cmat_ij]  = <Real>  ! C coupling matrix. i, j = \{``1'', or ``2''\} 
  beginning[e_tot]    = <Real>  ! Reference total energy in eV.
  beginning[eta_x]    = <Real>  ! x-axis dispersion
  beginning[eta_y]    = <Real>  ! y-axis dispersion
  beginning[etap_x]   = <Real>  ! x-axis dispersion derivative.
  beginning[etap_y]   = <Real>  ! y-axis dispersion derivative.
  beginning[p0c]      = <Real>  ! Reference momentum in eV.
  beginning[phi_a]    = <Real>  ! "a" mode phase.
  beginning[phi_b]    = <Real>  ! "b" mode phase.
  beginning[ref_time] = <Real>  ! Starting reference time.
  beginning[s]        = <Real>  ! Longitudinal starting position.
\end{example}
\index{e_tot}
The \vn{gamma_a}, \vn{gamma_b}, and \vn{gamma_c} (the coupling gamma
factor) will be kept consistent with the values set. If not set the
default values are all zero.  \vn{beginning[e_tot]} and
\vn{parameter[e_tot]} are equivalent and one or the other may be
set but not both. Similarly, \vn{beginning[p0c]} and
\vn{parameter[p0c]} are equivalent.

\index{x_position}\index{y_position}\index{z_position}
\index{theta_position}\index{phi_position}\index{psi_position}
For any lattice the \vn{beginning} statement can be used to set the
starting floor position of the first lattice branch
(see~\sref{s:global}). The syntax is
\begin{example}
  beginning[x_position]     = <Real>  ! X position
  beginning[y_position]     = <Real>  ! Y position
  beginning[z_position]     = <Real>  ! Z position
  beginning[theta_position] = <Real>  ! Angle on floor
  beginning[phi_position]   = <Real>  ! Angle of attack
  beginning[psi_position]   = <Real>  ! Roll angle
\end{example}
If the floor position is not specified, the default is to place
beginning element at the origin with all angles set to zero.

\index{root branch}\index{branch}
The \vn{beginning} statement is useful in situations where only parameters for
the first branch need be specified. If this is not the case, the parameters for
any branch can be specified using a statement of the form
\begin{example}
  line_name[parameter] = <Value>
\end{example}
This construct is called a \vn{line parameter} statement
Here \vn{line_name} is the name of a line and \vn{parameter} is the
name of a parameter. The parameters that can be set here are the same
parameters that can be set with the \vn{beginning} statement with the additional
parameters from the \vn{parameter} statement:
\index{geometry}\index{particle}\index{default_tracking_species}
\begin{example}
  geometry
  particle
  default_tracking_species
\end{example}
Example:
\begin{example}
  x_ray_fork: fork, to_line = x_ray
  x_ray = (...)
  x_ray[E_tot] = 100
\end{example}

Rules:
  \begin{enumerate}
  \item
The floor position of a line can only be set if the line is used for a 
\vn{root} \vn{branch}. 
  \item
Line parameters statements must come after the associated line. This
rule is similar to the rule that element attribute redefinitions must
come after the definition of the element.
 \end{enumerate}

%-----------------------------------------------------------------
\section{Bmad Global Parameters}
\label{s:bmad.params}
\index{Bmad!general parameters|hyperbf}

\index{max_aperture_limit}\index{d_orb(6)}
\index{grad_loss_sr_wake}\index{default_ds_step}
\index{rel_tol_tracking}\index{abs_tol_tracking}
\index{rel_tol_adaptive_tracking}\index{abs_tol_adaptive_tracking}
\index{taylor_order}\index{default_integ_order}
\index{sr_wakes_on}\index{lr_wakes_on}
\index{mat6_track_symmetric}\index{auto_bookkeeper}
\index{space_charge_on}\index{coherent_synch_rad_on}
\index{spin_tracking_on}\index{radiation_damping_on}
\index{radiation_fluctuations_on}\index{be_thread_safe}
\index{absolute_time_tracking_default}
\index{bmad_common_struct|hyperbf}
Some overall parameters are stored in the \vn{bmad_common_struct}
structure. The instance name here is \vn{bmad_com}. The parameters of
this structure along with the default values are:
\begin{example}
  type bmad_common_struct
    real(rp) max_aperture_limit = 1e3          ! Max Aperture.
    real(rp) d_orb(6)           = 1e-5         ! for the make_mat6_tracking routine.
    real(rp) default_ds_step    = 0.2          ! Integration step size.  
    real(rp) significant_length = 1e-10        ! meter 
    real(rp) rel_tol_tracking = 1e-8           ! Closed orbit relative tolerance.
    real(rp) abs_tol_tracking = 1e-10          ! Closed orbit absolute tolerance.
    real(rp) rel_tol_adaptive_tracking = 1e-8  ! Runge-Kutta tracking relative tolerance.
    real(rp) abs_tol_adaptive_tracking = 1e-10 ! Runge-Kutta tracking absolute tolerance.
    real(rp) init_ds_adaptive_tracking = 1e-3  ! Initial step size.
    real(rp) min_ds_adaptive_tracking = 0      ! Minimum step size to use.
    real(rp) fatal_ds_adaptive_tracking = 1e-8 ! Threshold for loosing particles.
    integer taylor_order = 3                   ! 3rd order is default
    integer default_integ_order = 2            ! PTC integration order
    integer ptc_max_fringe_order = 2           ! PTC max fringe order (2 => Quadrupole !).
    logical sr_wakes_on = T                    ! Short range wake fields?
    logical lr_wakes_on = T                    ! Long range wake fields
    logical mat6_track_symmetric = T           ! symmetric offsets
    logical auto_bookkeeper = T                ! Automatic bookkeeping?
    logical space_charge_on = F                ! Space charge switch
    logical coherent_synch_rad_on = F          ! csr 
    logical spin_tracking_on = F               ! spin tracking?
    logical radiation_damping_on = F           ! Damping toggle.
    logical radiation_fluctuations_on = F      ! Fluctuations toggle.
    logical conserve_taylor_maps = T           ! Enable bookkeeper to set
                                               ! ele%taylor_map_includes_offsets = F?
    logical absolute_time_tracking_default = F ! Default for lat%absolute_time_tracking
    logical aperture_limit_on = T              ! Use aperture limits in tracking.
    logical debug = F                          ! Used for code debugging.
  end type
\end{example}

\begin{description}
\index{aperture_limit_on}
\item[\vn{\%aperture_limit_on]}] \Newline
Aperture limits may be set for elements in the lattice
(\sref{s:limit}). Setting \vn{aperture_limit_on} to \vn{False} will
disable all set apertures. \vn{True} is the default.

\item[\vn{\%max_aperture_limit}] \Newline 
Sets the maximum amplitude a particle can have
during tracking. If this amplitude is exceeded, the particle is lost
even if there is no element aperture set. Having a maximum aperture
limit helps prevent numerical overflow in the tracking calculations.

\item[\vn{\%d_orb}] \Newline 
Sets the orbit displacement used in the routine that calculates the
transfer matrix through an element via tracking.  \vn{%d_orb} needs to
be large enough to avoid significant round-off errors but not so large
that nonlinearities will affect the results. Also see
\vn{%mat6_track_symmetric}.

\item[\vn{\%default_ds_step}] \Newline
Step size for tracking code \sref{c:methods} that uses a fixed step
size. For example, \vn{symp_lie_ptc} and \vn{boris} tracking.

\item[\vn{\%significant_length}] \Newline
Sets the scale to decide if two length values are significantly
different. For example, The superposition code will not create any
super_slave elements that have a length less then this.

\item[\vn{\%rel_tol_tracking}] \Newline
Relative tolerance to use in tracking. Specifically, Tolerance to use
when finding the closed orbit.

\item[\vn{\%abs_tol_tracking}] \Newline
Absolute tolerance to use in tracking. Specifically, Tolerance to use
when finding the closed orbit.

\item[\vn{\%rel_tol_adaptive_tracking}] \Newline
Relative tolerance to use in adaptive tracking. This is used in
\vn{runge_kutta} and \vn{time_runge_kutta} tracking (\sref{s:integ}).

\item[\vn{\%abs_tol_adaptive_tracking}] \Newline
Absolute tolerance to use in adaptive tracking. This is used in
\vn{runge-kutta} and \vn{time_runge_kutta} tracking (\sref{s:integ}).

\item[\vn{\%init_ds_adaptive_tracking}] \Newline
Initial step to use for adaptive tracking. This is used in
\vn{runge-kutta} and \vn{time_runge_kutta} tracking (\sref{s:integ}).

\item[\vn{\%min_ds_adaptive_tracking}] \Newline
This is used in \vn{runge-kutta} and \vn{time_runge_kutta} tracking
(\sref{s:integ}). Minimum step size to use for adaptive tracking. If
To be useful, \vn{%min_ds_adaptive_tracking} must be set larger than
the value of \vn{%fatal_ds_adaptive_tracking}. In this case,
particles are never lost due to taking too small a step.

\item[\vn{\%fatal_ds_adaptive_tracking}] \Newline
This is used in \vn{runge-kutta} and \vn{time_runge_kutta} tracking
(\sref{s:integ}).  If the step size falls below the value set for
\vn{%fatal_ds_adaptive_tracking}, a particle is considered lost.
This prevents a program from ``hanging'' due to taking a large number
of extremely small steps. The most common cause of small step size is
an ``unphysical'' magnetic or electric field.

\item[\vn{\%taylor_order}] \Newline
Cutoff Taylor order of maps produced by \vn{sym_lie_ptc}.

\item[\vn{\%default_integ_order}] \Newline
Order of the the integrator used by \'Etienne Forest's PTC code (\sref{s:libs}).

\item{ptc_max_fringe_order} \Newline
Maximum order for computing fringe field effects in PTC. 

\item[\vn{\%sr_wakes_on}] \Newline
Toggle for turning on or off short-range higher order mode wake field effects.

\item[\vn{\%lr_wakes_on}] \Newline
Toggle for turning on or off long-range higher order mode wake field effects.

\item[\vn{\%mat6_track_symmetric}] \Newline
Toggle to turn off whether the transfer matrix from tracking routine
(\Hyperref{r:twiss.from.tracking}{twiss_from_tracking}) tracks 12 particles at both plus and
minus \vn{%d_orb} values or only tracks 7 particles to save time.

\item[\vn{\%auto_bookkeeper}] \Newline
Toggles automatic or intelligent bookkeeping. See
section~\sref{s:lat.bookkeeping} for more details.

\item[\vn{\%space_charge_on}] \Newline
Toggle to turn on or off high energy space charge effect in particle tracking.

\item[\vn{\%coherent_synch_rad_on}] \Newline
Toggle to turn on or off the coherent space charge calculation.

\item[\vn{\%spin_tracking_on}] \Newline
Determines if spin tracking is performed or not.

\item[\vn{\%radiation_damping_on}] \Newline
Toggle to turn on or off effects due to radiation damping in particle tracking.

\item[\vn{\%radiation_fluctuations_on}] \Newline
Toggle to turn on or off effects due to radiation fluctuations in particle tracking.

\item[\vn{\%conserve_taylor_maps}] \Newline
Toggle to determine if the Taylor map for an element include any
element ``misalignments''.  See Section~\sref{s:mapoff} for more
details.

\item[\vn{\%absolute_time_tracking_default}] \Newline
Default setting to be applied to a lattice if
\vn{absolute_time_tracking} (\sref{s:param}) is not specified in a
lattice file. Additionally, if an element that is not associated with a lattice
is tracked, \vn{%absolute_time_tracking_default} will be used to
determine whether absolute time tracking is used. 

To change between absolute and relative time tracking
(\sref{s:rf.time}) after lattice file parsing, the
\vn{%absolute_time_tracking} component of a \vn{lat_struct}
(\sref{s:abs.time}) can be appropriately set.

\item[\vn{\%debug}] \Newline
Used for communication between program units for debugging purposes.

\end{description}

%-----------------------------------------------------------------
\section{Bmad_Com}
\label{s:bmad.com}

\index{bmad_com}
The parameters of the \vn{bmad_com} instance of the
\vn{bmad_common_struct} structure (\sref{s:bmad.common}) can be set in
the lattice file using the syntax
\begin{example}
  bmad_com[parm-name] = value
\end{example}
where \vn{parm-name} is the name of a component of
\vn{bmad_common_struct}. For example:
\begin{example}
  bmad_com[rel_tol_tracking] = 1e-7
\end{example}

Be aware that setting a \vn{bmad_com} parameter value in a lattice
file will affect all computations of a program even if the program
reads in additional lattice files. That is, setting of \vn{bmad_com}
components is ``sticky'' and persists even when other lattice files
are read in. There are two exceptions: A program is always free to
override settings of \vn{bmad_com} parameters. Additionally, a second
lattice file can also override the setting made in a prior lattice
file.
