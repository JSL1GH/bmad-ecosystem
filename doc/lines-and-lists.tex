\chapter{Beam Lines and Replacement Lists}
\label{c:sequence}

\index{branch}
This chapter describes how to define the ordered list of elements that make up a lattice branch
(\sref{s:branch.def}).  In a lattice, branches may be connected together using \vn{fork} or
\vn{photon fork} elements (\vn{s:fork}), or by using \vn{multipass} (\sref{s:multipass}).

%-----------------------------------------------------------------------------
\section{Branch Construction Overview}
\label{s:branch.construct}

\index{list}
A lattice branch (\sref{s:branch.def}) is defined in a lattice file using what are called \vn{beam lines}
(\sref{s:lines.wo.arg}) and \vn{replacement lists} (\sref{s:replace.list}).  The \vn{beam lines} are
divided into two types - lines with (\sref{s:lines.with.arg}) and lines without
(\sref{s:lines.wo.arg}) \vn{replacement arguments}. This essentially corresponds to the \mad
definition of lines and lists. There can be multiple \vn{beam lines} and \vn{replacement lists}
defined in a lattice file and lines and lists can be nested inside other lines and lists.

Since lines can be nested within other lines, The same element name may be repeated multiple times
in a branch. To distinguish between multiple elements of the same name, lines and lists may be
\vn{tagged} (\sref{s:tag}) to produce unique element names.

A marker element named \vn{END} will be automatically placed at the ends of the branches unless a
\vn{parameter[no_end_marker]} statement (\sref{s:param}) is used to suppress the insertion.
Additionally, if an ending marker named \vn{END} is already present in the lattice file, no extra
marker will be created.

Branches are ordered in an array (\sref{s:lattice.def}) and each branch is assigned an index number
starting with index 0. When there are multiple branches in a lattice, the reference orbit
(\sref{s:ref}) of a branch must not depend upon details of branches later on in the array.  \bmad
depends upon this and calculates the reference orbits of the branches one at a time starting with
the first branch.

%-----------------------------------------------------------------------------
\section{Beam Lines and Lattice Expansion}
\label{s:lines.wo.arg}
\index{line|hyperbf}

A \vn{beam line} without arguments has the format
\begin{example}
  label: line = (member1, member2, ...)
\end{example}
where \vn{member1}, \vn{member2}, etc. are either elements, other \vn{beam lines} or \vn{replacement
lists}, or sublines enclosed in parentheses.  Example:
\begin{example}
  line1: line = (a, b, c)
  line2: line = (d, line1, e)
  use, line2
\end{example}
The \vn{use} statement is explained in Section~\sref{s:use}.  This example shows how a \vn{beam
line} member can refer to another \vn{beam line}. This is helpful if the same sequence of elements
appears repeatedly in the lattice.

The process of constructing the ordered sequences of elements that comprise the branches of the
lattice is called \vn{lattice expansion}. In the example above, when \vn{line2} is expanded to form
the lattice (in this case there is only one branch so \vn{lattice} and \vn{branch} can be considered
synonymous), the definition of \vn{line1} will be inserted in to produce the following lattice:
\begin{example}
  beginning, d, a, b, c, e, end
\end{example}
The \vn{beginning} and \vn{end} marker elements are automatically inserted at the beginning and end
of the lattice. The \vn{beginning} element will always exist but insertion of the \vn{end} element
can be suppressed by inserting into the lattice:
\begin{example}
 parameter[no_end_marker] = T    ! See: \sref{s:param}
\end{example}
Lattice expansion occurs either at the end after the lattice file has been parsed, or, during parsing, at the
point where an \vn{expand_lattice} statement (\sref{s:expand}) is found.

Each element is assigned an \vn{element index} number starting from 0 for the \vn{beginning}
element, 1 for the next element, etc.

In the expanded lattice, any \vn{null_Ele} type elements (\sref{s:null.ele}) will be discarded. For
example, if element \vn{b} in the above example is a \vn{null_Ele} then the actual expanded lattice
will be:
\begin{example}
  beginning, d, a, c, e, end
\end{example}

\index{reflection of elements}
A member that is a line or list can be ``reflected'' (elements taken in reverse order) if a negative
sign is put in front of it. For example:
\begin{example}
  line1: line = (a, b, c)
  line2: line = (d, -line1, e)
\end{example}
\vn{line2} when expanded gives
\begin{example}
  d, c, b, a, e
\end{example}
It is important to keep in mind that line reflection is \vn{not} the same as going backwards through
elements. For example, if an \vn{sbend} or \vn{rbend} element (\sref{s:bend}) is reflected, the
face angle of the upstream edge (\sref{s:ref.construct}) is still specified by the \vn{e1} attribute
and not the \vn{e2} attribute. True element reversal can be accomplished as discussed in \Sref{s:ele.reverse}.

Reflecting a subline will also reflect any sublines of the subline. For example:
\begin{example}
  line0: line = (y, z)
  line1: line = (line0, b, c)
  line2: line = (d, -line1, e)
\end{example}
\vn{line2} when expanded gives
\begin{example}
  d, c, b, z, y, e
\end{example}
\index{sbend}\index{rbend}

A repetition count, which is an integer followed by an asterisk, 
means that the member is
repeated. For example
\begin{example}
  line1: line = (a, b, c)
  line2: line = (d, 2*line1, e)
\end{example}
\vn{line2} when expanded gives
\begin{example}
  d, a, b, c, a, b, c, e
\end{example}
Repetition count can be combined with reflection. For example
\begin{example}
  line1: line = (a, b, c)
  line2: line = (d, -2*line1, e)
\end{example}
\vn{line2} when expanded gives
\begin{example}
  d, c, b, a, c, b, a, e
\end{example}
Instead of the name of a line, subline members can also be given as an explicit list using
parentheses. For example, the previous example could be rewritten as
\begin{example}
  line2: line = (d, -2*(a, b, c), e)
\end{example}

Lines can be defined in any order in the lattice file so a subline does not have to come before a
line that references it. Additionally, element definitions can come before or after any lines that
reference them.

A line can have the \vn{multipass} attribute. This is covered in \sref{s:multipass}.

%-----------------------------------------------------------------------------
\section{Line Slices}
\label{s:line.slice}
\index{line slice}

A line ``\vn{slice}'' is a section of a line from some starting element to some ending element.  A
line slice can be used to construct a new line similar to how an unsliced line is used to construct
a new line. An example will make this clear:
\begin{example}
  line1: line = (a, b, c, d, e)
  line2: line = (z1, line1[b:d], z2)
\end{example}
The line slice \vn{line1[b:d]} that is used to construct \vn{line2} consists of the elements in
\vn{line1} from element \vn{b} to element \vn{d} but not elements \vn{a} or \vn{e}. When \vn{line2}
is expanded, it will have the elements:
\begin{example}
  z1, b, c, d, z2
\end{example}

The general form for line slices is
\begin{example}
  line_name[element1:element2]
\end{example}
where \vn{line_name} is the name of the line and \vn{element1} and \vn{element2} delimit the
beginning and ending positions of the slice. The beginning and ending element names may be omitted
and, if not present, the default is the beginning element and ending element of the line
respectively. Thus, for example, ``\vn{line4[:q1]}'' represents the list of elements from the start
of \vn{line4} up to, and including the element \vn{q1}.

If there are multiple elements of the same name, the double hash \vn{\#\#} symbol
(\sref{s:ele.match}) can be use to denote the N\Th element of a given name. If double hash is not
used, the first instance of a given element name is assumed. That is, something like ``\vn{q1}'' is
equivalent to ``\vn{q1\#\#1}''.

Wild card characters and \vn{class::element_name} syntax (\sref{s:ele.match}) are not allowed with
slice element names.

Line slicing of a given line occurs after the line has been expanded (all sublines and line slices
substituted in). Thus, the following makes sense:
\begin{example}
  line1: line = (a, b, c, d, e)
  line2: line = (z1, line1, z2)
  line3: line = (line2[z1:c])
\end{example}

%-----------------------------------------------------------------------------
\section{Element Reversal}
\label{s:ele.reverse}
\index{element reversal}

An element is \vn{reversed} if particles traveling through it enter at the ``exit'' end and leave at
the ``entrance'' end. Being able to reverse elements is useful, for example, in describing the
interaction region of a pair of rings where particles of one ring are going in the opposite
direction relative to the particles in the other ring.

Element reversal is indicated by using a double negative sign ``--'' prefix. The double negative sign
prefix can be applied to individual elements or to a line. If it is applied to a line, the line is
both reflected (same as if a single negative sign is used) and each element is reflected. For
example:
\begin{example}
  line1: line = (a, b, --c)
  line2: line = (--line1)
  line3: line = (c, --b, --a)
\end{example}
In this example, \vn{line2} and \vn{line3} are identical. Notice that the reversal of a reversed
element makes the element unreversed.

Another example involving element reversal is given in Section~\sref{s:reverse}.

Reversed elements, unlike other elements, have their local $z$-axis pointing in the
opposite direction to the local $s$-axis (\sref{s:ref.construct}). This means that there
must be a \vn{reflection patch} (\sref{s:reflect.patch}) between reversed and unreversed
elements. See \sref{s:ex.patch} for an example. Since this complicates matters, it is
generally only useful to employ element reversal in cases where there are multiple
intersecting lines with particle beams going in opposite directions through some elements
(for example, colliding beam interaction regions). In this case, element reversal is
typically used with \vn{multipass} (\sref{s:multipass}).

Where reversed elements are not needed, it is simple to define elements that are
effectively reversed. For example:
\begin{example}
  b00: bend, angle = 0.023, e1 = ...
  b00_rev: b00, angle = -b00[angle], e1 = -b00[e2], e2 = -b00[e1]
\end{example}
and \vn{b00_rev} serves as a reversed version of \vn{b00}.

Internally, \bmad associates an \vn{orientation} attribute with each element. This attribute is set
to -1 for reversed elements and 1 for unreversed elements. This attribute. If a program can print
out the attributes for an element, checking the \vn{orientation} attribute will show if an element
is reversed or not.

%-----------------------------------------------------------------------------
\section{Beam Lines with Replaceable Arguments}
\label{s:lines.with.arg}
\index{line!with arguments}

\vn{Beam lines} can have an argument list using the following syntax
\begin{example}
  line_name(dummy_arg1, dummy_arg2, ...): LINE = (member1, member2, ...)
\end{example}
The dummy arguments are replaced by the actual arguments when the line is used
elsewhere. For example:
\begin{example}
  line1(DA1, DA2): line = (a, DA2, b, DA1)
  line2: line = (h, line1(y, z), g)
\end{example}
When \vn{line2} is expanded the actual arguments of \vn{line1}, in this case \vn(y, z), replaces the
dummy arguments \vn{(DA1, DA2)} to give for \vn{line2}
\begin{example}
  h, a, z, b, y, g
\end{example} 
\index{MAD}
Unlike \mad, \vn{beam line} actual arguments can only be elements or \vn{beam lines}. 
Thus the following is not allowed
\begin{example}
  line2: line = (h, line1(2*y, z), g)   ! NO: 2*y NOT allowed as an argument.
\end{example}

%-----------------------------------------------------------------------------
\section{Lists}
\label{s:replace.list}
\index{list|hyperbf}

When a lattice is expanded, all the lattice members that correspond to a name of a \vn{list} are
replaced successively, by the members in the \vn{list}. The general syntax is
\begin{example}
  label: LIST = (member1, member2, ...)
\end{example}
For example:
\begin{example}
  my_list1 list = (a, b, c)
  line1: line = (z1, my_list, z2, my_list, z3, my_list, z4, my_list)
  use, line1
\end{example}
When the lattice is expanded the first instance of \vn{my_list} in \vn{line1} is replaced by \vn{a}
(which is the first element of \vn{my_list}), the second instance of \vn{my_list} is replaced by
\vn{b}, etc. If there are more instances of \vn{my_list} in the lattice then members of
\vn{my_list}, the replacement starts at the beginning of \vn{my_list} after the last member of
\vn{my_list} is used. In this case the lattice would be:
\begin{example}
  z1, a, z2, b, z3, c, z4, a
\end{example}
members of a \vn{replacement list} can only be simple elements and not other lines or lists. 
For example, the following is not allowed:
\begin{example}
  line1: line = (a, b)
  my_list: list = (2*line1)  ! Lines cannot be list members.
\end{example}
A repetition count is permitted
\begin{example}
  my_list1: list = (2*a, b) 
  my_list2: list = (a, a, b) ! Equivalent to my_list1
\end{example}

%-----------------------------------------------------------------------------
\section{Use Statement}
\label{s:use}

\index{use statement|hyperbf}
The particular line or lines that defines the root branches (\sref{s:lattice.def}) to be used in the
lattice is selected by the \vn{use} statement. The general syntax is
\begin{example}
  use, line1, line2 ...
\end{example}
For example, \vn{line1} may correspond to one ring and \vn{line2} may correspond to the other ring
of a dual ring colliding beam machine. In this case, \vn{multipass} (\sref{s:multipass}) will be
needed to describe the common elements of the two rings. Example
\begin{example}
  use, e_ring, p_ring
\end{example}
would pick the lines \vn{e_ring} and \vn{p_ring} for analysis.  These will be the \vn{root}
branches.

\vn{use} statements can come anywhere in the lattice, even before the definition of the lines they
refer to. Additionally, there can be multiple \vn{use} statements.  The last \vn{use} statement in
the file defines which \vn{line} to use.

The total number of branches in the lattice is equal to the number of lines that appear on the
\vn{use} statement plus the number of \vn{fork} and \vn{photon_fork} elements that branch to a new
branch.

To set such things as the geometry of a branch, beginning Twiss parameters, etc., see Section
\vn{s:beginning}.

%-----------------------------------------------------------------------------
\section{Tagging Lines and Lists}
\index{tags for lines and lists|hyperbf}
\label{s:tag}

When a lattice has repeating lines, it can be desirable to differentiate
between repeated elements. This can be done by tagging lines with a \vn{tag}. 
An example will make this clear:
\begin{example}
  line1: line = (a, b)
  line2: line = (line1, line1)
  use, line2
\end{example}
When expanded the lattice would be:
\begin{example}
  a, b, a, b
\end{example}
The first and third elements have the same name ``a'' and the second and fourth
elements have the same name ``b''. Using tags the lattice elements can be given
unique names. lines or lists are tagged  
using the at (@) sign. The general syntax is:
\begin{example}
  tag_name@line_name                           ! Syntax for lines
  tag_name@list_name                           ! Syntax for lists
  tag_name@replacement_line(arg1, arg2, ...)   ! Syntax for replacement lines.
\end{example}
Thus to differentiate the lattice elements in the above example \vn{line2} needs to
be modified using tags:
\begin{example}
  line1: line = (a, b)
  line2: line = (t1@line1, t2@line1)
  use, line2
\end{example}
In this case the lattice elements will have names of the form:
\begin{example}
  tag_name.element_name
\end{example}
In this particular example, the lattice with tagging will be:
\begin{example}
  t1.a, t1.b, t2.a, t2.b
\end{example}
Of course with this simple example one could have just as easily not used tags:
\begin{example}
  t1.a: a;   t2.a: a
  t1.b: b;   t2.b: b
  line1: line = (t1.a, t1.b, t2.a, t2.b)
  use, line2
\end{example}
But in more complicated situations tagging can make for compact lattice files.

When lines are nested, the name of an element is formed by concatenating the tags together with dots
in between in the form:
\begin{example}
  tag_name1.tag_name2. ... tag_name_n.element_name
\end{example}
An example will make this clear:
\begin{example}
  list1 = (g, h)
  line1(y, z) = (a, b)
  line2: line = (t1@line1(a, b))
  line3: line = (line2, hh@list1)
  line4: line = (z1@line3, z2@line3)
  use, line4
\end{example}
The lattice elements in this case are:
\begin{example}
  z1.t1.a, z1.t1.b, z1.hh.g, z2.t1.a, z2.t1.b, z1.hh.h 
\end{example}

\index{expand_lattice}
To modify a particular tagged element the lattice must be expanded
first (\sref{s:expand}). For example:
\begin{example}
  line1: line = (a, b)
  line2: line = (t1@line1, t2@line1)
  use, line2
  expand_lattice
  t1.b[k1] = 1.37
  b[k1] = 0.63       ! This statement generates an error
\end{example}
After the lattice has been expanded there is no connection between the original \vn{a} and \vn{b}
elements and the elements in the lattice like \vn{t1.b}. Thus the last line in the example where the
\vn{k1} attribute of\vn{b} is modified generates an error since there are no elements named \vn{b}
in the lattice.
