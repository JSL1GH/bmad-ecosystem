\chapter{Beam Lines and Replacement Lists}
\label{c:sequence}

\index{branch}
This chapter describes how to define the ordered list of elements that
make up a lattice branch (\sref{s:branch.def}).  In a lattice,
branches may be connected together using \vn{branch} or \vn{photon
branch} elements (\vn{s:branch}), or by using \vn{multipass} (\sref{s:multipass}).

%-----------------------------------------------------------------------------
\section{Branch Construction Overview}
\label{s:branch.construct}

\index{list}
A lattice branch is defined in a lattice file using what are called
\vn{beam lines} (\sref{s:lines.wo.arg}) and \vn{replacement lists}
(\sref{s:replace.list}).  The \vn{beam lines} are divided into two
types - lines with (\sref{s:lines.with.arg}) and lines without
(\sref{s:lines.wo.arg}) \vn{replacement arguments}. This essentially
corresponds to the \mad definition of lines and lists. There can be
multiple \vn{beam lines} and \vn{replacement lists} defined in a
lattice file and lines and lists can be nested inside other lines and
lists.

Since lines can be nexted within other lines, The same element name
may be repeated multiple times in a brach. To distinguish between
mutiple elements of the same name, lines and lists may be \vn{tagged} 
(\sref{s:tag}) to produce unique element names.

There will also be a marker element named \vn{END} automatically
placed at the end of the lattice. This end marker will not be
automatically placed in the lattice if a marker named \vn{end} is
defined in the lattice file at the end of the lattice. The
\vn{parameter[no_end_marker]} statement (\sref{s:param}) can be used
to suppress the insertion of the end marker.

%-----------------------------------------------------------------------------
\section{Beam Lines and Lattice Expansion}
\label{s:lines.wo.arg}
\index{line|hyperbf}

A \vn{beam line} without arguments has the format
\begin{example}
  label: line = (member1, member2, ...)
\end{example}
where \vn{member1}, \vn{member2}, etc. are either elements, other \vn{beam
lines} or \vn{replacement lists}, or sublines enclosed in parentheses.
Example:
\begin{example}
  line1: line = (a, b, c)
  line2: line = (d, line1, e)
  use, line2
\end{example}
The \vn{use} statement is explained in Section~\sref{s:use}.
This example shows how a \vn{beam line} member can refer to another
\vn{beam line}. This is helpful if the same sequence of elements
appears repeatedly in the lattice. 

The process of constructing the ordered sequence of elements that
comprise the lattice is called \vn{lattice expansion}. In the example
above, when \vn{line2} is expanded to form the lattice, the definition
of \vn{line1} will be inserted in to produce the following lattice:
\begin{example}
  beginning, d, a, b, c, e, end
\end{example}
The \vn{beginning} and \vn{end} marker elements are automatically inserted
at the beginning and end of the lattice. The \vn{beginning} element will
always exist but insertion of the \vn{end} element can be supressed by inserting
into the lattice:
\begin{example}
 parameter[no_end_marker] = T    ! See: \sref{s:param}
\end{example}
Lattice expansion occurs at the end when a lattice file has been
parsed or if an \vn{expand_lattice} statement (\sref{s:expand}) is
present.

Each element is assigned an \vn{element index} number starting from 0
for the \vn{beginning} element, 1 for the next element, etc.

In the expanded lattice, any \vn{null_Ele} type elements
(\sref{s:null.ele}) will be discarded. For example, if element \vn{b}
in the above example is a \vn{null_Ele} then the actual expanded
lattice will be:
\begin{example}
  beginning, d, a, c, e, end
\end{example}

\index{reflection of elements}
A member that is a line or list can be ``reflected''
(elements taken in reverse order) if
a negative sign is put in front of it. For example:
\begin{example}
  line1: line = (a, b, c)
  line2: line = (d, -line1, e)
\end{example}
\vn{line2} when expanded gives
\begin{example}
  d, c, b, a, e
\end{example}
Reflecting a subline will also reflect any sublines of the subline. For
example:
\begin{example}
  line0: line = (y, z)
  line1: line = (line0, b, c)
  line2: line = (d, -line1, e)
\end{example}
\vn{line2} when expanded gives
\begin{example}
  d, c, b, z, y, e
\end{example}
\index{sbend}\index{rbend}

A repetition count, which is an integer followed by an asterisk, 
means that the member is
repeated. For example
\begin{example}
  line1: line = (a, b, c)
  line2: line = (d, 2*line1, e)
\end{example}
\vn{line2} when expanded gives
\begin{example}
  d, a, b, c, a, b, c, e
\end{example}
Repetition count can be combined with reflection. For example
\begin{example}
  line1: line = (a, b, c)
  line2: line = (d, -2*line1, e)
\end{example}
\vn{line2} when expanded gives
\begin{example}
  d, c, b, a, c, b, a, e
\end{example}
Instead of the name of a line, subline members can also be given as an explicit 
list using parentheses. For example, the previous example could be rewritten as
\begin{example}
  line2: line = (d, -2*(a, b, c), e)
\end{example}

Lines can be defined in any order in the lattice file so a subline
does not have to come before a line that references it. Additionally,
element definitions can come before or after any lines that reference
them.

A line can have the \vn{multipass} attribute. This is covered in
\sref{s:multipass}.

%-----------------------------------------------------------------------------
\section{Element Reversal}
\label{s:ele.reverse}
\index{element reversal}

An element is \vn{reversed} if particles traveling through it enter at
the ``exit'' end and leave at the ``entrance'' end. Being able to
reverse elements is useful, for example, in describing the interaction
region of a pair of rings where particles of one ring are going in the
opposite direction relative to the particles in the other ring.

Elment reversal is indicated by using a double negative sign ``--''
prefix. The double negative sign prefix can be applied to individual
elements or to a line. If it is applied to a line, the line is both
reflected (same as if a single negative sign is used) and each element
is reflected. For example:
\begin{example}
  line1: line = (a, b, --c)
  line2: line = (--line1)
  line3: line = (c, --b, --a)
\end{example}
In this example, \vn{line2} and \vn{line3} are identical. Notice that
the reversal of a reversed element makes the element unreversed.

%-----------------------------------------------------------------------------
\section{Beam Lines with Replaceable Arguments}
\label{s:lines.with.arg}
\index{line!with arguments}

\vn{Beam lines} can have an argument list using the following syntax
\begin{example}
  line_name(dummy_arg1, dummy_arg2, ...): LINE = (member1, member2, ...)
\end{example}
The dummy arguments are replaced by the actual arguments when the line is used
elsewhere. For example:
\begin{example}
  line1(DA1, DA2): line = (a, DA2, b, DA1)
  line2: line = (h, line1(y, z), g)
\end{example}
When \vn{line2} is expanded the actual arguments of \vn{line1}, in this
case \vn(y, z), replaces the dummy arguments \vn{(DA1, DA2)} to give for
\vn{line2}
\begin{example}
  h, a, z, b, y, g
\end{example} 
\index{MAD}
Unlike \mad, \vn{beam line} actual arguments can only be elements or \vn{beam lines}. 
Thus the following is not allowed
\begin{example}
  line2: line = (h, line1(2*y, z), g)   ! NO: 2*y NOT allowed as an argument.
\end{example}

%-----------------------------------------------------------------------------
\section{Replacement Lists}
\label{s:replace.list}
\index{list|hyperbf}

When a lattice is expanded, all the lattice members that correspond to
a name of a \vn{replacement list} are replaced successively, by the
members in the \vn{replacement list}. The general syntax is
\begin{example}
  label: LIST = (member1, member2, ...)
\end{example}
For example:
\begin{example}
  list1: list = (a, b, c)
  line1: line = (z1, list1, z2, list1, z3, list1, z4, list1)
  use, line1
\end{example}
When the lattice is expanded the first instance of \vn{list1} in
\vn{line1} is replaced by \vn{a} (which is the first element of
\vn{list1}), the second instance of \vn{list1} is replaced by \vn{b},
etc. If there are more instances of \vn{list1} in the lattice then
members of \vn{list1}, the replacement starts at the beginning of
\vn{list1} after the last member of \vn{list1} is used. In this case the
lattice would be:
\begin{example}
  z1, a, z2, b, z3, c, z4, a
\end{example}
\index{MAD}
Unlike \mad members of a \vn{replacement list} can only be simple elements 
without reflection or repetition count and not other lines or lists. 
For example the following is not allowed:
\begin{example}
  list1: list = (2*a, b)  ! NO: No repetition count allowed.
\end{example}

%-----------------------------------------------------------------------------
\section{Use Statement}
\label{s:use}

\index{use statement|hyperbf}
The particular line or lines that defines the root branches
(\sref{s:lattice.def}) to be used in the lattice is selected by the
\vn{use} statement. The general syntax is
\begin{example}
  use, line1, line2 ...
\end{example}
For example, \vn{line1} may correspond to one ring and \vn{line2} may
correspond to the other ring of a dual ring colliding beam machine. In
this case, \vn{multipass} (\sref{s:multipass}) will be needed to
describe the common elements of the two rings. Example
\begin{example}
  use, e_ring, p_ring
\end{example}
would pick the lines \vn{e_ring} and \vn{p_ring} for analysis. 

\vn{use} statements can come anywhere in the lattice, even before the
definition of the lines they refer to. Additionally, there can be
multiple \vn{use} statements.  The last \vn{use} statement in the file
defines which \vn{line} to use.

%-----------------------------------------------------------------------------
\section{Line and List Tags}
\index{tags for lines and lists|hyperbf}
\label{s:tag}

When a lattice has repeating lines, it can be desirable to differentiate
between repeated elements. This can be done by tagging lines with a \vn{tag}. 
An example will make this clear:
\begin{example}
  line1: line = (a, b)
  line2: line = (line1, line1)
  use, line2
\end{example}
When expanded the lattice would be:
\begin{example}
  a, b, a, b
\end{example}
The first and third elements have the same name ``a'' and the second and fourth
elements have the same name ``b''. Using tags the lattice elements can be given
unique names. lines or lists are tagged  
using brackets \vn{[...]}. The general syntax is:
\begin{example}
  line_name[tag_name]                           ! Syntax for lines
  list_name[tag_name]                           ! Syntax for lists
  replacement_line[tag_name](arg1, arg2, ...)   ! Syntax for replacement lines.
\end{example}
Thus to differentiate the lattice elements in the above example \vn{line2} needs to
be modified using tags:
\begin{example}
  line1: line = (a, b)
  line2: line = (line1[t1], line1[t2])
  use, line2
\end{example}
In this case the lattice elements will have names of the form:
\begin{example}
  tag_name.element_name
\end{example}
In this particular example, the lattice with tagging will be:
\begin{example}
  t1.a, t1.b, t2.a, t2.b
\end{example}
Of course with this simple example one could have just as easily not used tags:
\begin{example}
  t1.a: a;   t2.a: a
  t1.b: b;   t2.b: b
  line1: line = (t1.a, t1.b, t2.a, t2.b)
  use, line2
\end{example}
But in more complicated situations tagging can make for compact lattice files.

When lines are nested, the name of an element is formed by concatenating the tags
together with dots in between in the form:
\begin{example}
  tag_name1.tag_name2. ... tag_name_n.element_name
\end{example}
An example will make this clear:
\begin{example}
  list1 = (g, h)
  line1(y, z) = (a, b)
  line2: line = (line1[t1](a, b))
  line3: line = (line2, list1[hh])
  line4: line = (line3[z1], line3[z2])
  use, line4
\end{example}
The lattice elements in this case are:
\begin{example}
  z1.t1.a, z1.t1.b, z1.hh.g, z2.t1.a, z2.t1.b, z1.hh.h 
\end{example}

\index{expand_lattice}
To modify a particular tagged element the lattice must be expanded
first (\sref{s:expand}). For example:
\begin{example}
  line1: line = (a, b)
  line2: line = (line1[t1], line1[t2])
  use, line2
  expand_lattice
  t1.b[k1] = 1.37
  b[k1] = 0.63       ! This statement does not have any effect
\end{example}
After the lattice has been expanded there is no connection between 
the original \vn{a} and \vn{b} elements and the elements in the lattice like
\vn{t1.b}. Thus the last line in the example where the \vn{k1} attribute of\vn{b} 
is modified do not have any effect on the lattice elements. 
