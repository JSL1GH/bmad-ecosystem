\chapter{Lattice File Conversion}
\label{c:lat.convert}
\index{conversion to other lattice formats}

A \bmad Distribution (\sref{s:tao.intro}) contains a number of translation programs between \bmad
and other formats.


%-----------------------------------------------------------------------------
\section{MAD Conversion}
\label{s:mad.convert}
\index{MAD!conversion}

%-----------------------------------------------------------------------------
\subsection{Convert MAD to Bmad}
\label{s:mad.bmad.uap}

Python scripts to convert from MAD8 and MADX are available at:
\begin{example}
  util_programs/mad_to_bmad
\end{example}
Due to differences in language definitions, conversions must be done with some care. The following
differences should be noted:
  \begin{itemize}
  \item
\bmad, unlike \mad, does not have any ``action'' commands. An action command is a command that makes
a calculation. Examples include \mad's \vn{SURVEY} and \vn{TWISS} commands.
  \item
In \bmad all variables must be defined before being used (\sref{s:arith}) while \mad will simply take
a variable's value to be zero if it is not defined.
  \item
\bmad, unlike \mad, does not allow variable values to be redefined.
  \end{itemize}

%-----------------------------------------------------------------------------
\subsection{Convert Bmad to MAD}
\label{s:bmad.mad}

\index{wiggler!conversion to MAD}
\index{sol_quad!conversion to MAD}
To convert to MAD8 or MADX, the \tao program can be used. Additionally, the program
\vn{util_programs/bmad_to_mad_or_sad} is available. Since \mad does not have a \vn{wiggler} or a
\vn{sol_quad} element, this conversion routine makes ``equivalent'' substitution. For a
\vn{sol_quad}, the equivalent substitution will be a drift-matrix-drift series of elements. For a
\vn{wiggler}, a series of bend and drift elements will be used (the program can also use a
drift-matrix-drift model here but that is not as accurate). The bends and drifts for the
\vn{wiggler} model are constructed so that the global geometry of the lattice does not
change. Additionally the bends and drifts are constructed to most nearly match the wiggler's
\begin{example}
  Transfer matrix
  $I_2$ and $I_3$ synchrotron radiation integrals (\sref{s:synch.ints})
\end{example}
Note that the resulting model will not have the vertical cubic nonlinearity that the actual wiggler
has.

%-----------------------------------------------------------------------------
\subsection{Convert to PTC}
\label{s:to.ptc}

A PTC ``\vn{flatfile}'' can be constructed using the \tao program with the following commands:
\begin{example}
  Tao> ptc init
  Tao> write ptc
\end{example}

%---------------------------------------------------------------------------
\section{SAD Conversion}
\label{s:sad.convert}
\index{SAD}

Conversion from \vn{SAD}\cite{b:sad} to \bmad is accomplished using the Python script
\begin{example}
  util_programs/sad_to_bmad/sad_to_bmad.py
\end{example}
Currently, the following restrictions on SAD lattices apply:
  \begin{itemize}
  \item
SAD \vn{mult} elements cannot have an associated RF field
  \item
Misalignments in a \vn{sol} element with \vn{geo} = 1 cannot be handled.
  \end{itemize}

\bmad to \vn{SAD} to conversion can be done with the \tao program or the program 
\vn{util_programs/bmad_to_mad_or_sad}.
