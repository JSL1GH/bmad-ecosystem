\chapter{MAD/XSIF/SAD/PTC Lattice Conversion}
\label{c:lat.convert}
\index{conversion to other lattice formats}

%-----------------------------------------------------------------------------
\section{MAD Conversion}
\label{s:mad.convert}
\index{MAD!conversion}


%-----------------------------------------------------------------------------
\subsection{Convert MAD to Bmad Via UAP}
\label{s:mad.bmad.uap}

Conversion of lattice files from \mad to \bmad format can be done
using the \vn{Universal Accelerator Parser} (\sref{s:aml}). Due to
differences in language definitions, the conversions must be done with
some care. The following differences should be noted:
  \begin{itemize}
  \item
\bmad, unlike \mad, does not have any ``action'' commands. An action
command is a command that makes a calculation. Examples include \mad's
\vn{SURVEY} and \vn{TWISS} commands.
  \item
In \bmad all variables must be defined. In \mad undefined variables
will default to 0.
  \item
In \bmad all variables must be defined before being used
(\sref{s:arith}) while \mad does not have this constraint.
  \item
\bmad, unlike \mad, does not allow variable values to be redefined.
  \item
Elements like a \vn{sad_mult} cannot be translated.
  \end{itemize}

%-----------------------------------------------------------------------------
\subsection{Convert MAD-X to Bmad Via PTC}
\label{s:mad.bmad.ptc}

\mad-X lattices can be converted to \bmad by first using \mad-X to create a PTC
``flat'' file and then reading in the PTC flat file. First, take a \mad-X lattice
and append the following to it:
\begin{example}
  use, period = LINE_NAME_HERE;
  ptc_create_universe;
  ptc_create_layout, model = 1, method = 2, nst = 1; 
  !! ptc_create_layout, model=1, method=2, nst=1, exact;  ! Use this if exact is wanted
  ptc_script, file="create_flat.ptc";
\end{example}
Substitute the actual sequence or line name for ``LINE_NAME_HERE''.

Second, create a file called \vn{create_flat.ptc} with the following in it:
\begin{example}
  select layout
  1
  ! print new flat file
  print flat file
  XXX.flat
  return
\end{example}
The flat file name \vn{XXX.flat} can be changed if desired.

Now run \mad-X with the lattice file. A flat file will be created. This flat file can
converted to \bmad as documented in \sref{s:ptc.convert}


%-----------------------------------------------------------------------------
\subsection{Convert Bmad to MAD}
\label{s:mad.bmad.uap}

\index{wiggler!conversion to MAD}
\index{sol_quad!conversion to MAD}
Besides using the \vn{Universal Accelerator Parser} for conversion
from \bmad to \mad, there is a conversion routine called
\Hyperref{r:write.lattice.in.foreign.format}{write_lattice_in_foreign_format}. 
The advantage of this routine is that since \mad does not have a
\vn{wiggler} or a \vn{sol_quad} element, this conversion routine can
make an ``equivalent'' substitution. For a \vn{sol_quad}, the
equivalent substitution will be a drift-matrix-drift series of
elements. For a \vn{wiggler}, a series of bend and drift elements will
be used (the program can also use a drift-matrix-drift model here but
that is not as accurate). The bends and drifts for the \vn{wiggler}
model are constructed so that the global geometry of the lattice does
not change. Additionally the bends and drifts are constructed to most
nearly match the wiggler's
\begin{example}
  Transfer matrix
  $I_2$ and $I_3$ synchrotron radiation integrals (\sref{s:synch.ints})
\end{example}
Note that the resulting model will not have the vertical cubic
nonlinearity that the actual wiggler has.

The \vn{bmad_to_mad_or_xsif} routine is embeded in the program
\vn{util_programs/bmad_to_mad_or_xsif} which can be used for \bmad to
\mad  or \vn{XSIF} conversions.

%---------------------------------------------------------------------------
\section{XSIF Conversion}
\label{s:xsif.convert}
\index{XSIF!conversion}

XSIF\cite{b:xsif}, developed at SLAC, stands for ``Extended Standard
Input Format.''  XSIF is essentially a subset of the
\mad\cite{b:maduser} input format.

\bmad has software to directly parse XSIF files so XSIF files may be
used in place of \bmad lattice files.  With some restrictions, an XSIF
lattice file may be called from within a \bmad lattice file. See
Section~\sref{s:call} for details.

\index{parameter statement!geometry}
Since XSIF does not have a \vn{parameter[geometry]} statement
(\sref{s:param}), the type of the lattice (whether circular or linear)
is determined by the presence or absence of any \vn{lcavity} elements
in the XSIF file. This is independent of whether \vn{lcavity} elements
are actually used in the lattice.

Note: One point that is not covered in the XSIF documentation is that
for a \vn{MATRIX} element, unlike \mad, the \vn{R$ii$} terms (the
diagonal terms of the linear matrix) are not unity by default. Thus
\index{MAD}
\begin{example}
  m: matrix
\end{example}
in an XSIF file will give a matrix with all elements being zero.

To convert between XSIF and \bmad the \vn{Universal Accelerator
Parser} can be used (\sref{s:aml}). Additionally, the
\vn{bmad_to_mad_or_xsif} routine in \bmad can convert from \bmad to
XSIF (cf.~\sref{s:mad.convert}).

%---------------------------------------------------------------------------
\section{PTC Conversion}
\label{s:ptc.convert}
\index{ptc!conversion}

For a programmer, 
For conversion from \bmad to PTC there are two possibilites

%---------------------------------------------------------------------------
\section{SAD Conversion}
\label{s:sad.convert}
\index{SAD}

Conversion from \vn{SAD}\cite{b:sad} to \bmad is accomplished using the Python script
\begin{example}
  util_programs/sad_to_bmad/sad_to_bmad.py
\end{example}
Currently, a converter from \bmad to SAD is planned but has not been implemented.

Currently, the following restrictions on SAD lattices apply:
  \begin{itemize}
  \item
SAD must elements cannot have an associated RF field
  \item
Misalignments in a \vn{sol} element with \vn{geo} = 1 cannot be handled.
  \end{itemize}

%----------------------------------------------------------------------------
\section{Translation Using the Universal Accelerator Parser}
\label{s:aml}
\index{Accelerator Markup Language (AML)}
\index{Universal Accelerator Parser (UAP)}

\index{MAD}
The \vn{Accelerator Markup Language} (\vn{AML}) / \vn{Universal
Accelerator Parser} (\vn{UAP}) project\cite{b:aml} is a collaborative
effort with the aim of 1) creating a lattice format (the \vn{AML}
part) that can be used to fully describe accelerators and storage
rings, and 2) producing software (the \vn{UAP} part) that can parse
\vn{AML} lattice files. A side benefit of this project is that the
\vn{UAP} code has been extended to be able to translate between
\vn{AML}, \bmad, \mad-8, \mad-X, and \vn{XSIF}. 

The program
\vn{translate_driver} which comes with the \vn{UAP} code can be used
for conversions. To get help with how to run this program use the command
\begin{example}
  path-to-uap-dir/bin/translate_driver -help
\end{example}
Example:
\begin{example}
  path-to-uap-dir/bin/translate_driver -constants_first -bmad xxx.madx
\end{example}
This will convert a \mad-X file \vn{xxx.madx} to \bmad format.
