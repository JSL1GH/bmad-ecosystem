\newcommand{\comma}{\> ,}
\newcommand{\period}{\> .}

\newcommand{\Begineqs}{\begin{eqnarray}}
\newcommand{\Endeqs}{\end{eqnarray}}
\newcommand{\AND}{&& \hskip -17pt\relax}
\newcommand{\CR}{\\}
\newcommand{\CRNO}{\nonumber \\}
\newcommand{\dstyle}{\displaystyle}

\newcommand{\Begineq}{\begin{equation}}
\newcommand{\Endeq}{\end{equation}}
\newcommand{\NoPrint}[1]{}

\newcommand{\pow}[1]{\cdot 10^{#1}}
\newcommand{\Bf}[1]{{\bf #1}}

\newcommand{\bmad}{{\sl Bmad}\xspace}
\newcommand{\tao}{{\sl Tao}\xspace}
\newcommand{\mad}{{\sl MAD}\xspace}
\newcommand{\cesr}{{\sl CESR}\xspace}

\newcommand{\sref}[1]{\S\ref{#1}}
\newcommand{\cref}[1]{Chapter~\ref{#1}}

\newcommand{\Newline}{\hfil \\}

\newcommand{\eq}[1]{{(\protect\ref{#1})}}
\newcommand{\Eq}[1]{{Eq.~(\protect\ref{#1})}}
\newcommand{\Eqs}[1]{{Eqs.~(\protect\ref{#1})}}

\newcommand{\vn}{\ttcmd}           % For variable names
\newcommand{\vni}{\ttcmdindx}
\newcommand{\cs}{\ttcmd}           % For code source
\newcommand{\cmd}{\ttcmd}          % For Unix commands
\newcommand{\rn}{\ttcmd}           % For Routine names
\newcommand{\tn}{\ttcmd}           % For Type (structure) names
\newcommand{\bn}[1]{{\bf #1}}       
\newcommand{\toffset}{\vskip 0.01in}
\newcommand{\rot}[1]{\begin{rotate}{-45}#1\end{rotate}}

\newcommand{\data}{{\mbox{data}}}
\newcommand{\reference}{{\mbox{ref}}}
\newcommand{\model}{{\mbox{model}}}
\newcommand{\base}{{\mbox{base}}}
\newcommand{\design}{{\mbox{design}}}
\newcommand{\meas}{{\mbox{meas}}}
\newcommand{\var}{{\mbox{var}}}

\newcommand{\merit}{{\mbox{merit}}}
\newcommand{\weight}{{\mbox{weight}}}
\newcommand{\actual}{{\mbox{actual\_value}}}
\newcommand{\target}{{\mbox{target\_value}}}

\newcommand\ttcmd{\begingroup\catcode`\_=11 \catcode`\%=11 \dottcmd}
\newcommand\dottcmd[1]{\texttt{#1}\endgroup}

\newcommand\slcmd{\begingroup\catcode`\_=11 \catcode`\%=11 \doslcmd}
\newcommand\doslcmd[1]{\textsl{#1}\endgroup}

\newcommand\ttcmdindx{\begingroup\catcode`\_=11 \catcode`\%=11 \dottcmdindx}
\newcommand\dottcmdindx[1]{\texttt{#1}\endgroup\index{#1}}

\newcommand{\St}{$^{st}$\xspace}
\newcommand{\Nd}{$^{nd}$\xspace}
\newcommand{\Th}{$^{th}$\xspace}
\newcommand{\B}{$\backslash$}
\newcommand{\W}{$^\wedge$}

\newlength{\dPar}
\newlength{\ExBeg}
\newlength{\ExEnd}
\setlength{\dPar}{1.5ex}
\setlength{\ExBeg}{-\dPar}
\addtolength{\ExBeg}{-1.5ex}
\setlength{\ExEnd}{-\dPar}
\addtolength{\ExEnd}{-1.0ex}

\newenvironment{example}
  {\vspace{\ExBeg} \begin{alltt}}
  {\end{alltt} \vspace{\ExEnd}}

\newcommand\Strut{\rule[-2ex]{0mm}{6ex}}

\newenvironment{Example}{\vspace{-1.5ex} \begin{alltt}}{\end{alltt}}

\newenvironment{Itemize}
  {\begin{list}{$\bullet$}
    {\addtolength{\topsep}{-1.5ex} 
     \addtolength{\itemsep}{-1ex}
    }
  }
  {\end{list} \vspace*{1ex}}

\newcommand{\Section}[1]{\section{#1}\indent\vspace{-3ex}}

\newcommand{\SECTION}[1]{\section*{#1}\indent\vspace{-3ex}}

% From pg 64 of The LaTex Companion.

\newenvironment{ventry}[1]
  {\begin{list}{}
    {\renewcommand{\makelabel}[1]{\textsf{##1}\hfil}
     \settowidth{\labelwidth}{\textsf{#1}}
     \addtolength{\itemsep}{-1.5ex}
     \addtolength{\topsep}{-1.0ex} 
     \setlength{\leftmargin}{5em}
    }
  }
  {\end{list}}
