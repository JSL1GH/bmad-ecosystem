\documentclass[11pt]{article}
\usepackage{moreverb}    % Defines {listing} environment.
\usepackage{tocloft}
\usepackage{geometry}            % See geometry.pdf to learn the layout options. There are lots.
\usepackage{xspace}
\geometry{letterpaper}           % ... or a4paper or a5paper or ... 
%\usepackage[parfill]{parskip}   % To begin paragraphs with an empty line rather than an indent
\usepackage{graphicx}
\usepackage{amssymb}
\usepackage{amsmath}
\usepackage{alltt}
\usepackage[T1]{fontenc}   % so _, <, and > print correctly in text.
\usepackage[strings]{underscore}    % to use "_" in text
\usepackage[pdftex,colorlinks=true]{hyperref}

\newcommand{\fig}[1]{Figure~\ref{#1}}
\newcommand{\lux}{Lux\xspace}
\newcommand{\chess}{CHESS\xspace}
\newcommand{\Begineq}{\begin{equation}}
\newcommand{\Endeq}{\end{equation}}
\newcommand{\CRNO}{\nonumber \\}

\newcommand\ttcmd{\begingroup\catcode`\_=11 \catcode`\%=11 \dottcmd}
\newcommand\dottcmd[1]{\texttt{#1}\endgroup}
\newcommand{\vn}{\ttcmd}   

\newcommand{\Newline}{\hfil \\}
\newcommand{\sref}[1]{$\S$\ref{#1}}

\newenvironment{example}
  {\vspace{\ExBeg} \begin{alltt}}
  {\end{alltt} \vspace{\ExEnd}}

%---------------------------------------------------------------------------------

\newlength{\dPar}
\newlength{\ExBeg}
\newlength{\ExEnd}
\setlength{\dPar}{1.5ex}
\setlength{\ExBeg}{-\dPar}
\addtolength{\ExBeg}{-0.5ex}
\setlength{\ExEnd}{-\dPar}
\addtolength{\ExEnd}{-0.0ex}

\setlength{\textwidth}{6.25in}
\setlength{\hoffset}{0.0in}
\setlength{\oddsidemargin}{0.25in}
\setlength{\evensidemargin}{0.0in}
\setlength{\textheight}{8.5in}
\setlength{\topmargin}{0in}

\setlength{\parskip}{\dPar}
\setlength{\parindent}{0ex}

\setlength\cftparskip{0pt}
\setlength\cftbeforesecskip{3pt}
\setlength\cftaftertoctitleskip{15pt}

%---------------------------------------------------------------------------------

\title{Lux Program}
\author{D. Sagan, K. Finkelstein}
\date{Revision 1.1 \\ August 1, 2013}

%---------------------------------------------------------------------------------

\begin{document}
\maketitle

\pdfbookmark[1]{Contents}{Contents}
\tableofcontents

%------------------------------------------------------------------
\section{Introduction} 
\label{s:intro}

\lux is a program for Monte Carlo simulations of X-rays using either
coherent or incoherent ray tracing. \lux works by generating
Photons at a source element and tracking them through to the
detector. \lux uses the Bmad subroutine library\cite{b:bmad} for
tracking and lattice bookkeeping.

%------------------------------------------------------------------
\section{Running \lux} 
\label{s:run}

Syntax for invoking \lux:
\begin{example}
  lux \{options\} \{<master_input_file_name>\}
\end{example}
The \vn{<master_input_file_name>} optional argument is used to set the
master input file name. The default is ``\vn{lux.init}''. 

Currently there are no \{options\}.

%------------------------------------------------------------------
\section{Fortran Namelist}
\label{s:namelist}

Fortran namelist syntax is used for \lux. The general form
of a namelist is
\begin{example}
  &<namelist_name>
    <var1> = ...
    <var2> = ...
    ...
  /
\end{example}
The tag \vn{"\&<namelist_name>"} starts the namelist where
\vn{<namelist_name>} is the name of the namelist. The namelist ends
with the slash \vn{"/"} tag. Anything outside of this is
ignored. Within the namelist, anything after an exclamation mark
\vn{"!"} is ignored including the exclamation mark. \vn{<var1>},
\vn{<var2>}, etc. are variable names. Example:
\begin{example}
  &section_def section =   0.0, "arc_std", "elliptical", 0.045, 0.025 /
\end{example}
here \vn{section_def} is the namelist name and \vn{section} is a variable
name.  Notice that here \vn{section} is a ``structure'' which has five
components.

%------------------------------------------------------------------
\section{Master Input File} 
\label{s:master.file}

The master input file consists of a single \vn{namelist} called \vn{params}.
An example is:
\begin{example}
  &params
    lattice_file = 'test.bmad'                     ! Defines experimental layout.

    photon_init%photon_type = 'INCOHERENT'           ! Or 'COHERENT'
    photon_init%energy_distribution = 'GAUSSIAN'     ! Or 'UNIFORM'. 
    photon_init%spatial_distribution = 'GAUSSIAN'    ! Or 'UNIFORM'. 
    photon_init%velocity_distribution = 'GAUSSIAN'   ! Or 'SHPERICAL', 'UNIFORM'. 
    photon_init%sig_x = 0
    photon_init%sig_y = 0
    photon_init%sig_z = 0
    photon_init%sig_vx = 0
    photon_init%sig_vy = 0
    photon_init%sig_E = 0                            ! Initial photon energy width (eV).
    photon_init%dE_center = 0                        ! Initial photon energy shift (eV).
    photon_init%dE_relative_to_ref = True            ! dE_center relative to reference E?
    photon_init%emit_x = 0                          
    photon_init%emit_Y = 0 
    photon_init%e_field_x = 0                        ! Polarization. x & y = 0 -> random
    photon_init%e_field_y = 0
    photon_init%normalize_field = False

    random_seed = 0                                ! 0 => Use system clock
    intensity_normalization_coef = 1e6             ! photon intensities norm.

    lux_param%simulation_type = 'ROCKING_CURVE'    ! Or 'NORMAL'
    lux_param%source_element = ''                  ! Name of source element

    lux_param%stop_total_intensity = 10            ! Cumulative intensity to stop at.
    lux_param%stop_num_photons = 0                 ! Max number of photons to track.

    det_pix_out_file = 'lux.det_pix#'
    lux_param%intensity_min_det_pixel_cutoff = 0.1 ! det_pix_out file Cutoff

    photon1_out_file = 'lux.photon1'               ! Individual photon positions.
    ix_ele_photon1_file = -1
    lux_param%intensity_min_photon1_cutoff = 1e-3  ! photon1_out file cutoff
    reject_dead_at_det_photon1 = False
  /
\end{example}

The components of this namelist are:
  \begin{description}
  \item[\vn{lattice_file}] \Newline
This file defines the X-ray optics setup. See
Section~\sref{s:lat.file} and the Bmad manual for more details.

  \item[\vn{lux_param\%simulation_type}] \Newline
Type of simulation. Possible settings are:
\begin{example}
  NORMAL
  ROCKING_CURVE      
\end{example}
The \vn{ROCKING_CURVE} setting indicates that a rocking curve simulation
will be done.
The \vn{NORMAL} setting indicates that the usual photon tracking is to be done.

  \item[\vn{lux_param\%source_element}] \Newline
Used with \vn{lux_param%simulation_type} = 'NORMAL'. Name of the lattice element
that is the source of the photons.

  \item[\vn{photon_init\%photon_type}] \Newline
Whether coherent or incoherent photons are tracked. 
Possible settings are:
\begin{example}
  COHERENT
  INCOHERENT
\end{example}
The \vn{INCOHERENT} setting should only be used when ``long-range''
phase correlations are important such as when there is a
\vn{diffraction_plate} element.

  \item[\vn{photon_init\%energy_spectrum}] \Newline
Sets the type of energy spectrum for emitted photons
(\sref{s:photon.start}). \break
Used with \vn{lux_param%simulation_type} =
'X_RAY_INIT'.  Possible settings are:
\begin{example}
  GAUSSIAN
  UNIFORM
\end{example}
The \vn{'GAUSSIAN'} setting gives a Gaussian distribution with width
set by \vn{photon_init%sig_E}. The \vn{'UNIFORM'} setting gives a flat
distribution in the range
\begin{example}
  [-photon_init%sig_E, photon_init%sig_E]
\end{example}

  \item[\vn{photon_init\%spatial_distribution}] \Newline
Sets spacial $(x, y, z)$ spectrum of emitted photons
(\sref{s:photon.start}).  Used with \vn{lux_param%simulation_type} =
'X_RAY_INIT'.  Possible settings are:
\begin{example}
  GAUSSIAN
  UNIFORM
\end{example}
The \vn{'GAUSSIAN'} setting gives a Gaussian distribution with width
set by \vn{photon_init%sig_E}. The \vn{'UNIFORM'} setting gives a flat
distribution in the range
\begin{example}
  [-photon_init%sig_E, photon_init%sig_E]
\end{example}
The \vn{'GAUSSIAN'} setting gives a Gaussian distribution with width
$\sigma$ where $sigma$ is 
\begin{example}
  photon_init%sig_x     for x distribution
  photon_init%sig_y     for y distribution
  photon_init%sig_z     for z distribution
\end{example}
The \vn{'UNIFORM'} setting gives a flat
distribution in the range $[-\sigma, \sigma]$.

  \item[\vn{photon_init\%velocity_distribution}] \Newline
Sets the transverse $(v_x/c, v_y/c)$ velocity spectrum of emitted
photons. \break
Used with \vn{lux_param%simulation_type} = 'X_RAY_INIT'. The
longitudinal velocity is always computed to make $v_x^2 + v_y^2 +
v_z^2 = c^2$ Possible settings are:
\begin{example}
  GAUSSIAN
  SPHERICAL
  UNIFORM
\end{example}
The \vn{'GAUSSIAN'} setting gives a Gaussian distribution with width
$\sigma$ where $sigma$ is 
\begin{example}
  photon_init%sig_vx     for vx/c distribution
  photon_init%sig_vy     for vy/c distribution
\end{example}
The \vn{'UNIFORM'} setting gives a flat distribution in the range
$[-\sigma, \sigma]$.
The \vn{SPHERICAL} setting gives flat distribution in all directions.

  \item[\vn{photon_init\%dE_center}] \Newline
Average initial photon energy in eV. If
\vn{photon_init%dE_relative_to_ref} is set to True (default)
then \vn{photon_init%dE_center} will be relative to the reference energy
(\sref{s:photon.start}).

  \item[\vn{photon_init\%sig_x, photon_init\%sig_y, photon_init\%sig_z}] \Newline
Width of emitted photons in $x$, $y$ and $z$ directions. See
\vn{photon_init%spatial_distribution} for more details.

  \item[\vn{photon_init\%sig_vx, photon_init\%sig_vy}] \Newline
Width of emitted photons in $v_x/c$ and $v_y/c$ directions. See
\vn{photon_init%velocity_distribution} for more details.

  \item[\vn{photon_init\%sig_E}] \Newline
Energy width in eV. See \vn{photon_init%energy_distribution} for more
details.

  \item[\vn{photon_init\%dE_relative_to_ref}] \Newline
Is \vn{photon_init%dE_center} with respect to the reference energy or
an absolute value (\sref{s:photon.start})?

  \item[\vn{photon_init\%emit_x, photon_init\%emit_y}] \Newline
$x$ and $y$ charged particle beam emittances. 
Used when the photon source is a bend, wiggler, or undulator.

  \item[\vn{photon_init\%e_field_x}, \vn{photon_init\%e_field_y}] \Newline
Electric field component of initial photons in the $x$ and $y$ planes
(\sref{s:photon.start}).  If both are set to 0 then a random field is
chosen with unit intensity $E_x^2 + E_y^2 = 1$.

  \item[\vn{photon_init\%normalize_field}] \Newline
Scale the field so that $E_x^2 + E_y^2 = 1$ Only used if at least
one field component is non-zero.


  \item[\vn{random_seed}] \Newline
Random number seed used in by the random number generator. If set to
0, the system clock will be used. That is, if set to 0, the output
results will vary from run to run.

  \item[\vn{intensity_normalization_coef}] \Newline
Coefficient used to normalize the photon intensity with (\sref{s:det}).

  \item[\vn{photon_init\%stop_total_intensity}] \Newline
Cumulative unnormalized intensity at the detector to stop at. (\sref{s:det}).

  \item[\vn{photon_init\%stop_num_photons}] \Newline
Maximum number of photons to track (\sref{s:det}).

  \item[\vn{det_pix_out_file}] \Newline
Name of the output data file recording the X-ray intensity on the
detector. See section~\sref{s:det.pix.file} for more details. The file
name may be ``numbered'' using a ``\#'' character (\sref{s:number}). A
blank file name will prevent a file being generated.

  \item[\vn{lux_param\%intensity_min_det_pixel_cutoff}] \Newline
Minimum intensity, relative to the maximum pixel intensity, below
which a pixel will not be included in the det_pix_out file (\sref{s:det.pix.file}).

  \item[\vn{photon1_out_file}] \Newline
Name of the output data file recording the starting and ending
positions of individual photons. See section~\sref{s:photon1.file} for
more details. The file name may be ``numbered'' using 
a ``\#'' character (\sref{s:number}). A blank file name will prevent a
file being generated.

  \item[\vn{ix_ele_photon1}] \Newline
The photon position recorded in the photon1_out file will be the
photon position at the lattice element whose lattice index is given by
\vn{ix_ele_photon1}. If \vn{ix_ele_photon1} is less than 1 then the
lattice element used will be the detector element (\sref{s:photon1.file}).

  \item[\vn{photon_init\%intensity_min_photon1_cutoff}] \Newline
Cutoff intensity below which a photon will not be included in
the photon1_out file (\sref{s:photon1.file}).

  \item[\vn{reject_dead_at_det_photon1}] \Newline
If \vn{reject_dead_to_det_photon1} is set to True, do not print to the
photon_out file any photons that do not reach the detector. This switch
is only relavent when the lattice element at which the photon positions
are being evaluated is not the detector (\sref{s:photon1.file}).

  \end{description}

%------------------------------------------------------------------
\section{Lattice Description File}
\label{s:lat.file}

The name of the lattice description file which defines the
experimental setup is given by the \vn{lattice_file} variable in the
master input file (\sref{s:master.file}). Bmad standard syntax is
used\cite{b:bmad}. 

%------------------------------------------------------------------
\subsection{Example Lattice Using an X_RAY_INIT Source}
\label{s:spherical.lat}

An example lattice using an \vn{x_ray_init} element as the photon source is:
\begin{listing}{1}
beginning[e_tot] = 1e4
parameter[particle] = photon
parameter[no_end_marker] = T

c_rad = 75e-3  ! Crystal transverse radius 
r0 = 0.2
angle  = 80 * pi / 180
qq = r0 * tan(pi/2-graze_angle)
dft_len = sqrt(qq^2 + r0^2)

source: x_ray_init, l = 0, x_offset = 0, y_offset = 0, z_offset = 0, tilt = 0,
        x_pitch = 0, y_pitch = 0
drift1: pipe, l = dft_len, x_limit = 0.01, y_limit = 0.01
cryst: crystal, crystal_type = 'Si(444)', b_param = -1, tilt = pi, 
        a2_trans_curve = 1 / (2 * c_rad), a4_trans_curve = 1 / (8 * c_rad^3)
drift2: drift, l = dft_len
det: marker, x_limit = 0.01, y_limit = 0.01

lux_line: line = (source, drift1, cryst, drift2, det)
use, lux_line
\end{listing}

The list of lattice elements is given in line 20.  This lattice
constructs of five elements: The source, a crystal, and a detector
with two drift spaces in between. The source element must be the first
element of the lattice. The detector element must be the last
line. Line 3 prevents a final end marker element from being inserted
into the lattice and being mistaken for the detector element.

The definitions of the lattice elements is given in lines 11 through
17.  The reference photon energy is specified in line 1 as 10~KeV and
line 2 sets photons as the reference particle.

%------------------------------------------------------------------
\section{Photon Description}
\label{s:photon.descrip}

Photons are described by:
\begin{example}
  (x, vx, y, vy, z, vz)   ! six dimensional phase space vector 
  E                       ! Photon energy 
  e_field_x, e_field_y    ! Electric field vector
  phase_x, phase_y        ! Electric field phase 
\end{example}
$(x, y, z)$ is the spatial position of the photon with $z$ being the
longitudinal coordinate and $z=0$ corresponds to the beginning of the
element the photon is passing though (Thus $z$ is generally not
interesting). $(v_x, v_y, v_z)$ is the photon velocity normalized to 1.
Generally photons that make it to the detector will have $v_z$ close to
1 and $v_x$ and $v_y$ relatively ``small''.

%------------------------------------------------------------------
\subsection{'SPHERICAL' Source Type}

Photons are generated from the first element of the lattice. 
Example source definition in the lattice input file:
\begin{example}
  source: instrument, l = 0, x_offset = 0, y_offset = 0, z_offset = 0,
        x_pitch = 0, y_pitch = 0, tilt = 0, x_half_length = 0, y_half_length = 0
\end{example}
The physical extent of the source is given by the parameters
\begin{example}
  l              ! Longitudinal length (\(2 \sigma\))
  x_half_length  ! Half length in x-direction (\(1 \sigma\))
  y_half_length  ! Half length in the y-direction (\(1 \sigma\)).
\end{example}
With Gaussian spatial distributions, \vn{l} is the $2\sigma$
longitudinal length of the source and \vn{x_half_length} and
\vn{y_half_length} are the $1\sigma$ transverse extents.

The source can be moved spatially by setting the parameters
\begin{example}
  x_offset, y_offset, z_offset
  x_pitch, y_pitch, tilt
\end{example}
See the Bmad manual for more details.

The emission spectrum is governed by the master input file parameters:
\begin{example}
  photon_init%sig_E
  photon_init%dE_center
  photon_init%dE_relative_to_ref
  photon_init%energy_distribution
  photon_init%spatial_distribution     ! (x, y) spacial distribution
  photon_init%velocity_distribution     ! (x, y) spacial distribution
\end{example}
If \vn{photon_init%dE_relative_to_ref} is set to True, The average
starting photon energy is
\begin{example}
  E_average = photon_init%dE_center + beginning[e_tot]
\end{example}
where the reference energy \vn{beginning[e_tot]} is set in the lattice file. 
If \vn{photon_init%dE_relative_to_ref} is set to False, The average
starting photon energy is simply \vn{photon_init%dE_center}.

The width of the energy distribution, in eV, is set by \vn{%sig_E}.
The \vn{%energy_spectrum} and \vn{%transverse_distribution} parameters
establish the type of energy and spatial spectrum used. Possible
settings for these parameters are:
\begin{example}
  "UNIFORM"
  "GAUSSIAN"
\end{example}
A \vn{UNIFORM} spectrum has a constant probability of emission out to
1~sigma and zero outside of this range.

The simulation assumes that in the actual experiment that photons will
be generated uniformly in all directions. Most of these direction will
be ``uninteresting''. An uninteresting direction is a direction in
which a photon has no chance of making its way to the detector. To
reduce computation time, \lux tries to only generate photons in
interesting directions. The prescription for doing this is as follows.

%------------------------------------------------------------------
\subsection{'INSERTION_DEVICE' Source Type}

The \vn{'INSERTION_DEVICE'} source type simulates readiation coming from dipole
bending magnets, wigglers, or undulators. In this case, photons are
generated using the first element in the lattice. There must be a
\vn{photon_branch} element in the lattice that branches to the x-ray
line. 

%------------------------------------------------------------------
\subsection{Polarization Init}

The polarization of the photons is set in the master input file by
\begin{example}
  photon_init%e_field_x     ! Polarization along the x-axis
  photon_init%e_field_y     ! polarization along the y-axis
\end{example}
\lux will scale the field of each photon to 1 at the start of
tracking. If both field components are set to zero, random
polarization will be set. The intensity of a photon is defined as the
square of the field so the initial photon intensity is one. As photons
travel, they can loose intensity via, for example, diffraction from a
crystal. This photon intensity is called the ``unnormalized'' photon
intensity. [See \sref{s:det} for an explanation of ``normalized'' intensity.]

%------------------------------------------------------------------
\section{Photon Detection}
\label{s:det}

Photons are tracked until they are lost (hit an aperture) or until
they get to the detector which is the last element in the lattice.

Two parameters in the master input file determine when the simulation
is stopped:
\begin{example}
    lux_param%stop_total_intensity
    lux_param%stop_num_photons
\end{example}
If \vn{lux_param%stop_total_intensity} is positive, the simulation
ends when the total accumulated unnormalized intensity at the detector
passes this threshold. Intensity is the square of the photon electric
field and the unnormalized photon intensity will vary from zero to one for
each photon.

If \vn{lux_param%stop_num_photons} is positive, the simulation is
stopped when the number of photons generated passes this threshold. If
both stop parameters are positive, the simulation is stopped when either
the intensity or the number of photons passes the set threshold.

To properly normalize the intensity at the detector in the output
files, a normalization factor $f_n$ is applied.
\Begineq
  I (\mbox{normalized}) = f_n I (\mbox{unnormalized}) 
\Endeq
$f_n$ is constructed so that the normalized intensity will be
independent of the number of simulation photons and independent of the
tiling parameters. Explicitly:
\Begineq
  f_n = \frac{C_f \, A_t}{4 \, \pi \, N_p}
\Endeq
where $A_t$ is the total area of the live tiles, $N_p$ is the number
of photons tracked, and $C_f$ is a normalization coefficient set by
the \vn{intensity_normalization_coef} parameter in the master input
file, $C_f$ can be used, for example, to scale the output numbers to
correspond to actual measured values.

%------------------------------------------------------------------
\section{Numbering the Output Data Files}
\label{s:number}

The names of the output data files are specified by the following
variables (\sref{s:master.file}):
\begin{example}
  photon1_out_file
  det_pix_out_file
\end{example}

If an output data file name contains a pound character ``\#'', then a
number will substituted for the pound character and the number
substituted will be increased by one each time \lux is run. For example,
if \vn{det_pix_out_file} is defined by
\begin{example}
  det_pix_out_file = 'lux.det_pix#'
\end{example}
Then the first time \lux is run, the pixel data file will be named
``lux.det_pix1''.  The next time \lux is run, the data file will be
named ``lux.det_pix2'', etc.

Using a numbering system prevents data files from being
overwritten. If more than one output file name has a pound character,
all such files will receive the same number on a given run. To set the
number given, one can edit the file:
\begin{example}
  lux_out_file.number
\end{example}

%------------------------------------------------------------------
\section{Photon1 Data File}
\label{s:photon1.file}

The \vn{photon1_out_file} parameter (set in the master input file) is
the name of an output data file containing beginning and ``end''
coordinates of a set of photons.

The location where the ``end'' coordinates are evaluated is set by the
\vn{ix_ele_photon} parameter from the master input file. If
\vn{ix_ele_photon} is less than 1 then the end coordinates are
evalueated at the detector element. If \vn{ix_ele_photon} is 1 or more
then the end coordinates are evaluated at the exit end of the lattice
element with index \vn{ix_ele_photon}. 

If \vn{reject_dead_at_det_photon1} is set to True, photons that do not
make it to the detector will not be included in the photon1_out
file. In any case, a photon that does not reach the \vn{ix_ele_photon}
element will not be included. Thus, if the lattice element where the
photon position is being evaluated at is the detector, it will always
be the case that photons that do not make it to the detector will not
be included in the photon1_out file irregardless of the setting of
\vn{reject_dead_at_det_photon1}.

If the file name is blank then no file will be generated. The file
name may be ``numbered'' using a ``\#'' character (\sref{s:number}).

If a photon at the detector has an intensity of less than
\vn{lux_param%intensity_min_photon1_cutoff} then the photon is not
counted.

Each row in the file is a single photon. The columns in the file are:
\begin{example}
  n_live           ! Index of this photon.
  beginning_orbit  ! Five columns: (x, vx, y, vy, z) at the source
  orbit_at_ix_ele  ! Five columns: (x, vx, y, vy, z) at the ``end''
  energy           ! Photon energy (eV).
  intensity_x      ! x-axis unnormalized photon intensity
  intensity_y      ! y-axis unnormalized photon intensity
\end{example}

%------------------------------------------------------------------
\section{Detector Pixel Data Files}
\label{s:det.pix.file}

The \vn{det_pix_out_file} parameter (set in the master input file) is
the name of an output data file containing pixel-by-pixel information.

If the file name is blank then no file will be generated. The file
name may be ``numbered'' using a ``\#'' character (\sref{s:number}).

The intensity of a pixel is the accumulated intensity of the photons
hitting the pixel. Only pixels whose intensity is greater than
\vn{lux_param%intensity_min_det_pixel_cutoff} are recorded in the
output file. This cutoff is relative to the intensity of the pixel
with the highest intensity.
\vn{lux_param%intensity_min_det_pixel_cutoff} is useful for keeping
the data file sizes small.

Each row in this file represents a single pixel. The columns in this
file are:
\begin{example}
  ix_pix            ! $x$-axis index
  iy_pix            ! $y$-axis index
  x_pix             ! $x$ coordinate of pixel center
  y_pix             ! $y$ coordinate of pixel center
  intensity         ! Normalized pixel intensity
  n_photon          ! Number of photons hitting pixel
  E_ave             ! Average energy deviation from reference (eV)
  E_rms             ! RMS about the average energy (eV)
\end{example}
\vn{E_ave} is an intensity weighted average photon energy deviation
from the reference energy at a pixel. \vn{E_rms} is the RMS variation
of the photon energy with respect to \vn{E_ave}.

In addition to the pixel-by-pixel data file, two additional files will
be generated containing data projected on the $x$ and $y$ detector
alux. The name of these files will be the same as
\vn{det_pix_out_file} with ``.x'' and ``.y'' suffilux indicating
projected data on the $x$ and $y$ alux respectively. The columns for
the $x$-axis projected data is
\begin{example}
  ix_pix            ! x-axis index
  x_pix             ! x coordinate of pixel center
  intensity         ! Normalized pixel intensity sum
  n_photon          ! Number of photons hitting pixels
  E_ave             ! Average energy deviation from reference (eV)
  E_rms             ! RMS about the average energy (eV)
\end{example}
With similar columns for the $y$-axis file. The columns for the
$x$-axis projected file records the sum of the intensities for all
pixels with a common \vn{y_pix}. Since this can include pixels not
recorded in the pixel-by-pixel file (due to a non-zero
\vn{lux_param%intensity_min_det_pixel_cutoff} setting), quantities
like the total number of photon need not match between the
pixel-by-pixel file and the projected files.

%------------------------------------------------------------------
\section{Data Visualization}
\label{s:vis}

There is a python script for making an intensity plot of the
\vn{det_pix_out_file} in:
\begin{example}
  lux/plot/det_pix_plot.py
\end{example}
For help run the script
\begin{example}
  <path_to_script>/det_pix_plot.py -help
\end{example}

%------------------------------------------------------------------
\begin{thebibliography}{9}

\bibitem{b:bmad}
D. Sagan,
"Bmad: A Relativistic Charged Particle Simulation Library"
Nuc.\ Instrum.\ \& Methods Phys.\ Res.\ A, {\bf 558}, pp 356-59 (2006).

\end{thebibliography}
\end{document}  
