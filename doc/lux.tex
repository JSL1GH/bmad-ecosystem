\documentclass[11pt]{article}
\usepackage{moreverb}    % Defines {listing} environment.
\usepackage{tocloft}
\usepackage{geometry}            % See geometry.pdf to learn the layout options. There are lots.
\usepackage{xspace}
\geometry{letterpaper}           % ... or a4paper or a5paper or ... 
%\usepackage[parfill]{parskip}   % To begin paragraphs with an empty line rather than an indent
\usepackage{graphicx}
\usepackage{amssymb}
\usepackage{amsmath}
\usepackage{alltt}
\usepackage[T1]{fontenc}   % so _, <, and > print correctly in text.
\usepackage[strings]{underscore}    % to use "_" in text
\usepackage[pdftex,colorlinks=true]{hyperref}

\newcommand{\fig}[1]{Figure~\ref{#1}}
\newcommand{\lux}{Lux\xspace}
\newcommand{\chess}{CHESS\xspace}
\newcommand{\Begineq}{\begin{equation}}
\newcommand{\Endeq}{\end{equation}}
\newcommand{\CRNO}{\nonumber \\}

\newcommand\ttcmd{\begingroup\catcode`\_=11 \catcode`\%=11 \dottcmd}
\newcommand\dottcmd[1]{\texttt{#1}\endgroup}
\newcommand{\vn}{\ttcmd}   

\newcommand{\Newline}{\hfil \\}
\newcommand{\sref}[1]{$\S$\ref{#1}}

\newenvironment{example}
  {\vspace{\ExBeg} \begin{alltt}}
  {\end{alltt} \vspace{\ExEnd}}

%---------------------------------------------------------------------------------

\newlength{\dPar}
\newlength{\ExBeg}
\newlength{\ExEnd}
\setlength{\dPar}{1.5ex}
\setlength{\ExBeg}{-\dPar}
\addtolength{\ExBeg}{-0.5ex}
\setlength{\ExEnd}{-\dPar}
\addtolength{\ExEnd}{-0.0ex}

\setlength{\textwidth}{6.25in}
\setlength{\hoffset}{0.0in}
\setlength{\oddsidemargin}{0.25in}
\setlength{\evensidemargin}{0.0in}
\setlength{\textheight}{8.5in}
\setlength{\topmargin}{0in}

\setlength{\parskip}{\dPar}
\setlength{\parindent}{0ex}

\setlength\cftparskip{0pt}
\setlength\cftbeforesecskip{3pt}
\setlength\cftaftertoctitleskip{15pt}

%---------------------------------------------------------------------------------

\title{Lux Program}
\author{D. Sagan, K. Finkelstein}
\date{Revision 1.1 \\ August 1, 2013}

%---------------------------------------------------------------------------------

\begin{document}
\maketitle

\pdfbookmark[1]{Contents}{Contents}
\tableofcontents

%------------------------------------------------------------------
\section{Introduction} 
\label{s:intro}

\lux is a program for Monte Carlo simulations of X-rays using either
coherent or incoherent ray tracing. \lux works by generating
Photons at a source element and tracking them through to the
detector. \lux uses the Bmad subroutine library\cite{b:bmad} for
tracking and lattice bookkeeping.

%------------------------------------------------------------------
\section{Running \lux} 
\label{s:run}

Syntax for invoking \lux:
\begin{example}
  lux \{options\} \{<master_input_file_name>\}
\end{example}
The \vn{<master_input_file_name>} optional argument is used to set the
master input file name. The default is ``\vn{lux.init}''. 

Currently there are no \{options\}.

%------------------------------------------------------------------
\section{Fortran Namelist}
\label{s:namelist}

Fortran namelist syntax is used for \lux. The general form
of a namelist is
\begin{example}
  &<namelist_name>
    <var1> = ...
    <var2> = ...
    ...
  /
\end{example}
The tag \vn{"\&<namelist_name>"} starts the namelist where
\vn{<namelist_name>} is the name of the namelist. The namelist ends
with the slash \vn{"/"} tag. Anything outside of this is
ignored. Within the namelist, anything after an exclamation mark
\vn{"!"} is ignored including the exclamation mark. \vn{<var1>},
\vn{<var2>}, etc. are variable names. Example:
\begin{example}
  &section_def section =   0.0, "arc_std", "elliptical", 0.045, 0.025 /
\end{example}
here \vn{section_def} is the namelist name and \vn{section} is a variable
name.  Notice that here \vn{section} is a ``structure'' which has five
components.

%------------------------------------------------------------------
\section{Master Input File} 
\label{s:master.file}

The master input file consists of a single \vn{namelist} called \vn{params}.
An example is:
\begin{example}
  &params
  	lattice_file = 'test.bmad'                     ! Defines experimental layout.
    det_pix_out_file = 'lux.det_pix#'

    photon1_out_file = 'lux.photon1'               ! Individual photon positions.
    lux_param%intensity_min_photon1_cutoff = 1e-3  ! photon1_out file cutoff
    ix_ele_photon1 = -1                            ! Index of element for photon1_out file.
    reject_dead_at_det_photon1 = False             ! Reject photons dead at detector.

    random_seed = 0                                ! 0 => Use system clock
    intensity_normalization_coef = 1e6             ! photon intensities norm.

    lux_param%sig_E = 0.000                        ! Init energy width (eV).
    lux_param%dE_center = 0                        ! Init energy shift (eV).
    lux_param%dE_relative_to_ref = True            ! dE_center is Relative or absolute?
    lux_param%energy_spectrum = 'GAUSSIAN'         ! or 'UNIFORM'.
    lux_param%transverse_distribution = 'GAUSSIAN' ! or 'UNIFORM'. 

    lux_param%source_type = 'SPHERICAL'            ! or 'PLANAR' or 'BEND'
    lux_param%use_tile = False
    lux_param%y_max = 0.1                          ! total tile height.
    lux_param%phi_max = 0.1                        ! total tile width.

    lux_param%e_field_x = 0                        ! Polarization. 
    lux_param%e_field_y = 0                        ! x & y = 0 -> random

    lux_param%stop_total_intensity = 10            ! Cumulative intensity to stop at.
    lux_param%stop_num_photons = 0                 ! Max number of photons to track.

    lux_param%dx_det_pixel = 50e-6                 ! Detector pixel width
    lux_param%dy_det_pixel = 50e-6                 ! Detector pixel height
    lux_param%intensity_min_det_pixel_cutoff = 0.1 ! det_pix_out file Cutoff
  /
\end{example}

The components of this namelist are:
  \begin{description}
  \item[\vn{lattice_file}] \Newline
This file defines the X-ray optics setup. See
Section~\sref{s:lat.file} and the Bmad manual for more details.
  \item[\vn{photon1_out_file}] \Newline
Name of the output data file recording the starting and ending
positions of individual photons. See section~\sref{s:photon1.file} for
more details. The file name may be ``numbered'' using 
a ``\#'' character (\sref{s:number}). A blank file name will prevent a
file being generated.
  \item[\vn{det_pix_out_file}] \Newline
Name of the output data file recording the X-ray intensity on the
detector. See section~\sref{s:det.pix.file} for more details. The file
name may be ``numbered'' using a ``\#'' character (\sref{s:number}). A
blank file name will prevent a file being generated.
  \item[\vn{random_seed}] \Newline
Random number seed used in by the random number generator. If set to
0, the system clock will be used. That is, if set to 0, the output
results will vary from run to run.
  \item[\vn{intensity_normalization_coef}] \Newline
Coefficient used to normalize the photon intensity with (\sref{s:det}).
  \item[\vn{lux_param\%sig_E}] \Newline
Energy width of initial photon distribution in eV (\sref{s:photon.start}).
  \item[\vn{lux_param\%dE_center}] \Newline
Average initial photon energy in eV. If
\vn{lux_param%dE_relative_to_ref} is set to True (default)
then \vn{lux_param%dE_center} will be relative to the reference energy
(\sref{s:photon.start}).
  \item[\vn{lux_param\%dE_relative_to_ref}] \Newline
Is \vn{lux_param%dE_center} with respect to the reference energy or
an absolute value (\sref{s:photon.start})?
  \item[\vn{lux_param\%energy_spectrum}] \Newline
Type of energy spectrum (\sref{s:photon.start}). 
  \item[\vn{lux_param\%transverse_distribution}] \Newline
Type of spacial spectrum (\sref{s:photon.start}).
  \item[\vn{lux_param\%source_type}] \Newline
Type of source used for photon generation (\sref{s:photon.start}).
  \item[\vn{lux_param\%use_tile}] \Newline
Use all direction tiles within the rectangle defined by \vn{lux_param%del_y}
and \vn{lux_param%del_phi} (\sref{s:photon.start})?
  \item[\vn{lux_param\%y_max}] \Newline
Height extent of the direction tile (\sref{s:photon.start}).
  \item[\vn{lux_param\%phi_max}] \Newline
Width extent of the direction tile  (\sref{s:photon.start}).
  \item[\vn{lux_param\%e_field_x}] \Newline
Electric field component of initial photons in the x-plane (\sref{s:photon.start}).
  \item[\vn{lux_param\%e_field_y}] \Newline
Electric field component of initial photons in the y-plane (\sref{s:photon.start}).
  \item[\vn{lux_param\%stop_total_intensity}] \Newline
Cumulative unnormalized intensity at the detector to stop at. (\sref{s:det}).
  \item[\vn{lux_param\%stop_num_photons}] \Newline
Maximum number of photons to track (\sref{s:det}).
  \item[\vn{lux_param\%dx_det_pixel}] \Newline
Detector pixel width (\sref{s:det}).
  \item[\vn{lux_param\%dy_det_pixel}] \Newline
Detector pixel height (\sref{s:det}).
  \item[\vn{lux_param\%intensity_min_photon1_cutoff}] \Newline
Cutoff intensity below which a photon will not be included in
the photon1_out file (\sref{s:photon1.file}).
  \item[\vn{ix_ele_photon1}] \Newline
The photon position recorded in the photon1_out file will be the
photon position at the lattice element whose lattice index is given by
\vn{ix_ele_photon1}. If \vn{ix_ele_photon1} is less than 1 then the
lattice element used will be the detector element (\sref{s:photon1.file}).
  \item[\vn{reject_dead_at_det_photon1}] \Newline
If \vn{reject_dead_to_det_photon1} is set to True, do not print to the
photon_out file any photons that do not reach the detector. This switch
is only relavent when the lattice element at which the photon positions
are being evaluated is not the detector (\sref{s:photon1.file}).
  \item[\vn{lux_param\%intensity_min_det_pixel_cutoff}] \Newline
Minimum intensity, relative to the maximum pixel intensity, below
which a pixel will not be included in the det_pix_out file (\sref{s:det.pix.file}).
  \end{description}

%------------------------------------------------------------------
\section{Lattice Description File}
\label{s:lat.file}

The name of the lattice description file which defines the
experimental setup is given by the \vn{lattice_file} variable in the
master input file (\sref{s:master.file}). Bmad standard syntax is
used\cite{b:bmad}. 

%------------------------------------------------------------------
\subsection{Example Lattice File for SPHERICAL or PLANAR Sources}
\label{s:spherical.lat}

An example lattice is valid for \vn{'SPHERICAL'} or
\vn{'PLANAR'} type sources (cf.~\sref{s:photon.start}):
\begin{listing}{1}
beginning[beta_a] = 1
beginning[beta_b] = 1
beginning[e_tot] = 1e4
parameter[particle] = photon
parameter[no_end_marker] = T

c_rad = 75e-3  ! Crystal transverse radius 
r0 = 0.2
angle  = 80 * pi / 180
qq = r0 * tan(pi/2-graze_angle)
dft_len = sqrt(qq^2 + r0^2)

source: instrument, l = 0, x_offset = 0, y_offset = 0, z_offset = 0, tilt = 0,
        x_pitch = 0, y_pitch = 0, x_half_length = 0, y_half_length = 0
drift1: pipe, l = dft_len, x_limit = 0.01, y_limit = 0.01
cryst: crystal, crystal_type = 'Si(444)', b_param = -1, tilt = pi, 
        a2_trans_curve = 1 / (2 * c_rad), a4_trans_curve = 1 / (8 * c_rad^3)
drift2: drift, l = dft_len
det: marker, x_limit = 0.01, y_limit = 0.01

lux_line: line = (source, drift1, cryst, drift2, det)
use, lux_line
\end{listing}

The list of lattice elements is given in line 21.  This lattice
constructs of five elements: The source, a crystal, and a detector
with two drift spaces in between. The source element must be the first
element of the lattice. The detector element must be the last
line. Line 5 prevents a final end marker element from being inserted
into the lattice and being mistaken for the detector element.

The definitions of the lattice elements is given in lines 13 through
19.  The reference photon energy is specified in line 3 as 10~KeV and
line 4 sets photons as the reference particle. Lines 1 and 2 set the
``beta'' parameters at the beginning of the lattice. While these
parameters are not used by \lux, they need to be present to prevent
warning messages if this lattice is used by programs other than \lux.

%------------------------------------------------------------------
\subsection{Example Lattice File for BEND Sources}
\label{s:bend.lat}

For \vn{'BEND'} type sources (cf.~\sref{s:photon.start}), a dipole
bend, wiggler, or udulator element must be specified in the lattice
file. This source element must be in a branch suitable for tracking
electrons (or any other charged particle) and a \vn{photon_branch}
element must connect this branch to a branch representing the x-ray
line. Example:
\begin{listing}{1}

beginning[beta_a] = 1
beginning[beta_b] = 1
beginning[e_tot] = 1e4
parameter[particle] = photon
parameter[no_end_marker] = T

c_rad = 75e-3  ! Crystal transverse radius 
r0 = 0.2
angle  = 80 * pi / 180
qq = r0 * tan(pi/2-graze_angle)
dft_len = sqrt(qq^2 + r0^2)

source: instrument, l = 0, x_offset = 0, y_offset = 0, z_offset = 0, tilt = 0,
        x_pitch = 0, y_pitch = 0, x_half_length = 0, y_half_length = 0
drift1: pipe, l = dft_len, x_limit = 0.01, y_limit = 0.01
cryst: crystal, crystal_type = 'Si(444)', b_param = -1, tilt = pi, 
        a2_trans_curve = 1 / (2 * c_rad), a4_trans_curve = 1 / (8 * c_rad^3)
drift2: drift, l = dft_len
det: marker, x_limit = 0.01, y_limit = 0.01

lux_line: line = (source, drift1, cryst, drift2, det)
use, lux_line
\end{listing}

%------------------------------------------------------------------
\section{Photon Description}
\label{s:photon.descrip}

Photons are described by:
\begin{example}
  (x, vx, y, vy, z, vz)   ! six dimensional phase space vector 
  E                       ! Photon energy 
  e_field_x, e_field_y    ! Electric field vector
  phase_x, phase_y        ! Electric field phase 
\end{example}
$(x, y, z)$ is the spatial position of the photon with $z$ being the
longitudinal coordinate and $z=0$ corresponds to the beginning of the
element the photon is passing though (Thus $z$ is generally not
interesting). $(vx, vy, vz)$ is the photon velocity normalized to 1.
Generally photons that make it to the detector will have $vz$ close to
1 and $vx$ and $vy$ relatively ``small''.

%----------------------------------------------

The type of source that is used to generate photons is given by the
parameter \vn{lux_param%source_type}. This parameter may have the following
values
\begin{example}
  'SPHERICAL'
  'PLANAR'
  'BEND'      ! Includes wigglers and undulators
\end{example}

%------------------------------------------------------------------
\subsection{'SPHERICAL' Source Type}

Photons are generated from the first element of the lattice. 
Example source definition in the lattice input file:
\begin{example}
  source: instrument, l = 0, x_offset = 0, y_offset = 0, z_offset = 0,
        x_pitch = 0, y_pitch = 0, tilt = 0, x_half_length = 0, y_half_length = 0
\end{example}
The physical extent of the source is given by the parameters
\begin{example}
  l              ! Longitudinal length (\(2 \sigma\))
  x_half_length  ! Half length in x-direction (\(1 \sigma\))
  y_half_length  ! Half length in the y-direction (\(1 \sigma\)).
\end{example}
With Gaussian spatial distributions, \vn{l} is the $2\sigma$
longitudinal length of the source and \vn{x_half_length} and
\vn{y_half_length} are the $1\sigma$ transverse extents.

The source can be moved spatially by setting the parameters
\begin{example}
  x_offset, y_offset, z_offset
  x_pitch, y_pitch, tilt
\end{example}
See the Bmad manual for more details.

The emission spectrum is governed by the master input file parameters:
\begin{example}
  lux_param%sig_E
  lux_param%dE_center
  lux_param%dE_relative_to_ref
  lux_param%energy_spectrum
  lux_param%transverse_distribution     ! (x, y) spacial distribution
\end{example}
If \vn{lux_param%dE_relative_to_ref} is set to True, The average
starting photon energy is
\begin{example}
  E_average = lux_param%dE_center + beginning[e_tot]
\end{example}
where the reference energy \vn{beginning[e_tot]} is set in the lattice file. 
If \vn{lux_param%dE_relative_to_ref} is set to False, The average
starting photon energy is simply \vn{lux_param%dE_center}.

The width of the energy distribution, in eV, is set by \vn{%sig_E}.
The \vn{%energy_spectrum} and \vn{%transverse_distribution} parameters
establish the type of energy and spatial spectrum used. Possible
settings for these parameters are:
\begin{example}
  "UNIFORM"
  "GAUSSIAN"
\end{example}
A \vn{UNIFORM} spectrum has a constant probability of emission out to
1~sigma and zero outside of this range.

The simulation assumes that in the actual experiment that photons will
be generated uniformly in all directions. Most of these direction will
be ``uninteresting''. An uninteresting direction is a direction in
which a photon has no chance of making its way to the detector. To
reduce computation time, \lux tries to only generate photons in
interesting directions. The prescription for doing this is as follows.

A photon direction $(vx, vy, vz)$ can be represented as a point on the
unit sphere as shown in \fig{f:sphere}. Any point on the sphere can be
characterized by a $(\phi, y)$ where
\begin{align}
  y &= v_y \CRNO
  \tan(\phi) &= \frac{v_x}{v_z}
\end{align}
$y$ must be in the range $(-1, 1)$ and $\phi$ must be in the range
$(-\pi, \pi)$.  The surface of the sphere is broken up into a set of
``tiles''. Each tile covers a rectangular area in $(\phi, y)$ space of
\begin{example}
  lux_param%del_y       ! Tile height
  lux_param%del_phi     ! Tile width
\end{example}
When \lux starts, the program launches a number of photons in the
direction of each tile. If the program determines that a photon headed
in the direction of a given tile has a chance of making it to the
detector, that tile is marked as ``alive''. Otherwise the tile is
marked as ``dead''. When \lux now does the Monte Carlo simulation,
photons are generated only in the direction of live tiles. This is
done to reduce simulation time and should not affect the results.

To cut down on the time it takes to determine which tiles are alive,
only tiles within a rectangle centered about $(\phi, y) = (0, 0)$ are
sampled. The rest are assumed dead. The half with and height of this
rectangle is given in the master input file by
\begin{example}
  lux_param%y_max
  lux_param%phi_max
\end{example}

If \vn{lux_param%use_all_tiles} is set to \vn{True} then all tiles
within the rectangle determined by \vn{lux_param%y_max} and
\vn{lux_param%phi_max} will be used for generating photons.
This can be used as a check to see if the algorithm for deciding
which tiles are alive is accurate.

%------------------------------------------------------------------
\subsection{'PLANAR' Source Type}

The \vn{'PLANAR'} source type is similar to the \vn{'SPHERICAL'}
source type except with \vn{'PLANAR'} all photons are generated
traveling in the same direction. This direction is determined by the
setting of \vn{beam_start} in the lattice file (See the Bmad manual
for details on the \vn{beam_start} parameter).

With a \vn{'PLANAR'} source, there is only one direction tile and the
settings of the following parameters are ignored:
\begin{example}
  lux_param%del_y       
  lux_param%del_phi      
  lux_param%y_max
  lux_param%phi_max
\end{example}
Additionally, the electric field is determined from the source and the setting
of the field in the input file is ignored
\begin{example}
  lux_param%e_field_x
  lux_param%e_field_y
\end{example}

%------------------------------------------------------------------
\subsection{'BEND' Source Type}

The \vn{'BEND'} source type simulates readiation coming from dipole
bending magnets, wigglers, or undulators. In this case, photons are
generated using the first element in the lattice. There must be a
\vn{photon_branch} element in the lattice that branches to the x-ray
line. 

%------------------------------------------------------------------
\subsection{Polarization Init}

The polarization of the photons is set in the master input file by
\begin{example}
  lux_param%e_field_x     ! Polarization along the x-axis
  lux_param%e_field_y     ! polarization along the y-axis
\end{example}
\lux will scale the field of each photon to 1 at the start of
tracking. If both field components are set to zero, random
polarization will be set. The intensity of a photon is defined as the
square of the field so the initial photon intensity is one. As photons
travel, they can loose intensity via, for example, diffraction from a
crystal. This photon intensity is called the ``unnormalized'' photon
intensity. [See \sref{s:det} for an explanation of ``normalized'' intensity.]

%------------------------------------------------------------------
\section{Photon Detection}
\label{s:det}

Photons are tracked until they are lost (hit an aperture) or until
they get to the detector which is the last element in the lattice.

Two parameters in the master input file determine when the simulation
is stopped:
\begin{example}
    lux_param%stop_total_intensity
    lux_param%stop_num_photons
\end{example}
If \vn{lux_param%stop_total_intensity} is positive, the simulation
ends when the total accumulated unnormalized intensity at the detector
passes this threshold. Intensity is the square of the photon electric
field and the unnormalized photon intensity will vary from zero to one for
each photon.

If \vn{lux_param%stop_num_photons} is positive, the simulation is
stopped when the number of photons generated passes this threshold. If
both stop parameters are positive, the simulation is stopped when either
the intensity or the number of photons passes the set threshold.

To properly normalize the intensity at the detector in the output
files, a normalization factor $f_n$ is applied.
\Begineq
  I (\mbox{normalized}) = f_n I (\mbox{unnormalized}) 
\Endeq
$f_n$ is constructed so that the normalized intensity will be
independent of the number of simulation photons and independent of the
tiling parameters. Explicitly:
\Begineq
  f_n = \frac{C_f \, A_t}{4 \, \pi \, N_p}
\Endeq
where $A_t$ is the total area of the live tiles, $N_p$ is the number
of photons tracked, and $C_f$ is a normalization coefficient set by
the \vn{intensity_normalization_coef} parameter in the master input
file, $C_f$ can be used, for example, to scale the output numbers to
correspond to actual measured values.

The detector size is set by the aperture limits of the detector
element in the lattice file. The detector is divided up into a matrix
of pixels. The size of the pixels is given by
\begin{example}
  lux_param%dx_det_pixel
  lux_param%dy_det_pixel
\end{example}

%------------------------------------------------------------------
\section{Numbering the Output Data Files}
\label{s:number}

The names of the output data files are specified by the following
variables (\sref{s:master.file}):
\begin{example}
  photon1_out_file
  det_pix_out_file
\end{example}

If an output data file name contains a pound character ``\#'', then a
number will substituted for the pound character and the number
substituted will be increased by one each time \lux is run. For example,
if \vn{det_pix_out_file} is defined by
\begin{example}
  det_pix_out_file = 'lux.det_pix#'
\end{example}
Then the first time \lux is run, the pixel data file will be named
``lux.det_pix1''.  The next time \lux is run, the data file will be
named ``lux.det_pix2'', etc.

Using a numbering system prevents data files from being
overwritten. If more than one output file name has a pound character,
all such files will receive the same number on a given run. To set the
number given, one can edit the file:
\begin{example}
  lux_out_file.number
\end{example}

%------------------------------------------------------------------
\section{Photon1 Data File}
\label{s:photon1.file}

The \vn{photon1_out_file} parameter (set in the master input file) is
the name of an output data file containing beginning and ``end''
coordinates of a set of photons.

The location where the ``end'' coordinates are evaluated is set by the
\vn{ix_ele_photon} parameter from the master input file. If
\vn{ix_ele_photon} is less than 1 then the end coordinates are
evalueated at the detector element. If \vn{ix_ele_photon} is 1 or more
then the end coordinates are evaluated at the exit end of the lattice
element with index \vn{ix_ele_photon}. 

If \vn{reject_dead_at_det_photon1} is set to True, photons that do not
make it to the detector will not be included in the photon1_out
file. In any case, a photon that does not reach the \vn{ix_ele_photon}
element will not be included. Thus, if the lattice element where the
photon position is being evaluated at is the detector, it will always
be the case that photons that do not make it to the detector will not
be included in the photon1_out file irregardless of the setting of
\vn{reject_dead_at_det_photon1}.

If the file name is blank then no file will be generated. The file
name may be ``numbered'' using a ``\#'' character (\sref{s:number}).

If a photon at the detector has an intensity of less than
\vn{lux_param%intensity_min_photon1_cutoff} then the photon is not
counted.

Each row in the file is a single photon. The columns in the file are:
\begin{example}
  n_live           ! Index of this photon.
  beginning_orbit  ! Six columns: (x, vx, y, vy, z, vz) at the source
  orbit_at_ix_ele  ! Six columns: (x, vx, y, vy, z, vz)
  energy           ! Photon energy (eV).
  intensity        ! Unnormalized photon intensity
\end{example}

%------------------------------------------------------------------
\section{Detector Pixel Data Files}
\label{s:det.pix.file}

The \vn{det_pix_out_file} parameter (set in the master input file) is
the name of an output data file containing pixel-by-pixel information.

If the file name is blank then no file will be generated. The file
name may be ``numbered'' using a ``\#'' character (\sref{s:number}).

The intensity of a pixel is the accumulated intensity of the photons
hitting the pixel. Only pixels whose intensity is greater than
\vn{lux_param%intensity_min_det_pixel_cutoff} are recorded in the
output file. This cutoff is relative to the intensity of the pixel
with the highest intensity.
\vn{lux_param%intensity_min_det_pixel_cutoff} is useful for keeping
the data file sizes small.

Each row in this file represents a single pixel. The columns in this
file are:
\begin{example}
  ix_pix            ! $x$-axis index
  iy_pix            ! $y$-axis index
  x_pix             ! $x$ coordinate of pixel center
  y_pix             ! $y$ coordinate of pixel center
  intensity         ! Normalized pixel intensity
  n_photon          ! Number of photons hitting pixel
  E_ave             ! Average energy deviation from reference (eV)
  E_rms             ! RMS about the average energy (eV)
\end{example}
\vn{E_ave} is an intensity weighted average photon energy deviation
from the reference energy at a pixel. \vn{E_rms} is the RMS variation
of the photon energy with respect to \vn{E_ave}.

In addition to the pixel-by-pixel data file, two additional files will
be generated containing data projected on the $x$ and $y$ detector
alux. The name of these files will be the same as
\vn{det_pix_out_file} with ``.x'' and ``.y'' suffilux indicating
projected data on the $x$ and $y$ alux respectively. The columns for
the $x$-axis projected data is
\begin{example}
  ix_pix            ! $x$-axis index
  x_pix             ! $x$ coordinate of pixel center
  intensity         ! Normalized pixel intensity sum
  n_photon          ! Number of photons hitting pixels
  E_ave             ! Average energy deviation from reference (eV)
  E_rms             ! RMS about the average energy (eV)
\end{example}
With similar columns for the $y$-axis file. The columns for the
$x$-axis projected file records the sum of the intensities for all
pixels with a common \vn{y_pix}. Since this can include pixels not
recorded in the pixel-by-pixel file (due to a non-zero
\vn{lux_param%intensity_min_det_pixel_cutoff} setting), quantities
like the total number of photon need not match between the
pixel-by-pixel file and the projected files.

%------------------------------------------------------------------
\section{Data Visualization}
\label{s:vis}

There is a python script for making an intensity plot of the
\vn{det_pix_out_file} in:
\begin{example}
  lux/plot/det_pix_plot.py
\end{example}
For help run the script
\begin{example}
  <path_to_script>/det_pix_plot.py -help
\end{example}

%------------------------------------------------------------------
\begin{thebibliography}{9}

\bibitem{b:bmad}
D. Sagan,
"Bmad: A Relativistic Charged Particle Simulation Library"
Nuc.\ Instrum.\ \& Methods Phys.\ Res.\ A, {\bf 558}, pp 356-59 (2006).

\end{thebibliography}
\end{document}  
