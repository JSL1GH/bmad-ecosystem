% WARNING: this is automatically generated. DO NOT EDIT.

%% python beam ------------------------------------
\subsection{python beam}
\index{python!beam}
\label{p:beam}


Output beam parameters that are not in the beam_init structure.

\begin{example}
   python beam \{ix_uni\}
\end{example}
\begin{verbatim}
Where:
  {ix_uni} is a universe index. Defaults to s%global%default_universe.

Note: To set beam_init parameters use the "set beam" command.
\end{verbatim}

%% python beam_init ------------------------------------
\subsection{python beam_init}
\index{python!beam_init}
\label{p:beam.init}


Output beam_init parameters.

\begin{example}
   python beam_init \{ix_uni\}
\end{example}
\begin{verbatim}
Where:
  {ix_uni} is a universe index. Defaults to s%global%default_universe.

Note: To set beam_init parameters use the "set beam_init" command
\end{verbatim}

%% python bmad_com ------------------------------------
\subsection{python bmad_com}
\index{python!bmad_com}
\label{p:bmad.com}


Output bmad_com structure components.

\begin{example}
   python bmad_com
\end{example}
\begin{verbatim}

\end{verbatim}

%% python branch1 ------------------------------------
\subsection{python branch1}
\index{python!branch1}
\label{p:branch1}


Output lattice branch information for a particular lattice branch.

\begin{example}
   python branch1 \{ix_uni\}@\{ix_branch\}
\end{example}
\begin{verbatim}
Where:
  {ix_uni} is a universe index. Defaults to s%global%default_universe.
  {ix_branch} is a lattice branch index. Defaults to s%global%default_branch.
\end{verbatim}

%% python bunch_comb ------------------------------------
\subsection{python bunch_comb}
\index{python!bunch_comb}
\label{p:bunch.comb}


Outputs bunch parameters at a comb point. 
Also see the "write bunch_comb" and "show bunch -comb" commands.

\begin{example}
   python bunch_comb \{flags\} \{who\} \{ix_uni\}@\{ix_branch\} \{ix_bunch\}
\end{example}
\begin{verbatim}
Where:
  {flags} are optional switches:
      -array_out : If present, the output will be available in the tao_c_interface_com%c_real.
  {ix_uni} is a universe index. Defaults to s%global%default_universe.
  {ix_branch} is a branch index. Defaults to s%global%default_branch.
  {ix_bunch} is the bunch index. Defaults to 1.
  {who} is one of:
      x, px, y, py, z, pz, t, s, spin.x, spin.y, spin.z, p0c, beta     -- centroid 
      x.Q, y.Q, z.Q, a.Q, b.Q, c.Q where Q is one of: beta, alpha, gamma, phi, eta, etap,
                                                                sigma, sigma_p, emit, norm_emit
    sigma.IJ where I, J in range [1,6]
    rel_min.I, rel_max.I where I in range [1,6]
    charge_live, n_particle_live, n_particle_lost_in_ele, ix_ele

  Note: If ix_uni or ix_branch is present, "@" must be present.

Example:
  python bunch_comb py 2@1 1
\end{verbatim}

%% python bunch_params ------------------------------------
\subsection{python bunch_params}
\index{python!bunch_params}
\label{p:bunch.params}


Outputs bunch parameters at the exit end of a given lattice element.

\begin{example}
   python bunch_params \{ele_id\}|\{which\}
\end{example}
\begin{verbatim}
Where:
  {ele_id} is an element name or index.
  {which} is one of: "model", "base" or "design"

Example:
  python bunch_params end|model  ! parameters at model lattice element named "end".
\end{verbatim}

%% python bunch1 ------------------------------------
\subsection{python bunch1}
\index{python!bunch1}
\label{p:bunch1}


Outputs Bunch parameters at the exit end of a given lattice element.

\begin{example}
   python bunch1 \{ele_id\}|\{which\} \{ix_bunch\} \{coordinate\}
\end{example}
\begin{verbatim}
Where:
  {ele_id} is an element name or index.
  {which} is one of: "model", "base" or "design"
  {ix_bunch} is the bunch index.
  {coordinate} is one of: x, px, y, py, z, pz, "s", "t", "charge", "p0c", "state"

For example, if {coordinate} = "px", the phase space px coordinate of each particle
of the bunch is displayed. The "state" of a particle is an integer. A value of 1 means
alive and any other value means the particle has been lost.
\end{verbatim}

%% python building_wall_list ------------------------------------
\subsection{python building_wall_list}
\index{python!building_wall_list}
\label{p:building.wall.list}


Output List of building wall sections or section points

\begin{example}
   python building_wall_list \{ix_section\}
\end{example}
\begin{verbatim}
Where:
  {ix_section} is a building wall section index.

If {ix_section} is not present, a list of building wall sections is given.
If {ix_section} is present, a list of section points is given.
\end{verbatim}

%% python building_wall_graph ------------------------------------
\subsection{python building_wall_graph}
\index{python!building_wall_graph}
\label{p:building.wall.graph}


Output (x, y) points for drawing the building wall for a particular graph.

\begin{example}
   python building_wall_graph \{graph\}
\end{example}
\begin{verbatim}
Where:
  {graph} is a plot region graph name.

Note: The graph defines the coordinate system for the (x, y) points.
\end{verbatim}

%% python building_wall_point ------------------------------------
\subsection{python building_wall_point}
\index{python!building_wall_point}
\label{p:building.wall.point}


add or delete a building wall point

\begin{example}
   python building_wall_point \{ix_section\}^^\{ix_point\}^^\{z\}^^\{x\}^^\{radius\}^^\{z_center\}^^\{x_center\}
\end{example}
\begin{verbatim}
Where:
  {ix_section}    -- Section index.
  {ix_point}      -- Point index. Points of higher indexes will be moved up 
                       if adding a point and down if deleting.
  {z}, etc...     -- See tao_building_wall_point_struct components.
                  -- If {z} is set to "delete" then delete the point.
\end{verbatim}

%% python building_wall_section ------------------------------------
\subsection{python building_wall_section}
\index{python!building_wall_section}
\label{p:building.wall.section}


Add or delete a building wall section

\begin{example}
   python building_wall_section \{ix_section\}^^\{sec_name\}^^\{sec_constraint\}
\end{example}
\begin{verbatim}
Where:
  {ix_section}      -- Section index. Sections with higher indexes will be
                         moved up if adding a section and down if deleting.
  {sec_name}        -- Section name.
  {sec_constraint}  -- A section constraint name or "delete". Must be one of:
      delete          -- Delete section. Anything else will add the section.
      none
      left_side
      right_side
\end{verbatim}

%% python constraints ------------------------------------
\subsection{python constraints}
\index{python!constraints}
\label{p:constraints}


Output optimization data and variable parameters that contribute to the merit function.

\begin{example}
   python constraints \{who\}
\end{example}
\begin{verbatim}
Where:
  {who} is one of: "data" or "var"

Data constraints output is:
  data name
  constraint type
  evaluation element name
  start element name
  end/reference element name
  measured value
  ref value (only relavent if global%opt_with_ref = T)
  model value
  base value (only relavent if global%opt_with_base = T)
  weight
  merit value
  location where merit is evaluated (if there is a range)
Var constraints output is:
  var name
  Associated varible attribute
  meas value
  ref value (only relavent if global%opt_with_ref = T)
  model value
  base value (only relavent if global%opt_with_base = T)
  weight
  merit value
  dmerit/dvar
\end{verbatim}

%% python da_aperture ------------------------------------
\subsection{python da_aperture}
\index{python!da_aperture}
\label{p:da.aperture}


Output dynamic aperture data

\begin{example}
   python da_aperture \{ix_uni\}
\end{example}
\begin{verbatim}
Where:
  {ix_uni} is a universe index. Defaults to s%global%default_universe.
\end{verbatim}

%% python da_params ------------------------------------
\subsection{python da_params}
\index{python!da_params}
\label{p:da.params}


Output dynamic aperture input parameters

\begin{example}
   python da_params \{ix_uni\}
\end{example}
\begin{verbatim}
Where:
  {ix_uni} is a universe index. Defaults to s%global%default_universe.
\end{verbatim}

%% python data ------------------------------------
\subsection{python data}
\index{python!data}
\label{p:data}


Output Individual datum parameters.

\begin{example}
   python data \{ix_uni\}@\{d2_name\}.\{d1_name\}[\{dat_index\}]
\end{example}
\begin{verbatim}
Where:
  {ix_uni} is a universe index. Defaults to s%global%default_universe.
  {d2_name} is the name of the d2_data structure the datum is in.
  {d1_datum} is the name of the d1_data structure the datum is in.
  {dat_index} is the index of the datum.

Use the "python data-d1" command to get detailed info on a specific d1 array.

Example:
  python data 1@orbit.x[10]
\end{verbatim}

%% python data_d_array ------------------------------------
\subsection{python data_d_array}
\index{python!data_d_array}
\label{p:data.d.array}


Output list of datums for a given d1_data structure.

\begin{example}
   python data_d_array \{ix_uni\}@\{d2_name\}.\{d1_name\}
\end{example}
\begin{verbatim}
Where:
  {ix_uni} is a universe index. Defaults to s%global%default_universe.
  {d2_name} is the name of the containing d2_data structure.
  {d1_name} is the name of the d1_data structure containing the array of datums.

Example:
  python data_d_array 1@orbit.x
\end{verbatim}

%% python data_d1_array ------------------------------------
\subsection{python data_d1_array}
\index{python!data_d1_array}
\label{p:data.d1.array}


Output list of d1 arrays for a given data_d2.

\begin{example}
   python data_d1_array \{d2_datum\}
\end{example}
\begin{verbatim}
{d2_datum} should be of the form
  {ix_uni}@{d2_datum_name}
\end{verbatim}

%% python data_d2 ------------------------------------
\subsection{python data_d2}
\index{python!data_d2}
\label{p:data.d2}


Output information on a d2_datum.

\begin{example}
   python data_d2 \{ix_uni\}@\{d2_name\}
\end{example}
\begin{verbatim}
Where:
  {ix_uni} is a universe index. Defaults to s%global%default_universe.
  {d2_name} is the name of the d2_data structure.
\end{verbatim}

%% python data_d2_array ------------------------------------
\subsection{python data_d2_array}
\index{python!data_d2_array}
\label{p:data.d2.array}


Output data d2 info for a given universe.

\begin{example}
   python data_d2_array \{ix_uni\}
\end{example}
\begin{verbatim}
Where:
  {ix_uni} is a universe index. Defaults to s%global%default_universe.

Example:
  python data_d2_array 1
\end{verbatim}

%% python data_d2_create ------------------------------------
\subsection{python data_d2_create}
\index{python!data_d2_create}
\label{p:data.d2.create}


Create a d2 data structure along with associated d1 and data arrays.

\begin{example}
   python data_d2_create \{ix_uni\}@\{d2_name\}^^\{n_d1_data\}^^\{d_data_arrays_name_min_max\}
\end{example}
\begin{verbatim}
Where:
  {ix_uni} is a universe index. Defaults to s%global%default_universe.
  {d2_name} is the name of the d2_data structure to create.
  {n_d1_data} is the number of associated d1 data structures.
  {d_data_arrays_name_min_max} has the form
    {name1}^^{lower_bound1}^^{upper_bound1}^^....^^{nameN}^^{lower_boundN}^^{upper_boundN}
  where {name} is the data array name and {lower_bound} and {upper_bound} are the bounds of the array.

Example:
  python data_d2_create 2@orbit^^2^^x^^0^^45^^y^^1^^47
This example creates a d2 data structure called "orbit" with 
two d1 structures called "x" and "y".

The "x" d1 structure has an associated data array with indexes in the range [0, 45].
The "y" d1 structure has an associated data arrray with indexes in the range [1, 47].

Use the "set data" command to set created datum parameters.

Note: When setting multiple data parameters, 
      temporarily toggle s%global%lattice_calc_on to False
  ("set global lattice_calc_on = F") to prevent Tao trying to 
      evaluate the partially created datum and generating unwanted error messages.
\end{verbatim}

%% python data_d2_destroy ------------------------------------
\subsection{python data_d2_destroy}
\index{python!data_d2_destroy}
\label{p:data.d2.destroy}


Destroy a d2 data structure along with associated d1 and data arrays.

\begin{example}
   python data_d2_destroy \{ix_uni\}@\{d2_name\}
\end{example}
\begin{verbatim}
Where:
  {ix_uni} is a universe index. Defaults to s%global%default_universe.
  {d2_name} is the name of the d2_data structure to destroy.

Example:
  python data_d2_destroy 2@orbit
This destroys the orbit d2_data structure in universe 2.
\end{verbatim}

%% python data_parameter ------------------------------------
\subsection{python data_parameter}
\index{python!data_parameter}
\label{p:data.parameter}


Output an array of values for a particular datum parameter for a given array of datums, 

\begin{example}
   python data_parameter \{data_array\} \{parameter\}
\end{example}
\begin{verbatim}
{parameter} may be any tao_data_struct parameter.
Example:
  python data_parameter orbit.x model_value
\end{verbatim}

%% python data_set_design_value ------------------------------------
\subsection{python data_set_design_value}
\index{python!data_set_design_value}
\label{p:data.set.design.value}


Set the design (and base \& model) values for all datums.

\begin{example}
   python data_set_design_value
\end{example}
\begin{verbatim}
Example:
  python data_set_design_value

Note: Use the "data_d2_create" and "datum_create" first to create datums.
\end{verbatim}

%% python datum_create ------------------------------------
\subsection{python datum_create}
\index{python!datum_create}
\label{p:datum.create}


Create a datum.

\begin{example}
   python datum_create \{datum_name\}^^\{data_type\}^^\{ele_ref_name\}^^\{ele_start_name\}^^
                       \{ele_name\}^^\{merit_type\}^^\{meas\}^^\{good_meas\}^^\{ref\}^^
                       \{good_ref\}^^\{weight\}^^\{good_user\}^^\{data_source\}^^
                       \{eval_point\}^^\{s_offset\}^^\{ix_bunch\}^^\{invalid_value\}^^
                       \{spin_axis%n0(1)\}^^\{spin_axis%n0(2)\}^^\{spin_axis%n0(3)\}^^
                       \{spin_axis%l(1)\}^^\{spin_axis%l(2)\}^^\{spin_axis%l(3)\}
\end{example}
\begin{verbatim}
Note: The 3 values for spin_axis%n0, as a group, are optional. 
      Also the 3 values for spin_axis%l are, as a group, optional.
Note: Use the "data_d2_create" first to create a d2 structure with associated d1 arrays.
Note: After creating all your datums, use the "data_set_design_value" routine to 
      set the design (and model) values.
\end{verbatim}

%% python datum_has_ele ------------------------------------
\subsection{python datum_has_ele}
\index{python!datum_has_ele}
\label{p:datum.has.ele}


Output whether a datum type has an associated lattice element

\begin{example}
   python datum_has_ele \{datum_type\}
\end{example}
\begin{verbatim}

\end{verbatim}

%% python derivative ------------------------------------
\subsection{python derivative}
\index{python!derivative}
\label{p:derivative}


Output optimization derivatives

\begin{example}
   python derivative
\end{example}
\begin{verbatim}
Note: To save time, this command will not recalculate derivatives. 
Use the "derivative" command beforehand to recalcuate if needed.
\end{verbatim}

%% python ele:ac_kicker ------------------------------------
\subsection{python ele:ac_kicker}
\index{python!ele:ac_kicker}
\label{p:ele:ac.kicker}


Output element ac_kicker parameters

\begin{example}
   python ele:ac_kicker \{ele_id\}|\{which\}
\end{example}
\begin{verbatim}
Where: 
  {ele_id} is an element name or index.
  {which} is one of: "model", "base" or "design"

Example:
  python ele:ac_kicker 3@1>>7|model
This gives element number 7 in branch 1 of universe 3.
\end{verbatim}

%% python ele:cartesian_map ------------------------------------
\subsection{python ele:cartesian_map}
\index{python!ele:cartesian_map}
\label{p:ele:cartesian.map}


Output element cartesian_map parameters

\begin{example}
   python ele:cartesian_map \{ele_id\}|\{which\} \{index\} \{who\}
\end{example}
\begin{verbatim}
Where:
  {ele_id} is an element name or index
  {which} is one of: "model", "base" or "design"
  {index} is the index number in the ele%cartesian_map(:) array
  {who} is one of: "base", or "terms"

Example:
  python ele:cartesian_map 3@1>>7|model 2 base
This gives element number 7 in branch 1 of universe 3.
\end{verbatim}

%% python ele:chamber_wall ------------------------------------
\subsection{python ele:chamber_wall}
\index{python!ele:chamber_wall}
\label{p:ele:chamber.wall}


Output element beam chamber wall parameters

\begin{example}
   python ele:chamber_wall \{ele_id\}|\{which\} \{index\} \{who\}
\end{example}
\begin{verbatim}
Where:
  {ele_id} is an element name or index.
  {which} is one of: "model", "base" or "design"
  {index} is index of the wall.
  {who} is one of:
    "x"       ! Return min/max in horizontal plane
    "y"       ! Return min/max in vertical plane
\end{verbatim}

%% python ele:control_var ------------------------------------
\subsection{python ele:control_var}
\index{python!ele:control_var}
\label{p:ele:control.var}


Output list of element control variables.
Used for group, overlay and ramper type elements.

\begin{example}
   python ele:control_var \{ele_id\}|\{which\}
\end{example}
\begin{verbatim}
Where:
  {ele_id} is an element name or index.
  {which} is one of: "model", "base" or "design"

Example:
  python ele:control_var 3@1>>7|model
This gives element number 7 in branch 1 of universe 3.
\end{verbatim}

%% python ele:cylindrical_map ------------------------------------
\subsection{python ele:cylindrical_map}
\index{python!ele:cylindrical_map}
\label{p:ele:cylindrical.map}


Output element cylindrical_map

\begin{example}
   python ele:cylindrical_map \{ele_id\}|\{which\} \{index\} \{who\}
\end{example}
\begin{verbatim}
Where 
  {ele_id} is an element name or index.
  {which} is one of: "model", "base" or "design"
  {index} is the index number in the ele%cylindrical_map(:) array
  {who} is one of: "base", or "terms"

Example:
  python ele:cylindrical_map 3@1>>7|model 2 base
This gives map #2 of element number 7 in branch 1 of universe 3.
\end{verbatim}

%% python ele:elec_multipoles ------------------------------------
\subsection{python ele:elec_multipoles}
\index{python!ele:elec_multipoles}
\label{p:ele:elec.multipoles}


Output element electric multipoles

\begin{example}
   python ele:elec_multipoles \{ele_id\}|\{which\}
\end{example}
\begin{verbatim}
Where:
  {ele_id} is an element name or index.
  {which} is one of: "model", "base" or "design"

Example:
  python ele:elec_multipoles 3@1>>7|model
This gives element number 7 in branch 1 of universe 3.
\end{verbatim}

%% python ele:floor ------------------------------------
\subsection{python ele:floor}
\index{python!ele:floor}
\label{p:ele:floor}


Output element floor coordinates. The output gives two lines. "Reference" is
without element misalignments and "Actual" is with misalignments.

\begin{example}
   python ele:floor \{ele_id\}|\{which\} \{where\}
\end{example}
\begin{verbatim}
Where:
  {ele_id} is an element name or index.
  {which} is one of: "model", "base" or "design"
  {where} is an optional argument which, if present, is one of
    beginning  ! Upstream end
    center     ! Middle of element
    end        ! Downstream end (default)
Note: {where} ignored for photonic elements crystal, mirror, and multilayer_mirror.

Example:
  python ele:floor 3@1>>7|model
This gives element number 7 in branch 1 of universe 3.
\end{verbatim}

%% python ele:gen_attribs ------------------------------------
\subsection{python ele:gen_attribs}
\index{python!ele:gen_attribs}
\label{p:ele:gen.attribs}


Output element general attributes

\begin{example}
   python ele:gen_attribs \{ele_id\}|\{which\}
\end{example}
\begin{verbatim}
Where: 
  {ele_id} is an element name or index.
  {which} is one of: "model", "base" or "design"

Example:
  python ele:gen_attribs 3@1>>7|model
This gives element number 7 in branch 1 of universe 3.
\end{verbatim}

%% python ele:gen_grad_map ------------------------------------
\subsection{python ele:gen_grad_map}
\index{python!ele:gen_grad_map}
\label{p:ele:gen.grad.map}


Output element gen_grad_map 

\begin{example}
   python ele:gen_grad_map \{ele_id\}|\{which\} \{index\} \{who\}
\end{example}
\begin{verbatim}
Where: 
  {ele_id} is an element name or index.
  {which} is one of: "model", "base" or "design"
  {index} is the index number in the ele%gen_grad_map(:) array
  {who} is one of: "base", or "derivs".

Example:
  python ele:gen_grad_map 3@1>>7|model 2 base
This gives element number 7 in branch 1 of universe 3.
\end{verbatim}

%% python ele:grid_field ------------------------------------
\subsection{python ele:grid_field}
\index{python!ele:grid_field}
\label{p:ele:grid.field}


Output element grid_field

\begin{example}
   python ele:grid_field \{ele_id\}|\{which\} \{index\} \{who\}
\end{example}
\begin{verbatim}
Where:
  {ele_id} is an element name or index.
  {which} is one of: "model", "base" or "design"
  {index} is the index number in the ele%grid_field(:) array.
  {who} is one of: "base", or "points"

Example:
  python ele:grid_field 3@1>>7|model 2 base
This gives grid #2 of element number 7 in branch 1 of universe 3.
\end{verbatim}

%% python ele:head ------------------------------------
\subsection{python ele:head}
\index{python!ele:head}
\label{p:ele:head}


Output "head" Element attributes

\begin{example}
   python ele:head \{ele_id\}|\{which\}
\end{example}
\begin{verbatim}
Where: 
  {ele_id} is an element name or index.
  {which} is one of: "model", "base" or "design"

Example:
  python ele:head 3@1>>7|model
This gives element number 7 in branch 1 of universe 3.
\end{verbatim}

%% python ele:lord_slave ------------------------------------
\subsection{python ele:lord_slave}
\index{python!ele:lord_slave}
\label{p:ele:lord.slave}


Output the lord/slave tree of an element.

\begin{example}
   python ele:lord_slave \{ele_id\}|\{which\}
\end{example}
\begin{verbatim}
Where: 
  {ele_id} is an element name or index.
  {which} is one of: "model", "base" or "design"

Example:
  python ele:lord_slave 3@1>>7|model
This gives lord and slave info on element number 7 in branch 1 of universe 3.
Note: The lord/slave info is independent of the setting of {which}.

The output is a number of lines, each line giving information on an element (element index, etc.).
Some lines begin with the word "Element". 
After each "Element" line, there are a number of lines (possibly zero) that begin with the word "Slave or "Lord".
These "Slave" and "Lord" lines are the slaves and lords of the "Element" element.
\end{verbatim}

%% python ele:mat6 ------------------------------------
\subsection{python ele:mat6}
\index{python!ele:mat6}
\label{p:ele:mat6}


Output element mat6

\begin{example}
   python ele:mat6 \{ele_id\}|\{which\} \{who\}
\end{example}
\begin{verbatim}
Where: 
  {ele_id} is an element name or index.
  {which} is one of: "model", "base" or "design"
  {who} is one of: "mat6", "vec0", or "err"

Example:
  python ele:mat6 3@1>>7|model mat6
This gives element number 7 in branch 1 of universe 3.
\end{verbatim}

%% python ele:methods ------------------------------------
\subsection{python ele:methods}
\index{python!ele:methods}
\label{p:ele:methods}


Output element methods

\begin{example}
   python ele:methods \{ele_id\}|\{which\}
\end{example}
\begin{verbatim}
Where: 
  {ele_id} is an element name or index.
  {which} is one of: "model", "base" or "design"

Example:
  python ele:methods 3@1>>7|model
This gives element number 7 in branch 1 of universe 3.
\end{verbatim}

%% python ele:multipoles ------------------------------------
\subsection{python ele:multipoles}
\index{python!ele:multipoles}
\label{p:ele:multipoles}


Output element multipoles

\begin{example}
   python ele:multipoles \{ele_id\}|\{which\}
\end{example}
\begin{verbatim}
Where: 
  {ele_id} is an element name or index.
  {which} is one of: "model", "base" or "design"

Example:
  python ele:multipoles 3@1>>7|model
This gives element number 7 in branch 1 of universe 3.
\end{verbatim}

%% python ele:orbit ------------------------------------
\subsection{python ele:orbit}
\index{python!ele:orbit}
\label{p:ele:orbit}


Output element orbit

\begin{example}
   python ele:orbit \{ele_id\}|\{which\}
\end{example}
\begin{verbatim}
Where: 
  {ele_id} is an element name or index.
  {which} is one of: "model", "base" or "design"

Example:
  python ele:orbit 3@1>>7|model
This gives element number 7 in branch 1 of universe 3.
\end{verbatim}

%% python ele:param ------------------------------------
\subsection{python ele:param}
\index{python!ele:param}
\label{p:ele:param}


Output lattice element parameter

\begin{example}
   python ele:param \{ele_id\}|\{which\} \{who\}
\end{example}
\begin{verbatim}
Where: 
  {ele_id} is an element name or index.
  {which} is one of: "model", "base" or "design"
  {who} values are the same as {who} values for "python lat_list".
        Note: Here {who} must be a single parameter and not a list.

Example:
  python ele:param 3@1>>7|model e_tot
This gives E_tot of element number 7 in branch 1 of universe 3.

Note: On output the {variable} component will always be "F" (since this 
command cannot tell if a parameter is allowed to vary).

Also see: "python lat_list".
\end{verbatim}

%% python ele:photon ------------------------------------
\subsection{python ele:photon}
\index{python!ele:photon}
\label{p:ele:photon}


Output element photon parameters

\begin{example}
   python ele:photon \{ele_id\}|\{which\} \{who\}
\end{example}
\begin{verbatim}
Where: 
  {ele_id} is an element name or index.
  {which} is one of: "model", "base" or "design"
  {who} is one of: "base", "material", or "curvature"

Example:
  python ele:photon 3@1>>7|model base
This gives element number 7 in branch 1 of universe 3.
\end{verbatim}

%% python ele:spin_taylor ------------------------------------
\subsection{python ele:spin_taylor}
\index{python!ele:spin_taylor}
\label{p:ele:spin.taylor}


Output element spin_taylor parameters

\begin{example}
   python ele:spin_taylor \{ele_id\}|\{which\}
\end{example}
\begin{verbatim}
Where: 
  {ele_id} is an element name or index.
  {which} is one of: "model", "base" or "design"

Example:
  python ele:spin_taylor 3@1>>7|model
This gives element number 7 in branch 1 of universe 3.
\end{verbatim}

%% python ele:taylor ------------------------------------
\subsection{python ele:taylor}
\index{python!ele:taylor}
\label{p:ele:taylor}


Output element taylor map 

\begin{example}
   python ele:taylor \{ele_id\}|\{which\}
\end{example}
\begin{verbatim}
Where: 
  {ele_id} is an element name or index.
  {which} is one of: "model", "base" or "design"

Example:
  python ele:taylor 3@1>>7|model
This gives element number 7 in branch 1 of universe 3.
\end{verbatim}

%% python ele:twiss ------------------------------------
\subsection{python ele:twiss}
\index{python!ele:twiss}
\label{p:ele:twiss}


Output element Twiss parameters

\begin{example}
   python ele:twiss \{ele_id\}|\{which\}
\end{example}
\begin{verbatim}
Where: 
  {ele_id} is an element name or index.
  {which} is one of: "model", "base" or "design"

Example:
  python ele:twiss 3@1>>7|model
This gives element number 7 in branch 1 of universe 3.
\end{verbatim}

%% python ele:wake ------------------------------------
\subsection{python ele:wake}
\index{python!ele:wake}
\label{p:ele:wake}


Output element wake.

\begin{example}
   python ele:wake \{ele_id\}|\{which\} \{who\}
\end{example}
\begin{verbatim}
Where: 
  {ele_id} is an element name or index.
  {which} is one of: "model", "base" or "design"
  {Who} is one of:
      "sr_long"        "sr_long_table"
      "sr_trans"       "sr_trans_table"
      "lr_mode_table"  "base"

Example:
  python ele:wake 3@1>>7|model
This gives element number 7 in branch 1 of universe 3.
\end{verbatim}

%% python ele:wall3d ------------------------------------
\subsection{python ele:wall3d}
\index{python!ele:wall3d}
\label{p:ele:wall3d}


Output element wall3d parameters.

\begin{example}
   python ele:wall3d \{ele_id\}|\{which\} \{index\} \{who\}
\end{example}
\begin{verbatim}
Where: 
  {ele_id} is an element name or index.
  {which} is one of: "model", "base" or "design"
  {index} is the index number in the ele%wall3d(:) array (size obtained from "ele:head").
  {who} is one of: "base", or "table".
Example:
  python ele:wall3d 3@1>>7|model 2 base
This gives element number 7 in branch 1 of universe 3.
\end{verbatim}

%% python evaluate ------------------------------------
\subsection{python evaluate}
\index{python!evaluate}
\label{p:evaluate}


Output the value of an expression. The result may be a vector.

\begin{example}
   python evaluate \{flags\} \{expression\}
\end{example}
\begin{verbatim}
Where:
  Optional {flags} are:
      -array_out : If present, the output will be available in the tao_c_interface_com%c_real.
  {expression} is expression to be evaluated.

Example:
  python evaluate 3+data::cbar.11[1:10]|model
\end{verbatim}

%% python em_field ------------------------------------
\subsection{python em_field}
\index{python!em_field}
\label{p:em.field}


Output EM field at a given point generated by a given element.

\begin{example}
   python em_field \{ele_id\}|\{which\} \{x\} \{y\} \{z\} \{t_or_z\}
\end{example}
\begin{verbatim}
Where:
  {which} is one of: "model", "base" or "design"
  {x}, {y}  -- Transverse coords.
  {z}       -- Longitudinal coord with respect to entrance end of element.
  {t_or_z}  -- time or phase space z depending if lattice is setup for absolute time tracking.
\end{verbatim}

%% python enum ------------------------------------
\subsection{python enum}
\index{python!enum}
\label{p:enum}


Output list of possible values for enumerated numbers.

\begin{example}
   python enum \{enum_name\}
\end{example}
\begin{verbatim}
Example:
  python enum tracking_method
\end{verbatim}

%% python floor_plan ------------------------------------
\subsection{python floor_plan}
\index{python!floor_plan}
\label{p:floor.plan}


Output (x,y) points and other information that can be used for drawing a floor_plan.

\begin{example}
   python floor_plan \{graph\}
\end{example}
\begin{verbatim}

\end{verbatim}

%% python floor_orbit ------------------------------------
\subsection{python floor_orbit}
\index{python!floor_orbit}
\label{p:floor.orbit}


Output (x, y) coordinates for drawing the particle orbit on a floor plan.

\begin{example}
   python floor_orbit \{graph\}
\end{example}
\begin{verbatim}

\end{verbatim}

%% python global ------------------------------------
\subsection{python global}
\index{python!global}
\label{p:global}


Output global parameters.

\begin{example}
   python global
\end{example}
\begin{verbatim}
Output syntax is parameter list form. See documentation at the beginning of this file.

Note: The follow is intentionally left out:
  optimizer_allow_user_abort
  quiet
  single_step
  prompt_color
  prompt_string
\end{verbatim}

%% python help ------------------------------------
\subsection{python help}
\index{python!help}
\label{p:help}


Output list of "help xxx" topics

\begin{example}
   python help
\end{example}
\begin{verbatim}

\end{verbatim}

%% python inum ------------------------------------
\subsection{python inum}
\index{python!inum}
\label{p:inum}


Output list of possible values for an INUM parameter.
For example, possible index numbers for the branches of a lattice.

\begin{example}
   python inum \{who\}
\end{example}
\begin{verbatim}

\end{verbatim}

%% python lat_calc_done ------------------------------------
\subsection{python lat_calc_done}
\index{python!lat_calc_done}
\label{p:lat.calc.done}


Output if a lattice recalculation has been proformed since the last 
  time "python lat_calc_done" was called.

\begin{example}
   python lat_calc_done
\end{example}
\begin{verbatim}

\end{verbatim}

%% python lat_ele_list ------------------------------------
\subsection{python lat_ele_list}
\index{python!lat_ele_list}
\label{p:lat.ele.list}


Output lattice element list.

\begin{example}
   python lat_ele_list \{branch_name\}
\end{example}
\begin{verbatim}
{branch_name} should have the form:
  {ix_uni}@{ix_branch}
\end{verbatim}

%% python lat_branch_list ------------------------------------
\subsection{python lat_branch_list}
\index{python!lat_branch_list}
\label{p:lat.branch.list}


Output lattice branch list

\begin{example}
   python lat_branch_list \{ix_uni\}
\end{example}
\begin{verbatim}
Output syntax:
  branch_index;branch_name;n_ele_track;n_ele_max
\end{verbatim}

%% python lat_list ------------------------------------
\subsection{python lat_list}
\index{python!lat_list}
\label{p:lat.list}


Output list of parameters at ends of lattice elements

\begin{example}
   python lat_list \{flags\} \{ix_uni\}@\{ix_branch\}>>\{elements\}|\{which\} \{who\}
\end{example}
\begin{verbatim}
Where:
 Optional {flags} are:
  -no_slaves : If present, multipass_slave and super_slave elements will not be matched to.
  -track_only : If present, lord elements will not be matched to.
  -index_order : If present, order elements by element index instead of the standard s-position.
  -array_out : If present, the output will be available in the tao_c_interface_com%c_real or
    tao_c_interface_com%c_integer arrays. See the code below for when %c_real vs %c_integer is used.
    Note: Only a single {who} item permitted when -array_out is present.

  {which} is one of: "model", "base" or "design"

  {who} is a comma deliminated list of:
    orbit.floor.x, orbit.floor.y, orbit.floor.z    ! Floor coords at particle orbit.
    orbit.spin.1, orbit.spin.2, orbit.spin.3,
    orbit.vec.1, orbit.vec.2, orbit.vec.3, orbit.vec.4, orbit.vec.5, orbit.vec.6,
    orbit.t, orbit.beta,
    orbit.state,     ! Note: state is an integer. alive$ = 1, anything else is lost.
    orbit.energy, orbit.pc,
    ele.name, ele.key, ele.ix_ele, ele.ix_branch
    ele.a.beta, ele.a.alpha, ele.a.eta, ele.a.etap, ele.a.gamma, ele.a.phi,
    ele.b.beta, ele.b.alpha, ele.b.eta, ele.b.etap, ele.b.gamma, ele.b.phi,
    ele.x.eta, ele.x.etap,
    ele.y.eta, ele.y.etap,
    ele.ref_time, ele.ref_time_start
    ele.s, ele.l
    ele.e_tot, ele.p0c
    ele.mat6      ! Output: mat6(1,:), mat6(2,:), ... mat6(6,:)
    ele.vec0      ! Output: vec0(1), ... vec0(6)
    ele.{attribute} Where {attribute} is a Bmad syntax element attribute. (EG: ele.beta_a, ele.k1, etc.)
    ele.c_mat     ! Output: c_mat11, c_mat12, c_mat21, c_mat22.
    ele.gamma_c   ! Parameter associated with coupling c-matrix.

  {elements} is a string to match element names to.
    Use "*" to match to all elements.

Examples:
  python lat_list -track 3@0>>Q*|base ele.s,orbit.vec.2
  python lat_list 3@0>>Q*|base real:ele.s    

Also see: "python ele:param"
\end{verbatim}

%% python lat_param_units ------------------------------------
\subsection{python lat_param_units}
\index{python!lat_param_units}
\label{p:lat.param.units}


Output units of a parameter associated with a lattice or lattice element.

\begin{example}
   python lat_param_units \{param_name\}
\end{example}
\begin{verbatim}

\end{verbatim}

%% python matrix ------------------------------------
\subsection{python matrix}
\index{python!matrix}
\label{p:matrix}


Output matrix value from the exit end of one element to the exit end of the other.

\begin{example}
   python matrix \{ele1_id\} \{ele2_id\}
\end{example}
\begin{verbatim}
Where:
  {ele1_id} is the start element.
  {ele2_id} is the end element.
If {ele2_id} = {ele1_id}, the 1-turn transfer map is computed.
Note: {ele2_id} should just be an element name or index without universe, branch, or model/base/design specification.

Example:
  python matrix 2@1>>q01w|design q02w
\end{verbatim}

%% python merit ------------------------------------
\subsection{python merit}
\index{python!merit}
\label{p:merit}


Output merit value.

\begin{example}
   python merit
\end{example}
\begin{verbatim}

\end{verbatim}

%% python orbit_at_s ------------------------------------
\subsection{python orbit_at_s}
\index{python!orbit_at_s}
\label{p:orbit.at.s}


Output twiss at given s position.

\begin{example}
   python orbit_at_s \{ix_uni\}@\{ele\}->\{s_offset\}|\{which\}
\end{example}
\begin{verbatim}
Where:
  {ix_uni} is a universe index. Defaults to s%global%default_universe.
  {ele} is an element name or index. Default at the Beginning element at start of branch 0.
  {s_offset} is the offset of the evaluation point from the downstream end of ele. Default is 0.
     If {s_offset} is present, the preceeding "->" sign must be present. EG: Something like "23|model" will
  {which} is one of: "model", "base" or "design".

Example:
  python orbit_at_s Q10->0.4|model   ! Orbit at 0.4 meters from Q10 element exit end in model lattice.
\end{verbatim}

%% python place_buffer ------------------------------------
\subsection{python place_buffer}
\index{python!place_buffer}
\label{p:place.buffer}


Output the place command buffer and reset the buffer.
The contents of the buffer are the place commands that the user has issued.
See the Tao manual for more details.

\begin{example}
   python place_buffer
\end{example}
\begin{verbatim}

\end{verbatim}

%% python plot_curve ------------------------------------
\subsection{python plot_curve}
\index{python!plot_curve}
\label{p:plot.curve}


Output curve information for a plot.

\begin{example}
   python plot_curve \{curve_name\}
\end{example}
\begin{verbatim}

\end{verbatim}

%% python plot_lat_layout ------------------------------------
\subsection{python plot_lat_layout}
\index{python!plot_lat_layout}
\label{p:plot.lat.layout}


Output plot Lat_layout info

\begin{example}
   python plot_lat_layout \{ix_uni\}@\{ix_branch\}
\end{example}
\begin{verbatim}
Note: The returned list of element positions is not ordered in increasing
      longitudinal position.
\end{verbatim}

%% python plot_list ------------------------------------
\subsection{python plot_list}
\index{python!plot_list}
\label{p:plot.list}


Output list of plot templates or plot regions.

\begin{example}
   python plot_list \{r_or_g\}
\end{example}
\begin{verbatim}
where "{r/g}" is:
  "r"      ! list regions of the form ix;region_name;plot_name;visible;x1;x2;y1;y2
  "t"      ! list template plots of the form ix;name
\end{verbatim}

%% python plot_graph ------------------------------------
\subsection{python plot_graph}
\index{python!plot_graph}
\label{p:plot.graph}


Output graph info.

\begin{example}
   python plot_graph \{graph_name\}
\end{example}
\begin{verbatim}
{graph_name} is in the form:
  {p_name}.{g_name}
where
  {p_name} is the plot region name if from a region or the plot name if a template plot.
  This name is obtained from the python plot_list command.
  {g_name} is the graph name obtained from the python plot1 command.
\end{verbatim}

%% python plot_histogram ------------------------------------
\subsection{python plot_histogram}
\index{python!plot_histogram}
\label{p:plot.histogram}


Output plot histogram info.

\begin{example}
   python plot_histogram \{curve_name\}
\end{example}
\begin{verbatim}

\end{verbatim}

%% python plot_template_manage ------------------------------------
\subsection{python plot_template_manage}
\index{python!plot_template_manage}
\label{p:plot.template.manage}


Template plot creation or destruction.

\begin{example}
   python plot_template_manage \{template_location\}^^\{template_name\}^^
                          \{n_graph\}^^\{graph_names\}
\end{example}
\begin{verbatim}
Where:
  {template_location} is the location to place or delete a template plot. Use "@Tnnn" syntax for the location.
  {template_name} is the name of the template plot. If deleting a plot this name is immaterial.
  {n_graph} is the number of associated graphs. If set to -1 then any existing template plot is deleted.
  {graph_names} are the names of the graphs.  graph_names should be in the form:
     graph1_name^^graph2_name^^...^^graphN_name
  for N=n_graph names
\end{verbatim}

%% python plot_curve_manage ------------------------------------
\subsection{python plot_curve_manage}
\index{python!plot_curve_manage}
\label{p:plot.curve.manage}


Template plot curve creation/destruction

\begin{example}
   python plot_curve_manage \{graph_name\}^^\{curve_index\}^^\{curve_name\}
\end{example}
\begin{verbatim}
If {curve_index} corresponds to an existing curve then this curve is deleted.
In this case the {curve_name} is ignored and does not have to be present.
If {curve_index} does not not correspond to an existing curve, {curve_index}
must be one greater than the number of curves.
\end{verbatim}

%% python plot_graph_manage ------------------------------------
\subsection{python plot_graph_manage}
\index{python!plot_graph_manage}
\label{p:plot.graph.manage}


Template plot graph creation/destruction

\begin{example}
   python plot_graph_manage \{plot_name\}^^\{graph_index\}^^\{graph_name\}
\end{example}
\begin{verbatim}
If {graph_index} corresponds to an existing graph then this graph is deleted.
In this case the {graph_name} is ignored and does not have to be present.
If {graph_index} does not not correspond to an existing graph, {graph_index}
must be one greater than the number of graphs.
\end{verbatim}

%% python plot_line ------------------------------------
\subsection{python plot_line}
\index{python!plot_line}
\label{p:plot.line}


Output points used to construct the "line" associated with a plot curve.

\begin{example}
   python plot_line \{region_name\}.\{graph_name\}.\{curve_name\} \{x_or_y\}
\end{example}
\begin{verbatim}
Optional {x-or-y} may be set to "x" or "y" to get the smooth line points x or y 
component put into the real array buffer.
Note: The plot must come from a region, and not a template, since no template plots 
      have associated line data.
Examples:
  python plot_line r13.g.a   ! String array output.
  python plot_line r13.g.a x ! x-component of line points loaded into the real array buffer.
  python plot_line r13.g.a y ! y-component of line points loaded into the real array buffer.
\end{verbatim}

%% python plot_symbol ------------------------------------
\subsection{python plot_symbol}
\index{python!plot_symbol}
\label{p:plot.symbol}


Output locations to draw symbols for a plot curve.

\begin{example}
   python plot_symbol \{region_name\}.\{graph_name\}.\{curve_name\} \{x_or_y\}
\end{example}
\begin{verbatim}
Optional {x_or_y} may be set to "x" or "y" to get the symbol x or y 
positions put into the real array buffer.
Note: The plot must come from a region, and not a template, 
      since no template plots have associated symbol data.
Examples:
  python plot_symbol r13.g.a       ! String array output.
  python plot_symbol r13.g.a x     ! x-component of the symbol positions 
                                     loaded into the real array buffer.
  python plot_symbol r13.g.a y     ! y-component of the symbol positions 
                                     loaded into the real array buffer.
\end{verbatim}

%% python plot_transfer ------------------------------------
\subsection{python plot_transfer}
\index{python!plot_transfer}
\label{p:plot.transfer}


Output transfer plot parameters from the "from plot" to the "to plot" (or plots).

\begin{example}
   python plot_transfer \{from_plot\} \{to_plot\}
\end{example}
\begin{verbatim}
To avoid confusion, use "@Tnnn" and "@Rnnn" syntax for {from_plot}.
If {to_plot} is not present and {from_plot} is a template plot, the "to plots" 
 are the equivalent region plots with the same name. And vice versa 
 if {from_plot} is a region plot.
\end{verbatim}

%% python plot1 ------------------------------------
\subsection{python plot1}
\index{python!plot1}
\label{p:plot1}


Output info on a given plot.

\begin{example}
   python plot1 \{name\}
\end{example}
\begin{verbatim}
{name} should be the region name if the plot is associated with a region.
Output syntax is parameter list form. See documentation at the beginning of this file.
\end{verbatim}

%% python ptc_com ------------------------------------
\subsection{python ptc_com}
\index{python!ptc_com}
\label{p:ptc.com}


Output Ptc_com structure components.

\begin{example}
   python ptc_com
\end{example}
\begin{verbatim}

\end{verbatim}

%% python ring_general ------------------------------------
\subsection{python ring_general}
\index{python!ring_general}
\label{p:ring.general}


Output lattice branch with closed geometry info (emittances, etc.)

\begin{example}
   python ring_general \{ix_uni\}@\{ix_branch\}|\{which\}
\end{example}
\begin{verbatim}
where {which} is one of:
  model
  base
  design
Example:
  python ring_general 1@0|model
\end{verbatim}

%% python shape_list ------------------------------------
\subsection{python shape_list}
\index{python!shape_list}
\label{p:shape.list}


Output lat_layout or floor_plan shapes list

\begin{example}
   python shape_list \{who\}
\end{example}
\begin{verbatim}
{who} is one of:
  lat_layout
  floor_plan
\end{verbatim}

%% python shape_manage ------------------------------------
\subsection{python shape_manage}
\index{python!shape_manage}
\label{p:shape.manage}


Element shape creation or destruction

\begin{example}
   python shape_manage \{who\} \{index\} \{add_or_delete\}
\end{example}
\begin{verbatim}
{who} is one of:
  lat_layout
  floor_plan
{add_or_delete} is one of:
  add     -- Add a shape at {index}. 
             Shapes with higher index get moved up one to make room.
  delete  -- Delete shape at {index}. 
             Shapes with higher index get moved down one to fill the gap.

Example:
  python shape_manage floor_plan 2 add
Note: After adding a shape use "python shape_set" to set shape parameters.
This is important since an added shape is in a ill-defined state.
\end{verbatim}

%% python shape_pattern_list ------------------------------------
\subsection{python shape_pattern_list}
\index{python!shape_pattern_list}
\label{p:shape.pattern.list}


Output list of shape patterns or shape pattern points

\begin{example}
   python shape_pattern_list \{ix_pattern\}
\end{example}
\begin{verbatim}
If optional {ix_pattern} index is omitted then list all the patterns.
If {ix_pattern} is present, list points of given pattern.
\end{verbatim}

%% python shape_pattern_manage ------------------------------------
\subsection{python shape_pattern_manage}
\index{python!shape_pattern_manage}
\label{p:shape.pattern.manage}


Add or remove shape pattern

\begin{example}
   python shape_pattern_manage \{ix_pattern\}^^\{pat_name\}^^\{pat_line_width\}
\end{example}
\begin{verbatim}
Where:
  {ix_pattern}      -- Pattern index. Patterns with higher indexes will be moved up 
                                      if adding a pattern and down if deleting.
  {pat_name}        -- Pattern name.
  {pat_line_width}  -- Line width. Integer. If set to "delete" then section 
                                            will be deleted.
\end{verbatim}

%% python shape_pattern_point_manage ------------------------------------
\subsection{python shape_pattern_point_manage}
\index{python!shape_pattern_point_manage}
\label{p:shape.pattern.point.manage}


Add or remove shape pattern point

\begin{example}
   python shape_pattern_point_manage \{ix_pattern\}^^\{ix_point\}^^\{s\}^^\{x\}
\end{example}
\begin{verbatim}
Where:
  {ix_pattern}      -- Pattern index.
  {ix_point}        -- Point index. Points of higher indexes will be moved up
                                    if adding a point and down if deleting.
  {s}, {x}          -- Point location. If {s} is "delete" then delete the point.
\end{verbatim}

%% python shape_set ------------------------------------
\subsection{python shape_set}
\index{python!shape_set}
\label{p:shape.set}


Set lat_layout or floor_plan shape parameters.

\begin{example}
   python shape_set \{who\}^^\{shape_index\}^^\{ele_name\}^^\{shape\}^^\{color\}^^
                    \{shape_size\}^^\{type_label\}^^\{shape_draw\}^^
                    \{multi_shape\}^^\{line_width\}
\end{example}
\begin{verbatim}
{who} is one of:
  lat_layout
  floor_plan
\end{verbatim}

%% python show ------------------------------------
\subsection{python show}
\index{python!show}
\label{p:show}


Output the output from a show command.

\begin{example}
   python show \{line\}
\end{example}
\begin{verbatim}
{line} is the string to pass through to the show command.
Example:
  python show lattice -python
\end{verbatim}

%% python species_to_int ------------------------------------
\subsection{python species_to_int}
\index{python!species_to_int}
\label{p:species.to.int}


Convert species name to corresponding integer

\begin{example}
   python species_to_int \{species_str\}
\end{example}
\begin{verbatim}
Example:
  python species_to_int CO2++
\end{verbatim}

%% python species_to_str ------------------------------------
\subsection{python species_to_str}
\index{python!species_to_str}
\label{p:species.to.str}


Convert species integer id to corresponding

\begin{example}
   python species_to_str \{species_int\}
\end{example}
\begin{verbatim}
Example:
  python species_to_str -1     ! Returns 'Electron'
\end{verbatim}

%% python spin_invariant ------------------------------------
\subsection{python spin_invariant}
\index{python!spin_invariant}
\label{p:spin.invariant}


Output closed orbit spin axes n0, l0, or m0 at the ends of all lattice elements in a branch.
n0, l0, and m0 are solutions of the T-BMT equation.
n0 is periodic while l0 and m0 are not. At the beginning of the branch, the orientation of the 
l0 or m0 axes in the plane perpendicular to the n0 axis is chosen a bit arbitrarily.
See the Bmad manual for more details.

\begin{example}
   python spin_invariant \{flags\} \{who\} \{ix_uni\}@\{ix_branch\}|\{which\}
\end{example}
\begin{verbatim}
Where:
  {flags} are optional switches:
      -array_out : If present, the output will be available in the tao_c_interface_com%c_real.
  {who} is one of: l0, n0, or m0
  {ix_uni} is a universe index. Defaults to s%global%default_universe.
  {ix_branch} is a branch index. Defaults to s%global%default_branch.
  {which} is one of:
    model
    base
    design

Example:
  python spin_invariant 1@0|model

Note: This command is under development. If you want to use please contact David Sagan.
\end{verbatim}

%% python spin_polarization ------------------------------------
\subsection{python spin_polarization}
\index{python!spin_polarization}
\label{p:spin.polarization}


Output spin polarization information

\begin{example}
   python spin_polarization \{ix_uni\}@\{ix_branch\}|\{which\}
\end{example}
\begin{verbatim}
Where:
  {ix_uni} is a universe index. Defaults to s%global%default_universe.
  {ix_branch} is a branch index. Defaults to s%global%default_branch.
  {which} is one of:
    model
    base
    design

Example:
  python spin_polarization 1@0|model

Note: This command is under development. If you want to use please contact David Sagan.
\end{verbatim}

%% python spin_resonance ------------------------------------
\subsection{python spin_resonance}
\index{python!spin_resonance}
\label{p:spin.resonance}


Output spin resonance information

\begin{example}
   python spin_resonance \{ix_uni\}@\{ix_branch\}|\{which\} \{ref_ele\}
\end{example}
\begin{verbatim}
Where:
  {ix_uni} is a universe index. Defaults to s%global%default_universe.
  {ix_branch} is a lattice branch index. Defaults to s%global%default_branch.
  {which} is one of: "model", "base" or "design"
  {ref_ele} is an element name or index.
\end{verbatim}

%% python super_universe ------------------------------------
\subsection{python super_universe}
\index{python!super_universe}
\label{p:super.universe}


Output super_Universe parameters.

\begin{example}
   python super_universe
\end{example}
\begin{verbatim}

\end{verbatim}

%% python twiss_at_s ------------------------------------
\subsection{python twiss_at_s}
\index{python!twiss_at_s}
\label{p:twiss.at.s}


Output twiss parameters at given s position.

\begin{example}
   python twiss_at_s \{ix_uni\}@\{ele\}->\{s_offset\}|\{which\}
\end{example}
\begin{verbatim}
Where:
  {ix_uni} is a universe index. Defaults to s%global%default_universe.
  {ele} is an element name or index. Default at the Beginning element at start of branch 0.
  {s_offset} is the offset of the evaluation point from the downstream end of ele. Default is 0.
     If {s_offset} is present, the preceeding "->" sign must be present. EG: Something like "23|model" will
  {which} is one of: "model", "base" or "design".
\end{verbatim}

%% python universe ------------------------------------
\subsection{python universe}
\index{python!universe}
\label{p:universe}


Output universe info.

\begin{example}
   python universe \{ix_uni\}
\end{example}
\begin{verbatim}
Use "python global" to get the number of universes.
\end{verbatim}

%% python var ------------------------------------
\subsection{python var}
\index{python!var}
\label{p:var}


Output parameters of a given variable.

\begin{example}
   python var \{var\} slaves
\end{example}
\begin{verbatim}

\end{verbatim}

%% python var_create ------------------------------------
\subsection{python var_create}
\index{python!var_create}
\label{p:var.create}


Create a single variable

\begin{example}
   python var_create \{var_name\}^^\{ele_name\}^^\{attribute\}^^\{universes\}^^
                     \{weight\}^^\{step\}^^\{low_lim\}^^\{high_lim\}^^\{merit_type\}^^
                     \{good_user\}^^\{key_bound\}^^\{key_delta\}
\end{example}
\begin{verbatim}
{var_name} is something like "kick[5]".
Before using var_create, setup the appropriate v1_var array using 
the "python var_v1_create" command.
\end{verbatim}

%% python var_general ------------------------------------
\subsection{python var_general}
\index{python!var_general}
\label{p:var.general}


Output list of all variable v1 arrays

\begin{example}
   python var_general
\end{example}
\begin{verbatim}
Output syntax:
  {v1_var name};{v1_var%v lower bound};{v1_var%v upper bound}
\end{verbatim}

%% python var_v_array ------------------------------------
\subsection{python var_v_array}
\index{python!var_v_array}
\label{p:var.v.array}


Output list of variables for a given data_v1.

\begin{example}
   python var_v_array \{v1_var\}
\end{example}
\begin{verbatim}
Example:
  python var_v_array quad_k1
\end{verbatim}

%% python var_v1_array ------------------------------------
\subsection{python var_v1_array}
\index{python!var_v1_array}
\label{p:var.v1.array}


Output list of variables in a given variable v1 array

\begin{example}
   python var_v1_array \{v1_var\}
\end{example}
\begin{verbatim}

\end{verbatim}

%% python var_v1_create ------------------------------------
\subsection{python var_v1_create}
\index{python!var_v1_create}
\label{p:var.v1.create}


Create a v1 variable structure along with associated var array.

\begin{example}
   python var_v1_create \{v1_name\} \{n_var_min\} \{n_var_max\}
\end{example}
\begin{verbatim}
{n_var_min} and {n_var_max} are the lower and upper bounds of the var
Example:
  python var_v1_create quad_k1 0 45
This example creates a v1 var structure called "quad_k1" with an associated
variable array that has the range [0, 45].

Use the "set variable" command to set a created variable parameters.
In particular, to slave a lattice parameter to a variable use the command:
  set {v1_name}|ele_name = {lat_param}
where {lat_param} is of the form {ix_uni}@{ele_name_or_location}{param_name}]
Examples:
  set quad_k1[2]|ele_name = 2@q01w[k1]
  set quad_k1[2]|ele_name = 2@0>>10[k1]
Note: When setting multiple variable parameters, 
      temporarily toggle s%global%lattice_calc_on to False
  ("set global lattice_calc_on = F") to prevent Tao trying to evaluate the 
partially created variable and generating unwanted error messages.
\end{verbatim}

%% python var_v1_destroy ------------------------------------
\subsection{python var_v1_destroy}
\index{python!var_v1_destroy}
\label{p:var.v1.destroy}


Destroy a v1 var structure along with associated var sub-array.

\begin{example}
   python var_v1_destroy \{v1_datum\}
\end{example}
\begin{verbatim}

\end{verbatim}

%% python wave ------------------------------------
\subsection{python wave}
\index{python!wave}
\label{p:wave}


Output Wave analysis info.

\begin{example}
   python wave \{who\}
\end{example}
\begin{verbatim}
Where {who} is one of:
  params
  loc_header
  locations
  plot1, plot2, plot3
\end{verbatim}
