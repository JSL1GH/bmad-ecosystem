% WARNING: this is automatically generated. DO NOT EDIT.
\begin{description}
\item[beam] \Newline\begin{example}
Output beam parameters that are not in the beam_init structure.

Notes
-----
Command syntax:
  python beam {ix_universe}
where
  {ix_universe} is a universe index.
To set beam_init parameters use the "set beam" command


Parameters
----------
ix_universe : default=1

   
Returns
-------
string_list 


Examples
-------- 

Example: 1
 init: -init $ACC_ROOT_DIR/regression_tests/python_test/csr_beam_tracking/tao.init
 args:
   ix_universe: 1

\end{example}
\item[beam_init] \Newline\begin{example}
Output beam_init parameters.

Notes
-----
Command syntax:
  python beam_init {ix_universe}
where
  {ix_universe} is a universe index.
To set beam_init parameters use the "set beam_init" command


Parameters
----------
ix_universe : default=1

   
Returns
-------
string_list 


Examples
-------- 

Example: 1
 init: -init $ACC_ROOT_DIR/regression_tests/python_test/csr_beam_tracking/tao.init
 args:
   ix_universe: 1

\end{example}
\item[bmad_com] \Newline\begin{example}
Bmad_com structure components

Notes
-----
Command syntax:
  python bmad_com

Parameters
----------

   
Returns
-------
string_list 


Examples
-------- 

Example: 1
 init: -init $ACC_ROOT_DIR/regression_tests/python_test/cesr/tao.init
 args:

\end{example}
\item[branch1] \Newline\begin{example}
Lattice element list.

Notes
-----
Command syntax:
  python branch1 {ix_universe}@{ix_branch}
where
  {ix_universe} is a universe index
  {ix_branch} is a lattice branch index


Parameters
----------
ix_universe : default=1
ix_branch : default=0

   
Returns
-------
string_list


Examples
--------

Example: 1
 init: -init $ACC_ROOT_DIR/regression_tests/python_test/cesr/tao.init
 args:
   ix_universe: 1
   ix_branch: 0

\end{example}
\item[bunch_params] \Newline\begin{example}
Bunch parameters at the exit end of a given lattice element.

Notes
-----
Command syntax:
python bunch_params {ele_id}|{which}

Parameters
----------
ele_id
  Element name or index
which : default=model
  One of: "model", "base" or "design"

Returns
-------
string_list 

Examples
--------

Example: 1
 init: -init $ACC_ROOT_DIR/regression_tests/python_test/csr_beam_tracking/tao.init
 args:
   ele_id: end
   which: model


\end{example}
\item[bunch1] \Newline\begin{example}
Bunch parameters at the exit end of a given lattice element.

Notes
-----
Command syntax:
python bunch1 {ele_id}|{which} {ix_bunch} {coordinate}

Parameters
----------
ele_id
  Element name or index
coordinate
  If one of: x, px, y, py, z, pz, 's', 't', 'charge', 'p0c', 'state'
which : default=model
  One of: "model", "base" or "design"
ix_bunch : default=1

 
Returns
-------
real_array
  if coordinate != 'state'
integer_array
  if coordinate == 'state'


Examples
--------

Example: 1
 init: -init $ACC_ROOT_DIR/regression_tests/python_test/csr_beam_tracking/tao.init
 args:
   ele_id: end
   coordinate: x
   which: model
   ix_bunch: 1
  


\end{example}
\item[building_wall_list] \Newline\begin{example}
List of building wall sections or section points

Notes
-----
Command syntax:
  python building_wall_list {ix_section}
If {ix_section} is not present then a list of building wall sections is given.
If {ix_section} is present then a list of section points is given


Parameters
----------
ix_section : optional

   
Returns
-------
string_list 


Examples
--------

Example: 1
 init: -init $ACC_ROOT_DIR/regression_tests/python_test/tao.init_wall
 args:
   ix_section:

Example: 2
 init: -init $ACC_ROOT_DIR/regression_tests/python_test/tao.init_wall
 args:
   ix_section: 1

\end{example}
\item[building_wall_graph] \Newline\begin{example}
(x, y) points for drawing the building wall for a particular graph.

Notes
-----
The graph defines the coordinate system for the (x, y) points.
Command syntax:
  python building_wall_graph {graph}


Parameters
----------
graph

   
Returns
-------
string_list 


Examples
--------

Example: 1
 init: -init $ACC_ROOT_DIR/regression_tests/python_test/tao.init_wall
 args:
   graph: floor_plan.g

\end{example}
\item[building_wall_point] \Newline\begin{example}
add or delete a building wall point

Notes
-----
Command syntax:
  python building_wall_point {ix_section}^^{ix_point}^^{z}^^{x}^^{radius}^^{z_center}^^{x_center}
Where:
  {ix_section}    -- Section index.
  {ix_point}      -- Point index. Points of higher indexes will be moved up 
                       if adding a point and down if deleting.
  {z}, etc...     -- See tao_building_wall_point_struct components.
                  -- If {z} is set to "delete" then delete the point.


Parameters
----------
ix_section
ix_point
z
x
radius
z_center
x_center

   
Returns
-------
None 


Examples
--------

Example: 1
 init: -init $ACC_ROOT_DIR/regression_tests/python_test/tao.init_wall
 args:
   ix_section: 1
   ix_point: 1
   z: 0
   x: 0
   radius: 0
   z_center: 0
   x_center: 0

\end{example}
\item[building_wall_section] \Newline\begin{example}
add or delete a building wall section

Notes
-----
Command syntax:
  python building_wall_section {ix_section}^^{sec_name}^^{sec_constraint}
Where:
  {ix_section}      -- Section index. Sections with higher indexes will be
                         moved up if adding a section and down if deleting.
  {sec_name}        -- Section name.
  {sec_constraint}  -- Must be one of:
      delete     -- Delete section. Anything else will add the section.
      none
      left_side
      right_side


Parameters
----------
ix_section
sec_name
sec_constraint

   
Returns
-------
None


Examples
--------

Example: 1
 init: -init $ACC_ROOT_DIR/regression_tests/python_test/cesr/tao.init
 args:
   ix_section: 1
   sec_name: test
   sec_constraint: none

\end{example}
\item[constraints] \Newline\begin{example}
Optimization data and variables that contribute to the merit function.

Notes
-----
Command syntax:
  python constraints {who}
{who} is one of:
  data
  var
Data constraints output is:
  data name
  constraint type
  evaluation element name
  start element name
  end/reference element name
  measured value
  ref value (only relavent if global%opt_with_ref = T)
  model value
  base value (only relavent if global%opt_with_base = T)
  weight
  merit value
  location where merit is evaluated (if there is a range)
Var constraints output is:
  var name
  Associated varible attribute
  meas value
  ref value (only relavent if global%opt_with_ref = T)
  model value
  base value (only relavent if global%opt_with_base = T)
  weight
  merit value
  dmerit/dvar


Parameters
----------
who

   
Returns
-------
string_list 


Examples
--------

Example: 1
 init: -init $ACC_ROOT_DIR/regression_tests/python_test/tao.init_optics_matching
 args:
   who: data

Example: 2
 init: -init $ACC_ROOT_DIR/regression_tests/python_test/cesr/tao.init
 args:
   who:var

\end{example}
\item[da_aperture] \Newline\begin{example}
Dynamic aperture data

Notes
-----
Command syntax:
  python da_aperture {ix_uni}


Parameters
----------
ix_uni : default=1

   
Returns
-------
string_list


Examples
--------

\end{example}
\item[da_params] \Newline\begin{example}
Dynamic aperture input parameters

Notes
-----
Command syntax:
  python da_params {ix_uni}


Parameters
----------
ix_uni : default=1

   
Returns
-------
string_list


Examples
--------

\end{example}
\item[data] \Newline\begin{example}
Individual datum info.

Notes
-----
Command syntax:
  python data {ix_universe}@{d2_name}.{d1_datum}[{dat_index}]
Use the "python data-d1" command to get detailed info on a specific d1 array.
Output syntax is parameter list form. See documentation at the beginning of this file.
Example:
  python data 1@orbit.x[10]
Note : By default dat_index is 1.


Parameters
----------
d2_name
d1_datum
ix_universe : default=1
dat_index : default=1

   
Returns
-------
string_list


Examples
--------

Example: 1
 init: -init $ACC_ROOT_DIR/regression_tests/python_test/tao.init_optics_matching
 args:
   ix_universe:
   d2_name: twiss
   d1_datum: end 
   dat_index: 1  

Example: 2
 init: -init $ACC_ROOT_DIR/regression_tests/python_test/tao.init_optics_matching
 args:
   ix_universe: 1
   d2_name: twiss
   d1_datum: end
   dat_index: 1

\end{example}
\item[data_d2_create] \Newline\begin{example}
Create a d2 data structure along with associated d1 and data arrays.

Notes
-----
Command syntax:
  python data_d2_create {d2_name}^^{n_d1_data}^^{d_data_arrays_name_min_max}
{d2_name} should be of the form {ix_uni}@{d2_datum_name}
{n_d1_data} is the number of associated d1 data structures.
{d_data_arrays_name_min_max} has the form
  {name1}^^{lower_bound1}^^{upper_bound1}^^....^^{nameN}^^{lower_boundN}^^{upper_boundN}
where {name} is the data array name and {lower_bound} and {upper_bound} are the bounds 
of the array.

Example:
  python data_d2_create 2@orbit^^2^^x^^0^^45^^y^^1^^47
This example creates a d2 data structure called "orbit" with 
two d1 structures called "x" and "y".

The "x" d1 structure has an associated data array with indexes in the range [0, 45].
The "y" d1 structure has an associated data arrray with indexes in the range [1, 47].

Use the "set data" command to set a created datum parameters.
Note: When setting multiple data parameters, 
      temporarily toggle s%global%lattice_calc_on to False
  ("set global lattice_calc_on = F") to prevent Tao trying to 
      evaluate the partially created datum and generating unwanted error messages.


Parameters
----------

d2_name
n_d1_data
d_data_arrays_name_min_max
ix_uni : default=1
   
Returns
-------
None


Examples
--------

Example: 1
 init: -init $ACC_ROOT_DIR/regression_tests/python_test/tao.init_optics_matching
 args:
   ix_uni: 1
   d2_name: orbit
   n_d1_data: 2 
   d_data_arrays_name_min_max: x^^0^^45^^y^^1^^47

\end{example}
\item[data_d2_destroy] \Newline\begin{example}
Destroy a d2 data structure along with associated d1 and data arrays.

Notes
-----
Command syntax:
  python data_d2_destroy {d2_datum}
{d2_datum} should be of the form
  {ix_uni}@{d2_datum_name}


Parameters
----------
d2_datum
ix_uni : default=1
   
Returns
-------
None


Examples
--------

Example: 1
 init: -init $ACC_ROOT_DIR/regression_tests/python_test/cesr/tao.init
 args:
   d2_datum: 1@eta.x

\end{example}
\item[data_d2] \Newline\begin{example}
Information on a d2_datum.

Notes
-----
Command syntax:
  python data_d2 {d2_datum}
{d2_datum} should be of the form
  {ix_uni}@{d2_datum_name}


Parameters
----------
d2_datum
ix_uni : default=1

   
Returns
-------
string_list 


Examples
--------

Example: 1
 init: -init $ACC_ROOT_DIR/regression_tests/python_test/tao.init_optics_matching
 args:
   ix_uni: 1
   d2_datum: twiss

\end{example}
\item[data_d_array] \Newline\begin{example}
List of datums for a given data_d1.

Notes
-----
Command syntax:
  python data_d_array {d1_datum}
{d1_datum} should be for the form
  {ix_uni}@{d2_datum_name}.{d1_datum_name}
Example:
  python data_d_array 1@orbit.x


Parameters
----------
d1_datum
ix_uni : default=1

   
Returns
-------
string_list 


Examples
--------

Example: 1
 init: -init $ACC_ROOT_DIR/regression_tests/python_test/tao.init_optics_matching
 args:
   ix_uni: 1 
   d1_datum: twiss.end

\end{example}
\item[data_d1_array] \Newline\begin{example}
List of d1 arrays for a given data_d2.

Notes
-----
Command syntax:
  python data_d1_array {d2_datum}
{d2_datum} should be of the form
  {ix_uni}@{d2_datum_name}


Parameters
----------
d2_datum
ix_uni : default=1

   
Returns
-------
string_list 


Examples
--------

Example: 1
 init: -init $ACC_ROOT_DIR/regression_tests/python_test/tao.init_optics_matching
 args:
   ix_uni: 1 
   d2_datum: twiss

\end{example}
\item[data_parameter] \Newline\begin{example}
Given an array of datums, generate an array of values for a particular datum parameter.

Notes
-----
Command syntax:
  python data_parameter {data_array} {parameter}
{parameter} may be any tao_data_struct parameter.
Example:
  python data_parameter orbit.x model_value

Parameters
----------
data_array
parameter

   
Returns
-------
string_list


Examples
--------

Example: 1
 init: -init $ACC_ROOT_DIR/regression_tests/python_test/tao.init_optics_matching
 args:
   data_array: twiss.end 
   parameter: model_value

\end{example}
\item[data_d2_array] \Newline\begin{example}
Data d2 info for a given universe.

Notes
-----
Command syntax:
  python data_d2_array {ix_universe}
Example:
  python data_d2_array 1


Parameters
----------
ix_universe

   
Returns
-------
string_list  


Examples
--------

Example: 1
 init: -init $ACC_ROOT_DIR/regression_tests/python_test/cesr/tao.init
 args:
   ix_universe : 1 

\end{example}
\item[data_set_design_value] \Newline\begin{example}
Set the design (and base & model) values for all datums.

Notes
-----
Command syntax:
  python data_set_design_value
Example:
  python data_set_design_value

Note: Use the "data_d2_create" and "datum_create" first to create datums.

Parameters
----------

   
Returns
-------
None


Examples
--------

Example: 1
 init: -init $ACC_ROOT_DIR/regression_tests/python_test/tao.init_optics_matching
 args:

\end{example}
\item[datum_create] \Newline\begin{example}
Create a datum.

Notes
-----
Command syntax:
  python datum_create {datum_name}^^{data_type}^^{ele_ref_name}^^{ele_start_name}^^
                      {ele_name}^^{merit_type}^^{meas}^^{good_meas}^^{ref}^^
                      {good_ref}^^{weight}^^{good_user}^^{data_source}^^
                      {eval_point}^^{s_offset}^^{ix_bunch}^^{invalid_value}^^
                      {spin_axis%n0(1)}^^{spin_axis%n0(2)}^^{spin_axis%n0(3)}^^
                      {spin_axis%l(1)}^^{spin_axis%l(2)}^^{spin_axis%l(3)}

Note: The 3 values for spin_axis%n0, as a group, are optional. 
      Also the 3 values for spin_axis%l are, as a group, optional.
Note: Use the "data_d2_create" first to create a d2 structure with associated d1 arrays.
Note: After creating all your datums, use the "data_set_design_value" routine to 
      set the design (and model) values.


Parameters
----------
datum_name          ! EG: orb.x[3]
data_type           ! EG: orbit.x
ele_ref_name : optional
ele_start_name : optional
ele_name : optional
merit_type : optional
meas : default=0
good_meas : default=F
ref : default=0
good_ref : default=F
weight : default=0
good_user : default=T
data_source : default=lat
eval_point : default=END
s_offset : default=0
ix_bunch : default=0
invalid_value : default=0
spin_axis%n0(1) : optional
spin_axis%n0(2) : optional
spin_axis%n0(3) : optional
spin_axis%l(1) : optional
spin_axis%l(2) : optional
spin_axis%l(3) : optional

   
Returns
-------
string_list


Examples
--------

Example: 1
 init: -init $ACC_ROOT_DIR/regression_tests/python_test/tao.init_optics_matching
 args:
   datum_name: twiss.end[6]
   data_type: beta.y
   ele_ref_name:
   ele_start_name:
   ele_name: P1
   merit_type: target
   meas: 0
   good_meas: T
   ref: 0
   good_ref: T
   weight: 0.3
   good_user: T
   data_source: lat
   eval_point: END
   s_offset: 0
   ix_bunch: 1
   invalid_value: 0

\end{example}
\item[datum_has_ele] \Newline\begin{example}
Does datum type have an associated lattice element?

Notes
-----
Command syntax:
  python datum_has_ele {datum_type}


Parameters
----------
datum_type

   
Returns
-------
string_list 


Examples
--------

Example: 1
 init: -init $ACC_ROOT_DIR/regression_tests/python_test/tao.init_optics_matching
 args:
   datum_type: twiss.end 

\end{example}
\item[derivative] \Newline\begin{example}
Optimization derivatives

Notes
-----
Command syntax:
  python derivative
Note: To save time, this command will not recalculate derivatives. 
Use the "derivative" command beforehand to recalcuate if needed.

Parameters
----------


Returns
-------
string_list


Examples
--------

Example: 1
 init: -init $ACC_ROOT_DIR/regression_tests/python_test/tao.init_optics_matching
 args:

\end{example}
\item[ele:head] \Newline\begin{example}
"Head" Element attributes

Notes
-----
Command syntax:
  python ele:head {ele_id}|{which}
where {ele_id} is an element name or index and {which} is one of
  model
  base
  design
Example:
  python ele:head 3@1>>7|model
This gives element number 7 in branch 1 of universe 3.


Parameters
----------
ele_id 
which : default=model

   
Returns
-------
string_list 


Examples
--------

Example: 1
 init: -init $ACC_ROOT_DIR/regression_tests/python_test/cesr/tao.init
 args:
  ele_id: 1@0>>1
  which: model

\end{example}
\item[ele:methods] \Newline\begin{example}
Element methods

Notes
-----
Command syntax:
  python ele:methods {ele_id}|{which}
where {ele_id} is an element name or index and {which} is one of
  model
  base
  design
Example:
  python ele:methods 3@1>>7|model
This gives element number 7 in branch 1 of universe 3.


Parameters
----------
ele_id
which : default=model


Returns
-------
string_list


Examples
--------

Example: 1
 init: -init $ACC_ROOT_DIR/regression_tests/python_test/cesr/tao.init
 args:
  ele_id: 1@0>>1
  which: model

\end{example}
\item[ele:gen_attribs] \Newline\begin{example}
Element general attributes

Notes
-----
Command syntax:
  python ele:gen_attribs {ele_id}|{which}
where {ele_id} is an element name or index and {which} is one of
  model
  base
  design
Example:
  python ele:gen_attribs 3@1>>7|model
This gives element number 7 in branch 1 of universe 3.


Parameters
----------
ele_id
which : default=model


Returns
-------
string_list


Examples
--------

Example: 1
 init: -init $ACC_ROOT_DIR/regression_tests/python_test/cesr/tao.init
 args:
  ele_id: 1@0>>1
  which: model

\end{example}
\item[ele:multipoles] \Newline\begin{example}
Element multipoles

Notes
-----
Command syntax:
  python ele:multipoles {ele_id}|{which}
where {ele_id} is an element name or index and {which} is one of
  model
  base
  design
Example:
  python ele:multipoles 3@1>>7|model
This gives element number 7 in branch 1 of universe 3.


Parameters
----------
ele_id
which : default=model


Returns
-------
string_list


Examples
--------

Example: 1
 init: -init $ACC_ROOT_DIR/regression_tests/python_test/cesr/tao.init
 args:
  ele_id: 1@0>>1
  which: model

\end{example}
\item[ele:ac_kicker] \Newline\begin{example}
Element ac_kicker

Notes
-----
Command syntax:
  python ele:ac_kicker {ele_id}|{which}
where {ele_id} is an element name or index and {which} is one of
  model
  base
  design
Example:
  python ele:ac_kicker 3@1>>7|model
This gives element number 7 in branch 1 of universe 3.


Parameters
----------
ele_id
which : default=model


Returns
-------
string_list


Examples
--------

Example: 1
 init: -init $ACC_ROOT_DIR/regression_tests/python_test/cesr/tao.init
 args:
  ele_id: 1@0>>1
  which: model

\end{example}
\item[ele:cartesian_map] \Newline\begin{example}
Element cartesian_map

Notes
-----
Command syntax:
  python ele:cartesian_map {ele_id}|{which} {index} {who}
where {ele_id} is an element name or index and {which} is one of
  model
  base
  design
{index} is the index number in the ele%cartesian_map(:) array
{who} is one of:
  base
  terms
Example:
  python ele:cartesian_map 3@1>>7|model 2 base
This gives element number 7 in branch 1 of universe 3.


Parameters
----------
ele_id
index 
who
which : default=model

Returns
-------
string_list


Examples
--------

Example: 1
 init: -init $ACC_ROOT_DIR/regression_tests/python_test/tao.init_em_field
 args:
  ele_id: 1@0>>1
  which: model
  index: 1
  who: base

\end{example}
\item[ele:chamber_wall] \Newline\begin{example}
Element beam chamber wall

Notes
-----
Command syntax:
  python ele:chamber_wall {ele_id}|{which} {index} {who}
where {ele_id} is an element name or index and {which} is one of
  model
  base
  design
{index} is index of the wall.
{who} is one of:
  x       ! Return min/max in horizontal plane
  y       ! Return min/max in vertical plane


Parameters
----------
ele_id
index
who
which : default=model

   
Returns
-------
string_list 


Examples
--------

Example: 1
 init: -init $ACC_ROOT_DIR/regression_tests/python_test/tao.init_wall3d
 args:
  ele_id: 1@0>>1
  which: model
  index: 1
  who: x

\end{example}
\item[ele:cylindrical_map] \Newline\begin{example}
Element cylindrical_map

Notes
-----
Command syntax:
  python ele:cylindrical_map {ele_id}|{which} {index} {who}
where {ele_id} is an element name or index and {which} is one of
  model
  base
  design
{index} is the index number in the ele%cylindrical_map(:) array
{who} is one of:
  base
  terms
Example:
  python ele:cylindrical_map 3@1>>7|model 2 base
This gives map #2 of element number 7 in branch 1 of universe 3.


Parameters
----------
ele_id
index
who
which : default=model

   
Returns
-------
string_list


Examples
--------

Example: 1
 init: -init $ACC_ROOT_DIR/regression_tests/python_test/tao.init_em_field
 args:
  ele_id: 1@0>>5
  which: model
  index: 1
  who: base

\end{example}
\item[ele:taylor] \Newline\begin{example}
Element taylor

Notes
-----
Command syntax:
  python ele:taylor {ele_id}|{which}
where {ele_id} is an element name or index and {which} is one of
  model
  base
  design
Example:
  python ele:taylor 3@1>>7|model
This gives element number 7 in branch 1 of universe 3.


Parameters
----------
ele_id
which : default=model

   
Returns
-------
string_list


Examples
--------

Example: 1
 init: -init $ACC_ROOT_DIR/regression_tests/python_test/tao.init_taylor
 args:
  ele_id: 1@0>>34
  which: model

\end{example}
\item[ele:spin_taylor] \Newline\begin{example}
Element spin_taylor

Notes
-----
Command syntax:
  python ele:spin_taylor {ele_id}|{which}
where {ele_id} is an element name or index and {which} is one of
  model
  base
  design
Example:
  python ele:spin_taylor 3@1>>7|model
This gives element number 7 in branch 1 of universe 3.


Parameters
----------
ele_id
which : default=model

   
Returns
-------
string_list


Examples
--------

Example: 1
 init: -init $ACC_ROOT_DIR/regression_tests/python_test/tao.init_spin
 args:
  ele_id: 1@0>>2
  which: model

\end{example}
\item[ele:wake] \Newline\begin{example}
Element wake

Notes
-----
Command syntax:
  python ele:wake {ele_id}|{which} {who}
where {ele_id} is an element name or index and {which} is one of
  model
  base
  design
{Who} is one of
  base
  sr_long     sr_long_table
  sr_trans    sr_trans_table
  lr_mode_table
Example:
  python ele:wake 3@1>>7|model
This gives element number 7 in branch 1 of universe 3.


Parameters
----------
ele_id
which : default=model
who : default=base

   
Returns
-------
string_list


Examples
--------

Example: 1
 init: -init $ACC_ROOT_DIR/regression_tests/python_test/tao.init_wake
 args:
  ele_id: 1@0>>1
  which: model
  who: sr_long

\end{example}
\item[ele:wall3d] \Newline\begin{example}
Element wall3d

Notes
-----
Command syntax:
  python ele:wall3d {ele_id}|{which} {index} {who}
where {ele_id} is an element name or index and {which} is one of
  model
  base
  design
{index} is the index number in the ele%wall3d(:) array (size obtained from "ele:head").
{who} is one of:
  base
  table
Example:
  python ele:wall3d 3@1>>7|model 2 base
This gives element number 7 in branch 1 of universe 3.


Parameters
----------
ele_id
which : default=model
index : default=1
who : default=base

   
Returns
-------
string_list


Examples
--------

Example: 1
 init: -init $ACC_ROOT_DIR/regression_tests/python_test/tao.init_wall3d
 args:
  ele_id: 1@0>>1
  which: model
  index: 1
  who: table

\end{example}
\item[ele:twiss] \Newline\begin{example}
Element twiss

Notes
-----
Command syntax:
  python ele:twiss {ele_id}|{which}
where {ele_id} is an element name or index and {which} is one of
  model
  base
  design
Example:
  python ele:twiss 3@1>>7|model
This gives element number 7 in branch 1 of universe 3.


Parameters
----------
ele_id
which : default=model

   
Returns
-------
string_list


Examples
--------

Example: 1
 init: -init $ACC_ROOT_DIR/regression_tests/python_test/cesr/tao.init
 args:
  ele_id: 1@0>>1
  which: model

\end{example}
\item[ele:control_var] \Newline\begin{example}
List element control variables.
Used for group, overlay and ramper type elements

Notes
-----
Command syntax:
  python ele:control_var {ele_id}|{which}
where {ele_id} is an element name or index and {which} is one of
  model
  base
  design
Example:
  python ele:control_var 3@1>>7|model
This gives element number 7 in branch 1 of universe 3.


Parameters
----------
ele_id
which : default=model

   
Returns
-------
string_list


Examples
--------

Example: 1
 init: -init $ACC_ROOT_DIR/regression_tests/python_test/cesr/tao.init
 args:
  ele_id: 1@0>>873
  which: model

\end{example}
\item[ele:orbit] \Newline\begin{example}
Element orbit

Notes
-----
Command syntax:
  python ele:orbit {ele_id}|{which}
where {ele_id} is an element name or index and {which} is one of
  model
  base
  design
Example:
  python ele:orbit 3@1>>7|model
This gives element number 7 in branch 1 of universe 3.


Parameters
----------
ele_id
which : default=model

   
Returns
-------
string_list


Examples
--------

Example: 1
 init: -init $ACC_ROOT_DIR/regression_tests/python_test/cesr/tao.init
 args:
  ele_id: 1@0>>1
  which: model

\end{example}
\item[ele:mat6] \Newline\begin{example}
Element mat6

Notes
-----
Command syntax:
  python ele:mat6 {ele_id}|{which} {who}
where {ele_id} is an element name or index and {which} is one of
  model
  base
  design
{who} is one of:
  mat6
  vec0
  err
Example:
  python ele:mat6 3@1>>7|model mat6
This gives element number 7 in branch 1 of universe 3.


Parameters
----------
ele_id
which : default=model
who : default=mat6

   
Returns
-------
string_list


Examples
--------

Example: 1
 init: -init $ACC_ROOT_DIR/regression_tests/python_test/cesr/tao.init
 args:
  ele_id: 1@0>>1
  which: model
  who: mat6

\end{example}
\item[ele:taylor_field] \Newline\begin{example}
Element taylor_field

Notes
-----
Command syntax:
  python ele:taylor_field {ele_id}|{which} {index} {who}
where {ele_id} is an element name or index and {which} is one of
  model
  base
  design
{index} is the index number in the ele%taylor_field(:) array
{who} is one of:
  base
  terms
Example:
  python ele:taylor_field 3@1>>7|model 2 base
This gives element number 7 in branch 1 of universe 3.


Parameters
----------
ele_id
index
who
which : default=model

   
Returns
-------
string_list


Examples
--------

Example: 1
 init: -init $ACC_ROOT_DIR/regression_tests/python_test/tao.init_em_field
 args:
  ele_id: 1@0>>9
  which: model
  index: 1
  who: terms

\end{example}
\item[ele:grid_field] \Newline\begin{example}
Element grid_field

Notes
-----
Command syntax:
  python ele:grid_field {ele_id}|{which} {index} {who}
where {ele_id} is an element name or index and {which} is one of
  model, base, design
{index} is the index number in the ele%grid_field(:) array.
{who} is one of:
  base, points
Example:
  python ele:grid_field 3@1>>7|model 2 base
This gives grid #2 of element number 7 in branch 1 of universe 3.


Parameters
----------
ele_id
which : default=model
index : default=1
who : default=base
   
Returns
-------
string_list


Examples
--------

Example: 1
 init: -init $ACC_ROOT_DIR/regression_tests/python_test/tao.init_grid
 args:
  ele_id: 1@0>>1
  which: model
  index: 1
  who: base 

\end{example}
\item[ele:floor] \Newline\begin{example}
Element floor coordinates. The output gives two lines. "Reference" is
without element misalignments and "Actual" is with misalignments.

Notes
-----
Command syntax:
  python ele:floor {ele_id}|{which} {where}
where {ele_id} is an element name or index and {which} is one of
  model
  base
  design
{where} is an optional argument which, if present, is one of
  beginning  ! Upstream end
  center     ! Middle of element
  end        ! Downstream end (default)
Note: {where} ignored for photonic elements crystal, mirror, and multilayer_mirror.
Example:
  python ele:floor 3@1>>7|model
This gives element number 7 in branch 1 of universe 3.


Parameters
----------
ele_id
which : default=model
where : default=end

   
Returns
-------
string_list


Examples
--------

Example: 1
 init: -init $ACC_ROOT_DIR/regression_tests/python_test/cesr/tao.init
 args:
  ele_id: 1@0>>1
  which: model
  where: 

Example: 2
 init: -init $ACC_ROOT_DIR/regression_tests/python_test/cesr/tao.init
 args:
  ele_id: 1@0>>1
  which: model
  where: center

\end{example}
\item[ele:photon] \Newline\begin{example}
Element photon

Notes
-----
Command syntax:
  python ele:photon {ele_id}|{which} {who}
where {ele_id} is an element name or index and {which} is one of
  model
  base
  design
{who} is one of:
  base
  material
  surface
Example:
  python ele:photon 3@1>>7|model base
This gives element number 7 in branch 1 of universe 3.


Parameters
----------
ele_id
which : default=model
who : default=base

   
Returns
-------
string_list


Examples
--------

Example: 1
 init: -init $ACC_ROOT_DIR/regression_tests/python_test/tao.init_photon
 args:
  ele_id: 1@0>>1
  which: model
  who: base

\end{example}
\item[ele:lord_slave] \Newline\begin{example}
Lists the lord/slave tree of an element.

Notes
-----
Command syntax:
  python ele:lord_slave {ele_id}|{which}
where {ele_id} is an element name or index and {which} is one of
  model
  base
  design
Example:
  python ele:lord_slave 3@1>>7|model
This gives lord and slave info on element number 7 in branch 1 of universe 3.
Note: The lord/slave info is independent of the setting of {which}.

The output is a number of lines, each line giving information on an element (element index, etc.).
Some lines begin with the word "Element". 
After each "Element" line, there are a number of lines (possibly zero) that begin with the word "Slave or "Lord".
These "Slave" and "Lord" lines are the slaves and lords of the "Element" element.


Parameters
----------
ele_id
which : default=model

   
Returns
-------
string_list


Examples
--------

Example: 1
 init: -init $ACC_ROOT_DIR/regression_tests/python_test/cesr/tao.init
 args:
  ele_id: 1@0>>1
  which: model

\end{example}
\item[ele:elec_multipoles] \Newline\begin{example}
Element electric multipoles

Notes
-----
Command syntax:
  python ele:elec_multipoles {ele_id}|{which}
where {ele_id} is an element name or index and {which} is one of
  model
  base
  design
Example:
  python ele:elec_multipoles 3@1>>7|model
This gives element number 7 in branch 1 of universe 3.


Parameters
----------
ele_id
which : default=model

   
Returns
-------
string_list


Examples
--------

Example: 1
 init: -init $ACC_ROOT_DIR/regression_tests/python_test/cesr/tao.init
 args:
  ele_id: 1@0>>1
  which: model

\end{example}
\item[evaluate] \Newline\begin{example}
Evaluate an expression. The result may be a vector.

Notes
-----
Command syntax:
  python evaluate {flags} {expression}

Example:
  python evaluate data::cbar.11[1:10]|model


Parameters
----------
expression
flags : default=-array_out
  If -array_out, the output will be available in the tao_c_interface_com%c_real.!

Returns
-------
string_list
  if '-array_out' not in flags
real_array
  if '-array_out' in flags

Examples
--------

Example: 1
 init: -init $ACC_ROOT_DIR/regression_tests/python_test/cesr/tao.init
 args:
   expression: data::cbar.11[1:10]|model

\end{example}
\item[em_field] \Newline\begin{example}
EM field at a given point generated by a given element.

Notes
-----
Command syntax:
  python em_field {ele_id}|{which} {x} {y} {z} {t_or_z}
where {which} is one of:
  model
  base
  design
Where:
  {x}, {y}  -- Transverse coords.
  {z}       -- Longitudinal coord with respect to entrance end of element.
  {t_or_z}  -- time or phase space z depending if lattice is setup for absolute time tracking.


Parameters
----------
ele_id
x
y
z
t_or_z
which : default=model
   
Returns
-------
string_list


Examples
--------

Example: 1
 init: -init $ACC_ROOT_DIR/regression_tests/python_test/cesr/tao.init
 args:
   ele_id: 1@0>>22
   which: model
   x: 0
   y: 0
   z: 0
   t_or_z: 0

\end{example}
\item[enum] \Newline\begin{example}
List of possible values for enumerated numbers.

Notes
-----
Command syntax:
  python enum {enum_name}
Example:
  python enum tracking_method


Parameters
----------
enum_name

   
Returns
-------
string_list


Examples
--------

Example: 1
 init: -init $ACC_ROOT_DIR/regression_tests/python_test/cesr/tao.init
 args:
   enum_name: tracking_method

\end{example}
\item[floor_plan] \Newline\begin{example}
Floor plan elements

Notes
-----
Command syntax:
  python floor_plan {graph}


Parameters
----------
graph

   
Returns
-------
string_list


Examples
--------

Example: 1
 init: -init $ACC_ROOT_DIR/regression_tests/python_test/tao.init_optics_matching
 args:
   graph: r13.g

\end{example}
\item[floor_orbit] \Newline\begin{example}
(x, y) coordinates for drawing the particle orbit on a floor plan.

Notes
-----
Command syntax:
  python floor_orbit {graph}


Parameters
----------
graph

   
Returns
-------
string_list


Examples
--------

Example: 1
 init: -init $ACC_ROOT_DIR/regression_tests/python_test/tao.init_floor_orbit
 args:
   graph: r33.g 

\end{example}
\item[global] \Newline\begin{example}
Global parameters

Notes
-----
Command syntax:
  python global
Output syntax is parameter list form. See documentation at the beginning of this file.

Note: The follow is intentionally left out:
  optimizer_allow_user_abort
  quiet
  single_step
  prompt_color
  prompt_string


Parameters
----------


Returns
-------
string_list


Examples
--------

Example: 1
 init: -init $ACC_ROOT_DIR/regression_tests/python_test/cesr/tao.init
 args:

\end{example}
\item[help] \Newline\begin{example}
returns list of "help xxx" topics

Notes
-----
Command syntax:
  python help


Parameters
----------


Returns
-------
string_list


Examples
--------

Example: 1
 init: -init $ACC_ROOT_DIR/regression_tests/python_test/cesr/tao.init
 args:

\end{example}
\item[inum] \Newline\begin{example}
INUM

Notes
-----
Command syntax:
  python inum {who}


Parameters
----------
who

   
Returns
-------
string_list


Examples
--------

Example: 1
 init: -init $ACC_ROOT_DIR/regression_tests/python_test/cesr/tao.init
 args:
   who: ix_universe

\end{example}
\item[lat_calc_done] \Newline\begin{example}
Check if a lattice recalculation has been proformed since the last time
  "python lat_calc_done" was called.

Notes
-----
Command syntax:
  python lat_calc_done


Parameters
----------
branch_name


Returns
-------
string_list


Examples
--------

Example: 1
 init: -init $ACC_ROOT_DIR/regression_tests/python_test/cesr/tao.init
 args:
   branch_name: 1@0

\end{example}
\item[lat_ele_list] \Newline\begin{example}
Lattice element list.

Notes
-----
Command syntax:
  python lat_ele_list {branch_name}
{branch_name} should have the form:
  {ix_uni}@{ix_branch}


Parameters
----------
branch_name : default=0

   
Returns
-------
string_list


Examples
--------

Example: 1
 init: -init $ACC_ROOT_DIR/regression_tests/python_test/cesr/tao.init
 args:
   branch_name: 1@0

\end{example}
\item[lat_general] \Newline\begin{example}
Lattice general

Notes
-----
Command syntax:
  python lat_general {ix_universe}

Output syntax:
  branch_index;branch_name;n_ele_track;n_ele_max


Parameters
----------
ix_universe : default=1

   
Returns
-------
string_list


Examples
--------

Example: 1
 init: -init $ACC_ROOT_DIR/regression_tests/python_test/cesr/tao.init
 args:
   ix_universe: 1

\end{example}
\item[lat_list] \Newline\begin{example}
List of parameters at ends of lattice elements

Notes
-----
Command syntax:
  python lat_list {flags} {ix_uni}@{ix_branch}>>{elements}|{which} {who}
where:
 Optional {flags} are:
  -no_slaves : If present, multipass_slave and super_slave elements will not be matched to.
  -track_only : If present, lord elements will not be matched to.
  -index_order : If present, order elements by element index instead of the standard s-position.
  -array_out : If present, the output will be available in the tao_c_interface_com%c_real or
    tao_c_interface_com%c_integer arrays. See the code below for when %c_real vs %c_integer is used.
    Note: Only a single {who} item permitted when -array_out is present.

  {which} is one of:
    model
    base
    design

  {who} is a comma deliminated list of:
    orbit.floor.x, orbit.floor.y, orbit.floor.z    ! Floor coords at particle orbit.
    orbit.spin.1, orbit.spin.2, orbit.spin.3,
    orbit.vec.1, orbit.vec.2, orbit.vec.3, orbit.vec.4, orbit.vec.5, orbit.vec.6,
    orbit.t, orbit.beta,
    orbit.state,     ! Note: state is an integer. alive$ = 1, anything else is lost.
    orbit.energy, orbit.pc,
    ele.name, ele.ix_ele, ele.ix_branch
    ele.a.beta, ele.a.alpha, ele.a.eta, ele.a.etap, ele.a.gamma, ele.a.phi,
    ele.b.beta, ele.b.alpha, ele.b.eta, ele.b.etap, ele.b.gamma, ele.b.phi,
    ele.x.eta, ele.x.etap,
    ele.y.eta, ele.y.etap,
    ele.s, ele.l
    ele.e_tot, ele.p0c
    ele.mat6, ele.vec0
    ele.{attribute} Where {attribute} is a Bmad syntax element attribute. (ele.beta_a, etc.)

  {elements} is a string to match element names to.
    Use "*" to match to all elements.

Examples:
  python lat_list -track 3@0>>Q*|base ele.s,orbit.vec.2
  python lat_list 3@0>>Q*|base real:ele.s    

Note: vector layout of mat6(6,6) is: [mat6(1,:), mat6(2,:), ...mat6(6,:)]

Parameters
----------
elements 
who 
ix_uni : default=1
ix_branch : default=0
which : default=model
flags : optional, default=-array_out -track_only

Returns
-------
string_list
  if ('-array_out' not in flags) or (who in ['ele.name'])
real_array
   if ('-array_out' in flags or 'real:' in who) and (who not in ['orbit.state'])
integer_array
   if '-array_out' in flags and who in ['orbit.state']

Examples
--------

Example: 1
 init: -init $ACC_ROOT_DIR/regression_tests/python_test/cesr/tao.init
 args:
   ix_uni: 1  
   ix_branch: 0 
   elements: Q* 
   which: model
   who: orbit.floor.x



\end{example}
\item[lat_param_units] \Newline\begin{example}
Units of a parameter associated with a lattice or lattice element.

Notes
-----
Command syntax:
  python lat_param_units {param_name}


Parameters
----------
param_name

   
Returns
-------
string_list


Examples
--------

Example: 1
 init: -init $ACC_ROOT_DIR/regression_tests/python_test/cesr/tao.init
 args:
   param_name: L   

\end{example}
\item[matrix] \Newline\begin{example}
Matrix value from the exit end of one element to the exit end of the other.

Notes
-----
Command syntax:
  python matrix {ele1_id} {ele2_id}
where:
  {ele1_id} is the start element.
  {ele2_id} is the end element.
If {ele2_id} = {ele1_id}, the 1-turn transfer map is computed.
Note: {ele2_id} should just be an element name or index without universe, branch, or model/base/design specification.

Example:
  python matrix 2@1>>q01w|design q02w


Parameters
----------
ele1_id
ele2_id

   
Returns
-------
string_list


Examples
--------

Example: 1
 init: -init $ACC_ROOT_DIR/regression_tests/python_test/cesr/tao.init
 args:
   ele1_id: 1@0>>q01w|design
   ele2_id: q02w

\end{example}
\item[merit] \Newline\begin{example}
Merit value.

Notes
-----
Command syntax:
  python merit


Parameters
----------


Returns
-------
string_list


Examples
--------

Example: 1
 init: -init $ACC_ROOT_DIR/regression_tests/python_test/cesr/tao.init
 args:

\end{example}
\item[orbit_at_s] \Newline\begin{example}
Twiss at given s position.

Notes
-----
Command syntax:
  python orbit_at_s {ix_uni}@{ix_branch}>>{s}|{which}
where:
  {which} is one of:
    model
    base
    design
  {s} is the longitudinal s-position.


Parameters
----------
s 
ix_uni : default=1
ix_branch : default=0
which : default=model

   
Returns
-------
string_list


Examples
--------

Example: 1
 init: -init $ACC_ROOT_DIR/regression_tests/python_test/cesr/tao.init
 args:
   ix_uni: 1
   ix_branch: 0
   s: 0.001
   which: model

\end{example}
\item[place_buffer] \Newline\begin{example}
Output place command buffer and reset the buffer.
The contents of the buffer are the place commands that the user has issued.

Notes
-----
Command syntax:
  python place_buffer


Parameters
----------

   
Returns
-------
None


Examples
--------

Example: 1
 init: -init $ACC_ROOT_DIR/regression_tests/python_test/cesr/tao.init
 args:

\end{example}
\item[plot_curve] \Newline\begin{example}
Curve information for a plot


Notes
-----
Command syntax:
  python plot_curve {curve_name}


Parameters
----------
curve_name

   
Returns
-------
string_list


Examples
--------

Example: 1
 init: -init $ACC_ROOT_DIR/regression_tests/python_test/tao.init_optics_matching
 args:
   curve_name: r13.g.a

\end{example}
\item[plot_lat_layout] \Newline\begin{example}
Plot Lat_layout info

Notes
-----
Command syntax:
  python plot_lat_layout {ix_universe}@{ix_branch}
Note: The returned list of element positions is not ordered in increasing
      longitudinal position.


Parameters
----------
ix_universe: 1
ix_branch: 0

   
Returns
-------
string_list


Examples
--------

Example: 1
 init: -init $ACC_ROOT_DIR/regression_tests/python_test/cesr/tao.init
 args:
   ix_universe: 1
   ix_branch: 0 

\end{example}
\item[plot_list] \Newline\begin{example}
List of plot templates or plot regions.

Notes
-----
Command syntax:
  python plot_list {r_or_g}
where "{r/g}" is:
  "r"      ! list regions
  "t"      ! list template plots


Parameters
----------
r_or_g

   
Returns
-------
string_list


Examples
--------

Example: 1
 init: -init $ACC_ROOT_DIR/regression_tests/python_test/cesr/tao.init
 args:
   r_or_g: r

\end{example}
\item[plot_graph] \Newline\begin{example}
Graph

Notes
-----
Command syntax:
  python plot_graph {graph_name}
{graph_name} is in the form:
  {p_name}.{g_name}
where
  {p_name} is the plot region name if from a region or the plot name if a template plot.
  This name is obtained from the python plot_list command.
  {g_name} is the graph name obtained from the python plot1 command.


Parameters
----------
graph_name

   
Returns
-------
string_list


Examples
--------

Example: 1
 init: -init $ACC_ROOT_DIR/regression_tests/python_test/tao.init_optics_matching
 args:
   graph_name: beta.g

\end{example}
\item[plot_histogram] \Newline\begin{example}
Plot Histogram

Notes
-----
Command syntax:
  python plot_histogram {curve_name}


Parameters
----------
curve_name

   
Returns
-------
string_list


Examples
--------

Example: 1
 init: -init $ACC_ROOT_DIR/regression_tests/python_test/tao.init_optics_matching
 args:
   curve_name: r33.g.x

\end{example}
\item[plot_plot_manage] \Newline\begin{example}
Template plot creation or destruction.

Notes
-----
Command syntax:
  python plot_plot_manage {plot_location}^^{plot_name}^^
                         {n_graph}^^{graph1_name}^^{graph2_name}^^{graphN_name}
Use "@Tnnn" sytax for {plot_location} to place a plot. A plot may be placed in a 
spot where there is already a template.
Extra graph names can be included with ^^ connection. 
If {n_graph} is set to -1 then just delete the plot.


Parameters
----------
plot_location
plot_name
n_graph
graph1_name
graph2_name
graphN_name

   
Returns
-------
None


Examples
--------

Example: 1
 init: -init $ACC_ROOT_DIR/regression_tests/python_test/tao.init_optics_matching
 args:
   plot_location: @T1
   plot_name: beta
   n_graph: 1
   graph1_name: g1
   graph2_name: g2
   graphN_name: gN

\end{example}
\item[plot_curve_manage] \Newline\begin{example}
Template plot curve creation/destruction

Notes
-----
Command syntax:
  python plot_curve_manage {graph_name}^^{curve_index}^^{curve_name}
If {curve_index} corresponds to an existing curve then this curve is deleted.
In this case the {curve_name} is ignored and does not have to be present.
If {curve_index} does not not correspond to an existing curve, {curve_index}
must be one greater than the number of curves.


Parameters
----------
graph_name
curve_index
curve_name

   
Returns
-------
None


Examples
--------

Example: 1
 init: -init $ACC_ROOT_DIR/regression_tests/python_test/tao.init_optics_matching
 args:
   graph_name: beta.g
   curve_index: 1
   curve_name: r13.g.a

\end{example}
\item[plot_graph_manage] \Newline\begin{example}
Template plot graph creation/destruction

Notes
-----
Command syntax:
  python plot_graph_manage {plot_name}^^{graph_index}^^{graph_name}
If {graph_index} corresponds to an existing graph then this graph is deleted.
In this case the {graph_name} is ignored and does not have to be present.
If {graph_index} does not not correspond to an existing graph, {graph_index}
must be one greater than the number of graphs.


Parameters
----------
plot_name
graph_index
graph_name

   
Returns
-------
None


Examples
--------

Example: 1
 init: -init $ACC_ROOT_DIR/regression_tests/python_test/tao.init_optics_matching
 args:
   plot_name: beta
   graph_index: 1
   graph_name: beta.g

\end{example}
\item[plot_line] \Newline\begin{example}
Output points used to construct the "line" associated with a plot curve.

Notes
-----
Command syntax:
  python plot_line {region_name}.{graph_name}.{curve_name} {x_or_y}
Optional {x-or-y} may be set to "x" or "y" to get the smooth line points x or y 
component put into the real array buffer.
Note: The plot must come from a region, and not a template, since no template plots 
      have associated line data.
Examples:
  python plot_line r13.g.a   ! String array output.
  python plot_line r13.g.a x ! x-component of line points loaded into the real array buffer.
  python plot_line r13.g.a y ! y-component of line points loaded into the real array buffer.


Parameters
----------
region_name
graph_name
curve_name
x_or_y : optional

   
Returns
-------
string_list


Examples
--------

Example: 1
 init: -init $ACC_ROOT_DIR/regression_tests/python_test/tao.init_plot_line -external_plotting
 args:
   region_name: beta
   graph_name: g
   curve_name: a
   x_or_y:

\end{example}
\item[plot_symbol] \Newline\begin{example}
Locations to draw symbols for a plot curve.

Notes
-----
Command syntax:
  python plot_symbol {region_name}.{graph_name}.{curve_name} {x_or_y}
Optional {x_or_y} may be set to "x" or "y" to get the symbol x or y 
positions put into the real array buffer.
Note: The plot must come from a region, and not a template, 
      since no template plots have associated symbol data.
Examples:
  python plot_symbol r13.g.a       ! String array output.
  python plot_symbol r13.g.a x     ! x-component of the symbol positions 
                                     loaded into the real array buffer.
  python plot_symbol r13.g.a y     ! y-component of the symbol positions 
                                     loaded into the real array buffer.


Parameters
----------
region_name
graph_name
curve_name
x_or_y

   
Returns
-------
string_list


Examples
--------

Example: 1
 init: -init $ACC_ROOT_DIR/regression_tests/python_test/tao.init_plot_line -external_plotting
 args:
   region_name: r13
   graph_name: g
   curve_name: a
   x_or_y: 

\end{example}
\item[plot_transfer] \Newline\begin{example}
Transfer plot parameters from the "from plot" to the "to plot" (or plots).

Notes
-----
Command syntax:
  python plot_transfer {from_plot} {to_plot}
To avoid confusion, use "@Tnnn" and "@Rnnn" syntax for {from_plot}.
If {to_plot} is not present and {from_plot} is a template plot, the "to plots" 
 are the equivalent region plots with the same name. And vice versa 
 if {from_plot} is a region plot.


Parameters
----------
from_plot
to_plot

   
Returns
-------
None


Examples
--------

Example: 1
 init: -init $ACC_ROOT_DIR/regression_tests/python_test/tao.init_optics_matching
 args:
   from_plot: r13
   to_plot: r23 

\end{example}
\item[plot1] \Newline\begin{example}
Info on a given plot.

Notes
-----
Command syntax:
  python plot1 {name}
{name} should be the region name if the plot is associated with a region.
Output syntax is parameter list form. See documentation at the beginning of this file.


Parameters
----------
name

   
Returns
-------
string_list


Examples
--------

Example: 1
 init: -init $ACC_ROOT_DIR/regression_tests/python_test/tao.init_optics_matching
 args:
   name: beta

\end{example}
\item[shape_list] \Newline\begin{example}
lat_layout and floor_plan shapes list

Notes
-----
Command syntax:
  python shape_list {who}
{who} is one of:
  lat_layout
  floor_plan


Parameters
----------
who

   
Returns
-------
string_list


Examples
--------

Example: 1
 init: -init $ACC_ROOT_DIR/regression_tests/python_test/cesr/tao.init
 args:
   who: floor_plan  

\end{example}
\item[shape_manage] \Newline\begin{example}
element shape creation or destruction

Notes
-----
Command syntax:
  python shape_manage {who} {index} {add_or_delete}

{who} is one of:
  lat_layout
  floor_plan
{add_or_delete} is one of:
  add     -- Add a shape at {index}. 
             Shapes with higher index get moved up one to make room.
  delete  -- Delete shape at {index}. 
             Shapes with higher index get moved down one to fill the gap.

Example:
  python shape_manage floor_plan 2 add
Note: After adding a shape use "python shape_set" to set shape parameters.
This is important since an added shape is in a ill-defined state.


Parameters
----------
who
index
add_or_delete

   
Returns
-------
string_list


Examples
--------

Example: 1
 init: -init $ACC_ROOT_DIR/regression_tests/python_test/cesr/tao.init
 args:
   who: floor_plan
   index: 1
   add_or_delete: add

\end{example}
\item[shape_pattern_list] \Newline\begin{example}
List of shape patterns or shape pattern points

Notes
-----
Command syntax:
  python shape_pattern_list {ix_pattern}

If optional {ix_pattern} index is omitted then list all the patterns.
If {ix_pattern} is present, list points of given pattern.


Parameters
----------
ix_pattern : optional

   
Returns
-------
string_list


Examples
--------

Example: 1
 init: -init $ACC_ROOT_DIR/regression_tests/python_test/tao.init_shape
 args:
   ix_pattern: 

\end{example}
\item[shape_pattern_manage] \Newline\begin{example}
Add or remove shape pattern

Notes
-----
Command syntax:
  python shape_pattern_manage {ix_pattern}^^{pat_name}^^{pat_line_width}
where:
  {ix_pattern}      -- Pattern index. Patterns with higher indexes will be moved up 
                                      if adding a pattern and down if deleting.
  {pat_name}        -- Pattern name.
  {pat_line_width}  -- Line width. Integer. If set to "delete" then section 
                                            will be deleted.


Parameters
----------
ix_pattern
pat_name
pat_line_width

   
Returns
-------
None


Examples
--------

Example: 1
 init: -init $ACC_ROOT_DIR/regression_tests/python_test/tao.init_shape
 args:
   ix_pattern : 1
   pat_name : new_pat
   pat_line_width : 1

\end{example}
\item[shape_pattern_point_manage] \Newline\begin{example}
Add or remove shape pattern point

Notes
-----
Command syntax:
  python shape_pattern_point_manage {ix_pattern}^^{ix_point}^^{s}^^{x}
where:
  {ix_pattern}      -- Pattern index.
  {ix_point}        -- Point index. Points of higher indexes will be moved up
                                    if adding a point and down if deleting.
  {s}, {x}          -- Point location. If {s} is "delete" then delete the point.


Parameters
----------
ix_pattern
ix_point
s
x

   
Returns
-------
None


Examples
--------

Example: 1
 init: -init $ACC_ROOT_DIR/regression_tests/python_test/tao.init_shape
 args:
   ix_pattern: 1
   ix_point: 1
   s: 0
   x: 0

\end{example}
\item[shape_set] \Newline\begin{example}
lat_layout or floor_plan shape set

Notes
-----
Command syntax:
  python shape_set {who}^^{shape_index}^^{ele_name}^^{shape}^^{color}^^
                   {shape_size}^^{type_label}^^{shape_draw}^^
                   {multi_shape}^^{line_width}
{who} is one of:
  lat_layout
  floor_plan


Parameters
----------
who
shape_index
ele_name
shape
color
shape_size
type_label
shape_draw
multi_shape
line_width

   
Returns
-------
None


Examples
--------

Example: 1
 init: -init $ACC_ROOT_DIR/regression_tests/python_test/cesr/tao.init
 args:
   who: floor_plan
   shape_index: 1
   ele_name: Q1
   shape: circle
   color:
   shape_size:
   type_label:
   shape_draw:
   multi_shape:
   line_width:

\end{example}
\item[show] \Newline\begin{example}
Show command pass through

Notes
-----
Command syntax:
  python show {line}
{line} is the string to pass through to the show command.
Example:
  python show lattice -python


Parameters
----------
line

   
Returns
-------
string_list


Examples
--------

Example: 1
 init: -init $ACC_ROOT_DIR/regression_tests/python_test/cesr/tao.init
 args:
   line: -python

\end{example}
\item[species_to_int] \Newline\begin{example}
Convert species name to corresponding integer

Notes
-----
Command syntax:
  python species_to_int {species_str}
Example:
  python species_to_int CO2++


Parameters
----------
species_str

   
Returns
-------
string_list


Examples
--------

Example: 1
 init: -init $ACC_ROOT_DIR/regression_tests/python_test/cesr/tao.init
 args:
   species_str: electron

\end{example}
\item[species_to_str] \Newline\begin{example}
Convert species integer id to corresponding

Notes
-----
Command syntax:
  python species_to_str {species_int}
Example:
  python species_to_str -1     ! Returns 'Electron'


Parameters
----------
species_int

   
Returns
-------
string_list


Examples
--------

Example: 1
 init: -init $ACC_ROOT_DIR/regression_tests/python_test/cesr/tao.init
 args:
   species_int: -1

\end{example}
\item[spin_polarization] \Newline\begin{example}
Spin polarization information

Notes
-----
Command syntax:
  python spin_polarization {ix_uni}@{ix_branch}|{which}
where {which} is one of:
  model
  base
  design
Example:
  python spin 1@0|model

Note: This command is under development. If you want to use please contact David Sagan.


Parameters
----------
ix_uni : default=1
ix_branch : default=0
which : default=model

   
Returns
-------
string_list


Examples
--------

Example: 1
 init: -init $ACC_ROOT_DIR/regression_tests/python_test/cesr/tao.init
 args: 
   ix_uni: 1
   ix_branch: 0
   which: model

\end{example}
\item[spin_resonance] \Newline\begin{example}
Spin resonance information

Notes 
-----
Command syntax:
  python spin_resonance {ix_uni}@{ix_branch}|{which} {ref_ele}

Parameters
----------
ix_uni : default=1
ix_branch : default=0
which : default=model
ref_ele : default=0
  Reference element to calculate at.






\end{example}
\item[super_universe] \Newline\begin{example}
Super_Universe information

Notes
-----
Command syntax:
  python super_universe


Parameters
----------

   
Returns
-------
string_list


Examples
--------

Example: 1
 init: -init $ACC_ROOT_DIR/regression_tests/python_test/cesr/tao.init
 args: 

\end{example}
\item[twiss_at_s] \Newline\begin{example}
Twiss at given s position

Notes
-----
Command syntax:
  python twiss_at_s {ix_uni}@{ix_branch}>>{s}|{which}
where {which} is one of:
  model
  base
  design


Parameters
----------
s
ix_uni : default=1
ix_branch : default=0
which : default=model

   
Returns
-------
string_list


Examples
--------

Example: 1
 init: -init $ACC_ROOT_DIR/regression_tests/python_test/cesr/tao.init
 args: 
   ix_uni: 1
   ix_branch: 0
   s: 0
   which: model 

\end{example}
\item[universe] \Newline\begin{example}
Universe info

Notes
-----
Command syntax:
  python universe {ix_universe}
Use "python global" to get the number of universes.


Parameters
----------
ix_universe

   
Returns
-------
string_list


Examples
--------

Example: 1
 init: -init $ACC_ROOT_DIR/regression_tests/python_test/cesr/tao.init
 args: 
   ix_universe: 1

\end{example}
\item[var] \Newline\begin{example}
Info on an individual variable

Notes
-----
Command syntax:
  python var {var} slaves


Parameters
----------
var
slaves : optional

   
Returns
-------
string_list


Examples
--------

Example: 1
 init: -init $ACC_ROOT_DIR/regression_tests/python_test/tao.init_optics_matching
 args: 
   var: quad[1]
   slaves:

Example: 2
 init: -init $ACC_ROOT_DIR/regression_tests/python_test/tao.init_optics_matching
 args: 
   var: quad[1]
   slaves: slaves

\end{example}
\item[var_create] \Newline\begin{example}
Create a single variable

Notes
-----
Command syntax:
  python var_create {var_name}^^{ele_name}^^{attribute}^^{universes}^^
                    {weight}^^{step}^^{low_lim}^^{high_lim}^^{merit_type}^^
                    {good_user}^^{key_bound}^^{key_delta}
{var_name} is something like "kick[5]".
Before using var_create, setup the appropriate v1_var array using 
the "python var_v1_create" command.


Parameters
----------
var_name
ele_name
attribute
universes
weight
step
low_lim
high_lim
merit_type
good_user
key_bound
key_delta

   
Returns
-------
string_list


Examples
--------

Example: 1
 init: -init $ACC_ROOT_DIR/regression_tests/python_test/tao.init_optics_matching
 args:
   var_name: quad[1]
   ele_name: Q1
   attribute: L
   universes: 1
   weight: 0.001
   step: 0.001
   low_lim: -10
   high_lim: 10
   merit_type: 
   good_user: T
   key_bound: T
   key_delta: 0.01 

\end{example}
\item[var_general] \Newline\begin{example}
List of all variable v1 arrays

Notes
-----
Command syntax:
  python var_general
Output syntax:
  {v1_var name};{v1_var%v lower bound};{v1_var%v upper bound}


Parameters
----------


Returns
-------
string_list


Examples
--------

Example: 1
 init: -init $ACC_ROOT_DIR/regression_tests/python_test/cesr/tao.init
 args:

\end{example}
\item[var_v_array] \Newline\begin{example}
List of variables for a given data_v1.

Notes
-----
Command syntax:
  python var_v_array {v1_var}
Example:
  python var_v_array quad_k1


Parameters
----------
v1_var

   
Returns
-------
string_list


Examples
--------

Example: 1
 init: -init $ACC_ROOT_DIR/regression_tests/python_test/cesr/tao.init
 args:
   v1_var: quad_k1

\end{example}
\item[var_v1_array] \Newline\begin{example}
List of variables in a given variable v1 array

Notes
-----
Command syntax:
  python var_v1_array {v1_var}


Parameters
----------
v1_var

   
Returns
-------
string_list


Examples
--------

Example: 1
 init: -init $ACC_ROOT_DIR/regression_tests/python_test/cesr/tao.init
 args:
   v1_var: quad_k1 

\end{example}
\item[var_v1_create] \Newline\begin{example}
Create a v1 variable structure along with associated var array.

Notes
-----
Command syntax:
  python var_v1_create {v1_name} {n_var_min} {n_var_max}
{n_var_min} and {n_var_max} are the lower and upper bounds of the var
Example:
  python var_v1_create quad_k1 0 45
This example creates a v1 var structure called "quad_k1" with an associated
variable array that has the range [0, 45].

Use the "set variable" command to set a created variable parameters.
In particular, to slave a lattice parameter to a variable use the command:
  set {v1_name}|ele_name = {lat_param}
where {lat_param} is of the form {ix_uni}@{ele_name_or_location}{param_name}]
Examples:
  set quad_k1[2]|ele_name = 2@q01w[k1]
  set quad_k1[2]|ele_name = 2@0>>10[k1]
Note: When setting multiple variable parameters, 
      temporarily toggle s%global%lattice_calc_on to False
  ("set global lattice_calc_on = F") to prevent Tao trying to evaluate the 
partially created variable and generating unwanted error messages.


Parameters
----------
v1_name
n_var_min
n_var_max

   
Returns
-------
string_list


Examples
--------

Example: 1
 init: -init $ACC_ROOT_DIR/regression_tests/python_test/cesr/tao.init
 args:
   v1_name: quad_k1 
   n_var_min: 0 
   n_var_max: 45 

\end{example}
\item[var_v1_destroy] \Newline\begin{example}
Destroy a v1 var structure along with associated var sub-array.

Notes
-----
Command syntax:
  python var_v1_destroy {v1_datum}


Parameters
----------
v1_datum

   
Returns
-------
string_list


Examples
--------

Example: 1
 init: -init $ACC_ROOT_DIR/regression_tests/python_test/cesr/tao.init
 args:
   v1_datum: quad_k1

\end{example}
\item[wave] \Newline\begin{example}
Wave analysis info.

Notes
-----
Command syntax:
  python wave {what}
Where {what} is one of:
  params
  loc_header
  locations
  plot1, plot2, plot3


Parameters
----------
what

   
Returns
-------
string_list


Examples
--------

Example: 1
 init: -init $ACC_ROOT_DIR/regression_tests/python_test/cesr/tao.init
 args:
   what: params

\end{example}
\end{description}
