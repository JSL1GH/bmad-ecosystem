\chapter{Spin Dynamics}
\label{c:spin}
\index{spin|hyperbf}   

%-----------------------------------------------------------------   
\section{Equations of Motion}
\label{s:spin.dyn}

The propagation of the classical spin vector $\Bf S$ is described in the local reference
frame (\sref{s:ref}) by a modified Thomas-Bargmann-Michel-Telegdi
(T-BMT) equation\cite{b:spin.hoff}
\Begineq
  \frac{\mathrm{d}}{\mathrm{d}s} \Bf S = 
  \left\{ \frac{(1+\Bf r_t \dotproduct \bfg)}{c \, \beta_z} \, 
  \left( {\pmb\Omega}_{BMT} + {\pmb\Omega}_{EDM} \right) - 
  \bfg \times \bfhat z \right\} \times \mathbf{S}
  \label{tbmt}
\Endeq
where $\bfg$ is the bend curvature function which points away from the center of curvature
of the particle's reference orbit (see \fig{f:local.coords}), $\Bf r_t = (x, y)$ are the
transverse coordinates, $c \, \beta_z$ is the longitudinal component of the velocity, and
$\bfhat z$ is the unit vector in the $z$-direction. $\pmb\Omega_{BMT}$ is the usual T-BMT
precession vector due to the particle's magnetic moment and $\pmb\Omega_{EDM}$ is the
precession vector due to a finite electric dipole moment (EDM) \cite{b:silenko}. Note:
The value for the EDM is set by \vn{bmad_com%electric_dipole_moment} (\sref{s:bmad.common}).
\begin{align}
  {\pmb\Omega}_{BMT} (\mathbf{r,P},t) &= 
    - \frac{q}{m \, c} \left[ 
    \left(\frac{1}{\gamma} + a \right) \, c \, \Bf B -
    \frac{a \, \gamma \, c}{1 + \gamma} \, ( \bfbeta \dotproduct \Bf B ) \, \bfbeta -
    \left( a + \frac{1}{1 + \gamma} \right) \, \mathbf{\bfbeta \times  E} 
    \right] \\
  &= - \frac{q}{m \, c} \left[ 
    \left( \frac{1}{\gamma} + a \right) \, c \, \Bf B_{\perp} +
    \frac{(1 + a) \, c}{\gamma} \, \Bf B_\parallel -
    \left( a + \frac{1}{1 + \gamma} \right) \mathbf{\bfbeta \times E} 
    \right] \nonumber
\end{align}
and
\Begineq
  {\pmb\Omega}_{EDM} (\mathbf{r,P},t) = 
  - \frac{q \, \eta}{2 \, m \, c} \left[
  \Bf E - \frac{\gamma}{1 + \gamma} \, 
  ( \bfbeta \dotproduct \Bf E ) \, \bfbeta +
  c \, \mathbf{\bfbeta \times B}
  \right]
\Endeq
Here $\Bf E (\Bf r ,t)$ and $\Bf B (\Bf r ,t)$ are the electric and magnetic fields, $\Bf
B_\perp$ and $\Bf B_\parallel$ are the components perpendicular and parallel to the
particle's momentum, $\gamma$ is the particle's relativistic gamma factor, $q$, $m$, and $\eta$ are
the particle's charge, mass, and electric dipole moment, $\bfbeta$ is the normalized
velocity, and $a = (g-2)/2$ is the particle's anomalous gyro-magnetic g-factor (values
given in Table~\ref{t:constants}).

%-----------------------------------------------------------------   
\section{Quaternion Representation of Spin Rotations}
\label{s:quat}

\bmad uses a quaternion representation for spin rotations. The following is a brief introduction to
quaternions. For more information, the reader is referred to the Web. In particular, the
\vn{Quaternions and spatial rotation} article at \vn{wikipedia.org}. 

A quaternion $\bfq$ is a 4-vector $\bfq = (q_1, q_x, q_y, q_z)$ where the components are reals if they represent a
rotation and are Taylor series if they represent a rotation map (\sref{s:spin.map}). A quaternion can
also be represented in the form:
\Begineq
  \bfq = q_1 + q_x \, \cali + q_y \, \calj + q_z \calk
\Endeq
where $\cali$, $\calj$, and $\calk$ are the {\em fundamental quaternion units} with the properties
\Begineq
  \cali^2 = \calj^2 = \calk^2 = \cali \, \calj \, \calk = -1, \quad 
  \cali \, \calj = \calk, \quad \calj \, \cali = -\calk, \quad \text{etc.}
\Endeq
$\cali$, $\calj$, and $\calk$ do not commute among themselves but do commute with real numbers.

The $q_1$ component of a quaternion is called the \vn{real} (or sometimes \vn{scalar}) part and the other
three components are called the \vn{imaginary} (or sometimes \vn{vector}) part.

When a quaternion represents a rotation, $\cali$, $\calj$, and $\calk$ can be thought of as representing unit
vectors along the three Cartesian axes $\bfx$, $\bfy$, and $\bfz$ respectively. A rotation through an angle
$\theta$ around the unit axis $\bfu = (u_x, u_y, u_z)$ is represented by the quaternion
\Begineq
  \bfq = \cos\frac{\theta}{2} + (u_x \, \cali + u_y \, \calj + u_z \, \calk) \sin\frac{\theta}{2}
  \label{qt2ui}
\Endeq

A rotation quaternion $\bfq$ has normalization 1:
\Begineq
  1 = ||\bfq|| = \sqrt{q_1^2 + q_x^2 + q_y^2 + q_z^2}
\Endeq
Such a quaternion is called a \vn{unit} quaternion.
The quaternion conjugate $\bfq^*$ is defined by
\Begineq
  \bfq^* = q_1 - q_x \, \cali - q_y \, \calj - q_z \calk
\Endeq
and the inverse $\bfq^{-1}$ is given by
\Begineq
  \bfq^{-1} = \frac{\bfq^*}{||\bfq||^2}
\Endeq

Given an ordinary spatial vector $(\bfr_x, \bfr_y, \bfr_z)$, this vector is represented by a
quaternion $\bfr = (0, \bfr_x, \bfr_y, \bfr_z)$. The rotation of $\bfr$ through a rotation
represented by quaternion $\bfq$ to position $\bfr'$ is given by
\Begineq
  \bfr' = \bfq \, \bfr \, \bfq^*
  \label{rqrq}
\Endeq
The transformation matrix corresponding to the transformation of \Eq{rqrq} is
\Begineq
  \bfR = \begin{pmatrix}
    q_1^2 + q_x^2 - q_y^2 - q_z^2    & 2 \, q_x \ q_y - 2 \, q_1 \, q_z & 2 \, q_x \ q_z + 2 \, q_1 \, q_y \\
    2 \, q_x \ q_y + 2 \, q_1 \, q_z & q_1^2 - q_x^2 + q_y^2 - q_z^2    & 2 \, q_y \ q_z - 2 \, q_1 \, q_x \\
    2 \, q_x \ q_z - 2 \, q_1 \, q_y & 2 \, q_y \ q_z + 2 \, q_1 \, q_x & q_1^2 - q_x^2 - q_y^2 + q_z^2 
  \end{pmatrix}
  \label{rqqq}
\Endeq

In quaternion notation, two rotations $\bfq_1$ and $\bfq_2$ are combined into one rotation using the formula
\Begineq
  \bfq_{12} = \bfq_2 \, \bfq_1
\Endeq
where $\bfq_{12}$ is the result of rotation $\bfq_1$ followed by $\bfq_2$. Note that this equation
is analogous to how rotation matrices are combined.

%-----------------------------------------------------------------   
\section{Invariant Spin Field}
\label{s:isf}

In a storage ring, the \vn{invariant spin field} $\bfn(\bfr, s)$
\cite{b:spin.hoff,b:duan15}, at some phase space position $\bfr = (x, p_x, y, p_y, z, p_z)$ and at
some point $s$ in the ring, is the continuous function with unit amplitude that satisifies
\Begineq
  \bfn(\Cal M_r \bfr, s) = \Cal M_s(\bfr) \, \bfn(\bfr, s)
  \label{nmrs}
\Endeq
where $\Cal M_r$ is the orbital part of the 1-turn transfer map and $\Cal M_s(\bfr)$ is the spin
part of the map which is a function of $\bfr$. In general, it is not straightforward to calculate
$\bfn$. The exceptional case (besides the cases where there is a resonance) is if the particle is on
the closed orbit $\bfr_0$. In this case, since $\Cal M_r \bfr_0 = \bfr_0$, and since \Cal M_s(\bfr)
is a rotation matrix, \Eq{nmrs} can be solved to give the invariant spin field on the closed orbit
which is denoted $\bfn_0$.

Once the invariant spin field has been calculated, various quantities of interest can be
computed. For example, given some initial distribution of spins in a beam, the maximum possible time
averaged polarization $\left< \bfS \right>_{\max}$ is
\Begineq
  \left< \bfS \right>_{\max} = \int d\bfr \, \rho(\bfr) \, \bfn(\bfr)
\Endeq
where the integral is over the beam phase space space density $\rho$ and the longitudinal
$s$-dependence is implicit. The above equation neglects any single spin polarization or
depolarization processes. Notice that what is calculated is a time averaged quantity.
Instantaneously, the beam can be fully polarized but the average over many turns, at some given
position $s$, cannot exceed $\left< \bfS \right>_{\max}$.

Another quantity that can be computed from knowledge of $\bfn$ is the equilibrium polarization of a beam
$\bfP_{dk}$ using the Derbenev-Kondratenko-Mane formula \cite{b:barber}:
\Begineq
  \bfP_{dk} = -\bfn \, \frac{8}{5 \, \sqrt{3}}
  \frac{\displaystyle \oint ds \,\left< g^3 \, \what\bfb \cdot 
    \left( \bfn - \frac{\partial \bfn}{\partial \delta} \right) \right>_{ps}}
  {\displaystyle \oint ds \, \left< g^3 \left( 1 - \frac{2}{9} (\bfn \cdot \what\bfs)^2 + 
    \frac{11}{18} \left| \frac{\partial \bfn}{\partial \delta} \right|^2 \right) \right>_{ps}}
\Endeq
where $<>_{ps}$ denotes an $s$-dependent average over phase space, $g = 1/\rho$ is the bending
strength ($\rho$ is the bending radius), $\delta$ is the fractional energy deviation which, for
ultra-relativistic particles, is the same as phase space $p_z$, $\what\bfs$ is the unit vector in
the direction of motion, and $\what\bfb$ is defined to be
\Begineq
  \what\bfb \equiv \frac{\what\bfs \cross d\what\bfs/ds}{d\what\bfs/ds}
\Endeq
Notice that $\what\bfb$ is the direction of the magnetic field when $\what\bfs$ is perpendicular to
the magnetic field and when there is no electric field.

The time dependence of the polarization is \cite{b:barber})
\Begineq
  \bfP(t) = \bfP_{dk} \, \left( 1 - \exp{-t/\tau_{dk}} \right) + \bfP_0 \, \exp{-t/\tau_{dk}}
\Endeq
where $\bfP_0$ is the initial polarization and the polarization rate $\tau_{dk}^{-1}$ is 
\Begineq
  \tau_{dk}^{-1} = \frac{5 \, \sqrt{3}}{8} \frac{r_e \, \gamma^5 \, \hbar}{m}
  \frac{1}{C} \, \oint ds \, \left< g^3 \, \left( 1 - \frac{2}{9} (\bfn \cdot \what\bfs)^2 + 
  \frac{11}{18} \left| \frac{\partial \bfn}{\partial \delta} \right|^2 \right) \right>_{ps}
\Endeq

$\tau_{dk}^{-1}$ can be decomposed into two parts:
\Begineq
  \tau_{dk}^{-1} = \tau_{st}^{-1} + \tau_{dep}^{-1}
\Endeq
where $\tau_{st}^{-1}$ is the Sokolov-Ternov poarization rate and the depolarization rate
$\tau_{dep}^{-1}$ is given by the formula
\Begineq
  \tau_{dk}^{-1} = \frac{5 \, \sqrt{3}}{8} \frac{r_e \, \gamma^5 \, \hbar}{m}
  \frac{1}{C} \, \oint ds \, \left< g^3 \,
  \frac{11}{18} \left| \frac{\partial \bfn}{\partial \delta} \right|^2 \right>_{ps}
\Endeq

%-----------------------------------------------------------------   
\section{SLIM Formalism}
\label{s:slim}

The \vn{SLIM} formalism\cite{b:duan15,b:barber}, indroduced by Alex Chao, is a way to represent the
first order orbital + spin transport as an $8 \cross 8$ matrix which then can be
analyzed. The idea is to expand the transport map around the closed orbit $(\bfr_0, \bfn_0)$ where
$\bfr_0$ is the orbital closed orbit and $\bfn_0$ is the invariant spin field on the closed orbit.
The spin coordinates are expressed using coordinates $(\bfl_0(s), \bfn_0(s),
\bfm_0(s))$\footnote{Different authors will use different conventions for the ordering of $\bfl_0$,
$\bfn_0$, and $\bfm_0(s)$. The ordering used here reflects the fact that in many rings the $\bfn_0$
axis will point in the vertical $y$-direction.}, where $\bfl_0$ and $\bfm_0$ are axes constructed to
be periodic in $s$ and transverse to $\bfn_0$, with $(\bfl_0(s), \bfn_0(s), \bfm_0(s))$ forming a
right hand coordinate system. 

The variation of the spin component along the $\bfn_0$ axis will be second order and therefore, to
first order, can be ignored. Thus the first order map in the $(\bfr_0, (\bfl_0, \bfm_0))$ coordinate
system between two any points $s_1$ and $s_2$ is an $8 \cross 8$ matrix $\wt\bfM$ which is
written in the form
\Begineq
  \wt\bfM(s1, s2) = \begin{pmatrix}
    \bfM_{6\cross6} & \Bf 0_{6\cross2} \\
    \bfG_{2\cross6} & \bfD_{2\cross2}
  \end{pmatrix}
\Endeq
Where $\bfM$ is the $6\cross6$ orbital transport matrix, $\bfG$ represents the coupling between
orbital and spin coordinates, and $\bfD$ is the $2\cross2$ rotation matrix for the spin transport of
a particle on the closed orbit. The upper right block $\Bf 0_{6\cross2}$ in the $\wt\bfM$ matrix
is zero since Stern-Gerlach effects are ignored. If $\wt\bfM$ is the 1-turn matrix ($s_2 = s_1$),
the phase advance of $\bfD$ is the spin tune.

To compute $\wt\bfM$, the first step is to calculate the $(\bfl_0, \bfn_0, \bfm_0)$ coordinate
system. If the entire ring is being analyzed, $\bfn_0$ at some starting point can be calculated
(\sref{s:isf}).  If only part of the ring is being analyzed, the orientation of $\bfn_0$ at the
start of the section will be an input parameter (that is, it is given by the User and not
calculated). After $\bfn_0$ is known at some $s$-position, $\bfl_0$ and $\bfm_0$ at that
$s$-position can be choisen somewhat arbitrarily to form a right handed coordinate system.

After the $(\bfl_0, \bfn_0, \bfm_0)$ coordinates have been calculated at some initial point, the
axes can be transported using the $\bfq_0$ quaternion. When analyzing only a section of a ring,
there is no identifiable spin tune so nothing further needs to be done. In this case, the $\bfD$
matrix is just a unit matrix. When analyzing an entire ring, it is sometimes desireable to rotate
the $\bfl_0(s)$ and $\bfm_0(s)$ axes to give a uniform phase advance as a function of $s$.  When
analyzing one-turn maps, it is generally not necessary to rotate the $\bfl_0(s)$ and $\bfm_0(s)$
axes. In this case the spin phase advance as a function of $s$ will be zero except just before the
starting position where there will be a discontinuous jump in phase.

Once the $(\bfl_0, \bfn_0, \bfm_0)$ coordinates have been calculated, the matrices $\bfG$ and $\bfD$
can be calculated from the spin transport map (which \bmad calculates via PTC
(\sref{c:ptc})). To first order, the spin transport map $\bfq_s$, represented as a quaternion, can be
written in the form
\Begineq
  \bfq_s = \bfq_0 + \sum_{i = 1}^6 r_i \, \bfq_i
  \label{qqrq}
\Endeq
The quaternion $\bfq_0$ is the zeroth order part of the map and the six quaternions $\bfq_i$, $i =
1, \ldots, 6$ are the first order part with $\bfr$ being the orbital phase space point the map is
being evaluated at. Let $\bfq_{lnm}(s)$ be the quaternion that transforms from $(\bfl_0, \bfn_0,
\bfm_0)$ coordinates to $(x, y, z)$ coordinates at a given point $s$. With this, the spin transport
$\what\bfq$ from $s_1$ to $s_2$ in the $(\bfl_0, \bfn_0, \bfm_0)$ coordinate system is
\Begineq
  \what\bfq_s(s_1, s_2) = \bfq_{lnm}(s_2) \, \bfq_s(s_1, s_2) \, \bfq_{lnm}^{-1}(s_1)
\Endeq
The zeroth order part of this map 
\Begineq
  \what\bfq_0(s_1,s_2) = \bfq_{lnm}(s_2) \, \bfq_0(s_1, s_2) \, \bfq_{lnm}^{-1}(s_1) 
\Endeq
represents a rotation around the $\bfn_0$ axis. Converting this to a rotation matrix $bfR_0$ via
\Eq{rqqq}) the $\Bf D$ matrix is then
\Begineq
  \Bf D(s_1, s_2) = \begin{pmatrix}
      R_0(1,1) & R_0(1,3) \\
      R_0(3,1) & R_0(3,3)
  \end{pmatrix}
\Endeq
The $\bfG$ matrix is calulated from the first order part of $\what\bfq_s$
\Begineq
  \what\bfq_i = \bfq_{lnm}(s_2) \, \bfq_i(s_1, s_2) \, \bfq_{lnm}^{-1}(s_1)
\Endeq
Using \Eq{qqrq} in \Eq{rqqq} and keeping only first order terms gives
\begin{align}
  \bfG(1,j) &= 2 (\what q_{0,x} \, \what q_{j,y} + \what q_{0,y} \, \what q_{j,x} + 
                  \what q_{0,z} \, \what q_{j,z} + \what q_{0,1} \, \what q_{j,1}) \\
  \bfG(2,j) &= 2 (\what q_{0,1} \, \what q_{j,x} + \what q_{0,x} \, \what q_{j,1} + 
                  \what q_{0,y} \, \what q_{j,z} + \what q_{0,z} \, \what q_{j,y}),
  \quad j = 1, \ldots, 6
  \nonumber
\end{align}

Lattice design consists of minimizing elements of the $\bfG$ matrix since that will minimize
${\partial \bfn}/{\partial \delta}$.


%-----------------------------------------------------------------   
\section{Spinor Notation}

The following descibes the old spinor representation used by \bmad to represent spins. This
documentation is kept as an aid for comparison with the spin tracking literature.

In the SU(2) representation, a spin $\Bf s$ is written as a spinor $\Psi = \left( \psi_{1}, \psi_{2}
\right)^{T}$ where $\psi_{1,2}$ are complex numbers. The conversion between SU(2) and SO(3) is
\Begineq  
  \Bf S = \Psi^{\dagger} \Bf {\bfsig} \, \Psi 
  \qquad \longleftrightarrow \qquad
  \Psi  = \frac{e^{i \xi}}{\sqrt{2 \left(P+s_{3}\right)}}   
     \begin{pmatrix} P+s_{3} \\ s_{1}+i s_{2} \end{pmatrix}   
  \Endeq  
Where $\xi$ is an unmeasureable phase factor, and $P$ is the polarization. $P = 1$ for a single
particle. Also ${\bfsig} = (\sigma_x, \sigma_y, \sigma_z)$ are the three Pauli matrices
\Begineq
  \sigma_x = \begin{pmatrix} 0 &  1 \\ 1 &  0 \end{pmatrix}, \qquad
  \sigma_y = \begin{pmatrix} 0 & -i \\ i &  0 \end{pmatrix}, \qquad
  \sigma_z = \begin{pmatrix} 1 &  0 \\ 0 & -1 \end{pmatrix}
\Endeq
In polar coordinates
\Begineq   
  \Psi = \begin{pmatrix} \psi_{1} \\ \psi_{2} \end{pmatrix}
       = \sqrt{P} \, e^{i \xi}
         \begin{pmatrix} 
            \cos \frac{\theta}{2} \\   
            e^{i \phi} \, \sin \frac{\theta}{2}
         \end{pmatrix}
  \qquad \longleftrightarrow \qquad
  \Bf S = P \, \begin{pmatrix} \sin \theta \cos \phi \\   
                          \sin \theta \sin \phi \\   
                          \cos \theta \end{pmatrix}
  \label{pp1p2}
\Endeq
Due to the unitarity of the spin vector,   
$|\psi_{1}|^{2} + |\psi_{2}|^{2} = P$.
The spinor eigenvectors along the $x$, $y$ and $z$ axes are
\begin{align}
   \Psi_{x+} &= \frac{1}{\sqrt{2}} \, \begin{pmatrix} 1 \\ 1 \end{pmatrix} \, , 
  &\Psi_{x-} &= \frac{1}{\sqrt{2}} \, \begin{pmatrix} 1 \\ -1 \end{pmatrix} \, , \CRNO
   \Psi_{y+} &= \frac{1}{\sqrt{2}} \, \begin{pmatrix} 1 \\ i \end{pmatrix} \, , 
  &\Psi_{y-} &= \frac{1}{\sqrt{2}} \, \begin{pmatrix} 1 \\ -i \end{pmatrix} \, , \\
   \Psi_{z+} &=                       \begin{pmatrix} 1 \\ 0 \end{pmatrix} \, , 
  &\Psi_{z-} &=                       \begin{pmatrix} 0 \\ -1 \end{pmatrix} \, . \nonumber
\end{align}

In spinor notation, the T-BMT equation can be written as
  \Begineq   
    \frac{\mathrm{d}}{\mathrm{d} t} \Psi = - \frac{i}{2} \left( \bfsig \dotproduct   
    {\pmb\Omega} \right) \Psi = -\frac{i}{2} \begin{pmatrix}
    \Omega_z & \Omega_x - i \, \Omega_y \\
    \Omega_x + i \, \Omega_y & -\Omega_z \end{pmatrix}
    \Psi
  \Endeq   
The solution leads to a rotation of the spin vector by an angle   
$\alpha$ around a unit vector $\bfhat n$ represented as   
  \begin{align}   
    \Psi_f &= \exp \left[ -i \frac{\alpha}{2} \bfhat n \dotproduct \bfsig \right] \Psi_i \CRNO
         &= \left[ \cos \left( \frac{\alpha}{2} \right) \, \Bf 1_{2} - 
            i \, (\bfhat n \dotproduct \bfsig) \, \sin \left( \frac{\alpha}{2} \right) \right] \Psi_i \\
         &= \Bf A \Psi_i. \nonumber
  \end{align}   
where $\Psi_i$ is the initial spin state, $\Psi_f$ is the final spin state, and $\Bf A$,
which describes the spin transport, is the SU(2) matrix representation of the quaternian
$(a_0, \Bf a) = (\cos(\alpha/2), -\sin(\alpha/2) \, \bfhat n)$. $\Bf A$ has the
normalization condition $a_{0}^{2} + \boldsymbol{a}^{2} = 1$. Thus the three components
$\boldsymbol{a} = \left(a_{1}, a_{2}, a_{3}\right)$ completely describe $\Bf A$. 

With spinors, the matrix representation of the observable $S_{\Bf u}$
corresponding to the measurement of the spin along the unit vector
$\Bf u$ is
\begin{align}
  S_{\Bf u} &\equiv \frac{\hbar}{2} \, \bfsig \dotproduct \Bf u \\   
            &= \frac{\hbar}{2} 
                   \begin{pmatrix} 
                     u_z            & u_x - i \, u_y \\
                     u_x + i \, u_y & u_z
                   \end{pmatrix}
\end{align}
The expectation value of this operator, $\Psi^\dagger \, \Bf S_u \,
\Psi$, representing the spin of a particle, satisfies the equation of
motion of a classical spin vector in the particle's instantaneous rest
frame.

For a distribution of spins, the polarization $P_s$ along the unit
vector $\Bf u$ is defined as the absolute value of the average
expectation value of the spin over all N particles times
$\frac{2}{\hbar}$,
  \Begineq
    P_s = \frac{2}{\hbar} \frac{1}{N} \sum_{j=1}^{N} \Psi_j^\dagger S_{\Bf u} \Psi_j
  \Endeq  

See \S~\sref{s:spin.hard.fringe} for formulas for tracking a spin through a multipole
fring field.
