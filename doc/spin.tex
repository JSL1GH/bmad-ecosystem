\chapter{Spin Dynamics}
\label{c:spin}
\index{spin|hyperbf}   

%-----------------------------------------------------------------   
\section{Equations of Motion}
\label{s:spin.dyn}


The classical spin vector $\Bf S$ is described in the local reference
frame (\sref{s:ref}) by a modified Thomas-Bargmann-Michel-Telegdi
(T-BMT) equation\cite{b:spin}
\Begineq
  \frac{\mathrm{d}}{\mathrm{d}s} \Bf S = 
  \left\{ \frac{(1+\Bf r_t \dotproduct \bfg)}{c \, \beta_z} \, 
  \left( {\pmb\Omega}_{BMT} + {\pmb\Omega}_{EDM} \right) - 
  \bfg \times \bfhat z \right\} \times \mathbf{S}
  \label{tbmt}
\Endeq
where $\bfg$ is the bend curvature function which points away from the center of curvature
of the particle's reference orbit (see \fig{f:local.coords}), $\Bf r_t = (x, y)$ are the
transverse coordinates, $c \, \beta_z$ is the longitudinal component of the velocity, and
$\bfhat z$ is the unit vector in the $z$-direction. $\pmb\Omega_{BMT}$ is the usual T-BMT
precession vector due to the particle's magnetic moment and $\pmb\Omega_{EDM}$ is the
precession vector due to a finite electric dipole moment (EDM) \cite{b:silenko}. Note:
The value for the EDM is set by \vn{bmad_com%electric_dipole_moment} (\sref{s:bmad.common}).
\begin{align}
  {\pmb\Omega}_{BMT} (\mathbf{r,P},t) &= 
    - \frac{q}{m \, c} \left[ 
    \left(\frac{1}{\gamma} + a \right) \, c \, \Bf B -
    \frac{a \, \gamma \, c}{1 + \gamma} \, ( \bfbeta \dotproduct \Bf B ) \, \bfbeta -
    \left( a + \frac{1}{1 + \gamma} \right) \, \mathbf{\bfbeta \times  E} 
    \right] \\
  &= - \frac{q}{m \, c} \left[ 
    \left( \frac{1}{\gamma} + a \right) \, c \, \Bf B_{\perp} +
    \frac{(1 + a) \, c}{\gamma} \, \Bf B_\parallel -
    \left( a + \frac{1}{1 + \gamma} \right) \mathbf{\bfbeta \times E} 
    \right] \nonumber
\end{align}
and
\Begineq
  {\pmb\Omega}_{EDM} (\mathbf{r,P},t) = 
  - \frac{q \, \eta}{2 \, m \, c} \left[
  \Bf E - \frac{\gamma}{1 + \gamma} \, 
  ( \bfbeta \dotproduct \Bf E ) \, \bfbeta +
  c \, \mathbf{\bfbeta \times B}
  \right]
\Endeq
Here $\Bf E (\Bf r ,t)$ and $\Bf B (\Bf r ,t)$ are the electric and magnetic fields, $\Bf
B_\perp$ and $\Bf B_\parallel$ are the components perpendicular and parallel to the
momentum, $\gamma$ is the particle's relativistic gamma factor, $q$, $m$, and $\eta$ are
the particle's charge, mass, and magnetic_moment, $\bfbeta$ is the normalized
velocity, and $a = (g-2)/2$ is the particle's anomalous gyro-magnetic g-factor (values
given in Table~\ref{t:constants}).

%-----------------------------------------------------------------   
\section{Quaternion Representation of Spin Rotations}
\label{s:quat}

\bmad uses a quaternion representation for spin rotations. The following is a brief introduction to
quaternions. For more information, the reader is referred to the Web. In particular, the
\vn{Quaternions and spatial rotation} article at \vn{wikipedia.org}. 

A quaternion $\bfq$ is a 4-vector $\bfq = (q_1, q_x, q_y, q_z)$ where the components are reals if they represent a
rotation and are Taylor series if they represent a rotation map (\sref{s:spin.map}). A quaternion can
also be represented in the form:
\Begineq
  \bfq = q_1 + q_x \, \cali + q_y \, \calj + q_z \calk
\Endeq
where $\cali$, $\calj$, and $\calk$ are the {\em fundamental quaternion units} with the properties
\Begineq
  \cali^2 = \calj^2 = \calk^2 = \cali \, \calj \, \calk = -1, \quad 
  \cali \, \calj = \calk, \quad \calj \, \cali = -\calk, \quad \text{etc.}
\Endeq
$\cali$, $\calj$, and $\calk$ do not commute among themselves but do commute with real numbers.

The $q_1$ component of a quaternion is called the \vn{real} (or sometimes \vn{scalar}) part and the other
three components are called the \vn{imaginary} (or sometimes \vn{vector}) part.

When a quaternion represents a rotation, $\cali$, $\calj$, and $\calk$ can be thought of as representing unit
vectors along the three Cartesian axes $\bfx$, $\bfy$, and $\bfz$ respectively. A rotation through an angle
$\theta$ around the unit axis $\bfu = (u_x, u_y, u_z)$ is represented by the quaternion
\Begineq
  \bfq = \cos\frac{\theta}{2} + (u_x \, \cali + u_y \, \calj + u_z \, \calk) \sin\frac{\theta}{2}
  \label{qt2ui}
\Endeq

A rotation quaternion $\bfq$ has normalization 1:
\Begineq
  1 = ||\bfq|| = \sqrt{q_1^2 + q_x^2 + q_y^2 + q_z^2}
\Endeq
Such a quaternion is called a \vn{unit} quaternion.
The quaternion conjugate $\bfq^*$ is defined by
\Begineq
  \bfq^* = q_1 - q_x \, \cali - q_y \, \calj - q_z \calk
\Endeq
and the inverse $\bfq^{-1}$ is given by
\Begineq
  \bfq^{-1} = \frac{\bfq^*}{||\bfq||^2}
\Endeq

Given an ordinary spatial vector $(\bfr_x, \bfr_y, \bfr_z)$, this vector is represented by a
quaternion $\bfr = (0, \bfr_x, \bfr_y, \bfr_z)$. The rotation of $\bfr$ through a rotation
represented by quaternion $\bfq$ to position $\bfr'$ is given by
\Begineq
  \bfr' = \bfq \, \bfr \, \bfq^*
  \label{rqrq}
\Endeq
The transformation matrix corresponding to the transformation of \Eq{rqrq} is
\Begineq
  \bfR = \begin{pmatrix}
    q_1^2 + q_x^2 - q_y^2 - q_z^2    & 2 \, q_x \ q_y - 2 \, q_1 \, q_z & 2 \, q_x \ q_z + 2 \, q_1 \, q_y \\
    2 \, q_x \ q_y + 2 \, q_1 \, q_z & q_1^2 - q_x^2 + q_y^2 - q_z^2    & 2 \, q_y \ q_z - 2 \, q_1 \, q_x \\
    2 \, q_x \ q_z - 2 \, q_1 \, q_y & 2 \, q_y \ q_z + 2 \, q_1 \, q_x & q_1^2 - q_x^2 - q_y^2 + q_z^2 
  \end{pmatrix}
\Endeq

In quaternion notation, two rotations $\bfq_1$ and $\bfq_2$ are combined into one rotation using the formula
\Begineq
  \bfq_{12} = \bfq_2 \, \bfq_1
\Endeq
where $\bfq_{12}$ is the result of rotation $\bfq_1$ followed by $\bfq_2$. Note that this equation
is analogous to how rotation matrices are combined.

%-----------------------------------------------------------------   
\section{Invariant Spin Field}
\label{s:isf}

\etcetc

%-----------------------------------------------------------------   
\section{SLIM Formalism}
\label{s:slim}

The \vn{SLIM} formalism\cite{b:duan15,b:barber.ripkin}, indroduced by Alex Chao, is a way to represent the first order
orbital + spin transport as an $8 \cross 8$ matrix which then can be efficiently analyzed.

\etcetc


To compute the $\bfD$ and $\bfG$ matrices from the spin transport map (\sref{s:spin.map}), the spin
transport map is first truncated to linear order. In this case, the spin transport map $\cal M_s$
can be written in the form
\Begineq
  \cal M_s = \bfq_0 + \sum_{i = 1}^6 r_i \, \bfq_i
\Endeq
The quaternion $\bfq_0$ is the zeroth order part of the map and the six quaternions $\bfq_i$, $i =
1, \ldots, 6$ are the first order part with $\bfr$ being the orbital phase space point the map is
being evaluated at.

The next step is calculate the $(\bfl, \bfn_0, \bfm)$ coordinate system. If the entire ring is being
analyzed, $\bfn_0$ at some starting point can be calculated as discussed above. If only part of the
ring is being analyzed, the orientation of $\bfn_0$ at the start of the section will be an input
parameter (that is, it is given and not calculated). The $\bfl$ and $\bfm$ axes at the starting
point are chosen to give a right handed coordante system. After the initial $(\bfl, \bfn_0, \bfm)$
coordinates have been calculated, the axes are transported using the $\bfq_0$
quaternion. When analyzing only a section of a ring, there is no identifiable spin tune so the
second step of rotating the $\bfl$ and $\bfm$ axes to give a uniform phase advance is skipped. In
this case, the $\bfD$ matrix is just a unit matrix.  When analyzing an entire ring, the second step
of rotating the $\bfl$ and $\bfm$ axes to give a uniform phase advance is optional (except at the
end of the ring where the $\bfl$ and $\bfm$ axes must be rotated to keep the axes periodic) since,
for analysis purposes, it generally does not matter if the spin phase advance is uniform as a
funciton of longitudinal position.

Once the $(\bfl, \bfn_0, \bfm)$ coordinate system is established, the transformation from this
coordinate system to the laboratory $(x, y, z)$ coordinates is given by the matrix $(\bfl, \bfn_0,
\bfm)$ where each axes is a column of the matrix. Between two longitudinal positions $s_1$ and
$s_2$, the $\bfD$ matrix is calculated by first transforming $\bfq_0(s_1, s_2)$ to the $(\bfl,
\bfn_0, \bfm)$ coordinate system
\Begineq
  \bfq_{0\gamma} = \bft_2 \, \bfq_0 \, \bft_1^{-1}
\Endeq
$\bfq_{0\gamma}$ is a rotation around the $\bfn_0$ axis. When $\bfq_{0\gamma}$ is represented as a
matrix $\bfm_{0\gamma}$, the $\bfD$ matrix is then
\Begineq
  \bfD(s_1, s_2) = \begin{pmatrix}
      m_{0\gamma}(1,1) & m_{0\gamma}(1,3) \\
      m_{0\gamma}(3,1) & m_{0\gamma}(3,3)
  \end{pmatrix}
\Endeq



%-----------------------------------------------------------------   
\section{Spinor Notation}

The following descibes the old spinor representation used by \bmad to represent spins. This
documentation is kept as an aid for comparison with the spin tracking literature.

In the SU(2) representation, a spin $\Bf s$ is written as a spinor $\Psi = \left( \psi_{1}, \psi_{2}
\right)^{T}$ where $\psi_{1,2}$ are complex numbers. The conversion between SU(2) and SO(3) is
\Begineq  
  \Bf S = \Psi^{\dagger} \Bf {\bfsig} \, \Psi 
  \qquad \longleftrightarrow \qquad
  \Psi  = \frac{e^{i \xi}}{\sqrt{2 \left(P+s_{3}\right)}}   
     \begin{pmatrix} P+s_{3} \\ s_{1}+i s_{2} \end{pmatrix}   
  \Endeq  
Where $\xi$ is an unmeasureable phase factor, and $P$ is the polarization. $P = 1$ for a single
particle. Also ${\bfsig} = (\sigma_x, \sigma_y, \sigma_z)$ are the three Pauli matrices
\Begineq
  \sigma_x = \begin{pmatrix} 0 &  1 \\ 1 &  0 \end{pmatrix}, \qquad
  \sigma_y = \begin{pmatrix} 0 & -i \\ i &  0 \end{pmatrix}, \qquad
  \sigma_z = \begin{pmatrix} 1 &  0 \\ 0 & -1 \end{pmatrix}
\Endeq
In polar coordinates
\Begineq   
  \Psi = \begin{pmatrix} \psi_{1} \\ \psi_{2} \end{pmatrix}
       = \sqrt{P} \, e^{i \xi}
         \begin{pmatrix} 
            \cos \frac{\theta}{2} \\   
            e^{i \phi} \, \sin \frac{\theta}{2}
         \end{pmatrix}
  \qquad \longleftrightarrow \qquad
  \Bf S = P \, \begin{pmatrix} \sin \theta \cos \phi \\   
                          \sin \theta \sin \phi \\   
                          \cos \theta \end{pmatrix}
  \label{pp1p2}
\Endeq
Due to the unitarity of the spin vector,   
$|\psi_{1}|^{2} + |\psi_{2}|^{2} = P$.
The spinor eigenvectors along the $x$, $y$ and $z$ axes are
\begin{align}
   \Psi_{x+} &= \frac{1}{\sqrt{2}} \, \begin{pmatrix} 1 \\ 1 \end{pmatrix} \, , 
  &\Psi_{x-} &= \frac{1}{\sqrt{2}} \, \begin{pmatrix} 1 \\ -1 \end{pmatrix} \, , \CRNO
   \Psi_{y+} &= \frac{1}{\sqrt{2}} \, \begin{pmatrix} 1 \\ i \end{pmatrix} \, , 
  &\Psi_{y-} &= \frac{1}{\sqrt{2}} \, \begin{pmatrix} 1 \\ -i \end{pmatrix} \, , \\
   \Psi_{z+} &=                       \begin{pmatrix} 1 \\ 0 \end{pmatrix} \, , 
  &\Psi_{z-} &=                       \begin{pmatrix} 0 \\ -1 \end{pmatrix} \, . \nonumber
\end{align}

In spinor notation, the T-BMT equation can be written as
  \Begineq   
    \frac{\mathrm{d}}{\mathrm{d} t} \Psi = - \frac{i}{2} \left( \bfsig \dotproduct   
    {\pmb\Omega} \right) \Psi = -\frac{i}{2} \begin{pmatrix}
    \Omega_z & \Omega_x - i \, \Omega_y \\
    \Omega_x + i \, \Omega_y & -\Omega_z \end{pmatrix}
    \Psi
  \Endeq   
The solution leads to a rotation of the spin vector by an angle   
$\alpha$ around a unit vector $\bfhat n$ represented as   
  \begin{align}   
    \Psi_f &= \exp \left[ -i \frac{\alpha}{2} \bfhat n \dotproduct \bfsig \right] \Psi_i \CRNO
         &= \left[ \cos \left( \frac{\alpha}{2} \right) \, \Bf 1_{2} - 
            i \, (\bfhat n \dotproduct \bfsig) \, \sin \left( \frac{\alpha}{2} \right) \right] \Psi_i \\
         &= \Bf A \Psi_i. \nonumber
  \end{align}   
where $\Psi_i$ is the initial spin state, $\Psi_f$ is the final spin state, and $\Bf A$,
which describes the spin transport, is the SU(2) matrix representation of the quaternian
$(a_0, \Bf a) = (\cos(\alpha/2), -\sin(\alpha/2) \, \bfhat n)$. $\Bf A$ has the
normalization condition $a_{0}^{2} + \boldsymbol{a}^{2} = 1$. Thus the three components
$\boldsymbol{a} = \left(a_{1}, a_{2}, a_{3}\right)$ completely describe $\Bf A$. 

With spinors, the matrix representation of the observable $S_{\Bf u}$
corresponding to the measurement of the spin along the unit vector
$\Bf u$ is
\begin{align}
  S_{\Bf u} &\equiv \frac{\hbar}{2} \, \bfsig \dotproduct \Bf u \\   
            &= \frac{\hbar}{2} 
                   \begin{pmatrix} 
                     u_z            & u_x - i \, u_y \\
                     u_x + i \, u_y & u_z
                   \end{pmatrix}
\end{align}
The expectation value of this operator, $\Psi^\dagger \, \Bf S_u \,
\Psi$, representing the spin of a particle, satisfies the equation of
motion of a classical spin vector in the particle's instantaneous rest
frame.

For a distribution of spins, the polarization $P_s$ along the unit
vector $\Bf u$ is defined as the absolute value of the average
expectation value of the spin over all N particles times
$\frac{2}{\hbar}$,
  \Begineq
    P_s = \frac{2}{\hbar} \frac{1}{N} \sum_{j=1}^{N} \Psi_j^\dagger S_{\Bf u} \Psi_j
  \Endeq  

See \S~\sref{s:spin.hard.fringe} for formulas for tracking a spin through a multipole
fring field.
