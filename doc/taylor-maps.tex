\chapter{Taylor Maps}

%-----------------------------------------------------------------
\section{Taylor Maps}
\label{s:taylor.phys}
\index{taylor map|hyperbf}
\index{symplectic integration}

A transport map ${\cal M}: {\cal R}^6 \rightarrow {\cal R}^6$ through
an element or a section of a lattice is a function that maps the
starting phase space coordinates $\Bf r(\In)$ to the ending
coordinates $\Bf r(\Out)$
\begin{equation}
  \Bf r(\Out) = {\cal M} \, (\delta \bfr)
\end{equation}
where
\Begineq
  \delta \bfr = \bfr(\In) - \bfr_\Ref
\Endeq
$\bfr_\Ref$ is the reference orbit at the start of the map around which the map is
made. In many cases the reference orbit is the zero orbit. For a storage ring, the closed
orbit is commanly used for the reference orbit.

${\cal M}$ in the above equation is made up of six functions ${\cal M}_i: {\cal R}^6
\rightarrow {\cal R}$. Each of these functions maps to one of the $r(\Out)$
coordinates. Each of these functions can be expanded in a Taylor series and truncated at
some order. Each Taylor series is in the form
\Begineq
  r_i(\Out) = \sum_{j = 1}^N \, C_{ij} \, \prod_{k = 1}^6 \, (\delta r_k)^{e_{ijk}}
  \label{rcr}
\Endeq
Where the $C_{ij}$ are coefficients and the $e_{ijk}$ are integer exponents.
The order of a given term associated with index $i,j$ is the sum over the exponents
\Begineq
  \text{order}_{ij} = \sum_{k = 1}^6 e_{ijk} 
\Endeq
The order of the entire map is the order at which the map is truncated.

The standard \bmad routine for printing a Taylor map might produce something 
like this: 
\begin{example}
   Taylor Terms:
   Out      Coef             Exponents          Order       Reference
   --------------------------------------------------
    1:     -0.600000000000   0  0  0  0  0  0       0       0.200000000
    1:      1.000000000000   1  0  0  0  0  0       1
    1:      0.145000000000   2  0  0  0  0  0       2
   --------------------------------------------------
    2:     -0.185000000000   0  0  0  0  0  0       0       0.000000000
    2:      1.300000000000   0  1  0  0  0  0       1
    2:      3.800000000000   2  0  0  0  0  1       3
   --------------------------------------------------
    3:      1.000000000000   0  0  1  0  0  0       1       0.100000000
    3:      1.600000000000   0  0  0  1  0  0       1
    3:    -11.138187077310   1  0  1  0  0  0       2
   --------------------------------------------------
    4:      1.000000000000   0  0  0  1  0  0       1       0.000000000
   --------------------------------------------------
    5:      0.000000000000   0  0  0  0  0  0       0       0.000000000
    5:      0.000001480008   0  1  0  0  0  0       1
    5:      1.000000000000   0  0  0  0  1  0       1
    5:      0.000000000003   0  0  0  0  0  1       1
    5:      0.000000000003   2  0  0  0  0  0       2
   --------------------------------------------------
    6:      1.000000000000   0  0  0  0  0  1       1       0.000000000
\end{example}
Each line in the example represents a single \vn{Taylor term}. The
Taylor terms are grouped into 6 \vn{Taylor series}. There is one
series for each of the output phase space coordinate. The first
column in the example, labeled ``out'', (corresponding to the $i$
index in \Eq{rcr}) indicates the Taylor series: $1 = x(out)$, $2 =
p_x(out)$, etc. The 6 exponent columns give the $e_{ijk}$ of
\Eq{rcr}. In this example, the second Taylor series (\vn{out} = 2),
when expressed as a formula, would read:
\Begineq
  p_x(out) = -0.185 + 1.3 \, \delta p_x + 3.8 \, \delta x^2 \, \delta p_z
\Endeq

\index{taylor map!reference coordinates}
The reference column in the above example shows the input coordinates around
which the Taylor map is calculated. In this case, the reference
coordinates where 
\Begineq
  (x, p_x, y, p_y, z, p_z)_{ref} = (0.2, 0, 0.1, 0, 0, 0, 0)
\Endeq
The choice of the reference point will affect the values of the
coefficients of the Taylor map. As an example, consider the 1-dimension map
\Begineq
  x(out) = A \, \sin(k \, \delta x)
\Endeq
Then a Taylor map to 1\St order is
\Begineq
  x(out) = c_0 + c_1 \, \delta x
\Endeq
where
\begin{align}
  c_1 &= A \, k \, \cos(k \, x_{\text{ref}}) \\
  c_0 &= A \, \sin(k \, x_{\text{ref}})
\end{align}

%-----------------------------------------------------------------
\section{Spin Taylor Map}
\label{s:spin.map}
\index{spin taylor map}

A Taylor map that fully describes spin and orbital motion, would
consist of nine Taylor series (six for the orbital phase space
variables and three for the spin components) and each Taylor series
would be a polynomial in nine variables.

To simplify things, \bmad assumes that the effect on the orbital phase
space due to the spin orientation is negligible. That is, Stern-Gerlach
effects are ignored. With this assumption, the orbital part of the map
is only dependent on the six orbital variables. Furthermore, 
the three spin Taylor series are linear in the spin coordinates
and can be cast into matrix form:
\Begineq
  S_i(\Out) = \sum_{j} \Bf\Lambda_{ij}(\bfr(\In) - \bfr_\Ref) \, S_j(\In)
\Endeq
where $\Bf S = (s_x, s_y, s_z)$ is the spin and $\Bf\Lambda_{ij}$ are
Taylor series that are only dependent upon the orbit coordinates. The
proof of this assertion is derived from the
Thomas-Bargmann-Michel-Telegdi equation (\sref{tbmt}). Since the
orbital coordinates are assumed independent of the spin, the TBMT
equation is linear in the spin and thus the transport can be described
by a matrix. Furthermore, since the spin magnitude is an invarient,
for given initial orbit coordinates, The $\Bf\Lambda_{ij}$ matrix,
evaluated for some given initial orbital starting point, must just
represent a rotation about some axis. While use of the $\Bf\Lambda$
matrix increases the number of Taylor series needed from 9 to 15, all
the series are now only dependent on the six orbit coordinates.

The standard \bmad routine for printing a spin Taylor map will produce
something like this:
\begin{example}
  Spin Taylor Terms:
  Out      Coef_Sx         Coef_Sy         Coef_Sz      Exponents           Order
  -------------------------------------------------------------------------------
  Sx:      0.99757886     -0.02372254     -0.06537314   0  0  0  0  0  0        0
  Sx:      0.04802411      0.22401654      0.65154583   1  0  0  0  0  0        1
  Sx:      0.07391383      3.77338795     -0.24137562   0  1  0  0  0  0        1
  Sx:      0.00008802     -0.17584738      0.06515458   0  0  1  0  0  0        1
  Sx:      0.01148244     -0.20322945      0.24896717   0  0  0  1  0  0        1
  Sx:     -0.00457076     -0.22322148      0.01125358   0  0  0  0  0  1        1
  -------------------------------------------------------------------------------
  Sy:     -0.02460178      0.75885811     -0.65079114   0  0  0  0  0  0        0
  Sy:      0.25039365     -0.04801091     -0.06544895   1  0  0  0  0  0        1
  Sy:     -3.02631629     -0.07739091      0.02416143   0  1  0  0  0  0        1
  Sy:      0.17584738      0.00008802     -0.00654489   0  0  1  0  0  0        1
  Sy:      0.47579058      1.59361755      1.84025908   0  0  0  1  0  0        1
  Sy:      0.13046518     -0.46155512     -0.54313051   0  0  0  0  0  1        1
  -------------------------------------------------------------------------------
  Sz:      0.06504736      0.65082378      0.75643719   0  0  0  0  0  0        0
  Sz:     -0.64180483      0.06414595      0.00000000   1  0  0  0  0  0        1
  Sz:     -2.27815007      0.22777762     -0.00007328   0  1  0  0  0  0        1
  Sz:      0.06515785     -0.00651227      0.00000000   0  0  1  0  0  0        1
  Sz:      0.00385332     -1.86555986      1.60475990   0  0  0  1  0  0        1
  Sz:      0.11944181      0.53003513     -0.46630288   0  0  0  0  0  1        1
\end{example}
The reference orbit is seen by printing the orbital part of the map.

%-----------------------------------------------------------------
\section{Symplectification}
\label{s:symp.method}
\index{symplectification}

If the evolution of a system can be described using a Hamiltonian then
it can be shown that the linear part of any transport map (the Jacobian)
must obey the symplectic condition. If a matrix $\Bf M$ is not symplectic,
Healy\cite{b:healy} has provided an elegant method for finding a symplectic 
matrix that is ``close'' to $\Bf M$. The procedure is as follows:
From $\Bf M$ a matrix $\bfV$ is formed via
\begin{equation}
  \bfV = \Bf S (\Bf I - \Bf M)(\Bf I + \Bf M)^{-1} 
  \label{e:vsimi}
\end{equation}
where $\Bf S$ is the matrix
\Begineq
  \Bf S = 
  \begin{pmatrix} 
      0 &  1 &  0 &  0 &  0 &  0 \cr
     -1 &  0 &  0 &  0 &  0 &  0 \cr
      0 &  0 &  0 &  1 &  0 &  0 \cr
      0 &  0 & -1 &  0 &  0 &  0 \cr
      0 &  0 &  0 &  0 &  0 & -1 \cr
      0 &  0 &  0 &  0 & -1 &  0 \cr
  \end{pmatrix}
  \label{s0100}
\Endeq
$\bfV$ is symmetric if and only if $\Bf M$ is symplectic. In any case,
a symmetric matrix $\Bf W$ near $\bfV$ can be
formed via
\begin{equation}
  \Bf W = \frac{\bfV + \bfV^t}{2}
\end{equation}
A symplectic matrix $\Bf F$ is now obtained by inverting \eq{e:vsimi}
\Begineq
  \Bf F = (\Bf I + \Bf S \Bf W) (\Bf I - \Bf S \Bf W)^{-1}
\Endeq

%-----------------------------------------------------------------
\section{Map Concatenation and Feed-Down}
\label{s:map.concat}

\index{taylor map!feed-down}
Of importance in working with Taylor maps is the concept of
\vn{feed-down}.  This is best explained with an example. To keep the
example simple, the discussion is limited to one phase space
dimension so that the Taylor maps are a single Taylor series. Take the
map $M_1$ from point 0 to point 1 to be
\Begineq
  M_1: x_1 = x_0 + 2
  \label{xx2}
\Endeq
and the map $M_2$ from point 1 to point 2 to be
\Begineq
  M_2: x_2 = x_1^2 + 3 \, x_1
  \label{xx3x}
\Endeq
Then concatenating the maps to form the map $M_3$ from point 0 to point 2
gives
\Begineq
  M_3: x_2 = (x_0 + 2)^2 + 3 (x_0 + 2) = x_0^2 + 7 \, x_0 + 10
  \label{xx23x2}
\Endeq
However if we are evaluating our maps to only 1\St order the map $M_2$
becomes
\Begineq
  M_2: x_2 = 3 \, x_1
\Endeq
and concatenating the maps now gives
\Begineq
  M_3: x_2 = 3 (x_0 + 2) = 3 \, x_0 + 6
  \label{x3x23}
\Endeq
Comparing this to \Eq{xx23x2} shows that by neglecting the 2\Nd order
term in \Eq{xx3x} leads to 0\Th and 1\St order errors in
\Eq{x3x23}. These errors can be traced to the finite 0\Th order term in
\Eq{xx2}. This is the principal of feed--down: Given $M_3$ which is a map
produced from the concatenation of two other maps, $M_1$, and $M_2$
\Begineq
  M_3 = M_2(M_1)
\Endeq
Then if $M_1$ and $M_2$ are correct to n\Th order, $M_3$ will also be
correct to n\Th order as long as $M_1$ has no constant (0\Th order)
term. [Notice that a constant term in $M_2$ does not affect the
argument.]  What happens if we know there are constant terms in our
maps? One possibility is to go to a coordinate system where the
constant terms vanish. In the above example that would mean using the
coordinate $\widetilde x_0$ at point 0 given by
\Begineq
  \widetilde x_0 = x_0 + 2
\Endeq
\index{symplectic integration}

%-----------------------------------------------------------------
\section{Symplectic Integration}
\label{s:symp.integ}

Symplectic integration, as opposed to concatenation, never has
problems with feed--down. The subject of symplectic integration is too
large to be covered in this guide. The reader is referred to the book
``Beam Dynamics: A New Attitude and Framework'' by \'Etienne
Forest\cite{b:forest}. A brief synopsis: Symplectic integration uses
as input 1) The Hamiltonian that defines the equations of motion, and
2) a Taylor map $M_1$ from point 0 to point 1. Symplectic integration
from point 1 to point 2 produces a Taylor map $M_3$ from point 0 to
point 2. Symplectic integration can produce maps to arbitrary
order. In any practical application the order $n$ of the final map is
specified and in the integration procedure all terms of order higher
than $n$ are ignored. If one is just interested in knowing the final
coordinates of a particle at point 2 given the initial coordinates at
point 1 then $M_1$ is just the constant map
\Begineq
  M_1: x_1 = c_i
\Endeq
where $c_i$ is the initial starting point. The order of the
integration is set to 0 so that all non--constant terms are
ignored. The final map is also just a constant map
\Begineq
  M_3: x_2 = c_f
\Endeq
If the map from point 1 to point 2 is desired then the map $M_1$ is
just set to the identity map
\Begineq
  M_1: x_1 = x_0
\Endeq
In general it is impossible to exactly integrate any non--linear
system. In practice, the symplectic integration is achieved by slicing
the interval between point 1 and point 2 into a number of (generally
equally spaced) slices. The integration is performed, slice step by
slice step. This is analogous to integrating a function by evaluating
the function at a number of points. Using more slices gives better
results but slows down the calculation. The speed and accuracy of the
calculation is determined by the number of slices and the \vn{order}
of the integrator. The concept of integrator order can best be
understood by analogy by considering the trapezoidal rule for
integrating a function of one variable:
\Begineq
  \int_{y_a}^{y_b} f(y) \, dy = 
  h \left[ \frac{1}{2} f(y_a) + \frac{1}{2} f(y_b) \right] +
  o(h^3 \, f^{(2)})
\Endeq
In the formula $h = y_b - y_a$ is the slice width. $0(h^3 \, f^{(2)})$
means that the error of the trapezoidal rule scales as the second
derivative of $f$. Since the error scales as $f^{(2)}$ this is an
example of a second order integrator. To integrate a function between
points $y_1$ and $y_N$ we slice the interval at points $y_2 \ldots y_{N-1}$
and apply the trapezoidal rule to each interval. Examples of higher
order integrators can be found, for example, in Numerical
Recipes\cite{b:nr}. The concept of integrator order in symplectic
integration is analogous. 

The optimum number of slices is determined by the smallest number that
gives an acceptable error. The slice size is given by the \vn{ds_step}
attribute of an element (\sref{s:integ}).  Integrators of higher order
will generally need a smaller number of slices to achieve a given
accuracy. However, since integrators of higher order take more time
per slice step, and since it is computation time and not number of
slices which is important, only a measurement of error and calculation
time as a function of slice number and integrator order will
unambiguously give the optimum integrator order and slice width.  In
doing a timing test, it must be remembered that since the magnitude of
any non-linearities will depend upon the starting position, the
integration error will be dependent upon the starting map $M_1$. \bmad
has integrators of order 2, 4, and 6 (\sref{s:integ}). Timing tests
performed for some wiggler elements (which have strong nonlinearities)
showed that, in this case, the 2\Nd order integrator gave the fastest
computation time for a given accuracy. However, the higher order
integrators may give better results for elements with weaker
nonlinearities.
