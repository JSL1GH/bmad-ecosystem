\chapter{Introduction to Bmad Programming}
\label{c:program.info}

To get the general feel for how \bmad works before
getting into the nitty--gritty details in subsequent chapters, this
chapter analyzes an example test program.

%-----------------------------------------------------------------------------
\section{A First Program}
\label{s:first.program}
\index{programming!example program}

Consider the example program shown in \fig{f:program}.  This program
is provided with \bmad in the directory:
\begin{example}
  \$ACC_ROOT_DIR/examples/simple_bmad_program
\end{example}
The executable is at:
\begin{example}
  \$ACC_ROOT_DIR/production/bin/simple_bmad_program
\end{example}
When you run the program, be in the \vn{examples/simple_bmad_program} directory since
the program will look for the \vn{lat.bmad} lattice file that is there.

\begin{figure}[htp]
\index[routine]{bmad_parser}
\index[routine]{twiss_at_start}
\index[routine]{twiss_propagate_all}
\index[routine]{lat_ele_locator}
\index[routine]{type_ele}
\index{ele_struct}
\index{lat_struct}
\index{ele_pointer_struct}
\index{geometry}
\begin{listing}{1}
program test

use bmad                 ! Define the structures we need to know about.
implicit none
type (lat_struct), target :: lat   ! This structure holds the lattice info
type (ele_struct), pointer :: ele, cleo
type (ele_pointer_struct), allocatable :: eles(:)
type (all_pointer_struct) a_ptr
integer i, ix, n_loc
logical err

! Programs should always implement "intelligent bookkeeping".
bmad_com%auto_bookkeeper = .false.

! Read in a lattice, and modify the ks solenoid strength of "cleo_sol".

call bmad_parser ("lat.bmad", lat)  ! Read in a lattice.

call lat_ele_locator ('CLEO_SOL', lat, eles, n_loc, err)  ! Find element
cleo => eles(1)%ele                        ! Point to cleo_sol element.
call pointer_to_attribute (cleo, 'KS', .true., a_ptr, err) ! Point to KS attribute.
a_ptr%r = a_ptr%r + 0.001         ! Modify KS component.
call set_flags_for_changed_attribute (cleo, a_ptr%r)
call lattice_bookkeeper (lat)
call lat_make_mat6 (lat, cleo%ix_ele)      ! Remake transfer matrix

! Calculate starting Twiss params if the lattice is closed, 
! and then propagate the Twiss parameters through the lattice.

if (lat%param%geometry == closed$) call twiss_at_start (lat)
call twiss_propagate_all (lat)      ! Propagate Twiss parameters

! Print info on the first 11 elements

print *, ' Ix  Name              Ele_type                   S      Beta_a'
do i = 0, 10
  ele => lat%ele(i)
  print '(i4,2x,a16,2x,a,2f12.4)', i, ele%name, key_name(ele%key), ele%s, ele%a%beta
enddo

! print information on the CLEO_SOL element.

print *
print *, '!---------------------------------------------------------'
print *, '! Information on element: CLEO_SOL'
print *
call type_ele (cleo, .false., 0, .false., 0, .true., lat)

deallocate (eles)

end program
\end{listing}
\caption{Example Bmad program}
\label{f:program}
\end{figure}

%-----------------------------------------------------------------------------
\section{Explanation of the Simple_Bmad_Program}

\index{lat_struct!example use of}
A line by line explanation of the example program follows. The \vn{use
bmad} statement at line 3 defines the \bmad structures and defines the
interfaces (argument lists) for the \bmad subroutines. In particular,
the \vn{lat} variable (line 5), which is of type \vn{lat_struct}
(\sref{s:lat.struct}), holds all of the lattice information: The list
of elements, their attributes, etc.  The setting of
\vn{bmad_com%auto_bookkeeper} to \vn{False} in line 13 enables the
``intelligent'' bookkeeping of lattice attributes as discussed in
\sref{s:lat.bookkeeping}). The call to
\Hyperref{r:bmad.parser}{bmad_parser} (line 17) causes the lattice
file \vn{lat.bmad} to be parsed and the lattice information is stored
the \vn{lat} variable. Note: To get a listing of the \vn{lat_struct}
components or to find out more about \vn{bmad_parser} use the
\vn{getf} command as discussed in \sref{s:getf}.

The routine \Hyperref{r:lat.ele.locator}{lat_ele_locator} (\sref{s:lat.ele.find}) is used in line
19 to find the element in the lattice with the name \vn{CLEO_SOL}. Line 20 defines a pointer
variable named \vn{cleo} which is used here as shortcut notation rather than having to write
\vn{eles(1)%ele} when refering to this element. The call to \vn{pointer_to_attribute} in line 21
sets up a pointer structure \vn{a_ptr%r} to point to the \vn{KS} solenoid strength component of
\vn{cleo}. [\vn{a_ptr} also has components \vn{a_ptr%i} and \vn{a_ptr%l} to point to integrer
or logical components if needed. See \sref{s:lat.ele.attribute}.]

Line 22 changes the \vn{ks} solenoid strength of \vn{cleo}. Since an element attribute has been
changed, the call to \Hyperref{r:set.flags.for.changed.attribute}{set_flags_for_changed_attribute}
in line 23 is needed for \bmad to inform \bmad that this attribute has changed and the call to
\Hyperref{r:lattice.bookkeeper}{lattice_bookkeeper} does the necessary lattice bookkeeping
(\sref{s:lat.bookkeeping}).

The call to \Hyperref{r:lat.make.mat6}{lat_make_mat6} in line 25 recalculates the linear transfer
matrix for the \vn{CLEO_SOL} element.

In line 30, the program checks if the lattice is circular (\sref{s:param}) and, if so, uses the
routine \Hyperref{r:twiss.at.start}{twiss_at_start} to multiply the transfer matrices of the
individual elements together to form the 1--turn matrix from the start of the lat back to the
start. From this matrix \vn{twiss_at_start} calculates the Twiss parameters at the start of the
lattice and puts the information into \vn{lat%ele(0)} (\sref{s:twiss}).  The next call, to
\Hyperref{r:twiss.propagate.all}{twiss_propagate_all}, takes the starting Twiss parameters and,
using the transfer matrices of the individual elements, calculates the Twiss parameters at all the
elements. Notice that if the lattice is not circular, The starting Twiss parameters will need to
have been defined in the lattice file.

\index{ele_struct!\%x}
\index{ele_struct!\%key}
\index{ele_struct!\%s}
\index{lat_struct!\%ele(:)}
The program is now ready output some information. Lines 24 through 28
of the program print information on the first 11 elements in the
lattice.  The do-loop is over the array \vn{lat%ele(:)}.  Each element
of the array holds the information about an individual lattice element
as explained in Chapter~\ref{c:lat.struct}. The \vn{lat%ele(0)}
element is basically a marker element to denote the beginning of the
array (\sref{c:sequence}). Using the pointer \vn{ele} to point to the
individual elements (line 37) makes for a cleaner syntax and reduces
typing. The table that is produced is shown in lines 1 through 12 of
\fig{f:output}.  The first column is the element index $i$. The
second column, \vn{ele%name}, is the name of the element. The third
column, key_name(ele%key), is the name of the element
class. \vn{ele%key} is an integer denoting what type of element
(quadrupole, wiggler, etc.) it is.  \vn{key_name} is an array that
translates the integer key of an element to a printable string. The
fourth column, \vn{ele%s}, is the longitudinal position at the exit
end of the element. Finally, the last column, \vn{ele%x%beta}, is the
$a$--mode (nearly horizontal mode) beta function.

The \Hyperref{r:type.ele}{type_ele} routine on line 47 of the program is used to type out the
\vn{CLEO_SOL}'s attributes and other information as shown on lines 14 through 41 of the output (more
on this later).

This brings us to the lattice file used for the input to the program.
The call to \vn{bmad_parser} shows that this file is called 
\vn{simple_bmad_program/lat.bmad}.
In this file there is a call to another file
  \begin{example}
  call, file = "layout.bmad"
  \end{example}
\index{line}
It is in this second file
that the layout of the lattice is defined. In particular, the \vn{line} used
to define the element order looks like
\begin{example}
  cesr: line = (IP_L0, d001, DET_00W, d002, Q00W, d003, ...)
  use, cesr
\end{example}
If you compare this to the listing of the elements in
\fig{f:output} you will find differences. For example, element
\#2 in the program listing is named \vn{CLEO_SOL\B3}. From the
definition of the \vn{cesr} line this should be \vn{d001} which, if
you look up its definition in \vn{layout.bmad} is a drift.  The
difference between lattice file and output is due to the presence
the \vn{CLEO_SOL} element which appears in \vn{lat.bmad}:
\begin{example}
  ks_solenoid    := -1.0e-9 * clight * solenoid_tesla / beam[energy]
  cleo_sol: solenoid, l = 3.51, ks = ks_solenoid, superimpose 
\end{example}
\index{superimpose!example}
The solenoid is 3.51 meters long
and it is superimposed upon the lattice with its center at $s = 0$ (this
is the default if the position is not specified). 
When \vn{bmad_parser} constructs the lattice list of elements
the superposition of \vn{IP_L0}, which is a zero--length marker, with the
solenoid does not modify \vn{IP_L0}. The superposition of the
\vn{d001} drift with the solenoid gives a solenoid with the same
length as the drift. Since this is a ``new'' element, \vn{bmad_parser}
makes up a name that reflects that it is basically a section of the
solenoid it came from.  Next, since the \vn{CLEO_SOL} element happens to
only cover
part of the \vn{Q00W} quadrupole, \vn{bmad_parser} breaks the
quadrupole into two pieces. The piece that is inside the solenoid is a
\vn{sol_quad} and the piece outside the solenoid is a regular
quadrupole. See \sref{s:super} for more details. Since the
center of the \vn{CLEO_SOL} is at $s = 0$, half of it extends to
negative $s$. In this situation, \vn{bmad_parser} will wrap this half
back and superimpose it on the elements at the end of the lattice list
near $s = s_{lat}$ where $s_{lat}$ is the length of the lattice.  As
explained in Chapter~\ref{c:lat.struct}, the lattice list that is used
for tracking extends from \vn{lat%ele(0)} through \vn{lat%ele(n)}
where \vn{n = lat%n_ele_track}. The \vn{CLEO_SOL} element is put in the
section of \vn{lat%ele(n)} with \vn{n > lat%n_ele_track} since it is
not an element to be tracked through. The \vn{Q00W} quadrupole also
gets put in this part of the list.  The bookkeeping information that
the \vn{cleo_sol\B3} element is derived from the \vn{cleo_sol} is put
in the \vn{cleo_sol} element as shown in lines 33 through 41 of the
output.  It is now possible in the program to vary, say, the strength
of the \vn{ks} attribute of the \vn{CLEO_SOL} and have the \vn{ks}
attributes of the dependent (``\vn{super_slave}'') elements updated
with one subroutine call. For example, the following code increases the
solenoid strength by 1\%
\Hyperref{r:lattice.bookkeeper}{lattice_bookkeeper}
\Hyperref{r:lat.ele.locator}{lat_ele_locator}
\begin{example}
  call lat_ele_locator ('CLEO_SOL', lat, eles, n_loc, err)
  eles(1)%ele(ix)%value(ks$) = eles(1)%ele%value(ks$) * 1.01 
  call lattice_bookkeeper (lat)
\end{example}
\bmad takes care of the bookkeeping. In fact \vn{control_bookkeeper} is
automatically called when transfer matrices are remade so the direct call
to \vn{control_bookkeeper} may not be necessary.

Running the program gives the output as shown in \fig{f:output}.

\begin{figure}[ht]
\small
\begin{listing}{1}
  Ix  Name              Ele_type                   S      Beta_a
   0  BEGINNING         BEGINNING_ELE         0.0000      0.9381
   1  IP_L0             MARKER                0.0000      0.9381
   2  CLEO_SOL#3        SOLENOID              0.6223      1.3500
   3  DET_00W           MARKER                0.6223      1.3500
   4  CLEO_SOL#4        SOLENOID              0.6380      1.3710
   5  Q00W\CLEO_SOL     SOL_QUAD              1.7550      7.8619
   6  Q00W#1            QUADRUPOLE            2.1628     16.2350
   7  D003              DRIFT                 2.4934     27.4986
   8  DET_01W           MARKER                2.4934     27.4986
   9  D004              DRIFT                 2.9240     46.6018
  10  Q01W              QUADRUPOLE            3.8740     68.1771

 !---------------------------------------------------------
 ! Information on element: CLEO_SOL 

  Element #         871
  Element Name: CLEO_SOL
  Key: SOLENOID
  S:              1.7550
  Ref_time:   0.0000E+00
 
  Attribute values [Only non-zero values shown]:
      1   L                      =  3.5100000E+00
      7   KS                     = -8.5023386E-02
     32   P0C                    =  5.2890000E+09
     33   E_TOT                  =  5.2890000E+09
     34   BS_FIELD               = -1.5000000E+00
     50   DS_STEP                =  2.0000000E-01
 
          TRACKING_METHOD        =  Bmad_Standard
          MAT6_CALC_METHOD       =  Bmad_Standard
          FIELD_CALC             =  Bmad_Standard
          APERTURE_AT            =  Exit_End
          OFFSET_MOVES_APERTURE  =  F
          INTEGRATOR_ORDER:      =     2
          NUM_STEPS              =    18
          SYMPLECTIFY            =  F
          FIELD_MASTER           =  F
          CSR_CALC_ON            =  T
 
 Lord_status:  SUPER_LORD
 Slave_status: NO_MAJOR_LORDS
 Slaves: Number:   6
      Name                           Lat_index  Attribute           Coefficient
      Q00E\CLEO_SOL                        865  --------              3.182E-01
      CLEO_SOL#1                           866  --------              4.460E-03
      CLEO_SOL#2                           868  --------              1.773E-01
      CLEO_SOL#3                             2  --------              1.773E-01
      CLEO_SOL#4                             4  --------              4.460E-03
      Q00W\CLEO_SOL                          5  --------              3.182E-01
\end{listing}
\caption{Output from the example program}
\label{f:output}
\end{figure}

