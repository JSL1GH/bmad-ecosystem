\chapter{Lattice Concepts and Organization}
\label{c:lat.concepts}

This chapter presents the basic concepts, such as \vn{element},
\vn{branch}, and \vn{lattice}, that \bmad uses to describe such things
as LINACs, storage rings, X-ray beam lines, etc.  In fact, \bmad is
capable of simulating a whole machine complex of interconnected parts.
This includes simulating transfer lines connected to storage rings or 
simulating interconnected colliding beam storage rings.

In addition to simulating the 

Chapter~\sref{c:lat.file} covers the syntax of how to construct a
lattice file for input to \bmad based programs.

%---------------------------------------------------------------------------
\section{Lattice Elements}
\label{s:element.def}

\index{element}\index{marker element}
The basic component \bmad uses to describe a machine is the
\vn{element} (also called a \vn{lattice element}).  An element can be
a physical thing like a bending magnet, a quadrupole or a Bragg
crystal, or something like a \vn{marker} element (\sref{s:mark})
that is used to mark a particular point in the machine.

\index{controller element}
Another class of element are the \vn{controller} elements
(Table~\ref{t:control.classes}) that can be used for parameter control
of other elements (\sref{c:control}).

Chapter~\sref{c:elements} lists the complete set of different element
types that \bmad knows about.

%---------------------------------------------------------------------------
\section{Lattice Branches}
\label{s:branch.def}

\index{branch}
The next level up from an \vn{element} is the \vn{branch}
(\sref{s:branching}).  A \vn{branch}, also called a \vn{lattice
branch} to distinguish it from a \vn{branch} element
(\sref{s:branch}), is just an ordered sequence of elements that a
particle will travel through. A branch can represent a LINAC, X-Ray
line, storage ring or anything else that can be represented as a
simple ordered list of elements. 

Chapter~\sref{c:sequence} shows how a \vn{branch} is defined in a
lattice file with \vn{line}, \vn{list}, and \vn{use} statements.

\index{beginning element}\index{end element}
All elements in a branch are assigned a number starting from zero. The
zeroth \vn{init_ele} (\sref{s:init.ele}) element is automatically
included in every branch and is used as a marker for the beginning of
the branch.  The zeroth element is always named \vn{BEGINNING}.
Additionally, every branch will, by default, have a final marker
element (\sref{s:mark}) named \vn{END}.

%---------------------------------------------------------------------------
\section{Lattice}
\label{s:lattice.def}

\index{lattice}
\vn{branch element} A \vn{lattice} is just an array of branches that
can be interconnected together to describe an entire machine complex.
The array of \vn{branches} in a \vn{lattice} is numbered starting from
zero.

Branches can be interconnected in one of two ways: \vn{branch} and
\vn{photon_branch} elements (\sref{s:branch}) may be used to simulate
forking beam lines (\sref{s:branching}) such as what occurs when there
is a dump line connection or an X-ray beam line which is connected to
a wiggler or undulator. Additionally, \vn{multipass} lines
(\sref{s:multipass}) can be used to simulate branches that share
common elements such as the interaction region in colliding beam
machines.

%---------------------------------------------------------------------------
\section{Lord and Slave Elements}
\label{s:lattice}

Under construction...




