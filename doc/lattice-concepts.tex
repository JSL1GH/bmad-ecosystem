\chapter{Lattice Concepts and Structure}
\label{c:lat.concepts}

This chapter presents the basic concepts, such as \vn{element},
\vn{branch}, and \vn{lattice}, that \bmad uses to describe an
``accelerator''. Here ``accelerator'' can refer to ... machine
complex. Additionally, the organization of the structure used to
represent

\index{element}
The basic component \bmad uses to describe an accelerator is the
\vn{element} An element can be a physical thing like a bending magnet,
quadrupole, etc. or something like a \vn{marker} element that is used
to mark a particular point the machine. Chapter~\sref{c:elements}
lists the complete set of different element types that \bmad knows
about.

\index{branch}
The next level up from an \vn{element} is the \vn{branch}.  A
\vn{branch} is just a sequence of elements that a particle will travel
through. A 


\index{lattice}
A ``lattice'' is just a collection of branches or may sequence of
elements that a particle will travel through. The ``lattice'' may
refer to an enti accelerator complex or may refer to some specific
part.

--> If there is only one branch then it can be ignored.


along with the attributes (length, strength, orientation,
etc.) of the elements.  A lattice file (or files) (\sref{c:lat.file}
is a file that is used to describe an accelerator or storage ring.

\index{beginning element}\index{end element}
\bmad associates a number with each element staring at 1 for the first
element in the lattice, 2 for the second element, etc. Additionally,
\bmad always automatically creates a 0\Th \vn{init_ele}
(\sref{s:init.ele}) element to mark the beginning of the lattice. The
name of this element is always \vn{beginning}.


\index{MAD}
\index{lattice}
To \bmad, a ``lattice'' is the sequence of physical elements that is
to be studied. As explained below, the lattice is constructed in the
input lattice file using what are known as \vn{beam lines} and
\vn{replacement lists}. A simple example:
\begin{example}
  q1: quadrupole
  d1: drift
  my_machine: line = (q1, d1)
  use, my_machine
\end{example}
This lattice has two elements: A quadrupole \vn{q1} and a drift
\vn{d1}.  The line named \vn{my_machine} gives the ordered list of
elements and the \vn{use} statement chooses \vn{my_machine} as the
line to be used for the lattice.

There will also be a marker element named \vn{end} automatically
placed at the end of the lattice. This end marker will not be
automatically placed in the lattice if a marker named \vn{end} is
defined in the lattice file at the end of the lattice. The
\vn{parameter[no_end_marker]} statement (\sref{s:param}) can be used
to suppress the insertion of the end marker.

If there are multiple branches (\sref{s:branching}) in the lattice,
beginning and end markers will be placed at the beginning and end of
all branches.

One is also allowed in \bmad to work with multiple lattices that can
be interconnected togeter using \vn{branching} (\sref{s:branching})
and multipass \vn{s:multipass} to form a description of an accelerator
complex.


