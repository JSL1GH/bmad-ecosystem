\chapter{Variables}
\label{c:var}
\index{variables}

\index{change command}
\index{optimizer!variables}
For the \vn{model} lattice (or lattices if there are multiple \vn{universes}) the
\vn{change} command (\sref{s:change}) can be used to vary lattice parameters such as
element strengths, the initial Twiss parameters, etc.  Additionally, \vn{variables} can be
defined in the \tao initialization files (\sref{s:init.var}) that can also be used to vary
these \vn{model} lattice parameters.  A given \tao variable may control a single attribute
of one element in one or more universes.  There are a few reasons why one would want to
setup such variables.  For example, the optimizer (\sref{c:opti}) will only work with
\tao variables and blocks of these variables can be plotted for visual inspection.

\index{variables!v1_var}
Blocks of variables are associated with what is called a \vn{v1_var}
structure and each of these structures has a \vn{name} with which to
refer to them in \tao commands. For example, if \vn{quad_k1} is the
name of a \vn{v1_var}, then \vn{quad_k1[5]} referees to the variable 
with index 5 in the block. 

A set of variables within a \vn{v1_var} block
can be referred to by using using a comma \vn{,} to
separate their indexes. Additionally, a Colon \vn{:} can be use to
specify a range of variables. For example
\begin{example}
  quad_k1[3:6,23]
\end{example}
refers to variables 3, 4, 5, 6, and 23. Instead of a number, the
associated lattice element name can be used so if, in the above
example, the lattice element named \vn{q01} is associated with
\vn{quad_k1[1]}, etc., then the following is equivalent:
\begin{example}
  quad_k1[q03:q06,q23]
\end{example}
Using lattice names instead of numbers is not valid if the same
lattice element is associated with more than one variable in a
\vn{v1_var} array. This can happen, for example, if one variable controls
an element's \vn{x_offset} and another variable controls the same element's
\vn{y_offset}. 

In referring to variables, a ``\vn{*}'' can be used as a wild card to 
denote ``all''. Thus:
\begin{example}
  *                 ! All the variables
  quad_k1[*]|design ! All design values of quad_k1.
  quad_k1[]|model   ! No values. That is, the empty set.
  quad_k1|model     ! Same as quad_k1[*]|model
\end{example}

A given variable may control a single attribute of one element in a
\vn{model} lattice of a single universe or it can be configured to
simultaneously control an element attribute across multiple
universes. Any one variable cannot control more than one attribute of
one element. However, a variable may control an overlay or group
element which, in turn, can control numerous elements.

Each individual variable has a number of values associated with it:
The list of components that can be set or refereed to are:
\begin{example}
  ele_name     ! Associated lattice element name.
  attrib_name  ! Name of the attribute to vary.
  ix_attrib    ! Index in ele%value(:) array if appropriate.
  s            ! longitudinal position of ele.

  meas         ! Value of variable at time of a data measurement.
  ref          ! Value at time of the reference data measurement.
  model        ! Value in the model lattice.
  base         ! Value in the base lattice.
  design       ! Value in  the design lattice.
  correction   ! Value determined by a fit to correct the lattice.
  old          ! Scratch value.

  weight       ! Weight used in the merit function.
  delta_merit  ! Diff used to calculate the merit function term.
  merit        ! merit_term = weight * delta^2.
  merit_type   ! 'target' or 'limit'
  dMerit_dVar  ! Merit derivative.

  high_lim     ! High limit for the model_value.
  low_lim      ! Low limit for the model_value.
  step         ! For fitting/optimization: What is considered a small change.

  key_bound    ! Variable bound to keyboard key?
  ix_key_table ! Has a key binding?

  ix_v1        ! Index of this var in the s%v1_var(i)%v(:) array.
  ix_var       ! Index number of this var in the s%var(:) array.
  ix_dvar      ! Column in the dData_dVar derivative matrix.

  exists       ! Does the variable exist?
  good_var     ! The variable can be varied (set by \tao).
  good_user    ! The variable can be varied (set by the user).
  good_opt     ! For use by extension code.
  good_plot    ! For use by extension code
  useit_opt    ! Variable is to be used for optimizing.
  useit_plot   ! Variable is to be used for plotting.
\end{example}

  \index{variable!measured}\index{variable!reference}
  \index{variable!model}\index{variable!design}\index{variable!base}
  \begin{description}
  \item[attrib_name] \Newline
Name of the element attribute to vary. To see a list of attributes for a given element
consult the \bmad manual or use the \vn{show element -attributes} command.
  \item[base] \Newline
The value of the variable as derived from the \vn{base} lattice (\sref{s:universe}).
  \item[delta_merit] \Newline
Difference value used to calculate the contribution of the variable to the merit function (\Eq{m1}).
  \item[design] \Newline
The value of the variable as given in the \vn{design} lattice.
  \item[dMerit_dVar] \Newline
Derivative of the merit function with respect to the variable.
  \item[ele_name] \Newline
Associated lattice element name.
  \item[exists] \Newline
The variable exists. Non-existent variables can serve as place holders in a variable
array.
  \item[good_opt] \Newline
Logical not modified by Tao proper and reserved for use by extension code. See below.
  \item[good_plot] \Newline
Logical not modified by Tao proper and reserved for use by extension code. See below.
  \item[good_var] \Newline
Logical controlled by \tao and used to veto variables that should not be varied during
optimization. For example, variables that do not affect the merit function. See below.
  \item[good_user] \Newline
Logical set by the user using \vn{veto}, \vn{use}, and \vn{restore} commands to indicate
whether the variable should be used when optimizing. See below.
  \item[high_lim] \Newline
High limit for the model value during optimization (\sref{s:generalized.design}) beyond which
the contribution of the variable to the merit function is nonzero.
  \item[ix_attrib] \Newline
Index assigned by \bmad to the attribute being controlled. Used for diagnosis and not
of general interest.
  \item[ix_dvar] \Newline
Column index of the variable in the dData_dVar derivative matrix constructed by \tao.
Used for diagnostics and not of general interest.
  \item[ix_key_table] \Newline
Index of the variable in the key table (\sref{s:key.bind}).
  \item[ix_v1] \Newline
Index of this variable in the variable array of the associated \vn{v1_var} variable.
For example, a variable named \vn{q1_quad[10]} would have \vn{ix_v1} equal to 10.
  \item[ix_var] \Newline
For ease of computation, \tao establishes an array that holds all the variables.
\vn{ix_var} is the index number for this variable in this array. 
Used for diagnostics and not of general interest.
  \item[key_bound] \Newline
Variable bound to keyboard key (\sref{s:key.bind})?
  \item[measured] \Newline
The value of the variable as obtained at the time of a \vn{data} measurement.
  \item[merit] \Newline
The contribution to the merit function \Eq{m1} from the variable. Use the \vn{show top10}
command to set the variables and data which contribute most to the merit function.
  \item[merit_type] \Newline
T'target' or 'limit'
  \item[low_lim] \Newline
Lower limit for the model value during optimization (\sref{s:generalized.design}) beyond which
the contribution of the variable to the merit function is nonzero.
  \item[model] \Newline
The value of the variable as given in the \vn{model} lattice.
  \item[reference] \Newline
The Value of the variable as obtained at the time of a \vn{reference} data measurement
(\sref{s:lattice.correction}.
  \item[s] \Newline
longitudinal position of element whose attribute the variable is controlling.  Since a
variable may control multipole attributes in multiple elements at different s-positions,
The value of \vn{s} may not be relevant.
  \item[step] \Newline
What is considered a small change in the variable but large enough to be able to compute
derivatives by changing the variable by \vn{step}. Used for fitting/optimization.  
  \item[useit_opt] \Newline
Variable is to be used for optimization. See below.
  \item[useit_plot] \Newline
If True, variable is used when plotting variable values. See below.
  \item[weight] \Newline
Weight used in the merit function. $w_j$ in \Eq{m1}
  \end{description}


These components and others can be refereed to in expressions using the notation documented
in \Sref{s:var.token}.

Use the \vn{show var} (\sref{s:show}) command to see variable information

When using optimization for lattice correction or lattice design
(\sref{c:opti}), Individual variables can be excluded from the process
using the \vn{veto} (\sref{s:veto}), \vn{restore} (\sref{s:restore}),
and \vn{use} (\sref{s:use}) commands. These set the \vn{good_user}
component of a variable. This, combined with the setting \vn{exists},
\vn{good_var}, and \vn{good_opt} determine the setting of
\vn{useit_opt} which is the component that determines if the datum is
used in the computation of the merit function. 
\begin{example}
  useit_plot = exists \& good_user \& good_opt \& good_var
\end{example}
The settings of everything but \vn{good_user} and \vn{good_opt} is determined by \tao

If the \vn{useit_plot} component is True, the variable is used when when plotting
variables and is not used when \vn{useit_plot} is False. \vn{useit_plot} is set
by \tao using the state of three other components:
\begin{example}
  useit_plot = exists \& good_plot \& good_user
\end{example}
