\chapter{Electromagnetic Fields}

%-----------------------------------------------------------------
\section{Magnetic Static Fields}
\label{s:mag.field}
\index{magnetic fields|hyperbf}

\index{MAD}
Start with the assumption that the local magnetic field has no
longitudinal component (obviously this assumption does not work with,
say, a solenoid).  Following \mad, the vertical magnetic field along
the $y = 0$ axis is expanded in a Taylor series
\Begineq
  B_y(x, 0) = \sum_n B_n \, \frac{x^n}{n!}
  \label{byx0b}
\Endeq
This is not the most
general form for the magnetic field. Essentially all of the skew
components have been ignored here. Assuming that the
reference orbit is locally straight (there are correction terms if the
Reference Orbit is locally curved), the field up to $3^{rd}$ order is
\begin{alignat}{5}
  B_x &=           &&B_1 y \plus         &&B_2 \, xy       
                   && \plus && \frac{1}{6} B_3 (3x^2 y - y^3) \plus \ldots \\
  B_y &= B_0 \plus &&B_1 x + \frac{1}{2} &&B_2 (x^2 - y^2) 
                   && \plus && \frac{1}{6} B_3 (x^3 - 3x y^2) \plus \ldots
\end{alignat}
The normalized integrated multipole $K_nL$ is used when specifying magnetic
multipole components
\index{multipole!KnL, Tn|hyperbf}
\Begineq
  K_nL \equiv \frac{q \, L \, B_n}{P_0}
\Endeq
$L \, B_n$ is the integrated multipole component over a length $L$,
and $P_0$ is the reference momentum. Note that $P_0/q$ is sometimes
written as $B\rho$. This is just an old notation where $\rho$ is the
bending radius of a particle with the reference energy in a field of
strength $B$. [Note: If $q$ is in Coulombs, $P_0$ is in
kg*m/sec. If $q$ is in units of the elementary charge, $P_0$ is in
eV/c_light.]

Conventionally, \bmad always takes the sign of the charge $q$ to be
positive. This means that, for example, positive $K_1$ and positive
$B_1$ are always associated with horizontally focusing quadrupoles for
positrons as well as electrons. The only element where the sign of the
particle's charge matters is an \vn{elseparator} element (\sref{s:elsep}).

The kicks $\Delta p_x$ and $\Delta p_y$ that a
particle experiences going through a multipole field is
\begin{alignat}{5}
  \Delta p_x & = \frac{-q \, L \, B_y}{P_0} \label{pqlbp1} \\
             & = -K_0 L \;-\; 
             && K_1 L \, x \plus 
             \frac{1}{2} && K_2 L (y^2 - x^2) && \plus 
             && \frac{1}{6} K_3 L (3x y^2 - x^3) \plus \ldots 
             \nonumber \\
  \Delta p_y & = \frac{q \, L \, B_x}{P_0} \label{pqlbp2} \\
             & =     
             && K_1 L \, y \plus 
             && K_2 L \, xy && \plus 
             && \frac{1}{6} K_3L (3x^2 y - y^3) \plus \ldots \nonumber 
\end{alignat}
A positive $K_1L$ quadrupole component gives
horizontal focusing and vertical defocussing. The general form is
\begin{align}
  \Delta p_x &= \sum_{n = 0}^{\infty} \frac{K_n L}{n!} 
             \sum_{m = 0}^{\lfloor \frac{n}{2} \rfloor}
             \begin{pmatrix} n \cr 2m \end{pmatrix} \,
             (-1)^{m+1} \, x^{n-2m} \, y^{2m} \\
  \Delta p_y &= \sum_{n = 0}^{\infty} \frac{K_n L}{n!} 
             \sum_{m = 0}^{\lfloor \frac{n-1}{2} \rfloor}
             \begin{pmatrix} n \cr 2m+1 \end{pmatrix} \,
             (-1)^{m} \, x^{n-2m-1} \, y^{2m+1}
\end{align}

\index{multipole!KnL, Tn|hyperbf}
So far only the normal components of the field have been
considered. If the fields associated with a particular $B_n$ multipole
component are rotated in the $(x, y)$ plane by an angle $\theta_n$, the
magnetic field at a point $(x,y)$ can be expressed in complex notation
as
\Begineq
  B_y(x,y) + i B_x(x,y) = 
    \frac{1}{n!} B_n e^{-i(n+1)\theta_n} \, e^{i n \theta} \, r^n 
  \label{bib1nb}
\Endeq
where $(r, \theta)$ are the polar coordinates of the point $(x, y)$.

Instead of using magnitude $K_n$ and rotation angle $\theta_n$,
Another representation is using normal $\wt K_n$ and skew $\wt
KS_n$. The conversion between the two are
\begin{align}
  \wt K_n  &= K_n \, \cos((n + 1) \theta_n) \CRNO
  \wt KS_n &= K_n \, \sin((n + 1) \theta_n) 
\end{align}

\index{multipole!an, bn|hyperbf}
Another representation of the magnetic field used by \bmad divides
the fields into normal $b_n$ and skew $a_n$ components. In terms of
these components the magnetic field for the $n$\Th\ order multipole is
\Begineq
  \frac{q \, L}{P_0} \, (B_y + i B_x) = (b_n + i a_n) \, (x + i y)^n
  \label{qlpbb}
\Endeq
The conversion between $(a_n, b_n)$ and $(K_nL, \theta_n)$ is
\Begineq
  b_n + i a_n = \frac{1}{n!} \, K_nL \, e^{-i(n+1)\theta_n}
\Endeq
or
\begin{align}
  K_n L &= n! \, \sqrt{a_n^2 + b_n^2} \\
  \tan[(n+1) \theta_n] &= \frac{-a_n}{b_n}
\end{align}
To convert a normal magnet (a magnet with no skew component) into a skew
magnet (a magnet with no normal component) the magnet should be rotated
about its longitudinal axis with a rotation angle of
\Begineq
  (n+1) \theta_n = \frac{\pi}{2}
\Endeq
For example, a normal quadrupole rotated by $45^\circ$ becomes a
skew quadrupole.

\index{ab_multipole}
\index{radius}
The $a_n$, $b_n$ representation of multipole fields is in
\vn{AB_Multipole} elements (\sref{s:ab.m}) as well as in other types
of elements such as quadrupoles, sextupoles, etc. This allows error
fields to be represented.  When $a_n$ and $b_n$ multipole values are
associated with an element that is not an \vn{AB_Multipole} element,
and if the \vn{scale_multipoles} attribute (\sref{s:multip}) is not
set to \vn{False}, a measurement radius $r_0$ and a scale factor $F$
are used to scale the effect of the $a_n$ and $b_n$ so that the
multipole strength scales as the element strength. The scaling formula
is
\Begineq
  \bigl[ a_n (\text{actual}), b_n (\text{actual}) \bigr] =
  \bigl[ a_n (\text{input}), b_n (\text{input}) \bigr] 
  \cdot F \cdot \frac{r_0^{n_\text{ref}}}{r_0^n} 
  \label{ababf}
\Endeq
$a_n(\text{input})$ and $b_n(\text{input})$ are the multipole values as given in the
lattice file. $a_n(\text{actual})$ and $b_n(\text{actual})$ are the multipole values
that are used in any simulation calculations. $r_0$ is set by the
\vn{radius} attribute of an element. $F$ and $n_\text{ref}$ are set
automatically depending upon the type of element as shown in
Table~\ref{t:ab}.

\index{ab_multipole}
\index{multipole}
Note that the $n = 0$ component of an \vn{AB_Multipole} or \vn{Multipole}
element rotates the reference orbit essentially acting as a zero length bend.
This is not true for multipoles that are associated with 
non-multipole elements.

\index{kicker}
\index{hkicker}
\index{vkicker}
\index{rbend}
\index{sbend}
\index{elseparator}
\index{quadrupole}
\index{solenoid}
\index{sol_quad}
\index{sextupole}
\index{octupole}
\begin{table}[ht]
\centering
\begin{tabular}{|l|l|l|} \hline
\tt
  {\em Element} & $F$                              & $n_\text{ref}$ \\ \hline
  \vn{Elseparator} & $\sqrt{{\tt Hkick}^2 + {\tt Vkick}^2}$ & 0 \\
  \vn{Hkicker}     & Kick                                   & 0 \\
  \vn{Kicker}      & $\sqrt{{\tt Hkick}^2 + {\tt Vkick}^2}$ & 0 \\
  \vn{Rbend}       & G * L                                  & 0 \\
  \vn{Sbend}       & G * L                                  & 0 \\
  \vn{Vkicker}     & Kick                                   & 0 \\
  \vn{Wiggler}     & $\dsfrac{2 \, c \, {\tt L\_pole} \, B_{max}}{\pi \, {\tt p0c}}$ 
                                                            & 0 \\
  \vn{Quadrupole}  & K1 * L                                 & 1 \\
  \vn{Sol_Quad}    & K1 * L                                 & 1 \\
  \vn{Solenoid}    & KS * L                                 & 1 \\
  \vn{Sextupole}   & K2 * L                                 & 2 \\
  \vn{Octupole}    & K3 * L                                 & 3 \\ \hline
\end{tabular}
\caption{$F$ and $n_\text{ref}$ for various elements.}
\label{t:ab}
\end{table}

%-----------------------------------------------------------------
\section{Electric Static Fields}
\label{s:elec.field}
\index{electric fields}

Except for the \vn{elseparator} element, \bmad specifies DC electric
fields using normal $b_{en}$ and skew $a_{en}$ components. The
potential $\phi_n$ for the $n$\Th\ order multipole is
\Begineq
  \phi_n = -\re \left[ \frac{b_{en} - i a_{en}}{n + 1} \, (x + i y)^{n+1} \right]
\Endeq
and the electric field for the $n$\Th order
multipole is
\Begineq
  E_x - i E_y = (b_{en} - i a_{en}) \, (x + i y)^n
\Endeq
Notice that the magnetic multipole components $a_n$ and $b_n$ are
normalized by the element length and reference momentum (\Eq{qlpbb})
while their electric counterparts are not.

Using the paraxial approximation, The kick given a particle due to the electric field is
\Begineq
  \frac{dp_x}{ds} = \frac{q \, E_x}{P_0}, \qquad \frac{dp_y}{ds} = \frac{q \, E_y}{P_0}
\Endeq

%-----------------------------------------------------------------
\section{RF Fields}
\label{s:rf.fields.phys}
\index{rf fields|hyperbf}

The RF fields in an RF cavity can be parameterized as the sum over a
number of modes. This parameterization used by \bmad essentially
follows Abell\cite{b:rf.abell}. The electric field is written in the
form
\Begineq
  \bfE(\bfr) = \sum_{j=1}^M \, \bfE_j(\bfr) \, e^{-i \, (\omega_j \, t + \theta_{0j})}
  \label{eseei}
\Endeq
where $M$ is the number of modes. Each mode satisfies the vector Helmholtz
equation
\Begineq
  \nabla^2 \bfE_j + k_{tj}^2 \, \bfE_j = 0
  \label{bke}
\Endeq
where $k_{tj} = \omega_j/c$ with $\omega_j$ being the mode frequency.

For a given mode, the electric field can be characterized by the values on 
a cylindrical surface of some radius $R$. The field will be of the form
\begin{align}
  E_\rho(R, \phi, z) &= E_{\rho c}(R, z) \, \cos(m \, \phi - \phi_0) \CRNO
  E_\phi(R, \phi, z) &= E_{\phi s}(R, z) \, \sin(m \, \phi - \phi_0) \\
  \label{erpze}
  E_z(R, \phi, z)    &= E_{z c}(R, z)    \, \cos(m \, \phi - \phi_0) \nonumber
\end{align}
where in this, and subsequent equations, the mode index $j$ has been
dropped.

The azimuthal mode number $m_j$ is a non-negative integer. If $m_j
= 0$, there are two different modes, one with $\phi_{0j} = 0$ and
the other with $\phi_{0j} = -\pi/2$. The $\phi_{0j} = 0$ mode is an
accelerating mode with the electric field is in the form
\begin{align}
  E_\rho(\bfr) &= \sum_{n=-N/2+1}^{N/2} -e^{i \, k_{zn} \, z} \, 
    i \, k_{zn} \, e_0(n) \, \wt I_1(\kappa_n, \rho) \CRNO
  E_\phi(\bfr) &= 0 \\
  E_z(\bfr) &= \sum_{n=-N/2+1}^{N/2}e^{i \, k_{zn} \, z} \, 
    e_0(n) \, \wt I_0(\kappa_n, \rho) \nonumber
\end{align}
where $\wt I_m$ is
\Begineq
  \wt I_m (\kappa_n, \rho) \equiv \frac{I_m(\kappa_n \, \rho)}{\kappa_n^m}
\Endeq
with $I_m$ being a modified Bessel function and $\kappa_n$ is given by
\Begineq
  \kappa_n = \sqrt{k_{zn}^2 - k_t^2} = 
  \begin{cases}
    \sqrt{k_{zn}^2 - k_t^2} & |k_{zn}| > k_t \\
    -i \, \sqrt{k_t^2 - k_{zn}^2} & k_t > |k_{zn}|
  \end{cases}
\Endeq
with
\Begineq
  k_{zn} = \frac{2 \pi \, n}{N \, dz}
\Endeq
$N$ is the number of points where $E_{zc}$ is evaluated, and $dz$ is
the distance between points. When $\kappa_n$ is imaginary, $I_m(\kappa_n \, \rho)$
can be evaluated through the relation
\Begineq
  I_m(-i \, x) = i^{-m} \, J_m(x)
\Endeq
where $J_m$ is a Bessel function of the first kind.
The $e_0$ coefficients are obtained from the equation
\begin{align}
  e_0(n) &= \frac{1}{\wt I_0(\kappa_n, R)} \, \frac{1}{N} \, \sum_{p=0}^{N-1}
    e^{-2 \, \pi \, i \, n \, p / N} \, E_{zc}(R, p \, dz) \CRNO
  &\equiv \frac{\wt E_{zc}(R, n)}{\wt I_0(\kappa_n, R)} 
\end{align}

The non-accelerating $m = 0$ mode with $\phi_0 = -\pi/2$ has an
electric field in the form
\begin{align}
  E_\rho(\bfr) &= E_z(\bfr) = 0 \CRNO
  E_\phi(\bfr) &= \sum_{n=-N/2+1}^{N/2}e^{i \, k_{zn} \, z} \, 
    b_0(n) \, \wt I_1(\kappa_n, \rho)
\end{align}
where
\Begineq
  b_0(n) = \frac{\wt E_{\phi s}(R, n)}{\wt I_1(\kappa_n, R)} 
\Endeq

For positive $m$, the electric field is in the form
\begin{align}
  E_\rho(\bfr) &= \sum_{n=-N/2+1}^{N/2}
    -i \, e^{i \, k_{zn} \, z} \, 
    \left[ 
    k_{zn} \, e_m(n) \, \wt I_{m+1}(\kappa_n, \rho) +
    b_m(n) \, \frac{\wt I_m(\kappa_n, \rho)}{\rho}
    \right]
    \cos(m \, \phi - \phi_0) \CRNO
  E_\phi(\bfr) &= \sum_{n=-N/2+1}^{N/2} 
    -i \, e^{i \, k_{zn} \, z} \, 
    \left[
    k_{zn} \, e_m(n) \, \wt I_{m+1}(\kappa_n, \rho) \, + \right. \\
  & \left. \qquad \qquad \qquad \qquad \qquad \qquad
    b_m(n) \, \left( \frac{\wt I_m(\kappa_n, \rho)}{\rho} - 
    \frac{1}{m} \, \wt I_{m-1}(\kappa_n, \rho) \right)
    \right] 
    \sin(m \, \phi - \phi_0) \CRNO
  E_z(\bfr) &= \sum_{n=-N/2+1}^{N/2}e^{i \, k_{zn} \, z} \, 
    e_m(n) \, \wt I_m(\kappa_n, \rho) \cos(m \, \phi - \phi_0) \nonumber
\end{align}
with
\begin{align}
  e_m(n) &= \frac{\wt E_{zc}(R, n)}{\wt I_m(\kappa_n, R)} \CRNO
  b_m(n) &= \frac{R}{\wt I_m(\kappa_n, R)} \left[
    i \, \wt E_{\rho c} - k_{zn} \, e_m(n) \, \wt I_{m+1}(\kappa_n, R)
    \right]
\end{align}

The above mode decomposition was done in the gauge where the scalar
potential $\psi$ is zero. The electric and magnetic fields are thus
related to the vector potential $\bfA$ via
\Begineq
  \bfE = -\partial_t \, \bfA, \qquad \bfB = \nabla \times \bfA
\Endeq
Using \Eq{eseei}, the vector potential can be obtained from the
electric field via
\Begineq
  \bfA_j = \frac{-i \, \bfE_j}{\omega_j}
  \label{aiew}
\Endeq
 
Symplectic tracking through the RF field is discussed in
Section~\sref{s:symp.track}.  For the fundamental accelerating mode,
The vector potential can be analytically integrated using the identity
\Begineq
  \int dx \,\frac{x \, I_1 (a \, \sqrt{x^2+y^2})}{\sqrt{x^2+y^2}}  = 
  \frac{1}{a} \, I_0 (a \, \sqrt{x^2+y^2})
\Endeq

%-----------------------------------------------------------------
\section{Wake fields}
\label{s:wake.fields}
\index{wake fields|hyperbf}

%-----------------------------
\subsection{Short--Range Wakes}
\index{wake fields!short-range}

Wake field effects are divided into short--range (within a bunch) and
long--range (between bunches).

Only the transverse dipole and longitudinal monopole components of the
short--range wake field are modeled. The longitudinal monopole energy
kick $dE$ for the $i$\Th (trailing) macroparticle due to the wake from
the $j$\Th (leading) macroparticle is computed from the equation
\Begineq
  \Delta p_z(i) = \frac{-e \, L}{v \, P_0} \, \left(
        \frac{1}{2}\WlS(0) \,  |q_i| +
        \sum_{j \ne i} \WlS(dz_{ij}) \, |q_j| \right)
  \label{delvp}
\Endeq
where $v$ is the particle velocity, $e$ is the charge on an electron,
$q$ is the macroparticle charge, $L$ is the cavity length, $dz_{ij}$
is the longitudinal distance between the $i$\Th and $j$\Th
macroparticles, $\WlS$ is the short--range longitudinal wake field
function.

For the transverse kick, there are two types of wakes: Those that are
dependent upon transverse offset of the leading particle (but
independent of the position of the trailing particle) and those that
are dependent upon the transverse offset of the trailing particle (but
independent of the position of the leading particle. If the beam
chamber has azimuthal symmetry, the only wakes present are those that
are dependent upon the offset of the leading particle. 

\bmad assumes that the beam chamber wall is mirror symmetric with
respect to both the $x$-axis and $y$-axis. In this case, a horizontal
displacement (of either particle) results in a horizontal kick and
similarly for vertical displacements. That is, the horizontal and
vertical wakes are decoupled. 

With the above symmetry assumpltion, 
the transverse kick $\Delta p_x(i)$ for the $i$\Th macroparticle due to the 
dipole short--range transverse wake field is modeled with the equation
\Begineq
  \Delta p_x(i) = \frac{-e \, L \, \sum_j |q_j| \, x \, \WtS(dz_{ij})}
                 {v \, P_0}
  \label{pelqxw}
\Endeq
Where $x$ is the horizontal displacement of the leading or trailing
particle as appropriate. There is a similar equation for $\Delta
p_y(i)$. $\WtS$ is the transverse short--range wake function.

The wake field functions $\WlS$ and $\WtS$ can be specified in a \bmad
lattice file using ``pseudo'' modes where
\Begineq
  W(z) = \sum_i A_i \, e^{d_i z} \, \sin (k_i \, z + \phi_i)
  \label{wadzk}
\Endeq
The parameters $(A_i, d_i, k_i, \phi_i)$ are chosen to fit the
calculated wake potential (\sref{ss:short.range.wakes}). The reason
why the mode approach is used in \bmad is due to the fact that, using
pseudo modes, the calculation time scales as the number of particles
$N$ while a calculation based upon a table of wake vs $z$ would scale
as $N^2$. [The disadvantage is that initially the user must proform a
fit to the wake potential to generate the mode parameter values.]

%-----------------------------
\subsection{Long--Range Wakes}
\index{wake fields!long-range}

Following Chao\cite{b:chao} Eq.~2.88, the long--range wake fields are
characterized by a set of cavity modes. The wake function $W_m$ for a
mode of order $m$ is given by
\Begineq
  W_m(t) = -c \, \left( \frac{R}{Q} \right)_m \,\,
  e^{-\omega \, t/ 2 Q} \, \sin (\omega \, t)
  \label{wcrq}
\Endeq
The mode strength $(R/Q)_m$ has units of Volts/meter$^{2m}$.

Assuming that the macroparticle generating the wake is offset a
distance $r_w$ along the $x$--axis, a trailing macroparticle will see a kick
\begin{align}
  \Delta {\Bf p}_\perp &= 
    -C \, I_m \, W_m(t) \, m \, r^{m-1} \, \left( 
    \bfhat r \cos m\theta - {\bf\hat\theta} \sin m\theta \right) \\
  &= -C \, I_m \, W_m(t) \, m \, r^{m-1} \, \left( 
    \bfhat x \cos [(m-1) \theta] - 
    \bfhat y \sin [(m-1)\theta] \right) \CRNO
  \Delta p_z &= -C \, I_m \, W'_m(t) \, r^m \, \cos m\theta
\end{align}
where $m$ is the order of the mode, $C$ is given by
\Begineq
  C = \frac{e}{c \, P_0}
\Endeq
 and
\Begineq
  I_m = q_w \, r_w^m
\Endeq
with $q_w$ being the magnitude of the charge on the particle. 
Generalizing the above, a macroparticle at $(r_w, \theta_w)$ will generate a wake
\begin{align}
  -\Delta p_x + i\Delta p_y &= C \, I_m \, W_m(t) \, 
    m \, r^{m-1} \, e^{-i m \theta_w} \, e^{i (m-1) \theta} 
    \label{ppcimr} \\
  \Delta p_z &= C \, I_m \, W'_m(t) \, r^m \, \cos [m(\theta - \theta_w)]
    \label{pciwr}
\end{align}
Comparing \Eq{ppcimr} to \eq{bib1nb}, and using the relationship between
kick and field as given by \eq{pqlbp1} and \eq{pqlbp2}, shows that
the form of the wake field transverse kick is the same as for a
multipole of order $n = m - 1$. 

The wake field felt by a particle is due to the wake fields generated by
all the particles ahead of it. If the wake field kicks are computed by
summing over all particle pairs, the
computation will scale as $N^2$ where $N$ is the number of
particles. This quickly becomes computationally exorbitant. A better
solution is to keep track of the wakes in a cavity. When a particle
comes through, the wake it generates is simply added to the existing
wake. This computation scales as $N$ and makes simulations with large
number of particles practical. 

To add wakes together, a wake must be decomposed into its
components.  Spatially, there are normal and skew components and
temporally there are sin and cosine components. This gives 4
components which will be labeled $a_{\cos}$, $a_{\sin}$, $b_{\cos}$,
and $b_{\sin}$. For a mode of order $m$, a particle passing through at
a time $t_w$ with respect to the reference particle will produce
wake components
\begin{alignat}{2}
  \delta a_{\sin, m} &=  &c \, \left( \frac{R}{Q} \right)_m \,
    e^{\omega \, t_w/ 2 Q} \, \cos (\omega \, t_w) \, I_m \, \sin(m \theta_w) 
    \CRNO
  \delta a_{\cos, m} &= -&c \, \left( \frac{R}{Q} \right)_m \,
    e^{\omega \, t_w/ 2 Q} \, \sin (\omega \, t_w) \, I_m \, \sin(m \theta_w) 
    \label{ac2rq} 
    \\
  \delta b_{\sin, m} &=  &c \, \left( \frac{R}{Q} \right)_m \,
    e^{\omega \, t_w/ 2 Q} \, \cos (\omega \, t_w) \, I_m \, \cos(m \theta_w) 
    \CRNO
  \delta b_{\cos, m} &= -&c \, \left( \frac{R}{Q} \right)_m \,
    e^{\omega \, t_w/ 2 Q} \, \sin (\omega \, t_w) \, I_m \, \cos(m \theta_w) 
    \nonumber
\end{alignat}
These are added to the existing wake components. The total is
\Begineq
  a_{\sin, m} = \sum_{\text{particles}} \delta a_{\sin, m}
\Endeq
with similar equations for $a_{\cos, m}$ etc. Here the sum is over all particles
that cross the cavity before the kicked particle. To calculate the kick
due to wake, the normal and skew components are added together
\begin{align}
  a_m &= e^{-\omega \, t/ 2 Q} \, \left( 
    a_{\cos, m} \, \cos (\omega \, t) - a_{\sin, m} \, \sin (\omega \, t) \right) 
    \label{akz2q} \\
  b_m &= e^{-\omega \, t/ 2 Q} \, \left(
    b_{\cos, m} \, \cos (\omega \, t) - b_{\sin, m} \, \sin (\omega \, t) \right) \nonumber 
\end{align}
Here $t$ is the passage time of the particle with respect to the
reference particle. In analogy to \Eq{ppcimr} and \eq{pciwr}, the kick
is
\begin{align}
  -\Delta p_x + i\Delta p_y &= C \, 
    m \, (b_m + i a_m) \, r^{m-1} \, e^{i (m-1) \theta} 
    \label{ppcmbar} \\
  \Delta p_z &= -C \, r^m \, \left( 
    (b_m' + i a_m') e^{i m\theta} + (b_m' - i a_m') e^{-i m\theta} \right)
\end{align}
where $a' \equiv da/dt$ and $b' \equiv db/dt$.

When simulating trains of bunches, the exponential factor $\omega \, t_w / 2
Q$ in \Eq{ac2rq} can become very large. To prevent numerical overflow,
\bmad uses a reference time $z_{\text{ref}}$ so that all times
$t$ in the above equations are replaced by
\Begineq
  t \longrightarrow t - t_{\text{ref}}
\Endeq

The above equations were developed assuming cylindrical symmetry. With
cylindrical symmetry, the cavity modes are actually a pair of
degenerate modes. When the symmetry is broken, the modes no longer have the
same frequency. In this case, one has to consider a mode's polarization
angle $\phi$. Equations \eq{akz2q} and \eq{ppcmbar} are unchanged. 
In place of \Eq{ac2rq}, the contribution of a particle to a mode is
\begin{alignat}{2}
  \delta a_{\sin, m} &=  &c \, \left( \frac{R}{Q} \right)_m \,
    e^{\omega \, t_w/ 2 Q} \, \cos (\omega \, t_w) \, I_m \, \left[
    \sin(m \theta_w) \, \sin^2(m \phi) + 
    \cos(m \theta_w) \, \sin(m \phi) \, \cos(m\phi) \right]
    \CRNO
  \delta a_{\cos, m} &= -&c \, \left( \frac{R}{Q} \right)_m \,
    e^{\omega \, t_w/ 2 Q} \, \sin (\omega \, t_w) \, I_m \, \left[ 
    \sin(m \theta_w) \, \sin^2(m \phi) + 
    \cos(m \theta_w) \, \sin(m \phi) \, \cos(m\phi) \right]
    \\
  \delta b_{\sin, m} &=  &c \, \left( \frac{R}{Q} \right)_m \,
    e^{\omega \, t_w/ 2 Q} \, \cos (\omega \, t_w) \, I_m \, \left[
    \cos(m \theta_w) \, \cos^2(m \phi) + 
    \sin(m \theta_w) \, \sin(m \phi) \, \cos(m\phi) \right]
    \CRNO
  \delta b_{\cos, m} &= -&c \, \left( \frac{R}{Q} \right)_m \,
    e^{\omega \, t_w/ 2 Q} \, \sin (\omega \, t_w) \, I_m \, \left[
    \cos(m \theta_w) \, \cos^2(m \phi) + 
    \sin(m \theta_w) \, \sin(m \phi) \, \cos(m\phi) \right]
    \nonumber
\end{alignat}

Each mode is characterized by an $R/Q$, $Q$, $\omega$, and $m$. Notice
that $R/Q$ is defined so that it includes the cavity length. Thus the
long--range wake equations, as opposed to the short--range ones, do
not have any explicit dependence on $L$. 

To make life more interesting, different people define $R/Q$
differently. A common practice is to define an $R/Q$ ``at the beam
pipe radius''. In this case the above equations must be modified to
include factors of the beam pipe radius. Another convention uses a
``linac definition'' which makes $R/Q$ twice as large and adds a
factor of 2 in \Eq{wcrq} to compensate.

