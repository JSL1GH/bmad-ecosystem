\chapter{Electromagnetic Fields}

%-----------------------------------------------------------------
\section{Magnetostatic Multipole Fields}
\label{s:mag.field}
\index{magnetic fields|hyperbf}

\index{MAD}
Start with the assumption that the local magnetic field has no longitudinal component
(obviously this assumption does not work with, say, a solenoid).  Following \mad, ignoring
skew fields for the moment, the vertical magnetic field along the $y = 0$ axis is expanded
in a Taylor series
\Begineq
  B_y(x, 0) = \sum_n B_n \, \frac{x^n}{n!}
  \label{byx0b}
\Endeq
Assuming that the reference orbit is locally straight (there are correction terms if the
Reference Orbit is curved), the field is
\begin{alignat}{5}
  B_x &=           &&B_1 y \plus         &&B_2 \, xy       
                   && \plus && \frac{1}{6} B_3 (3x^2 y - y^3) \plus \ldots \\
  B_y &= B_0 \plus &&B_1 x + \frac{1}{2} &&B_2 (x^2 - y^2) 
                   && \plus && \frac{1}{6} B_3 (x^3 - 3x y^2) \plus \ldots
\end{alignat}
For some fields, the normalized integrated multipole $K_nL$ can be used when specifying
magnetic multipole components
\index{multipole!KnL, Tn|hyperbf}
\Begineq
  K_nL \equiv \frac{q \, L \, B_n}{P_0}
\Endeq
where $q$ is the charge of the reference particle (in units of the elementary charge), $L$ is the
element length, and $P_0$ is the reference momentum (in units of eV/c).  Note that $P_0/q$ is
sometimes written as $B\rho$. This is just an old notation where $\rho$ is the bending radius of a
particle with the reference energy in a field of strength $B$. Notice that $P_0$ is the local
reference momentum at the element which may not be the same as the reference energy at the beginning
of the lattice if there are \vn{lcavity} elements (\sref{s:lcav}) present.

The kicks $\Delta p_x$ and $\Delta p_y$ that a
particle experiences going through a multipole field is
\begin{alignat}{5}
  \Delta p_x & = \frac{-q \, L \, B_y}{P_0} \label{pqlbp1} \\
             & = -K_0 L \;-\; 
             && K_1 L \, x \plus 
             \frac{1}{2} && K_2 L (y^2 - x^2) && \plus 
             && \frac{1}{6} K_3 L (3x y^2 - x^3) \plus \ldots 
             \nonumber \\
  \Delta p_y & = \frac{q \, L \, B_x}{P_0} \label{pqlbp2} \\
             & =     
             && K_1 L \, y \plus 
             && K_2 L \, xy && \plus 
             && \frac{1}{6} K_3L (3x^2 y - y^3) \plus \ldots \nonumber 
\end{alignat}
A positive $K_1L$ quadrupole component gives
horizontal focusing and vertical defocusing. The general form is
\begin{align}
  \Delta p_x &= \sum_{n = 0}^{\infty} \frac{K_n L}{n!} 
             \sum_{m = 0}^{\lfloor \frac{n}{2} \rfloor}
             \begin{pmatrix} n \cr 2m \end{pmatrix} \,
             (-1)^{m+1} \, x^{n-2m} \, y^{2m} \\
  \Delta p_y &= \sum_{n = 0}^{\infty} \frac{K_n L}{n!} 
             \sum_{m = 0}^{\lfloor \frac{n-1}{2} \rfloor}
             \begin{pmatrix} n \cr 2m+1 \end{pmatrix} \,
             (-1)^{m} \, x^{n-2m-1} \, y^{2m+1}
\end{align}
where $\lfloor\alpha\rfloor$ means round down to the integer equal to or less than $\alpha$ and 
$\binom{a}{b}$ (``a choose b'') denotes a binomial coefficient.

\index{multipole!KnL, Tn|hyperbf}
So far only the normal components of the field have been
considered. If the fields associated with a particular $B_n$ multipole
component are rotated in the $(x, y)$ plane by an angle $\theta_n$, the
magnetic field at a point $(x,y)$ can be expressed in complex notation
as
\Begineq
  B_y(x,y) + i B_x(x,y) = 
    \frac{1}{n!} B_n e^{-i(n+1)\theta_n} \, e^{i n \theta} \, r^n 
  \label{bib1nb}
\Endeq
where $(r, \theta)$ are the polar coordinates of the point $(x, y)$.

\index{multipole}
Note that, for compatibility with MAD, the $K0L$ component of a \vn{Multipole} element
rotates the reference orbit essentially acting as a zero length bend. This is not true
for multipoles of any other type of element.

Instead of using magnitude $K_n$ and rotation angle $\theta_n$,
Another representation is using normal $\wt K_n$ and skew $\wt
S_n$. The conversion between the two are
\begin{align}
  \wt K_n  &= K_n \, \cos((n + 1) \theta_n) \CRNO
  \wt S_n &= K_n \, \sin((n + 1) \theta_n) 
\end{align}

\index{multipole!an, bn|hyperbf}
Another representation of the magnetic field used by \bmad divides
the fields into normal $b_n$ and skew $a_n$ components. In terms of
these components the magnetic field for the $n$\Th\ order multipole is
\Begineq
  \frac{q \, L}{P_0} \, (B_y + i B_x) = (b_n + i a_n) \, (x + i y)^n
  \label{qlpbb}
\Endeq
The $a_n$, $b_n$ representation of multipole fields can be used in elements such as
quadrupoles, sextupoles, etc. to allow ``error'' fields to be represented.  
The conversion between $(a_n, b_n)$ and $(K_nL, \theta_n)$ is
\Begineq
  b_n + i a_n = \frac{1}{n!} \, K_nL \, e^{-i(n+1)\theta_n}
\Endeq
or
\begin{align}
  K_n L &= n! \, \sqrt{a_n^2 + b_n^2} \\
  \tan[(n+1) \theta_n] &= \frac{-a_n}{b_n}
\end{align}
To convert a normal magnet (a magnet with no skew component) into a skew magnet (a magnet with no
normal component) the magnet should be rotated about its longitudinal axis with a rotation angle of
\Begineq
  (n+1) \theta_n = \frac{\pi}{2}
\Endeq
For example, a normal quadrupole rotated by $45^\circ$ becomes a skew quadrupole.

The multipole fields can be ``\vn{reference energy}'' scaled and/or ``\vn{element strength}''
scaled.  Scaling here means that the $a_n$ and $b_n$ values used in tracking are scaled from the
input values given in the lattice file.

\vn{Reference energy} scaling is applied if the \vn{field_master} attribute (\sref{s:field.master})
is True for an element so that the multipole values specified in the lattice file are not reference
energy normalized
\Begineq
  \bigl[ a_n, b_n \bigr] \longrightarrow
  \bigl[ a_n, b_n \bigr] \cdot \frac{q}{P_0}
  \label{ababq}
\Endeq

\index{r0_mag}
\index{F (multipole scale factor)}
\vn{Element strength} scaling is applied when the multipoles are associated with a non
\vn{AB_Multipole} element and if the \vn{scale_multipoles} attribute (\sref{s:multip}) is
\vn{True}. This scaling uses a measurement radius $r_0$ and a scale factor $F$:
\Begineq
  \bigl[ a_n, b_n \bigr] \longrightarrow
  \bigl[ a_n, b_n \bigr]
  \cdot F \cdot \frac{r_0^{n_\text{ref}}}{r_0^n}
  \label{ababf}
\Endeq
$r_0$ is set by the \vn{r0_mag} attribute of an element. $F$ and $n_\text{ref}$ are set
automatically depending upon the type of element as shown in Table~\ref{t:ab}. The
$\gamma_p$ term is

\index{kicker}\index{hkicker}\index{vkicker}\index{ac_kicker}
\index{rbend}\index{sbend}\index{elseparator}
\index{quadrupole}\index{solenoid}\index{sol_quad}
\index{sextupole}\index{octupole}
\begin{table}[ht]
\centering
\begin{tabular}{lll} \toprule
\tt
  {\em Element} & $F$                              & $n_\text{ref}$ \\ \midrule
  \vn{Elseparator} & $\sqrt{{\tt Hkick}^2 + {\tt Vkick}^2}$ & 0 \\
  \vn{Hkicker}     & Kick                                   & 0 \\
  \vn{Kicker},\vn{AC_Kicker}
                   & $\sqrt{{\tt Hkick}^2 + {\tt Vkick}^2}$ & 0 \\
  \vn{Rbend}       & G * L                                  & 0 \\
  \vn{Sbend}       & G * L                                  & 0 \\
  \vn{Vkicker}     & Kick                                   & 0 \\
  \vn{Wiggler}     & $\dsfrac{2 \, c \, {\tt L\_pole} \, B_{max}}{\pi \, {\tt p0c}}$ 
                                                            & 0 \\
  \vn{Quadrupole}  & K1 * L                                 & 1 \\
  \vn{Sol_Quad}    & K1 * L                                 & 1 \\
  \vn{Solenoid}    & KS * L                                 & 1 \\
  \vn{Sextupole}   & K2 * L                                 & 2 \\
  \vn{Octupole}    & K3 * L                                 & 3 \\ \bottomrule
\end{tabular}
\caption{$F$ and $n_\text{ref}$ for various elements.}
\label{t:ab}
\end{table}

%-----------------------------------------------------------------
\section{Electrostatic Multipole Fields}
\label{s:elec.field}
\index{electric fields}

Except for the \vn{elseparator} element, \bmad specifies DC electric fields using normal
$b_{en}$ and skew $a_{en}$ components (\sref{s:multip}. The potential $\phi_n$ for the
$n$\Th\ order multipole is
\Begineq
  \phi_n = -\re \left[ \frac{b_{en} - i a_{en}}{n + 1} \, \frac{(x + i y)^{n+1}}{r_0^n} \right]
  \label{pbian1}
\Endeq
where $r_0$ is a ``measurement radius'' set by the \vn{r0_elec} attribute of an element
(\sref{s:multip}).

The electric field for the $n$\Th order multipole is
\Begineq
  E_x - i E_y = (b_{en} - i a_{en}) \, \frac{(x + i y)^n}{r_0^n}
  \label{exiey}
\Endeq
Notice that the magnetic multipole components $a_n$ and $b_n$ are normalized by the
element length, reference charge, and reference momentum (\Eq{qlpbb}) while their electric
counterparts are not.

Using the paraxial approximation, The kick given a particle due to the electric field is
\Begineq
  \frac{dp_x}{ds} = \frac{q \, E_x}{\beta \, P_0 \, c}, \qquad \frac{dp_y}{ds} = \frac{q \, E_y}{\beta \, P_0 \, c}
\Endeq
Where $\beta$ is the normalized velocity.

%-----------------------------------------------------------------
\section{Exact Multipole Fields in a Bend}
\label{s:field.exact}

For static magnetic and electric multipole fields in a bend, the spacial dependence of the
field is different from multipole fields in an element with a straight geometry as given
by \Eqs{qlpbb} and \eq{exiey}. The analysis of the multipole fields in a bend here follows
McMillan\cite{b:mcmillan}.  

In the rest of this section, normalized coordinates $\rw = r / \rho$, $\xw / = x /
\rho$, and $\yw = y / \rho$ will be used where $\rho$ is the bending radius of the
reference coordinate system, $r$ is the distance, in the plane of the bend, from the bend
center to the observation point, $x$ is the distance in the plane of the from the reference
coordinates to the observation point and $y$ is the distance out-of-plane. With this
convention $\rw = 1 + \xw$.

An electric or magnetic multipole can be characterized by a scalar potential $\phi$ with
the field given by $-\nabla \phi$.  The potential is a solution to Laplace's equation
\Begineq
  \frac{1}{\rw} \, \frac{\partial}{\partial \, \rw} 
  \left( \rw \, \frac{\partial \, \phi}{\partial \, \rw} \right) +
  \frac{\partial^2 \phi}{\partial \, \yw^2} = 0
\Endeq
As McMillian shows, it is also possible to calculate the magnetic field by constructing the
appropriate vector potential. However, from a practical point of view, it is simpler to use the
scalar potential for both the magnetic and electric fields.

Solutions to Laplace's equation can be found in form
\Begineq
  \phi_{n}^r = \frac{-1}{1+n} \sum_{p = 0}^{\lfloor (n+1)/2 \rfloor} 
             \begin{pmatrix} n + 1 \cr 2 \, p \end{pmatrix} \,
             (-1)^{p} \, F_{n+1-2p}(\rw) \, \yw^{2p}
  \label{pspn1}
\Endeq
and in the form
\Begineq
  \phi_{n}^i = \frac{-1}{1+n} \sum_{p = 0}^{\lfloor n/2 \rfloor} 
             \begin{pmatrix} n + 1 \cr 2p +1 \end{pmatrix} \,
             (-1)^{p} \, F_{n-2p}(\rw) \, \yw^{2p+1}
  \label{pspn2}
\Endeq
where $\binom{a}{b}$ (``a choose b'') denotes a binomial coefficient, $n$ is the order
number which can range from 0 to infinity, $\lfloor\alpha\rfloor$ means round down to the
integer equal to or less than $\alpha$. [Notice that here $n$ is related to $m$ in
McMillian's paper by $m = n + 1$. Also note that the $\phi^r$ and $\phi^i$ here have a
normalization factor that is different from McMillian.]

In \Eq{pspn2} the $F_p(\rw)$ are related by
\Begineq
  F_{p+2} = (p + 1) \, (p + 2) \, \int_1^{\rw} \frac{d\rw}{\rw} 
  \left[ \int_1^{\rw} d\rw \, \rw \, F_{p} \right]
\Endeq
with the ``boundary condition'':
\begin{align}
  F_0(\rw) &= 1 \CRNO
  F_1(\rw) &= \ln \, \rw
\end{align}
This condition ensures that the number of terms in the sums in \Eqs{pspn1} and \eq{pspn2}
are finite. With this condition, all the $F_p$ can be constructed:
\begin{align}
  F_1 &= \ln \, \rw = \xw - \frac{1}{2}\xw^2 + \frac{1}{3}\xw^3 - \ldots \CRNO
  F_2 &= \frac{1}{2} (\rw^2 - 1) - \ln \rw = \xw^2 - \frac{1}{3}\xw^3 + \frac{1}{4} \xw^4 - \ldots \CRNO
  F_3 &= \frac{3}{2} [-(\rw^2 - 1) + (\rw^2 + 1) \ln \rw] = \xw^3 - \frac{1}{2} \xw^4 + \frac{7}{20} \xw^5 - \ldots 
         \label{ffff} \\
  F_4 &= 3 [ \frac{1}{8} (\rw^4 - 1) + \frac{1}{2} (\rw^2 - 1) - (\rw^2 + \frac{1}{2}) \ln \rw] = 
         \xw^4 - \frac{2}{5} \xw^5 + \frac{3}{10} \xw^6 - \ldots \CRNO
  &\text{Etc...} \nonumber
\end{align}
In terms of implementing these functions on a computer, the exact $\rw$-dependent functions are
problematical due to round off error near $\xw = 0$. For example, Evaluating $F_4(\rw)$ at $\xw =
10^{-4}$ results in a complete loss of accuracy (no significant digits!) when using double precision
numbers. In practice, \bmad uses a Pade approximant for $\xw$ small enough and then switches to the
$\rw$-dependent formulas for $\xw$ away from zero.

For magnetic fields, the ``real'' $\phi_n^r$ solutions will correspond to skew fields and the
``imaginary'' $\phi_n^i$ solutions will correspond to normal fields
\Begineq
  \bfB = -\frac{P_0}{q \, L} \, 
    \sum_{n = 0}^\infty \rho^n \, \left[ a_n \, \widetilde \nabla \phi_n^r + b_n \, \widetilde \nabla \phi_n^i \right]
  \label{bpql}
\Endeq
where the gradient derivatives of $\widetilde \nabla$ are with respect to the normalized
coordinates. In the limit of infinite bending radius $\rho$, the above equations converge
to the straight line solution given in \Eq{qlpbb}.

For electric fields, the ``real'' solutions will correspond to normal fields and the
``imaginary'' solutions are used for skew fields
\Begineq
  \bfE = -\sum_{n = 0}^\infty \rho^n \, \left[ a_{en} \, \widetilde \nabla \phi_n^i + 
  b_{en} \, \widetilde \nabla \phi_n^r \right]
  \label{enrn}
\Endeq
And this will converge to \Eq{exiey} in the straight line limit.

In the vertical plane, with $\xw = 0$, the solutions $\phi_n^r$ and $\phi_n^i$ have the same
variation in $\yw$ as the multipole fields with a straight geometry. For example, the field
strength of an $n = 1$ (quadrupole) multipole will be linear in $\yw$ for $\xw = 0$. However, in the
horizontal direction, with $\yw = 0$, the multipole field will vary like $dF_2/d\xw$ which has
terms of all orders in $\xw$. In light of this, the solutions $\phi_n^r$ and $\phi_n^i$ are
called ``vertically pure'' solutions.

It is possible to construct ``horizontally pure'' solutions as well. That is, it is possible to
construct solutions that in the horizontal plane, with $\yw = 0$, behave the same as the corresponding
multipole fields with a straight geometry. A straight forward way to do this, for some given
multipole of order $n$, is to construct the horizontally pure solutions, $\psi_n^r$ and $\psi_n^i$,
as linear superpositions of the vertically pure solutions
\Begineq
  \psi_n^r = \sum_{k = n}^\infty C_{nk} \, \phi_k^r, \qquad
  \psi_n^i = \sum_{k = n}^\infty D_{nk} \, \phi_k^i
  \label{p1rc}
\Endeq
with the normalizations $C_{nn} = D_{nn} = 1$. The $C_{nk}$ and $D_{nk}$ are chosen, order
by order, so that $\psi_n^r$ and $\psi_n^i$ are horizontally pure. For the real
potentials, the $C_{nk}$, are obtained from a matrix $\bfM$ where $M_{ij}$ is the
coefficient of the $\xw^j$ term of $(dF_i/d\xw)/i$ when $F_i$ is expressed as an expansion in
$\xw$ (\Eq{ffff}). $C_{nk}$, $k = 0, \ldots, \infty$ are the row vectors of the inverse
matrix $\bfM^{-1}$. For the imaginary potentials, the $D_{nk}$ are constructed similarly
but in this case the rows of $\bfM$ are the coefficients in $\xw$ for the functions $F_i$.
To convert between field strength coefficients, \Eqs{bpql} and \eq{enrn} and \Eqs{p1rc}
are combined
\begin{alignat}{3}
  a_n &= \sum_{k = n}^\infty \frac{1}{\rho^{k-n}} \, C_{nk} \, \alpha_k, \quad 
  &a_{en} &= \sum_{k = n}^\infty \frac{1}{\rho^{k-n}} \, D_{nk} \, \alpha_{ek}, \CRNO
  b_n &= \sum_{k = n}^\infty \frac{1}{\rho^{k-n}} \, D_{nk} \, \beta_k, \quad
  &b_{en} &= \sum_{k = n}^\infty \frac{1}{\rho^{k-n}} \, D_{nk} \, \beta_{ek}
\end{alignat}
where $\alpha_k$, $\beta_k$, $\alpha_{ek}$, and $\beta_{ek}$ are the corresponding coefficients
for the horizontally pure solutions.

When expressed as a function of $\rw$ and $\yw$, the vertically pure solutions $\phi_n$ have a
finite number of terms (\Eqs{pspn1} and \eq{pspn2}). On the other hand, the horizontally
pure solutions $\psi_n$ have an infinite number of terms.

The vertically pure solutions form a complete set. That is, any given field that satisfies
Maxwell's equations and is independent of $z$ can be expressed as a linear combination of
$\phi_n^r$ and $\phi_n^i$. Similarly, the horizontally pure solutions form a complete
set. [It is, of course, possible to construct other complete sets in which the basis
functions are neither horizontally pure nor vertically pure.]

\index{exact_multipoles}
This brings up an important point. To properly simulate a machine, one must first of all
understand whether the multipole values that have been handed to you are for horizontally
pure multipoles, vertically, pure multipoles, or perhaps the values do not correspond to
either horizontally pure nor vertically pure solutions! Failure to understand this point
can lead to differing results. For example, the chromaticity induced by a horizontally
pure quadrupole field will be different from the chromaticity of a vertically pure
quadrupole field of the same strength. With \bmad, the \vn{exact_multipoles}
(\sref{s:bend}) attribute of a bend is used to set whether multipole values are for
vertically or horizontally pure solutions. [Note to programmers: PTC always assumes
coefficients correspond to horizontally pure solutions. The \bmad PTC interface will
convert coefficients as needed.]

%-----------------------------------------------------------------
\section{Map Decomposition of Magnetic and Electric Fields}
\label{s:field.map}
\index{electric fields!map decomposition}
\index{magnetic fields!map decomposition}

Electric and magnetic fields can be parameterized as the sum over a number of functions
with each function satisfying Maxwell's equations. These functions are also referred to as
``maps'', ``modes'', or ``terms''. \bmad has two parameterizations:
\begin{example}
  Cartesian_Map      ! \sref{s:cart.map.phys}.
  Cylindrical_Map    ! \sref{s:cylind.map.phys}
\end{example}
These parameterizations are two of the four \vn{field map} parameterizations that \bmad
defines \sref{s:fieldmap}.

The \vn{cartesian_map} decomposition involves a set of terms, each term a solution the
Laplace equation solved using separation of variables in Cartesian coordinates. This
decomposition can be used for DC but not AC fields. See \sref{s:cart.map.phys}.
for more details. The syntax for specifying the \vn{cartesian_map} decomposition
is discussed in \sref{s:cart.map}.

The \vn{cylindrical_map} decomposition can be used for both DC and AC fields. See
\sref{s:cylind.map.phys} for more details. The syntax for specifying the
\vn{cylindrical_map} decomposition is discussed in \sref{s:cylind.map}.

%-----------------------------------------------------------------
\section{Cartesian Map Field Decomposition}
\label{s:cart.map.phys}
\index{cartesian_map}

Electric and magnetic fields can be parameterized as the sum over a number of functions
with each function satisfying Maxwell's equations. These functions are also referred to as
``maps'', ``modes'', or ``terms''. \bmad has two types. The ``\vn{Cartesian}''
decomposition is explained here. The other type is the \vn{cylindrical} decomposition
(\sref{s:cylind.map.phys}).

The \vn{Cartesian} decomposition implemented by \bmad involves a set of terms, each
term a solution the Laplace equation solved using separation of variables in Cartesian
coordinates. This decomposition is for DC electric or magnetic fields. No AC Cartesian Map
decomposition is implemented by \bmad. In a lattice file, a \vn{Cartesian} map is specified using
the \vn{cartesian_map} attribute as explained in Sec.~\sref{s:cart.map}.

The \vn{Cartesian} decomposition is modeled using an extension of the method of Sagan,
Crittenden, and Rubin\cite{b:wiggler}. In this decomposition, the magnetic(or electric
field is written as a sum of terms $B_i$ (For concreteness the symbol $B_i$ is used but
the equations below pertain equally well to both electric and magnetic fields) with:
\Begineq
  \bfB(x,y,z) = \sum_i \bfB_i(x, y, z; A, k_x, k_y, k_z, x_0, y_0, \phi_z, family)
\Endeq
Each term $B_i$ is specified using seven numbers $(A, k_x, k_y, k_z,
x_0, y_0, \phi_z)$ and a switch called \vn{family} which can be one of:
\begin{example}
  x,  qu
  y,  sq
\end{example}
Roughly, taking the offsets $x_0$ and $y_0$ to be zero (see the equations below), the \vn{x}
\vn{family} gives a field on-axis where the $y$ component of the field is zero. that is, the \vn{x}
family is useful for simulating, say, magnetic vertical bend dipoles. The \vn{y} \vn{family} has a
field that on-axis has no $x$ component. The \vn{qu} \vn{family} has a magnetic quadrupole like
(which for electric fields is skew quadrupole like) field on-axis and the \vn{sq} \vn{family} has a
magnetic skew quadrupole like field on-axis. Additionally, assuming that the $x_0$ and $y_0$ offsets
are zero, the \vn{sq} family, unlike the other three families, has a nonzero on-axis $z$ field
component.

Each family has three possible forms These are designated as ``\vn{hyper-y}'',
``\vn{hyper-xy}'', and ``\vn{hyper-x}''. 

For the \vn{x} \vn{family} the \vn{hyper-y} form is:
\begin{alignat}{4}
  B_x &=  &&A \, &\dfrac{k_x}{k_y} & \cos(\kxx) \, \cosh(\kyy) \, \cos(\kzz) \CRNEG
  B_y &=  &&A \, &                 & \sin(\kxx) \, \sinh(\kyy) \, \cos(\kzz) \CRNEG
  B_s &= -&&A \, &\dfrac{k_z}{k_y} & \sin(\kxx) \, \cosh(\kyy) \, \sin(\kzz) \label{cm1} \\
  &&&&& \text{with} \,\, k_y^2 = k_x^2 + k_z^2 \nonumber
\end{alignat}
The \vn{x} \vn{family} \vn{hyper-xy} form is:
\begin{alignat}{4}
  B_x &=  &&A \, &\dfrac{k_x}{k_z} & \cosh(\kxx) \, \cosh(\kyy) \, \cos(\kzz) \CRNEG
  B_y &=  &&A \, &\dfrac{k_y}{k_z} & \sinh(\kxx) \, \sinh(\kyy) \, \cos(\kzz) \CRNEG
  B_s &= -&&A \, &                 & \sinh(\kxx) \, \cosh(\kyy) \, \sin(\kzz) \label{cm3} \\
  &&&&& \text{with} \,\, k_z^2 = k_x^2 + k_y^2 \nonumber
\end{alignat}
And the \vn{x} \vn{family} \vn{hyper-x} form is:
\begin{alignat}{4}
  B_x &=  &&A \, &                 & \cosh(\kxx) \, \cos(\kyy) \, \cos(\kzz) \CRNEG
  B_y &= -&&A \, &\dfrac{k_y}{k_x} & \sinh(\kxx) \, \sin(\kyy) \, \cos(\kzz) \CRNEG
  B_s &= -&&A \, &\dfrac{k_z}{k_x} & \sinh(\kxx) \, \cos(\kyy) \, \sin(\kzz) \label{cm5} \\
  &&&&& \text{with} \,\, k_x^2 = k_y^2 + k_z^2 \nonumber
\end{alignat}

The relationship between $k_x$, $k_y$, and $k_z$ ensures that
Maxwell's equations are satisfied. Notice that which form
\vn{hyper-y}, \vn{hyper-xy}, and \vn{hyper-x} a particular $\bfB_i$
belongs to can be computed by \bmad by looking at the values of $k_x$,
$k_y$, and $k_z$.

Using a compact notation where $\Ch \equiv \cosh$, subscript $x$ is $\kxx$, subscript $z$
is $\kzz$, etc., the \vn{y} \vn{family} of forms is:
\begin{alignat}{7}
  & \text{Form} \quad  && \text{hyper-y} \quad && \text{hyper-xy} \quad && \text{hyper-x} \quad \CRNO
  & B_x  
    &-& A \, \dfrac{k_x}{k_y} \, \Se_x \, \Sh_y \, \Ce_z \quad
    & & A \, \dfrac{k_x}{k_z} \, \Sh_x \, \Sh_y \, \Ce_z \quad
    & & A \, \hphphp          \, \Sh_x \, \Se_y \, \Ce_z \quad \CRNO
  & B_y
    & & A \, \hphphp          \, \Ce_x \, \Ch_y \, \Ce_z \quad
    & & A \, \dfrac{k_y}{k_z} \, \Ch_x \, \Ch_y \, \Ce_z \quad
    & & A \, \dfrac{k_y}{k_x} \, \Ch_x \, \Ce_y \, \Ce_z \quad \label{family.y} \\
  & B_z
    &-& A \, \dfrac{k_z}{k_y} \, \Ce_x \, \Sh_y \, \Se_z \quad
    &-& A \, \hphphp          \, \Ch_x \, \Sh_y \, \Se_z \quad
    &-& A \, \dfrac{k_z}{k_x} \, \Ch_x \, \Se_y \, \Se_z \quad \CRNO
  & \text{with} 
    && k_y^2 = k_x^2 + k_z^2 
    && k_z^2 = k_x^2 + k_y^2
    && k_x^2 = k_y^2 + k_z^2 \nonumber
\end{alignat}

the \vn{qu} \vn{family} of forms is:
\begin{alignat}{7}
  & \text{Form} \quad  && \text{hyper-y} \quad && \text{hyper-xy} \quad && \text{hyper-x} \quad \CRNO
  & B_x  
    & & A \, \dfrac{k_x}{k_y} \, \Ce_x \, \Sh_y \, \Ce_z \quad
    & & A \, \dfrac{k_x}{k_z} \, \Ch_x \, \Sh_y \, \Ce_z \quad
    & & A \, \hphphp          \, \Ch_x \, \Se_y \, \Ce_z \quad \CRNO
  & B_y
    & & A \, \hphphp          \, \Se_x \, \Ch_y \, \Ce_z \quad
    & & A \, \dfrac{k_y}{k_z} \, \Sh_x \, \Ch_y \, \Ce_z \quad
    & & A \, \dfrac{k_y}{k_x} \, \Sh_x \, \Ce_y \, \Ce_z \quad \\
  & B_z
    &-& A \, \dfrac{k_z}{k_y} \, \Se_x \, \Sh_y \, \Se_z \quad
    &-& A \, \hphphp          \, \Sh_x \, \Sh_y \, \Se_z \quad
    &-& A \, \dfrac{k_z}{k_x} \, \Sh_x \, \Se_y \, \Se_z \quad \CRNO
  & \text{with} 
    && k_y^2 = k_x^2 + k_z^2 
    && k_z^2 = k_x^2 + k_y^2
    && k_x^2 = k_y^2 + k_z^2 \nonumber
\end{alignat}

the \vn{sq} \vn{family} of forms is:
\begin{alignat}{7}
  & \text{Form} \quad  && \text{hyper-y} \quad && \text{hyper-xy} \quad && \text{hyper-x} \quad \CRNO
  & B_x  
    &-& A \, \dfrac{k_x}{k_y} \, \Se_x \, \Ch_y \, \Ce_z \quad
    & & A \, \dfrac{k_x}{k_z} \, \Sh_x \, \Ch_y \, \Ce_z \quad
    &-& A \, \hphphp          \, \Sh_x \, \Ce_y \, \Ce_z \quad \CRNO
  & B_y
    & & A \, \hphphp          \, \Ce_x \, \Sh_y \, \Ce_z \quad
    & & A \, \dfrac{k_y}{k_z} \, \Ch_x \, \Sh_y \, \Ce_z \quad
    & & A \, \dfrac{k_y}{k_x} \, \Ch_x \, \Se_y \, \Ce_z \quad \label{bsq} \\
  & B_z
    &-& A \, \dfrac{k_z}{k_y} \, \Ce_x \, \Ch_y \, \Se_z \quad
    &-& A \, \hphphp          \, \Ch_x \, \Ch_y \, \Se_z \quad
    & & A \, \dfrac{k_z}{k_x} \, \Ch_x \, \Ce_y \, \Se_z \quad \CRNO
  & \text{with} 
    && k_y^2 = k_x^2 + k_z^2 
    && k_z^2 = k_x^2 + k_y^2
    && k_x^2 = k_y^2 + k_z^2 \nonumber
\end{alignat}


The singular case where $k_x = k_y = k_z = 0$ is not allowed. If a
uniform field is needed, a term with very small $k_x$, $k_y$, and
$k_z$ can be used. Notice that since $k_y$ must be non-zero for the
\vn{hyper-y} forms (remember, $k_y^2 = k_x^2 + k_z^2$ for these forms
and not all $k$'s can be zero), and $k_z$ must be non-zero for the
\vn{hyper-xy} forms, and $k_x$ must be nonzero for the \vn{hyper-x}
forms. The magnetic field is always well defined even if one of the
$k$'s is zero.

%-----------------------------------------------------------------
\section{Cylindrical Map Decomposition}
\label{s:cylind.map.phys}
\index{cylindrical map}

Electric and magnetic fields can be parameterized as the sum over a number of functions
with each function satisfying Maxwell's equations. These functions are also referred to as
``maps'', ``modes'', or ``terms''. \bmad has two types. The ``\vn{cylindrical}''
decomposition is explained here. The other type is the \vn{Cartesian} decomposition
(\sref{s:cylind.map.phys}).

In a lattice file, a \vn{cylindrical} map is specified using the \vn{cylindrical_map}
attribute as explained in Sec.~\sref{s:cylind.map}.

The \vn{cylindrical} decomposition takes one of two forms depending upon whether the
fields are time varying or not. The DC decomposition is explained in
Sec.~\sref{s:cylind.dc} while the RF decomposition is explained in
Sec.~\sref{s:cylind.ac}. 

%-----------------------------------------------------------------
\subsection{DC Cylindrical Map Decomposition}
\label{s:cylind.dc}

The DC \vn{cylindrical} parametrization used by \bmad essentially follows Venturini et
al.\cite{b:vent.map}. See Section~\sref{s:fieldmap} for details on the synax used to cylindrical
maps in \bmad. The electric and magnetic fields are both described by a scalar potential
\Begineq
  \bfB = -\nabla \, \psi_B, \qquad \bfE = \nabla \, -\psi_E
\Endeq
The scalar potentials both satisfy the Laplace equation $\nabla^2 \, \psi = 0$.
The scalar potentials are decomposed as a sum of modes indexed by an integer $m$
\Begineq
  \psi_B = \re \left[ \sum_{m = 0}^\infty \, \psi_{Bm} \right]
\Endeq
[Here and below, only equations for the magnetic field will be shown, the equations for the electric
fields are similar.] The $\psi_{Bm}$ are decomposed in $z$ using a discrete Fourier
sum.\footnote{Venturini uses a continuous Fourier transformation but \bmad uses a discrete
transformation so that only a finite number of coefficients are needed.}
Expressed in cylindrical coordinates the decomposition of $\psi_{Bm}$ is
\Begineq
  \psi_{Bm} = \sum_{n=-N/2}^{N/2-1} \psi_{Bmn} =
  \sum_{n=-N/2}^{N/2-1} \frac{-1}{k_n} \, e^{i \, k_n \, z} \,
  \cos (m \, \theta - \theta_{0m}) \, b_m(n) \, I_m(k_n \, \rho)
  \label{psps1k}
\Endeq
where $I_m$ is a modified Bessel function of the first kind, and the
$b_m(n)$ are complex coefficients. [For electric fields, $e_m(n)$ is
substituted for $b_m(n)$] In \Eq{psps1k} $k_n$ is
given by
\Begineq
  k_n = \frac{2 \pi \, n}{N \, dz}
\Endeq
where $N$ is the number of ``sample points'', and $dz$ is the longitudinal ``distance between
points''. That is, the above equations will only be accurate over a longitudinal length $(N-1)
\, dz$. Note: Typically the sum in \Eq{psps1k} and other equations below runs from $0$ to $N-1$.
Using a sum from $-N/2$ to $N/2-1$ gives exactly the same field at the sample points ($z = 0, dz,
2\,ds, \ldots$) and has the virtue that the field is smoother in between.

The field associated with $\psi_{Bm}$ is for $m = 0$:
\begin{align}
  B_\rho &= \re \left[ 
    \sum_{n=-N/2}^{N/2-1} e^{i \, k_n \, z} \, b_0(n) \,
    I_1(k_n \, \rho) \right] \CRNO
  B_\theta &= 0 \\
  B_z &= \re \left[ 
    \sum_{n=-N/2}^{N/2-1} i \, e^{i \, k_n \, z} \, b_0(n) \,
    I_0(k_n \, \rho) \right]
    \nonumber
\end{align}

And for $m \neq 0$:
\begin{align}
  B_\rho &= \re \left[ 
    \sum_{n=-N/2}^{N/2-1} \frac{1}{2} \, e^{i \, k_n \, z} \, 
    \cos (m \, \theta - \theta_{0m}) \, b_m(n) \,
    \Big[ I_{m-1}(k_n \, \rho) + I_{m+1}(k_n \, \rho) \Big] \right] \CRNO
  B_\theta &= \re \left[ 
    \sum_{n=-N/2}^{N/2-1} \frac{-1}{2} \, e^{i \, k_n \, z} \, 
    \sin (m \, \theta - \theta_{0m}) \, b_m(n) \,
    \Big[ I_{m-1}(k_n \, \rho) - I_{m+1}(k_n \, \rho) \Big] \right] \\
  B_z &= \re \left[ 
    \sum_{n=-N/2}^{N/2-1} i \, e^{i \, k_n \, z} \, 
    \cos (m \, \theta - \theta_{0m}) \, b_m(n) \,
    I_m(k_n \, \rho) \right]
    \nonumber
\end{align}

While technically $\psi_{Bm0}$ is not well defined due to the $1/k_n$ factor
that is present, the field itself is well behaved. Mathematically,
\Eq{psps1k} can be corrected if, for $n = 0$, the term $I_m(k_n \,
\rho) / k_n$ is replaced by
\Begineq
  \frac{I_m(k_0 \, \rho)}{k_0} \rightarrow 
  \begin{cases}
    \rho   &\text{if } m = 0 \\
    \rho/2 &\text{if } m = 1 \\
    0      &\text{otherwise}
  \end{cases}
\Endeq

The magnetic vector potential for $m = 0$ is constructed such that
only $A_\theta$ is non-zero
\begin{align}
  A_\rho &= 0 \CRNO
  A_\theta &= \re \left[ 
    \sum_{n=-N/2}^{N/2-1} \frac{i}{k_n} \, e^{i \, k_n \, z} \, b_0(n) \, I_1(k_n \, \rho) \right] \\
  A_z    &= 0 \nonumber
\end{align}
For $m \ne 0$, the vector potential is chosen so that $A_\theta$ is zero.

\begin{align}
  A_\rho &= \re \left[ 
    \sum_{n=-N/2}^{N/2-1} \frac{-i \, \rho}{2 \, m} \, e^{i \, k_n \, z} \, 
    \cos (m \, \theta - \theta_{0m}) \, b_m(n) \,
    \Big[ I_{m-1}(k_n \, \rho) - I_{m+1}(k_n \, \rho) \Big] \right] \CRNO
  A_\theta &= 0 \\
  A_z    &= \re \left[ 
    \sum_{n=-N/2}^{N/2-1} \frac{-i \, \rho}{m} \, e^{i \, k_n \, z} \, 
    \cos (m \, \theta - \theta_{0m}) \, b_m(n) \,
    I_m(k_n \, \rho) \right] \nonumber
\end{align}

Note: The description of the field using \vn{``generalized gradients''}\cite{b:newton} is similar to
the above equations. The difference is that, with the generalized gradient formalism, terms in
$\theta$ and $\rho$ are expanded in a Taylor series in $x$ and $y$.

%-----------------------------------------------------------------
\subsection{AC Cylindrical Map Decomposition}
\label{s:cylind.ac}
\index{rf fields|hyperbf}

For RF fields, the \vn{cylindrical} mode parameterization used by \bmad essentially
follows Abell\cite{b:rf.abell}. The electric field is the real part of the complex field
\Begineq
  \bfE(\bfr) = \sum_{j=1}^M \, \bfE_j(\bfr) \, \exp[{-2\, \pi \, i \, (f_j \, t + \phi_{0j})}]
  \label{eseei}
\Endeq
where $M$ is the number of modes. Each mode satisfies the vector Helmholtz
equation
\Begineq
  \nabla^2 \bfE_j + k_{tj}^2 \, \bfE_j = 0
  \label{bke}
\Endeq
where $k_{tj} = 2 \, \pi \, f/c$ with $f_j$ being the mode frequency.

The individual modes vary azimuthally as $\cos(m \, \theta - \theta_0)$ where $m$ is a non-negative
integer.  [in this and in subsequent equations, the mode index $j$ has been dropped.]  For the $m =
0$ modes, there is an accelerating mode whose electric field is in the form
\begin{align}
  E_\rho(\bfr) &= \sum_{n=-N/2}^{N/2-1} -e^{i \, k_n \, z} \, 
    i \, k_n \, e_0(n) \, \wt I_1(\kappa_n, \rho) \CRNO
  E_\theta(\bfr) &= 0 \\
  E_z(\bfr) &= \sum_{n=-N/2}^{N/2-1}e^{i \, k_n \, z} \, 
    e_0(n) \, \wt I_0(\kappa_n, \rho) \nonumber
\end{align}
where $\wt I_m$ is
\Begineq
  \wt I_m (\kappa_n, \rho) \equiv \frac{I_m(\kappa_n \, \rho)}{\kappa_n^m}
\Endeq
with $I_m$ being a modified Bessel function first kind, and $\kappa_n$ is given by
\Begineq
  \kappa_n = \sqrt{k_n^2 - k_t^2} = 
  \begin{cases}
    \sqrt{k_n^2 - k_t^2} & |k_n| > k_t \\
    -i \, \sqrt{k_t^2 - k_n^2} & k_t > |k_n|
  \end{cases}
\Endeq
with
\Begineq
  k_n = \frac{2 \pi \, n}{N \, dz}
\Endeq
$N$ is the number of points where $E_{zc}$ is evaluated, and $dz$ is
the distance between points. The length of the field region is $(N-1) \, dz$. When
$\kappa_n$ is imaginary, $I_m(\kappa_n \, \rho)$ can be evaluated
through the relation
\Begineq
  I_m(-i \, x) = i^{-m} \, J_m(x)
\Endeq
where $J_m$ is a Bessel function of the first kind.
The $e_0$ coefficients can be obtained given knowledge of the field at some radius $R$ via
\begin{align}
  e_0(n) &= \frac{1}{\wt I_0(\kappa_n, R)} \, \frac{1}{N} \, \sum_{p=0}^{N-1}
    e^{-2 \, \pi \, i \, n \, p / N} \, E_{z}(R, p \, dz)
\end{align}

The non-accelerating $m = 0$ mode has an electric field in the form
\begin{align}
  E_\rho(\bfr) &= E_z(\bfr) = 0 \CRNO
  E_\theta(\bfr) &= \sum_{n=-N/2}^{N/2-1}e^{i \, k_n \, z} \, 
    b_0(n) \, \wt I_1(\kappa_n, \rho)
\end{align}
where the $b_0$ coefficients can be obtained given knowledge of the field at some radius $R$ via
\Begineq
  b_0(n) = \frac{1}{\wt I_1(\kappa_n, R)} \, \frac{1}{N} \, \sum_{p=0}^{N-1}
    e^{-2 \, \pi \, i \, n \, p / N} \, E_{\theta}(R, p \, dz)
\Endeq

For positive $m$, the electric field is in the form
\begin{align}
  E_\rho(\bfr) &= \sum_{n=-N/2}^{N/2-1}
    -i \, e^{i \, k_n \, z} \, 
    \left[ 
    k_n \, e_m(n) \, \wt I_{m+1}(\kappa_n, \rho) +
    b_m(n) \, \frac{\wt I_m(\kappa_n, \rho)}{\rho}
    \right]
    \cos(m \, \theta - \theta_{0m}) \CRNO
  E_\theta(\bfr) &= \sum_{n=-N/2}^{N/2-1} 
    -i \, e^{i \, k_n \, z} \, 
    \left[
    k_n \, e_m(n) \, \wt I_{m+1}(\kappa_n, \rho) \, + \right. \\
  & \left. \qquad \qquad \qquad \qquad \qquad \qquad
    b_m(n) \, \left( \frac{\wt I_m(\kappa_n, \rho)}{\rho} - 
    \frac{1}{m} \, \wt I_{m-1}(\kappa_n, \rho) \right)
    \right] 
    \sin(m \, \theta - \theta_{0m}) \CRNO
  E_z(\bfr) &= \sum_{n=-N/2}^{N/2-1}e^{i \, k_n \, z} \, 
    e_m(n) \, \wt I_m(\kappa_n, \rho) \cos(m \, \theta - \theta_{0m}) \nonumber
\end{align}
The \vn{e_m} and \vn{b_m} coefficients can be obtained given knowledge of the field at some radius $R$ via
\begin{align}
  e_m(n) &= \frac{1}{\wt I_m(\kappa_n, R)} \, \frac{1}{N} \, \sum_{p=0}^{N-1}
    e^{-2 \, \pi \, i \, n \, p / N} \, E_{zc}(R, p \, dz) \CRNO
  b_m(n) &= \frac{R}{\wt I_m(\kappa_n, R)} \left[
    \frac{1}{N} \, \sum_{p=0}^{N-1}
    i \, e^{-2 \, \pi \, i \, n \, p / N} \, E_{\rho c}(R, p \, dz) -
    k_n \, e_m(n) \, \wt I_{m+1}(\kappa_n, R)
    \right]
\end{align}
where $E_{\rho c}$, $E_{\theta s}$, and $E_{z c}$ are defined by
\begin{align}
  E_\rho(R, \theta, z) &= E_{\rho c}(R, z) \, \cos(m \, \theta - \theta_{0m}) \CRNO
  E_\theta(R, \theta, z) &= E_{\theta s}(R, z) \, \sin(m \, \theta - \theta_{0m}) \\
  \label{erpze}
  E_z(R, \theta, z)    &= E_{z c}(R, z)    \, \cos(m \, \theta - \theta_{0m}) \nonumber
\end{align}

The above mode decomposition was done in the gauge where the scalar potential $\psi$ is zero. The
electric and magnetic fields are thus related to the vector potential $\bfA$ via
\Begineq
  \bfE = -\partial_t \, \bfA, \qquad \bfB = \nabla \times \bfA
\Endeq
Using \Eq{eseei}, the vector potential can be obtained from the electric field via
\Begineq
  \bfA_j = \frac{-i \, \bfE_j}{2 \, \pi \, f_j}
  \label{aiew}
\Endeq
 
Symplectic tracking through the RF field is discussed in Section~\sref{s:symp.track}.  For the
fundamental accelerating mode, the vector potential can be analytically integrated using the
identity
\Begineq
  \int dx \,\frac{x \, I_1 (a \, \sqrt{x^2+y^2})}{\sqrt{x^2+y^2}}  = 
  \frac{1}{a} \, I_0 (a \, \sqrt{x^2+y^2})
\Endeq

%-----------------------------------------------------------------
\section{Field Modeling Using Taylor Maps}
\label{s:taylor.field.phys}
\index{taylor field modeling}

\bmad has a number of \vn{field map} models that can be used to model electric or magnetic fields
(\sref{s:fieldmap}). One model involves a set of Taylor maps. This model is restricted to modeling
DC fields. In a lattice file, the \vn{Taylor} field model is specified using the \vn{taylor_field}
attribute as exaplined in Sec.~\sref{s:taylor.field}.

The Taylor field model specifies the field using a set of Taylor maps. Each map defines the field in
the transverse $(x, y)$ plane at constant longitudinal $z$: That is, each Taylor map is comprised of
three Taylor series, one for each field component, and each Taylor series is a polynomial in $x$ and
$y$:
\Begineq
  \Bf B = (B_x(x,y;i), B_y(x,y;i), B_z(x,y;i))
\Endeq
where the $B_x, B_y, B_z$ on the right hand side represent the tree Taylor series, $i$
denotes the Taylor map at $z_i = i \cdot dz$ where $dz$ is the spacing between maps. The
maps are restricted to be equally spaced to enable higher order integration schemes in PTC
(See \vn{default_integ_order} in \sref{s:bmad.com}).  [Note: Each Taylor field model
specifes either an electric or magnetic field. In this section, the symbol $B$ can refer
to either magnetic or electric fields.]

Interpolation of the field in \bmad is done using a cubic spline fit. Given a position
$(x, y, z)$, the field $\Bf B$ and transverse field derivatives $\partial \Bf B / \partial
\Bf r$, are evaluated at two positions $\Bf r_0 = (x, y, z_{i0})$ and $\Bf r_1 = (x, y,
z_{i1})$ with $z_{i0}$ being the positions of the maps to either side of $z$.
The longitudinal field derivatives are derived from the transverse ones using Maxwell's equations:
\Begineq
  \frac{\partial B_x}{\partial z} = \frac{\partial B_z}{\partial x}, \qquad
  \frac{\partial B_y}{\partial z} = \frac{\partial B_z}{\partial y}, \qquad
  \frac{\partial B_z}{\partial z} = 
    - \left( \frac{\partial B_x}{\partial x} + \frac{\partial B_y}{\partial y} \right)
\Endeq

Once the field and longitudinal derivatives are known at $\Bf r_0$ and $\Bf r_1$, a
standard cubic spline interpolation is done.

Note: If the Taylor field model uses curved coordinates, There will be inaccuracies in the
computed field to the extent that Maxwell's equations have to be modified to take into
account the coordinate curvature.

%-----------------------------------------------------------------
\section{RF fields for Field_Calc = Bmad_Standard}
\label{s:rf.fields}
\index{RF fields|hyperbf}

The following describes the how RF fields are calculated when the the \vn{field_calc}
attribute of an RF element is set to \vn{bmad_standard}.

The traveling wave model for RF cavities the field, in cylindrical coordinates is
\begin{align}
  E_s(r, \phi, s, t) &= G \, \cos\bigl( k \, s - \omega \, t + 2 \, \pi \, \phi \bigr) \CRNO
  E_r(r, \phi, s, t) &= \frac{1}{2} \, G \, k \, r \, \sin\bigl( k \, s - \omega \, t + 2 \, \pi \, \phi \bigr) \\
  B_\phi(r, \phi, s, t) &= \frac{1}{2 \, c} \, G \, k \, r \, \sin\bigl( k \, s - \omega \, t + 2 \, \pi \, \phi \bigr) \nonumber
  \label{egot}
\end{align}
where $G$ is the accelerating gradient, $k = \omega / c$ is the wave number with $\omega$ beging the
RF frequency.

For standing wave cavities, the RF fields are modeled as $N$ half-wave cells, each having a length
of $\lambda/2$ where $\lambda = 2 \, \pi / k$ is the wavelength. If the length of the RF element is
not equal to the length of $N$ cells, the ``active region'' is centered in the element and the
regions to either side are treated as field free.

The field in the standing wave cell is modeled either with a $p = 0$ or $p = 1$ longitudinal
mode. The $p = 1$ longitudinal mode (which is the default if the longitudinal mode is not set in the
element), models the fields as a pillbox with the transverse wall at infinity as detailed in Chapter
3, Section VI of reference \cite{b:lee}
\begin{align}
  E_s(r, \phi, s, t)    &= 2 \, G \,                         \cos(k \, s) \, \cos(\omega \, t + 2 \, \pi \, \phi) \CRNO
  E_r(r, \phi, s, t)    &= G \, k \, r \,                    \sin(k \, s) \, \cos(\omega \, t + 2 \, \pi \, \phi) \\
  B_\phi(r, \phi, s, t) &= \frac{-1}{c} \, G \, \, k \, r \, \cos(k \, s) \, \sin(\omega \, t + 2 \, \pi \, \phi) \nonumber
  \label{egot2}
\end{align}
The overall factor of 2 in the equation is present to ensure that an ultra-relativistic particle
entering with $\phi = 0$ will experience an average gradient equal to $G$.

For the $p = 0$ longitudinal mode, a ``pseudo TM$_{010}$'' mode is used that has the correct symmetry:
\begin{align}
  E_s(r, \phi, s, t)    &= 2 \, G \,                        \sin(k \, s) \, \sin(\omega \, t + 2 \, \pi \, \phi) \CRNO
  E_r(r, \phi, s, t)    &= -G \, k \, r \,                  \cos(k \, s) \, \sin(\omega \, t + 2 \, \pi \, \phi) \\
  B_\phi(r, \phi, s, t) &= \frac{1}{c} \, G \, \, k \, r \, \sin(k \, s) \, \cos(\omega \, t + 2 \, \pi \, \phi) \nonumber
  \label{egot3}
\end{align}

%-----------------------------------------------------------------
\section{Wake fields}
\label{s:wake.fields}
\index{wake fields|hyperbf}

%-----------------------------
\subsection{Short--Range Wakes}
\label{s:wake.short}
\index{wake fields!short-range}

Wake field effects are divided into short--range (within a bunch) and
long--range (between bunches).

Only the transverse dipole and longitudinal monopole components of the short--range wake field are
modeled. The longitudinal monopole energy kick $dE$ for the $i$\Th (trailing) macroparticle due to
the wake from the $j$\Th (leading) macroparticle is computed from the equation
\Begineq
  \Delta p_z(i) = \frac{-e \, L}{v \, P_0} \, \left( \frac{1}{2}\WlS(0) \,  |q_i| +
        \sum_{j \ne i} \WlS(dz_{ij}) \, |q_j| \right)
  \label{delvp}
\Endeq
where $v$ is the particle velocity, $e$ is the charge on an electron, $q$ is the macroparticle
charge, $L$ is the cavity length, $dz_{ij}$ is the longitudinal distance between the $i$\Th and
$j$\Th macroparticles, $\WlS$ is the short--range longitudinal wake field function.

For the transverse kick, there are two types of wakes: Those that are dependent upon transverse
offset of the leading particle (but independent of the position of the trailing particle) and those
that are dependent upon the transverse offset of the trailing particle (but independent of the
position of the leading particle. If the beam chamber has azimuthal symmetry, the only wakes present
are those that are dependent upon the offset of the leading particle.

\bmad assumes that the beam chamber wall is mirror symmetric with respect to both the $x$-axis and
$y$-axis. In this case, a horizontal displacement (of either particle) results in a horizontal kick
and similarly for vertical displacements. That is, the horizontal and vertical wakes are decoupled.

With the above symmetry assumpltion, the transverse kick $\Delta p_x(i)$ for the $i$\Th
macroparticle due to the dipole short--range transverse wake field is modeled with the equation
\Begineq
  \Delta p_x(i) = \frac{-e \, L \, \sum_j |q_j| \, x \, \WtS(dz_{ij})}{v \, P_0}
  \label{pelqxw}
\Endeq
Where $x$ is the horizontal displacement of the leading or trailing particle as appropriate. There
is a similar equation for $\Delta p_y(i)$. $\WtS$ is the transverse short--range wake function.

The wake field functions $\WlS$ and $\WtS$ can be specified in a \bmad lattice file using ``pseudo''
modes where
\Begineq
  W(z) = A_{amp} \, \sum_i A_i \, e^{d_i z} \, \sin (k_i \, z + \phi_i)
  \label{wadzk}
\Endeq
The parameters $(A_i, d_i, k_i, \phi_i)$ are chosen to fit the calculated wake potential
(\sref{s:sr.wake.file}). The dimensionless overall amplitude scale $A_{amp}$ is indroduced as a
convenient way to scale the overall wake. The reason why the mode approach is used in \bmad is due
to the fact that, using pseudo modes, the calculation time scales as the number of particles $N$
while a calculation based upon a table of wake vs $z$ would scale as $N^2$. [The disadvantage is
that initially the user must proform a fit to the wake potential to generate the mode parameter
values.]

%-----------------------------
\subsection{Long--Range Wakes}
\label{s:wake.long}
\index{wake fields!long-range}

Following Chao\cite{b:chao} Eq.~2.88, the long--range wake fields are characterized by a set of
cavity modes. The wake function $W_m$ for a mode of order $m$ is given by
\Begineq
  W_m(t) = -A_{amp} \, c \, \left( \frac{R}{Q} \right)_m \,\,
  e^{-\omega \, t/ 2 Q} \, \sin (\omega \, t)
  \label{wcrq}
\Endeq
The mode strength $(R/Q)_m$ has units of Ohms/meter$^{2m}$. The dimensionless overall amplitude
scale $A_{amp}$ is indroduced as a convenient way to scale the overall wake. The lattice syntax for
defining long-range wakes is discussed in \Sref{s:lr.wake.file}.

Assuming that the macroparticle generating the wake is offset a distance $r_w$ along the $x$--axis,
a trailing macroparticle will see a kick
\begin{align}
  \Delta {\Bf p}_\perp &= 
    -C \, I_m \, W_m(t) \, m \, r^{m-1} \, \left( 
    \bfhat r \cos m\theta - {\bf\hat\theta} \sin m\theta \right) \\
  &= -C \, I_m \, W_m(t) \, m \, r^{m-1} \, \left( \bfhat x \cos [(m-1) \theta] - 
    \bfhat y \sin [(m-1)\theta] \right) \CRNO
  \Delta p_z &= -C \, I_m \, W'_m(t) \, r^m \, \cos m\theta
\end{align}
where $m$ is the order of the mode, $C$ is given by
\Begineq
  C = \frac{e}{c \, P_0}
\Endeq
 and
\Begineq
  I_m = q_w \, r_w^m
\Endeq
with $q_w$ being the magnitude of the charge on the particle.  Generalizing the above, a
macroparticle at $(r_w, \theta_w)$ will generate a wake
\begin{align}
  -\Delta p_x + i\Delta p_y &= C \, I_m \, W_m(t) \, 
    m \, r^{m-1} \, e^{-i m \theta_w} \, e^{i (m-1) \theta} 
    \label{ppcimr} \\
  \Delta p_z &= C \, I_m \, W'_m(t) \, r^m \, \cos [m(\theta - \theta_w)]
    \label{pciwr}
\end{align}
Comparing \Eq{ppcimr} to \eq{bib1nb}, and using the relationship between kick and field as given by
\eq{pqlbp1} and \eq{pqlbp2}, shows that the form of the wake field transverse kick is the same as
for a multipole of order $n = m - 1$.

The wake field felt by a particle is due to the wake fields generated by all the particles ahead of
it. If the wake field kicks are computed by summing over all particle pairs, the computation will
scale as $N^2$ where $N$ is the number of particles. This quickly becomes computationally
exorbitant. A better solution is to keep track of the wakes in a cavity. When a particle comes
through, the wake it generates is simply added to the existing wake. This computation scales as $N$
and makes simulations with large number of particles practical.

To add wakes together, a wake must be decomposed into its components.  Spatially, there are normal
and skew components and temporally there are sin and cosine components. This gives 4 components
which will be labeled $a_{\cos}$, $a_{\sin}$, $b_{\cos}$, and $b_{\sin}$. For a mode of order $m$, a
particle passing through at a time $t_w$ with respect to the reference particle will produce wake
components
\begin{alignat}{2}
  \delta a_{\sin, m} &=  &c \, \left( \frac{R}{Q} \right)_m \,
    e^{\omega \, t_w/ 2 Q} \, \cos (\omega \, t_w) \, I_m \, \sin(m \theta_w) 
    \CRNO
  \delta a_{\cos, m} &= -&c \, \left( \frac{R}{Q} \right)_m \,
    e^{\omega \, t_w/ 2 Q} \, \sin (\omega \, t_w) \, I_m \, \sin(m \theta_w) 
    \label{ac2rq} 
    \\
  \delta b_{\sin, m} &=  &c \, \left( \frac{R}{Q} \right)_m \,
    e^{\omega \, t_w/ 2 Q} \, \cos (\omega \, t_w) \, I_m \, \cos(m \theta_w) 
    \CRNO
  \delta b_{\cos, m} &= -&c \, \left( \frac{R}{Q} \right)_m \,
    e^{\omega \, t_w/ 2 Q} \, \sin (\omega \, t_w) \, I_m \, \cos(m \theta_w) 
    \nonumber
\end{alignat}
These are added to the existing wake components. The total is
\Begineq
  a_{\sin, m} = \sum_{\text{particles}} \delta a_{\sin, m}
\Endeq
with similar equations for $a_{\cos, m}$ etc. Here the sum is over all particles that cross the
cavity before the kicked particle. To calculate the kick due to wake, the normal and skew components
are added together
\begin{align}
  a_m &= e^{-\omega \, t/ 2 Q} \, \left( 
    a_{\cos, m} \, \cos (\omega \, t) - a_{\sin, m} \, \sin (\omega \, t) \right) 
    \label{akz2q} \\
  b_m &= e^{-\omega \, t/ 2 Q} \, \left(
    b_{\cos, m} \, \cos (\omega \, t) - b_{\sin, m} \, \sin (\omega \, t) \right) \nonumber 
\end{align}
Here $t$ is the passage time of the particle with respect to the reference particle. In analogy to
\Eq{ppcimr} and \eq{pciwr}, the kick is
\begin{align}
  -\Delta p_x + i\Delta p_y &= C \, 
    m \, (b_m + i a_m) \, r^{m-1} \, e^{i (m-1) \theta} 
    \label{ppcmbar} \\
  \Delta p_z &= -C \, r^m \, \left( 
    (b_m' + i a_m') e^{i m\theta} + (b_m' - i a_m') e^{-i m\theta} \right)
\end{align}
where $a' \equiv da/dt$ and $b' \equiv db/dt$.

When simulating trains of bunches, the exponential factor $\omega \, t_w / 2 Q$ in \Eq{ac2rq} can
become very large. To prevent numerical overflow, \bmad uses a reference time $z_{\text{ref}}$ so
that all times $t$ in the above equations are replaced by
\Begineq
  t \longrightarrow t - t_{\text{ref}}
\Endeq

The above equations were developed assuming cylindrical symmetry. With cylindrical symmetry, the
cavity modes are actually a pair of degenerate modes. When the symmetry is broken, the modes no
longer have the same frequency. In this case, one has to consider a mode's polarization angle
$\phi$. Equations \eq{akz2q} and \eq{ppcmbar} are unchanged.  In place of \Eq{ac2rq}, the
contribution of a particle to a mode is
\begin{alignat}{2}
  \delta a_{\sin, m} &=  &c \, \left( \frac{R}{Q} \right)_m \,
    e^{\omega \, t_w/ 2 Q} \, \cos (\omega \, t_w) \, I_m \, \left[
    \sin(m \theta_w) \, \sin^2(m \phi) + 
    \cos(m \theta_w) \, \sin(m \phi) \, \cos(m\phi) \right]
    \CRNO
  \delta a_{\cos, m} &= -&c \, \left( \frac{R}{Q} \right)_m \,
    e^{\omega \, t_w/ 2 Q} \, \sin (\omega \, t_w) \, I_m \, \left[ 
    \sin(m \theta_w) \, \sin^2(m \phi) + 
    \cos(m \theta_w) \, \sin(m \phi) \, \cos(m\phi) \right]
    \\
  \delta b_{\sin, m} &=  &c \, \left( \frac{R}{Q} \right)_m \,
    e^{\omega \, t_w/ 2 Q} \, \cos (\omega \, t_w) \, I_m \, \left[
    \cos(m \theta_w) \, \cos^2(m \phi) + 
    \sin(m \theta_w) \, \sin(m \phi) \, \cos(m\phi) \right]
    \CRNO
  \delta b_{\cos, m} &= -&c \, \left( \frac{R}{Q} \right)_m \,
    e^{\omega \, t_w/ 2 Q} \, \sin (\omega \, t_w) \, I_m \, \left[
    \cos(m \theta_w) \, \cos^2(m \phi) + 
    \sin(m \theta_w) \, \sin(m \phi) \, \cos(m\phi) \right]
    \nonumber
\end{alignat}

Each mode is characterized by an $R/Q$, $Q$, $\omega$, and $m$. Notice that $R/Q$ is defined so that
it includes the cavity length. Thus the long--range wake equations, as opposed to the short--range
ones, do not have any explicit dependence on $L$.

To make life more interesting, different people define $R/Q$ differently. A common practice is to
define an $R/Q$ ``at the beam pipe radius''. In this case the above equations must be modified to
include factors of the beam pipe radius. Another convention uses a ``linac definition'' which makes
$R/Q$ twice as large and adds a factor of 2 in \Eq{wcrq} to compensate.

