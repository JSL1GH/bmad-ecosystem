\chapter{Syntax}
\label{c:syntax}

%------------------------------------------------------------------------
\section{Element List Format}
\label{s:ele.list.format}

The syntax for specifying a set of lattice elements is called
\vn{element list} format. Each item of the list is one of:
\begin{center}
\begin{tabular}{ll}
  {\it Item Type} & {\it Example} \\ \hline     
  An element name.                                & "5@q*"               \\
  An element index.                               & "23", "2>>183"       \\
  A range of elements.                            & "b23w:67"            \\
  A class::name specification.                    & "sbend::b*"          \\
\end{tabular}
\break

\end{center}
Items in a list are separated by a blank character or a comma. Example:
\begin{example}
  23, 45:74 quad::q*
\end{example}

An element name item is the name of an element or elements. The
wild card characters ``*'' and/or ``\%'' can be used. The ``*''wildcard
matches any number of characters, The ``\%'' wildcard matches a single
character. For example, ``q\%1*'' matches any element whose name
begins with ``q'' and whose third character is ``1''.  If there are
multiple elements in the lattice that match a given name, all such
elements are included. Thus ``d12'' will match to all elements of that
name. Element names may be prefixed by the universe number followed by
the ``\vn{\@}'' sign. If a universe is not specified, the current universe
is used. Examples
\begin{example}
  "5@q*"       ! All elements whose name begins with "q" of universe 5.
  "*@sex10w"   ! Element "sex10w" of all universes.
  "b37"        ! Element "b37" of the current universe.
  "0@b37"      ! Same as the previous example.
\end{example}
Note: element names are {\em not} case sensitive.

An element index item is simply the index of the number in the lattice
list of elements. A prefix followed by the string ">>" can be used to
specify a branch. As with element names, a universe prefix can be 
given. Example
\begin{example}
  2@2>>183   ! Element number 183 of branch \# 2 of universe 2.
\end{example}

A range of elements is specified using the format:
\begin{example}
  \{<class>::\}<ele1>:<ele2>
\end{example}
\vn{<ele1>} is the element at the beginning of the range and
\vn{<ele2>} is the element at the end of the range. Either an element
name or index can be used to specify \vn{<ele1>} and \vn{<ele2>}. Both
\vn{<ele1>} and \vn{<ele2>} are part of the range. The optional \vn{<class>} 
prefix can be used to select only those elements in the range that match the class.
Example:
\begin{example}
  quad::sex10w:sex20w
\end{example}
This will select all quadrupoles between elements \vn{sex10w} and \vn{sex20w}.

\index{class::name}
A \vn{class::name} item
selects elements based upon their class (Eg: \vn{quadrupole},
\vn{marker}, etc.), and their name. The syntax is:
\begin{example}
  <element class>::<element name>
\end{example}
where \vn{<element class>} is an element class and \vn{<element
name>} is the element name that can (and generally does) contain the wild card characters
``\%'' and ``*''. Essentially this is an extension of the \vn{element name}
format. As with element names, a universe prefix can be 
given. Example:
\begin{example}
  "4@quad::q*"   ! All quadrupole whose name starts with "q" of universe 4.
\end{example}

%------------------------------------------------------------------------
\section{Arithmetic Expressions}
\index{arithmetic Expressions}
\label{s:arithmetic.exp}

\tao is able to handle arithmetic expressions within commands
(\sref{c:command}) and in strings in a \tao initialization file.
Arithmetic expressions can be used in a place where a real value or an
array of real values are required.  The standard operators are
defined: \hfil\break \hspace*{0.15in}
\begin{tabular}{ll}
  $a + b$           & Addition        \\
  $a - b$           & Subtraction     \\
  $a \, \ast \, b$  & Multiplication  \\
  $a \; / \; b$     & Division        \\
  $a \, \land \, b$ & Exponentiation  \\
\end{tabular} \newline
The following intrinsic functions are also recognized (this is the
same list as the \bmad parser): \hfil\break
\index{intrinsic functions}
\hspace*{0.15in}
\begin{tabular}{ll}
  \vn{sqrt}(x)      & Square Root    \\
  \vn{log}(x)       & Logarithm      \\
  \vn{exp}(x)       & Exponential    \\
  \vn{sin}(x)       & Sine           \\
  \vn{cos}(x)       & Cosine         \\
  \vn{tan}(x)       & Tangent        \\
  \vn{asin}(x)      & Arc sine       \\
  \vn{acos}(x)      & Arc cosine     \\
  \vn{atan}(y)      & Arc Tangent    \\
  \vn{atan2}(y, x)  & Arc Tangent    \\
  \vn{abs}(x)       & Absolute Value \\
  \vn{factorial(x)} & Factorial \\
  \vn{ran}()        & Random number between 0 and 1 \\
  \vn{ran_gauss}()  & Gaussian distributed random number with unit RMS \\
  \vn{int(x)}       & Nearest integer with magnitude less then x \\
  \vn{nint(x)}      & Nearest integer to x \\
  \vn{floor(x)}     & Nearest integer less than x \\
  \vn{ceiling(x)}   & Nearset integer greater than x \\
\end{tabular} \newline
Both \vn{ran} and \vn{ran_gauss} use a seeded random number generator. 
Setting the seed is described in Section~\sref{s:globals}.

See \sref{s:param.syntax} for the syntax for using data, variable, and
lattice parameter values in an expression.

%------------------------------------------------------------------------
\section{Syntax for Specifying Data, Variable, and Lattice parameter Values}
\label{s:param.syntax}

Data (\sref{s:data.org}) and variable values
(\sref{c:var}), along with lattice parameters can be
referenced in expressions. The general form of a term for
data and variables is
\begin{example}
  \{[<universe(s)>]@\}<source>::<var_or_data_name>[<index_list>]\{|<component>\}
\end{example}
For lattice and beam parameters the format is
\begin{example}
  \{[<universe(s)>]@\}<source>::<param_name>[\{<ref_element>&\}<element_list>]\{|<component>\}
\end{example}
And for element parameters the format is
\begin{example}
  \{<universe_range>@\}ele::<element_list>[<param_name>]\{|<component>\}
  \{<universe_range>@\}ele_mid::<element_list>[<param_name>]\{|<component>\}
\end{example}
\vn{lat::} will evaluate the parameter at the exit end of the element and 
\vn{lat_mid::} will evaluate the parameter at the middle of the element.
The components of this syntax are:
\begin{example}
  <universe_range>    Optional universe specification (\sref{s:universe})
  <var_or_data_name>  Variable or data name.
  <source>            Source of the data.
  <param_name>        Name of the parameter, datum, or variable.
  <index_list>        List of indexes.
  <ref_point>         Optional reference element (with lat or beam source only).
  <eval_points>       Evaluation point or points.
  <component>         Optional component. 
\end{example}
Examples:
\begin{example}
  3@lat::orbit.x[34:37]          Array of orbits at element 34 through 37 in universe 3.
  3@lat::orbit.x[34]|model       Orbit.x model value at element 34
  3@ele::34[orbit_x]             Same as above. 
  beam::sigma.x[q10w]            Beam sigma at element q10w.
  beam::n_particle_loss[2&56]    Particle loss between elements 2 and 56.
  3@lat::quad::q*[k1]            k1 of all quads in the model lattice of universe 3 
                                   with names beginning with "q".
  lat::3[l]|design               Length of the design lattice element #3 
                                   in the default universe.
  data::orbit.x[2:7,8]|model
\end{example}

The \vn{<source>} field may be one of:
\begin{example}
  beam        ! Value is from multiparticle beam tracking.
  data        ! Value is from a \tao datum in a data array (\sref{s:data.org}).
  ele         ! Value at exit end of element.
  ele_mid     ! Value at middle of element.
  lat         ! Value is from the lattice.
  var         ! Value is from a \tao variable (\sref{c:var}).
\end{example}

When a term has a \vn{data} source, The value of the term comes from a 
data structure. With a \vn{data} source, \vn{<var_or_dat_name>} is the name
of a \vn{d2_name.d1_name} data array (\sref{s:data.org}) and 
\vn{<index_list>} is a list of indexes. \vn{<index_list>} will determine
how many elements are in the array. For example, \vn{orbit.x[10:21,44]} 
represents an array of 13 elements. Finally, the optional \vn {<component>}
indicates what component of the datum is to be used. Possible components are
listed in Section~\sref{s:datum.values}. The default is \vn{model}.

When a term has a \vn{var} source, the value of the term comes from a
\tao variable (\sref{c:var}). With a \vn{var} source,
\vn{<var_or_data_name>} is the name of a \vn{v1_var} variable array. Like
terms with a \vn{data} source, \vn{<index_list>} is a list of indexes.
The optional \vn{<component>} indicates what component of the variable
is to be used. Possible components are listed in Section~\sref{c:var}.
The default is \vn{model}.

% Note: when combining components from the same \vn{data} or \vn{var} source
% in an expression, the common prefix can be eliminated. 
% For example
% \begin{example}
%   orbit.x[2:7,8]|model - 2*orbit.x[2:7,8]|meas
% \end{example}
% is equivalent to
% \begin{example}
%   orbit.x[2:7,8]|model - 2*meas
% \end{example}

When a term has a \vn{lat} or \vn{beam} source, the value of the term
comes from evaluation of the lattice or beam tracking. The
\vn{<param_name>} will be a datum type from Table~\ref{t:data.types}
table. Element list format (\sref{s:ele.list.format}) is used for the
\vn{element_list}. In this case, a universe specification is not
allowed as part of the element list. For example, the following is not
allowed
\begin{example}
  lat::orbit.x[3@q10w]    ! Bad syntax
\end{example}
The correct syntax is
\begin{example}
  3@lat::orbit.x[q10w]    ! Good syntax
\end{example}

The optional \vn{<ref_element>} specifies a reference
element for the evaluation. For example
\begin{example}
  lat::r.56[q0\&qa:qb]
\end{example}  
is an array of the $r(5,6)$ matrix element of the transport map
between element \vn{q0} and each element in the range from element
\vn{qa} and \vn{qb}. The optional \vn{<component>} indicates what
lattice is to be used. Possible components are
\begin{example}
  model (default)
  base
  design
\end{example}
The default is \vn{model}.

Notice the difference between, say, ``\vn{lat::orbit.x[10]}'' and
``\vn{data::orbit.x[10]}''. With the ``\vn{lat::}'' source, the element 
index, in this case \vn{10}, refers to the 10th lattice element. With the
``\vn{data::}'' source, ``\vn{10}'' refers to the 10\Th element in
the \vn{orbit.x} data array which may or may not correspond to the
10\Th lattice element and which may not even refer to orbit data.

Due to historical circumstances, the notation is a bit different for
terms with an \vn{ele} or \vn{ele_mid} source as opposed to terms with
a \vn{lat} or \vn{beam} source. With \vn{ele} or \vn{ele_mid}, the
\vn{<element_list>} comes before the \vn{<param_name>}. Additionally,
With \vn{ele} or \vn{ele_mid}, the \vn{<param_name>} uses the \bmad
naming convention as opposed to the \tao convention. For example, the
\tao name for the horizontal orbit is \vn{orbit.x} while the \bmad
name is \vn{orbit_x}.

Logical attributes of an element can be referred to. For example,
``\vn{ele::Q01W[is_on]}'' gives the on/off status of element
\vn{Q01W}. If used in an expression, True values are converted to a
value of \vn{1} and False value are converted to \vn{0}.

the value of the term comes from an element parameter such as the
element strength.  Element list format (\sref{s:ele.list.format}) is
used for the \vn{<element_list>} so an array of elements can be
defined.  The optional \vn{<component>} indicates what lattice is to
be used and is the same as for terms with a \vn{lat} or \vn{beam}
source (see above).

Generally, if the \vn{<source>} field is not present, for example, 
``\vn{orbit.x[10]}'', \tao will search 
for a match assuming that the source is either \vn{data} or \vn{var}. 
Exceptions will be noted in the manual.
