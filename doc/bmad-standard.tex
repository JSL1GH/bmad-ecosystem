\chapter{Bmad\_Standard Transfer Map Calculations}
\label{c:bmad.std}

For tracking and transfer map calculations (here generically called
``tracking''), \bmad has various methods that can be applied to a
given element (\sref{c:methods}). This chapter discusses the
\vn{bmad_standard} calculation that is the default for almost all
element types.

%-----------------------------------------------------------------
\section{Tracking with Non-Zero Offsets, Pitches, or Tilt}
\label{s:misalign.track}

\index{element reference coordinates}
The general procedure for tracking through an element makes use of the
\vn{element reference coordinates} Without any offsets, pitches or
tilt, (henseforth called ``misalignments''), the element reference
coordinates are the same as the \vn{laboratory reference coordinates}
(\sref{s:ref}). With misalignments, the \vn{element reference
coordinates} will stay fixed relative to the element and will
therefore shift as the element shifts in the laboratory fram.

\index{crystal}\index{mirror}
For \vn{crystal} (\sref{s:crystal} and \vn{mirror} (\sref{s:mirror})
elements, the non-straight reference trajectory through the element
complicates the calculation. For these elements, two element reference
coordinates are used. The first set of coordinates are called the
\vn{entrance element coordinates} and the second set the \vn{exit
element coordinates} as shown in \ref{f:local.coords}.
the \vn{element coordinates}, the vector normal to the surface is in
the $x-z$ plane.

Tracking a particle through an element is therefore a three
step transformation:
\begin{enumerate}
\item
At the entrance end of the element, transform from the
laboratory reference coordinates to the (entrance) \vn{element coordinates}.
\item
Track through the element ignoring any misalignments. For \vn{crystal}
and \vn{mirror} elements, this includes the transformation from
\vn{entrance} to \vn{exit element coordinates}.
\item
At the exit end of the element, transform from the \vn{element
reference from} to the \vn{laboratory reference frame}.
\end{enumerate}

The transformation between \vn{local} and \vn{element} reference frames
will depend upon whether \vn{local} reference frame through the
element is straight or not. 

%-----------------------------------------------------------------
\section{Transformation for a Straight Element Between 
Laboratory and Element Coordinates}

It is assumed that pitches are are small so that second
order terms can be ignored.

The transformation from the laboratory coordinates to
element coordinates for an element that has a
straight reference trajectory through it is accomplished in a number of steps.
\begin{enumerate}
\item
Track as in a drift a distance \vn{s_offset_tot}.
\item
Apply offsets and pitches
\begin{align}
  x_1    &= x_0 - x_{off} + \frac{L}{2} x'_{pitch} \CRNO
  p_{x1} &= p_{x0} - (1 + p_{z0}) \, x'_{pitch} \CRNO
  y_1    &= y_0 - y_{off} + \frac{L}{2} y'_{pitch} \\
  p_{y1} &= p_{x0} - (1 + p_{z0}) \, y'_{pitch} \CRNO
  z_1    &= z_0 + x'_{pitch} \, x_1 + y'_{pitch} \, y_1 - 
    \frac{L}{4} (x^{\prime2}_{pitch} + y^{\prime2}_{pitch}) \nonumber
\end{align}
Notice that $z_1$ is written in terms of $x_1$ and $y_1$
\item
Apply any tilt $\theta_t$
\begin{align}
  x_2    &=  x_1    \, \cos\theta_t + y_1    \, \sin\theta_t \CRNO
  p_{x2} &=  p_{x1} \, \cos\theta_t + p_{y1} \, \sin\theta_t \CRNO
  y_2    &= -x_1    \, \sin\theta_t + y_1    \, \cos\theta_t \\
  p_{y2} &= -p_{x1} \, \sin\theta_t + p_{y1} \, \cos\theta_t \nonumber
\end{align}
\end{enumerate}

The back transformation from element to laboratory coordinates is
accomplished by the transformation
\begin{enumerate}
\item
Apply any tilt $\theta_t$
\begin{align}
  x_1    &=  x_0    \, \cos\theta_t - y_0    \, \sin\theta_t \CRNO
  p_{x1} &=  p_{x0} \, \cos\theta_t - p_{y0} \, \sin\theta_t \CRNO
  y_1    &=  x_0    \, \sin\theta_t + y_0    \, \cos\theta_t \\
  p_{y1} &=  p_{x0} \, \sin\theta_t + p_{y0} \, \cos\theta_t \nonumber
\end{align}
\item
Apply offsets and pitches
\begin{align}
  x_2    &= x_1 + x_{off} + \frac{L}{2} x'_{pitch}     \CRNO
  p_{x2} &= p_{x1} + (1 + p_{z1}) \, x'_{pitch}        \CRNO
  y_2    &= y_1 + y_{off} + \frac{L}{2} y'_{pitch}     \\
  p_{y2} &= p_{x1} + (1 + p_{z1}) \, y'_{pitch}        \CRNO
  z_2    &= z_1 + x'_{pitch} \, x_1 + y'_{pitch} \, y_1 - 
    \frac{L}{4} (x^{\prime2}_{pitch} + y^{\prime2}_{pitch})      \nonumber
\end{align}
\item
Track as in a drift a distance \vn{-s_offset_tot}.
\end{enumerate}


%-----------------------------------------------------------------
\section[Mirror and Crystal Elements]{Transformation for Mirror and Crystal Elements Between 
Laboratory and Element Coordinates}

\index{mirror}\index{crystal}
It is assumed that pitches are are small so that second
order terms can be ignored.

\index{s_offset_tot}
With photons, the intensities must also be transformed.
The transformation from the the entrance laboratory coordinates to
the entrance element coordinates is:
\begin{enumerate}
\item
Track as in a drift a distance \vn{s_offset_tot}.
\item
\index{x_offset}\index{x_pitch}\index{y_offset}\index{y_pitch}
Apply offsets and pitches: The effective ``length'' of the element is
zero (\sref{s:mirror.coords}) so the origin of the element coordinates
is the same point around which the element is pitched so
\begin{align}
  x_1    &= x_0 - x_{off} \CRNO
  p_{x1} &= p_{x0} - (1 + p_{z0}) \, x'_{pitch} \CRNO
  y_1    &= y_0 - y_{off} \\
  p_{y1} &= p_{x0} - (1 + p_{z0}) \, y'_{pitch} \CRNO
  z_1    &= z_0 + x'_{pitch} \, x_1 + y'_{pitch} \, y_1 \nonumber
\end{align}
where $x_{off} \equiv \vn{x_offset}$, $x'_{pitch} \equiv \vn{x_pitch}$, etc.
\item
Apply \vn{tilt} and \vn{tilt_err}:
\begin{align}
  \begin{pmatrix} x_2 \\ y_2 \end{pmatrix} &=
    \bR (\theta_{tot}) \,   
  \begin{pmatrix} x_1 \\ y_1 \end{pmatrix} \CRNO
  \begin{pmatrix} p_{x2} \\ p_{y2} \end{pmatrix} &=
    \bR (\theta_{tot}) \, 
  \begin{pmatrix} p_{x1} \\ p_{y1} \end{pmatrix} \label{xyrtxy} \\ 
  \begin{pmatrix} \bE_{x2} \\ \bE_{y2} \end{pmatrix} &=
    \bR (\theta_{tot}) \,   \begin{pmatrix} \bE_{x1} \\ \bE_{y1} \end{pmatrix} \nonumber
\end{align}
where $\bE$ is shorthand notation for
\Begineq
  \bE \equiv \sqrt{I} \, e^{i \, \phi}
\Endeq
with $I$ being the field intensity and $\phi$ being the field phase angle.
In the above equations $\bR$ is the rotation matrix
\Begineq
  \bR(\theta) = \begin{pmatrix} \cos\theta & \sin\theta \\ -\sin\theta & \cos\theta \end{pmatrix}
\Endeq
\index{tilt}\index{tilt_err}\index{tilt_corr}
with $\theta_{tot}$ being 
\Begineq
  \theta_{tot}  = 
  \begin{cases}
    \vn{tilt} + \vn{tilt_err} + \vn{tilt_corr} & \vn{for crystal elements} \\
    \vn{tilt} + \vn{tilt_err} & \vn{for mirror elements}
  \end{cases}
\Endeq
The \vn{tilt_corr} correction is explained below.
\item
\index{graze_angle_err}
Apply \vn{graze_angle_err} ($\theta_{gerr}$):
\begin{align}
  p_{x3} &= p_{x2} + \theta_{gerr} \CRNO
  z_3    &= z_2 - x_2 \, \theta_{gerr} 
\end{align}
\end{enumerate}

The back transformation from exit element coordinates to exit
laboratory coordinates is accomplished by the transformation
  \begin{enumerate}
  \item
Unapply \vn{graze_angle_err}:
\begin{align}
  p_{x1} &= p_{x0} - \theta_{gerr} \CRNO
  z_1    &= z_0 + x_0 \, \theta_{gerr} 
\end{align}
  \item
Unapply \vn{tilt} and \vn{tilt_err}: \vn{tilt} rotates the exit
laboratory coordinants with respect to the exit element coordinates in
the same way \vn{tilt} rotates the entrance laboratory coordinates
with respect to the entrance element coordinants. The forward and back
transformations are thus just inverses of each other.  With
\vn{tilt_err}, this is not true. \vn{tilt_err}, unlike \vn{tilt}, does
not rotate the output laboratory coordinates.  There is the further
complication in that \vn{tilt_err} is a rotation about the {\em
entrance} laboratory coordinates. The first step is to express
\vn{tilt_err} with respect to the exit coordinants. This is done with
the help of the $\bS$ matrix of \Eq{r00ls} with $\alpha$ given by
\Eq{agg}. The effect of the \vn{tilt_err} can be modeled as a rotation
vector $\Bf e_{in}$ in the entrance laboratory coordinates pointing
along the $z$-axis
\Begineq
 \Bf e_{in} = (0, 0, \mbox{tilt_err})
\Endeq
In the exit laboritory coordinates, the vector $\Bf e_{out}$ is
\Begineq
  \Bf e_{out} = \bS \, \Bf e_{in}
\Endeq
The $z$ component of $\Bf e_{out}$ combines with \vn{titlt} to give
the transformation
\begin{align}
  \begin{pmatrix} x_2 \\ y_2 \end{pmatrix} &=
    \bR (-\theta_{t}) \,   \begin{pmatrix} x_1 \\ y_1 \end{pmatrix} \CRNO
  \begin{pmatrix} p_{x2} \\ p_{y2} \end{pmatrix} &=
    \bR (-\theta_{t}) \,   \begin{pmatrix} p_{x1} \\ p_{y1} \end{pmatrix} \\
  \begin{pmatrix} \bE_{x2} \\ \bE_{y2} \end{pmatrix} &=
    \bR (-\theta_{t}) \,   \begin{pmatrix} \bE_{x1} \\ \bE_{y1} \end{pmatrix} \nonumber
\end{align}
where $\theta_t$ is $\mbox{tilt} + \Bf e_{out,z}$. The $x$ and $y$ components
of $\Bf e_{out}$ give rotations around the $x$ and $y$ axes
\begin{align}
  p_{x3} &= p_{x2} - \Bf e_{out,y} \CRNO
  p_{y3} &= p_{y2} + \Bf e_{out,x} \\
  z_3    &= z_2 + x_2 \, \Bf e_{out,y} - y_2 \, \Bf e_{out,x}
\end{align}
  \item
Unapply pitches: Since pitches are defined with
respect to the entrance laboratory coordinates, they have to be
translated to the exit laboratory coordinates
\Begineq
  \bP_{out} = \bS \, \bP_{in}
\Endeq
where $\bP_{in} = (x'_{pitch}, y'_{pitch}, 0)$ is the pitch vector in
the entrance laboratory frame and $\bP_{out}$ is the vetor in the exit
laboratory frame. The transformation is then
\begin{align}
  p_{x4} &= p_{x3} - \bP_{out,y} \CRNO
  p_{y4} &= p_{y3} + \bP_{out,x} \\
  z_4    &= z_3 + x_3 \, \bP_{out,y} - y_3 \, \bP_{out,x}
\end{align}
  \item
Unapply offsets: Again, offsets are defined with respect to the
entrance laboratory coordinates. Like pitches, the translation is
\Begineq
  \bO_{out} = \bS \, \bO_{in}
\Endeq
where $\bO_{in} = (x_{off}, y_{off}, s_{off})$ is the offset in the
entrance laboratory frame. The transformation is
\begin{align}
  x_5 &= x_4 + \bO_{out,x} - p_{x4} \, \bO_{out,z} \CRNO
  y_5 &= y_4 + \bO_{out,y} - p_{y4} \, \bO_{out,z} \\
  z_5 &= z_4 + \bO_{out,z} 
\end{align}
  \end{enumerate}

%-----------------------------------------------------------------
\section{Crystal Element Tracking}
\label{s:crystal.tracking}

\textit{\large [Crystal tracking developed by Jing Yee Chee, Ken Finkelstein, and David Sagan]}

Crystal diffraction is modeled using dynamical diffraction
theory\cite{b:batterman}.


%-----------------------------------------------------------------
\section{Hamiltonian for a Magnetic Element}
\label{s:mag.hamiltonian}

Without any electric fields, the Hamiltonian is
\Begineq
  H = p_z - (1 + g \, x) \sqrt{(1 + p_z)^2 - (p_x - a_x)^2 - (p_y - a_y)^2} - 
  (1 + g \, x) \, a_s
  \label{h1gx1}
\Endeq
Here $(x, p_x, y, p_y, z, p_z)$ are the canonical coordinates
(\sref{s:phase.space}), $g = 1/\rho$ with $\rho$ being the local
radius of curvature of the reference particle, and $\Bf a(x,y,s)$ is
the normalized vector potential which is related to the vector
potential $\Bf A(x,y,s)$ via
\Begineq
  \Bf a = \frac{q \, \Bf A}{P_0 \, c}
\Endeq
The equations of motion are
\begin{align}
  \frac{dq_i}{ds} &= \frac{\partial H}{\partial p_i} \CRNO
  \frac{dp_i}{ds} &= -\frac{\partial H}{\partial q_i}
  \label{rshp}
\end{align}

Assuming mid--plane symmetry of the magnetic field, so
that $a_x$ and $a_y$ can be set to zero\cite{b:madphysics}, The vector
potential up to second order is (cf.~\Eq{byx0b})
\Begineq
  a_s = -k_0 \left( x - \frac{g \, x^2}{2 (1 + g\, x)} \right) -
  \frac{1}{2} k_1 \left( x^2 - y^2 \right)
\Endeq

\label{paraxial approximation}
Using the paraxial approximation (\sref{s:phase.space}), \Eq{h1gx1} becomes
\Begineq
  H = \frac{(p_x - a_x)^2}{2 (1 + p_z)} + \frac{(p_y - a_y)^2}{2 (1 + p_z)} - 
  (1 + g \, x) \, a_s 
  \label{hpapa}
\Endeq

Once the transverse trajectory has been calculated, the longitudinal position
$z_2$ at the exit end of an element is obtained from symplectic
integration of \Eq{hpapa}
\Begineq
  z_2 = z_1 - \frac{1}{2 (1 + p_{z1})^2} \int \! ds \, 
  \left[ (p_x - a_x)^2 + (p_y - a_y)^2 \right] - \int \! ds \, g \, x
  \label{zz121p}
\Endeq
where $z_1$ is the longitudinal position at the entrance end of the element.
Using the equations of motion \Eqs{rshp} this can also be rewritten as
\Begineq
  z_2 = z_1 - \frac{1}{2} \int \! ds \, 
  \left[ \left( \frac{dx}{ds} \right)^2 + \left( \frac{dy}{ds} \right)^2 \right] - 
  \int \! ds \, g \, x
  \label{zz12sx}
\Endeq

For some elements, \vn{bmad_standard} uses a truncated Taylor map for
tracking.  For elements without electric fields where the particle
energy is a constant, the transfer map for a given coordinate $r_i$
may be expanded in a Taylor series
\Begineq
  r_{i,2} \rightarrow m_i + \sum_{j = 1}^4 m_{ij} \, r_{j,1} + 
  \sum_{j = 1}^4 \sum_{k = j}^4 m_{ijk} \, r_{j,1} \, r_{k,1} + \ldots
\Endeq
where the map coefficients $m_{ij\cdots}$ are functions of $p_z$.  For
linear elements, the transfer map is linear for the transverse
coordinates and quadratic for $r_i = z$.

%-----------------------------------------
\section{BeamBeam Element Tracking}
\label{s:beambeam.std}
\index{beambeam}

A beam-beam element (\sref{s:bbi}) simulates the effect on a tracked
particle of an opposing beam of particles moving in the opposite
direction. The opposing beam, called the ``strong'' beam, is assumed
to be Gaussian in shape.

The strong beam is divided up into \vn{n_slice} equal charge (not
equal thickness) slices. Propagation through the strong beam involves
a kick at the charge center of each slice with drifts in between the
kicks. The kicks are calculated using the standard Bassetti--Erskine
complex error function formula\cite{b:talman}.  Even though the strong
beam can have a finite \vn{sig_z} the length of the element is always
considered to be zero. This is achieved by adding drifts at either end
of any tracking so that the longitudinal starting point and ending
point are identical. The longitudinal $s$--position of the
\vn{BeamBeam} element is at the center of the strong bunch. For
example, with \vn{n_slice} = 2 the calculation would proceed as
follows:
\begin{enumerate}
  \item  Start with the reference particle at the center of the strong bunch.
  \item  Propagate (drift) backwards to the center of the first slice.
  \item  Apply the beam--beam kick due to the first slice.
  \item  Propagate (drift) forwards to the center of the second slice.
  \item  Apply the beam--beam kick due to the second slice.
  \item  Propagate (drift) backwards to end up with the reference particle
     at the center of the strong bunch.
\end{enumerate}

%-----------------------------------------
\section{Bend Element: Fringe Tracking}
\label{s:.bend.fringe.std}
\index{sbend}

The transformation for the entrance face of an \vn{sbend} is
\begin{align}
  p_{x2} &= p_{x1} + k_x \, x_1 \CRNO
  p_{y2} &= p_{y1} + k_y \, y_1
\end{align}
where
\begin{align}
  k_x &= g_{tot} \, \tan(e_1) \CRNO
  k_y &= -g_{tot} \, \tan \left[ e_1 - 2 \, |g_{tot}| \, f_{int} \,  h_{gap} \, 
    \frac{1 + \sin(e1)^2}{\cos(e_1)} \right]
\end{align}
where $g_{tot}$ is the total bending strength (design +
error). Similar equations are used for tracking the exit edge of the
bend.

%-----------------------------------------
\section{Bend Element: Body Tracking}
\label{s:bend.body.std}
\index{sbend}

The Hamiltonian for the body of an \vn{sbend} is
\Begineq
  H = (k_0 - g) x - g \, x \, p_z + 
  \frac{1}{2}\left( (k_1 + g \, k_0) x^2 - k_1 \, y^2 \right) +
  \frac{p_x^2 + p_y^2}{2 (1 + p_z)} 
\Endeq

This is simply solved
\begin{align}
  x_2    &= c_x \, (x - x_c) + s_x \, \frac{p_{x1}}{1 + p_{z1}} + x_c \CRNO
  p_{x2} &= \tau_x \, \om_x^2 \, \, (1 + p_{z1}) \, s_x \, (x -x_c) + c_x \, p_{x1} \CRNO
  y_2    &= c_y \, y_1 + s_y \, \frac{p_{y1}}{1 + p_{z1}} \CRNO
  p_{y2} &= \tau_y \, \om_y^2 \, \, (1 + p_{z1}) \, s_y \, y_1 + c_y \, p_{y1} \\
  z_2    &= z_1 + m_5 + m_{51} (x - x_c) + m_{52} p_{x1} + \CRNO
         &\hspace*{20ex} m_{511} \, (x-x_c)^2 + m_{512} \, (x-x_c) \, p_{x1} + m_{522} \, p_{x1}^2 + 
                         m_{533} \, y^2 + m_{534} \, y_1 \, p_{y1} + m_{544} \, p_{y1}^2 \CRNO
  p_{z2} &= p_{z1} \nonumber
\end{align}
where 
\begin{alignat}{2}
  k_x &= k_1 + g \, k_0 & \qqquad
  \om_x &\equiv \sqrt{\frac{|k_x|}{1 + p_{z1}}} \CRNO
  x_c &= \frac{g \, (1 + p_{z1}) - k_0}{k_x} & \qqquad
  \om_y &\equiv \sqrt{\frac{|k_1|}{1 + p_{z1}}} 
\end{alignat}
and
\begin{alignat}{6}
         &\hspace*{3ex}  && k_x > 0          &\hspace*{3ex}& k_x < 0 & \qqquad
         &\hspace*{3ex}  && k_1 > 0          &\hspace*{3ex}& k_1 < 0 \CRNO
     c_x &=   && \cos  (\om_x \, L)               && \cosh (\om_x \, L) & \qqquad
     c_y &=   && \cosh (\om_y \, L)               && \cos  (\om_y \, L) \CRNO
     s_x &=   && \frac{\sin  (\om_x \, L)}{\om_x} && \frac{\sinh (\om_x \, L)}{\om_x} & \qqquad
     s_y &=   && \frac{\sinh (\om_y \, L)}{\om_y} && \frac{\sin  (\om_y \, L)}{\om_y} \\
  \tau_x &=   && {-}1             && {+}1             & \qqquad
  \tau_y &=   && {+}1             && {-}1             \nonumber
\end{alignat}
and
\begin{alignat}{2}
  m_5     &= -g \, x_c \, L & \qqquad & \CRNO
  m_{51}  &= -g \, s_x & \qqquad
  m_{52}  &= \frac{\tau_x \, g}{1 + p_{z1}} \, \frac{1 - c_x}{\om_x^2} \CRNO
  m_{511} &= \frac{\tau_x \,\, \om_x^2}{4} \, (L - c_x \, s_x) & \qqquad
  m_{533} &= \frac{\tau_y \,\, \om_y^2}{4} \, (L - c_y \, s_y) \CRNO
  m_{512} &= \frac{-\tau_x \,\, \om_x^2}{2 \, (1 + p_{z1})} \, s_x^2 & \qqquad
  m_{534} &= \frac{-\tau_y \,\, \om_y^2}{2 \, (1 + p_{z1})} \, s_y^2 \CRNO
  m_{522} &= \frac{-1}{4 \, (1 + p_{z1})^2} \, (L + c_x \, s_x) & \qqquad
  m_{544} &= \frac{-1}{4 \, (1 + p_{z1})^2} \, (L + c_y \, s_y) \nonumber
\end{alignat}

%-----------------------------------------
\section{Drift Element Tracking}
\label{s:drift.std}
\index{drift} 

\Eq{h1gx1} for a drift has $\Bf a = 0$ and $g = 0$. The Hamiltonian for a
drift is then
\Begineq
  H = \frac{p_x^2 + p_y^2}{2 (1 + p_z)} 
\Endeq
This gives the map
\begin{align}
  x_2    &= x_1 + \frac{L \, p_{x1}}{1 + p_{z1}} \CRNO
  p_{x2} &= p_{x1}  \CRNO
  y_2    &= y_1 + \frac{L \, p_{y1}}{1 + p_{z1}} \CRNO
  p_{y2} &= p_{y1}  \\
  z_2    &= z_1 - \frac{L \, (p_{x1}^2 + p_{y1}^2)}{2 (1 + p_{z1})^2} \CRNO
  p_{z2} &= p_{z1} \nonumber
\end{align}

%-----------------------------------------
\section{Kicker, Hkicker, Vkicker, and Elseparator Element Tracking}
\label{s:kicker.std}
\index{kicker}
\index{hkicker}
\index{vkicker}
\index{elseparator}

The Hamiltonian for a horizontally deflecting kicker or separator is
\Begineq
  H = \frac{p_x^2 + p_y^2}{2 (1 + p_z)} - k_0 \, x 
\Endeq
This gives the map
\begin{align}
  x_2    &= x_1 + \frac{1}{1 + p_{z1}} \, \left( L \, p_{x1} + \frac{1}{2} k_0 \, L^2 \right) \CRNO
  p_{x2} &= p_{x1} + k_0 \, L \CRNO
  y_2    &= y_1 + \frac{L \, p_{y1}}{1 + p_{z1}} \CRNO
  p_{y2} &= p_{y1}  \\
  z_2    &= z_1 - \frac{L}{2 (1 + p_{z1})^2} \, 
    \left( p_{x1}^2 + p_{y1}^2 + p_{x1} \, k_0 \, L + \frac{1}{3} k_0^2 \, L^2 \right) \CRNO
  p_{z2} &= p_{z1} \nonumber
\end{align}
The generalization when the kick is not in the horizontal plane is easily derived.

%-----------------------------------------
\section{LCavity Element Tracking}
\label{s:lcavity.std}
\index{lcavity}

The transverse trajectory through an \vn{Lcavity} is modeled using equations
developed by Rosenzweig and Serafini\cite{b:rosenzweig} with
\Begineqs
  b_0 &= 1 \CRNO
  b_{-1} &= 1 
\Endeqs
and all other $b_n$ set to zero.

The transport through the body (R\&S Eq.~(9)) has been modified to give the 
correct phase-space area at non ultra-relativistic energies:
\Begineq
  \begin{pmatrix}
    x \\ 
    x'
  \end{pmatrix}_2 = 
  \begin{pmatrix}
    m_{11}                      & \beta_1 \, m_{12} \\
    \frac{1}{\beta_2} \, m_{21} & \frac{\beta_1}{\beta_2} \, m_{22} 
  \end{pmatrix}
  \,
  \begin{pmatrix}
    x \\ 
    x'
  \end{pmatrix}_1
\Endeq
where the $m_{ij}$ are the matrix elements from R\&S Eq.~(9) and the 
$\beta$ are the standard relativistic factors. With this, the determinate 
of the matrix is $\beta_1 \, \gamma_1 / \beta_2 \, \gamma_2$.

The change in $z$ going through a cavity is calculated by first calculating the particle
transit time $\Delta t$
\begin{align}
  c \, \Delta t &= \int_{s_1}^{s_2} \!\! ds \,\, \frac{1}{\beta(s)} \CRNO
  &= \int_{s_1}^{s_2} \!\! ds \, \frac{E}{\sqrt{E^2 - (mc^2)^2}} \\
  &= \frac{c \, P_{z2} - c \, P_{z1}}{G} \nonumber
\end{align}
where it has been assumed that the accelerating gradient $G$ is
constant through the cavity. In this equation $\beta = v / c$, $E$ is
the energy, and $P_{z1}$ and $P_{z2}$ are the entrance and exit
momenta. Using \Eq{zbctt}, the change in $z$ is thus
\Begineq
  z_2 = \frac{\beta_2}{\beta_1} \, z_1 - 
  \beta_2 \, 
  \left(
  \frac{c \, P_{z2} - c \, P_{z1}}{G} - 
  \frac{c \, \Pbar_{z2} - c \, \Pbar_{z1}}{\BAR G}
  \right)
\Endeq
where $\Pbar$ and $\BAR G$ are the momentum and gradient of the
reference particle.

The derivatives are straight forward if tedious
\begin{align}
  m_{55} &= \frac{dz_2}{dz_1} = 
    \frac{\beta_2}{\beta_1} + 
    \frac{z_2 \, m_{65}}{\beta_2} \frac{d\beta_2}{dp_{z2}} - 
    \frac{\beta_2}{G} \frac{dcP_2}{dz_1} +
    \frac{\beta_2 \, c \, (P_2 - P_1) \, c P_2}{L \, G^2 \, E_2} 
      \frac{dcP_2}{dz_1} \CRNO
  m_{56} &= \frac{dz_2}{dp_{z1}} = 
    \frac{-\beta_2 \, z_1}{\beta_1^2} \frac{d\beta_1}{dp_{z1}} + 
    \frac{z_2 \, m_{66}}{\beta_2} \frac{d\beta_2}{dp_{z2}} -
    \frac{\beta_2 ( c \Pbar_{2} \, m_{66} - c \Pbar_{1})}{G} \CRNO
  m_{65} &= \frac{dp_{z2}}{dz1} =
    \frac{E_2}{cP_2 \, c\Pbar_{2}} \frac{cP_1 \, c\Pbar_{1}}{E_1}  \\
  m_{66} &= \frac{dp_{z2}}{dp_{z1}} = 
    \frac{E_2}{cP_2} \frac{G \, L}{c \Pbar_{2}} \frac{2 \, \pi \, f \, \sin\phi}{c}
    \nonumber
\end{align}
where
\begin{align}
  \frac{d\beta_1}{dp_{z1}}  &= \frac{(mc^2)^2}{E_1^3} \, c\Pbar_{1} \CRNO
  \frac{d\beta_2}{dp_{z2}}  &= \frac{(mc^2)^2}{E_2^3} \, c\Pbar_{2} \\
  \frac{dcP_2}{dz_1}        &= m_{65} \, c\Pbar_{2}  \nonumber
\end{align}

%-----------------------------------------
\section{Mirror Element Tracking}
\label{s:mirror.std}
\index{mirror}

Tracking through

%-----------------------------------------
\section{Octupole Element Tracking}
\label{s:octupole.std}
\index{octupole}

The Hamiltonian for an upright octupole is
\Begineq
  H = \frac{p_x^2 + p_y^2}{2 (1 + p_z)} + \frac{k_3}{24} (x^4 - 6 \, x^2 \, y^2 + y^4)
\Endeq

An octupole is modeled using a kick-drift-kick model.

%-----------------------------------------
\section{Quadrupole Element Tracking}
\label{s:quadrupole.std}
\index{quadrupole}

The \vn{bmad_standard} calculates the transfer map through an upright
quadrupole and then transforms that map to the laboratory frame.

The Hamiltonian for an upright quadrupole is
\Begineq
  H = \frac{p_x^2 + p_y^2}{2 (1 + p_z)} + \frac{k_1}{2} (x^2 - y^2)
\Endeq
This is simply solved
\begin{align}
  x_2    &= c_x \, x_1 + s_x \, \frac{p_{x1}}{1 + p_{z1}} \CRNO
  p_{x2} &= \tau_x \, \om^2 \, \, (1 + p_{z1}) \, s_x \, x_1 + c_x \, p_{x1} \CRNO
  y_2    &= c_y \, y_1 + s_y \, \frac{p_{y1}}{1 + p_{z1}} \CRNO
  p_{y2} &= \tau_y \, \om^2 \, \, (1 + p_{z1}) \, s_y \, y_1 + c_y \, p_{y1} \\
  z_2    &= z_1 + m_{511} \, x_1^2 + m_{512} \, x_1 \, p_{x1} + m_{522} \, p_{x1}^2 + 
                   m_{533} \, y_1^2 + m_{534} \, y_1 \, p_{y1} + m_{544} \, p_{y1}^2 \CRNO
  p_{z2} &= p_{z1} \nonumber
\end{align}
where 
\Begineq
  \om \equiv \sqrt{\frac{|k_1|}{1 + p_{z1}}}
\Endeq
and
\begin{alignat}{6}
         &\hspace*{3ex}  && k_1 > 0          &\hspace*{3ex}& k_1 < 0 & \qqquad
         &\hspace*{3ex}  && k_1 > 0          &\hspace*{3ex}& k_1 < 0 \CRNO
     c_x &=   && \cos  (\om \, L) && \cosh (\om \, L) & \qqquad
     c_y &=   && \cosh (\om \, L) && \cos  (\om \, L) \CRNO
     s_x &=   && \frac{\sin  (\om \, L)}{\om} && \frac{\sinh (\om \, L)}{\om} & \qqquad
     s_y &=   && \frac{\sinh (\om \, L)}{\om} && \frac{\sin  (\om \, L)}{\om} \\
  \tau_x &=   && {-}1             && {+}1             & \qqquad
  \tau_y &=   && {+}1             && {-}1             \nonumber
\end{alignat}
with this
\begin{alignat}{2}
  m_{511} &= \frac{\tau_x \,\, \om^2}{4} \, (L - c_x \, s_x) & \qqquad
  m_{533} &= \frac{\tau_y \,\, \om^2}{4} \, (L - c_y \, s_y) \CRNO
  m_{512} &= \frac{-\tau_x \,\, \om^2}{2 \, (1 + p_{z1})} \, s_x^2 & \qqquad
  m_{534} &= \frac{-\tau_y \,\, \om^2}{2 \, (1 + p_{z1})} \, s_y^2 \\
  m_{522} &= \frac{-1}{4 \, (1 + p_{z1})^2} \, (L + c_x \, s_x) & \qqquad
  m_{544} &= \frac{-1}{4 \, (1 + p_{z1})^2} \, (L + c_y \, s_y) \nonumber
\end{alignat}


%-----------------------------------------
\section{RFcavity Element Tracking}
\label{s:rfcavity.std}
\index{rfcavity}

%-----------------------------------------
\section{Sextupole Element Tracking}
\label{s:sextupole.std}
\index{sextupole}

The Hamiltonian for an upright octupole is
\Begineq
  H = \frac{p_x^2 + p_y^2}{2 (1 + p_z)} + \frac{k_2}{6} (x^3 - 3 \, x \, y^2)
\Endeq

A sextupoleis modeled using a kick-drift-kick model.

%-----------------------------------------
\section{Sol\_Quad Element Tracking}
\label{s:sol.quad.std}
\index{sol_quad}

The Hamiltonian is
\Begineq
  H = \frac{(p_x + \frac{k_s }{2}\, y)^2}{2 (1 + p_z)} + 
  \frac{(p_y - \frac{k_s}{2} \, x)^2}{2 (1 + p_z)} + \frac{k_1}{2} (x^2 - y^2)
\Endeq
Solving the equations of motion gives
\begin{align}
  x_2    &= m_{11} \, x_1 + m_{12} \, p_{x1} + m_{13} \, y_1 + m_{14} \, p_{y1} \CRNO
  p_{x2} &= m_{21} \, x_1 + m_{22} \, p_{x1} + m_{23} \, y_1 + m_{24} \, p_{y1} \CRNO
  y_2    &= m_{31} \, x_1 + m_{32} \, p_{x1} + m_{33} \, y_1 + m_{34} \, p_{y1} \CRNO
  p_{y2} &= m_{41} \, x_1 + m_{42} \, p_{x1} + m_{43} \, y_1 + m_{44} \, p_{y1} \\
  z_2    &= z_1 + \sum_{j = 1}^4 \sum_{k = j}^4 m_{5jk} \, r_j \, r_k  \CRNO
  p_{z2} &= p_{z1} \nonumber
\end{align}
where
\begin{alignat}{2}
  m_{11} &= \frac{1}{2 \, f} \, \left( f_{0+} \, c + f_{0-} \, c_h \right) & \qqquad
  m_{31} &= -m_{24} \CRNO
  m_{12} &= \frac{1}{2 \, f \, (1 + p_{z1})} \, 
            \left( \frac{f_{++}}{\om_+} \,  s + \frac{f_{--}}{\om_-} \, s_h \right) & \qqquad
  m_{32} &= -m_{14} \CRNO
  m_{13} &= \frac{\ks}{4 \, f} \, 
            \left( \frac{f_{+-}}{\om_+} \, s +\frac{f_{-+}}{\om_-} \, s_h \right) & \qqquad
  m_{33} &= \frac{1}{2 \, f} \, \left( f_{0-} \, c + f_{0+} \, c_h \right) \CRNO
  m_{14} &= \frac{\ks}{f \, (1 + p_{z1})} \, \left( -c + c_h \right) & \qqquad
  m_{34} &= \frac{1}{2 \, f \, (1 + p_{z1})} \, 
            \left( \frac{f_{+-}}{\om_+} \, s + \frac{f_{-+}}{\om_-} \, s_h \right) \CRNO
  m_{21} &= \frac{-(1 + p_{z1})}{8 \, f} \, 
            \left( \frac{\xi_{1+}}{\om_+} \, s + \frac{\xi_{2+}}{\om_-} s_h \right) & \qqquad
  m_{41} &= -m_{23} \\
  m_{22} &= m_{11} & \qqquad
  m_{42} &= -m_{13} \CRNO
  m_{23} &= \frac{\ks^3 \, (1 + p_{z1})}{4 \, f} \, \left( c - c_h \right) & \qqquad
  m_{43} &= \frac{-(1 + p_{z1})}{8 \, f} \, 
            \left( \frac{\xi_{1-}}{\om_+} \, s + \frac{\xi_{2-}}{\om_-} \, s_h \right) \CRNO
  m_{24} &= \frac{\ks}{4 \, f} \, 
            \left( \frac{f_{++}}{\om_+} \, s + \frac{f_{--}}{\om_-} \, s_h \right) & \qqquad
  m_{44} &= m_{33} \nonumber
\end{alignat}
and
\begin{alignat}{2}
  \kone        &= \frac{k_1}{1 + p_{z1}} & \qqquad 
  \ks          &= \frac{k_s}{1 + p_{z1}} \CRNO
  f            &= \sqrt{\ks^4 + 4 \, \kone^2} & \qqquad
  f_{\pm0}     &= f \pm \ks^2 \CRNO
  f_{0\pm}     &= f \pm 2 \, \kone & \qqquad
  f_{\pm\pm}   &= f \pm \ks^2 \pm 2 \, \kone \CRNO
  \om_+        &= \sqrt{\frac{f_{+0}}{2}} & \qqquad
  \om_-        &= \sqrt{\frac{f_{-0}}{2}} \\
  s            &= \sin (\om_+ \, L) & \qqquad
  s_h          &= \sinh (\om_- \, L) \CRNO
  c            &= \cos (\om_+ \, L) & \qqquad
  c_h          &= \cosh (\om_- \, L) \CRNO
  \xi_{1\pm} &= \ks^2 \, f_{+\mp} \pm 4 \, \kone \, f_{+\pm} & \qqquad
  \xi_{2\pm} &= \ks^2 \, f_{-\pm} \pm 4 \, \kone \, f_{-\mp} \nonumber
\end{alignat}

The $m_{5jk}$ terms are obtained via \Eq{zz121p}
\begin{align}
  m_{5jk} = - \frac{\tau_{jk}}{2 (1 + p_{z1})^2} \int \! ds \, 
  & \left[ 
    \left( m_{2j} + \frac{k_s}{2} \, m_{3j} \right) \, 
    \left( m_{2k} + \frac{k_s}{2} \, m_{3k} \right)   
  \right. + \\
  & \hspace*{15ex} \left.
    \left( m_{4j} - \frac{k_s}{2} \, m_{1j} \right) \, 
    \left( m_{4k} - \frac{k_s}{2} \, m_{1k} \right) 
  \right] \nonumber
\end{align}
where
\Begineq
  \tau_{jk} = 
  \begin{cases}
    1 & j = k \\
    2 & j \ne k 
  \end{cases}
\Endeq
The needed integrals involve the product of two trigonometric or
hyperbolic functions. These integrals are trivial to do but the
explicit equations for $m_{5jk}$ are quite long and in the interests of
brevity are not reproduced here.

%-----------------------------------------
\section{Solenoid Element Tracking}
\label{s:solenoid.std}
\index{solenoid}

The Hamiltonian is
\Begineq
  H = \frac{ \left( p_x + \frac{k_s}{2} \, y \right)^2}{2 \, (1 + p_z)} + 
  \frac{ \left( p_y - \frac{k_s}{2} \, x \right)^2}{2 \, (1 + p_z)} 
\Endeq
The solution is
\begin{align}
  x_2    &= \frac{1 + c}{2} \, x_1 + \frac{s}{k_s} \, p_{x1} +
           \frac{s}{2} \, y_1 + \frac{1 - c}{k_s} \, p_{y1} \CRNO
  p_{x2} &= \frac{-k_s \, s}{4} \, x_1 + \frac{1 + c}{2} \, p_{x1} - 
           \frac{k_s \, (1 - c)}{4} \, y_1 + \frac{s}{2} \, p_{y1} \CRNO
  y_2    &= \frac{-s}{2} \, x_1 - \frac{1 - c}{k_s} \, p_{x1} +
           \frac{1 + c}{2} \, y_1 + \frac{s}{k_s} \, p_{y1} \\      
  p_{y2} &= \frac{k_s \, (1 - c)}{4} \, x_1 + \frac{-s}{2} \, p_{x1} -
            \frac{k_s \, s}{4} \, y_1 + \frac{1 + c}{2} \, p_{y1} \CRNO 
  z_2    &= z_1 + \frac{L}{2 \, (1 + p_{z1})^2} \, 
                   \left[ \left( p_{x1} + \frac{k_s}{2} \, y_1 \right)^2 +
                          \left( p_{y1} - \frac{k_s}{2} \, x_1 \right)^2 \right] \CRNO
  p_{z2} &= p_{z1} \nonumber
\end{align}
where
\begin{align}
  c &= \cos \left( \frac{k_s}{2} \, L \right) \CRNO
  s &= \sin \left( \frac{k_s}{2} \, L \right)
\end{align}

%-----------------------------------------
\section{Map\_type Wiggler Element Tracking}
\label{s:wiggler.map.std}
\index{wiggler}

\index{wiggler}
As discussed in \sref{s:wiggler}, \bmad wiggler elements are
split into two classes: \vn{map type} and \vn{periodic type}. The 
\vn{map type} wigglers are modeled using the method of Sagan, Crittenden,
and Rubin\cite{b:wiggler}. In this model the magnetic field is written
as a sum of terms $B_i$
\Begineq
  \bfB(x,y,s) = \sum_i \bfB_i(x, y, s; C, k_x, k_y, k_s, \phi_s)
\Endeq 
Each term $B_i$ is specified using five numbers: 
$(C, k_x, k_y, k_s, \phi_s)$. A term can take one of three forms: The first
form is
\begin{alignat}{4}
  B_x &= -&C &\dfrac{k_x}{k_y} & \sin(\kxx) \sinh(\kyy) \cos(\ksss) \CRNEG
  B_y &=  &C &                 & \cos(\kxx) \cosh(\kyy) \cos(\ksss) \CRNEG
  B_s &= -&C &\dfrac{k_s}{k_y} & \cos(\kxx) \sinh(\kyy) \sin(\ksss) \label{f1} \\
  & \makebox[1pt][l]{with $k_y^2 = k_x^2 + k_s^2$ .} &&&  \nonumber
\end{alignat}
The second form is
\begin{alignat}{4}
  B_x &=  &C &\dfrac{k_x}{k_y} & \sinh(\kxx) \sinh(\kyy) \cos(\ksss) \CRNEG
  B_y &=  &C &                 & \cosh(\kxx) \cosh(\kyy) \cos(\ksss) \CRNEG
  B_s &= -&C &\dfrac{k_s}{k_y} & \cosh(\kxx) \sinh(\kyy) \sin(\ksss) \label{f2} \\
  & \makebox[1pt][l]{with $k_y^2 = k_s^2 - k_x^2$ ,} &&&  \nonumber
\end{alignat}
The third form is
\begin{alignat}{4}
  B_x &=  &C &\dfrac{k_x}{k_y} & \sinh(\kxx) \sin(\kyy) \cos(\ksss) \CRNEG
  B_y &=  &C &                 & \cosh(\kxx) \cos(\kyy) \cos(\ksss) \CRNEG
  B_s &= -&C &\dfrac{k_s}{k_y} & \cosh(\kxx) \sin(\kyy) \sin(\ksss) \label{f3} \\
  & \makebox[1pt][l]{with $k_y^2 = k_x^2 - k_s^2$ .} &&& \nonumber
\end{alignat}
The relationship between $k_x$, $k_y$, and $k_s$ ensures that
Maxwell's equations are satisfied.

There is no \vn{bmad_standard} tracking for a \vn{map_type}
\vn{wiggler}. The following describes the \vn{symp_lie_bmad} type tracking.
The Hamiltonian for the wiggler is 
\Begineq
  H = H_x + H_y + H_z
\Endeq
where
\begin{align}
  H_x &= \frac{p_x^2}{2 (1 + \delta)} \CRNO
  H_y &= \frac{(p_y - A_y)^2}{2 (1 + \delta)} \\
  H_s &= - A_s \nonumber
\end{align}
And the vector potential $\bfA$ is obtained from $\bfB$ via
\begin{align}
  A_x(x,y,s) &= 0 \CRNO
  A_y(x,y,s) &=  \int_0^x d\tilde x \, B_z(\tilde x, y, s) 
  \label{a0a0a0} \\
  A_s(x,y,s) &= -\int_0^x d\tilde x \, B_y(\tilde x, y, s) \nonumber
\end{align}

For tracking, the wiggler is broken up into a number of slices set by
the element's \vn{ds_step} attribute. For each slice, the tracking
uses a quadratic symplectic integrator $I$:
\Begineq
  I = T_{s/2} \; I_{x/2} \; I_{y/2} \; I_s \; I_{y/2} \; I_{x/2} \; T_{s/2}
\Endeq
$T_{s/2}$ is just a translation of the $s$ variable:
\Begineq
  s \rightarrow s + \frac{ds}{2}
\Endeq
And the other integrator components are
\begin{align}
  I_{x/2} &= \exp \left( : -\frac{ds}{2} H_x : \right) \CRNO
  I_{y/2} &= \exp \left( : -\frac{ds}{2} H_y : \right) \\
  I_{s}   &= \exp \left( : -ds \, H_s : \right) \nonumber
\end{align}
The evaluation if $I_{y/2}$ is tracky since it involves both transverse
position and momentum variables. The trick is to split the integration into three parts
\begin{align}
  I_{y/2} &= \exp \left( : -\frac{ds}{2} \frac{(p_y - A_y)^2}{2 (1 + \delta)} : \right) \CRNO
  &= \exp \left( : -\int A_y \, dy : \right) \,
     \exp \left( : -\frac{ds}{2} \frac{p_y^2}{2 (1 + \delta)} : \right) \,
     \exp \left( : \int A_y \, dy : \right)
  \label{ids2}
\end{align}
Given the form of the of the magnetic field of \Eqs{f1}, \eq{f2}, and
\eq{f3}, the intgrals in \Eq{a0a0a0} and \eq{ids2} are easily calculated.

%-----------------------------------------
\section{Periodic\_Type Wiggler Element Tracking}
\label{s:wiggler.periodic.std}
\index{wiggler}

The horizontal motion looks like a drift with a superimposed
sinusoidal oscillation. It is assumed that there is an integer number
of periods in the oscillation so that the exit horizontal coordinates
can be calculated from the initial coordinates using the equations for
a drift. The vertical motion is a quadratic superimposed with a
octupole. Vertical motion is calculated using a kick-drift-kick model.

\vn{Periodic type} wigglers use a simplified model where the magnetic
field components are
\begin{alignat}{1}
  B_y &= \hphantom{-} B_{\max} \, \cosh(k_s \, y) \, \cos(ksss) \CRNO
  B_s &= -B_{\max} \, \sinh(k_s \, y) \, \sin(ksss) 
  \label{bbkyks}
\end{alignat}
where $B_{\max}$ is the maximum field on the centerline and $k$ is
given in terms of the pole length (\vn{l_pole}) by
\Begineq
  k_s = \frac{\pi}{l_{\mbox{pole}}}
\Endeq
This type of wiggler has infinitely wide poles. With
\vn{bmad_standard} tracking and transfer matrix calculations the
vertical focusing is assumed small so averaged over a period the
horizontal motion looks like a drift and the vertical motion is
modeled as a combination focusing quadrupole and focusing octupole
giving a kick\cite{b:corbett}
\Begineq
  \frac{dp_y}{ds} = k_1 \left( y + \frac{2}{3} \, k_s^2 \, y^3 \right)
\Endeq
where
\begin{alignat}{1}
  k_1 &= \frac{-1}{2} \, \left( \frac{e \, B_{\max}}{P_0 \, (1 + p_s)} \right)^2 
\end{alignat}
with $k_1$ being the linear focusing constant. For radiation
calculations the true horizontal trajectory with $y = 0$ is needed
\Begineq
  x = \frac{\sqrt{2 \, |k_1|}}{k_s^2} \, \cos (k_s \, s)
\Endeq

With \vn{periodic type} wigglers and \vn{bmad_standard} tracking, the
phase $\phi_s$ in \Eqs{bbkyks} is irrelevant. When the tracking
involves Taylor maps and symplectic integration, the phase is
important. Here the phase is chosen so that $B_y$ is symmetric about
the center of the wiggler
\Begineq
  \phi_s = \frac{-k_y \, L}{2}
\Endeq
With this choice, a particle that enters the wiggler on-axis will
leave the wiggler on-axis provided there is an even number of poles.
