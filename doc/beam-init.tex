\chapter{Beam Initialization}
\label{c:beam.init}

Some \bmad based programs track beams of particles instead of tracking individual particles
one-by-one. This can be useful for several reasons. For example, tracking beams is useful when
inter-bunch or intra-bunch effects are to be simulated. Also tracking beams can simplify the
bookkeeping a program needs to do to calculate such quantities such as the bunch size.

A \bmad based program has two standard ways to specify the initial distribution of a beam. One is
using a \vn{beam_init_struct} \vn{structure} (\sref{s:struct}) which holds parameters (for
example, the beam emittances) from which a distribution of particles can be constructed. The
\vn{beam_init_struct} structure is explained in Section~\sref{s:beam.init.struct}. The other way is to
specify the inital beam distribution via a file that has the individual particle positions. This is
covered in Section~\sref{s:beam.init.file}.

%-----------------------------------------------------------------
\section{Beam_Init_Struct Structure}
\label{s:beam.init.struct}
\index{beam initialization parameters|hyperbf}

\index{beam_init_struct|hyperbf}
The \vn{beam_init_struct} \vn{structure} (\sref{s:struct}) holds parameters which are used to
initialize a beam. The parameters of this structure are:
\begin{example}
  type beam_init_struct
    character(200) :: file_name = ''       ! Initialization file name.
    character distribution_type(3)         ! "ELLIPSE", "KV", "GRID", "" (default).
    type (ellipse_beam_init_struct) ellipse(3) ! For ellipse beam distribution
    type (kv_beam_init_struct) KV              ! For KV beam distribution
    type (grid_beam_init_struct) grid(3)       ! For grid beam distribution
    !!! The following are for Random distributions
    character random_engine          ! "pseudo" (default) or "quasi". 
    character random_gauss_converter ! "exact" (default) or "quick". 
    real random_sigma_cutoff = -1    ! -1 => no cutoff used.
    real center_jitter(6) = 0.0      ! Bunch center rms jitter
    real emit_jitter(2)   = 0.0      ! %RMS a and b mode bunch emittance jitter
    real sig_z_jitter     = 0.0      ! bunch length RMS jitter 
    real sig_e_jitter     = 0.0      ! energy spread RMS jitter 
    integer n_particle = 0               ! Number of simulated particles per bunch.
    logical renorm_center = T            ! Renormalize centroid?
    logical renorm_sigma = T             ! Renormalize sigma?
    !!! The following are used  by all distribution types
    real(rp) spin(3)                 ! Spin (x, y, z)
    real a_norm_emit                 ! a-mode normalized emittance (= \(\gamma\,\epsilon\))
    real b_norm_emit                 ! b-mode normalized emittance (= \(\gamma\,\epsilon\))
    real a_emit                      ! a-mode emittance (= \(\gamma\,\epsilon\))
    real b_emit                      ! b-mode emittance (= \(\gamma\,\epsilon\))
    real dPz_dz = 0                  ! Correlation of Pz with long position.
    real center(6) = 0               ! Bench center offset.
    real dt_bunch                    ! Time between bunches.
    real sig_z                       ! Z sigma in m.
    real sig_e                       ! dE/E (pz) sigma.
    real bunch_charge                ! Charge in a bunch.
    integer n_bunch = 1                  ! Number of bunches.
    character species                    ! Species. Default is reference particle.
    logical full_6D_coupling_calc = F    ! Use 6x6 1-turn matrix to match distribution?  
    logical use_t_coords = F        ! If true, the distributions will be 
                                    !   calculated using time coordinates  
    logical use_z_as_t   = F        ! Only used if  use_t_coords = T
                                    !   If True,  particles will be distributed in t
                                    !   If False, particles will be distributed in s
  end type
\end{example}
Note: The \vn{z} coordinate value given to particles of a bunch is with respect to the
nominal center of the bunch. Therefore, if there are multiple bunches, and there is an RF
cavity whose frequency is not commensurate with the spacing between bunches, absolute time
tracking (\sref{s:rf.time}) must be used.


\begin{description}
\item[\%file_name] \Newline
\vn{%file_name} sets the name of the file to be read in containing the particle coordinates.  Input
from a file is triggered if not-blank. The format of the file is discused in
Section~\sref{s:beam.init.file}.
%
\item[\%a_emit, \%b_emit, \%a_norm_emit, \%b_norm_emit] \Newline
Normalized and unnormalized emittances. Either \vn{a_norm_emit} or \vn{a_emit}
may be set but not both. similarly, either \vn{b_norm_emit} or
\vn{b_emit} may be set but not both.
\item[\%bunch_charge] \Newline
%
\item[\%center(6)] \Newline
%
\item[\%center_jitter, \%emit_jitter, \%sig_z_jitter, \%sig_e_jitter] \Newline
These components can be used to provide a bunch-to-bunch 
random variation in the emittance and bunch center.
%
\item[\%distribution_type(3)] \Newline
The \vn{%distributeion_type(:)} array determines what algorithms are used to generate
the particle distribution for a bunch. \vn{%distributeion_type(1)} sets the distribution 
type for the $(x, p_x)$ 2D phase space, etc. 
Possibilities for \vn{%distributeion_type(:)} are:
\begin{example}
  "", or "RAN_GAUSS"  ! Random distribution (default).
  "ELLIPSE"           ! Ellipse distribution (\sref{s:ellipse.init})
  "KV"                ! Kapchinsky-Vladimirsky distribution (\sref{s:kv.init})
  "GRID"              ! Uniform distribution.
\end{example}
Since the Kapchinsky-Vladimirsky distribution is for a 4D
phase space, if the Kapchinsky-Vladimirsky distribution is used,
\vn{"KV"} must appear exactly twice in the \vn{%distributeion_type(:)}
array. 

Unlike all other distribution types, the \vn{GRID} distribution is independent of the
Twiss parameters at the point of generation.  For the non-\vn{GRID} distributions, the
distributions are adjusted if there is local $x$-$y$ coupling (\sref{s:coupling}). For
lattices with a closed geometry, if \vn{full_6D_coupling_calc} is set to \vn{True}, the
full 6-dimensional coupling matrix is used. If \vn{False}, which is the default, The
4-dimensional $\bfV$ matrix of \Eq{vgicc1} is used.

Note: The total number particles generated is the product of the individual
distributions. For example:
\begin{example}
  type (beam_init_struct) bi
  bi%distribution_type = ELLIPSE", "ELLIPSE", "GRID"
  bi%ellipse(1)%n_ellipse = 4
  bi%ellipse(1)%part_per_ellipse = 8
  bi%ellipse(2)%n_ellipse = 3
  bi%ellipse(2)%part_per_ellipse = 100
  bi%grid(3)%n_x = 20
  bi%grid(3)%n_px = 30
\end{example}
The total number of particles per bunch will be $32 \times 300 \times
600$. The exception is that when \vn{RAN_GAUSS} is mixed with other
distributions, the random distribution is overlaid with the other distributions
instead of multiplying. For example:
\begin{example}
  type (beam_init_struct) bi
  bi%distribution_type = RAN_GAUSS", "ELLIPSE", "GRID"
  bi%ellipse(2)%n_ellipse = 3
  bi%ellipse(2)%part_per_ellipse = 100
  bi%grid(3)%n_x = 20
  bi%grid(3)%n_px = 30
\end{example}
Here the number of particle is $300 \times 600$. Notice that when
\vn{RAN_GAUSS} is mixed with other distributions, the value of
\vn{beam_init%n_particle} is ignored.
%
\item[\%full_6D_coupling_calc] \Newline
If set \vn{True}, coupling between the transverse and longitudinal modes is taken into
account when calculating the beam distribution. 
The default \vn{False} decouples the transverse and longitudinal calculations.
%
\item[\%dPz_dz] \Newline
Correlation between $p_z$ and $z$ phase space coordinates. 
%
\item[\%dt_bunch] \Newline
Time between bunches
%
\item[\%ellipse(3)] \Newline
The \vn{%ellipse(:)} array sets the parameters for the 
\vn{ellipse} distribution (\sref{s:ellipse.init}). 
Each component of this array looks like
\begin{example}
  type ellipse_beam_init_struct
    integer part_per_ellipse  ! number of particles per ellipse.
    integer n_ellipse         ! number of ellipses.
    real(rp) sigma_cutoff     ! sigma cutoff of the representation.
  end type
\end{example}
%
\item[\%grid(3)] \Newline
The \vn{%grid} component of the \vn{beam_init_struct} sets the parameters 
for a uniformly spaced grid of particles. The components of \vn{%grid}
are:
\begin{example}
  type grid_beam_init_struct
    integer n_x        ! number of columns.
    integer n_px       ! number of rows.
    real(rp) x_min     ! Lower x limit.
    real(rp) x_max     ! Upper x limit.
    real(rp) px_min    ! Lower px limit.
    real(rp) px_max    ! Upper px limit.
  end type
\end{example}
%
\item[\%KV] \Newline
The \vn{%kv} component of the \vn{beam_init_struct} sets the parameters for the 
Kapchinsky-Vladimirsky distribution (\sref{s:kv.init}). The components of \vn{%KV}
are:
\begin{example}
  type kv_beam_init_struct
    integer part_per_phi(2)    ! number of particles per angle variable.
    integer n_I2               ! number of I2
    real(rp) A                 ! A = I1/e
  end type
\end{example}
%
\item[\%n_bunch] \Newline
The number of bunches in the beam is set by \vn{n_bunch}. Default is one.
%
\item[\%n_particle] \Newline
Number of particles generated when the \vn{%distribution_type} is \vn{"RAN_GAUSS"}.
Ignored for other distribution types.
%
\item[\%random_engine] \Newline
This component sets the algorithm to use in generating a uniform distribution
of random numbers in the interval [0, 1]. \vn{"pseudo"} is a pseudo random
number generator and "quasi" is a quasi random generator. "quasi random" is
a misnomer in that the distribution generated is fairly uniform.
%
\item[\%random_gauss_converter, \%random_sigma_cutoff] \Newline
To generate Gaussian random numbers, a conversion algorithm from the
flat distribution generated according to \vn{%random_engine} is
needed.  \vn{%random_gauss_converter} selects the algorithm. The
\vn{"exact"} conversion uses an exact conversion. The \vn{"quick"}
method is somewhat faster than the \vn{"exact"} method but not as accurate.
With either conversion method, if \vn{%random_sigma_cutoff} is set to a positive number,
this limits the maximum sigma generated.
%
\item[\%renorm_center, \%renorm_sigma] \Newline 
If set to True, these components will ensure that the actual beam center 
and sigmas will correspond to the input values. 
Otherwise, there will be fluctuations due to the finite number of 
particles generated.
%
\item[\%sig_e, \%sig_z] \Newline
Longitudinal sigmas. \vn{%sig_e} is the fractional energy spread dE/E. 
This, along with \vn{%dPz_dz} determine the longitudinal profile.
%
\item[\%species] \Newline
Name of the species tracked. If not set then the default tracking particle type is used.
%
\item[\%spin] \Newline
Particle spin in Cartesian $(x, y, z)$ coordinates.
%
\item[\%use_lattice_center] \Newline
If \vn{%use_lattice_center} is set to \vn{True} (default is \vn{False}), 
the center of the bunch is determined by the 
\vn{beam_start} (\sref{s:beam.start}) setting in the lattice file rather than the 
setting of \vn{%center} in the \vn{beam_init_struct}.
%
\item[\%use_t_coords, \%use_z_as_t] \Newline 
If \vn{use_t_coords} is true, then the
distributions are taken as describing particles in $t$-coordinates
(\sref{s:time.phase.space}).  Furthermore, if \vn{use_z_as_t} is true,
then the $z$ coordinates from the distribution will be taken as
describing the time coordinates. For example, particles may originate
at a cathode at the same $s$, but different times.  If false, then the
$z$ coordinate from the distribution describes particles at the same
time but different $s$ positions, and each particle gets
\vn{%location=inside\$}. In this case, the bunch will need to be
tracked with a tracking method that can handle inside particles, such
as \vn{time_runge_kutta}.  All particles are finally converted to
proper $s$-coordinate distributions for Bmad to use.
\end{description}


%-----------------------------------------------------------------
\section{File Based Beam Initialization}
\label{s:beam.init.file}
\index{beam initialization parameters|hyperbf}

A beam initialization file specifies the coordinates of all the particles in a beam. If a \bmad
based program uses a \vn{beam_init_struct} (\sref{s:beam.init.file}) for inputting initialization
parameters, the file name for file based beam initialization can be set using the \vn{%file_name}
component of the structure.

There are two formats for the beam initialization file: ASCII and binary. Currently, the binary file
format is undergoing revision and will not be discussed here. The ASCII file format is:
\begin{example}
  <ix_ele>         ! Lattice element index. This is ignored.
  <n_bunch>        ! Number of bunches.
  <n_particle>     ! Number of particles per bunch to use
  [bunch loop: ib = 1 to n_bunch]
    BEGIN_BUNCH    ! Marker to mark the beginning of a bunch specification block.
    <species_name> ! Species of particle
    <bunch_charge> ! Charge of bunch. 0 => Use <particle_charge>.
    <z_center>     ! z position at center of bunch.
    <t_center>     ! t position at center of bunch.
    [particle loop: Stop when END_BUNCH marker found]
      <x> <px> <y> <py> <z> <pz> <charge> <state> <spin_x> <spin_y> <spin_z> 
    [end particle loop]
    END_BUNCH      ! Marker to mark the end of the bunch specification block
  [end bunch loop]
\end{example}
Example:
\begin{example}
  0       ! ix_ele
  1       ! n_bunch
  25000   ! n_particle
  BEGIN_BUNCH
    POSITRON
    3.2E-9   ! bunch_charge
    0.0      ! z_center
    0.0      ! t_center
   -6.5E-3  9.6E-3 -1.9E-2  8.8E-3  2.2E-2 -2.4E-2  1.2E-13  1 1.0 0.0 0.0
    8.5E-3  5.5E-3  4.0E-2 -1.9E-2 -4.9E-3  2.1E-2  1.2E-13  1 1.0 0.0 0.0
    1.1E-2 -1.9E-2 -2.5E-2  1.0E-2 -1.8E-2 -7.1E-3  1.2E-13  1 1.0 0.0 0.0
   -3.4E-2 -2.7E-3 -4.1E-3  1.3E-2  1.3E-2  1.0E-2  1.2E-13  1 1.0 0.0 0.0
    6.8E-3 -4.5E-3  2.5E-3  1.4E-2 -2.3E-3  7.3E-2  1.2E-13  1 1.0 0.0 0.0
    1.2E-2 -9.8E-3  1.7E-3  6.4E-3 -9.8E-3 -7.2E-2  1.2E-13  1 1.0 0.0 0.0
    1.1E-2 -3.5E-4  1.2E-2  1.8E-2  5.4E-3  1.4E-2  1.2E-13  1 1.0 0.0 0.0
       ... etc. ...
  END_BUNCH
\end{example}

The first line of the file gives \vn{ix_ele}, the index of the lattice element at which the
distribution was created. This is ignored when the file is Read. The second line gives
\vn{<n_bunch>}, the number of bunches. The third line gives \vn{n_particle} the number of particles
in a bunch. After this, there are \vn{<n_bunch>} blocks of data, one for each bunch. Each one of
these blocks starts with a \vn{BEGIN_BUNCH} line to mark the beginning of the block and ends with a
\vn{END_BUNCH} marker line. In between, the first four lines give the \vn{species} name,
\vn{bunch_charge}, \vn{z_center}, and \vn{t_center} values. The \vn{species} name may be one of:
\begin{example}
  positron  ! default
  electron
  proton
  antiproton
  muon
  antimuon
  photon
\end{example}

The lines following the \vn{t_center} line specify particle coordinates. One line for each particle.
Only the first six numbers, which are the phase space coordinates, need to be specified for each
particle. If \vn{<particle_charge>} is not present, or is zero, it defaults to
\vn{bunch_charge/n_particle}. The \vn{<state>} parameter indicates whether a particle is alive or
dead. Values are
\begin{example}
  1     ! Alive
  2-7   ! Dead
\end{example}

The particle spin is specified by $x$, $y$ and $z$ components.

The number rows specifying particle coordinates may be more then \vn{<n_particle>}. In this case,
particles will be discarded so that the the beam has \vn{<n_particle>} particles. If
\vn{beam_init%n_particle}, if set in the \tao input file, this will override the setting of
\vn{<n_particle>} in the beam file.

Each particle has an associated \vn{<particle_charge>}. If \vn{<bunch_charge>} is set to a non-zero
value, the charge of all the particles will be scaled by a factor to make the bunch charge equal to
\vn{<bunch_charge>}. Additionally, if \vn{beam_init%bunch_charge} is set in the \tao input file,
this will override the setting of \vn{bunch_charge} in the beam file.

\index{change}\index{beam_start}\index{beam_init}
When the particle coordinates are read in from the \vn{beam0_file}, the centroid will be shifted by
the setting of \vn{beam_init%center}.
