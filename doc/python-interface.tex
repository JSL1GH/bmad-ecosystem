\chapter{Python/GUI Interface}
\index{python interface}
\label{c:python}

It is sometimes convenient to be able to run Tao via Python. For example, in an online control
system environment. \tao has scripts for doing this in one of two ways. One way is using the
\vn{ctypes} module. The other way is using the \vn{pexpect} module. A Web search will point to
documentation on \vn{ctypes} and \vn{pexpect}. The advantage of \vn{ctypes} is that it directly
accesses \tao code which makes communication between Python and \tao more robust. The disadvantage of
\vn{ctypes} is that it needs a shared-object version of the \vn{Tao} library. [See the Bmad web site
for information on building shared-object libraries.] The disadvantage of \vn{pexpect} is that it is
slower and it is possible for \vn{pexpect} to time out waiting for a response from \tao.

%--------------------------------------------------------------------------
\section{Python Interface Via Pexpect}

A python module, \vn{tao_pipe.py}, for interfacing \tao to \vn{Python}
is provided in the directory \vn{tao/python/tao_pexpect}.

The \vn{tao_pipe} module uses the \vn{pexpect} module. The
\vn{pexpect} module is a general purpose tool for interfacing
Python with programs like \tao. If \vn{pexpect} is not present
your system, it can be downloaded from
\vn{www.noah.org/wiki/pexpect}. 

Example:
\begin{example}
  >>> import tao_pipe                                       # import module
  >>> p = tao_pipe.tao_io("../bin/tao -lat my_lat.bmad")    # init session
  >>> p.cmd_in("show global")               # Command to Tao
  >>> print(p.output)                       # print the output from Tao
  >>> p.cmd("show global")                  # Like p.cmd_in() excepts prints the output too.
\end{example}

After each call to \vn{tao_io.cmd} and \vn{tao_io.cmd_in}, the
\vn{tao_io.output} variable is set to the multi-line output string
returned by \tao. To chop this string into lines, use the splitlines()
string method.

\index{python}
To get information from \tao into Python, the output from \tao,
contained in \vn{tao_io.output}, needs to be parsed. For long term
maintainability of python scripts, use the \vn{python} (\sref{s:python}) command 
as opposed to the \vn{show} command . See the \vn{python} command for more details.

%--------------------------------------------------------------------------
\section{Python Interface Via Ctypes}

A \vn{ctypes} based python module \vn{pytao.py} for interfacing \tao to \vn{Python} is provided in
the directory \vn{tao/python/pytao}.

A test driver script named \vn{pytao_example.py} is in the same directory. See the documentation in both of
these files for further information.

%--------------------------------------------------------------------------
\section{Tao Python command}
\label{s:python.python}

To get output from \tao that can be easily parsed by Python, use the \vn{python} command
(\sref{s:python}). The output of this command are semi-colon delimited lists. Besides being easily
parsed, the syntax of the output from the \vn{python} command will not change over time as the \tao
program is developed. [However, the content of the lists will change when \tao's internal structures
are modified in the course of \tao program development.]

Documentation on the \vn{python} command is contained in the code file itself at:
\begin{example}
  tao/code/tao_python_cmd.f90
\end{example}

Example: The command
\begin{example}
  python global
\end{example}
will produce as output:
\begin{example}
  lm_opt_deriv_reinit;REAL;T; -1.0000000000000000E+00
  de_lm_step_ratio;REAL;T;  1.0000000000000000E+00
  de_var_to_population_factor;REAL;T;  5.0000000000000000E+00
  lmdif_eps;REAL;T;  9.9999999600419720E-13
  svd_cutoff;REAL;T;  9.9999997473787516E-06
  unstable_penalty;REAL;T;  1.0000000474974513E-03
  ... etc ...
\end{example}

To simplify the process of parsing parameter lists, a Python class called \vn{tao_parameter} is
defined in the file:
\begin{example}
  tao/python/pytao/util/parameters.py
\end{example}
This class defines components
\begin{example}
    name:       The name of the parameter
    type:       "STR", "INT", "REAL", "LOGIC", "ENUM", etc...
    can_vary:   Either 'T', 'F', or 'I', indicating whether or not the
                user may change the value of the paramter. 'I' indicates
                that the parameter is to be ignored by a gui when displaying parameters.
    value:      The value held in the parameter, should be of the
                appropriate type for the specified param_type
                (or 'T'/'F' for LOGIC)
\end{example}
Further documentation on this class is in this file. The following Python program will parse a parameter
list data file and crate a dictionary holding the information
\begin{example}
  from pytao.util.parameters import *
  df = open('spin.dat', 'r')
  param_dict = tao_parameter_dict(df.readlines())
  print (str(param_dict))
\end{example}
Note: The \bmad setup scripts will append the directory \vn{tao/python} to the \vn{PYTHONPATH}
environment variable.


%--------------------------------------------------------------------------
\section{Plotting Issues}
\label{s:gui.plot}

When using \tao with a \vn{GUI}, and when the \vn{GUI} is doing the plotting, the \vn{-noplot} and
\vn{-external_plotting} options (\sref{s:command.line}) should be used when starting \tao. The
\vn{-noplot} option (which sets \vn{global%plot_on}) prevents \tao from opening a plotting
window. Note: Both of these options can also be set, after startup, with the \vn{set global} command
and the setting of both can be viewd using the \vn{show global} command.

With \vn{-external_plotting} set, the \vn{place} command will save its input to a buffer which then
can be read out by the \vn{python place_buffer} command. The reason for this is that with external
plotting, it is the external scripts that should handle the placement of plots in regions and it would
be potentially distruptive if a user tired to bypass this (which could inadventantly happen when running
command files).

Normally when \tao is not displaying the plot page when the \vn{-noplot} option is used, \tao will,
to save time, not calculate the points needed for plotting curves. The exception is if \vn{-external_plotting}
is turned on. In this case 

To make plot references unambiguous, plot can be referred to by their index number. The plot index
number can be viewed using the \vn{python plot_list} command. Template plots can be referenced using
the syntax ``\vn{@Tnnn}'' where \vn{nnn} is the index number. For example, \vn{@T3} referrers to the
template plot with index 3. Similarly, the displayed plots (plots that are associated with plot
regions) can be referred to using the syntax ``\vn{@Rnnn}''. This is useful when doing external
plotting.
