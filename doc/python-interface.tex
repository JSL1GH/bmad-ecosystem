\chapter{Python/Tao Interface}
\index{python interface}
\label{c:python}

It is sometimes convenient to interface \tao to an external program.\footnote
  { 
Commonly the external program is \vn{Python}. 
  }
For example, to analyze Tao generated data in an external program. Another application is to
interface \tao to an online control system environment. In this case, \tao can be used for such
things as flattening orbits.

To aid in interfacing, \tao has the \vn{python} command (\sref{s:python.python}).\footnote
  {
Dispite its name, \tao's \vn{python} command can be used with any external program. 
  }
The \vn{python} command defaines a standardized syntax with which to communicate with \tao.

Another aid is the \vn{PyTao} package which is an interface layer to be used between \tao and
\vn{Python}. See \sref{s:pytao} for more details.

%--------------------------------------------------------------------------
\section{PyTao Interface}
\label{s:pytao}

The \vn{PyTao} package is an interface layer to be used between \tao and
\vn{Python}. \vn{PyTao} is hosted on \vn{GitHub}
(independent of \bmad distributions) at:
\vspace{-2ex}
\begin{itemize}
  \item[] \url{https://bmad-sim.github.io/pytao}
\end{itemize}
\vspace{-2ex}
Documentation for setup and using PyTao is at:
\begin{example}
  bmad-sim.github.io/pytao/
\end{example}
See the \vn{PyTao} documentation for installation instructions, examples, etc. In this chapter, some
simple examples will be given.

There are two ways to interface with Python/PyTao. One way is using the Python \vn{ctypes}
library. The other way is using the \vn{pexpect} module. A Web search will point to documentation on
\vn{ctypes} and \vn{pexpect}. 

\vn{ctypes} is a foreign function library for Python which can be used to link to a \tao shared
library.  The \vn{pexpect} module is a general purpose tool for interfacing Python with programs
like \tao. If \vn{pexpect} is not present your system, it can be downloaded from
\vn{www.noah.org/wiki/pexpect}.

The advantage of \vn{ctypes} is that it directly accesses \tao code
which makes communication between Python and \tao more robust. The disadvantage of \vn{ctypes} is
that it needs a shared-object version of the \vn{Tao} library. [See the Bmad web site for
information on building shared-object libraries.] The disadvantage of \vn{pexpect} is that it is
slower and it is possible for \vn{pexpect} to time out waiting for a response from \tao.

%--------------------------------------------------------------------------
\subsection{Python/PyTao Via Pexpect}

For communicaiton via \vn{pexpect} (\sref{s:pytao}), the python module \vn{tao_pipe.py}, is
provided by \vn{PyTao} in the directory \vn{pytao/tao_pexpect}.

Example Python session:
\begin{example}
  >>> from pytao.tao_pexpect import tao_pipe  # import module
  >>> p = tao_pipe.tao_io("-lat my_lat.bmad") # init session
  >>> out = p.cmd_in("show global")           # Command to Tao
  >>> print(out)                              # print the output from Tao
  >>> p.cmd("show global")                    # Like p.cmd_in() excepts prints the output too.
\end{example}

%--------------------------------------------------------------------------
\subsection{Python/PyTao Interface Via Ctypes}

A \vn{ctypes} based python module \vn{pytao.py} for interfacing \tao to \vn{Python} is provided by
\vn{PyTao} (\sref{s:pytao}) in the directory \vn{pytao/tao_pexpect}.

A test driver script named \vn{pytao_example.py} is in the same directory. See the documentation in
both of these files for further information.

%--------------------------------------------------------------------------
\section{Tao's Python Command}
\label{s:python.python}

\tao's \vn{python} (\sref{s:python}) command was developed to:
%
\begin{itemize}
\item 
Standardize output of information (data, parameters, etc.) from \tao to simplify the task of
interfacing \tao to external programs especially scripting languages like \vn{Python}.
% 
\item 
Act as an intermediate layer for the control of \tao by such things as machine online control
programs or the planned graphical user interface for \tao.
\end{itemize}
Dispite its name, \tao's \vn{python} command can be used with any external program. However, the
\vn{PyTao} package has code that uses the \vn{python} command to ease integration with \vn{Python}.

Using the \vn{python} command to control \tao will not be covered here. The interested reader is
invited to read the sections of this manual on the coding of \tao and look at the \tao code itself
(which is heavily documented).

Using the \vn{python} command is far superior to using the \vn{show} command when interfacing to an
external program. For one, the \vn{python} command is formatted for ease of parsing. Another reason
to use the \vn{python} command is that, as \tao is developed over time, the output format of the
\vn{python} command is much more stable than output from the \vn{show} command.\footnote
  {
The output of the \vn{python} command will change when \tao's or \bmad's internal structures are
modified. This is in contrast to the \vn{show} command whose output is formated to be human readible
and whose output format may change on a whim.
  }
Thus the risk of User developed interface code breaking is much reduced by using the \vn{python} command.

The general form of the \vn{python} command is:
\begin{example}
  python <subcommand> <arguments>
\end{example}
The \vn{python} command has a number of \vn{subcommands} (over 100) that are listed in
\Sref{s:python.sub}. The sub-commands can be divided into two categories. One category are the
``\vn{action}'' subcommands which allow the user to control \tao (for example, creating variables
and data for use in an optimization). The other category are the ``\vn{output}'' subcommands which
output information from \tao.

The output of the \vn{python} command are semi-colon delimited lists. Example: With the
\vn{python global} the output looks like:
\begin{example}
  lm_opt_deriv_reinit;REAL;T; -1.0000000000000000E+00
  de_lm_step_ratio;REAL;T;  1.0000000000000000E+00
  de_var_to_population_factor;REAL;T;  5.0000000000000000E+00
  unstable_penalty;REAL;T;  1.0000000474974513E-03
  n_opti_cycles;INT;T;20
  track_type;ENUM;T;single
  derivative_uses_design;LOGIC;T;F
  ... etc ...
\end{example}

Most \vn{output} subcommands use ``\vn{parameter list form}'' format where each line has four fields
separated by semicolons:
\begin{example}
  {name};{type};{variable};{value(s)}
\end{example}
The fields are:
\begin{example}
    name:       The name of the parameter

    type:       The type of the parameter:
        INT           Integer number
        REAL          Real number
        COMPLEX       Complex number. A complex number is output as Re;Im
        REAL_ARR      Real array
        LOGIC         Logical: "T" or "F".
        INUM          Integer whose allowed values can be obtained 
                        using the "python inum" command.
        ENUM          String whose allowed values can be obtained 
                        using the "python enum" command.
        FILE          Name of file.
        CRYSTAL       Crystal name string. EG: "Si(111)"
        DAT_TYPE      Data type string. EG: "orbit.x"
        DAT_TYPE_Z    Data type string if plot%x_axis_type = 'data'. 
                        Otherwise is a data_type_z enum.
        SPECIES       Species name string. EG: "H2SO4++"
        ELE_PARAM     Lattice element parameter string. EG "K1"
        STR           String that does not fall into one of the above string categories.
        STRUCT        Structure. In this case {component_value(s)} is of the form:
                        {name1};{type1};{value1};{name2};{type2};{value2};...
        COMPONENT     For curve component parameters.

    can_vary:   Either 'T', 'F', or 'I', indicating whether or not the
                user may change the value of the parameter. 'I' indicates
                that the parameter is to be ignored by a GUI when displaying parameters.

    value(s):   The value or values of the the parameter. If a parameter has multiple
                values (EG an array), the values will be separated by semicolons.
\end{example}

%--------------------------------------------------------------------------
\section{Plotting Issues}
\label{s:gui.plot}

When using \tao with a \vn{GUI}, and when the \vn{GUI} is doing the plotting, the \vn{-noplot} and
\vn{-external_plotting} options (\sref{s:command.line}) should be used when starting \tao. The
\vn{-noplot} option (which sets \vn{global%plot_on}) prevents \tao from opening a plotting
window. Note: Both of these options can also be set, after startup, with the \vn{set global} command
and the setting of both can be viewed using the \vn{show global} command.

With \vn{-external_plotting} set, the external code should handle how plots are assigned to plot
regions and it would be potentially disruptive if a user tired to place plots (which could
inadvertently happen when running command files). To avoid this, with \vn{-external_plotting} set,
the \vn{place} command will not do any placement but rather save the \vn{place} arguments (which is
the name of a template plot and a region name) to a buffer which then can be read out by the
external code using the \vn{python place_buffer} command. The external code may then decide how to
proceed. The external code is able to bypass the buffering and perform placements by using
\vn{place} with the \vn{-no_buffer} switch (\sref{s:place}). Notice: \tao never processes place
command information put in the buffer. It is up to the external code to decide on a course of action.

Normally when \tao is not displaying the plot page when the \vn{-noplot} option is used, \tao will,
to save time, not calculate the points needed for plotting curves. The exception is if
\vn{-external_plotting} is turned on. In this case, to make plot references unambiguous, plot can be
referred to by their index number. The plot index number can be viewed using the \vn{python
plot_list} command. Template plots can be referenced using the syntax ``\vn{@Tnnn}'' where \vn{nnn}
is the index number. For example, \vn{@T3} referrers to the template plot with index 3. Similarly,
the displayed plots (plots that are associated with plot regions) can be referred to using the
syntax ``\vn{@Rnnn}''.

%--------------------------------------------------------------------------
\section{Python subcommands}
\label{s:python.sub}

The \vn{python} command has the following subcommands:

% WARNING: this is automatically generated. DO NOT EDIT.
\begin{description}
\item[beam] \Newline\begin{example}
Output beam parameters that are not in the beam_init structure.
Command syntax:
  python beam {ix_universe}
where
  {ix_universe} is a universe index.
To set beam_init parameters use the "set beam" command


Parameters
----------
ix_universe

   
Returns
-------
?? 

\end{example}
\item[beam_init] \Newline\begin{example}
Output beam_init parameters.
Command syntax:
  python beam_init {ix_universe}
where
  {ix_universe} is a universe index.
To set beam_init parameters use the "set beam_init" command


Parameters
----------
ix_universe

   
Returns
-------
?? 

\end{example}
\item[bmad_com] \Newline\begin{example}
Bmad_com structure components
Command syntax:
  python bmad_com


\end{example}
\item[branch1] \Newline\begin{example}
Lattice element list.
Command syntax:
  python branch1 {ix_universe}@{ix_branch}
where
  {ix_universe} is a universe index
  {ix_branch} is a lattice branch index


Parameters
----------
ix_universe
ix_branch

   
Returns
-------
?? 

\end{example}
\item[bunch1] \Newline\begin{example}
Bunch parameters at the exit end of a given lattice element.
Command syntax:
  python bunch1 {ele_id}|{which} {ix_bunch} {coordinate}
 where {ele_id} is an element name or index  and {which} is one of:
  model
  base
  design

Optional {coordinate} is one of:
  x, px, y, py, z, pz, 's', 't', 'charge', 'p0c'
and will return an array.


Parameters
----------
ele_id
which
ix_bunch
coordinate

   
Returns
-------
?? 

\end{example}
\item[building_wall_list] \Newline\begin{example}
List of building wall sections or section points
Command syntax:
  python building_wall_list {ix_section}
If {ix_section} is not present then a list of building wall sections is given.
If {ix_section} is present then a list of section points is given


Parameters
----------
ix_section

   
Returns
-------
?? 

\end{example}
\item[building_wall_graph] \Newline\begin{example}
(x, y) points for drawing the building wall for a particular graph.
The graph defines the coordinate system for the (x, y) points.
Command syntax:
  python building_wall_graph {graph}


Parameters
----------
graph

   
Returns
-------
?? 

\end{example}
\item[building_wall_point] \Newline\begin{example}
add or delete a building wall point
Command syntax:
  python building_wall_point {ix_section}^^{ix_point}^^{z}^^{x}^^{radius}^^{z_center}^^{x_center}
Where:
  {ix_section}    -- Section index.
  {ix_point}      -- Point index. Points of higher indexes will be moved up 
                       if adding a point and down if deleting.
  {z}, etc...     -- See tao_building_wall_point_struct components.
                  -- If {z} is set to "delete" then delete the point.


Parameters
----------
ix_section
ix_point
z
x
radius
z_center
x_center

   
Returns
-------
?? 

\end{example}
\item[building_wall_section] \Newline\begin{example}
add or delete a building wall section
Command syntax:
  python building_wall_section {ix_section}^^{sec_name}^^{sec_constraint}
Where:
  {ix_section}      -- Section index. Sections with higher indexes will be
                         moved up if adding a section and down if deleting.
  {sec_name}        -- Section name.
  {sec_constraint}  -- Must be one of:
      delete     -- Delete section. Anything else will add the section.
      none
      left_side
      right_side


Parameters
----------
ix_section
sec_name
sec_constraint

   
Returns
-------
?? 

\end{example}
\item[constraints] \Newline\begin{example}
Optimization data and variables that contribute to the merit function.
Command syntax:
  python constraints {who}
{who} is one of:
  data
  var
Data constraints output is:
  data name
  constraint type
  evaluation element name
  start element name
  end/reference element name
  measured value
  ref value (only relavent if global%opt_with_ref = T)
  model value
  base value (only relavent if global%opt_with_base = T)
  weight
  merit value
  location where merit is evaluated (if there is a range)
Var constraints output is:
  var name
  Associated varible attribute
  meas value
  ref value (only relavent if global%opt_with_ref = T)
  model value
  base value (only relavent if global%opt_with_base = T)
  weight
  merit value
  dmerit/dvar


Parameters
----------
who

   
Returns
-------
?? 

\end{example}
\item[da_aperture] \Newline\begin{example}
Dynamic aperture data
Command syntax:
  python da_aperture {ix_uni}


Parameters
----------
ix_uni

   
Returns
-------
?? 

\end{example}
\item[da_params] \Newline\begin{example}
Dynamic aperture input parameters
Command syntax:
  python da_params {ix_uni}


Parameters
----------
ix_uni

   
Returns
-------
?? 

\end{example}
\item[data] \Newline\begin{example}
Individual datum info.
Command syntax:
  python data {ix_universe}@{d2_name}.{d1_datum}[{dat_index}]
Use the "python data-d1" command to get detailed info on a specific d1 array.
Output syntax is parameter list form. See documentation at the beginning of this file.
Example:
  python data_d1 1@orbit.x[10]


Parameters
----------
ix_universe
d2_name
d1_datum
dat_index

   
Returns
-------
?? 

\end{example}
\item[data_d2_create] \Newline\begin{example}
Create a d2 data structure along with associated d1 and data arrays.

Command syntax:
  python data_d2_create {d2_name}^^{n_d1_data}^^{d_data_arrays_name_min_max}
{d2_name} should be of the form {ix_uni}@{d2_datum_name}
{n_d1_data} is the number of associated d1 data structures.
{d_data_arrays_name_min_max} has the form
  {name1}^^{lower_bound1}^^{upper_bound1}^^....^^{nameN}^^{lower_boundN}^^{upper_boundN}
where {name} is the data array name and {lower_bound} and {upper_bound} are the bounds 
of the array.

Example:
  python data_d2_create 2@orbit^^2^^x^^0^^45^^y^^1^^47
This example creates a d2 data structure called "orbit" with 
two d1 structures called "x" and "y".

The "x" d1 structure has an associated data array with indexes in the range [0, 45].
The "y" d1 structure has an associated data arrray with indexes in the range [1, 47].

Use the "set data" command to set a created datum parameters.
Note: When setting multiple data parameters, 
      temporarily toggle s%global%lattice_calc_on to False
  ("set global lattice_calc_on = F") to prevent Tao trying to 
      evaluate the partially created datum and generating unwanted error messages.


Parameters
----------
d2_name
n_d1_data
d_data_arrays_name_min_max

   
Returns
-------
?? 

\end{example}
\item[data_d2_destroy] \Newline\begin{example}
Destroy a d2 data structure along with associated d1 and data arrays.
Command syntax:
  python data_d2_destroy {d2_datum}
{d2_datum} should be of the form
  {ix_uni}@{d2_datum_name}


Parameters
----------
d2_datum

   
Returns
-------
?? 

\end{example}
\item[data_d2] \Newline\begin{example}
Information on a d2_datum.
Command syntax:
  python data_d2 {d2_datum}
{d2_datum} should be of the form
  {ix_uni}@{d2_datum_name}


Parameters
----------
d2_datum

   
Returns
-------
?? 

\end{example}
\item[data_d_array] \Newline\begin{example}
List of datums for a given data_d1.
Command syntax:
  python data_d_array {d1_datum}
{d1_datum} should be for the form
  {ix_uni}@{d2_datum_name}.{d1_datum_name}
Example:
  python data_d_array 1@orbit.x


Parameters
----------
d1_datum

   
Returns
-------
?? 

\end{example}
\item[data_d1_array] \Newline\begin{example}
List of d1 arrays for a given data_d2.
Command syntax:
  python data_d1_array {d2_datum}
{d2_datum} should be of the form
  {ix_uni}@{d2_datum_name}


Parameters
----------
d2_datum

   
Returns
-------
?? 

\end{example}
\item[data_parameter] \Newline\begin{example}
Given an array of datums, generate an array of values for a particular datum parameter.
Command syntax:
  python data_parameter {data_array} {parameter}
{parameter} may be any tao_data_struct parameter.
Example:
  python data_parameter orbit.x model_value

Parameters
----------
ix_universe

   
Returns
-------
?? 

\end{example}
\item[data_d2_array] \Newline\begin{example}
Data d2 info for a given universe.
Command syntax:
  python data_d2_array {ix_universe}
Example:
  python data_d2_array 1


Parameters
----------
ix_universe

   
Returns
-------
?? 

\end{example}
\item[data_set_design_value] \Newline\begin{example}
Set the design (and base & model) values for all datums.
Command syntax:
  python data_set_design_value
Example:
  python data_set_design_value

Note: Use the "data_d2_create" and "datum_create" first to create datums.


\end{example}
\item[datum_create] \Newline\begin{example}
Create a datum.
Command syntax:
  python datum_create {datum_name}^^{data_type}^^{ele_ref_name}^^{ele_start_name}^^
                      {ele_name}^^{merit_type}^^{meas}^^{good_meas}^^{ref}^^
                      {good_ref}^^{weight}^^{good_user}^^{data_source}^^
                      {eval_point}^^{s_offset}^^{ix_bunch}^^{invalid_value}^^
                      {spin_axis%n0(1)}^^{spin_axis%n0(2)}^^{spin_axis%n0(3)}^^
                      {spin_axis%l(1)}^^{spin_axis%l(2)}^^{spin_axis%l(3)}

Note: The 3 values for spin_axis%n0, as a group, are optional and the 3 values 
      for spin_axis%l are, as a group, optional.
Note: Use the "data_d2_create" first to create a d2 structure with associated d1 arrays.
Note: After creating all your datums, use the "data_set_design_value" routine to 
      set the design (and model) values.


Parameters
----------
datum_name
data_type
ele_ref_name
ele_start_name
ele_name
merit_type
meas
good_meas
ref
good_ref
weight
good_user
data_source
eval_point
s_offset
ix_bunch
invalid_value
spin_axis%n0(1)
spin_axis%n0(2)
spin_axis%n0(3)
spin_axis%l(1)
spin_axis%l(2)
spin_axis%l(3)

   
Returns
-------
?? 

\end{example}
\item[datum_has_ele] \Newline\begin{example}
Does datum type have an associated lattice element?
Command syntax:
  python datum_has_ele {datum_type}


Parameters
----------
datum_type

   
Returns
-------
?? 

\end{example}
\item[derivative] \Newline\begin{example}
Optimization derivatives
Command syntax:
  python derivative
Note: To save time, this command will not recalculate derivatives. 
Use the "derivative" command beforehand to recalcuate if needed.


\end{example}
\item[ele:head] \Newline\begin{example}
"Head" Element attributes
Command syntax:
  python ele:head {ele_id}|{which}
where {ele_id} is an element name or index and {which} is one of
  model
  base
  design
Example:
  python ele:head 3@1>>7|model
This gives element number 7 in branch 1 of universe 3.


Parameters
----------
ele_id
which

   
Returns
-------
?? 

\end{example}
\item[ele:methods] \Newline\begin{example}
Element methods
Command syntax:
  python ele:methods {ele_id}|{which}
where {ele_id} is an element name or index and {which} is one of
  model
  base
  design
Example:
  python ele:methods 3@1>>7|model
This gives element number 7 in branch 1 of universe 3.


Parameters
----------
ele_id
which

   
Returns
-------
?? 

\end{example}
\item[ele:gen_attribs] \Newline\begin{example}
Element general attributes
Command syntax:
  python ele:gen_attribs {ele_id}|{which}
where {ele_id} is an element name or index and {which} is one of
  model
  base
  design
Example:
  python ele:gen_attribs 3@1>>7|model
This gives element number 7 in branch 1 of universe 3.


Parameters
----------
ele_id
which

   
Returns
-------
?? 

\end{example}
\item[ele:multipoles] \Newline\begin{example}
Element multipoles
Command syntax:
  python ele:multipoles {ele_id}|{which}
where {ele_id} is an element name or index and {which} is one of
  model
  base
  design
Example:
  python ele:multipoles 3@1>>7|model
This gives element number 7 in branch 1 of universe 3.


Parameters
----------
ele_id
which

   
Returns
-------
?? 

\end{example}
\item[ele:ac_kicker] \Newline\begin{example}
Element ac_kicker
Command syntax:
  python ele:ac_kicker {ele_id}|{which}
where {ele_id} is an element name or index and {which} is one of
  model
  base
  design
Example:
  python ele:ac_kicker 3@1>>7|model
This gives element number 7 in branch 1 of universe 3.


Parameters
----------
ele_id
which

   
Returns
-------
?? 

\end{example}
\item[ele:cartesian_map] \Newline\begin{example}
Element cartesian_map
Command syntax:
  python ele:cartesian_map {ele_id}|{which} {index} {who}
where {ele_id} is an element name or index and {which} is one of
  model
  base
  design
{index} is the index number in the ele%cartesian_map(:) array
{who} is one of:
  base
  terms
Example:
  python ele:cartesian_map 3@1>>7|model 2 base
This gives element number 7 in branch 1 of universe 3.


Parameters
----------
ele_id
which
index
who

   
Returns
-------
?? 

\end{example}
\item[ele:chamber_wall] \Newline\begin{example}
Element beam chamber wall
Command syntax:
  python ele:chamber_wall {ele_id}|{which} {index} {who}
where {ele_id} is an element name or index and {which} is one of
  model
  base
  design
{index} is index of the wall.
{who} is one of:
  x       ! Return min/max in horizontal plane
  y       ! Return min/max in vertical plane


Parameters
----------
ele_id
which
index
who

   
Returns
-------
?? 

\end{example}
\item[ele:cylindrical_map] \Newline\begin{example}
Element cylindrical_map
Command syntax:
  python ele:cylindrical_map {ele_id}|{which} {index} {who}
where {ele_id} is an element name or index and {which} is one of
  model
  base
  design
{index} is the index number in the ele%cylindrical_map(:) array
{who} is one of:
  base
  terms
Example:
  python ele:cylindrical_map 3@1>>7|model 2 base
This gives map #2 of element number 7 in branch 1 of universe 3.


Parameters
----------
ele_id
which
index
who

   
Returns
-------
?? 

\end{example}
\item[ele:taylor] \Newline\begin{example}
Element taylor
Command syntax:
  python ele:taylor {ele_id}|{which}
where {ele_id} is an element name or index and {which} is one of
  model
  base
  design
Example:
  python ele:taylor 3@1>>7|model
This gives element number 7 in branch 1 of universe 3.


Parameters
----------
ele_id
which

   
Returns
-------
?? 

\end{example}
\item[ele:spin_taylor] \Newline\begin{example}
Element spin_taylor
Command syntax:
  python ele:spin_taylor {ele_id}|{which}
where {ele_id} is an element name or index and {which} is one of
  model
  base
  design
Example:
  python ele:spin_taylor 3@1>>7|model
This gives element number 7 in branch 1 of universe 3.


Parameters
----------
ele_id
which

   
Returns
-------
?? 

\end{example}
\item[ele:wake] \Newline\begin{example}
Element wake
Command syntax:
  python ele:wake {ele_id}|{which} {who}
where {ele_id} is an element name or index and {which} is one of
  model
  base
  design
{Who} is one of
  base
  sr_long     sr_long_table
  sr_trans    sr_trans_table
  lr_mode_table
Example:
  python ele:wake 3@1>>7|model
This gives element number 7 in branch 1 of universe 3.


Parameters
----------
ele_id
which
who

   
Returns
-------
?? 

\end{example}
\item[ele:wall3d] \Newline\begin{example}
Element wall3d
Command syntax:
  python ele:wall3d {ele_id}|{which} {index} {who}
where {ele_id} is an element name or index and {which} is one of
  model
  base
  design
{index} is the index number in the ele%wall3d(:) array (size obtained from "ele:head").
{who} is one of:
  base
  table
Example:
  python ele:wall3d 3@1>>7|model 2 base
This gives element number 7 in branch 1 of universe 3.


Parameters
----------
ele_id
which
index
who

   
Returns
-------
?? 

\end{example}
\item[ele:twiss] \Newline\begin{example}
Element twiss
Command syntax:
  python ele:twiss {ele_id}|{which}
where {ele_id} is an element name or index and {which} is one of
  model
  base
  design
Example:
  python ele:twiss 3@1>>7|model
This gives element number 7 in branch 1 of universe 3.


Parameters
----------
ele_id
which

   
Returns
-------
?? 

\end{example}
\item[ele:control] \Newline\begin{example}
Element control
Command syntax:
  python ele:control {ele_id}|{which}
where {ele_id} is an element name or index and {which} is one of
  model
  base
  design
Example:
  python ele:control 3@1>>7|model
This gives element number 7 in branch 1 of universe 3.


Parameters
----------
ele_id
which

   
Returns
-------
?? 

\end{example}
\item[ele:orbit] \Newline\begin{example}
Element orbit
Command syntax:
  python ele:orbit {ele_id}|{which}
where {ele_id} is an element name or index and {which} is one of
  model
  base
  design
Example:
  python ele:orbit 3@1>>7|model
This gives element number 7 in branch 1 of universe 3.


Parameters
----------
ele_id
which

   
Returns
-------
?? 

\end{example}
\item[ele:mat6] \Newline\begin{example}
Element mat6
Command syntax:
  python ele:mat6 {ele_id}|{which} {who}
where {ele_id} is an element name or index and {which} is one of
  model
  base
  design
{who} is one of:
  mat6
  vec0
  err
Example:
  python ele:mat6 3@1>>7|model mat6
This gives element number 7 in branch 1 of universe 3.


Parameters
----------
ele_id
which
who

   
Returns
-------
?? 

\end{example}
\item[ele:taylor_field] \Newline\begin{example}
Element taylor_field
Command syntax:
  python ele:taylor_field {ele_id}|{which} {index} {who}
where {ele_id} is an element name or index and {which} is one of
  model
  base
  design
{index} is the index number in the ele%taylor_field(:) array
{who} is one of:
  base
  terms
Example:
  python ele:taylor_field 3@1>>7|model 2 base
This gives element number 7 in branch 1 of universe 3.


Parameters
----------
ele_id
which
index
who

   
Returns
-------
?? 

\end{example}
\item[ele:grid_field] \Newline\begin{example}
Element grid_field
Command syntax:
  python ele:grid_field {ele_id}|{which} {index} {who}
where {ele_id} is an element name or index and {which} is one of
  model, base, design
{index} is the index number in the ele%grid_field(:) array.
{who} is one of:
  base, points
Example:
  python ele:grid_field 3@1>>7|model 2 base
This gives grid #2 of element number 7 in branch 1 of universe 3.


Parameters
----------
ele_id
which
index
who

   
Returns
-------
?? 

\end{example}
\item[ele:floor] \Newline\begin{example}
Element floor coordinates. The output gives two lines. "Reference" is
without element misalignments and "Actual" is with misalignments.
Command syntax:
  python ele:floor {ele_id}|{which} {where}
where {ele_id} is an element name or index and {which} is one of
  model
  base
  design
{where} is an optional argument which, if present, is one of
  beginning  ! Upstream end
  center     ! Middle of element
  end        ! Downstream end (default)
Note: {where} ignored for photonic elements crystal, mirror, and multilayer_mirror.
Example:
  python ele:floor 3@1>>7|model
This gives element number 7 in branch 1 of universe 3.


Parameters
----------
ele_id
which
where

   
Returns
-------
?? 

\end{example}
\item[ele:photon] \Newline\begin{example}
Element photon
Command syntax:
  python ele:photon {ele_id}|{which} {who}
where {ele_id} is an element name or index and {which} is one of
  model
  base
  design
{who} is one of:
  base
  material
  surface
Example:
  python ele:photon 3@1>>7|model base
This gives element number 7 in branch 1 of universe 3.


Parameters
----------
ele_id
which
who

   
Returns
-------
?? 

\end{example}
\item[ele:lord_slave] \Newline\begin{example}
Lists the lord/slave tree of an element.
Command syntax:
  python ele:lord_slave {ele_id}|{which}
where {ele_id} is an element name or index and {which} is one of
  model
  base
  design
Example:
  python ele:lord_slave 3@1>>7|model
This gives lord and slave info on element number 7 in branch 1 of universe 3.
Note: The lord/slave info is independent of the setting of {which}.

The output is a number of lines, each line giving information on an element (element index, etc.).
Some lines begin with the word "Element". 
After each "Element" line, there are a number of lines (possibly zero) that begin with the word "Slave or "Lord".
These "Slave" and "Lord" lines are the slaves and lords of the "Element" element.


Parameters
----------
ele_id
which

   
Returns
-------
?? 

\end{example}
\item[ele:elec_multipoles] \Newline\begin{example}
Element electric multipoles
Command syntax:
  python ele:elec_multipoles {ele_id}|{which}
where {ele_id} is an element name or index and {which} is one of
  model
  base
  design
Example:
  python ele:elec_multipoles 3@1>>7|model
This gives element number 7 in branch 1 of universe 3.


Parameters
----------
ele_id
which

   
Returns
-------
?? 

\end{example}
\item[evaluate] \Newline\begin{example}
Evaluate an expression. The result may be a vector.
Command syntax:
  python evaluate {expression}
Example:
  python evaluate 2*data::cbar.11[1:10]|model


Parameters
----------
expression

   
Returns
-------
?? 

\end{example}
\item[em_field] \Newline\begin{example}
EM field at a given point generated by a given element.
Command syntax:
  python em_field {ele_id}|{which} {x}, {y}, {z}, {t_or_z}
where {which} is one of:
  model
  base
  design
Where:
  {x}, {y}  -- Transverse coords.
  {z}       -- Longitudainal coord with respect to entrance end of element.
  {t_or_z}  -- time or phase space z depending if lattice is setup for absolute time tracking.


Parameters
----------
ele_id
which
x
y
z
t_or_z

   
Returns
-------
?? 

\end{example}
\item[enum] \Newline\begin{example}
List of possible values for enumerated numbers.
Command syntax:
  python enum {enum_name}
Example:
  python enum tracking_method


Parameters
----------
enum_name

   
Returns
-------
?? 

\end{example}
\item[floor_plan] \Newline\begin{example}
Floor plan elements
Command syntax:
  python floor_plan {graph}


Parameters
----------
graph

   
Returns
-------
?? 

\end{example}
\item[floor_orbit] \Newline\begin{example}
(x, y) coordinates for drawing the particle orbit on a floor plan.
Command syntax:
  python floor_orbit {graph}


Parameters
----------
graph

   
Returns
-------
?? 

\end{example}
\item[global] \Newline\begin{example}
Global parameters
Command syntax:
  python global
Output syntax is parameter list form. See documentation at the beginning of this file.

Note: The follow is intentionally left out:
  optimizer_allow_user_abort
  quiet
  single_step
  prompt_color
  prompt_string


\end{example}
\item[help] \Newline\begin{example}
returns list of "help xxx" topics
Command syntax:
  python help


\end{example}
\item[inum] \Newline\begin{example}
INUM
Command syntax:
  python inum {who}


Parameters
----------
who

   
Returns
-------
?? 

\end{example}
\item[lat_calc_done] \Newline\begin{example}
Check if a lattice recalculation has been proformed since the last time
  "python lat_calc_done" was called.
Command syntax:
  python lat_calc_done


\end{example}
\item[lat_ele_list] \Newline\begin{example}
Lattice element list.
Command syntax:
  python lat_ele {branch_name}
{branch_name} should have the form:
  {ix_uni}@{ix_branch}


Parameters
----------
branch_name

   
Returns
-------
?? 

\end{example}
\item[lat_general] \Newline\begin{example}
Lattice general
Command syntax:
  python lat_general {ix_universe}

Output syntax:
  branch_index;branch_name;n_ele_track;n_ele_max


Parameters
----------
ix_universe

   
Returns
-------
?? 

\end{example}
\item[lat_list] \Newline\begin{example}
List of parameters at ends of lattice elements
Command syntax:
  python lat_list -no_slaves -track_only -index_order 
                  {ix_uni}@{ix_branch}>>{elements}|{which} {who}
where:
  -no_slaves is optional. If present, multipass_slave and super_slave elements
             will not be matched to.
  -track_only is optional. If present, lord elements will not be matched to.
  -index_order is optional. If present, order elements by element index instead 
               of the standard s-position.

  {which} is one of:
    model
    base
    design

  {who} is a comma deliminated list of:
    orbit.floor.x, orbit.floor.y, orbit.floor.z    ! Floor coords at particle orbit.
    orbit.spin.1, orbit.spin.2, orbit.spin.3,
    orbit.vec.1, orbit.vec.2, orbit.vec.3, orbit.vec.4, orbit.vec.5, orbit.vec.6,
    orbit.t, orbit.beta,
    orbit.state,     ! Note: state is an integer. alive$ = 1, anything else is lost.
    orbit.energy, orbit.pc,
    ele.name, ele.ix_ele, ele.ix_branch
    ele.a.beta, ele.a.alpha, ele.a.eta, ele.a.etap, ele.a.gamma, ele.a.phi,
    ele.b.beta, ele.b.alpha, ele.b.eta, ele.b.etap, ele.b.gamma, ele.b.phi,
    ele.x.eta, ele.x.etap,
    ele.y.eta, ele.y.etap,
    ele.s, ele.l
    ele.e_tot, ele.p0c
    ele.mat6, ele.vec0

  {elements} is a string to match element names to.
    Use "*" to match to all elements.

Note: vector layout of mat6(6,6) is: [mat6(1,:), mat6(2,:), ...mat6(6,:)]
Note: To output through the real array buffer, add the prefix "real:" to {who}.
Note: Only a single item permitted with real buffer out.

Examples:
  python lat_list -track 3@0>>Q*|base ele.s,orbit.vec.2
  python lat_list 3@0>>Q*|base real:ele.s    


\end{example}
\item[lat_param_units] \Newline\begin{example}
Units of a parameter associated with a lattice or lattice element.
Command syntax:
  python lat_param_units {param_name}


Parameters
----------
param_name

   
Returns
-------
?? 

\end{example}
\item[matrix] \Newline\begin{example}
Matrix value from the exit end of one element to the exit end of the other.
Command syntax:
  python matrix {ele1_id} {ele2_id}
where:
  {ele1_id} is the start element.
  {ele2_id} is the end element.
If {ele2_id} = {ele1_id}, the 1-turn transfer map is computed.
Note: {ele2_id} should just be an element name or index without universe, branch, or model/base/design specification.

Example:
  python matrix 2@1>>q01w|design q02w

\end{example}
\item[merit] \Newline\begin{example}
Merit value.
Command syntax:
  python merit


\end{example}
\item[orbit_at_s] \Newline\begin{example}
Twiss at given s position.
Command syntax:
  python orbit_at_s {ix_uni}@{ix_branch}>>{s}|{which}
where:
  {which} is one of:
    model
    base
    design
  {s} is the longitudinal s-position.


Parameters
----------
ix_uni
ix_branch
s
which

   
Returns
-------
?? 

\end{example}
\item[place_buffer] \Newline\begin{example}
Output place command buffer and reset the buffer.
The contents of the buffer are the place commands that the user has issued.
Command syntax:
  python place_buffer


\end{example}
\item[plot_curve] \Newline\begin{example}
Curve information for a plot
Command syntax:
  pyton plot_curve {curve_name}


Parameters
----------
curve_name

   
Returns
-------
?? 

\end{example}
\item[plot_lat_layout] \Newline\begin{example}
Plot Lat_layout info
Command syntax:
  python plot_lat_layout {ix_universe}@{ix_branch}
Note: The returned list of element positions is not ordered in increasing
      longitudinal position.


Parameters
----------
ix_universe
ix_branch

   
Returns
-------
?? 

\end{example}
\item[plot_list] \Newline\begin{example}
List of plot templates or plot regions.
Command syntax:
  python plot_list {r_or_g}
where "{r/g}" is:
  "r"      ! list regions
  "t"      ! list template plots


Parameters
----------
r_or_g

   
Returns
-------
?? 

\end{example}
\item[plot_graph] \Newline\begin{example}
Graph
Command syntax:
  python plot_graph {graph_name}
{graph_name} is in the form:
  {p_name}.{g_name}
where
  {p_name} is the plot region name if from a region or the plot name if a template plot.
  This name is obtained from the python plot_list command.
  {g_name} is the graph name obtained from the python plot1 command.


Parameters
----------
graph_name

   
Returns
-------
?? 

\end{example}
\item[plot_histogram] \Newline\begin{example}
Plot Histogram
Command syntax:
  python plot_histograph {curve_name}


Parameters
----------
curve_name

   
Returns
-------
?? 

\end{example}
\item[plot_plot_manage] \Newline\begin{example}
Template plot creation or destruction.
Command syntax:
  pyton plot_plot_manage {plot_location}^^{plot_name}^^
                         {n_graph}^^{graph1_name}^^{graph2_name}...{graphN_name}
Use "@Tnnn" sytax for {plot_location} to place a plot. A plot may be placed in a 
spot where there is already a template.
If {n_graph} is set to -1 then just delete the plot.


Parameters
----------
plot_location
plot_name
n_graph
graph1_name
graph2_name
graphN_name

   
Returns
-------
?? 

\end{example}
\item[plot_curve_manage] \Newline\begin{example}
Template plot curve creation/destruction
Command syntax:
  pyton plot_curve_manage {graph_name}^^{curve_index}^^{curve_name}
If {curve_index} corresponds to an existing curve then this curve is deleted.
In this case the {curve_name} is ignored and does not have to be present.
If {curve_index} does not not correspond to an existing curve, {curve_index}
must be one greater than the number of curves.


Parameters
----------
graph_name
curve_index
curve_name

   
Returns
-------
?? 

\end{example}
\item[plot_graph_manage] \Newline\begin{example}
Template plot graph creation/destruction
Command syntax:
  pyton plot_graph_manage {plot_name}^^{graph_index}^^{graph_name}
If {graph_index} corresponds to an existing graph then this graph is deleted.
In this case the {graph_name} is ignored and does not have to be present.
If {graph_index} does not not correspond to an existing graph, {graph_index}
must be one greater than the number of graphs.


Parameters
----------
plot_name
graph_index
graph_name

   
Returns
-------
?? 

\end{example}
\item[plot_line] \Newline\begin{example}
Points used to construct a smooth line for a plot curve.
Command syntax:
  python plot_line {region_name}.{graph_name}.{curve_name} {x_or_y}
Optional {x-or-y} may be set to "x" or "y" to get the smooth line points x or y 
component put into the real array buffer.
Note: The plot must come from a region, and not a template, since no template plots 
      have associated line data.
Examples:
  python plot_line r13.g.a   ! String array output.
  python plot_line r13.g.a x ! x-component of line points loaded into the real array buffer.
  python plot_line r13.g.a y ! y-component of line points loaded into the real array buffer.


Parameters
----------
region_name
graph_name
curve_name
x_or_y

   
Returns
-------
?? 

\end{example}
\item[plot_symbol] \Newline\begin{example}
Locations to draw symbols for a plot curve.
Command syntax:
  python plot_symbol {region_name}.{graph_name}.{curve_name} {x_or_y}
Optional {x_or_y} may be set to "x" or "y" to get the symbol x or y 
positions put into the real array buffer.
Note: The plot must come from a region, and not a template, 
      since no template plots have associated symbol data.
Examples:
  python plot_symbol r13.g.a       ! String array output.
  python plot_symbol r13.g.a x     ! x-component of the symbol positions 
                                     loaded into the real array buffer.
  python plot_symbol r13.g.a y     ! y-component of the symbol positions 
                                     loaded into the real array buffer.


Parameters
----------
region_name
graph_name
curve_name
x_or_y

   
Returns
-------
?? 

\end{example}
\item[plot_transfer] \Newline\begin{example}
Transfer plot parameters from the "from plot" to the "to plot" (or plots).
Command syntax:
  python plot_transfer {from_plot} {to_plot}
To avoid confusion, use "@Tnnn" and "@Rnnn" syntax for {from_plot}.
If {to_plot} is not present and {from_plot} is a template plot, the "to plots" 
 are the equivalent region plots with the same name. And vice versa 
 if {from_plot} is a region plot.


Parameters
----------
from_plot
to_plot

   
Returns
-------
?? 

\end{example}
\item[plot1] \Newline\begin{example}
Info on a given plot.
Command syntax:
  python plot1 {name}
{name} should be the region name if the plot is associated with a region.
Output syntax is parameter list form. See documentation at the beginning of this file.


Parameters
----------
name

   
Returns
-------
?? 

\end{example}
\item[shape_list] \Newline\begin{example}
lat_layout and floor_plan shapes list
Command syntax:
  python shape_list {who}
{who} is one of:
  lat_layout
  floor_plan


Parameters
----------
who

   
Returns
-------
?? 

\end{example}
\item[shape_manage] \Newline\begin{example}
element shape creation or destruction
Command syntax:
  python shape_manage {who} {index} {add_or_delete}

{who} is one of:
  lat_layout
  floor_plan
{add_or_delete} is one of:
  add     -- Add a shape at {index}. 
             Shapes with higher index get moved up one to make room.
  delete  -- Delete shape at {index}. 
             Shapes with higher index get moved down one to fill the gap.

Example:
  python shape_manage floor_plan 2 add
Note: After adding a shape use "python shape_set" to set shape parameters.
This is important since an added shape is in a ill-defined state.


Parameters
----------
who
index
add_or_delete

   
Returns
-------
?? 

\end{example}
\item[shape_pattern_list] \Newline\begin{example}
List of shape patterns
Command syntax:
  python shape_pattern_list {ix_pattern}
If optional {ix_pattern} index is omitted then list all the patterns


Parameters
----------
ix_pattern

   
Returns
-------
?? 

\end{example}
\item[shape_pattern_manage] \Newline\begin{example}
Add or remove shape pattern
Command syntax:
  python shape_pattern_manage {ix_pattern}^^{pat_name}^^{pat_line_width}
where:
  {ix_pattern}      -- Pattern index. Patterns with higher indexes will be moved up 
                                      if adding a pattern and down if deleting.
  {pat_name}        -- Pattern name.
  {pat_line_width}  -- Line width. Integer. If set to "delete" then section 
                                            will be deleted.


Parameters
----------
ix_pattern
pat_name
pat_line_width

   
Returns
-------
?? 

\end{example}
\item[shape_pattern_point_manage] \Newline\begin{example}
Add or remove shape pattern point
Command syntax:
  python shape_pattern_point_manage {ix_pattern}^^{ix_point}^^{s}^^{x}
where:
  {ix_pattern}      -- Pattern index.
  {ix_point}        -- Point index. Points of higher indexes will be moved up
                                    if adding a point and down if deleting.
  {s}, {x}          -- Point location. If {s} is "delete" then delete the point.


Parameters
----------
ix_pattern
ix_point
s
x

   
Returns
-------
?? 

\end{example}
\item[shape_set] \Newline\begin{example}
lat_layout or floor_plan shape set
Command syntax:
  python shape_set {who}^^{shape_index}^^{ele_name}^^{shape}^^{color}^^
                   {shape_size}^^{type_label}^^{shape_draw}^^
                   {multi_shape}^^{line_width}
{who} is one of:
  lat_layout
  floor_plan


Parameters
----------
who
shape_index
ele_name
shape
color
shape_size
type_label
shape_draw
multi_shape
line_width

   
Returns
-------
?? 

\end{example}
\item[show] \Newline\begin{example}
Show command pass through
Command syntax:
  python show {line}
{line} is the string to pass through to the show command.
Example:
  python show lattice -python


Parameters
----------
line

   
Returns
-------
?? 

\end{example}
\item[species_to_int] \Newline\begin{example}
Convert species name to corresponding integer
Command syntax:
  python species_to_int {species_str}
Example:
  python species_to_int CO2++


Parameters
----------
species_str

   
Returns
-------
?? 

\end{example}
\item[species_to_str] \Newline\begin{example}
Convert species integer id to corresponding
Command syntax:
  python species_to_str {species_int}
Example:
  python species_to_str -1     ! Returns 'Electron'


Parameters
----------
species_int

   
Returns
-------
?? 

\end{example}
\item[spin_polarization] \Newline\begin{example}
Spin information
Command syntax:
  python spin {ix_uni}@{ix_branch}|{which}
where {which} is one of:
  model
  base
  design
Example:
  python spin 1@0|model

Note: This command is under development. If you want to use please contact David Sagan.


Parameters
----------
ix_uni
ix_branch
which

   
Returns
-------
?? 

\end{example}
\item[super_universe] \Newline\begin{example}
Super_Universe information
Command syntax:
  python super_universe


\end{example}
\item[twiss_at_s] \Newline\begin{example}
Twiss at given s position
Command syntax:
  python twiss_at_s {ix_uni}@{ix_branch}>>{s}|{which}
where {which} is one of:
  model
  base
  design


Parameters
----------
ix_uni
ix_branch
s
which

   
Returns
-------
?? 

\end{example}
\item[universe] \Newline\begin{example}
Universe info
Command syntax:
  python universe {ix_universe}
Use "python global" to get the number of universes.


Parameters
----------
ix_universe

   
Returns
-------
?? 

\end{example}
\item[var] \Newline\begin{example}
Info on an individual variable
Command syntax:
  python var {var}        or
  python var {var} slaves



Parameters
----------
var

   
Returns
-------
?? 

\end{example}
\item[var_create] \Newline\begin{example}
Create a single variable
Command syntax:
  python var_create {var_name}^^{ele_name}^^{attribute}^^{universes}^^
                    {weight}^^{step}^^{low_lim}^^{high_lim}^^{merit_type}^^
                    {good_user}^^{key_bound}^^{key_delta}
{var_name} is something like "kick[5]".
Before using var_create, setup the appropriate v1_var array using 
the "python var_v1_create" command.


Parameters
----------
var_name
ele_name
attribute
universes
weight
step
low_lim
high_lim
merit_type
good_user
key_bound
key_delta

   
Returns
-------
?? 

\end{example}
\item[var_general] \Newline\begin{example}
List of all variable v1 arrays
Command syntax:
  python var_general
Output syntax:
  {v1_var name};{v1_var%v lower bound};{v1_var%v upper bound}


\end{example}
\item[var_v_array] \Newline\begin{example}
List of variables for a given data_v1.
Command syntax:
  python var_v_array {v1_var}
Example:
  python var_v_array quad_k1


Parameters
----------
v1_var

   
Returns
-------
?? 

\end{example}
\item[var_v1_array] \Newline\begin{example}
List of variables in a given variable v1 array
Command syntax:
  python var_v1_array {v1_var}


Parameters
----------
v1_var

   
Returns
-------
?? 

\end{example}
\item[var_v1_create] \Newline\begin{example}
Create a v1 variable structure along with associated var array.
Command syntax:
  python var_v1_create {v1_name} {n_var_min} {n_var_max}
{n_var_min} and {n_var_max} are the lower and upper bounds of the var
Example:
  python var_v1_create quad_k1 0 45
This example creates a v1 var structure called "quad_k1" with an associated
variable array that has the range [0, 45].

Use the "set variable" command to set a created variable parameters.
In particular, to slave a lattice parameter to a variable use the command:
  set {v1_name}|ele_name = {lat_param}
where {lat_param} is of the form {ix_uni}@{ele_name_or_location}{param_name}]
Examples:
  set quad_k1[2]|ele_name = 2@q01w[k1]
  set quad_k1[2]|ele_name = 2@0>>10[k1]
Note: When setting multiple variable parameters, 
      temporarily toggle s%global%lattice_calc_on to False
  ("set global lattice_calc_on = F") to prevent Tao trying to evaluate the 
partially created variable and generating unwanted error messages.


Parameters
----------
v1_name
n_var_min
n_var_max

   
Returns
-------
?? 

\end{example}
\item[var_v1_destroy] \Newline\begin{example}
Destroy a v1 var structure along with associated var sub-array.
Command syntax:
  python var_v1_destroy {v1_datum}


Parameters
----------
v1_datum

   
Returns
-------
?? 

\end{example}
\item[wave] \Newline\begin{example}
Wave analysis info.
Command syntax:
  python wave {what}
Where {what} is one of:
  params
  loc_header
  locations
  plot1, plot2, plot3


Parameters
----------
what

   
Returns
-------
?? 

\end{example}
\end{description}


