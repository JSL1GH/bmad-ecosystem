\chapter{Linear Optics}

%-----------------------------------------------------------------
\section{Coupling and Normal Modes}
\label{s:coupling}
\index{normal mode!Coupling}

The coupling formalism used by \bmad is taken from the paper of Sagan and
Rubin\cite{b:coupling}. The main equations are reproduced here with the notation change
that $\bfA$ and $\bfB$ is replaced by $\bfQ$ and $\bfW$. The reason for this is explained below.

A one--turn map $\bfT(s)$ for the transverse two--dimensional phase space $\bfx = (x, x',
y, y')$ starting and ending at some point $s$ can be written as
  \Begineq
    \bfT = \bfV \, \bfU \, \bfV\inv 
    , \label{tvuv}
  \Endeq 
where $\bfV$ is symplectic, and $\bfU$ is of the form
  \Begineq
    \bfU = 
    \begin{pmatrix}
      \bfQ & \Bf0 \cr 
      \Bf0 & \bfW \cr
    \end{pmatrix}
    . \label{ua00b}
  \Endeq
\index{normal mode!a--mode}
\index{normal mode!b--mode}
Since $\bfU$ is uncoupled the standard Twiss analysis can be performed on the matrices
$\bfQ$ and $\bfW$. The normal modes are labeled $q$ and $w$ and if the one--turn matrix
$\bfT$ is uncoupled then $w$ corresponds to the horizontal mode and $w$ corresponds to the
vertical mode.

The reason why $q$ and $w$ are used here to denote the modes is to avoid confusion with
the \bmad convention of using the labels $a$ and $b$ to represent the same modes
throughout the lattice. At the start of the lattice, by convention, the $a$ mode is the
same as the $q$ mode and the $b$ mode is the same as the $w$ mode. If there is a ``mode
flip'' (see Sagan and Rubin for an explanation of this term) at some spot in the lattice,
the $a$ mode before the mode flip will be the same physical mode after the mode flip (and
similarly for the $b$ mode.  That is, after the mode flip the $a$ mode will be the same as
the $w$ mode and the $b$ mode will be the same as the $q$ mode.

$\bfV$ is written in the form
  \Begineq
    \bfV = 
    \begin{pmatrix}
        \gamma \bfI & \bfC \cr 
        -\bfC^+     & \gamma \bfI \cr
    \end{pmatrix}
    , \label{vgicc1}
  \Endeq
where $\bfC$ is a 2x2 matrix and $+$ superscript 
denotes the symplectic conjugate:
\index{symplectic!conjugate}
  \Begineq
    \bfC^+ = 
    \begin{pmatrix}
       C_{22} & -C_{12} \cr 
      -C_{21} & C_{11} \cr
    \end{pmatrix}
    . \label{ccccc}
  \Endeq
Since we demand that $\bfV$ be symplectic we have the condition
  \Begineq               
    \gamma^2 + \, ||\bfC|| = 1
    , \label{gc1}
  \Endeq
and $\bfV\inv$ is given by
  \Begineq
    \bfV\inv = 
    \begin{pmatrix}
      \gamma \bfI & -\bfC \cr 
      \bfC^+ & \gamma \bfI \cr
    \end{pmatrix}
    . \label{vgicc2}
  \Endeq 
$\bfC$ is a measure of the coupling. 
$\bfT$ is uncoupled if and only if $\bfC = \Bf 0$. 

It is useful to normalize out the $\beta(s)$ variation in the the above
analysis. Normalized quantities being denoted by a bar above them. The
normalized normal mode matrix $\BAR\bfU$ is defined by
  \Begineq
    \BAR\bfU = \bfG \, \bfU \, \bfG\inv
    , \label{ugug}
  \Endeq
Where $\bfG$ is given by 
  \Begineq
    \bfG \equiv 
    \begin{pmatrix}
      \bfG_q & \Bf0 \cr 
      \Bf0 & \bfG_w
    \end{pmatrix}
    , \label{gg00g}
  \Endeq  
with 
  \Begineq
    \bfG_q = 
    \begin{pmatrix}
      \frac{\tstyle 1}{\tstyle \sqrt{\beta_q}} & 0 \cr
      \frac{\tstyle \alpha_q}{\tstyle \sqrt{\beta_q}} & \sqrt{\beta_q}
    \end{pmatrix}
    , \label{g1b0a} 
  \Endeq
with a similar equation for $\bfG_w$. With this definition, the corresponding
$\BAR\bfQ$ and $\BAR\bfW$ (cf.~\Eq{ua00b}) are just rotation matrices.
The relationship between $\bfT$ and $\BAR\bfU$ is 
  \Begineq
    \bfT = \bfG\inv \, \BAR\bfV \, \BAR\bfU \, \BAR\bfV\inv \, \bfG
    , \label{tgvuv}
  \Endeq
where
  \Begineq
    \BAR\bfV = \bfG \, \bfV \, \bfG\inv
    . \label{vgvg}
  \Endeq
Using \Eq{gg00g}, $\BAR\bfV$ can be written in the form
  \Begineq
    \BAR\bfV = 
    \begin{pmatrix}
      \gamma \bfI & \BAR\bfC \cr -\BAR\bfC^+ & \gamma \bfI
    \end{pmatrix}
    , \label{vgicc3}
  \Endeq
with the normalized matrix $\BAR\bfC$ given by
  \Begineq
    \BAR\bfC = \bfG_q \, \bfC \, \bfG_w\inv
    . \label{cgcg}
  \Endeq

The normal mode coordinates ${\bf q} = (q, q', w, w')$ are related to
the laboratory frame via
  \Begineq
    {\bf q} = \bfV\inv \, {\bf x}
    . \label{avx}
  \Endeq 
In particular the normal mode dispersion $\bfeta_q = (\eta_q,
\eta'_q, \eta_w, \eta'_w)$ is related to the laboratory frame
dispersion $\bfeta_x = (\eta_x, \eta'_x, \eta_y, \eta'_y)$ via
  \Begineq
    {\bfeta_q} = \bfV\inv \, {\bfeta_x}
    . \label{etaavx}
  \Endeq 
When there is no coupling ($\bfC = 0$), $\bfeta_q$ and $\bfeta_x$ are
equal to each other.

%-----------------------------------------------------------------
\section{Dispersion Calculation}
\label{s:dispersion}
\index{dispersion|hyperbf}

The dispersion ($\eta$) and the dispersion derivative ($\eta'$) are 
defined by the equations
\begin{align}
  \eta_x(s) &\equiv \left. \frac{dx}{dp_z} \right|_s \comma \qquad
    \eta'_x(s) \equiv \left. \frac{d\eta_x}{ds} \right|_s
    = \left. \frac{dx'}{dp_z} \right|_s \CRNO
  \eta_y(s) &\equiv \left. \frac{dy}{dp_z} \right|_s \comma \qquad
    \eta'_y(s) \equiv \left. \frac{d\eta_y}{ds} \right|_s
    = \left. \frac{dy'}{dp_z} \right|_s \\
  \eta_z(s) &\equiv \left. \frac{dz}{dp_z} \right|_s \nonumber
\end{align}

Given the dispersion at a given point, the dispersion at some other
point is calculated as follows: Let $\Bf r = (x, p_x, y, p_y, z, p_z)$
be the reference orbit, around which the dispersion is to be
calculated. Let $\bfV$ and $\bf M$ be the zeroth and first order
components of the transfer map between two points labeled 1 and 2:
\Begineq
  \Bf r_2 = \bfM \, \Bf r_1 + \bfV
  \label{rmrv}
\Endeq
Define the dispersion vector $\bfeta$ by
\Begineq
  \bfeta = 
  \left( 
    \eta_x, \eta'_x \, (1 + p_z), \eta_y, \eta'_y \, (1 + p_z), \eta_z, 1
  \right)
\Endeq
Differentiating \Eq{rmrv} with respect to energy, 
the dispersion at point 2 in terms of the dispersion at point 1 is
\Begineq
  \bfeta_2 = \left[ \frac{dp_{z2}}{dp_{z1}} \right]^{-1} \, 
    \left[ \bfM \, \bfeta_1 \right] + \bfV_\eta 
    \label{eppmev}
\Endeq
where
\Begineq
  \bfV_\eta = \left[ \frac{dp_{z2}}{dp_{z1}} \right]^{-1} \, 
  \frac{1}{1 + p_{z1}}
  \left(
  \begin{array}{c}
    M_{12} \, p_{x1} + M_{14} \, p_{y1} \\
    M_{22} \, p_{x1} + M_{24} \, p_{y1} \\
    M_{32} \, p_{x1} + M_{34} \, p_{y1} \\
    M_{42} \, p_{x1} + M_{44} \, p_{y1} \\
    M_{52} \, p_{x1} + M_{54} \, p_{y1} \\
    M_{62} \, p_{x1} + M_{64} \, p_{y1} \\
  \end{array}
  \right)
  -
  \left(
  \begin{array}{c}
    0 \\
    \frac{p_{x2}}{1 + p_{z2}} \\
    0 \\
    \frac{p_{y2}}{1 + p_{z2}} \\
    0 \\
    0 
  \end{array}
  \right)
\Endeq
The sixth row of the matrix equation gives $dp_{z1}/dp_{z2}$. 
Explicitly
\Begineq
  \frac{dp_{z2}}{dp_{z1}} =
  \sum_{i=1}^6 M_{6i} \, \eta_{1i} + 
  \frac{M_{62} \, p_{x1} + M_{64} \, p_{y1}}{1 + p_{z1}}
\Endeq
For everything except \vn{RFcavity} and \vn{Lcavity} elements, 
$dp_{z2}/dp_{z1}$ is 1.

For a non-circular machine, there are two ways one can imagine
defining the dispersion: Either with respect to changes in energy at
the beginning of the machine or with respect to the local change in
energy at the point of measurement. The former definition will be
called ``non-local dispersion'' and the latter definition will be
called ``local dispersion''. For a circular machine, local dispersion
is always used.  The dispersion defined in the above equations, which
is what \bmad uses in calculations, is the local dispersion. The
non-local dispersion $\wt\bfeta(s_1)$ at some point $s_1$ is related
to the local dispersion $\bfeta(s_1)$ via
\Begineq
  \wt\bfeta(s_1) = \frac{dp_{z1}}{dp_{z0}} \, \bfeta(s_1)
\Endeq
where $s_0$ is the beginning of the machine.

For a non-circular machine, there are advantages and disadvantages to
using either local or non-local dispersion. Local dispersion has the
problem that $dp_{z2}/dp_{z1}$ in \Eq{eppmev} may go through zero at a
point producing infinite dispersions at that point. The non-local
dispersion has the merit of reflecting what one would measure if the
starting energy of the beam is veried. The local dispersion, on the
other hand, reflects the correlations between the particle energy and
particle position within a beam.

