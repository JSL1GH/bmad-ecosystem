%-----------------------------------------------------------------
%\chapter{Introduction}
%\label{c:prog_intro} 

\chapter{Creating a Custom Version of \tao}
\label{c:prog_intro} 

\tao has been designed to be ready extensible with a minimum of
effort. The tutorial providesa simple example of a custom data type. Here each
the hook routines are explained and pointers are given for writing code. 

The process for customizing is summarized as follows: For the purposes of this
discussion assume that the directory that you are developing \tao in
is called \vn{ROOT}. 
\begin{enumerate}
\item 
The first step is to checkout from CVS (or download a
copy) of \tao. You will now have \tao in
\vn{ROOT/tao}. The library source code will be sitting in \vn{ROOT/tao/code}
and the vanilla \tao program is \vn{ROOT/tao/program/tao_cl.f90}
\item 
From the \vn{ROOT}/tao area build the \tao library using the
\vn{gmake} command. This will create a \tao library that includes dummy hook
routines and structures. When creating custom hooks these are overiden.
\item
To extend \tao you will want to make a new
directory, say, called \vn{ROOT/my_tao}. In this directory you write
the necessary routines to extend \tao. You will also need a standard \vn{Makefile}
 for building programs.
\item
You can compile and link your routines with the \tao routines using
\vn{gmake} in \vn{ROOT/my_tao}.
\end{enumerate}

The tutorial in Part I of this manual gave an example of how to carry out the
above steps. This is repeated below but in greater detail.

\section{Creating the \tao Library and a Custom \tao Directory}

After obtaining the \tao distribution the \tao library is created by typing 
\cmd{gmake} in the \cmd{ROOT/tao} directory. This will create two libraries
called \cmd{libtao.a} and \cmd{libtao_g.a} where the second is a debug version
in the directory \cmd{ROOT/lib}. If you then type \cmd{gmake -f M.tao} then
"vanilla" \tao will be compiled called \cmd{tao} and \cmd{tao_g} and placed in
\cmd{ROOT/bin}


\section{Modifying the Hook Routines and Structures}

The golden rule when writing routines to extend \tao is that you are
only allowed to replace routines or redefine structures that have the
name ``hook'' in them. The reason for this is to ensure that, as time
goes by, and revisions are made to the \tao routines to extend the
usefulness of \tao and to eliminate bugs, that these changes will
have a minimum impact on the specialized routines that will be written
by various people to extend \tao.  What happens if you need to replace
or modify a non--hook routine or structure?  The answer is to contact
the \tao programming team and we will modify \tao and create the hooks 
you need.

Before one can begin writing code one must understand the structures
that \tao uses. The structures are defined in a file
\vn{tao/code/tao_struct.f90}. It is a good idea to have a copy of
\vn{tao/code/tao_struct.f90} easily accessible. It will be refered to frequently
when customizing \tao. Examine the \vn{tao_super_universe_struct} and
\vn{tao_universe_struct} structures in particular. They reference all other
structures, either directly or indirectly, in \vn{tao_struct}. 
\tao is based upon the \bmad software
package for the simulation of relativistic charged particles and the
\tao structures have components that are defined in \bmad. For
information on these structures see the \bmad Reference Manual. Any hook 
structures are defined in a file \vn{tao_hook_mod.f90}.

Also, to get a good idea of how \tao works it is recommended to spend a
little bit of time going through the \cmd{tao/code/*.f90} files. This may also
provide pointers on how to make customizations in the hook routines.

The following is a run through of each of the hook routines. Each routine
is in a separate file called \vn{tao/hook/<hook_routine_name>.f90}. See these
files for subroutine headers and plenty of comments throughout the dummy code to aid
in the modification of these subroutines.

\subsection{tao\_hook\_command}

Any custom commands are placed here. The dummy subrotuine already has an
ectensive amount of code that replicates what is performed in
\vn{tao_command}. It also inlcudes some helper rotuines for deconstructing the
command line syntax -- just like what is performed in \vn{tao_command}. It isn't
neceddary to use this included infrastructure. All that is required is that the
subroutine interface remaines the same. Commands placed here are searched before
the standard \tao commands are searched. This allows for the overwriting of any
standard \tao command.

By default, there is one command included in here: \vn{`hook'}. This is just a
simple command that doesn't reallt do anything for the purposes of demonstrating
how a custom command would be implemented.

The only thing to call at the end of a custom command is
\vn{tao_cmd_end_calc} (as is done in the dummt subroutine). This will perform
all of the steps listed in Section~\ref{s:lat_calc}. If this isn't called than
whatever changes are made will not show up until a standard command is
performed.

\subsection{tao\_hook\_init\_design\_lattice}

This will do a custom lattice initialialization. The standard lattice
initialization just calls \vn{bmad_parser} or \vn{xsif_parser}. If anything more
complex needs to be done then do it here. This is also where any custom overlays
or other elements would be inserted after the parsing is complete. But in
general, anything placed here should, in principle, be something that can be
placed in a lattice file. 

\textbf{This is the only routine that should insert elements in the ring}. This is
beacuse the \tao data structures use the element index for each element
associated with the datum. If all the element indexes shift then the data
structures will break. If new elements need to be inserted then modify this
routine and recompile. You can alternatively create a custom initialization file
used by this rotuine
that reads in any elements to be inserted.

\subsection{tao\_hook\_init}

After the lattice and all global and universe structures are intialized then
\vn{tao_hook_init} is called. Here, any further intializations can be added. In
paricular, if any custom hook structures need to be initialized, here's the
place to do it.

\subsection{tao\_hook\_lattice\_calc}

The standard lattice calculation can either be performed for single particle
tracking or macroparticle tracking and will recalculate the orbit, transfer
matrices and twiss perameters. If something else needs to be performed whenever
the lattice is recalculated then it is placed here. A custom lattice calculation
can be performed any lattice separately, this allows for the possibility of, for
example, 
tracking a single particle for one lattice and macroparticles in another.

\subsection{tao\_hook\_load\_load\_data\_array}

Any custom data types are defined and calculated here. If a non-standard data
type is listed in the intialization files then a corresponding data type must
be placed in this routine. The tutorial uses this hook routine when
calculating the emittance. 

As exmplained in the dummy file, the datum merit type
affects how a datum's calue should be calculated. There is a helper subroutine
in the dummy hook routine called \vn{load_it} to aid in modifying the datum's value based on the
merit type. If there is a range of elements associated with datum then a merit
type other than \vn{target} requires that the entire range of elements
associated with the datum be searched for the appropriate value to be returened.
See Chapter~\ref{c:opti} for details on the merit type. 

For example, if the data type is \vn{orbit:x} (yes, this is already defined in
\tao) then the appropriate case in \vn{tao_hook_load_data_array} would be:
\begin{example}
select case (datum%data_type)

case ('orbit:x')
  call load_it (orb(:)%vec(1))
\end{example}
\vn{load_it} will then look at each datum. If the merit type is other than
\vn{target} then \vn{load_it} will search the appropriate range of elements for
either the minimum or maximum horizontal orbit value and this will be the
datum's value. Because, a datum may refer to range of elements, the entire orbit
array (from 0 to \vn{n_ele_max}) must be passed to \vn{load_it} for each \vn{orbit:x} datum.

If the only merit type that is going to be used is \vn{target} then \vn{load_it}
can be ignored.

\subsection{tao\_hook\_merit\_data}

A custom data merit type can be defined here. Table~\ref{t:con_type} lists the
standard merit types. If a custom merit type is used then \vn{load_it} in
\vn{tao_hook_load_data_array} may also need to be modified to handle this merit
type, additionally, all standard datd types may need to be overridden in 
\vn{tao_hook_load_data_array} in order for the custom \vn{load_it} to be used.
See \vn{tao/code/tao_merit.f90} for how the standard merit types are calculated.

\subsection{tao\_hook\_merit\_var}

Likewise, this hook will allow for a custom variable merit type. However, since
there is no corresponding data transfer, no \vn{load_it} routine needs to be modified.
See \vn{tao/code/tao_merit.f90} for how the standard merit types are calculated.

\subsection{tao\_hook\_optimizer}

If a non standard optimizer is needed then it can be implemented here. See the
\vn{tao/code/tao_*_optimizer.f90} files for how the standard optimizers are
implemented.

\subsection{tao\_hook\_post\_process\_data}

Here can be placed anything that needs to be done after the data arrays are
loaded. This routine is called immediately after the data arrays are called and
before the optimizer or plotting is done, so any final modifications to the
lattice or data can be performed here. For example, this is whewre BPM
resolution can be handled.

\subsection{tao\_hook\_mod}

Here any custom structures are defined. In the dummy hook routine there are
already a number of structures defined. These are used in
\vn{tao/code/tao_struct.f90} so even if they are not used there must be a dummy
structure defined so that the compiler doesn't complain.

\textbf{This module is different from all the other hook routine in that the \tao
library must be compiled after this file is modified so that} \vn{tao_struct}
\textbf{knows about the hook structure}. If hook structures are needed but only used in
hook routines then it is recommended that these be placed in a separate file so
that it can be compiled when the custom \tao program is compiled. Because of the
special status of this hook module, it is not placed in the standard hook
directory but in the \vn{tao/code/} directory.

%-----------------------------------------------------------------
\chapter{Plotting}
\label{s:prog_plotting} 

Plotting is based upon the \vn{quick_plot} subroutines which are
documented in the \bmad reference manual and you should review this
material if you are not familiar with concepts of ``graph'', ``box'',
and ``page''. 

\fbox{this chapter is yet to be completed!} 

