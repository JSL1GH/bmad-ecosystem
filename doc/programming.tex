\chapter{Programming Tao}
\index{programming}
\label{c:programming}

This chapter gives an overview of the coding structure of Tao.
Knowledge of this structure is needed in order to create custom
versions of \tao. See Chapter \sref{c:custom.tao} for details of
how to create custom versions.

%-----------------------------------------------------------------
\section{Overview}
\index{programming!overview}

The \tao code files are stored in the following directories:
\begin{example}
  tao/code
  tao/hooks
  tao/program
\end{example}
Here \vn{tao} is the root directory of \tao. Ask your local guru
where to find this directory.

The files in \vn{tao/code} should not be modified when creating
custom versions of \tao. The files in \vn{tao/hooks}, as
explained in Chapter \sref{c:custom.tao}, are templates used
for customization. Finally, the directory \vn{tao/program} holds
the program file \vn{tao_program.f90}.

The structures used by tao are defined in the file \vn{tao_struct.f90}.
All \tao structures begin with the prefix \vn{tao_} so any structure
encountered that does not begin with \vn{tao_} must be defined in some other library

%-----------------------------------------------------------------
\section{tao_super_universe_struct}
\index{tao_super_universe_struct}

The "root" structure in \tao is the \vn{tao_super_universe_struct}. 
The definition of this structure is:
\begin{example}
  type tao_super_universe_struct
    type (tao_global_struct) global                      ! Global variables.
    type (tao_common_struct) :: com                      ! Global variables
    type (tao_plotting_struct) :: plotting               ! Plot parameters.
    type (tao_v1_var_struct), allocatable :: v1_var(:)   ! V1 Variable array
    type (tao_var_struct), allocatable :: var(:)         ! Array of all variables.
    type (tao_universe_struct), allocatable :: u(:)      ! Array of universes.
    type (tao_mpi_struct) mpi
    integer, allocatable :: key(:)
    type (tao_building_wall_struct) :: building_wall
    type (tao_wave_struct) :: wave 
    integer n_var_used
    integer n_v1_var_used
  end type
\end{example}
An instance of this structure called \vn{s} is defined in \vn{tao_struct.f90}:
\begin{example}
  type (tao_super_universe_struct), save, target :: s
\end{example}
Thus \vn{s} is common to all of \tao's routines.

The components of the \vn{tao_super_universe_struct} are:
  \begin{description}
  \index{tao_global_struct}
  \item[\%global] \Newline
The \vn{%global} component contains global variables that a user can set
in an initialization file.
See \sref{s:globals} for more details.
  \index{tao_common_struct}
  \item[\%com] \Newline
The \vn{%com} component is for global variables that are not directly
user accessible.
  \index{tao_plotting_struct}
  \item[\%plotting] \Newline
The \vn{%plotting} component holds parameters used in plotting.
  \index{tao_v1_var_struct}
  \item[\%v1_var(:)] \Newline
The \vn{%v1_var(:)} component is an array of all the \vn{v1_var} blocks
(\sref{c:var}) that the user has defined. The range of valid blocks
goes from 1 to \vn{%n_v1_var_used}.
  \index{tao_var_struct}
  \item[\%var(:)]
The \vn{%var(:)} array holds a list of all variables (\sref{c:var})
that the user has defined. The range of valid variables goes from
1 to \vn{n_var_used}.
  \index{tao_universe_struct}
  \item[\%u(:)] \Newline
The \vn{%u(:)} component is an array of universes (\sref{s:universe}).
  \index{tao_mpi_struct}
  \item[\%mpi] \Newline
The \vn{%mpi} component holds parameters needed for parallel processing.
  \item[\%key(:)] \Newline
The \vn{%key(:)} component is an array of indexes used for key bindings 
(\sref{s:key.bind}). The value of \vn{%key(i)} is the index in the 
\vn{%var(:)} array associated with the $i$\th key.
  \index{tao_building_wall_struct}
  \item[\%building_wall] \Newline
The \vn{%building_wall} component holds parameters associated
with a building wall (\sref{s:building.wall}).
  \index{tao_wave_struct}
  \item[\%wave] \Newline
The \vn{%wave} component holds parameters needed for the wave analysis
(\sref{c:wave}).
  \end{description}
  


