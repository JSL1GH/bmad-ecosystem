%-----------------------------------------------------------------
%\chapter{Introduction}
%\label{c:intro} 

\tao has been designed to be ready extensible with a minimum of
effert.  The process is as follows: For the purposes of this
discussion assume that the directory that you are developing \tao in
is called \vn{ROOT}. The first step is to checkout from CVS (or download a
copy) of \tao and build it on your computer. You will now have \tao in
\vn{ROOT}/tao. The code will be sitting in \vn{ROOT/tao/code}, a copy of this
manual will be in \vn{ROOT/tao/doc}, and the vanilla \tao program will be
in \vn{ROOT/tao/program}. To extend \tao you will want to make a new
directory, say, called \vn{ROOT/my_tao}. In this directory you write
the necessary routines to extend \tao and then using \vn{gmake} you
can compile and link your routines with the \tao routines.

The golden rule when writing routines to extend \tao is that you are
only allowed to replace routines that have the name ``hook'' in them
and you are only allowed to redefine structures that have the name
``hook'' in them. The reason for this is to ensure that, as time goes
by, and revisions are made to the \tao routines to extend the
usefulness of \tao and to elmiminate bugs, that these changes will
have a minimum impact on the specialized routines that will be written
by various people to extend \tao.  What happens if you need to replace
or modify a non--hook routine or modify a non--hook structure?  The
answer is to contact the \tao programming team (OK so it's only me,
David Sagan) and we (well, I) will modify \tao and create the hooks
you need.


Before one can begin writing code one must understand the structures
that \tao uses. The structures are defined in a file \vn{tao_struct.f90}. 
\tao is based upon the \bmad software package for simulations of 
relativistic charaged particles and the \tao structures have components
that are defined in \bmad. For information on these structures see the \bmad
Reference Manual. Note that the hook--structures are defined in a file
\vn{tao_hook_mod.f90}.

%-----------------------------------------------------------------
%\section{Plotting}
%\label{s:plotting} 

Plotting is based upon the \vn{quick_plot} subroutines which are
documented in the \bmad reference manual and you should review this
material if you are not familier with concepts of ``graph'', ``box'',
and ``page''. 

