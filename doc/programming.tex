\chapter{Programming Tao}
\index{programming}
\label{c:programming}

This chapter gives an overview of the coding structure of Tao.
Knowledge of this structure is needed in order to create custom
versions of \tao. See Chapter \sref{c:custom.tao} for details of
how to create custom versions.

%-----------------------------------------------------------------
\section{Overview}
\index{programming!overview}

The \tao code files are stored in the following directories:
\begin{example}
  tao/code
  tao/hooks
  tao/program
\end{example}
Here \vn{tao} is the root directory of \tao. Ask your local guru
where to find this directory.

The files in \vn{tao/code} should not be modified when creating
custom versions of \tao. The files in \vn{tao/hooks}, as
explained in Chapter \sref{c:custom.tao}, are templates used
for customization. Finally, the directory \vn{tao/program} holds
the program file \vn{tao_program.f90}.

The structures used by tao are defined in the file \vn{tao_struct.f90}.
All \tao structures begin with the prefix \vn{tao_} so any structure
encountered that does not begin with \vn{tao_} must be defined in some other library

%-----------------------------------------------------------------
\section{tao_super_universe_struct}
\label{s:super.uni.struct}
\index{tao_super_universe_struct}

The "root" structure in \tao is the \vn{tao_super_universe_struct}. 
The definition of this structure is:
\begin{example}
  type tao_super_universe_struct
    type (tao_global_struct) global                      ! Global variables.
    type (tao_common_struct) :: com                      ! Global variables
    type (tao_plotting_struct) :: plotting               ! Plot parameters.
    type (tao_v1_var_struct), allocatable :: v1_var(:)   ! V1 Variable array
    type (tao_var_struct), allocatable :: var(:)         ! Array of all variables.
    type (tao_universe_struct), allocatable :: u(:)      ! Array of universes.
    type (tao_mpi_struct) mpi
    integer, allocatable :: key(:)
    type (tao_building_wall_struct) :: building_wall
    type (tao_wave_struct) :: wave 
    integer n_var_used
    integer n_v1_var_used
    type (tao_cmd_history_struct) :: history(1000)        ! command history
  end type
\end{example}
An instance of this structure called \vn{s} is defined in \vn{tao_struct.f90}:
\begin{example}
  type (tao_super_universe_struct), save, target :: s
\end{example}
This \vn{s} variable is common to all of \tao's routines and serves as a giant common block for \tao.

The components of the \vn{tao_super_universe_struct} are:
  \begin{description}
  \index{tao_global_struct}
  \item[\%global] \Newline
The \vn{%global} component contains global variables that a user can set
in an initialization file.
See \sref{s:globals} for more details.
  \index{tao_common_struct}
  \item[\%com] \Newline
The \vn{%com} component is for global variables that are not directly
user accessible.
  \index{tao_plotting_struct}
  \item[\%plot_page] \Newline
The \vn{%plot_page} component holds parameters used in plotting (\sref{s:s.plot.page}).
  \index{tao_v1_var_struct}
  \item[\%v1_var(:)] \Newline
The \vn{%v1_var(:)} component is an array of all the \vn{v1_var} blocks
(\sref{c:var}) that the user has defined (\sref{s:s.v1.var}).
  \index{tao_var_struct}
  \item[\%var(:)]
The \vn{%var(:)} array holds a list of all variables (\sref{c:var})
that the user has defined (\sref{s:s.var}).
  \index{tao_universe_struct}
  \item[\%u(:)] \Newline
The \vn{%u(:)} component is an array of universes (\sref{s:universe}) (\sref{s:s.u}).
  \index{tao_mpi_struct}
  \item[\%mpi] \Newline
The \vn{%mpi} component holds parameters needed for parallel processing (\sref{s:s.mpi}).
  \item[\%key(:)] \Newline
The \vn{%key(:)} component is an array of indexes used for key bindings 
(\sref{s:key.bind}). 
  \index{tao_building_wall_struct}
  \item[\%building_wall] \Newline
The \vn{%building_wall} component holds parameters associated
with a building wall (\sref{s:building.wall}).
  \index{tao_wave_struct}
  \item[\%wave] \Newline
The \vn{%wave} component holds parameters needed for the wave analysis
(\sref{c:wave}).
  \item[\%history] \Newline
The \vn{%history} component holds the command history (\sref{s:s.history}).
  \end{description}

%-----------------------------------------------------------------
\section{s%plot_page Component}
\lable{s:s.plot.page}

The \vn{s%plot_page} component of the \vn{super universe} (\sref{s:super.uni.struct} is
a \vn{tao_plot_page_struct} which has components:
\begin{example}
  type tao_plot_page_struct
    type (tao_title_struct) title(2)          ! Titles at top of page.
    type (qp_rect_struct) border              ! Border around plots edge of page.
    type (tao_drawing_struct) :: floor_plan
    type (tao_drawing_struct) :: lat_layout
    type (tao_shape_pattern_struct), allocatable :: pattern(:)
    type (tao_plot_struct), allocatable :: template(:)  ! Templates for the plots.
    type (tao_plot_region_struct), allocatable :: region(:)
    character(8) :: plot_display_type = 'X'   ! 'X' or 'TK'
    character(80) ps_scale                    ! scaling when creating PS files.
    real(rp) size(2)                          ! width and height of window in pixels.
    real(rp) :: text_height = 12              ! In points. Scales the height of all text
    real(rp) :: main_title_text_scale  = 1.3  ! Relative to text_height
    real(rp) :: graph_title_text_scale = 1.1  ! Relative to text_height
    real(rp) :: axis_number_text_scale = 0.9  ! Relative to text_height
    real(rp) :: axis_label_text_scale  = 1.0  ! Relative to text_height
    real(rp) :: legend_text_scale      = 0.7  ! Relative to text_height
    real(rp) :: key_table_text_scale   = 0.9  ! Relative to text_height
    real(rp) :: curve_legend_line_len  = 50   ! Points
    real(rp) :: curve_legend_text_offset = 10 ! Points
    real(rp) :: floor_plan_shape_scale = 1.0
    real(rp) :: lat_layout_shape_scale = 1.0
    integer :: n_curve_pts = 401              ! Default number of points for plotting a smooth curve.
    integer :: id_window = -1                 ! X window id number.
    logical :: delete_overlapping_plots = .true. ! Delete overlapping plots when a plot is placed?
  end type
\end{example}

\begin{description}
  \item[\%template(:)] \Newline
The \vn{%template(:)} array contains the array of plot templates defined by the user (\sref{s:template}) and/or
the default plot templates which are created in the routine \vn{tao_init_plotting}.

  \item[

%-----------------------------------------------------------------
\section {s%v1_var Component}
\lable{s:s.v1.var}

 The range of valid blocks
goes from 1 to \vn{%n_v1_var_used}.

%-----------------------------------------------------------------
\section {s%var Component}
\lable{s:s.var}


 The range of valid variables goes from
1 to \vn{n_var_used}.

%-----------------------------------------------------------------
\section {s%u Component}
\lable{s:s.u}

%-----------------------------------------------------------------
\section {s%mpi Component}
\lable{s:s.mpi}

%-----------------------------------------------------------------
\section {s%var Component}
\lable{s:s.var}

%-----------------------------------------------------------------
\section {s%key Component}
\lable{s:s.key}

The value of \vn{%key(i)} is the index in the 
\vn{%var(:)} array associated with the $i$\th key.

%-----------------------------------------------------------------
\section {s%building_wall Component}
\lable{s:s.building_wall}

%-----------------------------------------------------------------
\section {s%wave Component}
\lable{s:s.wave}

%-----------------------------------------------------------------
\section {s%history Component}
\lable{s:s.history}
