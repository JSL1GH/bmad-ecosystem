%-----------------------------------------------------------------
%\chapter{Introduction}
%\label{c:intro} 

\tao has been designed to be ready extensible with a minimum of
effort.  The process is as follows: For the purposes of this
discussion assume that the directory that you are developing \tao in
is called \vn{ROOT}. 
\begin{enumerate}
\item 
The first step is to checkout from CVS (or download a
copy) of \tao. You will now have \tao in
\vn{ROOT}/tao. The library source code will be sitting in \vn{ROOT/tao/code}
and the vanilla \tao program is in \vn{ROOT/tao/program/tao_cl.f90}
\item 
From the \vn{ROOT}/tao area build the \tao library using the
\vn{gmake} command.
\item
If you want to play around with the vanilla \tao program you can make it
from the \vn{ROOT}/tao directory using the \vn{gmake -f M.tao} command.
\item
There are some test input files in \vn{ROOT/tao/program} so if you run
\tao from this directory you can play around with \tao.
\item
To extend \tao you will want to make a new
directory, say, called \vn{ROOT/my_tao}. In this directory you write
the necessary routines to extend \tao. You will also need a standard \vn{Makefile} for building programs.
\item
You can compile and link your routines with the \tao routines using
\vn{gmake} in \vn{ROOT/my_tao}.
\end{enumerate}

The golden rule when writing routines to extend \tao is that you are
only allowed to replace routines or redefine structures that have the
name ``hook'' in them. The reason for this is to ensure that, as time
goes by, and revisions are made to the \tao routines to extend the
usefulness of \tao and to eliminate bugs, that these changes will
have a minimum impact on the specialized routines that will be written
by various people to extend \tao.  What happens if you need to replace
or modify a non--hook routine or structure?  The answer is to contact
the \tao programming team (OK so it's only me, David Sagan) and we
(well, I) will modify \tao and create the hooks you need.

Before one can begin writing code one must understand the structures
that \tao uses. The structures are defined in a file
\vn{tao/code/tao_struct.f90}.  \tao is based upon the \bmad software
package for simulations of relativistic charged particles and the
\tao structures have components that are defined in \bmad. For
information on these structures see the \bmad Reference Manual. Note
that the hook--structures are defined in a file \vn{tao_hook_mod.f90}.

%-----------------------------------------------------------------
%\section{Plotting}
%\label{s:plotting} 

Plotting is based upon the \vn{quick_plot} subroutines which are
documented in the \bmad reference manual and you should review this
material if you are not familiar with concepts of ``graph'', ``box'',
and ``page''. 

