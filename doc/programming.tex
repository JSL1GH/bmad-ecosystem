%-----------------------------------------------------------------
%\chapter{Introduction}
%\label{c:prog_intro} 

\chapter{Creating a Custom Version of \tao}\index{customizing}
\label{c:prog_customizing} 

\tao has been designed to be readily extensible with a minimum of
effort. The tutorial provides a simple example of a custom data type. Here each
the hook routines is explained and pointers are given on writing code. 

The process for customizing is summarized as follows. For the purposes of this
discussion assume that the directory that you are developing \tao in
is called \vn{ROOT}. 
\begin{enumerate}
\item 
The first step is to checkout from CVS (or download a
copy) of \tao. You will now have \tao in
\vn{ROOT/tao}. The library source code will be sitting in \vn{ROOT/tao/code}
and the vanilla \tao program is \vn{ROOT/tao/program/tao_cl.f90}
\item 
From the \vn{ROOT}/tao area build the \tao library using the
\vn{gmake} command. This will create a \tao library that includes dummy hook
routines and structures. When creating custom hooks these are overiden.
\item
To extend \tao you will want to make a new
directory, say, called \vn{ROOT/my_tao}. In this directory you write
the necessary routines to extend \tao. You will also need a standard \vn{Makefile}
 for building programs.
\item
You can compile and link your routines with the \tao routines using
\vn{gmake} in \vn{ROOT/my_tao}.
\end{enumerate}

The tutorial in Part I of this manual gave an example of how to carry out the
above steps.

\section{Creating the Tao Library and a Custom Tao Directory}

After obtaining the \tao distribution the \tao library is created by typing
\cmd{gmake} in the \cmd{ROOT/tao} directory. This will create two libraries
called \cmd{libtao.a} and \cmd{libtao_g.a} in the directory \cmd{ROOT/lib} where
the second is a debug version. If you then type \cmd{gmake -f M.tao} "vanilla"
\tao will be compiled into the executables called \cmd{tao} and \cmd{tao_g} and
placed in \cmd{ROOT/bin}


\section{Modifying the Hook Routines and Structures}\index{Customizing!Hook Routines}

The golden rule when writing routines to extend \tao is that you are
only allowed to replace routines or redefine structures that have the
name ``hook'' in them. The reason for this is to ensure that, as time
goes by, and revisions are made to the \tao routines to extend it's
usefulness and to eliminate bugs, these changes will
have a minimum impact on the specialized routines that will be written
by various people.  What happens if you need to replace
or modify a non--hook routine or structure?  The answer is to contact
the \tao programming team and we will modify \tao and provide the hooks 
you need so that you can then do your customization.

Before one can begin writing code one must understand the structures
that \tao uses. The structures are defined in a file
\vn{tao/code/tao_struct.f90}. It is a good idea to have a copy of
\vn{tao/code/tao_struct.f90} easily accessible. It will be refered to frequently
when customizing \tao. Examine the \vn{tao_super_universe_struct} and
\vn{tao_universe_struct} structures in particular. They reference all other
structures, either directly or indirectly, in \vn{tao_struct}. 
\tao is based upon the \bmad software
package for the simulation of relativistic charged particles and the
\tao structures have components that are defined in \bmad. For
information on these structures see the \bmad Reference Manual. Any hook 
structures can be defined in the file \vn{tao_hook_mod.f90} but see the entry
below on this file before adding any structures.

Also, to get a good idea of how \tao works it is recommended to spend a
little bit of time going through the \cmd{tao/code/*.f90} files. This may also
provide pointers on how to make customizations in the hook routines. Of
particular interest is the module \vn{tao_lattice_calc_mod.f90} where tracking
and lattice parameters are computed. The routines to calculate the data structures
are also called from within this module.

Plotting is based upon the \vn{quick_plot} subroutines which are
documented in the \bmad reference manual. If custom plotting is
desired this material should be reviewed to get familiar with the
concepts of ``graph'', ``box'', and ``page''.

The following is a run through of each of the hook routines. Each
routine is in a separate file called
\vn{tao/hook/<hook_routine_name>.f90}. See these files for subroutine
headers and plenty of comments throughout the dummy code to aid in the
modification of these subroutines.

\subsection{tao\_hook\_command}\index{Customizing!tao_hook_commad}

Any custom commands are placed here. The dummy subroutine already has
a bit of code that replicates what is performed in
\vn{tao_command}. Commands placed here are searched before the
standard \tao commands. This allows for the overwriting of any
standard \tao command.

By default, there is one command included in here: \vn{`hook'}. This
is just a simple command that doesn't really do anything and is for
the purposes of demonstrating how a custom command would be
implemented.

The only thing needed to be called at the end of a custom command is
\vn{tao_cmd_end_calc} (as is done in the dumyt subroutine). This will
perform all of the steps listed in Section~\ref{s:lat_calc}. If this
isn't called than whatever is performed in the custom command will not
show up until a standard command is performed.

\subsection{tao\_hook\_does\_data\_exist}
\index{Customizing!tao_hook_does_data_exist}

This routine is for use with custom data types. 
This routine sets \vn{datum%exists} to true for datums with no
associated datum%ele_name. This is necessary since otherwise
\vn{tao_init_global_and_universes} will flag the datum as
not existing.

\subsection{tao\_hook\_evaluate\_a\_datum}
\index{Customizing!tao_hook_evaluate\_a\_datum}

Any custom data types are defined and calculated here. If a
non-standard data type is listed in the intialization files then a
corresponding data type must be placed in this routine. The tutorial
uses this hook routine when calculating the emittance.

\tao evaluates data at each element while each lattice is being
calculated. At initialization time \tao determines which datums are to
be evaluated at each element then calls \vn{tao_evaluate_a_datum} at
each element for each datum that needs to be evaluated. If a range of
elements are specified for a datum then \vn{tao_evaluate_a_datum} is
called for this datum at the last element in the
range. \vn{tao_evaluate_a_datum} starts by calling
\vn{tao_hook_avaluate_a_datum} to evaluate the custom data types.

As explained in the dummy file, the datum merit type affects how a
datum's value should be calculated. There is a helper subroutine in
the dummy hook routine called \vn{load_it} to aid in modifying the
datum's value based on the merit type. If there is a range of elements
associated with datum then a merit type other than \vn{target}
requires that the entire range of elements associated with the datum
be searched for the appropriate value to be returned.  See
Chapter~\ref{c:opti} for details on the merit type.

For example, if the data type is \vn{orbit.x} (yes, this is already
defined in \tao) then the appropriate case item in
\vn{tao_hook_load_data_array} would be:
\begin{example}
select case (datum%data_type)

case ('orbit.x')
  call load_it (orb(:)%vec(1))
\end{example}
\vn{load_it} will then look at each datum. If the merit type is other
than \vn{target} then \vn{load_it} will search the appropriate range
of elements for either the minimum or maximum horizontal orbit value
and this will be the datum's value. Because, a datum may refer to a
range of elements, the entire orbit array (from 0 to \vn{n_ele_max})
is passed to \vn{load_it} for each \vn{orbit.x} datum.

If the only merit type that is going to be used is \vn{target} then
\vn{load_it} can be ignored.

\subsection{tao\_hook\_graph\_data\_setup}
\index{Customizing!tao_hook_graph_data_setup}

Use this to setup custom graph data for a plot.

\subsection{tao\_hook\_init}\index{Customizing!tao_hook_init}

After the lattice and all global and universe structures are
intialized then \vn{tao_hook_init} is called. Here, any further
intializations can be added. In paricular, if any custom hook
structures need to be initialized, here's the place to do it.

\subsection{tao\_hook\_init\_design\_lattice}
\index{Customizing!tao_hook_init_design_lattice}

This will do a custom lattice initialialization. The standard lattice
initialization just calls \vn{bmad_parser} or \vn{xsif_parser}. If
anything more complex needs to be done then do it here. This is also
where any custom overlays or other elements would be inserted after
the parsing is complete. But in general, anything placed here should,
in principle, be something that can be placed in a lattice file.

\textbf{This is the only routine that should insert elements in the
ring}. This is beacuse the \tao data structures use the element index
for each element associated with the datum. If all the element indexes
shift then the data structures will break. If new elements need to be
inserted then modify this routine and recompile. You can alternatively
create a custom initialization file used by this routine that reads in
any elements to be inserted.

\subsection{tao\_hook\_lattice\_calc}
\index{Customizing!tao_hook_lattice_calc}

The standard lattice calculation can be performed for single particle,
particle beam or macroparticle tracking and will recalculate the
orbit, transfer matrices, twiss perameters and load the data
arrays. If something else needs to be performed whenever the lattice
is recalculated then it is placed here. A custom lattice calculation
can be performed on any lattice separately, this allows for the
possibility of, for example, tracking a single particle for one
lattice and macroparticles in another.

\subsection{tao\_hook\_merit\_data}
\index{Customizing!tao_hook_merit_data}

A custom data merit type can be defined here. Table~\ref{t:con_type}
lists the standard merit types. If a custom merit type is used then
\vn{load_it} in \vn{tao_hook_load_data_array} may also need to be
modified to handle this merit type, additionally, all standard data
types may need to be overridden in \vn{tao_hook_load_data_array} in
order for the custom \vn{load_it} to be used.  See
\vn{tao/code/tao_merit.f90} for how the standard merit types are
calculated.

\subsection{tao\_hook\_merit\_var}
\index{Customizing!tao_hook_merit_var}

This hook will allow for a custom variable merit type. However, since
there is no corresponding data transfer, no \vn{load_it} routine needs
to be modified.  See \vn{tao/code/tao_merit.f90} for how the standard
merit types are calculated.

\subsection{tao\_hook\_optimizer}
\index{Customizing!tao_hook_optimizer}

If a non standard optimizer is needed then it can be implemented
here. See the \vn{tao/code/tao_*_optimizer.f90} files for how the
standard optimizers are implemented.

\subsection{tao\_hook\_plot\_graph}
\index{Customizing!tao_hook_plot_graph}

This will customize the plotting of a graph. See the \tao module
\vn{tao_plot_mod} for details on what it normally done. You will also need to know how
DCSLIB's \vn{quick_plot} works.

\subsection{tao\_hook\_plot\_data\_setup}
\index{Customizing!tao_hook_plot_data_setup}

Use this routine to overide the \vn{tao_plot_data_setup} routine which
essentially transfers the information from the \vn{s%u(:)%data} arrays
to the \vn{s%plot_page%region(:)%plot%graph(:)%curve(:)} arrays. This
may be useful if you want to make a plot that isn't simply the
information in a data or variable array.

\subsection{tao\_hook\_post\_process\_data}
\index{Customizing!tao_hook_post_process_data}

Here can be placed anything that needs to be done after the data
arrays are loaded. This routine is called immediately after the data
arrays are called and before the optimizer or plotting is done, so any
final modifications to the lattice or data can be performed here.

\subsection{tao\_hook\_mod}\index{Customizing!tao_hook_mod}

Here any custom structures are defined. In the dummy hook routine
there are already a number of structures defined. These are used in
\vn{tao/code/tao_struct.f90} so that even if they are not used there
is a dummy structure defined so that the compiler doesn't complain.

\textbf{This module is different from all the other hook routines in
that the \tao library must be compiled after this file is modified so
that} \vn{tao_struct} \textbf{knows about the hook structure}. If hook
structures are needed but do not need to be accessed from any
\vn{tao_struct.f90} structures then it is recommended that these be
placed in a separate file so that it can be compiled when the custom
\tao program is compiled. Because of the special status of this hook
module, it is not placed in the standard hook directory but in the
\vn{tao/code/} directory.

%-----------------------------------------------------------------
%\chapter{Plotting}
%\label{s:prog_plotting} 

%\fbox{this chapter is yet to be completed!} 

