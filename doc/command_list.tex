\chapter{Tao Commands}
\label{c:commands}

Commands are case sensitive. The list of commands that \tao knows
about is shown in Table~\ref{t:commands}

\begin{table}[h]
\centering
{\tt
\begin{tabular}{|l|l||l|l|} \hline
  {\it Command} & {\it Section}  & {\it Command} & {\it Section} \\ \hline
  alias    & \ref{s:alias}    &  quit    & \ref{s:quit}    \\ \hline
  call     & \ref{s:call}     &  restore & \ref{s:restore} \\ \hline
  change   & \ref{s:change}   &  run     & \ref{s:run}     \\ \hline
  clip     & \ref{s:clip}     &  set     & \ref{s:set}     \\ \hline
  exit     & \ref{s:exit}     &  scale   & \ref{s:scale}   \\ \hline
  flatten  & \ref{s:flatten}  &  show    & \ref{s:show}    \\ \hline
  output   & \ref{s:output}   &  use     & \ref{s:use}     \\ \hline
  place    & \ref{s:place}    &  veto    & \ref{s:veto}    \\ \hline
  plot     & \ref{s:plot}     &  view    & \ref{s:view}    \\ \hline
\end{tabular}
}
\caption{Table of \bmad\ elements.}
\label{t:commands}\center
\end{table}

Where noted, a ``*'' star can be put in front of a command. This
signifies that the command is to be applied to all universes and not
just the one on view.

\vfil
\break

%------------------------------------------------------------------------
\section{Alias}
\label{s:alias}

Format: 
\begin{example}
  alias <alias_name> <string>
\end{example}

\vskip 0.2in

\vn{Alias} is like Unix aliases and allows the defining of alias
commands. When using an alias up to 9 arguments may be substituted in the
\vn{<string>}. The i\Th argument is substituted in place of the sub-string
``[i]''.

Examples:
\begin{example}
    alias xyzzy plot [1] model - base 
    xyzzy top 
\end{example}
In the above example ``xyzzy'' is the alias for the string ``plot [1]
model - base''.  When the command xyzzy is used ``top'' is substituted
for ``[1]'' in the string.

%------------------------------------------------------------------------
\section{Call}
\label{s:call}

Format: 
\begin{example}
  call <filename> \{<arg_list>\}  \Strut
\end{example}

\vskip 0.2in
\vn{call} opens a command file and executes the commands in it.  \tao
first looks in the current directory for the file. If not found \tao
will look in the directory pointed to by the \vn{TAO_COMMAND_DIR}
directory.  Up to 9 arguments may be passed to the command file. The
i\Th argument is substituted in place of the string ``[i]'' in the
file. 

Examples:
\begin{example}
    call my_cmd_file abc def 
\end{example}
In the above example the argument ``abc'' is substituted for any
``[1]'' appearing the the file and ``def'' is substituted for any
``[2]''.

%------------------------------------------------------------------------
\section{Change}
\label{s:change}

Format:
\begin{example}
  change ele <name> <attribute> <number>
  change var <name> <index> <number>
\end{example}

\vskip 0.2in
\vn{change} changes element attribute values or variable values. Generally
\vn{<number>} is added to the existing value
\begin{example}
  set_value = init_value + <number>
\end{example}
If "@" is prepended to \vn{<number>}  then just the value of \vn{<number>} is used
to set the value
\begin{example}
  set_value = <number>
\end{example}
If "d" is prepended to \vn{<number>}  then the value relative to the design
value is used:
\begin{example}
  set_value = design_value + <number>
\end{example}

Examples:
\begin{example}
  change ele s24e x_offset  0.1  ! Offset element by 0.1
  change ele q02w k1 d1.2e-2     ! set the k1 strength of q02w relative to the design
  change var steering 34 @1e-3  ! set the strength steering #34 to 0.001
\end{example}


%------------------------------------------------------------------------
\section{Clip}
\label{s:clip}

Format:
\begin{example}
  clip \{<where> <limit1> <limit2>\}
\end{example}

\vskip 0.2in \vn{clip} vetoes data points for plotting and
optimizing. The points vetoed are those points whose $y$ values are
outside a certain range defined by \vn{<limit1>} \vn{<limit2>}. If
neither \vn{<limit1>} nor \vn{<limit2>} is present then the clip range
is taken to be outside the graph minimum and maximum $y$--axis
values. If only \vn{<limit1>} is present then the clip range is
outside the range from -\vn{<limit1>} to +\vn{<limit1>}. If both are
present than the range is from \vn{<limit1>} to \vn{<limit2>}.  The
graphs that are clipped is determined by the \vn{<where>} switch.  If
\vn{<where>} is not present all graphs are scalled.

Examples
\begin{example}
  clip top:x -3  7  ! clip the x graph in the top region
  clip bottom       ! clip the graphs in the bottom region
\end{example}

%------------------------------------------------------------------------
\section{Exit}
\label{s:exit}

Format:
\begin{example}
  exit
\end{example}

\vskip 0.2in
\vn{Exit} exits the program. Same as \vn{Quit}.

%------------------------------------------------------------------------
\section{Flatten}
\label{s:flatten}

Format:
\begin{example}
  flatten <optimizer>
\end{example}

\vskip 0.2in
\vn{Flatten} runs the optimizer to minimize the merit function. This is the 
same as \vn{run}. See the \vn{run} command for more details.

%------------------------------------------------------------------------
\section{Output}
\label{s:output}

Format:
\begin{example}
  output hard
\end{example}

\vskip 0.2in
\vn{Output} creates a PostScript file of the plot window. 

%------------------------------------------------------------------------
\section{Place}
\label{s:place}

Format:
\begin{example}
  place <region> <template>
  place <region> none
\end{example}

\vskip 0.2in The \vn{place} command is used to associate a
\vn{<template>} plot with a \vn{<region>} and thus create a visible
plot in that region. To erase a plot from a region use the \vn{none}
switch. Notice that by using multiple \vn{place} commands a
\vn{template} can be associated with more than one region. 

Examples:
\begin{example}
  place top orbit  ! place the orbit template in the top region
  place top none   ! erase any plots in the top region
\end{example}

%------------------------------------------------------------------------
\section{Plot}
\label{s:plot}

Format:
\begin{example}
  plot <region> <who>
\end{example}

\vskip 0.2in The \vn{plot} command is used to determine who is plotted
in the graphs of a given region. Use a ``-'' for baselines. 

Examples:
\begin{example}
  plot bottom model - design       ! Plot model - design in the bottom region
  plot top data - model + design - ref 
\end{example}

%------------------------------------------------------------------------
\section{Quit}
\label{s:quit}

Format:
\begin{example}
  quit
\end{example}

\vskip 0.2in
\vn{Quit} exits the program. Same as \vn{exit}.

%------------------------------------------------------------------------
\section{Restore}
\label{s:restore}

Format:
\begin{example}
  \{*\}restore data  <class> <locations>
  restore var <class> <locations>
\end{example}

\vskip 0.2in The \vn{restore} command cancels data or variable
vetoes. See also the \vn{use}
and \vn{veto} commands.

Examples:
\begin{example}
  restore data orbit:x  23, 34:56 ! unveto orbit:x data.
  *restore data orbit 34          ! unveto orbit data in all universes.
  restore var quad_k1 67     ! unveto variable
\end{example}


%------------------------------------------------------------------------
\section{Run}
\label{s:run}

Format:
\begin{example}
  run <optimizer>
\end{example}

\vskip 0.2in

%------------------------------------------------------------------------
\section{Set}
\label{s:set}

Format:
\begin{example}
  \{*\}set data <class> <component> = <value> \{<locations\}
  \{*\}set var <class> <component> = <value> \{<locations\}
  set global <component> = <value>
\end{example}

\vskip 0.2in
A ``*'' star in front means that the \vn{set} is applied
to all universes and not just the one on view.

To apply a set to all data or variable classes use "all"
in place of \vn{<class>}.

For the \vn{set data} version of \vn{set} possible 
\vn{<component>}s that can be set are:
\begin{example}
  weight      ! Weight for the merit function.
  data        ! Measured data value.
  ref         ! Reference data value.
\end{example}
Besides a numeric value \vn{<value>} can be:
\begin{example}
  data        ! Measured data value.
  ref         ! Reference data value.
  model       ! model lattice value.
  base        ! base lattice value.
  design      ! design lattice value
\end{example}

For \vn{set var} version of \vn{set} possible 
\vn{<component>}s that can be set are:
\begin{example}
  weight     ! Weight for the merit function.
  step       ! Sets what a "small" variation of the variable is.
  model      ! Model lattice value.
  base       ! Base lattice value
\end{example}
Besides a numeric value \vn{<value>} can be:
\begin{example}
  data        ! Value when data measurement was made.
  ref         ! Value when reference measurement was made.
  model       ! model lattice value.
  base        ! base lattice value.
  design      ! design lattice value
\end{example}

For \vn{set global} version of set possible 
\vn{<components>}s that can be set are:
\begin{example}
  opt_with_ref  = T/F ! Use the reference data with optimization.
  opt_with_base = T/F ! Use the base lattice with optimization.
\end{example}

Examples:
\begin{example}
  set data all ref = data       ! set all ref data = measured data.
  *set var quad_k1 weight = 0.1 ! set quad_k1 weights. 
\end{example}

%------------------------------------------------------------------------
\section{Scale}
\label{s:scale}

Format:
\begin{example}
  scale \{<where> <value1> <value2>\}
\end{example}

\vskip 0.2in \vn{scale} adjusts the vertical scale of graphs. If
neither \vn{<value1>} nor \vn{<value2>} is present then the scale is
adjusted so that all the data points are within the graph region.  If
only \vn{<value1>} is present then the scale is taken to be from
-\vn{<value1>} to +\vn{<value1>}. If both are present than the scale
is from \vn{<value1>} to \vn{<value2>}.  The graphs that are scalled
is determined by the \vn{<where>} switch. If \vn{<where>} is not
present  or \vn{<where>} is ``all'' then all graphs are scalled. 

Examples:
\begin{example}
  scale top:x -3  7  ! scale the x graph in the top region
  scale bottom       ! scale the graphs in the bottom region
  scale              ! scale everything
\end{example}


%------------------------------------------------------------------------
\section{Show}
\label{s:show}

Format:
\begin{example}
  show alias                     
  show data \{<class> <number>\} 
  show ele <ele_name>
  show global
  show lattice <ele1_name> <ele2_name>
  show optimizer
  show top10
  show var <class> \{<number>\}
\end{example}

\vskip 0.2in
\vn{Show} is used to display various information about the state of \tao.
\begin{description}
\item[show alias]
Shows a list of defined aliases. See the \vn{alias} command for more details.
\item[show data]
Shows data information. If \vn{<class>} and \vn{<number>} are not present shown 
is a list of d2\_data classes.
\item[show ele]
This shows information on lattice elements. \vn{<ele_name>} can contain wild--card
characters ``\%'' and ``*''. ``\%'' matches to any single character and
``*'' matches to any number of characters. If \vn{<ele_name>} contains a
wild card then a list of elements that match the name are shown. If no wild--card
is present then information about the element whose name matches \vn{<ele_name>}
is shown. \vn{<ele_name>} is a number $n$ then the $n$\Th element in the lattice 
list will be shown.
\item[show global]
Shows global information.
\item[show lattice]
Shows Twiss and orbit data for the \vn{model} lattice between elements
\vn{<ele1_name>} and \vn{<ele2_name>}. Either \vn{<ele1_name>} or
\vn{<ele2_name>} may be numbers indicating the place in the lattice
list to start or stop at.
\item[show optimizer]
Shows information pertinent to optimization: Data and variables used, etc.
\item[show top10]
Shows top contributors to the merit function, dMerit/dVariable
derivatives, and Largest changes in variable value.
\item[show var]
Shows variable information. If \vn{<class>} and \vn{<number>} are not present shown 
is a list of v1\_var classes.
\end{description}

Examples:
\begin{example}
  show data
  show data orbit
  show data orbit:x 35
  show ele q*          ! list all lattice elements with names beginning with "q".
  show ele q10w        ! show a praticular lattice element
  show ele 105         ! show element #105 in the lattice
\end{example}

%------------------------------------------------------------------------
\section{Use}
\label{s:use}

Format:
\begin{example}
  \{*\}use data  <class> <locations>
  use var <class> <locations>
\end{example}

\vskip 0.2in The \vn{use} command  unvetoes data or variables
and sets a veto for the rest of the data.
A ``*'' star in front means that the \vn{use} is applied
to all universes and not just the one on view. See also the \vn{restore}
and \vn{veto} commands.

Examples:
\begin{example}
  use data orbit:x  23, 34:56 ! use orbit:x data.
  *use data orbit 34          ! use orbit data in all universes.
  use var quad_k1 67          ! use variable
\end{example}


%------------------------------------------------------------------------
\section{Veto}
\label{s:veto}

Format:
\begin{example}
  \{*\}veto data  <class> <locations>
  veto var <class> <locations>
\end{example}

\vskip 0.2in The \vn{veto} command  vetoes data or variables.
A ``*'' star in front means that the \vn{veto} is applied
to all universes and not just the one on view. See also the \vn{restore}
and \vn{use} commands.

Examples:
\begin{example}
  veto data orbit:x  23, 34:56 ! veto orbit:x data.
  *veto data orbit 34          ! veto orbit data in all universes.
  veto var quad_k1 67     ! veto variable
\end{example}

%------------------------------------------------------------------------
\section{View}
\label{s:view}

Format:
\begin{example}
  view <number>
\end{example}

\vskip 0.2in The \vn{view} command changes which universe data is
taken from from plotting.  This also sets the default universe that
commands are applied to in the absense of a star prefix.  

Examples:
\begin{example}
  view 2   ! Make universe #2 the default.
\end{example}
