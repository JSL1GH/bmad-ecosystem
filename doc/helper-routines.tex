\chapter{Helper Routines}
\label{c:helper}

This chapter gives an overview of various computational helper routines.

%-----------------------------------------------------------------------------
\section{Nonlinear Optimization}
\label{s:opti}
\index{optimizers}

\index{numerical recipes!library}
\index{tao}
Nonlinear optimization is the process of finding a minimum (or
maximum) of a nonlinear function (the "merit" function). Nonlinear
optimization is frequently used for lattice design or matching of data
to a model. For more information on this see the \tao manual.

In terms of routines for implementing nonlinear optimization the
Numerical Recipes library (\sref{s:libs} that is distributed along
with \bmad contains several. In particular, 
the routine \hyperref[r:super.mrqmin]{super_mrqmin}
which implements the Levenberg--Marquardt is an excellent routine for
finding local minimum when the merit function can be expressed as the
sum of quadratic terms. Another routine, \vn{frprmn}, which is an
implementation of the Fletcher--Reeves algorithm, is also good at
finding local minimum and has the advantage that as input it does not
need a derivative matrix as does Levenberg--Marquardt. The
disadvantage of Fletcher--Reeves is that it is slower than
Levenberg--Marquardt. 

A second implementation of Levenberg--Marquardt available with \bmad
is \hyperref[r:opti.lmdif]{opti_lmdif} which is Fortran90 version of the popular
\vn{lmdif} routine. Also available is \hyperref[r:opti.de]{opti_de} which implements
the Differential Evolution algorithm of Storn and
Price\cite{b:storn}. This routine is good for finding global minima
but can be slow. 

Another routine that should be mentioned is the \vn{amoeba} routine
from Numerical Recipes that implements the downhill simplex method of
Neider and Mead. This routine is robust but slow but is easily
parallelized so it is a good routine for parallel processing.

%-----------------------------------------------------------------------------
\section{Matrix Manipulation}
\label{s:matrix}
\index{matrix manipulation}

There are a number of \bmad routines for matrix manipulation as listed
in \sref{r:mat}. In fact, Fortran90 has a number of intrinsic matrix
routines as well but this is outside the scope of this manual. The
following example shows some of the \bmad matrix routines
\index{routine!mat_det}
\index{routine!mat_make_unit}
\begin{example}
  real(rp) mat(6,6), mat_inv(6,6)
  call mat_make_unit (mat)    ! make a unit matrix
  call mat_inv (mat, mat_inv) ! Compute the inverse matrix.
\end{example}