\chapter{Data types}
\label{c:data_types}

%------------------------------------------------------------------------
\section{How \tao Handles Data}

As explained in chapter~\ref{c:overview}, \tao has special structures to hold
data to be analyzed. Any lattice or beam parameters that are needed to be
analyzed or plotted must have a data type defined for it. In general, a data
type can be any type of parameter that is not necessarily measurable in a real
world machine. For example, there is an orbit data type and a BPM data type. The
orbit data is the real x,y,z position of a particle in the lab frame whereas the BPM
data is the x and y reading on a simulated beam position monitor that includes
any offsets and tilts.

\tao includes many different data types already predefined and
these can be classified into three main catagories:
\begin{enumerate}
  \item \textbf{Lattice Parameters} \Newline
    For example lattice twiss parameters, coupling and floor position.
  \item \textbf{Single Particle Properties} \Newline
    For example particle orbit, BPM reading and phase advance.
  \item \textbf{Bunch Properties} \Newline
    For example bunch sigmas, emittance and beam twiss parameters.
\end{enumerate}
For a given data type the method used to calculate the datum can vary depending
on the tracking type. For {\it single} particle tracking all data is found from
the lattice parameters. For \textit{particle bunch} or \textit{macroparticle
bunch} tracking some of the datums  are found from the
particle distribution using the appropriate \bmad routines. For example, with
single particle tracking the beta function is found from the lattice transport
matrix, however, with particle or macroparticle tracking the beta function is
found from the particle distribution using the formula
\begin{equation}
  \beta = \frac{<x^{2}>}{\sqrt{<x^{2}> <x'^{2}> - <x x'>^{2}}}.
\end{equation}
The next section will describe how each predefined data type is calculated.
Custom data types need not fall into one of the above catagories and can be any
real number as calculated in the appropriate hook routine.

Also associated with a datum are one or two lattice elements called
\vn{ele} and \vn{ele2}. The \vn{data_types} are divided into two
categories: Those that are \vn{relative} and those who are not.  A
\vn{relative} \vn{data_type} means that the \vn{model} value for that
datum is determined by a difference between elements. For example, for
\vn{phase:x} the \vn{model} value is
\begin{example}
  model_value = \(\phi\sb{x}\)(ele) - \(\phi\sb{x}\)(ele2)
\end{example}
If there is no \vn{ele2} associated with a datum then the model value is
\begin{example}
  model_value = \(\phi\sb{x}\)(ele) - \(\phi\sb{x}\)(0)
\end{example}
where $\phi_x(0)$ is the phase at the 0\Th element (which is always 0).

For datums with \vn{non-relative} \vn{data_types} if there is also an
associated \vn{ele2} element then the \vn{model} value is dependent
upon the \vn{merit_type}. For example, with a \vn{beta:x} \vn{data_type} the
model value is determined by Table~\ref{t:eval2} where \vn{i} goes from the
\vn{ele} index to the \vn{ele2} index.
\begin{table}[ht]
\centering
{\tt
\begin{tabular}{|l|l|l|} \hline
  {\it Merit\_Type}       & {\it Model Value} \\ \hline 
  \vn{min}     & $\min \beta_x(i)$ \\ \hline 
  \vn{max}     & $\min \beta_x(i)$ \\ \hline 
  \vn{abs_min} & $\min |\beta_x(i)|$ \\ \hline 
  \vn{abs_max} & $\min |\beta_x(i)|$ \\ \hline 
  \vn{target}  & {\it Error}   \\ \hline 
\end{tabular}
}
\caption{\vn{Model} evaluation.}
\label{t:eval2}
\end{table}

%------------------------------------------------------------------------
\section{\tao Data Types}

The predefined \vn{data types} are given in table~\ref{t:data_types}.

Table~\ref{t:data_calc_method} lists how each data type is calculated depending
on the tracking type.
\textbf{More to be written here!}

\begin{table}[ht] 
\centering 
{\tt
\begin{tabular}{|l|l|l|} \hline
  {\it Data\_Type}        & {\it Description}                 &          \\ \hline 
    beta:x, beta:y        & Twiss parameter                   &          \\ \hline 
    alpha:x, alpha:y      & Twiss parameter                   &          \\ \hline 
    eta:x, eta:y          & Dispersion                        &          \\ \hline 
    etap:x, etap:y        & Dispersion derivative             &          \\ \hline 
    phase:x, phase:y      & Betatron phase                    & relative \\ \hline 
    orbit:x, orbit:y      & Transverse orbit                  &          \\ \hline 
    orbit:p\_x, orbit:p\_y& Tranverse momenta                 &          \\ \hline 
    orbit:z, orbit:z\_p   & Longitudinal orbit and momenta    &          \\ \hline 
    bpm:x, bpm:y          & Transverse BPM reading            &          \\ \hline 
    \begin{tabular}{@{}l}     
      cbar:11, cbar:12, \\ 
      cbar:21, cbar:22 
    \end{tabular} 
                          & Coupling                         &          \\ \hline 
    \begin{tabular}{@{}l}   
      coupling:11b, coupling:12a, \\ 
      coupling:12b, coupling:22a 
    \end{tabular} 
                          & Coupling                         &          \\ \hline 
    floor:x, floor:y, floor:z
                          & Global (``floor'') position      & relative \\ \hline 
    floor:theta           & Global (``floor'') angle         & relative \\ \hline 
    r:$ij$                & 
                           \begin{tabular}{l}
                             Term in linear transfer map \\
                             $1 \le i,j \le 6$
                           \end{tabular}
                                                             & relative \\ \hline 
    t:$ijk$               & 
                           \begin{tabular}{l}
                             Term in 2\Nd order transfer map \\
                              $1 \le i,j,k \le 6$
                           \end{tabular} 
                                                             & relative \\ \hline 
    tt:$ijklm\ldots$      & 
                           \begin{tabular}{l}
                             Term in n\Th order transfer map \\
                              $1 \le i,j,k,\ldots \le 6$
                           \end{tabular} 
                                                              & relative \\ \hline 
    i5a\_e6, i5b\_e6      & Normalized I5 radiation integral  &          \\ \hline
    s\_position           & longitudinal length constraint    & relative \\ \hline 
    beam\_energy          & Beam energy                       &          \\ \hline
    \begin{tabular}{@{}l}  
      norm\_emittance:x \\ 
      norm\_emittance:y 
    \end{tabular} 
                          & Normalized emittance              &          \\ \hline 
    \begin{tabular}{@{}l}   
      bunch\_sigma:x, bunch\_sigma:p\_x \\ 
      bunch\_sigma:y, bunch\_sigma:p\_y \\
      bunch\_sigma:z, bunch\_sigma:p\_z \\
    \end{tabular} 
                          & Bunch size                        &          \\ \hline 
\end{tabular}
} 
\caption{Predefined Data Types}
\label{t:data_types}
\end{table}

\vfill \break
{\vfill}

\index{Element!table of data calculation methods}
\begin{table}[ht] 
\centering {
\begin{tabular}{|l|c|c|c|} \hline
\rule{0pt}{80pt} 
\vn{Data_Type}       &
\begin{sideways}\vn{Single}\end{sideways} &
\begin{sideways}\vn{Particle Beam}\end{sideways} &
\begin{sideways}\vn{Macroparticle Beam}\end{sideways} 
\\ \hline
%   data type             & Single & Beam & Macroparticle
    \vn{beta}             &   L    &   B  &   B   \\ \hline
    \vn{alpha}            &   L    &   B  &   B   \\ \hline
    \vn{eta}              &   L    &   B  &   B   \\ \hline
    \vn{etap}             &   L    &   B  &   B   \\ \hline
    \vn{phase}            &   L    &   L  &   L   \\ \hline
    \vn{orbit}            &   B    &   B  &   B   \\ \hline
    \vn{bpm}              &   B    &   B  &   B   \\ \hline
    \vn{cbar}             &   L    &   L  &   L   \\ \hline
    \vn{coupling}         &   L    &   L  &   L   \\ \hline
    \vn{floor}            &   L    &   L  &   L   \\ \hline
    \vn{r}                &   L    &   L  &   L   \\ \hline
    \vn{t}                &   L    &   L  &   L   \\ \hline
    \vn{tt}               &   L    &   L  &   L   \\ \hline
    \vn{i5a_e6, i5b_e6}   &   L    &   L  &   L   \\ \hline      
    \vn{s_position}       &   L    &   L  &   L   \\ \hline           
    \vn{beam_energy}      &   L    &   L  &   L   \\ \hline  
    \vn{norm_emittance}   &   L\footnote{Uses the Courant-Snyder Invariant}    &   B  &   B   \\ \hline
    \vn{bunch_sigma}      &   X    &   B  &   B   \\ \hline
\end{tabular}
}
\caption{Data\_type Calculation method. L means this data type uses lattice parameters, B
means this data type uses beam parameters. X mean this data is unavailable for
this tracking type.}
\label{t:data_calc_method}
\end{table}

\vfill\break

