\chapter{Parameter Structures}
\label{c:param.structs}

A ``structure'' is a collection of parameters.  \bmad has various
structures which can be used for various tasks. For example, the
\vn{beam_init_struct} structure (\sref{s:beam.init}) is used to set
parameters used to initialize particle beams.

A given program may give the user access to some of these structures
so, in order to allow intelligent parameter setting, this chapter
gives an in-depth description of the most common ones.

Each structure has a \vn{``structure name''} (also called a \vn{``type
name''}) which identifies the list of parameters (also called
``components'') in the structure. Associated with a structure there
will be an \vn{``instance''} of this structure and this instance will
have an \vn{``instance name''} which is what the user uses to set
parameters. It is possible to have multiple instances of a
structure. For example, in the situation where a program is simulating
multiple particle beams, there could be multiple \vn{beam_init_struct}
(\sref{s:beam.init}) instances with one for each beam.

\bmad defines uses some structures to hold global parameters. That is,
parameters that shared by all code. Each of these structures has
a single associated instance. These are:
\begin{center}
  \begin{tabular}{ll} \toprule
  Structure           & Instance    \\
  \midrule
  bmad_common_stuct   & bmad_com    \\
  csr_parameter_stuct & csr_param   \\
  \bottomrule
  \end{tabular}
\end{center}
All other structures will have instance names that are program
specific.  That is, see the program documentation for the instance
name(s) used.

To set a particular component of an instance use the syntax
\begin{example}
  instance_name%parameter_name = value
\end{example}
Example:
\begin{example}
  bmad_com%max_aperture_limit = 10
\end{example}
this sets the \vn{max_aperture_limit} parameter of \vn{bmad_com}.

%-----------------------------------------------------------------
\section{Bmad_Common_Struct}
\label{s:bmad.common}
\index{Bmad!general parameters|hyperbf}

The \vn{bmad_common_struct} structure contains a set of global parameters. There is only
one instance (\sref{c:param.structs}) of this structure and this instance has the name
\vn{bmad_com}. The components of this structure along with the default values are:
\index{max_aperture_limit}\index{d_orb(6)}
\index{grad_loss_sr_wake}\index{default_ds_step}
\index{rel_tol_tracking}\index{abs_tol_tracking}
\index{rel_tol_adaptive_tracking}\index{abs_tol_adaptive_tracking}
\index{taylor_order}\index{default_integ_order}
\index{sr_wakes_on}\index{lr_wakes_on}
\index{mat6_track_symmetric}\index{auto_bookkeeper}
\index{space_charge_on}\index{coherent_synch_rad_on}
\index{spin_tracking_on}\index{radiation_damping_on}
\index{radiation_fluctuations_on}\index{be_thread_safe}
\index{absolute_time_tracking_default}
\index{bmad_common_struct|hyperbf}
\begin{example}
  type bmad_common_struct
    real(rp) max_aperture_limit = 1e3          ! Max Aperture.
    real(rp) d_orb(6)           = 1e-5         ! for the make_mat6_tracking routine.
    real(rp) default_ds_step    = 0.2          ! Integration step size.  
    real(rp) significant_length = 1e-10        ! meter 
    real(rp) rel_tol_tracking = 1e-8           ! Closed orbit relative tolerance.
    real(rp) abs_tol_tracking = 1e-10          ! Closed orbit absolute tolerance.
    real(rp) rel_tol_adaptive_tracking = 1e-8  ! Runge-Kutta tracking relative tolerance.
    real(rp) abs_tol_adaptive_tracking = 1e-10 ! Runge-Kutta tracking absolute tolerance.
    real(rp) init_ds_adaptive_tracking = 1e-3  ! Initial step size.
    real(rp) min_ds_adaptive_tracking = 0      ! Minimum step size to use.
    real(rp) fatal_ds_adaptive_tracking = 1e-8 ! Threshold for loosing particles.
    real(rp) autoscale_amp_abs_tol = 0.1_rp    ! Autoscale absolute amplitude tolerance (eV).
    real(rp) autoscale_amp_rel_tol = 1d-6      ! Autoscale relative amplitude tolerance
    real(rp) autoscale_phase_tol = 1d-5        ! Autoscale phase tolerance.
    real(rp) electric_dipole_moment = 0        ! Particle's EDM. Call set_ptc to transfer value to PTC.
    real(rp) ptc_cut_factor = 0.006            ! Cut factor for PTC tracking
    real(rp) sad_eps_scale = 5.0d-3            ! Used in sad_mult step length calc.
    real(rp) sad_amp_max = 5.0d-2              ! Used in sad_mult step length calc.
    integer sad_n_div_max = 1000               ! Used in sad_mult step length calc.
    integer taylor_order = 3                   ! 3rd order is default
    integer default_integ_order = 2            ! PTC integration order
    integer ptc_max_fringe_order = 2           ! PTC max fringe order (2 => Quadrupole !).
    integer max_num_runge_kutta_step = 10000   ! Max num RK steps before particle is lost.
    logical use_hard_edge_drifts = T           ! Insert drifts when tracking through cavity?
    logical sr_wakes_on = T                    ! Short range wake fields?
    logical lr_wakes_on = T                    ! Long range wake fields
    logical mat6_track_symmetric = T           ! symmetric offsets
    logical auto_bookkeeper = T                ! Automatic bookkeeping?
    logical space_charge_on = F                ! Space charge switch
    logical coherent_synch_rad_on = F          ! csr 
    logical spin_tracking_on = F               ! spin tracking?
    logical radiation_damping_on = F           ! Damping toggle.
    logical radiation_fluctuations_on = F      ! Fluctuations toggle.
    logical conserve_taylor_maps = T           ! Enable bookkeeper to set
                                               ! ele%taylor_map_includes_offsets = F?
    logical absolute_time_tracking_default = F ! Default for lat%absolute_time_tracking
    logical aperture_limit_on = T              ! Use aperture limits in tracking.
    logical debug = F                          ! Used for code debugging.
  end type
\end{example}

Note: \vn{bmad_com} parameters may always be set in a lattice file as
discussed in Section~\sref{s:bmad.com}. However, thought must be given
to setting \vn{bmad_com} parameters in a lattice file since that will
affect every program that uses the lattice.

Parameter description:
\begin{description}
\index{aperture_limit_on}
\item[\vn{\%abs_tol_adaptive_tracking}] \Newline
Absolute tolerance to use in adaptive tracking. This is used in
\vn{runge-kutta} and \vn{time_runge_kutta} tracking (\sref{s:integ}).

\item[\vn{\%abs_tol_tracking}] \Newline
Absolute tolerance to use in tracking. Specifically, Tolerance to use
when finding the closed orbit.

\item[\vn{\%aperture_limit_on]}] \Newline
Aperture limits may be set for elements in the lattice
(\sref{s:limit}). Setting \vn{aperture_limit_on} to \vn{False} will
disable all set apertures. \vn{True} is the default.

\item[\vn{\%auto_bookkeeper}] \Newline
Toggles automatic or intelligent bookkeeping. See
section~\sref{s:lat.bookkeeping} for more details.

\item[\vn{\%autoscale_amp_abs_tol}] \Newline
Used when \bmad autoscales (\sref{s:autoscale}) an elements field amplitude. This parameter sets the
absolute tolerance for the autoscale amplitude parameter.

\item[\vn{\%autoscale_amp_rel_tol}] \Newline
Used when \bmad autoscales (\sref{s:autoscale}) an elements field amplitude. This parameter sets the
relative tolerance for the autoscale amplitude parameter.

\item[\vn{\%autoscale_phase_tol}] \Newline
Used when \bmad autoscales (\sref{s:autoscale}) an elements AC phase. This parameter sets the
absolute tolerance for the autoscale parameter.

\item[\vn{\%d_orb}] \Newline 
Sets the orbit displacement used in the routine that calculates the transfer matrix through an
element via tracking. That is, when the \vn{mat6_calc_method} (\sref{s:xfer}) is set to
\vn{tracking}. \vn{%d_orb} needs to be large enough to avoid significant round-off errors but not so
large that nonlinearities will affect the results. The default value is $10^{-5}$. Also see
\vn{%mat6_track_symmetric}.

\item[\vn{\%default_ds_step}] \Newline
Step size for tracking code \sref{c:methods} that uses a fixed step
size. For example, \vn{symp_lie_ptc} and \vn{boris} tracking.

\item[\vn{\%default_integ_order}] \Newline
Order of the the integrator used by \'Etienne Forest's PTC code (\sref{s:libs}).

\item[\vn{\%electric_dipole_moment}] \Newline
The electric dipole moment value used in tracking a particle's spin (\sref{s:spin.dyn}).

\item[\vn{\%max_aperture_limit}] \Newline 
Sets the maximum amplitude a particle can have
during tracking. If this amplitude is exceeded, the particle is lost
even if there is no element aperture set. Having a maximum aperture
limit helps prevent numerical overflow in the tracking calculations.

\item[\vn{\%max_num_runge_kutta_step}] \Newline 
The maximum number of steps to take through an element with \vn{runge_kutta} or
\vn{time_runge_kutta} tracking. The default value is 10,000. If the number of steps
reaches this value, the particle being tracked is marked as lost and a warning message is
issued. Under ``normal'' circumstances, a particle will take far fewer steps to track
through an element. If a particle is not through an element after 10,000 steps, it
generally indicates that there is a problem with how the field is defined. That is, the
field does not obey Maxwell's Equations. Especially: discontinuities in the field can
cause problems.

\item[\vn{\%rel_tol_adaptive_tracking}] \Newline
Relative tolerance to use in adaptive tracking. This is used in
\vn{runge_kutta} and \vn{time_runge_kutta} tracking (\sref{s:integ}).

\item[\vn{\%init_ds_adaptive_tracking}] \Newline
Initial step to use for adaptive tracking. This is used in
\vn{runge-kutta} and \vn{time_runge_kutta} tracking (\sref{s:integ}).

\item[\vn{\%min_ds_adaptive_tracking}] \Newline
This is used in \vn{runge-kutta} and \vn{time_runge_kutta} tracking
(\sref{s:integ}). Minimum step size to use for adaptive tracking. If
To be useful, \vn{%min_ds_adaptive_tracking} must be set larger than
the value of \vn{%fatal_ds_adaptive_tracking}. In this case,
particles are never lost due to taking too small a step.

\item[\vn{\%fatal_ds_adaptive_tracking}] \Newline
This is used in \vn{runge-kutta} and \vn{time_runge_kutta} tracking
(\sref{s:integ}).  If the step size falls below the value set for
\vn{%fatal_ds_adaptive_tracking}, a particle is considered lost.
This prevents a program from ``hanging'' due to taking a large number
of extremely small steps. The most common cause of small step size is
an ``unphysical'' magnetic or electric field.

\item[\vn{\%rel_tol_tracking}] \Newline
Relative tolerance to use in tracking. Specifically, Tolerance to use
when finding the closed orbit.

\item[\vn{\%default_integ_order}] \Newline
Order of the the integrator used by \'Etienne Forest's PTC code (\sref{s:libs}).
The order of the PTC integrator is like the order of a Newton-Cotes method.
Higher order means the error term involves a higher order derivative of the field.

\item{ptc_max_fringe_order} \Newline
Maximum order for computing fringe field effects in PTC. 

\item[\vn{\%significant_length}] \Newline
Sets the scale to decide if two length values are significantly
different. For example, The superposition code will not create any
super_slave elements that have a length less then this.

\item[\vn{\%sr_wakes_on}] \Newline
Toggle for turning on or off short-range higher order mode wake field effects.

\item[\vn{\%lr_wakes_on}] \Newline
Toggle for turning on or off long-range higher order mode wake field effects.

\item[\vn{\%mat6_track_symmetric}] \Newline
Toggle to turn off whether the transfer matrix from tracking routine
(\Hyperref{r:twiss.from.tracking}{twiss_from_tracking}) tracks 12 particles at both plus and
minus \vn{%d_orb} values or only tracks 7 particles to save time but is less accurate..

\item[\vn{\%space_charge_on}] \Newline
Toggle to turn on or off the high energy space charge effect in particle tracking
(\sref{s:space.charge}). This is not to be confused with the space charge component of
coherent synchrotron radiation (\sref{s:csr}).

\item[\vn{\%coherent_synch_rad_on}] \Newline
Toggle to turn on or off the coherent space charge calculation.

\item[\vn{\%spin_tracking_on}] \Newline
Determines if spin tracking is performed or not.

\item[\vn{\%taylor_order}] \Newline
Cutoff Taylor order of maps produced by \vn{sym_lie_ptc}.

\item[\vn{\%radiation_damping_on}] \Newline
Toggle to turn on or off effects due to radiation damping in particle tracking.

\item[\vn{\%radiation_fluctuations_on}] \Newline
Toggle to turn on or off effects due to radiation fluctuations in particle tracking.

\item[\vn{\%conserve_taylor_maps}] \Newline
Toggle to determine if the Taylor map for an element include any
element ``misalignments''.  See Section~\sref{s:mapoff} for more
details.

\item[\vn{\%absolute_time_tracking_default}] \Newline
Default setting to be applied to a lattice if
\vn{absolute_time_tracking} (\sref{s:param}) is not specified in a
lattice file. Additionally, if an element that is not associated with a lattice
is tracked, \vn{%absolute_time_tracking_default} will be used to
determine whether absolute time tracking is used. 

To change between absolute and relative time tracking
(\sref{s:rf.time}) after lattice file parsing, the
\vn{%absolute_time_tracking} component of a \vn{lat_struct}
(\sref{s:abs.time}) can be appropriately set.

\item[\vn{\%debug}] \Newline
Used for communication between program units for debugging purposes.

\end{description}

%-----------------------------------------------------------------
\section{Bmad_Com}
\label{s:bmad.com}

\index{bmad_com}
The parameters of the \vn{bmad_com} instance of the
\vn{bmad_common_struct} structure (\sref{s:bmad.common}) can be set in
the lattice file using the syntax
\begin{example}
  bmad_com[parm-name] = value
\end{example}
where \vn{parm-name} is the name of a component of
\vn{bmad_common_struct}. For example:
\begin{example}
  bmad_com[rel_tol_tracking] = 1e-7
\end{example}

Be aware that setting a \vn{bmad_com} parameter value in a lattice
file will affect all computations of a program even if the program
reads in additional lattice files. That is, setting of \vn{bmad_com}
components is ``sticky'' and persists even when other lattice files
are read in. There are two exceptions: A program is always free to
override settings of \vn{bmad_com} parameters. Additionally, a second
lattice file can also override the setting made in a prior lattice
file.
%-----------------------------------------------------------------
\section{Beam_Init_Struct}
\label{s:beam.init}
\index{beam initialization parameters|hyperbf}

\index{beam_init_struct|hyperbf}
Beams of particles are used for simulating inter-bunch intra-bunch
effects. The \vn{beam_init_struct} structure holds parameters which
are used to initialize a beam. The The parameters of this structure
are:
\begin{example}
  type beam_init_struct
    character distribution_type(3)         ! "ELLIPSE", "KV", "GRID", "" (default).
    type (ellipse_beam_init_struct) ellipse(3) ! For ellipse beam distribution
    type (kv_beam_init_struct) KV              ! For KV beam distribution
    type (grid_beam_init_struct) grid(3)       ! For grid beam distribution
    !!! The following are for Random distributions
    character random_engine          ! "pseudo" (default) or "quasi". 
    character random_gauss_converter ! "exact" (default) or "quick". 
    real random_sigma_cutoff = -1    ! -1 => no cutoff used.
    real center_jitter(6) = 0.0      ! Bunch center rms jitter
    real emit_jitter(2)   = 0.0      ! %RMS a and b mode bunch emittance jitter
    real sig_z_jitter     = 0.0      ! bunch length RMS jitter 
    real sig_e_jitter     = 0.0      ! energy spread RMS jitter 
    integer n_particle = 0               ! Number of simulated particles per bunch.
    logical renorm_center = T            ! Renormalize centroid?
    logical renorm_sigma = T             ! Renormalize sigma?
    !!! The following are used  by all distribution types
    real(rp) spin(3)                 ! Spin (x, y, z)
    real a_norm_emit                 ! a-mode normalized emittance (= \(\gamma\,\epsilon\))
    real b_norm_emit                 ! b-mode normalized emittance (= \(\gamma\,\epsilon\))
    real a_emit                      ! a-mode emittance (= \(\gamma\,\epsilon\))
    real b_emit                      ! b-mode emittance (= \(\gamma\,\epsilon\))
    real dPz_dz = 0                  ! Correlation of Pz with long position.
    real center(6) = 0               ! Bench center offset.
    real dt_bunch                    ! Time between bunches.
    real sig_z                       ! Z sigma in m.
    real sig_e                       ! dE/E (pz) sigma.
    real bunch_charge                ! Charge in a bunch.
    integer n_bunch = 1                  ! Number of bunches.
    character species                    ! Species. Default is reference particle.
    logical full_6D_coupling_calc = F    ! Use 6x6 1-turn matrix to match distribution?  
    logical use_t_coords = F        ! If true, the distributions will be 
                                    !   calculated using time coordinates  
    logical use_z_as_t   = F        ! Only used if  use_t_coords = T
                                    !   If True,  particles will be distributed in t
                                    !   If False, particles will be distributed in s
  end type
\end{example}
Note: The \vn{z} coordinate value given to particles of a bunch is with respect to the
nominal center of the bunch. Therefore, if there are multiple bunches, and there is an RF
cavity whose frequency is not commensurate with the spacing between bunches, absolute time
tracking (\sref{s:rf.time}) must be used.


  \begin{description}
  \item[\%a_emit, \%b_emit, \%a_norm_emit, \%b_norm_emit] \Newline
Normalized and unnormalized emittances. Either \vn{a_norm_emit} or \vn{a_emit}
may be set but not both. similarly, either \vn{b_norm_emit} or
\vn{b_emit} may be set but not both.
  \item[\%bunch_charge] \Newline
  \item[\%center(6)] \Newline
  \item[\%center_jitter, \%emit_jitter, \%sig_z_jitter, \%sig_e_jitter] \Newline
These components can be used to provide a bunch-to-bunch 
random variation in the emittance and bunch center.
  \item[\%distribution_type(3)] \Newline
The \vn{%distributeion_type(:)} array determines what algorithms are used to generate
the particle distribution for a bunch. \vn{%distributeion_type(1)} sets the distribution 
type for the $(x, p_x)$ 2D phase space, etc. 
Possibilities for \vn{%distributeion_type(:)} are:
\begin{example}
  "", or "RAN_GAUSS"  ! Random distribution (default).
  "ELLIPSE"           ! Ellipse distribution (\sref{s:ellipse.init})
  "KV"                ! Kapchinsky-Vladimirsky distribution (\sref{s:kv.init})
  "GRID"              ! Uniform distribution.
\end{example}
Since the Kapchinsky-Vladimirsky distribution is for a 4D
phase space, if the Kapchinsky-Vladimirsky distribution is used,
\vn{"KV"} must appear exactly twice in the \vn{%distributeion_type(:)}
array. 

Unlike all other distribution types, the \vn{GRID} distribution is independent of the
Twiss parameters at the point of generation.  For the non-\vn{GRID} distributions, the
distributions are adjusted if there is local $x$-$y$ coupling (\sref{s:coupling}). For
lattices with a closed geometry, if \vn{full_6D_coupling_calc} is set to \vn{True}, the
full 6-dimensional coupling matrix is used. If \vn{False}, which is the default, The
4-dimensional $\bfV$ matrix of \Eq{vgicc1} is used.

Note: The total number particles generated is the product of the individual
distributions. For example:
\begin{example}
  type (beam_init_struct) bi
  bi%distribution_type = ELLIPSE", "ELLIPSE", "GRID"
  bi%ellipse(1)%n_ellipse = 4
  bi%ellipse(1)%part_per_ellipse = 8
  bi%ellipse(2)%n_ellipse = 3
  bi%ellipse(2)%part_per_ellipse = 100
  bi%grid(3)%n_x = 20
  bi%grid(3)%n_px = 30
\end{example}
The total number of particles per bunch will be $32 \times 300 \times
600$. The exception is that when \vn{RAN_GAUSS} is mixed with other
distributions, the random distribution is overlaid with the other distributions
instead of multiplying. For example:
\begin{example}
  type (beam_init_struct) bi
  bi%distribution_type = RAN_GAUSS", "ELLIPSE", "GRID"
  bi%ellipse(2)%n_ellipse = 3
  bi%ellipse(2)%part_per_ellipse = 100
  bi%grid(3)%n_x = 20
  bi%grid(3)%n_px = 30
\end{example}
Here the number of particle is $300 \times 600$. Notice that when
\vn{RAN_GAUSS} is mixed with other distributions, the value of
\vn{beam_init%n_particle} is ignored.
  \item[\%full_6D_coupling_calc] \Newline
If set \vn{True}, coupling between the transverse and longitudinal modes is taken into
account when calculating the beam distribution. 
The default \vn{False} decouples the transverse and longitudinal calculations.
  \item[\%dPz_dz] \Newline
Correlation between $p_z$ and $z$ phase space coordinates. 
  \item[\%dt_bunch] \Newline
Time between bunches
  \item[\%ellipse(3)] \Newline
The \vn{%ellipse(:)} array sets the parameters for the 
\vn{ellipse} distribution (\sref{s:ellipse.init}). 
Each component of this array looks like
\begin{example}
  type ellipse_beam_init_struct
    integer part_per_ellipse  ! number of particles per ellipse.
    integer n_ellipse         ! number of ellipses.
    real(rp) sigma_cutoff     ! sigma cutoff of the representation.
  end type
\end{example}
  \item[\%grid(3)] \Newline
The \vn{%grid} component of the \vn{beam_init_struct} sets the parameters 
for a uniformly spaced grid of particles. The components of \vn{%grid}
are:
\begin{example}
  type grid_beam_init_struct
    integer n_x        ! number of columns.
    integer n_px       ! number of rows.
    real(rp) x_min     ! Lower x limit.
    real(rp) x_max     ! Upper x limit.
    real(rp) px_min    ! Lower px limit.
    real(rp) px_max    ! Upper px limit.
  end type
\end{example}
  \item[\%KV] \Newline
The \vn{%kv} component of the \vn{beam_init_struct} sets the parameters for the 
Kapchinsky-Vladimirsky distribution (\sref{s:kv.init}). The components of \vn{%KV}
are:
\begin{example}
  type kv_beam_init_struct
    integer part_per_phi(2)    ! number of particles per angle variable.
    integer n_I2               ! number of I2
    real(rp) A                 ! A = I1/e
  end type
\end{example}
  \item[\%n_bunch] \Newline
The number of bunches in the beam is set by \vn{n_bunch}. Default is one.
  \item[\%n_particle] \Newline
Number of particles generated when the \vn{%distribution_type} is \vn{"RAN_GAUSS"}.
Ignored for other distribution types.
  \item[\%random_engine] \Newline
This component sets the algorithm to use in generating a uniform distribution
of random numbers in the interval [0, 1]. \vn{"pseudo"} is a pseudo random
number generator and "quasi" is a quasi random generator. "quasi random" is
a misnomer in that the distribution generated is fairly uniform.
  \item[\%random_gauss_converter, \%random_sigma_cutoff] \Newline
To generate Gaussian random numbers, a conversion algorithm from the
flat distribution generated according to \vn{%random_engine} is
needed.  \vn{%random_gauss_converter} selects the algorithm. The
\vn{"exact"} conversion uses an exact conversion. The \vn{"quick"}
method is somewhat faster than the \vn{"exact"} method but not as accurate.
With either conversion method, if \vn{%random_sigma_cutoff} is set to a positive number,
this limits the maximum sigma generated.
  \item[\%renorm_center, \%renorm_sigma] \Newline 
If set to True, these components will ensure that the actual beam center 
and sigmas will correspond to the input values. 
Otherwise, there will be fluctuations due to the finite number of 
particles generated.
  \item[\%sig_e, \%sig_z] \Newline
Longitudinal sigmas. \vn{%sig_e} is the fractional energy spread dE/E. 
This, along with \vn{%dPz_dz} determine the longitudinal profile.
  \item[\%species] \Newline
Name of the species tracked. If not set then the default tracking particle type is used.
  \item[\%spin] \Newline
Particle spin in Cartesian $(x, y, z)$ coordinates.
  \item[\%use_lattice_center] \Newline
If \vn{%use_lattice_center} is set to \vn{True} (default is \vn{False}), 
the center of the bunch is determined by the 
\vn{beam_start} (\sref{s:beam.start}) setting in the lattice file rather than the 
setting of \vn{%center} in the \vn{beam_init_struct}.
  \item[\%use_t_coords, \%use_z_as_t] \Newline 
If \vn{use_t_coords} is true, then the
distributions are taken as describing particles in $t$-coordinates
(\sref{s:time.phase.space}).  Furthermore, if \vn{use_z_as_t} is true,
then the $z$ coordinates from the distribution will be taken as
describing the time coordinates. For example, particles may originate
at a cathode at the same $s$, but different times.  If false, then the
$z$ coordinate from the distribution describes particles at the same
time but different $s$ positions, and each particle gets
\vn{%location=inside\$}. In this case, the bunch will need to be
tracked with a tracking method that can handle inside particles, such
as \vn{time_runge_kutta}.  All particles are finally converted to
proper $s$-coordinate distributions for Bmad to use.
\end{description}

%-----------------------------------------------------------------
\section{CSR_Parameter_Struct}
\label{s:csr.params}
\index{csr parameters|hyperbf}

The Coherent Synchrotron Radiation (CSR) calculation is discussed in Section~\sref{s:csr}.

Besides the parameters discussed below, the \vn{coherent_synch_rad_on} parameter in
Section~\sref{s:bmad.com} must be set True to enable the CSR calculation. Additionally, tracking
with \vn{CSR} will only be done through elements where the parameter \vn{csr_calc_on}
(\sref{s:integ}) has been set to \vn{True}. This is done so that the computationally intensive
\vn{CSR} calculation can be restricted to places where the \vn{CSR} effect is significant.

The CSR parameter structure has a \vn{type name}
of \vn{csr_parameter_struct} and an \vn{instance name} of \vn{csr_param}.
This structure has components
\begin{example}
  type csr_parameter_struct 
    real(rp) ds_track_step = 0        ! Tracking step size
    real(rp) beam_chamber_height = 0  ! Used in shielding calculation.
    real(rp) sigma_cutoff = 0.1       ! Cutoff for the lsc calc. If a bin sigma
                                      !  is < cutoff * sigma_ave then ignore.
    integer n_bin = 0                 ! Number of bins used
    integer particle_bin_span = 2     ! Longitudinal particle length / dz_bin
    integer n_shield_images = 0       ! Chamber wall shielding. 0 = no shielding.
    integer sc_min_in_bin = 10        ! Min number of particle needed to compute sigmas.
    logical lcsr_component_on = T     ! Longitudinal csr component
    logical lsc_component_on = T      ! Longitudinal space charge component
    logical lsc_kick_transverse_dependence = F
                                      ! Include longitudinal SC transverse dependence?
    logical tsc_component_on = T      ! Transverse space charge component
    logical small_angle_approx = T    ! Use lcsr small angle approximation?
    logical print_taylor_warning = T  ! Print Taylor element warning?
  end type
\end{example}
The values for the various quantities shown above are their default values. 

\vn{ds_track_step} is the nominal longitudinal distance traveled by
the bunch between CSR kicks. The actual distance between kicks within
a lattice element is adjusted so that there is an integer number of
steps from steps from the element entrance to the element exit.  This
parameter must be set to something positive otherwise an error will
result. Larger values will speed up the calculation at the expense of
accuracy.

\vn{beam_chamber_height} is the height of the beam chamber in
meters. This parameter is used when shielding is taken into account.
See also the description of the parameter \vn{n_shield_images}.

\vn{sigma_cutoff} is used in the longitudinal space charge (LSC)
calculation and is used to prevent bins with only a few particles in
them to give a large contribution to the kick when the computed
transverse sigmas are abnormally low.

\vn{n_bin} is the number of bins used. The bind width is dynamically
adjusted at each kick point so that the bins will span the bunch
length.  This parameter must be set to something positive. Larger
values will slow the calculation while smaller values will lead to
inaccuracies and loss of resolution. \vn{n_bin} should also not be set
so large that the average number of particles in a bin is too small. 
``Typical'' values are in the range 100 --- 1000.

\vn{particle_bin_span} is the width of a particle's triangular density
distribution (cf.~\sref{s:csr}) in multiples of the bin width. A
larger span will give better smoothing of the computed particle
density with an attendant loss in resolution.

\vn{n_shield_images} is the number of shielding current layers used in
the shielding calculation. A value of zero results in no
shielding. See also the description of the parameter
\vn{beam_chamber_height}. The proper setting of this parameter depends
upon how strong the shielding is. Larger values give better accuracy
at the expense of computation speed. ``Typical'' values are in the
range 0 --- 5.

the \vn{sc_min_in_bin} parameter sets the minimum number of particle in a bin needed to
compute the transverse beam sigmas for that bin. If the number of particles is less than
this number, the beam sigmas are taken to be equal to the beam sigmas of a nearby bin
where there are enough particle to compute the sigma. The beam sigmas are needed for the
CS calculation but not need for the CSR calculation.

\vn{lcsr_component_on} toggles on or off the (longitudinal) CSR kick.  Note: Since \bmad
uses a ``1-Dimensional'' model for the CSR kick, there is no transverse CSR kick in
the model.

The \vn{lsc_component_on} parameter toggles on or off the longitudinal space charge kick.

The \vn{tsc_component_on} parameter toggles on or off the transverse space charge kick.

The \vn{lsc_kick_transverse_dependence} parameter toggles on or off whether the longitudinal
space charge kick is evaluated along the beam centerline (\vn{lsc_kick_transverse_dependence}
set to False) or whether the transverse displacement of a particle is used with the calculation
(\vn{lsc_kick_transverse_dependence} set to True). Simulating the transverse dependence to the
longitudinal SC kick is more physical but will take more time.

Note: \vn{Taylor} map elements (\sref{s:taylor}) that have a finite length cannot be
subdivided for the CSR calculation. \bmad will ignore any \vn{taylor} elements present in
the lattice but will print a warning that it is doing so. So suppress the warning,
\vn{print_taylor_warning} should be set to False.

%-----------------------------------------------------------------
\section{Opti_DE_Param_Struct}
\label{s:de.params}
\index{de optimizer parameters|hyperbf}

\index{opti_de_param_struct}
The Differential Evolution (\vn{DE}) optimizer is used in nonlinear
optimization problems. This optimizer is based upon the work of Storn
and Price\cite{b:de}. There are a number of parameters that can be
varied to vary how the optimizer works. These parameters are are
contained in a structure named \vn{opti_de_param_struct}. the instance
name is \vn{opti_de_param}.  This structure has components
\begin{example}
                              Default
  type opti_de_param_struct
    real(rp) CR             = 0.8    ! Crossover Probability.
    real(rp) F              = 0.8    !
    real(rp) l_best         = 0.0    ! Percentage of best solution used.
    logical  binomial_cross = False  ! IE: Default = Exponential.
    logical  use_2nd_diff   = False  ! use F * (x_4 - x_5) term
    logical  randomize_F    = False  !
    logical  minimize_merit = True   ! F => maximize the Merit func.
  end type
\end{example}

The "perturbed vector" is
  v = x_1 + l_best * (x_best - x_1) + F * (x_2 - x_3) + F * (x_4 - x_5)
The last term F * (x_4 - x_5) is only used if \vn{use_2nd_diff} = T.

The crossover can be either "Exponential" or "Binary". 
Exponential crossover is what is described in the paper.
With Exponential crossover the crossover parameters from a contiguous block
and the average number of crossover parameters is approximately
    average crossovers $\sim$ min(D, CR / (1 - CR))
where D is the total number of parameters.
With Binary crossover the probability of crossover of a parameter is 
uncorrelated with the probability of crossover of any other parameter and
the average number of crossovers is
    average crossovers = D * CR

\vn{randomize_F} = True means that the F that is used for a given 
generation  is randomly chosen to be within the range 
[0, 2*F] with average F.

%-----------------------------------------------------------------
\section{Dynamic Aperture Simulations: Aperture_Param_Struct}

\index{dynamic_aperture_struct|hyperbf}
The \vn{dynamic_aperture_struct} is used for dynamic aperture
calculations. This structure has components:
\begin{example}
  type aperture_param_struct
    real(rp) :: min_angle = 0
    real(rp) :: max_angle = pi
    integer :: n_angle   = 9
    integer :: n_turn = 100                ! Number of turns a particle must survive
    real(rp) :: x_init = 1e-3              ! Initial estimate for horizontal aperture
    real(rp) :: y_init = 1e-3              ! Initial estimate for vertical aperture
    real(rp) :: accuracy = 1e-5            ! Resolution of bracketed aperture.
  end type
\end{example}

