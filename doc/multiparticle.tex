\chapter{Multiparticle Simulation}

%-----------------------------------------------------------------
\section{Bunch Initialization}
\label{s:bunch.init}
\index{bunch initialization|hyperbf}

\textit{[Developed by Michael Saelim]}

To better visualize the evolution of a particle beam, it is sometimes convenient to initialize the
beam with the particles regularly spaced. The following two algorithms are implemented in \bmad for
such a purpose.

See Chapter~\vn{c:beam.init} for details on the standard input format used by \bmad based programs for
reading in bunch initialization parameters.

%----------------------------------------------
\subsection{Elliptical Phase Space Distribution}
\label{s:ellipse.init}

To observe nonlinear effects on the beam, it is sometimes convenient to
initialize a bunch of particles in a way that puts more particles in the
tails of the bunch than one would normally have with the standard method
of seeding particles using a Gaussian distribution. In order to preserve
the emittance, a distribution with more particles in the tail needs to
decrease the charge per tail particle relative to the core. 
This feature, along with a regular distribution,
are contained in the following \vn{``ellipse''}
distribution algorithm. 

Consider the two dimensional phase space $(x, p_x)$. 
The transformation to action-angle coordinates,
$(J, \phi)$, is
\begin{align}
  J &= \frac{1}{2}[\gamma x^2 + 2 \alpha x x' + \beta x'^2] \\
  \tan\phi &= \frac{-\beta \, (x' + \alpha \, x)}{x}
\end{align}
The inverse is
\Begineq
  \begin{pmatrix} 
    x \\ x' 
  \end{pmatrix} 
  = \sqrt{2J} 
  \begin{pmatrix} 
    \sqrt{\beta} & 0 \\ -\frac{\alpha}{\sqrt{\beta}} & 
    -\frac{1}{\sqrt{\beta}} 
  \end{pmatrix}
  \begin{pmatrix} 
    \cos\phi \\ 
    \sin\phi 
  \end{pmatrix}.
\Endeq
In action-angle coordinates, the normalized Gaussian phase space 
distribution, $\rho(J, \phi)$, is
\Begineq
  \rho(J,\phi) = \frac{1}{2\pi\varepsilon} e^{-\frac{J}{\varepsilon}}.
  \label{eq:rho}
\Endeq
where the emittance $\varepsilon$ is just the average of $J$ over the distribution
\Begineq
  \varepsilon = \langle J \rangle \equiv \int dJ \, d\phi \, J\rho(J,\phi).
  \label{eq:eps}
\Endeq
The beam sizes $\sigma$ and $\sigma'$ are
\begin{align}
  \sigma  & = \sqrt{\langle x^2 \rangle} = \sqrt{\varepsilon\beta}  \\
  \sigma' & = \sqrt{\langle x'^2 \rangle} = \sqrt{\varepsilon\gamma},
  \label{eq:rms}
\end{align}
and the covariance is
\Begineq
  \langle xx' \rangle = -\varepsilon\alpha.
  \label{eq:corr}
\Endeq

The \vn{ellipse} algorithm starts by partitioning phase space
into regions bounded by ellipses of constant $J = B_n$, $n = 0, \ldots N_J$. 
The boundary values $B_n$ are chosen so that, except for the last boundary,
the $\sqrt{B_n}$ are equally spaced
\Begineq
  B_n = 
  \begin{cases}
    \frac{\varepsilon}{2} \, \left( \frac{n_\sigma \, n}{N} \right)^2 & 
                  \text{for } 0 \le n < N_J \\
    \infty & \text{for } n = N_J
  \end{cases}
\Endeq
where $n_\sigma$ is called the \vn{``boundary sigma cutoff''}.
Within each region, an elliptical shell of constant $J_n$ is constructed with
$N_\phi$ particles equally spaced in $\phi$. The charge $q_n$ of each
particle of the $n$\Th ellipse is chosen so that the total charge of all
the particles of the ellipse is equal to the total charge within the
region
\Begineq
  N_\phi \, q_n = 
  \int_{B_{n-1}}^{B_{n}} \!\! dJ \int_{0}^{2\pi} \!\! d\phi \, \rho(J,\phi) 
  = 
    \exp \left( -\frac{B_{n-1}}{\varepsilon} \right) - 
    \exp \left( -\frac{B_{n}}{\varepsilon} \right)
\Endeq
The value of $J_n$ is chosen to coincide with the average $J$ within the region
\Begineq
  N_\phi \, q_n \, J_n = 
  \int_{B_{n-1}}^{B_{n}} \!\! dJ \int_{0}^{2\pi} \!\! d\phi \, J \, \rho(J,\phi) 
  = \varepsilon (\xi + 1) e^{-\xi} 
    \biggr\vert_{\frac{B_{n}}{\varepsilon}}^{\frac{B_{n-1}}{\varepsilon}}
\Endeq
The \vn{ellipse} phase space distribution is thus
\Begineq
  \rho_{model}(J, \phi) = q_{tot} \, 
  \sum_{n=1}^{N_J} q_{n} \, \delta(J - J_{n}) \, 
  \sum_{m=1}^{N_\phi} \, \delta(\phi - 2\pi \frac{m}{N_{\phi}})
  \label{eq:rhomodel}
\Endeq
where $q_{tot}$ is the total charge. At a given point in the lattice, where
the Twiss parameters are known, the input parameters needed to construct
the \vn{ellipse} phase space distribution is $n_\sigma$, $N_J$, $N_\phi$, 
and $q_{tot}$.

The \vn{ellipse} distribution is two dimensional in nature but can easily be 
extended to six dimensions.

%----------------------------------------------
\subsection{Kapchinsky-Vladimirsky Phase Space Distribution}
\label{s:kv.init}

The Kapchinsky-Vladimirsky (\vn{KV}) distribution can be thought of as
a four dimensional analog of the \vn{ellipse} distribution with only one
elliptical shell. Consider a 4D phase space $(x,x', y,y')$.  
Using this framework, a 4D Gaussian distribution is
\begin{align}
  \rho(J_x, \phi_x, J_y, \phi_y) &= 
    \frac{1}{(2\pi)^2 \varepsilon_x \varepsilon_y}\; 
    exp(-\frac{J_x}{\varepsilon_x})\; exp(-\frac{J_y} {\varepsilon_y}) \\
  &= \frac{1}{(2\pi)^2 \varepsilon_x \varepsilon_y}\; 
    exp(-\frac{I_1}{\varepsilon}) ,
\end{align}
where the orthogonal action coordinates are:
\begin{align}
  I_1 &= \left(  \frac{J_x}{\varepsilon_x} + \frac{J_y}{\varepsilon_y} \right) \varepsilon \\
  I_2 &= \left( -\frac{J_x}{\varepsilon_y} + \frac{J_y}{\varepsilon_x} \right) \varepsilon
\end{align}
with $\varepsilon = (\frac{1}{\varepsilon_x^2} + \frac{1}{\varepsilon_y^2})^{-1/2}$.  
The reverse transformation is:
\begin{align}
   J_x & = \left( \frac{I_1}{\varepsilon_x} - \frac{I_2}{\varepsilon_y} \right) 
      \varepsilon  \\
   J_y & = \left( \frac{I_1}{\varepsilon_y} + \frac{I_2}{\varepsilon_x} \right) 
      \varepsilon.
\end{align}

The \vn{KV} distribution is
\Begineq
  \rho(I_1,I_2,\phi_x,\phi_y) = \frac{1}{A} \delta(I_1 - \xi),
\Endeq
where $A = \frac{\varepsilon_x \varepsilon_y}{\varepsilon^2} \xi (2\pi)^2$ 
is a constant which normalizes the distribution to 1.  
By choosing a particular $\xi$, and iterating over the domain of the three remaining
coordinates, one can populate a 3D subspace of constant density.

The range in $I_2$ to be iterated over is constrained by $J_x$, $J_y \geq 0$.  
Thus $I_2 is in the range [-\frac{\varepsilon_x}{\varepsilon_y} I_1, 
\frac{\varepsilon_y}{\varepsilon_x} I_1]$. 
This range is divided into N regions of equal size, with a ring of 
particles placed in the middle of each region.  
The angle variables are also constrained to $\phi_x, \phi_y \in [0, 2\pi]$, 
with each range divided into $M_x$ and $M_y$ regions, respectively.  
Each of these regions will have a particle placed in its center.

The weight of a particle is determined by the total weight of the region 
of phase space it represents.  
Because the density $\rho$ is only dependent on $I_1$,
\begin{align}
   q &= \int_{0}^{\infty} dI_1 \int_{I_2}^{I_2 + \Delta I_2} 
     dI_2 \int_{\phi_x}^{\phi_x + \Delta \phi_x} d\phi_x 
    \int_{\phi_y}^{\phi_y + \Delta \phi_y} d\phi_y \; \frac{1}{A} \delta(I_1 - \xi) \\
   &= \frac{1}{A} \Delta I_2 \Delta \phi_x \Delta \phi_y.
\end{align}
To represent the distribution with particles of equal weight, 
we must partition $(I_2,\phi_x,\phi_y)$-space into regions of equal volume.

The weight of each particle is
\Begineq
  q = \frac{1}{N M_x M_y} = \frac{1}{N_{tot}}
\Endeq
where $N_{tot}$ is the total number of particles

%-----------------------------------------------------------------
\section{Touschek Scattering}
\label{s:touschek}
\index{Touschek Scattering}

\textit{[Developed by Michael Ehrlichman]}

Touschek scattering occurs when a single
scattering event between two particles in the same beam transfers
transverse momentum to longitudinal momentum, and the resulting change
in longitudinal momentum results in the loss of one or both particles.
In the case of storage rings, these losses impose a beam lifetime.  In
low-emittance storage rings, Touschek scattering can be the dominant
mechanism for particle loss.  In the case of linear accelerators,
these losses generate radiation in the accelerator tunnel.  When the
scattered particles collide with the beam chamber, x-rays are produced
which can damage equipment and impose a biohazard.  Studies of
Touschek scattering typically look at beam lifetime and locations
where scattering occurs and where particles are lost.

A commonly utilized theory for studying Touschek scattering is from
Piwinski \cite{b:piwinski}.  A basic outline of the derivation is,
\begin{enumerate}
\item Scatter two particles from a bunch in their COM frame using the relativistic
Moller cross-section.
\item Boost from COM frame to lab frame.  Changes to longitudinal momentum end up 
amplified by a factor of $\gamma$.
\item Integrate over 3D Gaussian distribution of particle positions and angles.
\end{enumerate}
During the derivation many approximations are made which lead to a relatively simple
formula.  The integration is set up such that only those collisions which will
result in particle loss are counted.  The formula takes the momentum aperture
as a parameter.  The resulting formula is reproduced here to give the reader
an idea of what influences the scattering rate, and how one might go about evaluating
the formula,
\begin{multline}
R=\frac{r_e^2 c\beta_x\beta_y\sigma_h N_p^2}{8\sqrt\pi\beta^2\gamma^4\sigma_{x\beta}^2
\sigma_{y\beta}^2\sigma_s\sigma_p}\int_{\tau_m}^\infty\Bigg(
\left(2+\frac{1}{\tau}\right)^2\left(\frac{\tau/\tau_m}{1+\tau}-1\right)+1
-\frac{\sqrt{1+\tau}}{\sqrt{\tau/\tau_m}}\\
-\frac{1}{2\tau}\left(4+\frac{1}{\tau}\right)\ln\frac{\tau/\tau_m}{1+\tau}\Bigg)
\frac{\sqrt\tau}{\sqrt{1+\tau}}e^{-B_1\tau}I_0\left[B_2\tau\right]d\tau,
\end{multline}
where $\tau_m=\beta^2\delta_m^2$ and $\delta_m$ is the momentum
aperture.  This formula gives the rate at which particles are
scattered out of the bunch.  It is assumed that two particles are lost
per scattering event, one with too much energy and one with too little
energy.  If a machine with an unsymmetric momentum aperture is being
studied, then the formula should be evaluated twice, once for each
aperture, and the results averaged.  Refer to \cite{b:piwinski} for
definitions of the parameters involved.  This formula is implemented
in BMAD as part of the {\tt touschek\_mod} module.

Different formulas for calculating the Touschek scattering rate exist
elsewhere in the literature.  For example,
Wiedemann~\cite{b:wiedemann}, presents a formula with a simpler
integrand.  This formula, originally from a paper by
LeDuff~\cite{b:leduff}, is derived in a fashion similar to Piwinski
except that the formula does not take dispersion into account and uses
a non-relativistic scattering cross-section.  Since Piwinski's formula
is the most robust, it is the one used in \bmad.

Particles are lost from Touschek scattering due to two effects.  In
storage rings, there is a momentum aperture defined by the RF system
that is often referred to as the RF bucket.  If the $\delta p$
imparted by a Touschek scattering event exceeds this RF bucket, then
the particle will no longer undergo synchrotron oscillations with the
rest of the bunch and will coast through the accelerator.  Second, if
the Touschek scattering event occurs in a dispersive region, the
scattered particles will take on a finite $J$ and undergo betatron
oscillations.  These oscillations can be large in amplitude and may
cause the particles to collide with the beam pipe.  To first order,
the amplitude of $J$ due to a scattering event that imparts a momentum
deviation of $\Delta p$ is,
\begin{equation}
  J\approx\gamma_0{\cal H}_0\frac{\Delta p^2}{2},
\end{equation}
where $\gamma_0$ is relativistic $\gamma$ and ${\cal H}_0$ is the dispersion invariant.

%-----------------------------------------------------------------
\section{Macroparticles}
\label{s:macro}
\index{macroparticles|hyperbf}

{\em Note: The macroparticle tracking code is not currently maintained in favor of tracking an
ensemble of particles where each particle is specified by a position without a sigma matrix. The 
following is present for historical reference only.}

A macroparticle\cite{b:transport.appendix} is
represented by a centroid position $\bfrbar$ and a $6 \times 6$
$\bfsig$ matrix which defines the shape of the macroparticle in
phase space. $\sigma_i = \sqrt{\bfsig(i,i)}$ is the RMS sigma for the $i$\Th
phase space coordinate. For example $\sigma_z = \sqrt{\bfsig(5,5)}$.

$\bfsig$ is a real, non-negative symmetric matrix. The equation that
defines the ellipsoid at a distance of $n$--sigma from the centroid is
\Begineq
  (\bfr - \bfrbar)^t \bfsig\inv (\bfr - \bfrbar) = n
\Endeq
where the $t$ superscript denotes the transpose. Given the sigma matrix
at some point $s = s_1$, the sigma matrix at a different point $s_2$ is
\Begineq
  \bfsig_2 = \bfM_{12} \, \bfsig_1 \, \bfM_{12}^t
\Endeq
where $\bfM_{12}$ is the Jacobian of the transport map from point
$s_1$ to $s_2$.

\index{dispersion}
The Twiss parameters can be calculated from the sigma matrix. The
dispersion is given by
\begin{align}
  \sigma(1,6) &= \eta_x \, \sigma(6,6) \CRNO
  \sigma(2,6) &= \eta'_x \, \sigma(6,6) \\
  \sigma(3,6) &= \eta_y \, \sigma(6,6) \CRNO
  \sigma(4,6) &= \eta'_y \, \sigma(6,6) \nonumber
\end{align}
Ignoring coupling for now, the betatron part of the sigma matrix can be
obtained from the linear equations of motion. For example, using
\Begineq
  x = \sqrt{2 \, \beta_x \, \epsilon_x} \cos \phi_x + \eta_x \, p_z
\Endeq
Solving for the first term on the RHS, squaring and averaging over all
particles gives
\Begineq
  \beta_x \, \epsilon_x = \sigma(1,1) - \frac{\sigma^2(1,6)}{\sigma(6,6)}
\Endeq
It is thus convenient to define the betatron part of the sigma matrix
\Begineq
  \sigma_\beta(i,j) \equiv \sigma(i,j) - \frac{\sigma(i,6) \, \sigma(j,6)}{\sigma(6,6)}
\Endeq
and in terms of the betatron part the emittance is
\Begineq
  \epsilon_x^2 = \sigma_\beta(1,1) \, \sigma_\beta(2,2) - \sigma_\beta^2(1,2)
\Endeq
and the Twiss parameters are
\Begineq
  \epsilon_x 
  \begin{pmatrix}
    \beta_x   & -\alpha_x \\
    -\alpha_x & \gamma_x
  \end{pmatrix} = 
  \begin{pmatrix}
    \sigma_\beta(1,1) & \sigma_\beta(1,2) \\
    \sigma_\beta(1,2) & \sigma_\beta(2,2) 
  \end{pmatrix}
\Endeq

If there is coupling, the transformation between the $4\times 4$
transverse normal mode sigma matrix $\bfsig_a$ and the $4\times 4$
laboratory matrix $\bfsig_x$ is
\Begineq
  \bfsig_x = \bfV \, \bfsig_a \bfV^t
\Endeq
where $\bfV$ is given by \Eq{vgicc1}.

The sigma matrix is the same for all macroparticles and is
determined by the local Twiss parameters:
\begin{align}
  \sigma(1,1) &= \epsilon_x \, \beta_x \CRNEG
  \sigma(1,2) &= -\epsilon_x \alpha_x  \CRNEG
  \sigma(2,2) &= \epsilon_x \, \gamma_x = 
      \epsilon_x \, (1 + \alpha_x^2) / \beta_x \CRNEG
  \sigma(3,3) &= \epsilon_y \, \beta_y \\
  \sigma(3,4) &= -\epsilon_y \alpha_b \CRNEG
  \sigma(3,4) &= \epsilon_y \, \gamma_y = 
      \epsilon_y \, (1 + \alpha_b^2) / \beta_y \CRNEG
  \sigma(i,j) &= 0 \quad \text{otherwise} \nonumber
\end{align}
The centroid energy of the $k$\Th macroparticle is
\Begineq
  E_k = E_b + \frac{(n_{mp} - 2 \, k + 1) \, \sigma_E \, N_{\sigma E}}{n_{mp}}
\Endeq
where $E_b$ is the central energy of the bunch, $n_{mp}$ is the number
of macroparticles, $\sigma_E$ is the energy sigma, and
$N_{\sigma E}$ is the number of sigmas in energy that the range of
macroparticle energies cover. The charge of each macroparticle is,
within a constant factor, the charge contained within the energy
region $E_k - dE_{mp}/2$ to $E_k + dE_{mp}/2$ assuming a Gaussian
distribution where the energy width $dE_{mp}$ is
\Begineq
  dE_{mp} = \frac{2 \, \sigma_E \, N_{\sigma E}}{n_{mp}}
\Endeq

