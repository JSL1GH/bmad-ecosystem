\chapter{Optimization}\index{Optimization}
\label{c:opti}

%------------------------------------------------------------------------
\section{Lattice Corrections}\index{Optimization!Lattice Corrections}

Examples of lattice corrections include flattening the orbit and
adjusting quadrupoles to correct the measured betatron phase. The
general idea is to vary an appropriate set of \vn{variables} with the
aim of minimizing a merit function \vn{M} that is a measure of how
well \vn{data_model}, the data as calculated from the \vn{model} fits
\vn{data_meas}, the measured data\index{Optimization!Merit Function}
\Begineq
  {\cal M} \equiv \sum_{i} w_i \,
    \bigl[ \data\_\model(i) -  \data\_\meas(i) \bigr]^2 + 
  \sum_{j} w_j \,
    \bigl[ \var\_\model(j) - \var\_\meas(j) \bigr]^2
  \label{m1}
\Endeq
\vn{var_model} is the value of a variable in the \vn{model} and
\vn{var_meas} is the value as measured at the time the data was taken
and the sum \vn{j} runs over all variables that are allowed to be
varied to minimize \vn{M}. The second term in the merit function
prevents degeneracies (or near degeneracies) in the problem which
would allow \tao to find solutions where \vn{data_model} matches
\vn{data_measured} with the \vn{var_model} having ``unphysical''
values (values far from \vn{var_meas}). The weights $w_i$ and $w_j$
need to be set depending upon how accurate the measured data is
relative to how accurate the calibrations for measuring the
\vn{var_meas} values are. With the second term in the merit function
the number of constraints (number of terms in the merit function) is
always larger than the number of variables and degeneracies can never
occur. 

\index{Optimization!Optimizer}
The algorithm used to vary the \vn{var_model} variables to minimize
\vn{M} is called an \vn{optimizer}. In \vn{command line mode} the
\vn{run} command is used to invoke an \vn{optimizer}. In \vn{single
mode} the \vn{g} key starts an optimizer and the \vn{.} key stops it.
Running an optimizer is also called ``fitting'' since one is trying to
get the \vn{data_model} to be equal to the \vn{data_meas}. With orbits
this is also called ``flattening'' since one generally wants to end up
with an orbit that is on--axis.

In a correction one wants to change the machine variables so that the
measured data corresponds to the design values \vn{data_design}. Thus
the change in the data that one wants is
\begin{example}
  data_change = data_design - data_meas
\end{example}
Once a fit has been made, and presuming that the \vn{data_model} is
resonably close to the \vn{data_meas} this data change within the
\vn{model} lattice can be accomplished by changing the variables by
\begin{example}
  var_change = var_design - var_model
\end{example}
This assumes the system is linear. For many situations this is true
since typically \vn{var_change} is ``small''. Since the variables have
a measured value of \vn{var_meas} the value that the variables should
be set to is
\begin{example}
  var_final = var_meas + (var_design - var_model)
\end{example}
Notice that the fitting process is independent of the \vn{design}
lattice. It is only when calculating the corrections to the
variables that the \vn{design} lattice plays a roll. 

Sometimes it is desired to fit to changes in data as opposed to the
absolute value of the data. For example, when closing an orbit bump
knob what is important is the difference in orbits before and after
the bump knob is varied. Designating one of these orbit the
\vn{reference}, the appropriate merit function is
\begin{alignat}{1}
  {\cal M} = &\sum_{i} w_i \,
    \Bigl[ \bigl( \data\_\model(i) - \data\_\design(i) \bigr) - 
      \bigl( \data\_\meas(i) - \data\_\reference(i) \bigr) \Bigr]^2 + \CRNO
  &\sum_{j} w_j \,
    \Bigl[ \bigl( \var\_\model(j) - \var\_\design(j) \bigr) -
     \bigl( \var\_\meas(i) - \var\_\reference(i) \bigr) \Bigr]^2 
  \label{m2}
\end{alignat}
where \vn{data_ref} and \vn{var_ref} refer to the reference
measurement.  This merit function is acceptable if the reference data
is taken with the machine reasonably near the design setup so that
nonlinearities can be ignored. If this is not the case then the
fitting becomes a two step process: The first step is to fit the
\vn{model} to the \vn{reference} data using the merit function of
\Eq{m1}. The \vn{base} lattice is then set equal to the \vn{model}
lattice. The second step is to fit the model using the merit function
\begin{alignat}{1}
  {\cal M} = &\sum_{i} w_i 
    \Bigl[ \bigl( \data\_\model(i) - \data\_\base(i) \bigr) - 
      \bigl( \data\_\meas(i) - \data\_\reference(i) \bigr) \Bigr]^2 + \CRNO
  &\sum_{j} w_j 
    \Bigl[ \bigl( \var\_\model(j) - \var\_\base(j) \bigr) -
     \bigl( \var\_\meas(i) - \var\_\reference(i) \bigr) \Bigr]^2 
  \label{m3}
\end{alignat}

Control of what data and what variables are to be used in the fitting
process is controlled by the \vn{use}, \vn{veto}, \vn{restore}, and
\vn{clip} commands.

%------------------------------------------------------------------------
\section{Lattice Design}\index{Optimization!Lattice Design}
\index{Optimization!Constraints}

Lattice design is the process of calculating \vn{variable} strengths
to meet a number of criteria called constraints. For example, one
constraint could be that the beta function in some part of the lattice
not exceed a certain value. In this case we can proceed as was done
for lattice correction and define a merit function to be minimized:
\Begineq
  {\cal M} = \sum_{i} w_i \, C_i^2
\Endeq
Typically constraints are either to limit values to some range so the
constraint would be of the form
\Begineq
  C = 
    \begin{cases}
    \mbox{model} - \mbox{limit}  & \mbox{model $>$ Limit} \\
    0                            & \mbox{otherwise}
    \end{cases}
\Endeq
or a constraint is used to keep the \vn{model} at a certain value so
the form of the constraint would be
\Begineq
  C = \mbox{model} - \mbox{target}  
\Endeq
Here \vn{model} is the value as calculated from the \vn{model}
lattice. \vn{target} and \vn{limit} are given numbers. Part of the
optimization process is in deciding what the values should be for any
\vn{target} or \vn{limit}.

%------------------------------------------------------------------------
\section{Generalized Design}\index{Optimization!Generalized Merit function}

Since lattice design and lattice corrections are similar, \tao
combines the two into one generalized correction process. In this
generalized process The merit function becomes
\Begineq
  {\cal M} = \sum_i w_i \, D_i^2 + \sum_j w_j \, V_j^2
\Endeq
The general form of the data merit terms $D_i$ is 
\Begineq
  D = 
    \begin{cases}
    \mbox{delta}  & \mbox{Non-zero Condition} \\
    0             & \mbox{Otherwise}
    \end{cases}
\Endeq
The \vn{Non-zero Condition} needed for a non--zero $C_i$ is dependent
upon the \vn{merit_type} of the datum. There are five \vn{merit_type} constraint
types as given in Table~\ref{t:con_type}.  The default \vn{merit_type} for
data is \vn{target}.
\begin{table}[ht]
\centering
{\tt
\begin{tabular}{|l|l|l|} \hline
  {\it Merit\_Type}       & {\it Non-zero Condition} \\ \hline 
  \vn{target}            & Any \vn{delta}   \\ \hline 
  \vn{min}, \vn{abs_min} & \vn{delta} $<$ 0 \\ \hline 
  \vn{max}, \vn{abs_max} & \vn{delta} $>$ 0 \\ \hline 
\end{tabular}
}
\caption{Constraint Type List.}
\label{t:con_type}
\end{table}

\index{Optimization! Optimize with Reference}
The form of \vn{delta} is determined by two global logicals called
\vn{opt_with_ref} and \vn{opt_with_base} as shown in
Table~\ref{t:d_i}. 
\begin{table}[ht] 
\centering 
{\tt
\begin{tabular}{|l|l|l|} \hline
  \vn{Opt_with_ref} & \vn{Opt_with_base} & \vn{delta} \\ \hline 
  F & F & Model - Meas                \\ \hline 
  T & F & Model - Meas + Ref - Design \\ \hline 
  F & T & Model - Meas - Base         \\ \hline 
  T & T & Model - Meas + Ref - Base   \\ \hline 
\end{tabular}
} 
\caption{$D_i$ Form}  
\label{t:d_i}
\end{table}
What kind of data is associated with \vn{model}, \vn{design} and
\vn{base} is determined by a datum's \vn{data_type}. Valid data types are given
in table~\ref{t:data_types}.

The form of the variable terms $V_i$ is determined by its \vn{merit_type}.
For variables the merit types are:
\begin{example}
  target
  limit
\end{example}
A \vn{target} \vn{merit_type} for a variable is the same as for
datum. In this case \vn{model} is just the value of the variable.
A \vn{limit} \vn{merit_type} has the form
\Begineq
  V = 
    \begin{cases}
    \mbox{model} - \mbox{high\_lim}  & \mbox{model} > \mbox{high\_lim} \\
    \mbox{model} - \mbox{low\_lim}   & \mbox{model} < \mbox{low\_lim} \\
    0                               & \mbox{Otherwise}
    \end{cases}
\Endeq
The default \vn{merit_type} for a variable is \vn{limit}.

Note: when doing lattice design \vn{opt_with_ref} and
\vn{opt_with_base} are both set to \vn{False} and the \vn{target} and
\vn{limit} values are identified with \vn{Meas}.



