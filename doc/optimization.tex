\chapter{Optimization: Lattice Correction and Design}
\label{c:opti}

\index{optimization|hyperbf}

``\vn{Optimization}'' is the process of varying \vn{model} (\sref{s:lattice}) lattice parameters so
that a given set of properties are as close to a given set of ``desired'' values as
possible. Optimization problems generally fall into one of two categories. One category involves
``\vn{lattice correction}'' (\sref{s:lat.correction}). Problems in this category involve matching
the \vn{model} lattice to actual measured data. For example, orbit flattening involves varying
steerings in the \vn{model} lattice so that the orbit calculated from the \vn{model} lattice matches
the measured orbit. The steering strengths in the \vn{model} lattice can then be used to calculate
what changes are needed to correct the orbit in the actual machine.

The other category of optimization problems involves ``\vn{lattice design}'' (\sref{s:lat.design})
where lattice parameters are varied to achieve some set of ideal properties. For example, varying
sextupole magnet strengths in order to give maximum dynamic aperture.

Note: The \bmad and \tao tutorial (\sref{s:tutorial}) has examples of how to construct both lattice
correction and lattice design optimizations.

%------------------------------------------------------------------------
\section{Optimization Overview}
\label{s:opt.main}

Optimization involves ``\vn{data}'' and ``\vn{variables}''. \vn{Data} are the parameters to be
optimized. For example, orbit positions when flattening an orbit or the value of the beta function
at a certain point when designing a lattice. \vn{Variables} are what is to be varied which can be
steering strengths, magnet positions, etc. Any lattice parameters can be used with optimization as
long as varying the variables will affect the data. For example, orbit flattening is not restricted
to varying steerings and can be done by, say, varying magnet misalignments.

How data is organized in \tao is discussed in Chapter~\sref{c:data}. How to define datums in a \tao
initialization file is discussed in Section~\sref{s:init.data}. In brief, each datum has a set of
associated parameters. For example, each datum has ``\vn{model}'' and ``\vn{design}'' values which
are the values of the datum as calculated from the \vn{model} and \vn{design} lattices. Each datum
also has a ``\vn{measured}'' value which is set by the User. This value can be from an actual
measurement which is typical when doing lattice correction or may be the desired value of the datum
which is typical when doing lattice design. Which of the defined datums are varied during optimization
is controlled by setting the \vn{good_user} component of a datum (\sref{s:datum.opt}).

How variables are organized in \tao is discussed in Chapter~\sref{c:var}. How to define variables in
a \tao initialization file is discussed in Section~\sref{s:init.var}. In brief, like datums, each
variable has a number of associated parameters. For example, each variable has a ``\vn{model}''
value which controls the corresponding value or values (a variable can control multiple parameters
simultaneously) in the \vn{model} lattice. There are also ``\vn{low_lim}'' and ``\vn{high_lim}''
parameters that can be set by the User that are used to keep the variable \vn{model} value within a
given range.

Typically, optimization involves minimizing one or more ``objectives'' or ``merit functions''. \tao
itself implements ``single objective'' optimization where there is only one merit
function\footnote{For ``multiple objective'' optimization, there is a separate program called
\vn{moga} that can be used (\sref{s:other}).}. The general form of the merit function $\cal M$ in \tao
is
\begin{equation}
  {\cal M} \equiv 
    \sum_{i} w_i \, \bigl[ \delta D_i \bigr]^2 + 
    \sum_{j} w_j \, \bigl[ \delta V_j \bigr]^2
  \label{m1}
\end{equation}
The first term on the RHS is a sum over the \vn{data} and the second term is a sum over the
\vn{variables}.  The $w_i$ and $w_j$ are weights, specified by the User and stored in the
\vn{weight} component of a datum or variable. The value of $\delta D$ (\sref{s:del.d}) for a datum or
the value of $\delta V$ (\sref{s:del.v}) for a variable is put in the \vn{delta_merit} component of
the datum or variable. The contribution to the merit function, which is $[\delta D]^2$ for a datum
and $[\delta V]^2$ for a variable, is put in the \vn{merit} component of the datum or variable.

In some cases it might be desired a datum or variable's value. For example, maximizing the dynamic
aperture. In this case, the datum or variable weight $w$ needs to be set negative and the
contribution to the merit function will be negative. Note: The \vn{lmdif} optimizer
(\sref{s:tao.opti}) will not function correctly if there are negative weights.

In the case where it is not possible to calculate \vn{delta_merit} for a datum, for example, in the
case where a datum is associated with a Twiss parameter but the lattice is unstable, the value of
\vn{delta_merit} is set equal to the value of the datum's \vn{invalid} component. This is
useful for ensuring that the optimization does not get stuck in an unstable state. Additionally, the
following \vn{data_type} (\sref{s:data.types}) can be used to drive the optimization away from an
instability:
\begin{example}
  unstable.eigen
  unstable.eigen.a, .b, .c
  unstable.lattice
  unstable.orbit
\end{example}

There are a number of \vn{show} commands (\sref{s:show}) that can be used to show optimization
parameters:
\begin{example}
  show constraints    ! \sref{s:show.constraints} List of constraints 
  show data           ! \sref{s:show.data} Show data info
  show derivative     ! \sref{s:show.derivative} Show derivative info
  show merit          ! \sref{s:show.merit} Top contributors to the merit function
  show optimizer      ! \sref{s:show.optimizer} Optimizer parameters
  show variable       ! \sref{s:show.variables} Show variable info
\end{example}

%------------------------------------------------------------------------
\section{Calculation of $\delta D$ for a Datum}
\label{s:del.d}

The $\delta D$ terms in the merit function (\Eq{m1}) (the value of which is put in the datum's
\vn{delta_merit} component) are used to drive the \vn{model} lattice parameters towards a certain
set of desirable properties. How $\delta D$ is calculated for a given datum is determined, in part,
by the setting of the datum's \vn{merit_type} component. The possible settings of this component
are:
\begin{example}
  "target"                       ! Default
  "average"                      ! Uses evaluation range
  "integral"                     ! Uses evaluation range
  "rms"                          ! Uses evaluation range
  "min", "max"
  "abs_min", "abs_max"
  "max-min"                      ! Uses evaluation range
\end{example}
As noted, some of these settings need an \vn{evaluation range} (\sref{s:dat.eval}). An evaluation
range occurs when a datum's \vn{ele_start_name} is set and the evaluation range covers the region
from the lattice element identified by \vn{ele_start_name} and ending at the \vn{ele_name} element.
Note: It does not make sense for a datum with a \vn{data_type} that does not have an associated
evaluation element (for example, a datum with an emittance related \vn{data_type}), to have an
evaluation range. Therefore, for such datums, it is not possible to use any one of the four
\vn{merit_type} settings that use an evaluation range.

The calculation of $\delta D$ is a three step process. The first step is calculating the
\vn{model}, \vn{base} and \vn{design} values for a datum which are calculated from the \vn{model},
\vn{base}, and \vn{design} lattices. The first step calculation depends upon the setting of
\vn{merit_type} as well as whether there is an associated evaluation range.

For datums that do not have an evaluation range, the value of the datum's \vn{model}, \vn{base}, or
\vn{design} component is the appropriate value of the \vn{data_type}. For example, the $b$-mode
emittance for a \vn{data_type} setting of \vn{"emit.b"} or the horizontal dispersion (for
\vn{data_type} set to \vn{"eta.x"}) at the evaluation element (named by \vn{ele_name}). If there is
an associated reference element, the value will be modified by subtracting off the \vn{data_type}
value at the reference element. The exception is that for \vn{merit_type} set to \vn{"abs_min"} or
\vn{"abs_max"}, the absolute value is taken (this is done after any reference value is subtracted).

For datums that do have an an evaluation region, the \vn{model}, \vn{base}, or \vn{design} datum values
are computed as described below. In all cases, if there is an associated reference element, the value
will be modified by subtracting off the \vn{data_type} value at the reference element.
\begin{description}[itemsep=0pt,topsep=0pt]
%
\item["target"] \Newline
With \vn{merit_type} set to\vn{"target"}, an evaluation range is not permitted since it does not
make sense to have one.
%
\item["average"] \Newline
With \vn{"average"} as the \vn{merit_type}, the value is the mean of the
\vn{data_type} over the evaluation region. This is just the integral normalized by the length of the
region. To save time, the data is only evaluated at the ends of elements
and the average is evaluated assuming linear variation between points.
%
\item["integral"] \Newline
With \vn{"integral"} as the \vn{merit_type}, the value is the integral of the \vn{data_type} over
the evaluation region. To save time, the data is only evaluated at the ends of elements and the
integral is evaluated assuming linear variation between points.
%
\item["min" or "max"] \Newline
With \vn{"min"} or \vn{"max"} as the \vn{merit_type}, the value
is the minimum or maximum value of the \vn{data_type} over the evaluation region.
%
\item["abs_min" or "abs_max"] \Newline
With \vn{"abs_min"} or \vn{"abs_max"} as the \vn{merit_type}, the value for a lattice is the minimum
or maximum value of the absolute value of the \vn{data_type} over the evaluation region.
%
\item["max-min"] \Newline
With \vn{"max-min"} as the \vn{merit_type}, the value for a lattice is the maximum value of the
\vn{data_type} over the evaluation region minus the minimum value over the evaluation region.
%
\item["rms"] \Newline
With \vn{"rms"} as the \vn{merit_type}, the value is the RMS of the
\vn{data_type} over the evaluation region. To save time, the data is only evaluated at the ends of elements
and the RMS is evaluated assuming linear variation between points.
\end{description}

  %
\begin{table}[tb] 
\centering 
{\tt
\begin{tabular}{lll} \toprule
  \vn{Opt_with_ref} & \vn{Opt_with_base} & Composite Value Expression \\ \midrule
  F & F & model - meas                    \\
  F & T & model - meas - base             \\
  T & F & (model - meas) - (design - ref) \\
  T & T & (model - meas) - (base - ref)   \\
\bottomrule
\end{tabular}
}
  \caption[Expression used to combine the values of \vn{model}, \vn{base}, \vn{design}, \vn{meas}, and
\vn{ref} into one ``composite'' value.]{
The expression used to combine the values of \vn{model}, \vn{base}, \vn{design}, \vn{meas}, and
\vn{ref} into one ``composite'' value is determined by the settings of \vn{opt_with_ref} and
\vn{opt_with_base} global parameters. These expressions are the same as Table~\ref{t:delta.v}.
  }
\label{t:delta.d}
\end{table}
  % 

After a datum's \vn{model}, \vn{base} and \vn{design} values are calculated, these values are
combined together along with the \vn{meas} and \vn{ref} values set by the User\footnote
  {
keep in mind that the \vn{ref} component is different from the \vn{ele_ref_name}
component (\sref{s:data.anatomy}). \vn{ele_ref_name} is used to set a reference element which is used
when evaluating \vn{model}, \vn{base}, and \vn{design} values (\sref{s:lat.correction}). The value of
the \vn{ref} component is set by the user.
  }. 
The formula used to combine these values into one ``\vn{composite}'' value is determined by the
setting of two global logicals \vn{opt_with_ref} and \vn{opt_with_base}. These parameters are in the
\vn{tao_global_struct} structure (\sref{s:tao.global.struct}). Table~\ref{t:delta.d} shows the
expressions.

After the \vn{composite} value is calculated, $\delta D$ will be set equal to the \vn{composite} value
except if the \vn{merit_type} is \vn{"min"}, \vn{"max"}, \vn{"abs_min"}, or \vn{"abs_max"}.  For 
\vn{"min"} and \vn{"abs_min"}, $\delta D$ will be
\begin{equation}
  \delta D = 
  \begin{cases}
    0 & \tv{composite} > 0 \\
    \tv{composite} & \tv{otherwise}
  \end{cases}
\end{equation}
That is, the datum only contributes to the merit function if the \vn{composite} is value is negative.
For \vn{"max"} and \vn{"abs_max"}, $\delta D$ will be
\begin{equation}
  \delta D = 
  \begin{cases}
    0 & \tv{composite} < 0 \\
    \tv{composite} & \tv{otherwise}
  \end{cases}
\end{equation}
That is, the datum only contributes to the merit function if the \vn{composite} is value is positive.

%------------------------------------------------------------------------
\section{Calculation of $\delta V$ for a Variable}
\label{s:del.v}

The $\delta V$ terms in the merit function (\Eq{m1}) serve one of two purposes. Such terms can be
used to keep variables within certain limits or can be used to guide the optimization towards a
solution where a variable has a certain value. How $\delta V$ is calculated for a given variable is
determined by the setting of the variable's \vn{merit_type} component. This component can
have one of two values:
\begin{example}
  "limit"     ! Default
  "target"
\end{example}

For \vn{merit_type} set to \vn{"limit"}, the value of $\delta V$ is determined by the setting of the
variable's \vn{high_lim} and \vn{low_lim} components. If not explicitly set, \vn{high_lim} defaults
to $10^{30}$ and \vn{low_lim} defaults to $-10^{30}$. The value of $\delta V$ in this case is
\begin{equation}
  \delta V = 
  \begin{cases}
    \tv{model} - \tv{high_lim} & \tv{model} > \tv{high_lim} \\
    \tv{model} - \tv{low_lim} & \tv{model} > \tv{low_lim} \\
    0 & \tv{otherwise}
  \end{cases}
\end{equation}

Note: When running the optimizer, if the parameter \vn{global%var_limits_on} (\sref{s:globals}) is
\vn{True}, and if the \vn{model} value of a variable is outside of the range set by the limits, \tao
will do two things: First, the \vn{model} value of the variable will be set to the value of the
nearest limit and, second, the variable's \vn{good_user} parameter (\sref{c:var}) is set to
False so that no further variation by the optimizer is done. This is done independent of the setting
of \vn{merit_type} for the variable and independent of whether the variable is being used in the
optimization. Sometimes it is convenient to not set the \vn{model} value to the nearest limit for
variables that are not being used in the optimization. In this case, the global parameter
\vn{global%only_limit_opt_vars} may be set to \vn{True}.  If this is done, only variables that the
optimizer is allowed to vary are restricted.

Note: The \vn{global%optimizer_var_limit_warn} parameter controls whether a warning is printed when a
variable value goes past a limit.  The default is \vn{True}.

  %
\begin{table}[ht] 
\centering 
{\tt
\begin{tabular}{lll} \toprule
  \vn{Opt_with_ref} & \vn{Opt_with_base} & \vn{Delta_Merit} ($\delta V$) Formula \\ \midrule
  F & F & model - meas                    \\
  F & T & model - meas - base             \\
  T & F & (model - meas) - (design - ref) \\
  T & T & (model - meas) - (base - ref)   \\
\bottomrule
\end{tabular}
}
  \caption[The formulas for evaluating $\delta V$ used in the Merit function.]{
When \vn{merit_type} is set to \vn{"target"}, the formula for evaluating $\delta V$ is determined by
the settings of \vn{opt_with_ref} and \vn{opt_with_base} global parameters as shown in the
table. These expressions are the same as Table~\ref{t:delta.d}.
  }
\label{t:delta.v}
\end{table}
  % 
For \vn{merit_type} set to \vn{"target"}, the formula used to compute $\delta V$ is determined by
the setting of two global logicals \vn{opt_with_ref} and \vn{opt_with_base}. These parameters are in
the \vn{tao_global_struct} structure (\sref{s:tao.global.struct}). Table~\ref{t:delta.v} shows the
expressions used to evaluate $\delta V$
In the table, \vn{model}, \vn{base} and \vn{design} are the values for the variable as set in the
\vn{model}, \vn{base}, and \vn{design} lattices, and the variable's \vn{meas} and \vn{ref} values
are set by the User. 

%------------------------------------------------------------------------
\section{Lattice Corrections}
\label{s:lat.correction}
\index{modeling data}
\index{lattice corrections}
\index{optimization!merit function}

Lattice correction is the process of varying a set of parameters in a machine achieve some desirable
state.  Typically, there are three stages. First there is a measurement. After this, corrections can
be calculated (the optimization stage). Finally, the calculated corrections are loaded into the
machine.

There are several variations of how optimization is done. To make the discussion concrete, the
case where a beta function measurement is used to calculate quadrupole \vn{K1} strength changes to
correct the machine optics is considered.\footnote
  { 
The beta function can be measured, for example, by pinging the
beam and observing the oscillations at the beam position monitors.
  }.
In this case, both data and variables will have their \vn{merit_type} set to \vn{"target"}, the
global \vn{opt_with_ref} and \vn{opt_with_base} will be False. With this, $\delta D$ and $\delta V$
in \Eq{m1} will be
\begin{align}
  \delta &\tv{D} = \tv{model_beta} - \tv{meas_beta} \CRNO
  \delta &\tv{V} = \tv{model_K1} - \tv{meas_K1}
  \label{dd1}
\end{align}
The \vn{meas_beta} data values will is the measured beta function and the variable \vn{meas_K1}
values will be the measured quadrupole \vn{K1} at the time the data was taken. The quadrupole
\vn{K1} could be measured from, for example, from the currents through the quadrupoles combined with
known current to field calibrations. The $\delta V_j$ terms in the merit function prevents
degeneracies (or near degeneracies) in the problem which would allow \tao to find solutions where
\vn{model_beta} matches \vn{meas_beta} with the \vn{model_k1} strengths having ``unphysical'' values
far from the measured strengths. The weights $w_i$ and $w_j$ in \Eq{m1} need to be set depending
upon how accurate the measured data is relative to the measured magnet strengths.

If the fit is good, the beta function is corrected by changing the
quadrupole strengths in the machine by an amount \vn{dK1} given by
\begin{example}
  dK1 = design_K1 - model_K1
\end{example}
where \vn{design_K1} is the \vn{design} value for the quadrupole strengths. The equation for
\vn{dK1} is derived using the following logic: Once a fit to the measured data has been made, the
\vn{model} lattice corresponds, more or less, to the actual state of the machine. On the other hand,
the desired state is given by the \vn{design} lattice . Thus the difference, \vn{design - model},
represents the desired state minus the actual state. The final state after the
correction will be
\begin{example}
  Final_State = Initial_State + Change
              = Actual_State + (Desired_State - Actual_State)
              = Desired_State
\end{example}
which is what is wanted.

Notice that the fitting process is independent of the strengths of the parameters in the
\vn{design} lattice. That is, the fit involves the actual machine state independent of what the
desired state is. It is not until the values needed for the correction are computed that the
parameter strengths in the \vn{design} lattice come into play.

To the extent that the measured beta function can be well fit determines the extent to which the
beta function can be corrected. For example, if an unwanted quadrupole error is generated by some
element at a spot that is far from any correctors (quadrupoles that can vary), it will not be possible to
fit the measured beta function well and it will not be possible to make a good correction. If, on
the other hand, an unwanted error is generated next to one corrector, the measured beta function can
be well fit and the \vn{model} lattice will have a strength change from the \vn{design} for that one
corrector. Varying that one corrector can then cancel out the unwanted kick.

Another point is that the correction algorithm will work with varying any set of parameters as long
as the variation in the parameters affect the \vn{model} data. Thus an analysis can be made of the
beam orbit using dipole rolls and/or quadrupole offsets as variables or any combination thereof. If
the fit is good, rolling the dipoles and moving the quadrupoles will correct the orbit. With \tao,
the User has complete freedom to vary any parameters in the fitting process.

Typically, at the start of the fit, the \vn{model} lattice is, by default, equal to the \vn{design}
lattice but this is not necessary. Generally, the actual machine state is near enough to the
\vn{design} machine state so that the machine will behave roughly linearly with the variation in the
parameters (if the machine parameters are far from the design values, it may not be possible to
store beam to take a measurement in the first place). This means that there will be only one minimum
merit function state so that the parameter values (quadrupole strengths in this case) at the end of
the fit are independent of the starting state. To put this in other terms, the User generally does
not have to worry about the initial state of the \vn{model} at the start of a fit. This is in
contrast to lattice design (\sref{s:lat.design}) where there are typically many local minima and it
can take days of work to find a good operating point. With lattice correction, on the other hand,
the near linear nature of the problem means that finding a solution in a machine with hundreds of
correctors and hundreds of BPM readings can be done in seconds with the \vn{lm} or \vn{lmdif}
optimizers (\sref{s:tao.opti}).

%------------------------------------------------------------------------
\section{Lattice Design}
\label{s:lat.design}
\index{optimization!lattice design}
\index{optimization!constraints}

Lattice design is the process of calculating \vn{variable} strengths to meet a number of criteria
called constraints. For example, one constraint could be that the beta function in some part of the
lattice not exceed a certain value or a constraint is used to keep the \vn{model} at a certain desired
value. To keep the nomenclature consistent, the desired value is labeled \vn{meas} even though it is not
a ``measured'' value.

Lattice design is an art in that there is no known algorithm that will guarantee that a global
minimum has been found. And indeed, there is no known general procedure for checking that a given
minimum is the global minimum. \tao has one optimizer for global minimum searching and that is the
\vn{de} optimizer (\sref{s:tao.opti}). Typical strategies involve using the \vn{de} optimizer to
find a minimum, then using the \vn{lm} or \vn{lmdif} optimizer to refine the answer, and then
switching back to the \vn{de} optimizer to see if a better minimum can be found. This is repeated
while varying the \vn{de} parameters and/or weights in the merit function (\Eq{m1}).

As an alternative, there is a \bmad based program separate from \tao called \vn{moga} that can be
used (\sref{s:other}). This program implements multi objective optimization.

%------------------------------------------------------------------------
\section{Optimizers in Tao}
\label{s:tao.opti}
\index{optimization!optimizer}

The algorithm used to vary the \vn{model} variables to minimize \vn{M} is called an \vn{optimizer}.
In \vn{command line mode} the \vn{run} command is used to invoke an \vn{optimizer}. In \vn{single
mode} the \vn{g} key starts an optimizer. In both modes the period key (\vn{``.''}) stops the
optimization (however, the \vn{global%optimizer_allow_user_abort} parameter (\sref{s:globals}) can
be set to False to prevent this). Running an optimizer is also called ``fitting'' since one is
trying to get the \vn{model} data to be equal to the \vn{measured} data. With orbits this is also
called ``flattening'' since one generally wants to end up with an orbit that is on--axis.

The optimizer that is used can be defined when using the \vn{run} command but the default optimizer
can be set in the \tao input file by setting the \vn{global%optimizer} component (\sref{s:globals}).

When the optimizer is run in \tao, the optimizer, after it initializes itself, takes a number of
\vn{cycles}. Each cycle consists of changing the values of the variables the optimizer is allowed to
change. The number of steps that the optimizer will take is determined by the parameter
\vn{global%n_opti_cycles} (\sref{s:globals}). When the optimizer initializes itself and goes through
\vn{global%n_opti_cycles}, it is said to have gone through one \vn{loop}. After going through
through \vn{global%n_opti_loops} loops, the optimizer will automatically stop.  To immediately stop
the optimizer the period key \vn{``.''} may be pressed. Note: In \vn{single_mode} (\sref{c:single}),
\vn{n_opti_loops} is ignored and the optimizer will loop forever.

There are currently three optimizers that can be used: 
  \begin{description}
  \index{lm optimizer}
  \item{\vn{lm}} \Newline
\vn{lm} is an optimizer based upon the Levenburg-Marquardt algorithm
\cite{b:nr}. This algorithm looks at the local derivative matrix of \vn{dData/dVariable} and
takes steps in variable space accordingly. The derivative matrix is calculated beforehand by varying
all the variables by an amount set by the variable's \vn{step} component (\sref{s:init.var}). The
\vn{step} size should be chosen large enough so that round-off errors will not make computation of
the derivatives inaccurate but the step size should not be so large that the derivatives are
effected by nonlinearities. By default, the derivative matrix will be recalculated each \vn{loop}
but this can be changed by setting the \vn{global%derivative_recalc} global parameter
(\sref{s:globals}). The reason to not recalculate the derivative matrix is one of time.  However, if
the calculated derivative matrix is not accurate (that is, if the variables have changed enough from
the last time the matrix was calculated and the nonlinearities in the lattice are large enough), the
\vn{lm} optimizer will not work very well.  In any case, this method will only find local minimum.
When running with \vn{lm}, the value of a parameter called \vn{a_lambda} will be printed. This
``damping factor'', which is always positive definite, is a measure of how well the variation of the
$\delta D_i$ terms with respect to the variables in \Eq{m1} matches the computed derivative matrix.
A small \vn{a_lambda}, much less than one, indicates good agreement while a larger \vn{a_lambda},
much greater than one, means that there is a mismatch. If \vn{a_lambda} is large, the optimizer will
not be able to make much progress. There are several reasons for a large \vn{a_lambda}. First, the
working point may have shifted enough so that the derivative matrix needs to be recalculated. If
recalculating the derivative matrix does not fix the problem, it may be that one or more variable
\vn{step} sizes are either too small or two large. If the value of \vn{step} is too small for a
variable, round-off error can cause the calculation of the variable's derivatives to be off. If the
value of the \vn{step} is too large, nonlinearities can throw off the derivative calculation. To
test if the step size is set correctly first use the \vn{show merit -derivative} command to see what
variables have the largest \vn{dMerit/dVariable} values as computed from the derivative
matrix. [Note: Generally the best thing is to concentrate on the variables with the largest
derivatives since these give the ``most bang for the buck''.] These values should be small when at
the true merit function minimum. The \vn{change variable} command can now be used with varying step
sizes to see if the actual \vn{dMerit/dVariable} change matches this. The output from the
\vn{change} command will show the \vn{dMerit/dVariable} value computed from varying the variable by
the amount given in the change command. If the two do not agree, there may be a problem. Warning! At
the $\cal M$ minimum the value of \vn{dMerit/dVariable} is very sensitive to the value of the
variables. To not get fooled, move away the minimum. For more information on \vn{a_lambda}, see the
Wikipedia article on the ``\vn{Levenberg–Marquardt algorithm}''. In this article the variable is
denoted $\lambda$.
  \index{lmdif optimizer}
  \item{\vn{lmdif}} \Newline
The \vn{lmdif} optimizer is like the \vn{lm} optimizer except that it builds up the information it
needs on the derivative matrix by initially taking small steps over the first \vn{n} cycles where
\vn{n} is the number of variables. The advantage of this is that you do not have to set a \vn{step}
size for the variables. The disadvantage is that for \vn{lmdif} to be useful, the number of
\vn{cycles} (set by \vn{set global n_cycles =<XXX>}) must be greater than the number of
variables. Again, like \vn{lm}, this method will only find local minimum.
  \index{de optimizer}
  \item{\vn{de}} \Newline
The \vn{de} optimizer stands for \vn{differential evolution}\cite{b:de}. The advantage of this
optimizer is that it looks for global minimum. The disadvantage is that it is slow to find the
bottom of a local minimum. A good strategy sometimes when trying to find a global minimum is to use
\vn{de} in combination with \vn{lm} or \vn{lmdif} one after the other. One important parameter with
the \vn{de} optimizer is the \vn{step} size. A larger step size means that the optimizer will tend
to explore larger areas of variable space but the trade off is that this will make it harder to find
minimum in the locally. One good strategy is to vary the \vn{step} size to see what is
effective. Remember, the optimal step size will be different for different problems and for
different starting points. The \vn{step} size that is appropriate of the \vn{de} optimizer will, in
general, be different from the \vn{step} size for the \vn{lm} optimizer. For this reason, and to
facilitate changing the step size, the actual step size used by the \vn{de} optimizer is the step
size given by a variable's \vn{step} component multiplied by the global variable
\vn{global%de_lm_step_ratio}. This global variable can be varied using the \vn{set} command
(\sref{s:set}). The number of trial solutions used in the optimization is
\begin{example}
  population = number_of_variables * global%de_var_to_population_factor
\end{example}
There are also a number of parameters that can be set that will affect how the optimizer works. See
Section~\sref{s:globals} for more details.
  \index{svd optimizer}
  \item{\vn{svd}} \Newline
The \vn{svd} optimizer uses a singular value decomposition calculation. With the \vn{svd}
optimizer, the setting of the \vn{global%n_opti_cycles} parameter is ignored. One optimization loop
consists of applying svd to the derivative matrix to locate a new set of variable values.  If the
merit function decreases with the new set, the new values are retained and the optimization loop is
finished. If the merit function increases, and if the global variable
\vn{global%svd_retreat_on_merit_increase} is True (the default), the variables are set to the
original variable settings. In either case, an increasing merit function will stop the execution of
additional loops.

The \vn{global%svd_cutoff} variable can be used to vary the cutoff that SVD uses to decide what
eigenvalues are singular.
  \end{description}

%------------------------------------------------------------------------
\section{Optimization Troubleshooting Tips}
\label{s:opt.trouble}
\index{optimization troubleshooting}

Optimization can be tricky. There are many parameters that affect the optimization and often it
comes down to trial and error to find an acceptable an acceptable solution. And even in expert
hands, optimizations can take days. The following are some tips if there are problems with an
optimization.

\begin{description}
%
\item[Show commands] \Newline
Commands that can be used to view optimizer parameters are:
\begin{example}
  show constraints
  show data
  show derivative
  show merit
  show optimizer
  show variable
\end{example}
%
\item[Set the optimizer to run longer] \Newline
One quick thing to do is to increase the number of optimization loops and/or optimization cycles:
\begin{example}
	set global n_opti_loops = 30
	set global n_opti_cycles = 50
\end{example}
The \vn{show optimizer} (\sref{s:show.optimizer}) command will show global parameters
associated with optimizations.
%
\item[Check merit function and weights] \Newline
One of the first things to check is the merit function, the top contributors can be seen with the
command \vn{show merit} (\sref{s:show.merit}). And individual contributions can be viewed using the
\vn{show variable} and \vn{show data} commands. If the weight of an individual datum is too small,
the optimizer will tend to ignore it. So one trick is to raise the weights for datums that are not
being well optimized. When running \tao this is done with the \vn{data data} command. For example:
\begin{example}
  set data twiss.a[1:2]|weight = 100*twiss.a[1:2]|weight
\end{example}
This example will increase the weight of datums \vn{twiss.a[1]} and \vn{twiss.a[2]} by a factor of
100.
%
\item[Check step size] \Newline
If using an optimizer that uses the derivative matrix (\vn{lm}, \vn{geodesic_lm} and \vn{svd}
optimizers), The variable \vn{step} sizes that are used to calculate the derivative should be
checked to make sure that the \vn{step} is not too small so that round-off is a problem but yet not
too large so that nonlinearities make the calculation inaccurate. One way to check that the step
size is adequate for a given variable is to vary the variable using the command \vn{change var}
(\sref{s:change}). This command will print out the the change in the merit function per change in
variable which can be compared to the derivatives as shown with the \vn{show merit -derivative}
(\sref{s:show.merit}) or the \vn{show derivative} (\sref{s:show.derivative}) command.
%
\item[Rotate optimizer usage] \Newline
Problems generally occur when there are many local minima. In this case, the \vn{de} optimizer
(\sref{s:tao.opti}) should be tried. This optimizer has several parameters which will need to be
varied by trial and error to find values that are suitable for the problem at hand. Optimization
strategies here include doing multiple optimizations one after another using the \vn{de} optimizer
every other optimization interlaced with one of the other optimizers. Also varying the merit
function weights between optimizations can help guide the optimization process towards an acceptable
solution.
\item[Use a non-\tao optimizer]
Using, for example, a Python based optimizer interfaced to \tao is possible. There is also the \vn{moga}
program which implements multi-objective optimization.
\end{description}


%------------------------------------------------------------------------
%++++++++++++++++++++++++++++++++++++++++++++++++++++++++++++++++++++++++
%------------------------------------------------------------------------

\endinput

\etcetc...

%------------------------------------------------------------------------
\section{Common Root Lattice (CRL) Analysis}
\label{s:crl}
\index{common root lattice}

Some data analysis problems involve varying variables in a both the \vn{model} and \vn{base}
lattices simultaneously. Such is the case with Orbit Response Matrix (\vn{ORM})
analysis\cite{b:orm}. With \vn{ORM}, the analysis starts with a set of difference orbits. A given
difference orbit is generated by varying a given steering by a known amount and the steering varied
is different for different difference orbits. Typically, The number $N$ of difference orbits is
equal to the number of steering elements in the machine. In \tao, this will result in the creation
of $N$ universes, one for each difference measurement. The \vn{model} lattice in a universe will
correspond to the machine with the corresponding steering set to what it was when the data was
taken. Conversely, the \vn{base} lattices in all the universes all correspond to the common
condition without any steering variation.

In \tao, this arrangement is called \vn{Common Root Lattice} (\vn{CRL}) analysis. To do a CRL
analysis, the \vn{common_lattice} switch must be set at initialization time (\sref{s:init.lat}).
With \vn{CRL}, \tao will set up a \vn{``common''} universe with index 0.  The \vn{model} lattice of
this common universe will be used as the \vn{base} lattice for all universes.

The variables (fit parameters) in a \vn{CRL} analysis can be divided into two classes. One class
consists of the parameters that were varied to get the data of the different universes. With
\vn{ORM}, these are the steering strengths. At initialization (\sref{s:init.var}), variables must be
set up that control these parameters. A single variable will control that particular parameter in a
particular universe, that was varied to create the data for that universe.

The second class of variables consists of everything that is to be varied in the common root
lattice. With \vn{ORM}, this generally will include such things as quadrupole and BPM error tilts,
etc. That is, parameters that did {\em not} change during data taking. The \tao variables that are
created for these parameters will control parameters of the \vn{model} lattice in the common
universe.

To cut down on memory usage when using \vn{CRL} (the number of data sets, hence the number of
universes, can be very large), \tao does not, except for the common \vn{model} lattice, reserve
separate memory for each \vn{model} lattice. Rather, it reserves memory for a single
\vn{``working''} lattice and the \vn{model} lattice for a particular universe is created by first
copying the common \vn{base} lattice to the \vn{working} lattice and then applying the variable(s)
(a steering in the case of \vn{ORM}) appropriate for that universe.  As a result, except for the
common \vn{model} lattice, it is not possible to vary a parameter of a \vn{model} lattice unless
that parameter has a \tao variable that associated with it. The \vn{change} command
(\sref{s:change}) is thus restricted to always vary parameters in the common \vn{model} lattice.

With \vn{CRL}, the \vn{opt_with_base} and \vn{opt_with_ref} (\sref{s:del.d}) global
logicals are generally set to True. Since \vn{opt_with_base}, and \vn{opt_with_ref} do not make
sense when applied to the data in the common universe, The contribution to the merit function from
data in this universe is always calculated as if \vn{opt_with_base} and \vn{opt_with_ref} were set
to False.

With \vn{opt_with_base} set to True, the \vn{base} value for a datum is evaluated by looking for a
corresponding datum in the common universe and using its \vn{model} value. To simplify the
bookkeeping, it is assumed that the structure of the data arrays is identical from universe to
universe. That is, the \vn{show data} command gives identical results independent of the default
universe.


Re: opt_with_ref and opt_with_base global parameters: An exception occurs when using a \vn{common
root lattice} (\sref{s:crl}). In this case, the common universe does not have base or reference
values associated with it. Thus all data and variables that are associated with the common universe
calculate their \vn{delta} as if both \vn{opt_with_ref} and \vn{opt_with_base} were set to
\vn{False}.

