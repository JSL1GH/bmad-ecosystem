\chapter{Overall Organization and Structure}
\label{c:organization}

\tao stands for ``Tool for Accelerator Optics''. \tao is a general
purpose program for simulating high energy particle beams in
accelerators and storage rings. This manual assumes you are already
familiar with the basics of particle beam dynamics and its
formalism. There are several books that introduce the topics very
well. A good place to start is, for example, \textit{The Physics of Particle
Accelerators} by Klaus Wille\cite{b:wille}.

\index{bmad}
\tao is based on the \bmad\cite{b:bmad} subroutine library. An
understanding of the nitty-gritty details of the routines that
comprise \bmad is not necessary, however, one should be familiar with
the conventions that \bmad uses and this is covered in the \bmad
manual.

So, what is \tao good for? A large variety of applications: Single and
multiparticle tracking, lattice simulation and analysis, lattice
design, machine commissioning and correction, etc. Furthermore, it is
designed to be extensible using interface ``hooks'' built into the
program.  This versatility has been used, for example, to enable \tao
to directly read in measurement data from Cornell's Cesr storage ring
and Jefferson Lab's FEL. Think of \tao as an accelerator design and
analysis environment. But even without any customizations, \tao will
do much analysis.

This chapter discusses how \tao is organized. After you are familiar
with the basics of \tao, you
might be interested to exploit its versatility by extending \tao to do
custom calculations. For this, see Chapter~\ref{c:custom.tao}.

%----------------------------------------------------------------
\section{The Organization of Tao: The Super\_Universe}
\label{s:organization}
\index{super_universe}

Many simulation problems fall into one of three categories: 
\begin{itemize}
\item 
Design a lattice subject to various constraints.
\item
Simulate errors and changes in machine parameters. For example, you want to
simulate what happens to the orbit, beta function, etc., when you change
something in the machine. 
\item 
Simulate machine commissioning including simulating data measurement and
correction. For example, you want to know what steering strength changes will
make an orbit flat.
\end{itemize}
Programs that are written to solve these types of problems have common
elements: You have variables you want to vary in your model of your
machine, you have "data" that you want to view, and, in the first two
categories above, you want to match the machine model to the data (in
designing a lattice the constraints correspond to the data).

With this in mind, \tao was structured to implement the essential
ingredients needed to solve these simulation problems.  
The information that \tao knows about can be divided into five
(overlapping) categories:
\begin{description}
  \index{lattice}
  \item[Lattice] \Newline   
Machine layout and component strengths, and the beam orbit (\sref{s:lattice}).
  \index{data}
  \item[Data] \Newline
Anything that can be measured.
For example: The orbit of a particle or the lattice beta 
functions, etc. (\sref{c:data})
  \index{variable}
  \item[Variables] \Newline
Essentially, any lattice parameter or initial condition that can be varied.
For example: quadrupole strengths, etc. (\sref{c:var}).
  \index{plotting}
  \item[Plotting]  \Newline
Information used to draw graphs, display the lattice 
floor plan, etc. (\sref{c:plotting}).
  \index{global parameters}
  \item[Global Parameters] \Newline
 \tao has a set of parameters to control every aspect of how it behaves from
the random number seed \tao uses to what optimizer is used for fitting data.
\end{description}

%------------------------------------------------------------------------
\section{The Super\_universe}
\label{s:super.uni}
\index{super_universe|hyperbf}

\index{structure|hyperbf}
The information in \tao deals is organized in a hierarchy of
\vn{``structures''}. At the top level, everything known to \tao is
placed in a single structure called the \vn{super_universe}.

\index{universe}
\index{variable}
Within the \vn{super_universe}, lies one or more \vn{universes}
(\sref{s:universe}), each \vn{universe} containing a particular
machine lattice and its associated data. This allows for the user to
do analysis on multiple machines or multiple configurations of a
single machine at the same time. The \vn{super_universe} also contains
the \vn{variable}, \vn{plotting}, and \vn{global parameter} information.

%------------------------------------------------------------------------
\section{The Universe}
\label{s:universe}
\index{universe|hyperbf}

\index{lattice}\index{design lattice}\index{model lattice}
\index{base lattice}\index{data}\index{super_universe}
The \tao \vn{super_universe} (\sref{s:super.uni}) contains one or
more \vn{universes}.  A \vn{universe} contains a \vn{lattice}
(\sref{s:lattice}) plus whatever data (\sref{c:data}) one wishes to
study within this lattice (i.e. twiss parameters, orbit, phase,
etc.). Actually, there are three lattices within each universe: the
\textbf{design} lattice, \textbf{model} lattice and \textbf{base}
lattice. Initially, when \tao is started, all three lattices are
identical and correspond to the lattice read in from the lattice
description file (\sref{s:init.lat}).

There are several situations in which multiple universes are
useful. One case where multiple universes are useful is where data has
been taken under different machine conditions. For example, suppose
that a set of beam orbits have been measured in a storage ring with
each orbit corresponding to a different steering element being set to some
non-zero value. To determine what
quadrupole settings will best reproduce the data, multiple universes can be
setup, one universe for each of the orbit measurements. Variables can be
defined to simultaneously vary the corresponding quadrupoles in each
universe and \tao's built in optimizer can vary the variables until
the data as determined from the \vn{model} lattice (\sref{s:lattice})
matches the measured data. This \vn{orbit response matrix} (ORM) analysis
is, in fact, a widely used procedure at many laboratories.

If multiple universes are present, it is important to be able to
specify, when issuing commands to tao and when constructing \tao
initialization files, what universe is being referred to when
referencing parameters such as data, lattice elements or other stuff
that is universe specific. [Note: \tao variables are {\em not}
universe specific.] If no universe is specified with a command, the
\vn{default} universe will be used. This default universe is set set
by the \vn{set default universe} command (\sref{s:set}). When \tao
starts up, the default universe is initially set to universe 1. Use
the \vn{show global} (\sref{s:show}) command to see the current
default universe.

the syntax used to specify a particular universe or range of universes
is attach a prefix of the form:
\begin{example}
  [<universe_range>]@<parameter>
\end{example}
Commas and colons can be used in the syntax for \vn{<universe_range>},
similar to the \vn{element list} format used to specify lattice
elements (\sref{s:ele.list.format}).  When there is only a single
Universe specified, the brackets \vn{[...]} are optional. When the
universe prefix is not present, the current default
universe is used. The current default universe
can also be specified using the number \vn{-1}. Additionally, a
``\vn{*}'' can be used as a wild card to denote all of the
universes. Examples:
\begin{example}
  [2:4,7]@orbit.x ! The \vn{orbit.x} data in universes 2, 3, 4 and 7.
  [2]@orbit.x     ! The \vn{orbit.x} data in universe 2. 
  2@orbit.x       ! Same as "2@orbit.x".
  orbit.x         ! The \vn{orbit.x} data in the current default universe.
  -1@orbit.x      ! Same as "orbit.x".
  *@orbit.x       ! orbit.x data in all the universes.
  *@*             ! All the data in all the universes. 
\end{example}

%------------------------------------------------------------------------
\section{Lattices}
\index{lattice|hyperbf}
\label{s:lattice}

\index{design lattice}\index{model lattice}
\index{base lattice}
A \vn{lattice} consists of a machine description (the strength and
placement of elements such as quadrupoles and bends, etc.), along with the
beam orbit through them. There are actually three types of lattices:
  \vspace*{-3ex}
  \begin{description}
  \index{design lattice|hyperbf}
  \item[Design Lattice] \Newline 
The \vn{design} lattice corresponds to the lattice read in from the
lattice description file(s) (\sref{s:init.lat}). In many instances, this
is the particular lattice that one wants the actual physical machine
to conform to. The \vn{design} lattice is fixed. Nothing is allowed to
vary in this lattice.
  \index{model lattice|hyperbf}
  \item[Model Lattice] \Newline
Initially the \vn{model} lattice is the same as the \vn{design} lattice. Except for some commands
that explicitly set the \vn{base} lattice, all \tao commands to vary lattice variables vary
quantities in the \vn{model} lattice. In particular, things like orbit correction involve varying
\vn{model} lattice variables until the \vn{data}, as calculated from the \vn{model}, matches the
\vn{data} as actually measured.
  \index{base lattice|hyperbf}
  \index{base lattice!using set command}
  \item[Base Lattice] \Newline
It is sometimes convenient to designate a reference lattice so that
changes in the \vn{model} from the reference point can be examined.
This reference lattice is called the \vn{base} lattice. The \vn{set}
command (\sref{s:set}) is used to transfer information from the
\vn{design} or \vn{model} lattices to the base lattice. Initally, the \vn{base} lattice is
set equal to the \vn{design} lattice by \tao.
  \end{description}

Lattices can have multiple \vn{branches}. For example, two
intersecting rings can be represented as a lattice with two branches,
one for each ring. See the \bmad manual for more details. Many \tao
commands operate on a particular lattice branch. For example, the
\vn{show lat} command prints the lattice elements of a particular
branch. If no branch is specified with a command, the default branch
is used. The default branch is set with the \vn{set default branch}
command (\sref{s:set}). Initially, when \tao is started, the default
branch is set to branch 0. Use the \vn{show global} (\sref{s:show})
command to see the current default branch.

%------------------------------------------------------------------------
\section{Tracking Types}
\index{tracking!types}

\index{track_type}
\index{tao_global_struct}
\index{global%track_type}
The are two types of tracking implemented in \tao: single particle
tracking and many particle multi-bunch tracking.  Single particle
tracking is just that, the tracking of a single particle through the
lattice. Many particle multi-bunch tracking creates a Gaussian
distribution of particles at the beginning of the lattice and tracks
each particle through the lattice, including any wakefields.  Single
particle tracking is used by default. The \vn{global%track_type}
parameter (\sref{s:globals}), which is set in the initialization file,
is used to set the tracking.

Particle spin tracking has also been set up for single particle and many
particle tracking. See Sections~\sref{s:globals} and \sref{s:beam.init} for
details on setting up spin tracking.

%------------------------------------------------------------------------
\section{Lattice Calculation}\index{lattice!calculation of}
\label{s:lat.calc}

After each \tao command is processed, the lattice and ``merit''
function are recalculated and the plot window is regenerated. The
merit function determines how well the \vn{model} fits the measured
data. See Chapter~\ref{c:opti} for more information on the merit
function and its use by the optimizer.

Below are the steps taken after each \tao command execution:
\begin{enumerate}
  \item 
The data and variables used by the optimizer are re-determined. This is
affected by commands such as \vn{use, veto,} and \vn{restore} and any
changes in the status of elements in the ring (e.g. if any elements
have been turned off).
  \item 
If changes have been made to the lattice (e.g. variables changed) then
the model lattice for all universes will be recalculated. The
\vn{model} orbit, linear transfer matrices and Twiss parameters are
recalculated for every element. All data types will also be calculated
at each element specified in the initialization file.  For single
particle tracking the linear transfer matrices and Twiss parameters
are found about the tracked orbit. Tracking is
performed using the tracking method defined for each element
(i.e. Bmad Standard, Symplectic Lie, etc...). See the \bmad Reference
manual for details on tracking and finding the linear transfer
matrices and Twiss parameters.
  \item 
The \vn{model} data is recalculated from the \vn{model} orbit, linear
transfer matrices, Twiss parameters, particle beam information and
global lattice parameters.  Any custom data type calculations are
performed \textit{before} the standard \tao data types are calculated.
  \item 
Any user specified data post-processing is performed in
\vn{tao_hook_post_process_data}.
  \item 
The contributions to the merit function from the variables and data are
computed.
  \item 
Data and variable values are transferred to the plotting structures.
  \item 
The plotting window is regenerated.
\end{enumerate}

If a closed orbit is to be calculated, \tao uses an iterative method
to converge on a solution where \tao starts with some initial orbit at
the beginning of the lattice, tracks from this initial orbit through
to the end of the lattice, and then adjusts the beginning orbit until
the end orbit matches the beginning orbit. A problem arises if the
tracked particle is lost before it reaches the end of the lattice
since \tao has no good way to calculate how to adjust the beginning
orbit to prevent the particle from getting lost. In this case, \tao,
in desperation, will try the orbit specified by \vn{particle_start} in the
\bmad lattice file (see the \bmad manual for more details on setting
\vn{particle_start}). Note: \vn{particle_start} can be varied while running
\tao using the \vn{set particle_start} (\sref{s:set}) or \vn{change
particle_start} (\sref{s:change}) commands.

If the recalculation takes a significant amount of time, the recalculation may be suppressed using
the \vn{set global lattice_calc_on} command (\sref{s:set.global}) or the \vn{set universe} command
(\sref{s:set.universe}).
