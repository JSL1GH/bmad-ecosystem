\chapter{Lattice Element Manipulation}
\label{c:ele.manip}

%--------------------------------------------------------------------------
\section{Creating Element Slices}
\label{s:ele.slice}

\index{sbend}\index{rbend}\index{wiggler}
It is sometimes convenient to split an element longitudinally into
``slices'' that represent a part of the element.  This is complicated
by the fact that elements are not necessarily uniform.  For example,
map type wigglers are nonuniform and bend elements have end effects.
Furthermore, attributes like \vn{hkick} need to be scaled with the
element length.

To create an element slice, the routine
\Hyperref{r:create.element.slice}{create_element_slice} can be used.
Example:
\begin{example}
  type (ele_struct) ele, sliced_ele
  ...
  sliced_ele = ele
  sliced_ele%value(l$) = l_slice ! Set the sliced element's length
  call create_element_slice (sliced_ele, ele, l_start, param, ...)
\end{example}
See the documentation on \vn{create_element_slice} for more details (\sref{s:getf}).

%----------------------------------------------------------------------------
\section{Adding and Deleting Elements From a Lattice}
\label{s:lat.add.delete}

Modifying the number of elements in a lattice involves a bit of
bookkeeping. To help with this there are a number of routines. 

The routine \Hyperref{r:remove.eles.from.lat}{remove_eles_from_lat} is
used to delete elements from a lattice.

For adding elements there are three basic routines: To add a lord
element, the \Hyperref{r:new.control}{new_control} routine is used.
To add a new element to the tracking part of the lattice, use the
\Hyperref{r:insert.element}{insert_element} routine. Finally, to split
an element into two pieces, the routine
\Hyperref{r:split.lat}{split_lat} is used. These basic routines are
then used in such routines as
\Hyperref{r:create.overlay}{create_overlay} that creates overlay
elements, \Hyperref{r:create.group}{create_group} which creates group
elements, \Hyperref{r:add.superimpose}{add_superimpose} which
superimposes elements, etc. Example:
\begin{example}
  type (lat_struct), target :: lat
  type (ele_struct), pointer :: g_lord, slave

  type (control_struct) con(1)
  integer ix, n
  logical err_flag
  ...
  call new_control (lat, ix)
  g_lord => lat%ele(ix)
  allocate (ele%control_var(1))
  ele%control_var(1)%name = 'A'
  call reallocate_expression_stack(con(1)%stack, 10))
  call expression_string_to_stack ('3.2*A^2', con(1)%stack, n, err_flag)
  con(1)%ix_attrib = k1$
  call lat_ele_locator ('Q1W', lat, eles)
  con(1)%slave = ele_to_lat_loc(eles(1)%ele)
  call create_group (g_lord, con, err_flag)
\end{example}
This example constructs a group element with one variable with name
\vn{A} controlling the \vn{K1} attribute of element \vn{Q1W} using the
expression ``$3.2 \cdot A^2$'' where \vn{A} is the name of the control
variable.

For constructing \vn{group} elements (but not \vn{overlay} elements),
the controlled attribute (set by \vn{con(1)%ix_attrib} in the above
example) can be set to, besides the set of element attributes, any one
in the following list:
\begin{example}
  accordion_edge$  ! Element grows or shrinks symmetrically
  start_edge$      ! Varies element's upstream edge s-position
  end_edge$        ! Varies element's downstream edge s-position
  s_position$      ! Varies element's overall s-position. Constant length.
\end{example}
See Section~\sref{s:group} for the meaning of these attributes

%---------------------------------------------------------------------------
\section{Finding Elements and Changing Attribute Values}
\label{s:lat.ele.change}

\index{ele_struct!\%ix_ele}
The routine \Hyperref{r:lat.ele.locator}{lat_ele_locator} 
can be used to search for an element
in a lattice by name or key type or a combination of both. Example:
\begin{example}
  type (lat_struct) lat
  type (ele_pointer_struct), allocatable :: eles(:)
  integer n_loc; logical err
  ...
  call lat_ele_locator ("quad::skew*", lat, eles, n_loc, err)
  print *, 'Quadrupole elements whose name begins with the string "SKEW":'
  print *, 'Name                 Branch_index        Element_index'
  do i = 1, n_loc  ! Loop over all elements found to match the search string.
    print *, eles(i)%ele%name, eles(i)%ele%ix_branch, eles(i)%ele%ix_ele
  enddo
\end{example}
This example finds all elements where \vn{ele%key} is \vn{quadrupole\$} 
and \vn{ele%name} starts with ``\vn{skew}''. See the documentation on 
\vn{lat_ele_locator} for more details on the syntax of the search string.

The \vn{ele_pointer_struct} array returned by \vn{lat_ele_locator} is
an array of pointers to \vn{ele_struct} elements
\begin{example}
  type ele_pointer_struct
    type (ele_struct), pointer :: ele
  end type
\end{example}
The \vn{n_loc} argument is the number of elements found and the \vn{err} argument
is set True on a decode error of the search string.

Once an element (or elements) is identified in the lattice,
it's attributes can be altered. However, care must be taken that an element's attribute
can be modified (\sref{s:depend}). The function \vn{attribute_free} will
check if an attribute is free to vary.
\begin{example}
  type (lat_struct) lat
  integer ix_ele
  ...
  call lat_ele_locator ('Q10W', lat, eles, n_loc, err)   ! look for an element 'Q10W'
  free = attribute_free (eles(i)%ele, 'K1', lat, .false.)
  if (.not. free) print *, 'Cannot vary k1 attribute of element Q10W'
\end{example}

With user input the routine \vn{pointer_to_attribute} is a convenient
way to obtain from an input string a pointer that points to the
appropriate attribute. For example:
\begin{example}
  type (lat_struct) lat
  type (ele_pointer_struct), allocatable :: eles(:)
  type (all_pointer_struct) attrib_ptr
  character(40) attrib_name, ele_name
  real(rp) set_value
  logical err
  integer ie
  ...
  write (*, '(a)', advance = 'no') ' Name of element to vary: '
  accept '(a)', ele_name
  write (*, '(a)', advance = 'no') ' Name of attribute to vary: '
  accept '(a)', attrib_name
  write (*, '(a)', advance = 'no') ' Value to set attribute at: '
  accept *, set_value

  call lat_ele_locator (name, lat, eles, n_loc, err)
  if (err) ....
  do ie = 1, size(eles)
    call pointer_to_attribute (eles(ie)%ele, attrib_name, .false., attrib_ptr, err)
    if (err) exit      ! Do nothing on an error
    if (associated(attrib_ptr%r)) then  ! Real attribute
      attrib_ptr%r = set_value  ! Set the attribute
    elseif (associated(attrib_ptr%i)) then ! Integer attribute
      attrib_ptr%i = nint(set_value)
    else
      print *, 'Not a real or integer type attribute!'
    endif
  enddo

  call lattice_bookkeeper (lat)  ! Bookkeeping needed due to parameter change 
\end{example}

changing an element attribute generally involves changing values in the 
\vn{%ele(i)%value(:)} array. This is done using the 
\Hyperref{r:set.ele.attribute}{set_ele_attribute} routine. For example:
\begin{example}
  type (lat_struct) lat
  type (ele_pointer_struct), allocatable :: eles(:)
  integer n_loc, n
  logical err_flag, make_xfer_mat
  ...
  call lat_ele_locator ('Q01W', lat, eles, n_loc, err_flag)
  do n = 1, n_loc
    call set_ele_attribute (eles(n)%ele, 'K1 = 0.1', lat, err_flag)
  enddo
\end{example}
\index{overlay}
This example sets the \vn{K1} attribute of all elements named \vn{Q01W}.
\vn{set_ele_attribute} checks whether an element is actually free to
be varied and sets the \vn{err_flag} logical accordingly. An element's
attribute may not be freely varied if, for example, the attribute is
controlled via an \vn{Overlay}.



