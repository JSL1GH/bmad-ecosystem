\chapter{HDF5}
\label{c:hdf5}
\index{hdf5}

\vn{HDF5}, which stands for ``Hierarchical Data Format'' version 5\cite{b:hdf5}, is a set of file
formats designed to store and organize large amounts of data. HDF5 has been developed by scientists
from a number of institutions including the National Center for Supercomputing Applications, the
University of Illinois at Urbana-Champaign, and Sandia National Laboratories. Tools for viewing and
editing HDF5 files are available from the HDF Group\cite{b:hdf5}. Programs include \vn{h5dump} and
\vn{HDFView} which can be used to directly view files. Interfaces so that HDF5 files can accessed
via Java or Python also exist.

\bmad uses HDF5 for storing beam particle (positions, spin, etc.) and \vn{grid_field}
(\sref{s:grid.field} data. Storage details are given in sections \sref{s:hdf5.beam} and
\sref{s:hdf5.grid} respectively. While \vn{HDF5} defines how data is formatted, \vn{HDF5} does not
define the syntax for how data is to be stored. For that, \bmad uses the syntax defined by the
\vn{Beam Physics} extension to the \vn{openPMD} standard\cite{b:openpmd}. To understand the rest of
this chapter, the reader should familiarize themselves with the \vn{openPMD} and \vn{Beam Physics}
standards.

%-----------------------------------------------------------------
\section{HDF5 Particle Beam Data Storage}
\label{s:hdf5.beam}
\index{hdf5 and particle beam data}

As per the \vn{openPMD}/\vn{Beam Physics} standard, particle beam data is stored in a tree structure
within a data file. The root ``\vn{group}'' (tree node) for each bunch of the beam has the path
within the file:
\begin{example}
  /data/%T/particles/
\end{example}
where \vn{%T} is an integer starting from ``1''.

Bunch parameters (``attributes'') stored in the root groop are:
\begin{example}
  speciesType     ! The name of the particle species using the \vn{SpeciesType} syntax.
  totalCharge     ! Total bunch charge.
  chargeLive      ! Charge of live particles.
  numParticles    ! Number of particles.
\end{example}

What per-particle data is stored is determined by whether the bunch particles are photons or
not. The following particle parameters are common for both types:
\begin{tabular}{lll} \toprule
\tt
  {\em Beam Physics Parameter} & {\em Bmad Equivalent}           & {\em Formula}         & {\em Notes}                    \midrule
  time                  & time - ref_time                        & -\%z / (c \%beta)     & See \Eq{zbctt}                 \\
  timeOffset            & reference time                         & \%t - time            &                                \\
  sPosition             & Position along $s$-axis                & \%s                   & See Fig.~\ref{f:local.coords}  \\
  weight                & macro bunch charge                     & \%charge              &                                \\
  branchIndex           & Lattice branch index                   & \%ix_branch           &                                \\
  elementIndex          & Lattice element index                  & \%ix_ele              &                                \\
  locationInElement     & Location in lattice element            & See below             &                                \\
  particleStatus        & Particle status                        & \%state               & See state table in \sref{s:coord.struct} \bottomrule
\end{tabular}
The \vn{formula} column gives the conversion between the Beam Physics Parameters and the
\vn{coord_struct} (\sref{s:coord.struct}) structure that contains the particle position information.
parameters with a ``\%'' suffix are \vn{coord_struct} components. For clarity's sake, the
\vn{%vec(1)} through \vn{%vec(6)} phase space coordinate components has been replaced in this table
and the tables below by \vn{%x}, \vn{%px}, $\ldots$, \vn{%z}, \vn{%pz}.

The \vn{locaionInElement} Beam Physics parameter is related to the \vn{%location} parameter of the \vn{coord_struct} using the following
transformation:
\begin{tabular}{rll} \toprule
\tt
  {\em locationInElement} & {\em \%location}    \midrule
  -1                      & upstream_end\\\$    \\
   0                      & inside\\\$          \\
   1                      & downstream_end\\\$  \bottomrule
\end{tabular}

The \vn{particleStatus} Beam Physics parameter is related to the \vn{

For photons, additional per-particle data is:
\begin{example}
  photonPolarizationAmplitude/x, y & Polarization                & 
  photonPolarizationPhase
  velocity/x, y, z
  position/x, y, z      & particle laboratory coordinates        & (\%x, \%y, \%z)      & See Fig.~\ref{f:local.coords}  \\
  pathLength
  totalMomentumOffset
\end{example}

and for non-photons, additional per-particle data is:
\begin{example}
  momentum/x, y, z
  totalMomentumOffset
  totalMomentum
  position/x, y, z      & particle laboratory coordinates        & (\%x, \%y, 0)            & See Fig.~\ref{f:local.coords}  \\
  spin/x, y, z
  chargeState           ! Used with non-fundamental particles only.
\end{example}


%-----------------------------------------------------------------
\section{HDF5 Grid\_Field Data Storage
\label{s:hdf5.grid}
\index{hdf5 and grid_field data}

