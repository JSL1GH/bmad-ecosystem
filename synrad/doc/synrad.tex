\documentclass[11pt]{article}
%%\documentclass{book}
\usepackage{geometry}            % See geometry.pdf to learn the layout options. There are lots.
\usepackage{xspace}
\geometry{letterpaper}           % ... or a4paper or a5paper or ... 
%\geometry{landscape}            % Activate for for rotated page geometry
%\usepackage[parfill]{parskip}   % To begin paragraphs with an empty line rather than an indent
\usepackage{graphicx}
\usepackage{amssymb}
\usepackage{alltt}
%\usepackage{epstopdf}
%\DeclareGraphicsRule{.tif}{png}{.png}{`convert #1 `dirname #1`/`basename #1 .tif`.png}
\usepackage[T1]{fontenc}   % so _, <, and > print correctly in text.
\usepackage[strings]{underscore}    % to use "_" in text

\usepackage{hyperref}

%---------------------------------------------------------------------------------

\newcommand{\sref}[1]{$\S$\ref{#1}}
\newcommand\ttcmd{\begingroup\catcode`\_=11 \catcode`\%=11 \dottcmd}
\newcommand\dottcmd[1]{\texttt{#1}\endgroup}
\newcommand{\Begineq}{\begin{equation}}
\newcommand{\Endeq}{\end{equation}}
\newcommand{\fig}[1]{Figure~\ref{#1}}
\newcommand{\vn}{\ttcmd}           
\newcommand{\Th}{$^{th}$\xspace}
\newcommand{\Newline}{\hfil \\}

\newlength{\dPar}
\newlength{\ExBeg}
\newlength{\ExEnd}
\setlength{\dPar}{1.5ex}
\setlength{\ExBeg}{-\dPar}
\addtolength{\ExBeg}{-0.5ex}
\setlength{\ExEnd}{-\dPar}
\addtolength{\ExEnd}{-0.0ex}

\newenvironment{example}
  {\vspace{\ExBeg} \begin{alltt}}
  {\end{alltt} \vspace{\ExEnd}}

%---------------------------------------------------------------------------------

\setlength{\textwidth}{6.25in}
\setlength{\hoffset}{0.0in}
\setlength{\oddsidemargin}{0.25in}
\setlength{\evensidemargin}{0.0in}
\setlength{\textheight}{8.5in}
\setlength{\topmargin}{0in}

\setlength{\parskip}{\dPar}
\setlength{\parindent}{0ex}

%---------------------------------------------------------------------------------

\title{ Synrad Information}
\author{}
\date{October 1, 2010}

\begin{document}
\maketitle

%------------------------------------------------------------------
\section{Introduction} 

Synrad is a program for calculating the synchrotron radiation power
deposition on the beam chamber walls in a storage ring or
accelerator. Synrad works by tracking photons from creation at the
beam to absorbion at the chamber wall. Tracking is in the horizontal
plane only and reflections from the wall are not considered.  The
vertical extent of the radiation stripe at the wall, important for
calculating the power deposition per unit area, is calculated using
the vertical emittance.

The chamber wall is specified in the horizontal plane by a set of
vertex points with a straight line between points. Exit and entrance
lines like x-ray beam lines or dumps can be modeled in
synrad. Shadowing of parts of the wall by other parts is taken into
account in the calculation.

Synrad output includes power deposition per longitudinal length, per
unit area on the beam center line and photon flux. The beam orbit is
not constrained to be zero so calculations to see the affect of
varying the orbit are possible.

Synrad is not to be confused with another program called Synrad3D.
The Synrad3D program uses a full three dimensional chamber model along
with a model for the scattering the photons. The aim here is to
simulate

%------------------------------------------------------------------
\section{Main Input File} 

The main input file can be specified on the command line invoking synrad3d.
If not given, the default name for the main input file is ``\vn{synrad.init}''.
Example main input file:
\begin{example}
  &synrad_params
    sr_param%lat_file = "lat.bmad"   ! Input lattice.
    sr_param%i_beam = 0.1            ! Single-beam current.
    sr_param%epsilon_y = 10e-12      ! Vertical emittance.
    sr_param%n_slice = 20            ! # of slices per element or wiggler pole
    seg_len = 0.1                    ! Segment length for calculation.
    beam_direction = 0               ! -1 = track backwards only,
                                     !  0 = track both directions, 1 = forward.
    wall_file = "wall.dat"           ! "NONE" => Use a wall with a fixed offset from the beam.
    wall_offset = 0.045              ! Used when wall_file is set to "NONE"
    forward_beam  = "POSITRON"       ! "POSITRON" or "ELECTRON"
    backward_beam = "ELECTRON"       !  This is important if there are elsep elements.
    use_ele_ix = 0   ! If using only a single element, this is the element index number to use
          ! Otherwise, use 0 for power from all elements
  /
\end{example}
 Fortran namelist input is used.  The tag \vn{"\&synrad3d_parameters"}
marks the start of the namelist and the namelist ends with the slash
\vn{"/"} tag. Anything outside of this is ignored. Within the
namelist, anything after an exclamation mark \vn{"!"} is ignored
including the exclamation mark.

  \begin{description}
  \item[\vn{sr_param\%lat_file}] \Newline
The \vn{lat_file_name} parameter specifies the lattice file to be
used.  Lattices are in Bmad standard format~\cite{b:bmad.web}.
  \item[\vn{sr_param\%i_beam}] \Newline
  \item[\vn{sr_param\%epsilon_y}] \Newline
  \item[\vn{sr_param\%n_slice}] \Newline
  \item[\vn{sr_param\%seg_len}] \Newline
  \item[\vn{beam_direction}] \Newline
  \item[\vn{wall_file}] \Newline
  \item[\vn{wall_offset}] \Newline
  \item[\vn{forward_beam}] \Newline
  \item[\vn{backward_beam}] \Newline
  \item[\vn{use_ele_ix}] \Newline
  \end{description}

%------------------------------------------------------------------
\section{Vacuum Chamber Wall Definition} 


%------------------------------------------------------------------
\section{Simulation Technique} 

Photon generation is based on the standard synchrotron radiation
formulas, applicable for dipoles quadrupoles, and wigglers. The
radiation is assumed to be incoherent, so this program cannot treat
undulator radiation. 



%------------------------------------------------------------------
\begin{thebibliography}{9}

\bibitem[Bmad]{b:bmad.web}
The Bmad web site:
\hfill\break
\hspace*{0.3in} \url{http://www.lepp.cornell.edu/~dcs/bmad}

\end{thebibliography}

\end{document}  
