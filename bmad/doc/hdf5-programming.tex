\chapter{HDF5}
\label{c:hdf5}
\index{hdf5}

\vn{HDF5}, which stands for ``Hierarchical Data Format'' version 5\cite{b:hdf5}, is a set of file
formats designed to store and organize large amounts of data. HDF5 has been developed by scientists
from a number of institutions including the National Center for Supercomputing Applications, the
University of Illinois at Urbana-Champaign, and Sandia National Laboratories. Tools for viewing and
editing HDF5 files are available from the HDF Group\cite{b:hdf5}. Programs include \vn{h5dump} and
\vn{HDFView} which can be used to directly view files. Interfaces so that HDF5 files can accessed
via Java or Python also exist.

\bmad uses HDF5 for storing beam particle (positions, spin, etc.) and \vn{grid_field}
(\sref{s:grid.field}) data. Storage details are given in sections \sref{s:hdf5.beam} and
\sref{s:hdf5.grid} respectively. While \vn{HDF5} defines how data is formatted, \vn{HDF5} does not
define the syntax for how data is to be stored. For that, \bmad uses the syntax defined by the
\vn{Beam Physics} extension to the \vn{openPMD} standard\cite{b:openpmd}. To understand the rest of
this chapter, the reader should familiarize themselves with the \vn{openPMD} and \vn{Beam Physics}
standards.

%-----------------------------------------------------------------
\section{HDF5 Particle Beam Data Storage}
\label{s:hdf5.beam}
\index{hdf5 and particle beam data}

The code for reading and writing beam data to/from HDF5 files is contained in the routines
\Hyperref{r:hdf5.read.beam}{hdf5_read_beam} and \Hyperref{r:hdf5.write.beam}{hdf5_write_beam}.

As per the \vn{openPMD}/\vn{Beam Physics} standard, particle beam data is stored in a tree structure
within a data file. The root ``\vn{group}'' (tree node) for each bunch of the beam has the path
within the file:
\begin{example}
  /data/%T/particles/
\end{example}
where \vn{%T} is an integer.

For any bunch, parameters (``attributes'') stored in the bunch root group are:
\begin{example}
  speciesType     ! The name of the particle species using the \vn{SpeciesType} syntax.
  totalCharge     ! Total bunch charge.
  chargeLive      ! Charge of live particles.
  numParticles    ! Number of particles.
\end{example}
The \vn{SpeciesType} syntax defined by the \vn{SpeciesType} extension to the \vn{openPMD} standard
is similar to the \bmad standard (\sref{s:param}) but there are differences. For one, the
\vn{SpeciesType} standard does not have an encoding for the charge state of atoms and
molecules. Another difference is that for fundamental particles the names are case sensitive while
for \bmad they are not (Note that atom and molecule names in \bmad are case sensitive).

What per-particle data is stored is determined by whether the bunch particles are photons or
not. The following particle parameters are common for both types:
\begin{center}
\tt
\begin{tabular}{lll} \toprule
  {\em Beam Physics Parameter} & {\em Bmad Equivalent}   & {\em Notes}                      \\ \midrule
  time                         & -\%vec(5) / (c \%beta)  & time - ref_time. See \Eq{zbctt}  \\
  timeOffset                   & \%t - time              & reference time                   \\
  totalMomentumOffset          & \%p0c                   &                                  \\
  sPosition                    & \%s                     & See Fig.~\ref{f:local.coords}    \\
  weight                       & \%charge                & Macro bunch charge               \\
  branchIndex                  & \%ix_branch             &                                  \\
  elementIndex                 & \%ix_ele                &                                  \\
  locationInElement            & \%location              & See below                        \\
  particleStatus               & \%state   & See the \%state table in \sref{s:coord.struct} \\ \bottomrule
\end{tabular}
\end{center}
The \vn{Bmad Equivalent} column gives the conversion between the Beam Physics parameters and the
\vn{coord_struct} (\sref{s:coord.struct}) structure components (the \vn{coord_struct} structure
contains the particle position information).  Parameters with a ``\%'' suffix are \vn{coord_struct}
components and \vn{%vec(5)} corresponds to the phase space $z$ coordinate. The \vn{particleState} is an integer
which corresponds to the \vn{coord_struct} \vn{%state} component. A value of 1 indicates that the particle is alive
(corresponding to the value of \vn{alive\$}) and any other value indicates that the particle is dead.

The \vn{locationInElement} Beam Physics parameter is related to the \vn{coord_struct} \vn{%location} parameter via
the following transformation:
\vspace{-1ex}
\begin{center}
\tt
\begin{tabular}{ll} \toprule
  {\em locationInElement Value} & {\em \%location Value}  \\ \midrule
  -1                            & upstream_end\$          \\
   0                            & inside\$                \\
   1                            & downstream_end\$        \\ \bottomrule
\end{tabular}
\end{center}

For photons, additional per-particle data is:
\vspace{-1ex}
\begin{center}
\tt
\begin{tabular}{ll} \toprule
  {\em Beam Physics Parameter}     & {\em Bmad Equivalent}  \\ \midrule
  velocity/x, y, z                 & (\%vx, \%vy, \%vz)     \\
  position/x, y, z                 & (\%x, \%y, \%z)        \\
  pathLength                       & \%path_len             \\
  photonPolarizationAmplitude/x, y & \%field                \\
  photonPolarizationPhase/x, y     & \%phase                \\ \bottomrule
\end{tabular}
\end{center}
For clarity's sake, the \vn{%vec(1)} through \vn{%vec(6)} phase space coordinate components in the
\vn{coord_struct} have been replaced by \vn{%x}, \vn{%vx}, $\ldots$, \vn{%z}, \vn{%vz} in the above table

For non-photons, additional per-particle data is:
\vspace{-1ex}
\begin{center}
\tt
\begin{tabular}{ll} \toprule
  {\em Beam Physics Parameter}    & {\em Bmad Equivalent}     \\ \midrule
  momentum/x, y, z                & \%p0c$\times$(\%px, \%py, sqrt((1 + \%pz)$^2$ - \%px$^2$ - \%py$^2$)) \\
  totalMomentum                   & \%p0c$\times$\%pz         \\
  position/x, y, z                & (\%x, \%y, 0)             \\
  spin/x, y, z                    & \%spin                    \\
  chargeState                     & Derived from \%species    \\ \bottomrule
\end{tabular}
\end{center}
For clarity's sake, the \vn{%vec(1)} through \vn{%vec(6)} phase space coordinate components in the
\vn{coord_struct} have been replaced by \vn{%x}, \vn{%px}, $\ldots$, \vn{%z}, \vn{%pz} in the above
table. Notice that the Beam Physics \vn{z} position (not to be confused with phase space \vn{z}) is
always zero by construction as shown in Fig.~\ref{f:local.coords}. 

%-----------------------------------------------------------------
\section{HDF5 Grid\_Field Data Storage}
\label{s:hdf5.grid}
\index{hdf5 and grid_field data}

The code for reading and writing \vn{grid_field} data to/from HDF5 files is contained in the
routines \Hyperref{r:hdf5.read.grid.field}{hdf5_read_grid_field} and
\Hyperref{r:hdf5.write.grid.field}{hdf5_write_grid_field}.

As per the \vn{openPMD}/\vn{Beam Physics} standard, \vn{grid_field} (\sref{s:grid.field} data is
stored in a tree structure within a data file. The root ``\vn{group}'' (tree node) for each \vn{grid_field}
has the path within the file:
\begin{example}
  /ExernalFieldmesh/%T/
\end{example}
where \vn{%T} is an integer.

For any \vn{grid_field}, parameters stored in the \vn{grid_field} root group are:
\vspace{-1ex}
\begin{center}
\tt
\begin{tabular}{lll} \toprule
  {\em Parameter in File}      & {\em Bmad Equivalent}      \\ \midrule
  gridGeometry                 & \%geometry                 \\ 
  masterParameter              & \%master_parameter         \\ 
  componentFieldScale          & \%field_scale              \\ 
  fieldScale                   & $\left\{ \text{
                                  \begin{tabular}{@{}ll}
                                    \%field_scale$\times$master param value & If master parameter set. \\
                                    \%field_scale & Otherwise.
                                 \end{tabular}} \right.$    \\ 
  harmonic                     & \%harmonic                 \\ 
  RFphase                      & $\left\{ \text{
                                 \begin{tabular}{@{}ll}
                                    \%harmonic$\times$\%phi0_fieldmap & For \vn{lcavity} elements \\
                                    \%harmonic$\times$(0.25 - \%phi0_fieldmap) & For all others.
                                 \end{tabular}} \right.$    \\ 
  eleAnchorPt                  & \%ele_anchor_pt            \\ 
  gridOriginOffset             & \%r0                       \\ 
  gridSpacing                  & \%dr                       \\ 
  interpolationOrder           & \%interpolation_order      \\ 
  gridLowerBound               & \%ptr\%pt lower bound      \\ 
  gridSize                     & \%ptr\%pt size             \\ 
  fundamentalFrequency         & ele%value(rf_frequency$)   \\
  gridCurvatureRadius          & ele%value(rho$)            \\ \bottomrule
\end{tabular}
\end{center}
The \vn{Bmad Equivalent} column gives the conversion between the Beam Physics parameters and the
\vn{grid_field_struct} structure components (that have a ``\%'' prefix). The value for
\vn{gridCurvatureRadius} is set to the value of \vn{rho} of the associated lattice element if
\vn{%curved_ref_frame} is True. 

Notice that the \vn{masterParameter} attribute is not part of the standard. If not present, which
could happen if a file is created by non-\bmad code, the default is a blank string indicating no
master parameter. If \vn{masterParameter} is set in the data file, there is a potential problem in
that it may not be possible to calculate \vn{%field_scale} if the value of the master parameter is
not equal to the value when the data was written. To get around this, if the non-standard
\vn{masterParameter} is present, the value of the non-standard \vn{componentFieldScale} (which has a
default value of one) will be used to set \vn{%field_scale} and the \vn{fieldScale} parameter will
be ignored. If \vn{masterParameter} is not present, \vn{componentFieldScale} is ignored and
\vn{%field_scale} is set from the value of \vn{fieldScale}.

When reading a data file, the setting of \vn{grid_field%field_type} is determined by what data is
stored in the file. If both electric and magnetic field data is present, \vn{%field_type} is set to
\vn{mixed\$}. Otherwise, \vn{%field_type} is set to \vn{magnetic\$} if magnetic field data is present
or \vn{electric\$} if electric field data is present.

The correspondence between the \vn{gridGeometry} parameter and the \vn{grid_field%geometry}
component is \vspace{-1ex}
\begin{center}
\tt
\begin{tabular}{ll} \toprule
  {\em gridGeometry Value}      & {\em \%geometry Value}      \\ \midrule
  "rectangular"                 & xyz\$                       \\
  "cylindrical"                 & rotationally_symmetric_rz\$ \\ \bottomrule
\end{tabular}
\end{center}

