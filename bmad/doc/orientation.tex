\chapter{Orientation}
\label{c:orient}

\section{What is Bmad?}

\bmad is an open-source software library (aka toolkit) for simulating charged particles and
X-rays. \bmad is not a program itself but is used by programs for doing calculations. The advantage
of \bmad over a stand-alone simulation program is that when new types of simulations need to be
developed, \bmad can be used to cut down on the time needed to develop such programs with the added
benefit that the number of programming errors will be reduced.

Over the years, \bmad has been used for a wide range of charged-particle and X-ray simulations. This
includes:
\begin{example}
Lattice design                                  X-ray simulations
Spin tracking                                   Wakefields and HOMs
Beam breakup (BBU) simulations in ERLs          Touschek Simulations
Intra-beam scattering (IBS) simulations         Dark current tracking
Coherent Synchrotron Radiation (CSR)            Frequency map analysis
\end{example}

%---------------------------------------------------------------------------
\section{Tao and Bmad Distributions}
\label{s:tao.intro}
\index{Tao}

The strength of \bmad is that, as a subroutine library, it provides a flexible framework from which
sophisticated simulation programs may easily be developed. The weakness of \bmad comes from its
strength: \bmad cannot be used straight out of the box. Someone must put the pieces together into a
program. To remedy this problem, the \tao program\cite{b:tao} has been developed. \tao, which uses
\bmad as its simulation engine, is a general purpose program for simulating particle beams in
accelerators and storage rings. Thus \bmad combined with \tao represents the best of both worlds:
The flexibility of a software library with the ease of use of a program.

Besides the \tao program, an ecosystem of \bmad based programs has been developed. These programs,
along with \bmad, are bundled together in what is called a \bmad \vn{Distribution} which can be
downloaded from the web. The following is a list of some of the more commonly used programs.
  \begin{description}
%
  \item[bmad_to_mad_sad_elegant] \Newline
The \vn{bmad_to_mad_sad_elegant} program converts \bmad lattice format files to \vn{MAD8},
\vn{MADX}, \vn{Elegant} and \vn{SAD} format.
%
  \item[bbu] \Newline
The \vn{bbu} program simulates the beam breakup instability in Energy Recovery Linacs (ERLs). 
%
  \item[dynamic_aperture] \Newline
The \vn{dynamic_aperture} program finds the dynamic aperture through tracking.
%
  \item[ibs_linac] \Newline
The \vn{ibs_linac} program simulates the effect of intra-beam scattering (IBS) for beams in a Linac.
%
  \item[ibs_ring] \Newline
The \vn{ibs_ring} program simulates the effect of intra-beam scattering (IBS) for beams
in a ring.
%
  \item[long_term_tracking] \Newline
The \vn{long_term_tracking_program} is for long term tracking of a particle or beam possibly
including tracking of the spin.
%
  \item[lux] \Newline
The \vn{lux} program simulates X-ray beams from generation through to experimental end stations.
%
  \item[mad8_to_bmad.py, madx_to_bmad.py] \Newline
These python programs will convert \vn{MAD8} and \vn{MADX} lattice files to to \bmad format. 
%
  \item[moga] \Newline
The \vn{moga} (multiobjective genetic algorithms) program does multiobjective optimization.
%
  \item[synrad] \Newline
The \vn{synrad} program computes the power deposited on the inside of a vacuum chamber
wall due to synchrotron radiation from a particle beam. The calculation is essentially two
dimensional but the vertical emittance is used for calculating power densities along the
centerline. Crotch geometries can be handled as well as off axis beam orbits. 
%
  \item[synrad3d] \Newline
The \vn{synrad3d} program tracks, in three dimensions, photons generated from a beam
within the vacuum chamber. Reflections at the chamber wall is included.
%
  \item[tao] \Newline
\tao is a general purpose simulation program.
  \end{description}

%---------------------------------------------------------------------------
\section{Resources: More Documentation, Obtaining Bmad, etc.}
\label{s:bmad.web}

More information and download instructions are readily available at the \bmad web site:
\begin{example}
  \url{\detokenize{www.classe.cornell.edu/bmad/}}
\end{example}
Links to the most up-to-date \bmad and \tao manuals can be found there as well as manuals for other
programs and instructions for downloading and setup.

The \bmad manual is organized as reference guide and so does not do a good job of instructing the
beginner as to how to use \bmad. For that there is an introduction and tutorial on \bmad and \tao
(\sref{s:tao.intro}) concepts that can be downloaded from the \bmad web page. Go to either the \bmad or
\tao manual pages and there will be a link for the tutorial.

%---------------------------------------------------------------------------
\section{PTC: Polymorphic Tracking Code}
\label{s:ptc.intro}
\index{PTC}

The PTC/FPP library of \'Etienne Forest handles Taylor maps to any arbitrary order. This is also
known as Truncated Power Series Algebra (TPSA). The core Differential Algebra (DA) package used by
PTC/FPP was developed by Martin Berz\cite{b:berz}. The PTC/FPP libraries are interfaced to \bmad so
that calculations that involve both \bmad and PTC/FPP can be done in a fairly seamless manner.

Basically, the FPP (``Fully Polymorphic Package'') part of the PTC/FPP code handles Taylor map
manipulation. This is purely mathematical. FPP has no knowledge of accelerators, magnetic fields,
particle tracking etc. PTC (``Polymorphic Tracking Code'') implements the physics and uses FPP to
handle the Taylor map manipulation. Since the distinction between \vn{FPP} and \vn{PTC} is
irrelevant to the non-programmer, ``PTC'' will be used to refer to the entire PTC/FPP package.

PTC is used by \bmad when constructing Taylor maps and when the \vn{tracking_method}
\sref{s:tkm}) is set to \vn{symp_lie_ptc}. All Taylor maps above first order are calculated
via PTC. No exceptions.

For more discussion of PTC see Chapter~\sref{c:ptc.use}. For the programmer, also see
Chapter~\sref{c:ptc.program}.

For the purposes of this manual, PTC and FPP are generally considered one package and
the combined PTC/FPP library will be referred to as simply ``PTC''.

