\chapter{Lattice File Global Parameters}

This chapter deals with statements that can be used to set ``global'' parameter values. That is,
parameter values that are associated with the lattice as a whole and not simply associated with a
single element.

Discussed elsewhere are the global structures (\sref{s:struct}) \vn{bmad_com} 
and \vn{ptc_com} (\sref{s:bmad.ptc.com}).

%-----------------------------------------------------------------------------
\section{Parameter Statements}
\label{s:param}
\index{parameter statement|hyperbf}


\index{lattice}\index{geometry}\index{photon_type}\index{live_branch}
\index{taylor_order}\index{e_tot}\index{p0c}\index{ran_seed}
\index{n_part}\index{no_end_marker}
\vn{Parameter} statements are used to set a number of global variables.  If multiple branches are
present (\sref{s:branch.def}), these variables pertain to the \vn{root} branch. The variables that
can be set by \vn{parameter} are
\begin{example}
  parameter[custom_attributeN]        = <string>    ! Defining custom attributes (\sref{s:cust.att}).
  parameter[default_tracking_species] = <Switch>    ! Default type of tracked particle. 
                                                    !    Default is ref_particle.
  parameter[e_tot]                    = <Real>      ! Reference total Energy. 
                                                    !      Default: 1000 * rest_energy.
  parameter[electric_dipole_moment]   = <Real>      ! Particle electric dipole moment.
  parameter[live_branch]              = <Logical>   ! Is branch fit for tracking?
  parameter[geometry]                 = <Switch>    ! Open or closed
  parameter[lattice]                  = <String>    ! Lattice name.
  parameter[machine]                  = <String>    ! Machine name.
  parameter[n_part]                   = <Real>      ! Number of particles in a bunch.
  parameter[no_end_marker]            = <Logical>   ! Default: False.
  parameter[p0c]                      = <Real>      ! Reference momentum.
  parameter[particle]                 = <speciesID> ! Reference species: positron, proton, etc.
  parameter[photon_type]              = <Switch>    ! Incoherent or coherent photons?
  parameter[ran_seed]                 = <Integer>   ! Random number generator init.
  parameter[taylor_order]             = <Integer>   ! Default: 3
\end{example}

\noindent
Examples
\begin{example}
  parameter[lattice]      = "L9A19C501.FD93S_4S_15KG"
  parameter[geometry]     = closed
  parameter[taylor_order] = 5
  parameter[E_tot]        = 5.6e9    ! eV
\end{example}

  \begin{description}
%
  \index{custom_attributeN}
  \item[{parameter[custom_attributeN]}] \Newline
Here \vn{N} is an integer between 1 and 40.  For more information on defining custom attributes, see
\sref{s:cust.att}.
%
  \index{machine}
Name of the machine the lattice simulates. Example: "LHC".
%
  \index{live_branch}
  \item[{parameter[live_branch]}] \Newline
Setting \vn{live_branch} to \vn{False} (default is \vn{True}) indicates to a program that no
tracking or other analysis of the root branch should be done. This can be useful if the lattice has
multiple branches and analysis of the root branch is not necessary. Other branches can also be
marked as alive/dead using line parameter statements (\sref{s:beginning}). Note that the \bmad
library itself ignores the setting of \vn{live_branch} and it is up to the program being run to
decide if this parameter is ignored or not. In particular, the \tao program (\sref{s:tao.intro})
{\em will} respect the setting of \vn{live_branch}.
%
  \index{default_tracking_species}
  \item[{parameter[default_tracking_species]}] \Newline
The \vn{parameter[default_tracking_species]} switch establishes the default type of particles to be
tracked. See \sref{s:species.name} for the syntax for naming particle species. In addition, this switch
can be set to:
\begin{example}
  ref_particle     ! default
  anti_ref_particle
\end{example}
By default, \vn{default_tracking_species} is set to \vn{ref_particle} so that the particle being
tracked is the same as the reference particle set by \vn{param[particle]}. In the case, for example,
where there are particles going one way and antiparticles going the another,
\vn{default_tracking_species} can be used to switch between tracking the particles or antiparticles.
%
  \index{e_tot}\index{p0c}\index{lcavity}\index{patch}
  \item[{parameter[e_tot], parameter[p0c]}] \Newline
The \vn{parameter[e_tot]} and \vn{parameter[p0c]} are the reference total energy and momentum at the
start of the lattice. Each element in a lattice has an individual reference \vn{e_tot} and \vn{p0c}
attributes which are dependent parameters. The reference energy and momentum will only change
between \vn{LCavity} or \vn{Patch} elements. The starting reference energy, if not set, will be set
to 1000 time the particle rest energy.  Note: \vn{beginning[e_tot]} and \vn{beginning[p0c]}
(\sref{s:beginning}) are equivalent to \vn{parameter[e_tot]} and \vn{parameter[p0c]}.
%
  \index{electric_dipole_moment}
  \item[{parameter[electric_dipole_moment]}] \Newline
The \vn{electric_dipole_moment} sets the electric dipole moment value $\eta$ for use when tracking
with spin (\sref{s:spin.dyn}).
%
  \index{geometry}\index{closed}\index{open}\index{lcavity!and geometry}
  \item[{parameter[geometry]}] \Newline
Valid \vn{geometry} settings are
\begin{example}
  closed  ! Default w/o LCavity element present.
  open    ! Default if LCavity elements present.
\end{example}
A machine with a \vn{closed} geometry is something like a storage ring where the particle beam
recirculates through the machine.  A machine with an \vn{open} geometry is something like a linac.
In this case, if the reference particle is not a photon, the initial Twiss parameters need to be
specified in the lattice file using the \vn{beginning} statement (\sref{s:beginning}). If the
\vn{geometry} is not specified, \vn{closed} is the default. The exception is that if there is an
\vn{Lcavity} element present or the reference particle is a photon, \vn{open} will be the default.

Notice that by specifying a \vn{closed} geometry it does {\em not} mean that the downstream end of
the last element of the lattice has the same global coordinates (\sref{s:global}) as the global
coordinates at the beginning. Setting the geometry to \vn{closed} simply signals to a program to
compute closed orbits and periodic Twiss parameters as opposed to calculating orbits and Twiss
parameters based upon initial orbit and Twiss parameters at the beginning of the lattice.  Indeed,
it is sometimes convenient to treat lattices as closed even though there is no closure in the global
coordinate sense. For example, when a machine has a number of repeating ``periods'', it may be
convenient to only use one period in a simulation. Since \bmad ignores closure in the global
coordinate sense, it is up to the lattice designer to ensure that a lattice is truly geometrically
closed if that is desired.
%
  \index{lattice}
  \item[{parameter[lattice]}] \Newline
Used to set the lattice name. The \vn{lattice} name is stored by \bmad for use by a program but it
does not otherwise effect any \bmad routines.
%
  \index{n_part}\index{beambeam}\index{lcavity}
  \item[{parameter[n_part]}] \Newline
The \vn{parameter[n_part]} is the number of particle in a bunch. This parameter is used in a number
of calculations, for example, with intrabeam scattering and to calculate the change in energy
through an \vn{Lcavity} (\sref{s:lcav}). Historically, this parameter has been used to set the
number of strong beam particle with \vn{BeamBeam} elements but it is strongly recommended to use
the \vn{beambeam} element's \vn{n_particle} parameter instead.
%
  \index{no_end_marker}\index{end element}
  \item[{parameter[no_end_marker]}] \Newline
Setting \vn{parameter[no_end_marker]} to True will suppress the automatic inclusion
of a marker named \vn{END} at the end of the lattice (\sref{s:branch.construct}). 
%
  \item[{parameter[p0c]}] \Newline
See \vn{parameter[e_tot]}.
%
  \index{particle}
  \item[{parameter[particle]}] \Newline
The \vn{parameter[particle]} switch sets the reference species. See \sref{s:species.name} for the
syntax for naming particle species.

The setting of the reference particle is used, for example, to determine the direction of the field
in a magnet and given the normalized field strength (EG: \vn{k1} for a quadrupole).  Generally, the
particles that by default are tracked through a lattice are the same as the reference particle. This
default behavior can be altered by setting \vn{parameter[default_tracking_species]}.

%
  \index{photon_type}
  \item[{parameter[photon_type]}] \Newline
The \vn{photon_type} switch is used to set the type of photons that are used in tracking. Possible
settings are:
\begin{example}
  incoherent    ! Default
  coherent 
\end{example}
The general rule is use incoherent tracking except when there is a \vn{diffraction_plate} element in
the lattice.
%
  \index{ran_seed}
  \item[{parameter[ran_seed]}] \Newline
For more information on \vn{parameter[ran_seed]} see \sref{s:functions}.
%
  \index{taylor_order}
  \item[{parameter[taylor_order]}] \Newline
The Taylor order (\sref{s:taylor.phys}) is set by \vn{parameter[taylor_order]} and is the maximum
order for a Taylor map.
  \end{description}

%-----------------------------------------------------------------------------
\section{Particle_Start Statements} 
\label{s:beam.start}
\index{particle_start statement|hyperbf}

\index{e_gun}
\index{x}\index{px}\index{y}\index{py}\index{z}\index{pz}
\index{emittance_a}\index{emittance_b}\index{emittance_z}
\index{field_x}\index{field_y}\index{phase_x}\index{phase_y}
\index{spin_x}\index{spin_y}\index{spin_z}
\vn{particle_start} statements are used, among other things to set the starting coordinates
for particle tracking. If multiple branches are present (\sref{s:branch.def}), these
variables pertain to the \vn{root} branch.
\begin{example}
  particle_start[x]                   = <Real>   ! Horizontal position.
  particle_start[px]                  = <Real>   ! Horizontal momentum.
  particle_start[y]                   = <Real>   ! Vertical position.
  particle_start[py]                  = <Real>   ! Vertical momentum.
  particle_start[z]                   = <Real>   ! Longitudinal position.
  particle_start[pz]                  = <Real>   ! Momentum deviation. Only for non-photons.
  particle_start[direction]           = +/-1     ! Longitudinal direction of travel.
  particle_start[E_photon]            = <Real>   ! Energy (eV). Only used for photons.
  particle_start[emittance_a]         = <Real>   ! A-mode emittance
  particle_start[emittance_b]         = <Real>   ! B-mode emittance
  particle_start[emittance_z]         = <Real>   ! Z-mode emittance
  particle_start[sig_z]               = <Real>   ! Beam sigma in z-direction
  particle_start[sig_pz]              = <Real>   ! Beam Sigma pz 
  particle_start[field_x]             = <Real>   ! Photon beam field along x-axis
  particle_start[field_y]             = <Real>   ! Photon beam field along y-axis
  particle_start[phase_x]             = <Real>   ! Photon beam phase along x-axis
  particle_start[phase_y]             = <Real>   ! Photon beam phase along y-axis
  particle_start[t]                   = <Real>   ! Absolute time
  particle_start[spin_x]              = <Real>   ! Spin polarization x-coordinate
  particle_start[spin_y]              = <Real>   ! Spin polarization y-coordinate
  particle_start[spin_z]              = <Real>   ! Spin polarization z-coordinate
\end{example}
Normally the absolute time, set by \vn{particle_start[t]}, is a dependent
parameter set by solving \Eq{zbctt} for $t$. The exception is when the
initial velocity is zero. (This can happen if there is an \vn{e_gun}
(\sref{s:e.gun}) element in the lattice). In this case, $z$ must be
zero and $t$ is an independent parameter that can be set.

The longitudinal direction of travel is set by \vn{particle_start[direction]}.  This can be set
to +1 (travel in the +s direction) or -1 for the reverse.  +1 is the default. Generally
\vn{particle_start[direction]} should not be set to -1 since most programs will not be
constructed to handle this situation. To track a particle in the reverse direction see
\sref{s:reverse}. 

For particles with spin, the spin can be specified using Cartesian coordinates with \vn{spin_x},
\vn{spin_y}, and \vn{spin_z}.

For photons, \vn{px}, \vn{py}, and \vn{pz} are the normalized velocity components
(Cf.~\Eq{xbybzb}). For photons \vn{pz} is a dependent parameter which will be set so that \Eq{bbb1}
is obeyed.

Note: \vn{particle_start} used to be called \vn{beam_start}. Since this was confusing (beam
initialization parameters are stored in a separate \vn{beam_init_struct} structure
(\sref{s:beam.init.struct})), the name was changed. Currently the use of the \vn{beam_start} name
is deprecated but still supported for backwards compatibility.

Example
\begin{example}
  particle_start[y] = 2 * particle_start[x]
\end{example}

%-----------------------------------------------------------------------------
\section{Beam Statement}
\label{s:beam.mad}
\index{beam statement|hyperbf}

\index{energy}
\index{particle}
\index{n_part}
\index{MAD!beam statement}
The \vn{beam} statement is provided for compatibility with \mad. The syntax is
\begin{example}
  beam, energy = GeV, pc = GeV, particle = <Switch>, n_part = <Real>
\end{example}
For example
\index{MAD}
\begin{example}
  beam, energy = 5.6  ! Note: GeV to be compatible with \mad
  beam, particle = electron, n_part = 1.6e10
\end{example}
Setting the reference energy using the \vn{energy} attribute is the
same as using \vn{parameter[e_tot]}. Similarly, setting \vn{pc} is
equivalent to setting \vn{parameter[p0c]}. Valid \vn{particle} switches
are the same as \vn{parameter[particle]}.

%--------------------------------------------------------------------------
\section{Beginning and Line Parameter Statements}
\label{s:beginning}
\index{beginning statement|hyperbf}

\index{beta_a}\index{alpha_a}
\index{phi_a}\index{eta_x}
\index{etap_x}\index{beta_b}
\index{alpha_b}\index{phi_b}
\index{eta_y}\index{etap_y}
\index{cmat_ij}\index{e_tot}
\index{p0c}\index{ref_time}
For non--circular lattices, the \vn{beginning} statement can be used to set the Twiss parameters and
beam energy at the beginning of the first lattice branch.
\begin{example}
  beginning[alpha_a]       = <Real>  ! "a" mode alpha
  beginning[alpha_b]       = <Real>  ! "b" mode alpha
  beginning[beta_a]        = <Real>  ! "a" mode beta
  beginning[beta_b]        = <Real>  ! "b" mode beta
  beginning[cmat_ij]       = <Real>  ! C coupling matrix. i, j = \{``1'', or ``2''\} 
  beginning[mode_flip]     = <logic> ! Set the mode flip status (\sref{s:coupling}). Default is False.
  beginning[e_tot]         = <Real>  ! Reference total energy in eV.
  beginning[eta_x]         = <Real>  ! x-axis dispersion
  beginning[eta_y]         = <Real>  ! y-axis dispersion
  beginning[etap_x]        = <Real>  ! x-axis dispersion derivative.
  beginning[etap_y]        = <Real>  ! y-axis dispersion derivative.
  beginning[p0c]           = <Real>  ! Reference momentum in eV.
  beginning[phi_a]         = <Real>  ! "a" mode phase.
  beginning[phi_b]         = <Real>  ! "b" mode phase.
  beginning[ref_time]      = <Real>  ! Starting reference time.
  beginning[s]             = <Real>  ! Longitudinal starting position.
  beginning[spin_dn_dpz_x] = <Real>   ! Spin dn/dpz x-coordinate
  beginning[spin_dn_dpz_y] = <Real>   ! Spin dn/dpz y-coordinate
  beginning[spin_dn_dpz_z] = <Real>   ! Spin dn/dpz z-coordinate
\end{example}
\index{e_tot}
The \vn{gamma_a}, \vn{gamma_b}, and \vn{gamma_c} (the coupling gamma factor) will be kept consistent
with the values set. If not set the default values are all zero.  \vn{beginning[e_tot]} and
\vn{parameter[e_tot]} are equivalent and one or the other may be set but not both. Similarly,
\vn{beginning[p0c]} and \vn{parameter[p0c]} are equivalent.

Setting \vn{etap_x} or \vn{etap_y} sets both types of dispersion derivatives (\sref{s:dispersion}). That
is, setting \vn{etap_x} sets both \vn{etap_x} and \vn{deta_x_ds} and similarly for \vn{etap_y}.

\index{x_position}\index{y_position}\index{z_position}
\index{theta_position}\index{phi_position}\index{psi_position}
For any lattice the \vn{beginning} statement can be used to set the starting floor position of the
first lattice branch (see~\sref{s:global}). The syntax is
\begin{example}
  beginning[x_position]     = <Real>  ! X position
  beginning[y_position]     = <Real>  ! Y position
  beginning[z_position]     = <Real>  ! Z position
  beginning[theta_position] = <Real>  ! Angle on floor
  beginning[phi_position]   = <Real>  ! Angle of attack
  beginning[psi_position]   = <Real>  ! Roll angle
\end{example}
If the floor position is not specified, the default is to place
beginning element at the origin with all angles set to zero.

\index{root branch}\index{branch}
The \vn{beginning} statement is useful in situations where only parameters for
the first branch need be specified. If this is not the case, the parameters for
any branch can be specified using a statement of the form
\begin{example}
  line_name[parameter] = <Value>
\end{example}
This construct is called a \vn{line parameter} statement Here \vn{line_name} is the name of a line
and \vn{parameter} is the name of a parameter. The parameters that can be set here are the same
parameters that can be set with the \vn{beginning} statement with the additional parameters from the
\vn{parameter} statement:
\index{geometry}\index{particle}\index{default_tracking_species}
\begin{example}
  default_tracking_species
  geometry
  high_energy_space_charge_on
  live_branch
  particle
  inherit_from_fork
\end{example}
Example:
\begin{example}
  x_ray_fork: fork, to_line = x_ray
  x_ray: line = (...)
  x_ray[E_tot] = 100
\end{example}
The \vn{inherit_from_fork} logical is used to determine if the reference energy and Twiss parameters as calculated
from the \vn{fork} element defining the branch is used to set the beginning element of the branch. This parameter
is ignored if the fork element does not fork to the beginning element of the branch. The default is True. If any reference energy or momentum or any Twiss parameter is set, \vn{inherit_from_fork} is implicitly set to False.

Rules:
  \begin{enumerate}
  \item
The floor position of a line can only be set if the line is used for a 
\vn{root} \vn{branch}. 
  \item
Line parameters statements must come after the associated line. This
rule is similar to the rule that element attribute redefinitions must
come after the definition of the element.
 \end{enumerate}
