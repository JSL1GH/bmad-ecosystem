\chapter{Physics}

%-----------------------------------------------------------------
\section{Units}
\label{s:units}
\index{Units|textbf}

\bmad\ uses SI (Syst\'eme International) units as shown in
Table~\ref{t:units}.  Note that \MAD uses different units. For example,
\mad's unit of Particle Energy is GeV not eV.
\begin{table}[h]
\centering
\begin{tabular}{|l|l|} \hline
  {\em Quantity}     & {\em Units}       \\ \hline
  Angles             &    radians        \\ 
  Phase Angles (RF)  &    radians/2$\pi$ \\ 
  Frequency          &    Hz             \\ 
  Current            &    Amps           \\ 
  Kick               &    radians        \\ 
  Length             &    meters         \\ 
  Magnetic Field     &    Tesla          \\ 
  Particle Energy    &    eV             \\ 
  Voltage            &    Volts          \\ \hline
\end{tabular}
\caption{Physical units used by \bmad.}
\label{t:units}
\end{table}


%-----------------------------------------------------------------
\section{Constants}
\label{s:constants}
\index{Constants|textbf}

\bmad\ defines commonly used physical and mathematical constants
shown in Table~\ref{t:constants}.  All symbols use straight SI units
except for \vn{e_mass} and \vn{p_mass} which are provided for
compatibility with \MAD.

\begin{table}
\centering
\begin{tabular}{|l|l|l|l|} \hline
  {\em Symbol}   & {\em Value}       & {\em Units} &  {\em Name}     \\ \hline
  pi             & 3.14159265359          &        &                   \\
  twopi          & 2 * pi                 &        &                   \\
  fourpi         & 4 * pi                 &        &                   \\
  sqrt\_2        & 1.4142135623731        &        &                   \\
  m\_electron    & $0.51099906 \pow{6}$   & eV     & Electron mass     \\
  m\_proton      & $0.938271998 \pow{9}$  & eV     & Proton mass       \\
  c\_light       & $2.99792458 \pow{8}$   & m/s    & Speed of light    \\
  r\_e           & $2.8179380 \pow{-15}$  & m      & Electron radius   \\
  r\_p           & $1.5346980 \pow{-18}$  & m      & Proton radius     \\
  e\_charge      & $1.6021892 \pow{-19}$  & C      & Electron charge   \\
  h\_planck      & $6.626196 \pow{-34}$   & J/Hz   & Planck's constant \\
  h\_bar\_planck & $1.054591 \pow{-34}$   & J s    & Planck / $2\pi$   \\
  e\_mass        & $0.51099906 \pow{-3}$  & GeV    & Electron mass     \\
  p\_mass        & $0.938271998$          & GeV    & Proton mass     \\ \hline
\end{tabular}
\caption{Physical and mathematical constants recognized by \bmad.}
\label{t:constants}
\end{table}


%-----------------------------------------------------------------
\section{Magnetic Fields}
\label{s:fields}
\index{Magnetic fields|textbf}

Start with the assumption that the local magnetic field has no
longitudinal component (obviously this assumption does not work with,
say, a solenoid).  Following \MAD, the vertical magnetic field is
expanded in a Taylor series
\Begineq
  B_y(x, 0) = \sum_n B_n \, \frac{x^n}{n!}
\Endeq
Assuming that the reference orbit is locally straight 
(there are correction terms if the Reference Orbit is locally curved), the 
field up to $3^{rd}$ order is
\begin{alignat}{5}
  B_x &=           &&B_1 y \plus         &&B_2 \, xy       && \plus && \frac{1}{6} B_3 (3x^2 y - y^3) \plus \ldots \\
  B_y &= B_0 \plus &&B_1 x + \frac{1}{2} &&B_2 (x^2 - y^2) && \plus && \frac{1}{6} B_3 (x^3 - 3x y^2) \plus \ldots
\end{alignat}
The normalized integrated multipole $K_nL$ is used when specifying magnetic
multipole components
\Begineq
  K_nL \equiv \frac{q \, L \, B_n}{P_0}
\Endeq
$L \, B_n$ is the integrated multipole component over a length $L$,
and $P_0$ is the reference momentum. Note that $P_0/q$ is sometimes
written as $B\rho$. This is just an old notation where $\rho$ is the
bending radius of a particle with the reference energy in a field of
strength $B$. The kicks $\Delta p_x$ and $\Delta p_y$ that a
particle experiences going through a multipole field is
\begin{alignat}{5}
  \Delta p_x & = \frac{-q \, L \, B_y}{P_0} \\
             & = -K_0 L \;-\; 
             && K_1 L \, x \plus 
             \frac{1}{2} && K_2 L (y^2 - x^2) && \plus 
             && \frac{1}{6} K_3 L (3x y^2 - x^3) \plus \ldots \nonumber \\
  \Delta p_y & = \frac{q \, L \, B_x}{P_0} \\
             & =     
             && K_1 L \, y \plus 
             && K_2 L \, xy && \plus 
             && \frac{1}{6} K_3L (3x^2 y - y^3) \plus \ldots \nonumber 
\end{alignat}
A positive $K_1L$ quadrupole component gives
horizontal focusing and vertical defocusing. 

If the fields associated with a particular $B_n$ multipole component
are rotated in the $(x, y)$ plane by an angle $\phi_n$ the magnetic
field at a point $(x,y)$ can be expressed in complex notation as
\Begineq
  B_y(x,y) + i B_x(x,y) = 
                \frac{1}{n!} B_n e^{-i(n+1)\phi_n} \, e^{i n \theta} \, r^n 
\Endeq
where $(r, \theta)$ are the polar coordinates of the point $(x, y)$.

Another representation of the magnetic field used by \bmad\ divides
the fields into normal $b_n$ and skew $a_n$ components. In terms of
these components the magnetic field for the $n$\Th\ component is
\Begineq
  \frac{q \, L}{P_0} \, (B_y + i B_x) = (b_n + i a_n) \, (x + i y)^n
\Endeq
The conversion between $(a_n, b_n)$ and $(K_nL, \phi_n)$ is
\Begineq
  b_n + i a_n = \frac{1}{n!} \, K_nL \, e^{-i(n+1)\phi_n}
\Endeq
or
\begin{align}
  K_n L &= n! \, \sqrt{a_n^2 + b_n^2} \\
  \tan[(n+1) \phi_n] &= \frac{-a_n}{b_n}
\end{align}

When the $a_n$ and $b_n$ are associated with a physical element (as
opposed to the $a_n$ and $b_n$ associated with an \vn{AB_Multipole}),
a measurement radius $r_0$ and a scale factor $F$ are used to scale
the $a_n$ and $b_n$ according to the formula
\begin{align}
  a_n &\rightarrow 
        a_n \cdot F \cdot \frac{r_0^{n_\text{ref}}}{r_0^n} \nonumber \\
  b_n &\rightarrow 
        b_n \cdot F \cdot \frac{r_0^{n_\text{ref}}}{r_0^n}
\end{align}
where $r_0$ is set by the \vnea{radius} attribute of an element. $F$ and
$n_\text{ref}$ are set automatically depending upon the type of
element as shown in Table~\ref{t:ab}.

Note that the $n = 0$ component of an \vne{AB_Multipole} or \vne{Multipole}
element rotates the reference orbit essentially acting as a zero length bend.
This is not true for multipoles that are associated with 
non-multipole elements.

\begin{table}[h]
\centering
\begin{tabular}{|l|l|l|} \hline
\tt
  {\em Element} & $F$                              & $n_\text{ref}$ \\ \hline
  \vne{Kicker}      & $\sqrt{{\tt Hkick}^2 + {\tt Vkick}^2}$ & 0 \\
  \vne{Hkicker}     & Kick                                   & 0 \\
  \vne{Vkicker}     & Kick                                   & 0 \\
  \vne{Rbend}       & G * L                                  & 0 \\
  \vne{Sbend}       & G * L                                  & 0 \\
  \vne{Elseparator} & $\sqrt{{\tt Hkick}^2 + {\tt Vkick}^2}$ & 0 \\
  \vne{Quadrupole}  & K1 * L                                 & 1 \\
  \vne{Solenoid}    & KS * L                                 & 1 \\
  \vne{Sol_Quad}    & K1 * L                                 & 1 \\
  \vne{Sextupole}   & K2 * L                                 & 2 \\
  \vne{Octupole}    & K3 * L                                 & 3 \\ \hline
\end{tabular}
\caption{$F$ and $n_\text{ref}$ for various elements.}
\label{t:ab}
\end{table}

%-----------------------------------------------------------------
\section{Taylor Maps}
\label{s:taylor_phys}
\index{Transfer map!Taylor|textbf}

A transport map ${\cal M}: {\cal R}^6 \rightarrow {\cal R}^6$ through
an element or a section of a lattice is a function that maps the
starting phase space coordinates $\Bf r(\In)$ to the ending
coordinates $\Bf r(\Out)$
\begin{equation}
  \Bf r(\Out) = {\cal M} \, \Bf r(\In)
\end{equation}
${\cal M}$ is made up of six functions ${\cal M}_i: {\cal R}^6
 \rightarrow {\cal R}$. Each of these functions maps to one of the $r(\Out)$
coordinates. These functions can be expanded in a Taylor
series and truncated at some order. Each Taylor series is in the form
\Begineq
  r_i(\Out) = \sum_{j = 1}^N \, C_{ij} \, \prod_{k = 1}^6 \, r_k^{e_{ijk}}(\In)
\Endeq
Where the $C_{ij}$ are coefficients and the $e_{ijk}$ are integer exponents.
The order of the map is
\Begineq
  \mbox{order} = \max_{i,j} \left( \sum_{k = 1}^6 e_{ijk} \right)
\Endeq

The standard \bmad routine for printing a Taylor map might produce something 
like this: 
\begin{example}
   Taylor Terms:
    Out     Coef              Exponents           Order        Reference
   ---------------------------------------------------
      1:     -0.600000000000  0  0  0  0  0  0        0       0.200000000
      1:      1.000000000000  1  0  0  0  0  0        1
      1:      0.145000000000  2  0  0  0  0  0        2
   ---------------------------------------------------
      2:     -0.185000000000  0  0  0  0  0  0        0       0.000000000
      2:      1.300000000000  0  1  0  0  0  0        1
      2:      3.800000000000  2  0  0  0  0  0        2
   ---------------------------------------------------
      3:      1.000000000000  0  0  1  0  0  0        1       0.100000000
      3:      1.600000000000  0  0  0  1  0  0        1
      3:    -11.138187077310  1  0  1  0  0  0        2
   ---------------------------------------------------
      4:      1.000000000000  0  0  0  1  0  0        1       0.000000000
   ---------------------------------------------------
      5:      0.000000000000  0  0  0  0  0  0        0       0.000000000
      5:      0.000001480008  0  1  0  0  0  0        1
      5:      1.000000000000  0  0  0  0  1  0        1
      5:      0.000000000003  0  0  0  0  0  1        1
      5:      0.000000000003  2  0  0  0  0  0        2
   ---------------------------------------------------
      6:      1.000000000000  0  0  0  0  0  1        1       0.000000000
\end{example}
The Taylor map has 6 Taylor series, one each output phase space coordinate.
In this example, the second Taylor series when expressed as a formula would
read:
\Begineq
  p_x(out) = -0.185 + 1.3 \, p_x(in) + 3.8 \, x^2(in)
\Endeq

\index{Transfer map!Taylor!reference coordinates}
The reference column in the above example shows the coordinates around
which the Taylor map is calculated. In this case the reference
coordinates where 
\Begineq
  (x, p_x, y, p_y, z, p_z) = (0.2, 0, 0.1, 0, 0, 0, 0)
\Endeq
For example, suppose that the exact map through an element looks like
\Begineq
  x(out) = A \, \sin(k \, x(in))
\Endeq
Then a Taylor map to 1\St order is
\Begineq
  x(out) = [A \, \sin(k \, x_{\mbox{ref}}] + 
  [A \, k \, \cos(k \, x_{\mbox{ref}}] \, x(in)
\Endeq
Thus the choice of the reference point will affect the values of the
coefficients of the Taylor map but once the coefficient values are
determined the reference point does not play any role when the Taylor
map is evaluated to determine the output coordinates as a function of
the input coordinates.

%-----------------------------------------------------------------
\section{Symplectification}
\label{s:symp_method}
\index{Symplectic!symplectification}

If the evolution of a system can be described using a Hamiltonian then
it can be shown that the linear part of any transport map (the Jacobian)
must obey the symplectic condition. If a matrix $\Bf M$ is not symplectic,
Healy\cite{b:healy} has provided an elegant method for finding a symplectic 
matrix that is ``close'' to $\Bf M$. The procedure is as follows:
From $\Bf M$ a matrix $\Bf V$ is formed via
\begin{equation}
  \Bf V = \Bf S (\Bf I - \Bf M)(\Bf I + \Bf M)^{-1} 
  \label{e:vsimi}
\end{equation}
where $\Bf S$ is the matrix
\Begineq
  \Bf S = 
  \begin{pmatrix} 
      0 &  1 &  0 &  0 &  0 &  0 \cr
     -1 &  0 &  0 &  0 &  0 &  0 \cr
      0 &  0 &  0 &  1 &  0 &  0 \cr
      0 &  0 & -1 &  0 &  0 &  0 \cr
      0 &  0 &  0 &  0 &  0 & -1 \cr
      0 &  0 &  0 &  0 & -1 &  0 \cr
  \end{pmatrix}
  \label{s0100}
\Endeq
$\Bf V$ is symmetric if and only if $\Bf M$ is symplectic. In any case,
a symmetric matrix $\Bf W$ near $\Bf V$ can be
formed via
\begin{equation}
  \Bf W = \frac{\Bf V + \Bf V^t}{2}
\end{equation}
A symplectic matrix $\Bf F$ is now obtained by inverting \eq{e:vsimi}
\Begineq
  \Bf F = (\Bf I + \Bf S \Bf W) (\Bf I - \Bf S \Bf W)^{-1}
\Endeq

%-----------------------------------------------------------------
\section{Wigglers}
\label{s:wiggler_phys}

As discussed in Section~\ref{s:wig}, \bmad \vne{Wiggler} elements are split into 
two classes: map type and periodic type.
 The map type \vn{Wigglers} are modeled using the method of Sagan, Crittenden, 
and Rubin\cite{b:wiggler}. In this model the magnetic field is written as 
a sum of terms $B_i$
\Begineq
  B(x,y,z) = \sum_i B_i(x, y, z; C, k_x, k_y, k_z, \phi_z)
\Endeq 
Each term $B_i$ is specified using five numbers: 
$(C, k_x, k_y, k_z, \phi_z)$. A term can take one of three forms: The first
form is
\begin{alignat}{4}
  B_x &= -&C &\dfrac{k_x}{k_y} & \sin(\kxx) \sinh(\kyy) \cos(\ksss) \CRNEG
  B_y &=  &C &                 & \cos(\kxx) \cosh(\kyy) \cos(\ksss) \CRNEG
  B_s &= -&C &\dfrac{k_s}{k_y} & \cos(\kxx) \sinh(\kyy) \sin(\ksss) \CRneg
  & \makebox[1pt][l]{with $k_y^2 = k_x^2 + k_s^2$ .} &&&  \label{f1}
\end{alignat}
The second form is
\begin{alignat}{4}
  B_x &=  &C &\dfrac{k_x}{k_y} & \sinh(\kxx) \sinh(\kyy) \cos(\ksss) \CRNEG
  B_y &=  &C &                 & \cosh(\kxx) \cosh(\kyy) \cos(\ksss) \CRNEG
  B_s &= -&C &\dfrac{k_s}{k_y} & \cosh(\kxx) \sinh(\kyy) \sin(\ksss) \CRneg
  & \makebox[1pt][l]{with $k_y^2 = k_s^2 - k_x^2$ ,} &&&  \label{f2}
\end{alignat}
The third form is
\begin{alignat}{4}
  B_x &=  &C &\dfrac{k_x}{k_y} & \sinh(\kxx) \sin(\kyy) \cos(\ksss) \CRNEG
  B_y &=  &C &                 & \cosh(\kxx) \cos(\kyy) \cos(\ksss) \CRNEG
  B_s &= -&C &\dfrac{k_s}{k_y} & \cosh(\kxx) \sin(\kyy) \sin(\ksss) \CRneg
  & \makebox[1pt][l]{with $k_y^2 = k_x^2 - k_s^2$ .} &&& \label{f3}
\end{alignat}
The relationship between $k_x$, $k_y$, and $k_z$ ensures that Maxwell's equations
are satisfied. Since the field is given by analytic equations, Lie algebraic
techniques can be use to construct Taylor maps to arbitrary order.

%-----------------------------------------------------------------
\section{Radiation Damping and Excitation}
\label{s:radiation}
\index{Synchrotron radiation!damping and excitation|textbf}

Emission of synchrotron radiation by a particle can be decomposed into
two parts. The deterministic average radiation emitted produces damping
while the stochastic fluctuating part produces excitation\cite{b:jowett}.

The treatment of radiation damping by \bmad\ essentially follows MAD.
The average change in energy $\Delta E$ of a particle going through a
section of magnet due to synchrotron radiation is
\Begineq
  \frac{\Delta E}{E_0} = -k_d \, (1 + p_z)
\Endeq
where
\Begineq
  k_d \equiv \frac{2 \, r_e}{3} \, \gamma_0^3 \, \ave{g_0^2} \, L_p \,  
  (1 + p_z)
  \label{k2r3g}
\Endeq
$r_e$ is the classical electron radius, $L_p$ is the actual path
length, $\gamma_0$ is the energy factor of an on-energy particle, $1/g_0$
is the bending radius of an on--energy particle, and $\ave{g_0^2}$ is an
average of $g_0^2$ over the actual path.

The energy lost is given by
\Begineq
  \frac{\Delta E}{E_0} = -k_f \, (1 + p_z)
\Endeq
where
\Begineq
  k_f \equiv \left( \frac{55 \, r_e \, \hbar \, c}{24 \, \sqrt{3} \, m_e} \, 
  L_p \, \gamma_0^5 \ave{g_0^3} \right)^{1/2} \, (1 + p_z) \, \xi
  \label{k55rh}
\Endeq
$\xi$ is a Gaussian distributed random number with unit sigma and zero mean.

Using \Eqs{k2r3g} and \eq{k55rh} The total change in $p_z$ can be written as
\Begineq
  \Delta p_z = \frac{\Delta E}{E_0} = -k_E \, (1 + p_z)
\Endeq
where
\Begineq
  k_E = k_d + k_f
\Endeq
Since the radiation is emitted in the forward direction the angles
$x'$ and $y'$ are invariant which leads to the following equations for
the changes in $p_x$ and $p_y$
\begin{align}
  \Delta p_x &= -k_E \, p_x \CRNO
  \Delta p_y &= -k_E \, p_y 
\end{align}

%-----------------------------------------------------------------
\section{Coupling and Normal Modes}
\label{s:coupling}
\index{Normal Mode!Coupling}

The coupling formalism used by \bmad\ is taken from the paper of Sagan
and Rubin\cite{b:coupling}. The main equations are reproduced here.  A
one--turn map $\bfT(s)$ for the transverse two--dimensional phase space
$\bfx = (x, x', y, y')$ starting and ending at some point $s$ can be
written as
  \Begineq
    \bfT = \bfV \, \bfU \, \bfV\inv 
    , \label{tvuv}
  \Endeq 
where $\bfV$ is symplectic, and $\bfU$ is of the form
  \Begineq
    \bfU = 
    \begin{pmatrix}
      \bfA & \Bf0 \cr 
      \Bf0 & \bfB \cr
    \end{pmatrix}
    . \label{ua00b}
  \Endeq
\index{Normal Mode!a--mode}
\index{Normal Mode!b--mode}
Since $\bfU$ is uncoupled the standard Twiss analysis can be
performed on the matrices $\bfA$ and $\bfB$. The normal modes
are labeled $a$ and $b$ and if the one--turn matrix $\bfT$ is
uncoupled then $a$ corresponds to the horizontal mode and $b$
corresponds to the vertical mode. 

$\bfV$ is written in the form
  \Begineq
    \bfV = 
    \begin{pmatrix}
        \gamma \bfI & \bfC \cr 
        -\bfC^+     & \gamma \bfI \cr
    \end{pmatrix}
    , \label{vgicc1}
  \Endeq
where the symplectic conjugate is 
\index{Symplectic!conjugate}
  \Begineq
    \bfC^+ = 
    \begin{pmatrix}
       C_{22} & -C_{12} \cr 
      -C_{21} & C_{11} \cr
    \end{pmatrix}
    . \label{ccccc}
  \Endeq
Since we demand that $\bfV$ be symplectic we have the condition
  \Begineq               
    \gamma^2 + \, ||\bfC|| = 1
    , \label{gc1}
  \Endeq
and $\bfV\inv$ is given by
  \Begineq
    \bfV\inv = 
    \begin{pmatrix}
      \gamma \bfI & -\bfC \cr 
      \bfC^+ & \gamma \bfI \cr
    \end{pmatrix}
    . \label{vgicc2}
  \Endeq 
$\bfC$ is a measure of the coupling. 
$\bfT$ is uncoupled if and only if $\bfC = \Bf 0$. 

It is useful to normalize out the $\beta(s)$ variation in the the above
analysis. Normalized quantities being denoted by a bar above them. The
normalized normal mode matrix $\BAR\bfU$ is defined by
  \Begineq
    \BAR\bfU = \bfG \, \bfU \, \bfG\inv
    , \label{ugug}
  \Endeq
Where $\bfG$ is given by 
  \Begineq
    \bfG \equiv 
    \begin{pmatrix}
      \bfG_a & \Bf0 \cr 
      \Bf0 & \bfG_b
    \end{pmatrix}
    , \label{gg00g}
  \Endeq  
with 
  \Begineq
    \Bf G_a = 
    \begin{pmatrix}
      \frac{\tstyle 1}{\tstyle \sqrt{\beta_a}} & 0 \cr
      \frac{\tstyle \alpha_a}{\tstyle \sqrt{\beta_a}} & \sqrt{\beta_a}
    \end{pmatrix}
    , \label{g1b0a} 
  \Endeq
with a similar equation for $\Bf G_b$. With this definition, the corresponding
$\BAR\bfA$ and $\BAR\bfB$ (cf.~\Eq{ua00b} are just rotation matrices.
The relationship between $\bfT$ and $\BAR\bfU$ is 
  \Begineq
    \bfT = \bfG\inv \, \BAR\bfV \, \BAR\bfU \, \BAR\bfV\inv \, \bfG
    , \label{tgvuv}
  \Endeq
where
  \Begineq
    \BAR\bfV = \bfG \, \bfV \, \bfG\inv
    . \label{vgvg}
  \Endeq
Using \Eq{gg00g}, $\BAR\bfV$ can be written in the form
  \Begineq
    \BAR\bfV = 
    \begin{pmatrix}
      \gamma \bfI & \BAR\bfC \cr -\BAR\bfC^+ & \gamma \bfI
    \end{pmatrix}
    , \label{vgicc3}
  \Endeq
with the normalized matrix $\BAR\bfC$ given by
  \Begineq
    \BAR\bfC = \bfG_a \, \bfC \, \bfG_b\inv
    . \label{cgcg}
  \Endeq

The normal mode coordinates ${\bf a} = (a, a', b, b')$ are related to
the laboratory frame via
  \Begineq
    {\bf a} = \bfV\inv \, {\bf x}
    . \label{avx}
  \Endeq 
In particular the normal mode dispersion $\bfeta_a = (\eta_a,
\eta'_a, \eta_b, \eta'_b)$ is related to the laboratory frame
dispersion $\bfeta_x = (\eta_x, \eta'_x, \eta_y, \eta'_y)$ via
  \Begineq
    {\bfeta_a} = \bfV\inv \, {\bfeta_x}
    . \label{etaavx}
  \Endeq 

%-----------------------------------------------------------------
\section{Macroparticles}
\label{s:macro}
\index{Macroparticles|textbf}

A macroparticle\cite{b:transport_appendix} is a bundle of particles
whose distribution is assumed to be Gaussian in shape. A macroparticle
is represented by a centroid position $\bfrbar$ and a $6 \times 6$
$\bfsig$ matrix which defines the shape of the macroparticle in
phase space. $\sigma_i = \sqrt{\bfsig(i,i)}$ is the RMS sigma for the $i$\Th
phase space coordinate. For example $\sigma_z = \sqrt{\bfsig(5,5)}$.

$\bfsig$ is a real, non-negative symmetric matrix. The equation that
defines the ellipsoid at a distance of $n$--sigma from the centroid is
\Begineq
  (\bfr - \bfrbar)^t \bfsig\inv (\bfr - \bfrbar) = n
\Endeq
where the $t$ superscript denotes the transpose. Given the sigma matrix
at some point $s = s_1$, the sigma matrix at a different point $s_2$ is
\Begineq
  \bfsig_2 = \Bf M_{21} \, \bfsig_1 \, \Bf M_{21}^t
\Endeq
where $\Bf M_{21}$ is the Jacobian of the transport map between points
$s_1$ and $s_2$.

The Twiss parameters can be calculated from the sigma matrix. The
dispersion is given by
\begin{align}
  \bfsig(1,6) &= \eta_x \, \bfsig(6,6) \CRNO
  \bfsig(2,6) &= \eta'_x \, \bfsig(6,6) \\
  \bfsig(3,6) &= \eta_y \, \bfsig(6,6) \CRNO
  \bfsig(4,6) &= \eta'_y \, \bfsig(6,6) \nonumber
\end{align}
Ignoring coupling for now the betatron part of the sigma matrix can be
obtained from the linear equations of motion. For example, using
\Begineq
  x = \sqrt{2 \, \beta_x \, \epsilon_x} \cos \phi_x + \eta_x \, p_z
\Endeq
Solving for the first term on the RHS, squaring and averaging over all
particles gives
\Begineq
  \beta_x \, \epsilon_x = \bfsig(1,1) - \frac{\bfsig^2(1,6)}{\bfsig(6,6)}
\Endeq
It is thus convienient to define the betatron part of the sigma martix
\Begineq
  \bfsigb(i,j) \equiv \bfsig(i,j) - \frac{\bfsig(i,6) \, \bfsig(j,6)}{\bfsig(6,6)}
\Endeq
and in terms of the betatron part the emittance is
\Begineq
  \epsilon_x^2 = \bfsigb(1,1) \, \bfsigb(2,2) - \bfsigb^2(1,2)
\Endeq
and the Twiss parameters are
\Begineq
  \epsilon_x 
  \begin{pmatrix}
    \beta_x   & -\alpha_x \\
    -\alpha_x & \gamma_x
  \end{pmatrix} = 
  \begin{pmatrix}
    \bfsigb(1,1) & \bfsigb(1,2) \\
    \bfsigb(1,2) & \bfsigb(2,2) 
  \end{pmatrix}
\Endeq

If there is coupling the transformation between the $4\times 4$
transverse normal mode sigma matrix $\bfsig_a$ and the $4\times 4$
laboratory matrix $\bfsig_x$ is
\Begineq
  \bfsig_x = \bfV \, \bfsig_a \bfV^t
\Endeq

For tracking a relativistic charged particle bunch is modeled by dividing it
longitudinally into a series of slices. Each slice is made up of a
number of macroparticles. It is assumed that the range of the
longitudinal positions of the macroparticles of a given slice do not
overlap the ranges of the other slices.

\bmad has a routine to initialize macroparticles to mirror the initial
distribution as set--up by the LIAR program\cite{b:liar}. The
initialization is as follows: The center of a
slice will be at a longitudinal position $z_j$ given by
\Begineq
  z_j = z_b + \frac{(n_s - 2 \, j + 1) \, \sigma_z \, N_{\sigma z}}{n_s}
\Endeq
where $z_b$ the overall offset of the bunch, 
$n_s$ is the number of slices, $j = 1, \ldots,
n_s$ is the index of the slice, $\sigma_z$ is the RMS bunch length,
and $N_{\sigma z}$ is the number of $\sigma_z$ the slices go out
to. The width $dz_s$ of each slice is
\Begineq
    dz_s = \frac{2 \, \sigma_z \, N_{\sigma z}}{n_s}
\Endeq
The charge associated with each slice is, within a constant factor,
the charge contained within the region between $z_j - dz_s/2$ and $z_j
+ dz_s/2$ assuming a Gaussian distribution.  The charge of all the
slices are multiplied by a factor so that the total charge of the
slices is equal to the input bunch charge.

The initialization of the macroparticles within a slice gives all the
macroparticles the same centroid position except with differing
centroid energies. 
The sigma matrix is the same for all macroparticles and is
determined by the input Twiss parameters:
\begin{align}
  \bfsig(1,1) &= \epsilon_x \, \beta_x \CRNEG
  \bfsig(1,2) &= -\epsilon_x \alpha_x  \CRNEG
  \bfsig(2,2) &= \epsilon_x \, \gamma_x = 
      \epsilon_x \, (1 + \alpha_x^2) / \beta_x \CRNEG
  \bfsig(3,3) &= \epsilon_y \, \beta_y \\
  \bfsig(3,4) &= -\epsilon_y \alpha_y \CRNEG
  \bfsig(3,4) &= \epsilon_y \, \gamma_y = 
      \epsilon_y \, (1 + \alpha_y^2) / \beta_y \CRNEG
  \bfsig(i,j) &= 0 \quad \mbox{otherwise} \nonumber
\end{align}
The centroid energy of the $k$\Th macroparticle is
\Begineq
  E_k = E_b + \frac{(n_{mp} - 2 \, k + 1) \, \sigma_E \, N_{\sigma E}}{n_{mp}}
\Endeq
where $E_b$ is the central energy of the bunch, $n_{mp}$ is the number
of macroparticles per slice, $\sigma_E$ is the energy sigma, and
$N_{\sigma E}$ is the number of sigmas in energy that the range of
macroparticle energies cover. The charge of each macroparticle is,
within a constant factor, the charge contained within the energy
region $E_k - dE_{mp}/2$ to $E_k + dE_{mp}/2$ assuming a Gaussian
distribution where the energy width $dE_{mp}$ is
\Begineq
  dE_{mp} = \frac{2 \, \sigma_E \, N_{\sigma E}}{n_{mp}}
\Endeq
The charge of all the macroparticles is adjusted by a constant factor
so that the total charge of the macroparticles within a slice is equal
to the charge associated with the slice.



%-----------------------------------------------------------------
\section{Wakefields}
\label{s:wakefields}
\index{Wakefields}

Wakefield modeling follows the LIAR program\cite{b:liar}. Wakefields
are applied in the center of LINAC accelerating cavities
(\vn{LCavity}). Wakefield effects are divided into short--range
(within a bunch) and long--range (between bunches).

Only the dipole component of the short--range wakefield are
modeled. The short--range longitudinal wakefield component for 
the $i$\Th macroparticle is computed from the equation
\Begineq
  dE_i = \frac{L \, \Wl(0)}{2} |Q_i| +
                  \sum_{j \ne i} \frac{L \, \tWl(dz_{ij})}{2} |Q_j|
\Endeq
where $L$ is the cavity length, $dz_{ij}$ is the longitudinal distance between 
the $i$\Th and $j$\Th macroparticles,
$\Wl$ is the short--range longitudinal wakefield function, and $\tWl$ is 
a modified wakefield function 
\Begineq
  \tWl(dz) = 
  \begin{cases}
    \Wl(0) \cdot \frac{dz + \sigma_{zij}}{2\sigma_{zij}} & 
                                    -\sigma_{zij} < dz < \sigma_{zij} \\
    \Wl(dz)                                            & \text{otherwise}
  \end{cases}
\Endeq
$\tWl$ is used instead of $\Wl$ to avoid simulation artifacts 
is due to the discontinuity of $\Wl$ at $dz = 0$. 
$\sigma_{zij}$ is the combined macroparticle length
\Begineq
  \sigma_{zij} = \sigma_{zi} + \sigma_{zj} + \sigma_{z0}
\Endeq
where $\sigma_{zi}$ and $\sigma_{zj}$ are the longitudinal sizes for
the $i$\Th and $j$\Th particles respectively, and $\sigma_{z0}$ is a
small fudge factor needed when $\sigma_{zi} = \sigma_{zj} = 0$.

The transverse kick $dp_x(i)$ for the $i$\Th macroparticle due to the 
dipole short--range transverse wakefield is modeled with the equation
\Begineq
  dp_x(i) = \frac{\sum_j |Q_j| \, x_j \, L \, \Wt(dz_{ij})}{P_0}
\Endeq
with a similar equation for $dp_y(i)$. $\Wt$ is the transverse short--range
wake function.

%-----------------------------------------------------------------
\section{Synchrotron Integrals}
\label{s:synch_ints}
\index{Synchrotron radiation!integrals}

The standard formulas for the synchrotron integrals used to compute
emittances, the energy spread, etc., for a storage ring assume no
coupling between the horizontal and vertical
planes\cite{b:helm,b:jowett}.  With coupling the equations need to be
generalized.

In the general case the curvature function $\Bf G = (G_x, G_y)$, which
points away from the center of curvature of the particle's orbit (see
Figure~\ref{f:local_coords}), does not lie in the horizontal
plane. $\Bf G$ has a magnitude $G = 1/\rho$ and a positive $G_y$
indicates a downward bend in the negative $y$ direction.  Similarly
the dispersion, $\bfeta\two = (\eta_x, \eta_y)$ will not lie in the
horizontal plane. With this notation the synchrotron Integrals for the
$a$ and $b$ normal modes are:
  \Begineqs
    I_1 &=& \oint ds \, \Bf G \cdot \bfeta 
         \equiv \oint ds \, (G_x \, \eta_x + G_y \, \eta_y) \\
    I_2 &=& \oint ds \, G^2 \\
    I_3 &=& \oint ds \, G^3 \\
    I_{4a} &=& \oint ds \, \left[ G^2 \, \Bf G \cdot \bfeta\two_a + 
         \nabla G^2 \cdot \bfeta\two_a \right] \\
    I_{4b} &=& \oint ds \, \left[ G^2 \, \Bf G \cdot \bfeta\two_b + 
         \nabla G^2 \cdot \bfeta\two_b \right] \\
    I_{4z} &=& \oint ds \, \left[ G^2 \, \Bf G \cdot \bfeta\two + 
         \nabla G^2 \cdot \bfeta\two \right] \\
    I_{5a} &=& \oint ds \, G^3 \, \calh_a \\
    I_{5b} &=& \oint ds \, G^3 \, \calh_b
  \Endeqs
where $\calh_a$ is 
  \Begineq
    \calh_a = \gamma_a \, \eta_a^2 + 2 \, \alpha_a \, \eta_a \, \eta_a' + 
      \beta_a \eta_a'^2 
  \Endeq
with a similar equation for $\calh_b$. Here $\bfeta\two_a =
(\eta_{ax}, \eta_{ay})$, and $\bfeta\two_b = (\eta_{bx}, \eta_{by})$
are the dispersion vectors for the $a$ and $b$ modes respectively in
$x--y$ space (these 2--vectors are not to be confused with the
dispersion 4--vectors used in the previous section). The position
dependence of the curvature function is:
  \Begineqs
    G_x(x,y) = G_{x} + x \, k_1 + y \, s_1 \CRNO
    G_y(x,y) = G_{y} + x \, s_1 - y \, k_1 
  \Endeqs
where $k_1$ is the quadrupole moment and $s_1$ is the skew--quadrupole moment.
Using this gives on--axis ($x = y = 0$)
  \Begineq
    \nabla G^2 = 2 \left( G_x k_1 + G_y s_1, \, G_x s_1 - G_y k_1 \right)
    \label{g2gkg}
  \Endeq

In a dipole a non--zero $e_1$ or $e_2$ gives a contribution to $I_4$
via the $\nabla G^2 \cdot \bfeta$ term. The edge field is modeled as a
thin quadrupole of length $\delta$ and stringth $k = -\tan(e) /
\delta$. It is assumed that $\Bf G$ rises linearly within the edge field
from zero on the outside edge of the edge field to its full value on the inside 
edge of the edge field. 
Using this in \Eq{g2gkg} and integrating over the edge field gives the contribution
to $I_4$ from a non--zero $e_1$ as
  \Begineq
    I_{4z} = -\tan(e_1) \, G^2
    \left( \cos(\theta) \, \eta_x + \sin(\theta) \, \eta_y \right)
    \label{iegct}
  \Endeq
With an analogous equation for a finite $e_2$. The extension to
$I_{4a}$ and $I_{4b}$ involves using $\bfeta\two_a$ and $\bfeta\two_b$
in place of $\bfeta\two$.  In \Eq{iegct} $\theta$ is the \vn{tilt}
angle which is non--zero if the bend is not in the horizontal plane.

The above integrals are invariant under rotation of the $(x,y)$ coordinate
system and reduce to the standard equations when $G_y = 0$ as they should.

There are various parameters that can be expressed in terms of these
integrals.  The $I_1$ integral can be related to the momentum
compaction $\alpha_p$ via
  \Begineq
    I_1 = \alpha_p \, L
  \Endeq
where $L$ is the storage ring circumferance. The energy loss per turn is
  \Begineq
    U_0 = \frac{2 \, r_e E_0^4}{3 \, mc^2} I_2
  \Endeq
where $E_0$ is the nominal energy and $r_e$ is the classical electron
radius (electrons are assumed here but the formulas are easily
generalized).

The damping partition numbers are
  \Begineq
    J_a = 1 - \frac{I_{4a}}{I_2} \comma \quad
    J_b = 1 - \frac{I_{4b}}{I_2} \comma \, \mbox{and} \quad \label{j1ii}
    J_z = 2 + \frac{I_{4z}}{I_2} \period
  \Endeq
Since 
  \Begineq          
    \bfeta\two_{a} + \bfeta\two_{b} = \bfeta\two
    \comma \label{eee}
  \Endeq
Robinson's theorem, $J_a + J_b + J_z = 4$, is satisfied.
Alternatively, the exponential damping coefficients per turn are
  \Begineq
    \alpha_a = \frac{U_0 \, J_a}{2 E_0} \comma \quad
    \alpha_b = \frac{U_0 \, J_b}{2 E_0} \comma \, \mbox{and} \quad
    \alpha_z = \frac{U_0 \, J_z}{2 E_0} \period
  \Endeq
The energy spread is given by
  \Begineq
    \sigma_{pz}^2 = \left( \frac{\sigma_E}{E_0} \right)^2 = 
    C_q \gamma_0^2 \frac{I_3}{2I_2 + I_{4z}}
  \Endeq
where $\gamma_0$ is the usual energy factor and 
  \Begineq
    C_q = \frac{55}{32 \, \sqrt{3}} \, \frac{\hbar}{mc} = 
    3.84 \times 10^{-13} \, \mbox{meter for electrons}
  \Endeq
If the synchrotron frequency is not too large the bunch length is given by
  \Begineq
    \sigma_z = \frac{I_1}{M(6,5)} \, \sigma_{pz}
  \Endeq
where $M(6,5)$ is the $(6,5)$ element for the 1--turn transfer matrix
of the storage ring. Finally the emittances are given by
  \Begineqs
    \epsilon_a \AND= C_q \, \gamma_0^2 \frac{I_{5a}}{I_2 - I_{4a}} \CRNO
    \epsilon_b \AND= C_q \, \gamma_0^2 \frac{I_{5b}}{I_2 - I_{4b}}
  \Endeqs

For a linac radiation integrals are still of interest if there are
bends but in this case the appropriate energy factors must be included
to take account the changing energy in the line. $I_1$ is not altered and
the $I_4$ integrals are not relavent. The other integrals become
  \Begineqs
    L_2 &=& \int ds \, G^2 \, \gamma_0^4 \\
    L_3 &=& \int ds \, G^3 \, \gamma_0^7 \\
    L_{5a} &=& \int ds \, G^3 \, \calh_a \, \gamma_0^6 \\
    L_{5b} &=& \int ds \, G^3 \, \calh_b \, \gamma_0^6
  \Endeqs
In terms of these integrals the energy loss through the linac is
  \Begineq
    U_0 = \frac{2 \, r_e \, mc^2}{3} L_2
  \Endeq
The energy spread assuming $\sigma_E$ is zero at the start and neglecting
any dampling is
  \Begineq
    \sigma_E^2 = \frac{4}{3} \, C_q \, r_e \, \left( m c^2 \right)^2 \, L_3
  \Endeq
and, again neglecting any initial beam width, the transverse beam size
at the end of the linac is
  \Begineqs
    \epsilon_a \AND= \frac{2}{3} \, C_q \, r_e \, 
    \frac{L_{5a}}{\gamma_f} \CRNO
    \epsilon_b \AND= \frac{2}{3} \, C_q \, r_e \, 
    \frac{L_{5b}}{\gamma_f} 
  \Endeqs
Where $\gamma_f$ is the final gamma
