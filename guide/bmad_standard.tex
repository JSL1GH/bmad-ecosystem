\documentclass{book}

\usepackage{graphicx}
\usepackage{moreverb}
\usepackage{amsmath}
\usepackage{alltt}
\usepackage{rotating}
\usepackage{subfigure}
\usepackage{toc}
\usepackage{xspace}
\usepackage{makeidx}

\newcommand{\extref}[1]{$\S$\ref*{#1}}   % No hyperlink. For external refs. \extref
\newcommand{\comma}{\> ,}
\newcommand{\period}{\> .}
\newcommand{\wt}{\widetilde}
\newcommand{\grv}{\textasciigrave}
\newcommand{\hyperbf}[1]{\textbf{\hyperpage{#1}}}
\newcommand{\Ss}{\(^*\)}
\newcommand{\Dd}{\(^\dagger\)}

\newcommand{\AND}{&& \hskip -17pt\relax}
\newcommand{\CR}{\\}
\newcommand{\CRNO}{\nonumber \\}
\newcommand{\dstyle}{\displaystyle}

\newcommand{\Begineq}{\begin{equation}}
\newcommand{\Endeq}{\end{equation}}
\newcommand{\NoPrint}[1]{}

\newcommand{\pow}[1]{\cdot 10^{#1}}
\newcommand{\Bf}[1]{{\bf #1}}
\newcommand{\bfr}{\Bf r}

\newcommand{\bmad}{{\sl Bmad}\xspace}
\newcommand{\tao}{{\sl Tao}\xspace}
\newcommand{\mad}{{\sl MAD}\xspace}
\newcommand{\cesr}{{\sl CESR}\xspace}

\newcommand{\sref}[1]{\S\ref{#1}}
\newcommand{\Sref}[1]{Sec.~\sref{#1}}
\newcommand{\cref}[1]{Chapter~\ref{#1}}

\newcommand{\Newline}{\hfil \\ \relax}

\newcommand{\eq}[1]{{(\protect\ref{#1})}}
\newcommand{\Eq}[1]{{Eq.~(\protect\ref{#1})}}
\newcommand{\Eqs}[1]{{Eqs.~(\protect\ref{#1})}}

\newcommand{\vn}{\ttcmd}           % For variable names
\newcommand{\vni}{\ttcmdindx}
\newcommand{\cs}{\ttcmd}           % For code source
\newcommand{\cmd}{\ttcmd}          % For Unix commands
\newcommand{\rn}{\ttcmd}           % For Routine names
\newcommand{\tn}{\ttcmd}           % For Type (structure) names
\newcommand{\bn}[1]{{\bf #1}}       
\newcommand{\toffset}{\vskip 0.01in}
\newcommand{\rot}[1]{\begin{rotate}{-45}#1\end{rotate}}

\newcommand{\data}{{\mbox{data}}}
\newcommand{\reference}{{\mbox{ref}}}
\newcommand{\model}{{\mbox{model}}}
\newcommand{\base}{{\mbox{base}}}
\newcommand{\design}{{\mbox{design}}}
\newcommand{\meas}{{\mbox{meas}}}
\newcommand{\var}{{\mbox{var}}}

\newcommand\ttcmd{\begingroup\catcode`\_=11 \catcode`\%=11 \dottcmd}
\newcommand\dottcmd[1]{\texttt{#1}\endgroup}

\newcommand\ttcmdindx{\begingroup\catcode`\_=11 \catcode`\%=11 \dottcmdindx}
\newcommand\dottcmdindx[1]{\texttt{#1}\endgroup\index{#1}}

\newcommand{\St}{$^{st}$\xspace}
\newcommand{\Nd}{$^{nd}$\xspace}
\newcommand{\Th}{$^{th}$\xspace}
\newcommand{\B}{$\backslash$}
\newcommand{\W}{$^\wedge$}

\newcommand{\cbar}[1]{\overline C_{#1}}

\newlength{\dPar}
\setlength{\dPar}{1.5ex}

\newenvironment{example}
  {\vspace{-3.0ex} \begin{alltt}}
  {\end{alltt} \vspace{-2.5ex}}

\newcommand\Strut{\rule[-2ex]{0mm}{6ex}}

\newenvironment{Itemize}
  {\begin{list}{$\bullet$}
    {\addtolength{\topsep}{-1.5ex} 
     \addtolength{\itemsep}{-1ex}
    }
  }
  {\end{list} \vspace*{1ex}}

\newcommand{\Section}[1]{\section{#1}\indent\vspace{-3ex}}

\newcommand{\SECTION}[1]{\section*{#1}\indent\vspace{-3ex}}

% From pg 64 of The LaTex Companion.

\newenvironment{ventry}[1]
  {\begin{list}{}
    {\renewcommand{\makelabel}[1]{\textsf{##1}\hfil}
     \settowidth{\labelwidth}{\textsf{#1}}
     \addtolength{\itemsep}{-1.5ex}
     \addtolength{\topsep}{-1.0ex} 
     \setlength{\leftmargin}{5em}
    }
  }
  {\end{list}}


\setlength{\textwidth}{6.25in}
\setlength{\hoffset}{0.0in}
\setlength{\oddsidemargin}{0.25in}
\setlength{\evensidemargin}{0.0in}
\setlength{\textheight}{8.5in}
\setlength{\topmargin}{0in}

\makeindex

\begin{document}

\setlength{\parskip}{\dPar}
\setlength{\parindent}{0ex}

\newcommand{\bfr}{{\bf r}}
\newcommand{\bfM}{{\bf M}}
\newcommand{\bfK}{{\bf K}}

%-----------------------------------------------------------------
\section{Bmad\_Standard Transfer Map Calculations}
\label{s:bmad_standard}

Without any electric fields, the Hamiltonian is
\Begineq
  H = p_z - (1 + g \, x) \sqrt{(1 + p_z)^2 - (p_x - a_x)^2 - (p_y - a_y)^2} - 
  (1 + g \, x) \, a_s
  \label{h1gx1}
\Endeq
Here $(x, p_x, y, p_y, z, p_z)$ are the canonical coordinates
(\sref{s:phase_space_coords}), $g = 1/\rho$ with $\rho$ being the
local radius of curvature of the reference particle, and
$\Bf a(x,y,s)$ is the normalized vector potential which is related to
the vector potential $\Bf A(x,y,s)$ via
\Begineq
  \Bf a = \frac{q \, \Bf A}{P_0 \, c}
\Endeq
The equations of motion are
\begin{align}
  \frac{dq_i}{ds} &= \frac{\partial H}{\partial p_i} \CRNO
  \frac{dp_i}{ds} &= -\frac{\partial H}{\partial q_i}
  \label{rshp}
\end{align}

Assuming mid--plane symmetry of the magnetic field, so
that $a_x$ and $a_y$ can set to zero\cite{b:madphysics}, The vector
potential up to second order is (cf.~\Eq{byx0b})
\Begineq
  a_s = -k_0 \left( x - \frac{g \, x^2}{2 (1 + g\, x)} \right) -
  \frac{1}{2} k_1 \left( x^2 - y^2 \right)
\Endeq

\vn{Bmad_standard} transfer map (\sref{s:xfer}) and (\sref{s:tkm})
calculations are meant to be quick. To this end the paraxial
approximation (\Eq{xpa1p}) is generally used.  Using the paraxial
approximation, \Eq{h1gx1} becomes
\Begineq
  H = \frac{(p_x - a_x)^2}{2 (1 + p_z)} + \frac{(p_y - a_y)^2}{2 (1 + p_z)} - 
  (1 + g \, x) \, a_s 
  \label{hpapa}
\Endeq

The change in $z$, once the transverse trajectory is
known, is obtained from symplectic integration of \Eq{hpapa} 
\Begineq
  z \rightarrow z - \frac{1}{2 (1 + p_z)^2} \int \! ds \, 
  \left[ (p_x - a_x)^2 + (p_y - a_y)^2 \right] - \int \! ds \, g \, x
  \label{zz121p}
\Endeq
Using the equations of motion \Eqs{rshp} this can also be rewritten as
\Begineq
  z \rightarrow z - \frac{1}{2} \int \! ds \, 
  \left[ \left( \frac{dx}{ds} \right)^2 + \left( \frac{dy}{ds} \right)^2 \right] - 
  \int \! ds \, g \, x
  \label{zz12sx}
\Endeq

For elements without electric fields the transfer map
for a given coordinate $r_i$ may be expanded in a Taylor series
\Begineq
  r_i \rightarrow m_i + \sum_{j = 1}^4 m_{ij} \, r_j + 
  \sum_{j = 1}^4 \sum_{k = j}^4 m_{ijk} \, r_j \, r_k + \ldots
\Endeq
where the map coefficients $m_{ij\cdots}$ are functions of $p_z$.  For
linear elements the transfer map is linear for the transverse
coordinates and quadratic for $z$.


%-----------------------------------------
\subsection{BeamBeam}
\index{BeamBeam}


%-----------------------------------------
\subsection{Bend\_Sol\_Quad}
\index{Bend_Sol_Quad}

%-----------------------------------------
\subsection{Drift}
\index{Drift} 

A drift has $\Bf a = 0$ and $g = 0$. The Hamiltonian for a drift is
\Begineq
  H = \frac{p_x^2 + p_y^2}{2 (1 + p_z)} 
\Endeq
This gives the map
\begin{align}
  x   &\rightarrow x + \frac{L \, p_x}{1 + p_z} \CRNO
  p_x &\rightarrow p_x  \CRNO
  y   &\rightarrow y + \frac{L \, p_y}{1 + p_z} \CRNO
  p_y &\rightarrow p_y  \\
  z   &\rightarrow z - \frac{L \, (p_x^2 + p_y^2)}{2 (1 + p_z)^2} \CRNO
  p_z &\rightarrow p_z \nonumber
\end{align}

%-----------------------------------------
\subsection{Kicker, Hkicker, Vkicker, Elseparator}
\index{Kicker}
\index{HKicker}
\index{VKicker}
\index{Elseparator}


%-----------------------------------------
\subsection{LCavity}
\index{LCavity}

For an \vn{LCavity} the problem is simplified by using $(-ct, E)$ for
the longitudinal coordinates. With this, The Hamiltonian is
\Begineq
  H = 
\Endeq

The traversal time across a cavity is
\Begineq
  t_L = \int_0^L \frac{ds}{\beta \, c}
\Endeq
Writing
\Begineq
  \frac{1}{\beta} = \frac{E}{cP} = \frac{E}{\sqrt{E^2 - m^2c^4}}
\Endeq
and assuming a constant gradient $G$, so that $E$ increases linearly
with length, the transit time is easily computed to be
\Begineq
  t_L = \frac{P_2 - P_1}{G}
\Endeq
Using this in \Eq{zbbzb} gives the change in $z$
\Begineq
  z_2 = \frac{\beta_2}{\beta_1} \, z_1 + 
  \beta_2 \, c \left[ \frac{P_{02} - P_{01}}{G_0} - \frac{P_2 - P_1}{G} \right]
\Endeq
where $P_{01}$ and $P_{02}$ are the reference momentum at the entrance
and exits of the \vn{LCavity} with $G_0$ the reference gradient. 

The derivatives are straight forward if tedious
\begin{align}
  m(5,5) &= \frac{dz_2}{dz_1} = 
    \frac{\beta_2}{\beta_1} + 
    \frac{z_2 \, m(6,5)}{\beta_2} \frac{d\beta_2}{dp_{z2}} - 
    \frac{\beta_2}{G} \frac{dcP_2}{dz_1} +
    \frac{\beta_2 \, c \, (P_2 - P_1) \, c P_2}{L \, G^2 \, E_2} 
      \frac{dcP_2}{dz_1} \CRNO
  m(5,6) &= \frac{dz_2}{dp_{z1}} = 
    \frac{-\beta_2 \, z_1}{\beta_1^2} \frac{d\beta_1}{dp_{z1}} + 
    \frac{z_2 \, m(6,6)}{\beta_2} \frac{d\beta_2}{dp_{z2}} -
    \frac{\beta_2 ( c P_{02} \, m(6,6) - c P_{01})}{G} \CRNO
  m(6,5) &= \frac{dp_{z2}}{dz1} =
    \frac{E_2}{cP_2 \, cP_{02}} \frac{cP_1 \, cP_{01}}{E_1}  \\
  m(6,6) &= \frac{dp_{z2}}{dp_{z1}} = 
    \frac{E_2}{cP_2} \frac{G \, L}{c P_{02}} \frac{2 \, \pi \, f \, \sin\phi}{c}
    \nonumber
\end{align}
where
\begin{align}
  \frac{d\beta_1}{dp_{z1}}  &= \frac{(mc^2)^2}{E_1^3} \, cP_{01} \CRNO
  \frac{d\beta_2}{dp_{z2}}  &= \frac{(mc^2)^2}{E_2^3} \, cP_{02} \\
  \frac{dcP_2}{dz_1}        &= m(6,5) \, cP_{02}  \nonumber
\end{align}

%-----------------------------------------
\subsection{Octupole}
\index{Octupole}

%-----------------------------------------
\subsection{Quadrupole}
\index{Quadrupole}

The \vn{bmad_standard} calculates the transfer map through an upright
quadrupole and then transforms that map to the laboratory frame.

The Hamiltonian for an upright quadrupole is
\Begineq
  H = \frac{p_x^2 + p_y^2}{2 (1 + p_z)} + \frac{k_1}{2} (x^2 - y^2)
\Endeq
This is simply solved
\begin{align}
  x   &\rightarrow c_x \, x + s_x \, \frac{p_x}{1 + p_z} \CRNO
  p_x &\rightarrow \tau_x \, \om^2 \, \, (1 + p_z) \, s_x \, x + c_x \, p_x \CRNO
  y   &\rightarrow c_y \, y + s_y \, \frac{p_y}{1 + p_z} \CRNO
  p_y &\rightarrow \tau_y \, \om^2 \, \, (1 + p_z) \, s_y \, y + c_y \, p_y \CRNO
  z   &\rightarrow z + m_{511} \, x^2 + m_{512} \, x \, p_x + m_{522} \, p_x^2 + 
                   m_{533} \, y^2 + m_{534} \, y \, p_y + m_{544} \, p_y^2 \CRNO
  p_z &\rightarrow p_z \nonumber
\end{align}
where 
\Begineq
  \om \equiv \sqrt{\frac{|k_1|}{1 + p_z}}
\Endeq
and
\begin{alignat}{6}
         &\hspace*{3ex}  && k_1 > 0          &\hspace*{3ex}& k_1 < 0 & \qqquad
         &\hspace*{3ex}  && k_1 > 0          &\hspace*{3ex}& k_1 < 0 \CRNO
     c_x &=   && \cos  (\om \, L) && \cosh (\om \, L) & \qqquad
     c_y &=   && \cosh (\om \, L) && \cos  (\om \, L) \CRNO
     s_x &=   && \frac{\sin  (\om \, L)}{\om} && \frac{\sinh (\om \, L)}{\om} & \qqquad
     s_y &=   && \frac{\sinh (\om \, L)}{\om} && \frac{\sin  (\om \, L)}{\om} \\
  \tau_x &=   && {-}1             && {+}1             & \qqquad
  \tau_y &=   && {+}1             && {-}1             \nonumber
\end{alignat}
with this
\begin{alignat}{2}
  m_{511} &= \frac{\tau_x \,\, \om^2}{4} \, (L - c_x \, s_x) & \qqquad
  m_{533} &= \frac{\tau_y \,\, \om^2}{4} \, (L - c_y \, s_y) \CRNO
  m_{512} &= \frac{-\tau_x \,\, \om^2}{2 \, (1 + p_z)} \, s_x^2 & \qqquad
  m_{534} &= \frac{-\tau_y \,\, \om^2}{2 \, (1 + p_z)} \, s_y^2 \CRNO
  m_{522} &= \frac{-1}{4 \, (1 + p_z)^2} \, (L + c_x \, s_x) & \qqquad
  m_{544} &= \frac{-1}{4 \, (1 + p_z)^2} \, (L + c_y \, s_y) \nonumber
\end{alignat}{2}


%-----------------------------------------
\subsection{RFcavity}
\index{RFcavity}

%-----------------------------------------
\subsection{Sbend, Fringe Effects}
\index{Sbend}


%-----------------------------------------
\subsection{Sbend, No k\_1}
\index{Sbend}



%-----------------------------------------
\subsection{Sbend, Finite k\_1}
\index{Sbend}

The Hamiltonian is
\Begineq
  H = (k_0 - g) x - g \, x \, p_z + 
  \frac{1}{2}\left( (k_1 + g \, k_0) x^2 - k_1 \, y^2 \right) +
  \frac{p_x^2 + p_y^2}{2 (1 + p_z)} 
\Endeq

This is simply solved
\begin{align}
  x   &\rightarrow c_x \, (x - x_c) + s_x \, \frac{p_x}{1 + p_z} + x_c \CRNO
  p_x &\rightarrow \tau_x \, \om_x^2 \, \, (1 + p_z) \, s_x \, (x -x_c) + c_x \, p_x \CRNO
  y   &\rightarrow c_y \, y + s_y \, \frac{p_y}{1 + p_z} \CRNO
  p_y &\rightarrow \tau_y \, \om_y^2 \, \, (1 + p_z) \, s_y \, y + c_y \, p_y \CRNO
  z   &\rightarrow z + m_5 + m_{51} (x - x_c) + m_{52} p_x + \CRNO
      &\hspace*{20ex} m_{511} \, (x-x_c)^2 + m_{512} \, (x-x_c) \, p_x + m_{522} \, p_x^2 + 
                   m_{533} \, y^2 + m_{534} \, y \, p_y + m_{544} \, p_y^2 \CRNO
  p_z &\rightarrow p_z \nonumber
\end{align}
where 
\begin{alignat}{2}
  k_x &= k_1 + g \, k_0 & \qqquad
  \om_x &\equiv \sqrt{\frac{|k_x|}{1 + p_z}} \CRNO
  x_c &= \frac{g \, (1 + p_z) - k_0}{k_x} & \qqquad
  \om_y &\equiv \sqrt{\frac{|k_1|}{1 + p_z}} \CRNO
\end{alignat}
and
\begin{alignat}{6}
         &\hspace*{3ex}  && k_x > 0          &\hspace*{3ex}& k_x < 0 & \qqquad
         &\hspace*{3ex}  && k_1 > 0          &\hspace*{3ex}& k_1 < 0 \CRNO
     c_x &=   && \cos  (\om_x \, L)               && \cosh (\om_x \, L) & \qqquad
     c_y &=   && \cosh (\om_y \, L)               && \cos  (\om_y \, L) \\
     s_x &=   && \frac{\sin  (\om_x \, L)}{\om_x} && \frac{\sinh (\om_x \, L)}{\om_x} & \qqquad
     s_y &=   && \frac{\sinh (\om_y \, L)}{\om_y} && \frac{\sin  (\om_y \, L)}{\om_y} \CRNO
  \tau_x &=   && {-}1             && {+}1             & \qqquad
  \tau_y &=   && {+}1             && {-}1             \nonumber
\end{alignat}
with this
\begin{alignat}{2}
  m_5     &= -g \, x_c \, L & \qqquad & \CRNO
  m_{51}  &= -g \, s_x & \qqquad
  m_{52}  &= \frac{\tau_x \, g}{1 + p_z} \, \frac{1 - c_x}{\om_x^2} \CRNO
  m_{511} &= \frac{\tau_x \,\, \om_x^2}{4} \, (L - c_x \, s_x) & \qqquad
  m_{533} &= \frac{\tau_y \,\, \om_y^2}{4} \, (L - c_y \, s_y) \CRNO
  m_{512} &= \frac{-\tau_x \,\, \om_x^2}{2 \, (1 + p_z)} \, s_x^2 & \qqquad
  m_{534} &= \frac{-\tau_y \,\, \om_y^2}{2 \, (1 + p_z)} \, s_y^2 \CRNO
  m_{522} &= \frac{-1}{4 \, (1 + p_z)^2} \, (L + c_x \, s_x) & \qqquad
  m_{544} &= \frac{-1}{4 \, (1 + p_z)^2} \, (L + c_y \, s_y) \nonumber
\end{alignat}

%-----------------------------------------
\subsection{Sextupole}
\index{Sextupole}

%-----------------------------------------
\subsection{Sol\_Quad}
\index{Sol_Quad}

The Hamiltonian is
\Begineq
  H = \frac{(p_x + \frac{k_s }{2}\, y)^2}{2 (1 + p_z)} + 
  \frac{(p_y - \frac{k_s}{2} \, x)^2}{2 (1 + p_z)} + \frac{k_1}{2} (x^2 - y^2)
\Endeq
Solving the equations of motion gives
\begin{align}
  x   &\rightarrow m_{11} \, x + m_{12} \, p_x + m_{13} \, y + m_{14} \, p_y \CRNO
  p_x &\rightarrow m_{21} \, x + m_{22} \, p_x + m_{23} \, y + m_{24} \, p_y \CRNO
  y   &\rightarrow m_{31} \, x + m_{32} \, p_x + m_{33} \, y + m_{34} \, p_y \CRNO
  p_y &\rightarrow m_{41} \, x + m_{42} \, p_x + m_{43} \, y + m_{44} \, p_y \CRNO
  z   &\rightarrow z + \sum_{j = 1}^4 \sum_{k = j}^4 m_{5jk} \, r_j \, r_k  \CRNO
  p_z &\rightarrow p_z \nonumber
\end{align}
where
\begin{alignat}{2}
  m_{11} &= \frac{1}{2 \, f} \, \left( f_{0+} \, c + f_{0-} \, c_h \right) & \qqquad
  m_{31} &= -m_{24} \CRNO
  m_{12} &= \frac{1}{2 \, f \, (1 + p_z)} \, 
            \left( \frac{f_{++}}{\om_+} \,  s + \frac{f_{--}}{\om_-} \, s_h \right) & \qqquad
  m_{32} &= -m_{14} \CRNO
  m_{13} &= \frac{\ks}{4 \, f} \, 
            \left( \frac{f_{+-}}{\om_+} \, s +\frac{f_{-+}}{\om_-} \, s_h \right) & \qqquad
  m_{33} &= \frac{1}{2 \, f} \, \left( f_{0-} \, c + f_{0+} \, c_h \right) \CRNO
  m_{14} &= \frac{\ks}{f \, (1 + p_z)} \, \left( -c + c_h \right) & \qqquad
  m_{34} &= \frac{1}{2 \, f \, (1 + p_z)} \, 
            \left( \frac{f_{+-}}{\om_+} \, s + \frac{f_{-+}}{\om_-} \, s_h \right) \CRNO
  m_{21} &= \frac{-(1 + p_z)}{8 \, f} \, 
            \left( \frac{\xi_{1+}}{\om_+} \, s + \frac{\xi_{2+}}{\om_-} s_h \right) & \qqquad
  m_{41} &= -m_{23} \CRNO
  m_{22} &= m_{11} & \qqquad
  m_{42} &= -m_{13} \CRNO
  m_{23} &= \frac{\ks^3 \, (1 + p_z)}{4 \, f} \, \left( c - c_h \right) & \qqquad
  m_{43} &= \frac{-(1 + p_z)}{8 \, f} \, 
            \left( \frac{\xi_{1-}}{\om_+} \, s + \frac{\xi_{2-}}{\om_-} \, s_h \right) \CRNO
  m_{24} &= \frac{\ks}{4 \, f} \, 
            \left( \frac{f_{++}}{\om_+} \, s + \frac{f_{--}}{\om_-} \, s_h \right) & \qqquad
  m_{44} &= m_{33} \nonumber
\end{alignat}
and
\begin{alignat}{2}
  \kone        &= \frac{k_1}{1 + p_z} & \qqquad 
  \ks          &= \frac{k_s}{1 + p_z} \CRNO
  f            &= \sqrt{\ks^4 + 4 \, \kone^2} & \qqquad
  f_{\pm0}     &= f \pm \ks^2 \CRNO
  f_{0\pm}     &= f \pm 2 \, \kone & \qqquad
  f_{\pm\pm}   &= f \pm \ks^2 \pm 2 \, \kone \CRNO
  \om_+        &= \sqrt{\frac{f_{+0}}{2}} & \qqquad
  \om_-        &= \sqrt{\frac{f_{-0}}{2}} \\
  s            &= \sin (\om_+ \, L) & \qqquad
  s_h          &= \sinh (\om_- \, L) \CRNO
  c            &= \cos (\om_+ \, L) & \qqquad
  c_h          &= \cosh (\om_- \, L) \CRNO
  \xi_{1\pm} &= \ks^2 \, f_{+\mp} \pm 4 \, \kone \, f_{+\pm} & \qqquad
  \xi_{2\pm} &= \ks^2 \, f_{-\pm} \pm 4 \, \kone \, f_{-\mp} \nonumber
\end{alignat}

The $m_{5jk}$ terms are obtained via \Eq{zz121p}
\begin{align}
  m_{5jk} = - \frac{\tau_{jk}}{2 (1 + p_z)^2} \int \! ds \, 
  & \left[ 
    \left( m_{2j} + \frac{k_s}{2} \, m_{3j} \right) \, 
    \left( m_{2k} + \frac{k_s}{2} \, m_{3k} \right)   
  \right. + \\
  & \hspace*{15ex} \left.
    \left( m_{4j} - \frac{k_s}{2} \, m_{1j} \right) \, 
    \left( m_{4k} - \frac{k_s}{2} \, m_{1k} \right) 
  \right] \nonumber
\end{align}
where
\Begineq
  \tau_{jk} = 
  \begin{cases}
    1 & j = k \\
    2 & j \ne k 
  \end{cases}
\Endeq
The needed integrals involve the product of two trigonometric or
hyperbolic functions. These integrals are trivial to do but the
explicit equations for $m_{5jk}$ are quite long and in the interests of
brevity are not reproduced here.

%-----------------------------------------
\subsection{Solenoid}
\index{Solenoid}

The Hamiltonian is
\Begineq
  H = \frac{ \left( p_x + \frac{k_s}{2} \, y \right)^2}{2 \, (1 + p_z)} + 
  \frac{ \left( p_y - \frac{k_s}{2} \, x \right)^2}{2 \, (1 + p_z)} 
\Endeq
The solution is
\begin{align}
  x   &\rightarrow \frac{1 + c}{2} \, x + \frac{s}{k_s} \, p_x +
           \frac{s}{2} \, y + \frac{1 - c}{k_s} \, p_y \CRNO
  p_x &\rightarrow \frac{-k_s \, s}{4} \, x + \frac{1 + c}{2} \, p_x - 
           \frac{k_s \, (1 - c)}{4} \, y + \frac{s}{2} \, p_y \CRNO
  y   &\rightarrow \frac{-s}{2} \, x - \frac{1 - c}{k_s} \, p_x +
           \frac{1 + c}{2} \, y + \frac{s}{k_s} \, p_y \\      
  p_y &\rightarrow \frac{k_s \, (1 - c)}{4} \, x + \frac{-s}{2} \, p_x -
            \frac{k_s \, s}{4} \, y + \frac{1 + c}{2} \, p_y \CRNO 
  z   &\rightarrow z + \frac{L}{2 \, (1 + p_z)^2} \, 
                   \left[ \left( p_x + \frac{k_s}{2} \, y \right)^2 +
                          \left( p_y - \frac{k_s}{2} \, x \right)^2 \right] \CRNO
  p_z &\rightarrow p_z \nonumber
\end{align}
where
\begin{align}
  c &= \cos \left( \frac{k_s}{2} \, L \right) \CRNO
  s &= \sin \left( \frac{k_s}{2} \, L \right)
\end{align}

%-----------------------------------------
\subsection{Wiggler, Map\_type}
\index{Wiggler}



%-----------------------------------------
\subsection{Wiggler, Periodic Type}
\index{Wiggler}

The horizontal motion looks like a drift with a superimposed
sinusoidal oscillation. It is assumed that there is an integer number
of periods in the oscillation so that the exit horizontal coordinates
can be calculated from the initial coordinates using the equations for
a drift. The vertical motion is a quadratic superimposed with a
octupole. Vertical motion is calculated using a kick-drift-kick model.

%-----------------------------------------
\subsection{Offset Transformation of a First Order Map}
\index{x_pitch}\index{y_pitch}\index{tilt}

Given a first order map
\Begineq
  \bfr \rightarrow \bfr \, \bfM + \bfK
\Endeq



\end{document}
