\chapter {Control Elements}
\label{c:control}
\index{Element!control}
\index{Element!lord}
\index{Element!slave}

It is possible to have elements controlling the attributes of other
elements. These \vn{lord} elements are meant, for example, to mimic
the effect of changing a knob in the control room. For this purpose
the lattice is split into two sections: The first section is the list
of elements that \bmad uses for any analysis (tracking, Twiss
parameter calculations, etc.). This first part is called the
\vn{regular} part of the lattice.
Elements in the \vn{regular} part are either
\vn{slave} elements if they have a controlling lord or \vn{free}
elements if they do not.  The second section consists solely of
\vn{lord} elements. \vn{Lord} elements can control other \vn{lord}
elements and a hierarchy of \vn{lord} elements may be established.

There are five types of lord elements: 
\begin{Itemize}
\item 
\index{Group}
\vn{Group} lord elements are used to make variations in attributes. For example, to
simulate the action of a control room knob that changes the beam tune
in a storage ring, a \vn{Group} can be used to vary the strength of
selected quads in a specified manner. \vn{Groups} are covered in \sref{s:group}.
\item
\index{Overlay}
An \vn{Overlay} lord is like a \vn{group} except that \vn{overlays}
set the value (not a change in) the attributes they
control. \vn{Overlays} are covered in \sref{s:overlay}.
\item
A \vn{superposition} lord is created when elements are superimposed on
top of other elements. This is covered in \sref{s:super}.
\item
\index{I_Beam}
\vn{I_Beam} elements mimic the effect of a support girder. This is
covered in \sref{s:i_beam}.
\item
\index{Multipass}
\vn{Multipass} elements are used when the beam recirculates through
the same element multiple times. This is covered in
\sref{s:multipass}.
\end{Itemize}

%-----------------------------------------------------------------------------
\section{Overlay Elements}
\label{s:overlay}

An \vn{overlay} element is used to control the attributes of other elements. 
For example: 
\begin{example}
  over1: overlay = \{a_ele, b_ele/2.0\}, hkick = 0.003
  over2: overlay = \{b_ele\}, hkick
  over2[hkick] = 0.9
  a_ele: quad, hkick = 0.05, ...
  b_ele: rbend, ...
  this_line: line = ( ... a_ele, ... b_ele, ... )
  use, this_line
\end{example}

In the example the overlay \vn{over1} controls the \vn{hkick}
attribute of the "slave" elements \vn{a_ele} and
\vn{b_ele}. \vn{over2} controls the hkick attribute of just
\vn{b_ele}. \vn{over1} has a \vn{hkick} value of 0.003 and \vn{over2}
has been assigned a value for \vn{hkick} of 0.9.

There are coefficients associated with the control of a slave element. 
The default coefficient is 1.0. To specify a coefficient use a slash "/" 
after the element name followed by the coefficient. In the above example 
the coefficient for the control of \vn{b_ele} from \vn{over1} is 2.0 
and for the others the default 1.0 is used. thus 
\begin{example}
  a_ele[hkick] = over1[hkick]
               = 0.003
  b_ele[hkick] = over2[hkick] + 2 * over1[hkick] 
               = 0.906
\end{example}

An \vn{overlay} will control all elements of a given name.  Thus, in
the above example, if there are multiple elements in \vn{this_line}
with the name \vn{b_ele} then the \vn{over1} and \vn{over2} overlays
will control the hkick attribute of all of them.

Note: Overlays completely determine the value of the attributes that
are controlled by the overlay. in the above example, the hkick of 0.05
assigned directly to \vn{a_ele} is overwritten by the overlay action
of \vn{over1}.

\noindent The default value for an overlay is 0 so for example
\begin{example}
  over3: overlay = \{c_ele\}, k1
\end{example}
will make \vn{c_ele[k1]} = 0. Overlays can also control more than one
type of attribute as the following example shows
\begin{example}
  over4: overlay = \{this_quad[k1]/3.4, this_sextupole[k2], ...\}, hkick
\end{example}


%-----------------------------------------------------------------------------
\section{Group Elements}
\label{s:group}
\index{Group}
 
\vn{group} is like \vn{overlay} in that a \vn{group} element controls
the attribute values of other ``slave'' elements.  A \vn{group}
element is used to make changes in value. This is unlike an
\vn{overlay} which sets a specific value directly. An example will
make this clear
\begin{example}
  gr: group = \{q1\}, k1 
  gr[command] = 0.34 
  q1, quad, l = ...
  q1[k1] = 0.57
\end{example}
In this example the group \vn{gr} controls the \vn{k1} attribute of
the element \vn{q1}. Unlike overlays, values are assigned to group
elements using the \vn{command} attribute.  When a lattice file is
read in then command values for any groups are always applied
last. This is independent of the order that they appear in the file.
Thus in this example the value of q1[k1] would be $0.91 = 0.57 + 0.34$.
When the changes are made to the slave attributes the value of
\vn{command} is stored in the \vn{group}'s \vn{old_command} attribute.
After the lattice is read in a program can change the \vn{gr[command]}
attribute and this change will be added to the value of
\vn{q1[k1]}. The bookkeeping routine that transfers the change from
\vn{gr[command]} to \vn{q1[k1]} doesn't care what the current value of
\vn{q1[k1]} is. It only knows it has to change it by the change in
\vn{gr[command]}.

A \vn{group} can be used to control an elements position and length
using the attributes
\begin{example}
  accordion_edge  ! Element grows or shrinks symmetrically
  start_edge      ! Varies element's starting edge s-position
  end_edge        ! Varies element's ending edge s-position
  symmetric_edge  ! Varies element's overall s-position. Constant length.
  s_offset        ! Same as symmetric_edge
\end{example}
In all cases the total length of the lattice is kept invariant.
\vn{accordion_edge} varies the edges of an element so that the center
of the element is fixed but the length varies. with
\vn{accordion_edge} a change of, say, 0.1 in a \vn{group}'s
\vn{command} attribute moves both edges of the element by 0.1 meters
so that the length of the element changes by 0.2 meters. To keep the
total lattice length invariant the lengths of the elements to either
side are varied accordingly.
For example
\begin{example}
  q10: quad, l = ...
  q11: quad, l = ...
  d1: drift, l = ...
  d2: drift, l = ...
  this_line: line = (... d1, q10, d2, q11, ...)
  gr2: group = \{q10\}, start_edge = 0.1
\end{example}
This last line that defines \vn{gr2} is just a shorthand notation for
\begin{example}
  gr2: group = \{q10\}, start_edge 
  gr2[command] = 0.1
\end{example}
The effect will be to lengthen the length of \vn{q10} and shorten the
length of \vn{d1}.

The full list of attributes of a group are
\begin{example}
  command         
  old_command     
  coef            
  type            ! See section \ref{s:string}
  alias           ! See section \ref{s:string}
  descrip         ! See section \ref{s:string}
\end{example}
The \vn{coef} attribute is not used by any \bmad\ routine. It is
defined for individual programs to store, say, a needed conversion
factor.

Like \vn{overlay}s, coefficients can be specified for the individual
elements under a \vn{group}'s control and \vn{group}s can control more
than one type of attribute. For example
\begin{example}
  gr3: group = \{q1[k1]/-1.0, q2[tilt], oct1/-2.0\}, k3
  gr3[command] = 2.0
  gr3[old_command] = 1.5
\end{example}
In this example \vn{gr3} controls 3 attributes of 3 different
elements.  The change in \vn{gr3} when the lattice is read in is $0.5
= 2.0 - 1.5$.  this 0.5 change will change \vn{q1[k1]} by $-0.5 = -1
\times 0.5$, \vn{q2[tilt]} will change by 0.5 and \vn{oct1[k3]} will
change by $-1.0 = -2.0 * 0.5$.

%-----------------------------------------------------------------------------
\section{Superposition}
\label{s:super}
\index{Superimpose|textbf}

In practice the field at a particular point in the lattice may be due
to more than one physical element. A common example is a quadrupole
inside a solenoid. \bmad has a mechanism to handle some of these
cases using what is called ``superposition''. A simple example shows
how this works:
\begin{example}
  Q: quad, l = 10
  D: drift, l = 4
  S: solenoid, l = 6, superimpose, ref = q, ref_end
  lat: line = (Q, D)
  use, lat
\end{example}
The \vn{superimpose} attribute of element \vn{S} superimposes \vn{S}
over the lattice \vn{(Q, D)}. The placement of \vn{S} is such that the
center of \vn{S} coincides with the end of \vn{Q} (more on how
superimposed elements get placed later). The regular part of the
lattice list (the part that one does tracking through) Looks like:
\begin{example}
        Element   Key         Length
  1)    Q{\B}        Quadrupole   7
  2)    Q{\B}S       Sol_quad     3
  3)    S{\B}1       Solenoid     3
  4)    D{\B}        Drift        1
\end{example}
What \bmad has done is to split the original elements \vn{(Q, D)} at
the edges of \vn{S}.  The first element \vn{Q\B} is the part of \vn{Q}
that is outside of \vn{S}. Since this is only part of \vn{Q} \bmad
has put a \vn{\B} in the name so that there will be no confusion.
(\vn{\B} has no special meaning other than the fact that \bmad uses
it for mangling names). The next element, \vn{Q{\B}S}, is the part of
\vn{Q} that is inside \vn{S}. \vn{Q{\B}S} is a combination
solenoid/quadrupole element as one could expect. \vn{S{\B}1} is the part
of \vn{S} that is outside \vn{Q} so this element is just a
solenoid. Finally, \vn{D\B} is the rest of the drift outside \vn{S}.

With the lattice broken up like this \bmad has constructed something
that can be easily analyzed. However, the original elements \vn{Q} and
\vn{S} still exist within the lord section of the lattice. \bmad has bookkeeping
routines so that if a change is made to the \vn{Q} or \vn{S} elements then these
changes can get propagated to the corresponding slaves.  It does
not matter which element is superimposed. Thus, in the above example,
\vn{S} could have been put in the Beam Line (with a drift before it)
and \vn{Q} could then have been superimposed on top and the result
would have been the same (except that the split elements could have
different names).

Superpositions are restricted in that \bmad may not have an
appropriate element for the superimposed fields. For example, a
solenoid can not be superimposed over an octupole.  If a zero length
element, such as a marker, is superimposed with some other element (or
vice versa) the element is just split in two. For example:
\begin{example}
  Q: quad, l = 10
  M: marker, superimpose, offset = 6
  lat: line = (Q)
  use, lat
\end{example}
The resulting lattice (first part) would be
\begin{example}
        Element   Key           Length
  1)    Q{\B}1       Quadrupole    6
  2)    M         Marker        0
  3)    Q{\B}2       Quadrupole    4
\end{example}
and the second part of the lattice would have the \vn{Q} element.
 
The placement of a superimposed element is determined by three
factors: A point on the superimposed element, a reference point in the
lattice line, and an offset between the points. The attributes that
determine these three quantities are:
\begin{example}
  ref = <element name in lattice>
  offset = <length>      (default = 0)
  ref_beginning
  ref_center             (default)
  ref_end
  ele_beginning
  ele_center             (default)
  ele_end
\end{example}
\vn{ele_beginning} \vn{ele_center} \vn{ele_end} makes the point on the
superimposed element the beginning edge, the center, or the
end edge respectively. If neither of these attributes are
given the default is to use the element center.

Superposition may be done with any element except \vn{drift},
and \vn{Group}, \vn{Overlay}, and \vn{I_Beam} control elements.
Additionally, \vn{Superpose} may not be used with \vn{close}.

%-----------------------------------------------------------------------------
\section{Multipass}
\label{s:multipass}
\index{Multipass|textbf}

Some lattices have the beam recirculating through the same element
multiple times. For example, an Energy Recovery Linac (ERL) will
circulate the beam back through the LINAC part to retrieve the energy
in the beam. In \bmad this situation can simulated using the
\vn{multipass} attribute. A simple example shows how this works.
\index{Expand_lattice!statement}
\begin{example}
  A: lcavity
  linac_part: line[multipass] = (A, ...)
  my_line: line = (linac_part, ..., linac_part)
  use, my_line
  expand_lattice
  A\B2[dphi0] = 0.5
\end{example}
The regular part of the lattice (the part that gets tracked through)
consists of two slave elements
\begin{example}
  A\B1, ..., A\B2, ...
\end{example}
Since the two elements are derived from a \vn{multipass} line they are
given unique names by adding a \vn{{\B}n} suffix. In addition there is
a lord element (that doesn't get tracked through) called \vn{A} in the
lord part of the lattice. Changes to attributes of the lord \vn{A}
element will be passed to the slave elements by \bmad's bookkeeping
routines. Assuming \vn{A\B1} is an accelerating cavity, to make
\vn{A\B2} a deaccelerating cavity the \vn{dphi0} attribute of
\vn{A\B2} is set to 0.5. This is the one attribute that \bmad's
bookkeeping routines will not touch when transferring attribute values
from \vn{A} to its slaves. Notice that the \vn{dphi0} attribute had to
be set after \vn{expand_lattice} is used to expand the lattice since
\bmad does immediate evaluation and \vn{A\B2} does not exist before
the lattice is expanded.

Sublines of a multipass line are automatically multipass:
\begin{example}
  a_line: line = (...)
  m_line: line[multipass] = (..., a_line, ...)
\end{example}
In this example \vn{a_line} is implicitly multipass.

Multiple elements of the same name in a multipass line are considered 
physically distinct:
\begin{example}
  m_line: line[multipass] = (A, A, B)
  u_line: line = (m_line, m_line)
  use, u_line
\end{example}
In this example the regular part of the lattice is
\begin{example}
  A\B1, A\B1, B\B1, A\B2, A\B2, B\B2
\end{example}
In the control section of the lattice there will be two multipass
lords called \vn{A} and one called \vn{B}. The first \vn{A} lord 
controls the 1\St and 4\Th elements in the regular part of the lattice 
and the second \vn{A} lord controls the 2\Nd and 5\Th elements.

If a multipass line is reversed then the elements are considered to be
transversed backwards:
\begin{example}
  m_line: line[multipass] = (A, A, B)
  u_line: line = (m_line, -m_line)
  use, u_line
\end{example}
In this example the regular part of the lattice is
\begin{example}
  A\B1, A\B1, B\B1, B\B2, A\B2, A\B2
\end{example}
Here the 1\St and 6\Th elements are connected to a multipass lord and the
2\Nd and 5\Th elements are connected to a different multipass lord.