\chapter{Elements}

Intro and basic syntax

\section{MAD elements recognized by Bmad}
Most element types available in \mad\ are provided in \bmad.  In some cases, however, the attributes
associated with a particular element type are not identical.  Table \ref{tab:mad_elements} summarizes the
element types which \bmad\ inherits from \mad.

\begin{table}
\label{tab:mad_elements}\center
{\tt
\begin{tabular}{|l|l|} \hline
    ab\_multipole &  type, a$n$, b$n$ $(0 \le n \le 20)$, radius, tilt     \\ \hline
    beambeam      &  type, sig\_x, sig\_y, charge, sig\_z, n\_slice        \\ \hline
    custom        &  type, l, val1, val2, val3, ..., val12                 \\ \hline
    drift         &  type, l                                               \\ \hline
    ecollimator   &  type, l, x\_limit, y\_limit                           \\ \hline
    elseparator   &  type, l, gap                                          \\ \hline
    group         &  type, command, old\_command, coef                     \\ \hline
    hkicker       &  type, l, kick, tilt                                   \\ \hline
    instument     &  type, l                                               \\ \hline
    kicker        &  type, l, hkick, vkick, h\_displace, v\_displace       \\ \hline
    lcavity       &  type, l, gradient, phi0, rf\_frequency \\ \hline
    marker        &  type \\ \hline
    monitor       &  type, l \\ \hline
    multipole     &  type, k$n$l, t$n$ $(0 \le n \le 20)$, tilt \\ \hline
    octupole      &  type, l, k3, tilt, b\_gradient \\ \hline
    overlay       &  {\sl see chap.~8} \\ \hline
    patch         &  type, x\_offset, x\_pitch, y\_offset, y\_pitch, z\_offset, \\
                  &  de\_offset \\ \hline
    quadrupole    &  type, l, k1, tilt, b\_gradient \\ \hline
    rbend         &  type, l, angle, e1, e2, k1, tilt, g, delta\_g, \\
                  &  roll, rho, b\_field \\ \hline
    rcollimator   &  type, l, x\_limit, y\_limit \\ \hline
    rfcavity      &  type, l, volt, harmon, phi0 (= lag)  \\ \hline
    sbend         &  type, l, angle, e1, e2, k1, tilt, g, delta\_g, \\
                  &  roll, rho, b\_field \\ \hline
    sextupole     &  type, l, k2, tilt, b\_gradient \\ \hline
    solenoid      &  type, l, ks, b\_field \\ \hline 
    sol\_quad     &  type, l, k1, ks, tilt \\ \hline
    taylor        &  type, \{out\_index : coef : e1 e2 e3 e4 e5 e6\} \\ \hline
    vkicker       &  type, l, kick, tilt \\ \hline
    wiggler       &  type, l, b\_max, n\_pole, tilt, b$n$ $(n=2,4,6,8)$, radius \\ \hline
    wiggler       &  type, l, polarity, z\_patch, \\
                  &  term($i$) = \{coef, kx, ky, kz, phi\_z\} \\ \hline
\end{tabular}}
\caption{Summary \bmad\ elements}
\end{table}

\subsection{Beambeam}
In \bmad, {\tt x\_offset} and {\tt y\_offset} are used to offset the {\tt beambeam} element instead of
the \mad\ standard attributes {\tt xma} and {\tt yma}.
\begin{itemize}
\item For a crossing angle use {\tt x\_pitch} and {\tt y\_pitch} (This is the full crossing angle,
not the half-angle).
\item {\tt n\_slice} is the number of equal charge chunks which the strong beam is sliced.
      Default is {\tt n\_slice} = 1.
\item For slicing, you need a non-zero {\tt sig\_z}.
\item {\tt charge}=-1: Opposite beam has the opposite charge (default).
\item {\tt charge}=+1: Opposite beam has the same charge.   
\end{itemize}

\subsection{Drift}
\subsection{Elseperator}
For an {\tt elseparator}, the kick is determined by {\tt hkick} and {\tt vkick}. The {\tt gap} for an
{\tt elseparator} is used to compute the electric field for a given kick. The voltage is a dependent
attribute determined by:

Voltage (V) = $10^{-9}$ * {\tt kick} * {\tt E} [Gev] * {\tt gap} [m] / {\tt L} [m] 

\subsection{Kicker}
\subsection{Marker}
\subsection{Multipole}
\subsection{Octupole}
\subsection{Quadrupole}
\subsection{Rbend}
\subsection{Rfcavity}
\subsection{Sbend}
\subsection{Sextupole}
\subsection{Solenoid}

\bmad\ provides additional element types corresponding to physical elements, as well as special
types which control other elements, or represent superimposed or hybrid elements.
Table \ref{tab:bmad_elements} summarizes element types unique to \bmad.

\begin{table}\label{tab:bmad_elements}\center
{\tt\begin{tabular}{|l|l|} \hline

\end{tabular}}
\caption{Summary of element types unique to \bmad}
\end{table}
\subsection{Ab{\tt\_}multipole}
\subsection{Custom}
\subsection{Group}
\subsection{Lcavity}
\subsection{Overlay}
\subsection{Dol{\tt\_}quad}
\subsection{Taylor}
\subsection{Wiggler} Two of these?
\subsection{Patch}
