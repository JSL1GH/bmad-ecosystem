\chapter{Elements}

A lattice for a storage ring or linac is made up of a collection of
elements --- Quadrupoles, Bends, etc. This chapter discusses the
various types of elements available in \bmad.


\section{Bmad Element List}


Most element types available in \mad\ are provided in \bmad.
Additionally, \bmad\ provides a number of element types that are not
available in \mad.  A word of caution: In some cases where both \mad\
and \bmad\ provide the the same element type, there will be an overlap of 
the attributes available but the two sets of attributes will not be the same.
The list of element types known to \bmad\ is shown in Table~\ref{tab:elements}
\begin{table}[h]
\centering
{\tt
\begin{tabular}{|l|l||l|l|} \hline
  {\it Element} & {\it Section}  & {\it Element} & {\it Section} \\ \hline
  ab\_multipole & \ref{s:ab_m}   &  octupole     & \ref{s:oct}   \\ \hline
  beambeam      & \ref{s:bbi}    &  overlay      & \ref{s:over}  \\ \hline
  custom        & \ref{s:custom} &  patch        & \ref{s:patch} \\ \hline
  drift         & \ref{s:drift}  &  quadrupole   & \ref{s:quad}  \\ \hline
  ecollimator   & \ref{s:ecol}   &  rbend        & \ref{s:rbend} \\ \hline
  elseparator   & \ref{s:elsep}  &  rcollimator  & \ref{s:rcol}  \\ \hline
  group         & \ref{s:group}  &  rfcavity     & \ref{s:rfcav} \\ \hline
  hkicker       & \ref{s:hk}     &  sbend        & \ref{s:sbend} \\ \hline
  instument     & \ref{s:inst}   &  sextupole    & \ref{s:sex}   \\ \hline
  kicker        & \ref{s:k}      &  solenoid     & \ref{s:sol}   \\ \hline
  lcavity       & \ref{s:lcav}   &  sol\_quad    & \ref{s:sq}    \\ \hline
  marker        & \ref{s:mark}   &  taylor       & \ref{s:tay}   \\ \hline
  monitor       & \ref{s:mon}    &  vkicker      & \ref{s:vk}    \\ \hline
  multipole     & \ref{s:m}      &  wiggler      & \ref{s:wig}   \\ \hline
\end{tabular}}
\caption{\bmad\ elements.}
\label{tab:elements}\center
\end{table}

%-----------------------------------------------------------------
\subsection{AB\_Multipole}
\label{s:ab_m}


\begin{table}[h]
\centering {
\begin{tabular}{|l|l||l|l|} \hline
  {\sl Attribute} & {\sl Description}  & {\sl Attribute} & {\sl description} \\ \hline
  tilt         = Real & \ref{s:offset} &  type               = String  & \ref{s:string} \\ \hline
  x\_offset    = Real & \ref{s:offset} &  alias              = String  & \ref{s:string} \\ \hline
  y\_offset    = Real & \ref{s:offset} &  descrip            = String  & \ref{s:string} \\ \hline
  s\_offset    = Real & \ref{s:offset} &  mat6\_calc\_method = Switch  & \ref{s:track}  \\ \hline
  x\_limit     = Real & \ref{s:limit}  &  tracking\_method   = Switch  & \ref{s:track}  \\ \hline
  y\_limit     = Real & \ref{s:limit}  &  is\_on             = Logical & \ref{s:is_on}  \\ \hline
  aperture     = Real & \ref{s:limit}  &  a$n$, b$n$         = Real    & \ref{s:ab}     \\ \hline  
  beam\_energy = Real & \ref{s:energy} &  radius             = Real    & \ref{s:ab}     \\ \hline
\end{tabular}}
\end{table}

%-----------------------------------------------------------------
\subsection{BeamBeam}
\label{s:bbi}


In \bmad, \vn{x_offset} and \vn{y_offset} are used to offset the
\vn{beambeam} element instead of the \mad\ standard attributes
\vn{xma} and \vn{yma}.
\begin{itemize}
\item For a crossing angle use \vn{x_pitch} and \vn{y_pitch} (This is the full crossing angle,
not the half-angle).
\item \vn{n_slice} is the number of equal charge chunks which the strong beam is sliced.
      Default is \vn{n_slice} = 1.
\item For slicing, you need a non-zero \vn{sig_z}.
\item \vn{charge}=-1: Opposite beam has the opposite charge (default).
\item \vn{charge}=+1: Opposite beam has the same charge.   
\end{itemize}

%-----------------------------------------------------------------
\subsection{Custom}
\label{s:custom}

%-----------------------------------------------------------------
\subsection{Drift}
\label{s:drift}

%-----------------------------------------------------------------
\subsection{Ecollimator}
\label{s:ecol}

%-----------------------------------------------------------------
\subsection{Elseperator}
\label{s:elsep}

For an \vn{elseparator}, the kick is determined by \vn{hkick} and
\vn{vkick}. The \vn{gap} for an \vn{elseparator} is used to compute
the electric field for a given kick. The voltage is a dependent
attribute determined by:
\begin{example}
  Voltage (V) = kick * E [ev] * gap [m] / L [m] 
\end{example}

%-----------------------------------------------------------------
\subsection{Group}
\label{s:group}

%-----------------------------------------------------------------
\subsection{Hkicker}
\label{s:hk}

%-----------------------------------------------------------------
\subsection{Instrument}
\label{s:inst}

%-----------------------------------------------------------------
\subsection{Kicker}
\label{s:k}

%-----------------------------------------------------------------
\subsection{Lcavity}
\label{s:lcav}

%-----------------------------------------------------------------
\subsection{Marker}
\label{s:mark}

%-----------------------------------------------------------------
\subsection{Monitor}
\label{s:mon}

%-----------------------------------------------------------------
\subsection{Multipole}
\label{s:m}

%-----------------------------------------------------------------
\subsection{Octupole}
\label{s:oct}

%-----------------------------------------------------------------
\subsection{Overlay}
\label{s:over}

%-----------------------------------------------------------------
\subsection{Patch}
\label{s:patch}

%-----------------------------------------------------------------
\subsection{Quadrupole}
\label{s:quad}

%-----------------------------------------------------------------
\subsection{Rbend}
\label{s:rbend}

%-----------------------------------------------------------------
\subsection{Rcollimator}
\label{s:rcol}

%-----------------------------------------------------------------
\subsection{Rfcavity}
\label{s:rfcav}

%-----------------------------------------------------------------
\subsection{Sbend}
\label{s:sbend}

%-----------------------------------------------------------------
\subsection{Sextupole}
\label{s:sex}

%-----------------------------------------------------------------
\subsection{Solenoid}
\label{s:sol}

%-----------------------------------------------------------------
\subsection{Sol\_Quad}
\label{s:sq}

%-----------------------------------------------------------------
\subsection{Taylor}
\label{s:tay}

%-----------------------------------------------------------------
\subsection{Vkicker}
\label{s:vk}

%-----------------------------------------------------------------
\subsection{Wiggler} 
\label{s:wig}



