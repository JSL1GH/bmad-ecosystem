\documentclass{book}
\usepackage{graphicx}
\newcommand{\extref}[1]{$\S$\ref*{#1}}   % No hyperlink. For external refs. \extref
\newcommand{\comma}{\> ,}
\newcommand{\period}{\> .}
\newcommand{\wt}{\widetilde}
\newcommand{\grv}{\textasciigrave}
\newcommand{\hyperbf}[1]{\textbf{\hyperpage{#1}}}
\newcommand{\Ss}{\(^*\)}
\newcommand{\Dd}{\(^\dagger\)}

\newcommand{\AND}{&& \hskip -17pt\relax}
\newcommand{\CR}{\\}
\newcommand{\CRNO}{\nonumber \\}
\newcommand{\dstyle}{\displaystyle}

\newcommand{\Begineq}{\begin{equation}}
\newcommand{\Endeq}{\end{equation}}
\newcommand{\NoPrint}[1]{}

\newcommand{\pow}[1]{\cdot 10^{#1}}
\newcommand{\Bf}[1]{{\bf #1}}
\newcommand{\bfr}{\Bf r}

\newcommand{\bmad}{{\sl Bmad}\xspace}
\newcommand{\tao}{{\sl Tao}\xspace}
\newcommand{\mad}{{\sl MAD}\xspace}
\newcommand{\cesr}{{\sl CESR}\xspace}

\newcommand{\sref}[1]{\S\ref{#1}}
\newcommand{\Sref}[1]{Sec.~\sref{#1}}
\newcommand{\cref}[1]{Chapter~\ref{#1}}

\newcommand{\Newline}{\hfil \\ \relax}

\newcommand{\eq}[1]{{(\protect\ref{#1})}}
\newcommand{\Eq}[1]{{Eq.~(\protect\ref{#1})}}
\newcommand{\Eqs}[1]{{Eqs.~(\protect\ref{#1})}}

\newcommand{\vn}{\ttcmd}           % For variable names
\newcommand{\vni}{\ttcmdindx}
\newcommand{\cs}{\ttcmd}           % For code source
\newcommand{\cmd}{\ttcmd}          % For Unix commands
\newcommand{\rn}{\ttcmd}           % For Routine names
\newcommand{\tn}{\ttcmd}           % For Type (structure) names
\newcommand{\bn}[1]{{\bf #1}}       
\newcommand{\toffset}{\vskip 0.01in}
\newcommand{\rot}[1]{\begin{rotate}{-45}#1\end{rotate}}

\newcommand{\data}{{\mbox{data}}}
\newcommand{\reference}{{\mbox{ref}}}
\newcommand{\model}{{\mbox{model}}}
\newcommand{\base}{{\mbox{base}}}
\newcommand{\design}{{\mbox{design}}}
\newcommand{\meas}{{\mbox{meas}}}
\newcommand{\var}{{\mbox{var}}}

\newcommand\ttcmd{\begingroup\catcode`\_=11 \catcode`\%=11 \dottcmd}
\newcommand\dottcmd[1]{\texttt{#1}\endgroup}

\newcommand\ttcmdindx{\begingroup\catcode`\_=11 \catcode`\%=11 \dottcmdindx}
\newcommand\dottcmdindx[1]{\texttt{#1}\endgroup\index{#1}}

\newcommand{\St}{$^{st}$\xspace}
\newcommand{\Nd}{$^{nd}$\xspace}
\newcommand{\Th}{$^{th}$\xspace}
\newcommand{\B}{$\backslash$}
\newcommand{\W}{$^\wedge$}

\newcommand{\cbar}[1]{\overline C_{#1}}

\newlength{\dPar}
\setlength{\dPar}{1.5ex}

\newenvironment{example}
  {\vspace{-3.0ex} \begin{alltt}}
  {\end{alltt} \vspace{-2.5ex}}

\newcommand\Strut{\rule[-2ex]{0mm}{6ex}}

\newenvironment{Itemize}
  {\begin{list}{$\bullet$}
    {\addtolength{\topsep}{-1.5ex} 
     \addtolength{\itemsep}{-1ex}
    }
  }
  {\end{list} \vspace*{1ex}}

\newcommand{\Section}[1]{\section{#1}\indent\vspace{-3ex}}

\newcommand{\SECTION}[1]{\section*{#1}\indent\vspace{-3ex}}

% From pg 64 of The LaTex Companion.

\newenvironment{ventry}[1]
  {\begin{list}{}
    {\renewcommand{\makelabel}[1]{\textsf{##1}\hfil}
     \settowidth{\labelwidth}{\textsf{#1}}
     \addtolength{\itemsep}{-1.5ex}
     \addtolength{\topsep}{-1.0ex} 
     \setlength{\leftmargin}{5em}
    }
  }
  {\end{list}}


%\makeatletter    % internal ``@'' commands can now be used. 
%\@addtoreset{chapter}{part}
%\makeatother     % Internal ``@'' commands are locked.
%\renewcommand{\thechapter}{\thepart.\arabic{chapter}}


%----------------------------------------------------------------

\begin{document}

\title{The \bmad Reference Manual}

\date{11 June, 2003}
\maketitle

%----------------------------------------------------------------
\section*{Overview}

\bmad (Otherwise known as "Baby MAD" or "Better MAD" or just plain "BE MAD!")
is a software subroutine library used for simulating 
relativistic charged--particle dynamics in high energy accelerators
and storage rings. \bmad can be used to read in lattice files, compute 
Twiss parameters, track particles, etc. 
These subroutines, written in  Fortran90, have been developed to:
\begin{enumerate}
\item Cut down on the time needed to develop programs,
\item Cut down on programming errors, and
\item Provide a standard input format for specifying lattices.
\end{enumerate}
\bmad has been developed at Cornell University's Laboratory for Elementary
Particle Physics and has been in use since the 1990's.
\vfill
\break
%----------------------------------------------------------------
\section*{Introduction}

The strength of \bmad is that as a subroutine library it provides a flexible
framework from which sophisticated simulation programs may easily be developed.
The weakness of \bmad comes from its strength: Someone must put the pieces 
together into a program. As a consequence this manual serves two masters:

The \bmad lattice input standard was developed using the MAD lattice
input standard as a starting point. MAD (Methodical Accelerator
Design) is a widely used stand--alone program developed at CERN by
Christoph Iselin for charged--particle optics calculations. The
limitations of the MAD program was the impetus for writing
\bmad. Since it can be convenient to do simulations with both MAD and
\bmad, differences and similarities between the two input formats are
noted in this guide.

Errors and omissions are a fact of life for any reference work and
comments from you, dear reader, are therefore most welcome. Please
send any missives (or chocolate, etc.) to:
\begin{verbatim}
  David Sagan <dcs16@cornell.edu>
\end{verbatim}


%----------------------------------------------------------------
\tableofcontents

\listoffigures

\listoftables

%----------------------------------------------------------------
%----------------------------------------------------------------
\part{Conventions and Physics}

%----------------------------------------------------------------
\chapter{Conventions}


%----------------------------------------------------------------
%----------------------------------------------------------------
\part{Language Reference}


%----------------------------------------------------------------
\chapter{Syntax}

%----------------------------------------------------------------
\chapter{Arithmetic Expressions}

%----------------------------------------------------------------
\chapter{Physical Units and Constants}

%----------------------------------------------------------------
\chapter{Parameters}

%----------------------------------------------------------------
\chapter{Elements}

%----------------------------------------------------------------
\chapter{Element Sequencing}

%----------------------------------------------------------------
\chapter{Superposition of Elements}

%----------------------------------------------------------------
\chapter{Elements Controlling Other Elements}

%----------------------------------------------------------------
\chapter{Tracking Methods}

%----------------------------------------------------------------
\chapter{Transfer Matrix Calculation Methods}

%----------------------------------------------------------------
%----------------------------------------------------------------
\part{Programmer's Guide}

%----------------------------------------------------------------
\chapter{The \bmad\ Distribution}

\section{Libraries in the \bmad\ Distribution}

When installing \bmad\ on a computer what one gets is not only the \bmad\
subroutine library but additionally subsidiary libraries upon which
subroutines in the \bmad\ library depend. There are 5 other libraries
that are used: cesr\_utils, forest, numerical\_recipes, pgplot, and dcslib.
\begin{description}
\item[cesr\_utils] This is a small low level library that primarily defines 
the precision that \bmad\ works at (see below) and defines the physical
and mathematical constants (pi, c\_light, etc.) that \bmad\ knows
about.
\item[dcslib] This library defines a set of miscellaneous helper routines. 
Routines include spline fitting, Gaussian random number generation,
etc. The library name comes from its creator.
\item[forest] This is the FPP/PTC 
(Fully Polymorphic Package/ Polymorphic Tracking Code) library of
Etienne Forest that handles Taylor maps to any arbitrary order (this
is also known as Truncated Power Series Algebra (TPSA)). The FPP part
handles the TPSA and the PTC part does the physics of tracking through
elements using Lie Algebra with a Hamiltonian.  \bmad\ uses this
software to crate Taylor Maps, track particles, etc.  FPP/PTC is a
very general package and \bmad\ only makes use of a small part of its
features. For more information see the FPP/PTC web site at
\begin{verbatim} 
    <http://bc1.lbl.gov/CBP_pages/educational/TPSA_DA/Introduction.html>
\end{verbatim}
\item[recipes] Numerical Recipes is a set of subroutines for doing 
scientific computing including Runge--Kutta integration, FFT's,
interpolation and extrapolation, etc., etc. The writeup for this
library is the book ``Numerical Recipes, The Art of Scientific
Computing''\cite{?}. For \bmad\ this library has been modified to handle
both single and double precision reals.
\item[pgplot] The PGPLOT Graphics Subroutine Library is a Fortran or 
C-callable, device-independent graphics package for making simple
scientific graphs.  One
disadvantage of PGPLOT is that it is not the most friendly software
for the programmer. To remidy this, there is a set of Fortran90
wrapper subroutines called quick\_plot. The quick\_plot suite is part
of the dcslib library. More information may be obtained from the PGPLOT
web site at 
\begin{verbatim}
    <http://www.astro.caltech.edu/~tjp/pgplot>.
\end{verbatim}

\end{description}

\section{Precision}

\bmad\ comes in two flavors: One where the real numbers are single
precision and a version with double precision reals. Which version you
are working with is controlled by the parameter \rp\ (Real Precision)
which is defined in cesr\_utils. [Note: For compatibility with older
programs the parameter \rdef\ is defined to be equal to \rp.]  On most
machines single precision has \rp\ = 4 and double precision has \rp\ =
8. Normally the double precision version is used since round-off
errors can be significant in some calculations. Long--term tracking is
an example where the single precision version is not adequate. 

To define your variables with the correct precision use the syntax
{\it real(rp)}. For example:
\begin{verbatim}
    real(rp) var1, var2, var3
\end{verbatim}
When you want to define a literal constant, for example to pass an
argument to a subroutine, add the suffix {\it \_rp} to the end of the
constant. For example: {\it 2.0\_rp} is equivalent to {\it 2.0D0} if
\rp is defined to be double precision. Notice that this is not
equivalent to {\it 2\_rp} which defines an integer (not a real) constant.


\section{Helper programs}

The {\it listf} command is used to locate routines and structures in
the cesr\_utils, dcslib and \bmad\ libraries. The form of the command is
\begin{verbatim}
    listf <name>
\end{verbatim}
This searches for any routine or structure with the name
<name>. <name> may contain the wild--cards ``*'' and ``\%'' where
``*'' matches to any number of characters and ``\%'' matches to any
single character. For example:
\begin{verbatim}
    listf ring_struct
    listf twiss_at_%
\end{verbatim}
The second example will match to {\it twiss\_at\_s} but not {\sl
twiss\_at\_start}.

The {\it getf} command is like the {\it listf} command with the
addition that the header comments that are in the source code files
will be printed out for each routine match and the structure definition
will be printed for each structure matched. The {\it getf} command is
thus more verbose than the {\it listf} command.

%----------------------------------------------------------------
%% \chapter{The \bmad Module Hierarchy}




%----------------------------------------------------------------
\chapter{The Ele\_struct}
\label{c:ele_struct}

This chapter describes the \tn{ele_struct} which is the structure that
holds all the information about an individual element: quadrupoles,
separators, wigglers, etc. This structure is somewhat
complicated. However, in practice, a lot of the complexity is
generally hidden  by the \bmad\ bookkeeping routines.

Part of the substructure of the \tn{ele_struct} is shown
in figure~\ref{f:ele_struct} (use \vn{getf} to see the entire structure 
definition). Some of the components of the \vn{ele_struct} like \vn{%name}, 
\vn{%tracking_method}, etc.\ have an obvious correspondence with 
attributes set in the lattice file and will not be discussed further.

\begin{figure}[htb]
\centering
\small
\begin{verbatim}
  type ele_struct
    character(16) name                ! name of element
    character(16) type                ! type name
    character(16) alias               ! Another name
    type (twiss_struct)  x, y, z      ! Twiss parameters at end of element
    type (floor_position_struct) position
    real(rp) value(n_attrib_maxx)     ! attribute values
    real(rp) vec0(6)                  ! 0th order transport vector
    real(rp) mat6(6,6)                ! 1st order transport matrix
    real(rp) c_mat(2,2)               ! 2x2 C coupling matrix
    real(rp) gamma_c                  ! gamma associated with C matrix
    real(rp) s                        ! longitudinal position at the end
    real(rp), pointer :: a(:) => null()  ! skew multipole component
    real(rp), pointer :: b(:) => null()  ! normal multipole component
    type (taylor_struct) :: taylor(6) ! Taylor terms
    type (wake_struct) wake           ! Wakefields
    integer key                       ! key value
    integer sub_key                   ! For wigglers: map_type$, periodic_type$
    integer control_type              ! SUPER_SLAVE$, OVERLAY_LORD$, etc.
    integer mat6_calc_method          ! bmad_standard$, taylor$, etc.
    integer tracking_method           ! bmad_standard$, taylor$, etc.
    integer field_calc                ! Used with integrators (Runge-Kutta et. al)
    integer num_steps                 ! number of slices for DA_maps
    integer integration_ord           ! For Etiennes' PTC: 2, 4, or 6.
    logical symplectify               ! Symplectify mat6 matrices.
    logical exact_rad_int_calc        ! Exact radiation integral calculation?
    logical field_master              ! Calculate strength from the field value?
    logical is_on                     ! For turning element on/off.
  end type
\end{verbatim}
\caption{The \tn{ele\_struct}. Only part of the substructure is shown.}
\label{f:ele_struct}
\end{figure}

%--------------------------------------------------------------------------
\section{Initialization and Element Key}

The \vn{%key} integer component gives the type of element (\vn{quadrupole},
\vn{RFcavity}, etc.). In general to get the correct index for an element type
just add a ``\$'' suffix to the type name. The \vn{key_name} array convets
from integer to the appropriate string. For example
\begin{example}
  type (ele_struct) ele
  call init_ele (ele)                      ! Initialize
  ele%key = wiggler\$                       ! Turn the element into a quadruple
  print *, 'This element: ', key_name(ele%key)  ! Prints: 'WIGGLER'
\end{example}
The call to \vn{init_ele} is needed since this element is being made up
from scratch.  The elements that are part of a \vn{ring_stuct} variable
will get automatically initalized.

%--------------------------------------------------------------------------
\section{Twiss Parameters, etc.}

There are a class of components of the \vn{ele_struct} whose values 
vary along the length of the element. In such a case the value of the 
component will be the value at the exit edge of the element. The 
components are:
\begin{example}
  type (twiss_struct)  x, y, z          ! Twiss parameters
  type (floor_position_struct) position ! Floor position
  real(rp) c_mat(2,2)                   ! Coupling c matrix
  real(rp) gamma_c                      ! Coupling parameter
\end{example}
To get the Twiss parameters, floor position, etc.\ for the beginning
of the element you need to look at the preceding element in the
\vn{ring%ele_(:)} array 
\begin{example}
  ring%ele_(i-1)%x%beta  ! Beta_x at beginning of ith element.
\end{example}
To get the parameters at a position within an element you can use the
routines \vn{twiss_and_track_at_s} or \vn{twiss_and_track_partial}.

The \vn{%x}, \vn{%y}, and \vn{%z} components are themselves
\vn{twiss_struct} structures that hold the Twiss parameters for the
$a$, $b$ and $z$ modes respectively. Yes it is known that the labeling
is misleading. Unfortunately it is a bit entrenched now. The definition of
the \vn{twiss_struct} structure is
\begin{example}
  type twiss_struct
    real(rp) beta, alpha, gamma, phi, eta, etap
    real(rp) eta_lab, etap_lab   ! dispersion along the x or y axis
    real(rp) sigma
  end type 
\end{example} 
The $a$ mode is the ``nearly horizontal'' mode and the $b$ mode is the
``nearly vertical'' mode. Remember: The Twiss parameters are
associated with the normal modes. With coupling there is no Twiss
parameter associated solely with the horizontal axis.  \vn{eta} and
\vn{etap} are also the normal mode dispersion and dispersion derivative.
The true horizontal and vertical dispersions are given by \vn{eta_lab}
and \vn{etap_lab}.

%--------------------------------------------------------------------------
\section{Attribute Values: Dependent and Otherwise}
\label{s:ele_dep}

Most of the real valued attributes of an element are held in the
\vn{%value(:)} array. For example, the value of the \vn{k1} attribute
for a quadrupole element is stored in \vn{%value(k1\$)} where
\vn{k1\$} is an integer parameter. In general to get the correct index
in \vn{%value(:)} for a given attribute just add a ``\$" as a
suffix. To convert from an attribute name to its index in the
\vn{%value} array use the \vn{attribute_index} routine.  To go back
from an index in the \vn{%value} array to a name use the
\vn{attribute_name} routine
\begin{example}
  type (ele_struct) ele
  ele%key = quadrupole$  ! Set element to be a quadrupole
  print *, 'Index for Quad K1:  ', attribute_index(ele, 'K1') ! prints: `4' (= k1\$)
  print *, 'Name for Quad k1\$: ', attrbute_name (ele, k1\$)   ! prints: `K1' 
\end{example}
The list of attributes in the \vn{%value(:)} array for a given element
type is given in the writeup for the different element in
Chapter~\ref{c:elements}. The real valued attribute that are {\em not}
found in the \vn{%value(:)} array are
\begin{example}
  an, bn                                  ! ab_multipole components
  knl tn                                  ! multipole components
  \{out: coef, e1, e2, e3, e4, e5, e6\}     ! taylor term for a Taylor element
  term(i) = Wig_term                      ! Wiggler term
\end{example}

\vn{attribute_bookkeeper} is the routine that makes sure the dependent
variables (See Section~\ref{s:dependent}) of an element are keep
up--to--date. \vn{attribute_bookkeeper} is called behind the scenes
when \vn{make_mat6} is called to make transfer matrices. The general
rule is that elment attributes are changed, and if you don't call
\vn{make_mat6}, \vn{ring_make_mat6}, or \vn{control_bookkeeper}
(Section~\ref{s:ring_control}) then you must call
\vn{attribute_bookkeeper}.

If you are designing a program that will let a user decide what
attribute to vary use the routine \vn{pointer_to_attribute} which will
return an error flag if the attribute to be varied should not be. For
example, trying to vary the strength of a bend by varying the \vn{rho}
attribute is an exercise in futility. An example
\begin{example}
  type (ring_struct) ring
  character(16) attrib_name, ele_name
  real(rp), pointer :: attrib_ptr
  real(rp) set_value
  logical err_flag
  integer ix_attrib, ie
  ...
  write (*, '(a)', advance = 'no') ' Name of element to vary: '
  accept '(a)', ele_name
  write (*, '(a)', advance = 'no') ' Name of attribute to vary: '
  accept '(a)', attrib_name
  write (*, '(a)', advance = 'no') ' Value to set attribute at: '
  accept *, set_value
  do ie = 1, ring%n_ele_max
    if (ring%ele_(ie)%name == ele_name) then
      call pointer_to_attribute (ring%ele_(ie), attrib_name, &
                            .false., attrib_ptr, ix_attrib, err_flag)
      if (err_flag) exit      ! Do nothing on an error
      attrib_ptr = set_value  ! Set the attribute
    endif
  enddo
\end{example}

\vn{%b_field_master} is the logical within appropriate elements that sets 
whether it is the strength or field that is the 
independent variable. See Section~\ref{s:dependent} for more details.

%--------------------------------------------------------------------------
\section{Transfer Maps}

The first order transfer map through a element is stored in \vn{vec0}
and \vn{mat6}. Thus with \vn{Linear} tracking the appropriate formula is
\begin{example}
  orbit_out = %vec0 + %mat6 * orbit_in
\end{example}
The \bmad\ routines that compute \vn{%mat6} (for example \vn{ring_make_mat6})
take a reference orbit as an argument and the resulting \vn{%mat6} matrix
is the Jacobian about the reference orbit.

%--------------------------------------------------------------------------
\section{Taylor Maps}

\vn{taylor_order} is the order of the Taylor map (see section
\ref{s:taylor_phys}). The map itself is stored in
\vn{%taylor(1:6)}. Each \vn{%taylor(i)} is a \vn{taylor_struct}
structure that defines a Taylor series. The structure is defined as
\begin{example}
  type taylor_struct
    real (rp) ref
    type (taylor_term_struct), pointer :: term(:) => null()
  end type
\end{example}
Each Taylor series is an array of \vn{taylor_term_struct} term defined as
\begin{example}
  type taylor_term_struct
    real(rp) :: coef
    integer :: exp(6)
  end type
\end{example}

To see if there is a Taylor map associated with an element you check the
association status of \vn{%taylor(1)%term}.
As an example the following finds the order of a Taylor map.
\begin{example}
  type (ele_struct) ele
  ...
  if (associated(ele%taylor(1)%term) then  ! Taylor map exists
    taylor_order = 0
    do i = 1, 6
      do j = 1, size(ele%taylor(i)%term)
        taylor_order = max(taylor_order, sum(ele%taylor(i)%term(j)%exp)
      enddo
    enddo
  else  ! Taylor map does not exist
    taylor_order = -1  ! flag non-existance
  endif
\end{example}

The Taylor map is made up around some reference phase space point
corresponding to the coordinates at the enntrance of the element.
This reference point is saved in \vn{%taylor(1:6)%ref}.  Once a Taylor map is
made the reference point is not needed in subsequent
calculations. However, The Taylor map itself will depend upon what
reference point is chosen.


%--------------------------------------------------------------------------
\section {Multipoles}

The multipole components of an element (See setion~\ref{s:fields}) are
stored in the pointers \vn{%a(:)} and \vn{%b(:)}. If \vn{%a} and
\vn{%b} are allocated they always have a range \vn{%a(0:n_pole_maxx)}
and \vn{%b(0:n_pole_maxx)}. Currently \vn{n_pole_maxx} = 20. If the
element is a \vn{multipole} then \vn{%a(n)} is taken to be the
integrated multipole strength \vn{KnL}, and \vn{%b(n)} is taken to be
the tilt \vn{Tn}. Routines for manipulating multipoles can be found in
Section~\ref{r:multi}.

%--------------------------------------------------------------------------
\section{General Use Components}

There are three components of an \vn{ele_struct} that are gauranteed
to never be used by any \bmad\ routine and so are available for use by
someone writing a program. These components are
\begin{example}
   real(rp), pointer :: r(:) => null()  ! For general use. Not used by \bmad. 
   integer ix_pointer                   ! For general use. Not used by \bmad.
   logical logic                        ! For general use. Not used by \bmad.
\end{example}

%--------------------------------------------------------------------------
\section{Initialization and Pointers}

Generally most \tn{ele_struct} elements are part of a
\tn{ring_struct} variable so you generally don't have to worry about
allocation/deallocation issues directly. In case you do have a local
\tn{ele_struct} variable within a subroutine then you either have to
deallocate the pointers within it with a call to
\rn{deallocate_ele_pointers} or you use the save attribute.
\begin{example}
  type (ele_struct), save :: ele     ! Either this at the beginning ...
  ...
  call deallocate_ele_pointers (ele) ! ... Or do this at the end.
\end{example}

\noindent
The equal sign in the assignment
\begin{example}
  ele1 = ele2
\end{example}
is overloaded by the routine \vn{ele_equal_ele} to ensure that the
pointers of \vn{ele1} do not point to the same memory locations as the
pointers of \vn{ele2}.


\chapter{The ring\_struct}
\label{c:ring_struct}

The \tn{ring_struct} is the structure that holds of all the information 
about a lattice.   Despite its name, \bmad\
makes no assumption about whether an \tn{ring_struct} is circular as
with a storage ring or open as with a LINAC.

%----------------------------------------------------------------------------
\section{Elements Within the ring\_struct}

\begin{figure}[htb]
\centering
\begin{verbatim}
type ring_struct
  type (mode_info_struct)  x, y, z  ! tunes, etc.
  character*16 name            ! Name in USE statement
  character*40 lattice         ! Lattice name
  character*80 input_file_name ! Lattice input file name
  character*80 title           ! From TITLE statement
  type (param_struct) param    ! parameters
  integer version              ! Version number for digested files
  integer n_ele_ring           ! number of physical ring elements
  integer n_ele_use            ! number of elements used
  integer n_ele_max            ! Index of last element used
  integer n_ele_maxx           ! Index of last element allocated
  integer n_control_array      ! last index used in CONTROL_ array
  integer n_ic_array           ! last index used in IC_ array
  integer input_taylor_order   ! As set in the input file
  integer ic_(n_control_maxx)  ! index to %control_(:)
  type (ele_struct), pointer :: ele_(:)    ! Array of ring elements
  type (ele_struct)  ele_init              ! For use by any program
  type (control_struct)  control_(n_control_maxx)  ! control list
end type
\end{verbatim}
\caption{Definition of the \tn{ring\_struct}.}
\label{f:ring_struct}
\end{figure}

The definition of the \tn{ring_struct} is shown in
Figure~\ref{f:ring_struct}. The array \vn{%ele(:)} holds the elements of
the lattice. This array is always allocated with zero as the lower bound.
\vn{%ele(0)} is a marker element with the name \vn{BEGINNING}.
\vn{%ele(0)%mat6} is always the unit matrix. \vn{%ele(0:)} is divided up
into two parts: A ``physical'' part (also called the ``regular'' part)
and a ``control'' part (also called the ``lord'' part). The physical
part of this array holds the elements that are tracked through. The
control part holds the overlay and group elements, and those elements
that are ``mangled'' when elements are superimposed upon other elements.
The bounds of these two parts is given in Table~\ref{tab:part_extent}.
\begin{table}[htb]
\begin{center}
\begin{tabular}{|l|l|l|}
\hline
              & \multicolumn{2}{c|} {\em index n}         \\ \hline
{\em section} & {\em min}          & {\em max}            \\ \hline
physical      & 0                  & \vn{%n_ele_ring}     \\ \hline
control       & \vn{%n_ele_ring}+1 & \vn{%n_ele_max}      \\ \hline
physical size & 0                  & \vn{%n_ele_maxx}     \\ \hline
\end{tabular} 
\caption{Bounds of the regular and control parts 
of the array \vn{\%ele(:)}.}
\end{center}
\label{tab:part_extent}
\end{table}
Note \vn{%n_ele_use} is identical to \vn{%n_ele_ring}

The \vn{%ele_init} component within the \vn{ring_struct} is not used
by \bmad\ and is available for general program use.

%----------------------------------------------------------------------------
\section{Param\_struct Structure}

The \vn{%param} variable within the \vn{ring_struct} is a
\vn{param_struct} structure whose definition is shown in
Figure~\ref{f:param_struct}
\begin{figure}[htb]
\centering
\begin{verbatim}
  type param_struct
    real(rp) beam_energy        ! beam energy in eV
    real(rp) n_part             ! Particles/bunch (for BeamBeam elements).
    real(rp) charge             ! MacroParticle charge (used by LCavities).
    real(rp) total_length       ! total_length of ring
    real(rp) growth_rate        ! growth rate/turn if not stable
    real(rp) t1_mat6(6,6)       ! Full 1-turn 6x6 matrix
    real(rp) t1_mat4(4,4)       ! Transverse 1-turn 4x4 matrix (RF off).
    integer particle            ! +1 = positrons, -1 = electrons
    integer iy                  ! Not currently used.
    integer ix_lost             ! If lost at what element?
    integer lattice_type        ! linac_lattice$, circular_lattice$, etc...
    integer ixx                 ! Integer for general use
    logical stable              ! is closed ring stable?
    logical aperture_limit_on   ! use apertures in tracking?
    logical lost                ! for use in tracking
  end type
\end{verbatim}
\caption{Definition of the \tn{param\_struct}.}
\label{f:param_struct}
\end{figure}

\vn{%param%n_part} is the number of particles in a bunch and is used
by \vn{BeamBeam} element to determine the strength of the beambeam
interaction. \vn{%param%charge} is the charge of a particle or
macroparticle and is used by \vn{LCavity} elements for wakefield
calculations. The two variables are independent and setting one will not
affect the other. 

\vn{%param%lost} is set by tracking routines to signal if a particle
is lost. \vn{%param%aperture_limit_on} determines is apertures are
checked to begin with. \vn{%param%ix_lost} gives the index of the
element at which a particle is lost.

\vn{%param%t1_mat6} and \vn{%param%t1_mat4} are the 1--turn transfer
matrices from the start of the lattice to the end. \vn{%param%t1_mat6}
is the full transfer matrix with RF on. \vn{%param%t1_mat4} is the
transverse transfer matrix with RF off. \vn{%param%t1_mat4} is used to
compute the Twiss parameters. When computing the Twiss parameters
\vn{%param%stable} is set according to whether the matrix is stable or
not. If the matrix is not stable the Twiss parameters cannot be
computed. If unstable, \vn{%param%growth_rate} will be set to the
growth rate per turn of the unstable mode.  \vn{%param%t1_mat6} and
\vn{%param%t1_mat4} are set by various routines. Other routines use
these matrices as input for calculations.




%----------------------------------------------------------------------------
\section{Elements Controlling Other Elements}
\label{s:ring_control}

Generally a programmer does not have to worry about how the lord
elements in the \vn{%ele(:)} array control other elements. The
bookkeeping routine \vn{control_bookkeeper} takes care of that. This
routine is automatically called when the transfer matrices for the
elements are computed by \vn{ring_make_mat6}. If \vn{ring_make_mat6}
is not called then \vn{control_bookkeeper} needs to be
called. \vn{control_bookkeeper} will call \vn{attribute_bookkeeper} so
the attribute bookkeeping will be taken care of too (see
Section~\ref{s:dependent}).

The element control information is stored in the \vn{%control_(:)} array, 
Each element of this array is a \vn{control_struct} structure 
\begin{example}
  type control_struct
    real(rp) coef                ! control coefficient
    integer ix_lord                ! index to lord element
    integer ix_slave               ! index to slave element
    integer ix_attrib              ! index of attribute controlled
  end type
\end{example}
\vn{%ix_lord} and \vn{%ix_slave} give the indices in the \vn{%ele_(:)}
array of a lord element and an element it controls. A lord element
\vn{%ele_(i_lord)} has assigned to it a block \vn{%control_(:)}
elements.  The following example prints the names and controlled
attributes of the slaves of a particular lord element. If the lord is
an \vn{overlay} or \vn{group} then \vn{%control_(:)%ix_attrib} and 
\vn{%control_(:)%coef} give the attribute index of the controlled
attriube andthe appropriate  coefficient 
\begin{example}
  type (ring_struct) ring
  ...
  ix1 = ring%ele_(i_lord)%ix1_slave  ! start of block
  ix2 = ring%ele_(i_lord)%ix2_slave  ! end of block
  print *, 'Slaves for lord: ', ring%ele_(i_lord)%name
  do i = ix1, ix2
    i_slave = ring%control_(i)%ix_slave
    if (ring%ele_(i_lord)%control_type == super_lord$) then
      print *, '  ', i, '  ', ring%ele_(i_slave)%name
    else   ! must be overlay or slave
      ixa = ring%control_(i)%ix_attrib
      attrib_name = attribute_name (ring%ele_(i_slave), ixa)
      print *, '  ', i, '  ', ring%ele_(i_slave)%name, attrib_name
    endif
  enddo
\end{example}

Going backward from a slave to its lords goes through one level of
indirection using the \vn{%ic_} array as shown in the next example
\begin{example}
  ix1 = ring%ele_(i_slave)%ic1_lord
  ix2 = ring%ele_(i_slave)%ic2_lord
  print *, 'Lords for slave: ', ring%ele_(i_slave)%name
  do i = ix1, ix2
    ix = ring%ic_(i)
    i_lord = ring%control_(ix)%ix_lord
    print *, '  ', i, '  ', ring%ele_(i_lord)%name
  enddo
\end{example}
For historical reasons, since a \vn{group} element only makes changes
in the values of the attributes it controls, \vn{group} elements are
not included in the list of lords generated by the above example.

%----------------------------------------------------------------------------
\section{Pointers}

Since the \tn{ring_struct} has pointers within it there is an extra burden on
the programmer to make sure that allocation and deallocation is done
properly. To this end the equal sign has been overloaded by the
routine \rn{ring_equal_ring} so that when one writes
\begin{example}
    ring1 = ring2
\end{example}
the pointers will be handled properly. The result will be that ring1
will hold the same information as \vn{ring2} but the pointers in
\vn{ring1} will point to different locations in physical memory so
that changes to one ring will not affect the other.

Initial allocation of the pointers in a \tn{ring_struct} variable is generally
handled by the \rn{bmad_parser} and \rn{ring_equal_ring} routines.
Once allocated,
local \tn{ring_struct} variables must have the save attribute or the
pointers within must be appropriately deallocated before leaving the
routine.
\begin{example}
  type (ring_struct), save :: ring     ! Either do this at the start or ...
  ...
  call deallocate_ring_pointers (ring) ! ... Do this at the end.
\end{example}
Using the save attribute will generally be faster but will use more
memory. Typically using the save attribute will be the best choice.

\chapter{The Coord\_struct}

  Meaning of vec elements

  Meaning of vec elements in PTC.


%----------------------------------------------------------------
\chapter{Reading and Writing Lattices}
\section{Digested files}
\section{bmad\_parser and bmad\_parser2}
%----------------------------------------------------------------
\chapter{Tracking and Transfer Maps}
\label{c:tracking}
\index{Tracking}

\bmad can do two types of tracking. One type uses a single particle
and tracks its coordinates throughout the lattice. The other type
takes a beam distribution and tracks the centroids and sigmas of
``macroparticles''. Discussion of macroparicle tracking will be
deferred until the last section of this chapter.

%----------------------------------------------------------------
\section{The coord_struct}
\index{Coord_struct}

For single particle tracking the starting point is the
\vn{coord_struct} whose definition is 
\begin{example}
  type coord_struct
    real(rp) vec(6)   ! (x, p_x, y, p_y, z, p_z)
  end type
\end{example}
The \vn{coord_struct} defines the phase
space vector of the particle at a certain longitudinal location.

To get an orbit, that is, the particle position at every element in a
lattice, you will need an array of \vn{coord_struct}s. Since the
number of elements in the lattice is not known in advance the array
must be declared to be allocatable.
\begin{example}
  type (coord_struct), allocatable :: orbit(:)
\end{example}
An example of how to do multi-turn tracking (assuming a circular lattice) is
\begin{example}
  type (lat_struct) lattice             ! lattice to track through
  type (coord_struct), allocatable :: orbit(:)
  ...
  call bmad_parser ('this_lattice', lattice)
  ...
  call reallocate_coords (orbit, lattice%n_ele_max)
  orbits(0)%vec = (/ 0.01, 0.2, 0.3, 0.4, 0.0, 0.0 /) ! initialization
  do i = 1, n_turns
    call track_all (lattice, orbit)
    orbit(0) = orbit(lattice%n_ele_track)
  end do
\end{example}
\vn{orbit(n)} holds the particle's position at the exit end of the
$n$\Th element. With \vn{lat%n_ele_max} elements in the lattice
(\sref{s:lat_struct}) the \vn{orbit(:)} array needs to be made this
large. The call to \vn{reallocate_coords} does this allocation. Since
\vn{lat%ele(0)} is essentially a marker element \vn{orbit(0)} is the
orbit at the start of the lattice.  \vn{track_all} takes \vn{orbit(0)}
and tracks through the list of lattice elements until it gets to the
last trackable element \vn{lattice%n_ele_track} (\sref{s:lat_struct}).

If you are writing a routine where the \vn{coord_struct} array is
local (not passed as an argument to the routine) then you have to
decide how to cleanup the allocated \vn{coord_struct} memory at the
end of the routine. In general you have two choices: 1) Deallocate
the array. This is the cleanest solution but it can be slow since you
have to allocate afresh each time the routine is called. 2) Use the
save attribute so that the array stays around until the next time the
routine is called 
\begin{example}
  type (coord_struct), allocatable, save :: orb(:) 
\end{example}
Saving the \vn{coord_stuct} is faster but leaves memory tied up. 

%----------------------------------------------------------------
\section{Tracking Through the Elements}

The routine \vnr{track1} is the routine that tracks through one
element in the lattice. The routine \vnr{track_all} calls \vn{track1}
in a loop over all elements to track through the entire
lattice. Alternatively the routine \vnr{track_many} can be used to
track through a selective number of elements or to track backwards
(See \sref{s:reverse_track}). The routines used for tracking
and closed orbit calculations are listed in Section \sref{r:track}.

\index{lat_param_struct!lost}
\index{lat_param_struct!ix_lost}
\index{Tracking!example}
The \vn{track_all} routine serves as a good example of how tracking
works. \vn{track_all} tracks a particle through a lattice from
beginning to end. Its code, condensed slightly, is shown in
Figure~\ref{f:track_all}.  The \vn{reallocate_coord} call (line~13) is
done in case the number of elements in the lattice has changed. The
call to \vn{track1} (line~18) tracks through one element from the exit
end of the $n-1$\St\ element to the exit end of the $n$\Th
particle. \vn{lattice%param%lost} is a logical that signals the
calling routine whether a particle has been lost.  This generally
happens when the particle's position is larger then the aperture. When
a particle is lost \vn{lattice%param%ix_lost} is used to record in
what element the loss occured.

\begin{figure}[htb]
\begin{centering}
\small
\begin{listing}{1}
  subroutine track_all (lattice, orbit)
    use bmad_struct
    use bmad_interface
    implicit none
    type (lat_struct)  lattice
    type (coord_struct), allocatable :: orbit(:)
    integer n

  ! Init

    lattice%param%lost = .false.
    if (size(orbit) < lattice%n_ele_max+1) &
                    call reallocate_coord (orbit, lattice%n_ele_max)

  ! Track through the elements and check for lost particles.

    do n = 1, lattice%n_ele_track
      call track1 (orbit(n-1), lattice%ele(n), lattice%param, orbit(n))
      if (lattice%param%lost) then
        lattice%param%ix_lost = n
        return
      endif
    enddo
  end subroutine
\end{listing}
\caption{Condensed track_all code.}
\end{centering}
\label{f:track_all}
\end{figure}

%----------------------------------------------------------------
\section{Closed Orbit}
\index{Closed orbit}

For a circular lattice the closed orbit may be calculated using
\vn{closed_orbit_calc}. By default this routine will track in the
forward direction which is acceptable unless the particle you are
trying to simulate is traveling in the reverse direction and there is
radiation damping on. In this case you must tell
\vn{closed_orbit_calc} to do backward tracking. This routine works by
iteratively converging on the closed orbit using the 1--turn matrix to
calculate the next guess. On rare occasions if the nonlinearities are
strong enough, this can fail to converge. An alternative routine is
\vn{closed_orbit_from_tracking} which tries to do things in a more
robust way but with a large speed penalty.

%----------------------------------------------------------------
\section{Apertures}
\index{Tracking!apertures}

\index{lat_param_struct!aperture_limit_on}
\index{bmad_common_struct!max_aperture_limit}
The logical \vn{lattice%param%aperture_limit_on} determines if element
apertures (See \sref{s:limit}) are used to determine if a
particle has been lost in tracking.  The default
\vn{lattice%param%aperture_limit_on} is True.  Even if this is False
there is a ``hard'' aperture limit set by
\vn{bmad_com%max_aperture_limit}. This hard limit is used to prevent
floating point overflows. The default hard aperture limit is 1000
meters. Additionally, even if a particle is within the hard limit,
some routines will mark a particle as lost if the tracking calculation
will result in an overflow.

\index{lat_param_struct!end_lost_at}
\index{lat_param_struct!lost}
\index{lat_param_struct!ix_lost}
\index{Entrance_end}
\index{Exit_end}
\vn{lattice%param%lost} is the logical to check to see if a particle has
been lost. \vn{lattice%param%ix_lost} gives the index of the element
at which a particle is lost and \vn{%param%end_lost_at} gives which
end the particle was lost at. The possible values for
\vn{lattice%param%end_lost_at} are:
\begin{example}
  entrance_end$
  exit_end$
\end{example}
When tracking forward, if a particle is lost at the exit end of an
element then the place where the orbit was outside the aperture is at
\vn{orbit(ix)} where \vn{ix} is the index of the element where the
particle is lost (given by \vn{lattice%param%ix_lost}). If the
particle is lost at the entrance end then the appropriate index is one
less (remember that \vn{orbit(i)} is the orbit at the exit end of an
element). To sort this out and to determine in what plane the particle
is lost in use the routine \vnr{lost_particle_info}.

%----------------------------------------------------------------
\section {Tracking Methods}

\index{Ele_struct!%tracking_method}
For each element the method of tracking may be set either via the
input lattice file (see \sref{s:tkm}) or directly in the
program by setting the \vn{%tracking_method} attribute of an element
\begin{verbatim}
  type (ele_struct) ele
  ...
  ele%tracking_method = boris$  ! for boris tracking
\end{verbatim}
To form the corresponding parameter to a given tracking method just
put ``\$'' after the name. For example, the \vn{bmad_standard}
tracking method is specified by the \vn{bmad_standard\$}
parameter.

\index{Ele_struct!%mat6}
It should be noted that except for \vni{Linear} tracking, none of the
\bmad tracking routines make use of the \vn{ele%mat6} transfer
matrix. The reverse, however, is not true.  The transfer matrix
routines (\vn{lattice_make_mat6}, etc.)  will do tracking.

\index{Synchrotron radiation!calculating}
\bmad simulates radiation damping and excitation by applying a kick
just before and after each element. To turn on radiation damping
and/or excitation use the \vnr{setup_radiation_tracking} routine.

%----------------------------------------------------------------
\section{Taylor Maps}
\label{s:taylor_track}
\index{Taylor Map}

A list of routines for manipulating Taylor maps is given
in~\sref{r:taylor}. The order of the Taylor maps is set in the lattice
file using the \vn{parameter} statement (\sref{s:param}). In a program
this can be overridden using the routine \vnr{set_taylor_order}. The
routine \vnr{taylor_coef} can be used to get the coefficient of any
given term.

\index{Symp_Lie_Bmad}
\index{Symp_Lie_PTC}
\index{Symp_Map}
\index{Taylor}
\index{Taylor!deallocating}
Transfer Taylor maps for an element are generated as needed when the
\vn{ele%tracking_method} or \vn{ele%mat6_calc_method} is set to
\vn{Symp_Lie_Bmad}, \vn{Symp_Lie_PTC}, \vn{Symp_Map}, or
\vn{Taylor}. Since generating a map can take an appreciable time,
\bmad follows the rule that once generated, these maps are never
regenerated unless an element attribute is changed.  To generate a
Taylor map within an element irregardless of the
\vn{ele%tracking_method} or \vn{ele%mat6_calc_method} settings use the
routine \vnr{ele_to_taylor}. This routine will kill any old Taylor map
before making any new one. To kill a Taylor map (which frees up the
memory it takes up) use the routine \vnr{kill_taylor}.

To test whether a \vn{taylor_struct} variable has an associated Taylor
map. That is, to test whether memory has been allocated for the map,
use the Fortran associated function:
\begin{example}
  type (bmad_taylor) taylor(6)
  ...
  if (associated(taylor(1)%term)) then  ! If has a map ...
    ...
\end{example}

To concatinate the Taylor maps in a set of elements the routine
\vnr{concat_taylor} can be used
\begin{example}
  type (lat_struct) lat          ! lattice
  type (taylor_struct) taylor(6)  ! taylor map
  ...
  call taylor_make_unit (taylor)  ! Make a unit map
  do i = i1+1, i2
    call concat_taylor (taylor, lat%ele(i)%taylor, taylor)
  enddo
\end{example}
The above example forms the transfer Taylor map starting at the end of
element \vn{i1} to the end of element \vn{i2}. Note: This example
assumes that all the elements have a Taylor map. The problem with
concatinating maps is that if there is a constant term in the map
``feed down'' can make the result inaccurate (\sref{s:taylor_phys}. To
get around this one can ``track'' a taylor map through an element
using symplectic integration.
\begin{example}
  type (lat_struct) lat          ! lattice
  type (taylor_struct) taylor(6)  ! taylor map
  ...
  call taylor_make_unit (taylor)  ! Make a unit map
  do i = i1+1, i2
    call call taylor_propagate1 (taylor, lat%ele(i), lat%param)
  enddo
\end{example}
\index{Ds_step}
\index{Integrator_order}
Symplectic integration is typically much slower than concatination.
The number of slices used for the integration is given by
\vn{%ele%value(n_step\$)}. The order of the integrator is given by
\vn{%ele%integrator_order}

%----------------------------------------------------------------
\section{Macroparticle Tracking}
\index{Macroparticles!tracking}
\index{Tracking!Macroparticles}
\label{s:macro_track}

\begin{figure}[!tb]
\small
\begin{listing}{1}
program macroparticle_test

  use bmad
  use macroparticle_mod

  implicit none

  type (lat_struct) lattice
  type (macro_beam_struct) beam
  type (macroparticle_struct) mp
  type (macro_init_struct) init
  type (macro_bunch_params_struct) params

  integer i, j, k, n

  namelist / beam_init / init

!

  bmad_com%use_liar_lcavity = .true.
  call bmad_parser ('test.bmad', lattice)

  open (1, file = 'test.init', status = 'old')
  init%E_0 = lattice%ele(0)%value(E_tot$)
  read (1, nml = beam_init)
  close (1)

  call init_macro_distribution (beam, init, .true.)
  call track_macro_beam (lattice, beam)

  print *
  print '(13x, a)', "x          x'    sig(1,1)    sig(1,2)    sig(2,2)"
  do i = 1, size(beam%bunch)
    do j = 1, size(beam%bunch(i)%slice)
      do k = 1, size(beam%bunch(i)%slice(j)%macro)
        mp = beam%bunch(i)%slice(j)%macro(k)
        call mp_to_angle_coords (mp, lattice%ele(n)%value(E_tot$))
        print '(3i2, 1p, 5e12.4)', i, j, k, mp%r%vec(1), mp%r%vec(2), &
                  mp%sigma(s11$), mp%sigma(s12$), mp%sigma(s22$)
      enddo
    enddo
  enddo

  call calc_macro_bunch_params (beam%bunch(1), lattice%ele(lattice%n_ele_track), &
                                params)

  print *
  print '(a, e12.4)', "Normalized y emittance at end of lattice: ", params%y%emitt

end program
\end{listing}
\caption{Example program showing macroparticle tracking.}
\label{f:macro_program}
\end{figure}

Macroparticles are particles that also have an associated size. 
See \sref{s:macro} for more details. Note: Macroparticles where implemented 
into \bmad when bunch tracking was first introduced. Macroparticles have the
theoretical advantage over particle tracking since the number of macroparticles
needed to characterize a bunch is smaller. However, transport of the macroparticle
sigma matrix adds complexity and recent development in \bmad in terms of tracking
bunches have generally focused on striaght particle tracking (\sref{s:part_track}).

The definition of the \vn{macro_struct} structure that defines a
macroparticle is shown in Figure~\ref{f:macro_struct}.  
\begin{figure}[htb]
\centering
\small
\begin{verbatim}
  type macro_struct
    type (coord_struct) r   ! Center of the macroparticle
    real(rp) sigma(21)      ! Sigma matrix.
    real(rp) :: sig_z = 0   ! longitudinal macroparticle length.
    real(rp) k_loss         ! loss factor (V/m). scratch variable for tracking.
    real(rp) charge         ! charge in a macroparticle (Coul).
    integer :: iz = 0       ! index to ordering of particles in z.
  end type
\end{verbatim}
\caption{The \vn{macro_struct}\ that defines a macroparticle.}
\label{f:macro_struct}
\end{figure}

\vn{%r} is the
coordinates of the macroparticle center. \vn{%sigma(:)} holds the
sigma matrix. Since the $6 \times 6$ sigma matrix is symmetric only 21
elements need to be stored. Symbolic constants \vn{s$ij$\$} ($i$, $j$
= $1, \ldots, 6$) are defined to map to between the linear
\vn{%sigma(:)} array and the matrix. Example
\begin{example}
  type (macro_struct) macro
  sigma_z = sqrt(macro%sigma(s55\$))
\end{example}
To convert \vn{%sigma(:)} into an actual matrix use the
\vn{mp_sigma_to_mat} routine.

\index{LIAR}
\bmad uses canonical phase space
coordinates(See \sref{s:phase_space_coords}). The LIAR program uses
``angle'' phase space coordinates
\begin{equation}
  (x, x', y, y', -z, E)
\end{equation}
The notation can be confusing here since in LIAR's notation positive
$z$ is in the opposite direction to the \bmad convention. Thus in
LIAR's notation the LIAR phase space coordinates are
\begin{equation}
  (x, x', y, y', z, E)
\end{equation}
To convert between \bmad and LIAR coordinates there are two routines
called \vn{mp_to_angle_coords} and \vn{mp_to_canonical_coords}.


Macroparticles are grouped into slices, an array of slices make up a
bunch, and a number of bunches make up a beam. The routine
\vn{init_macro_distribution} can be used to initialize a
\vn{macro_beam_struct} structure. \vn{init_macro_distribution}
implements the LIAR initialization algorithm (See
\sref{s:macro}). Once a beam is initialized, tracking can be performed
using \vn{track_macro_beam}. An example program is shown in
figure~\ref{f:macro_program}.

The equal sign in the assignments
\begin{example}
  beam1 = beam2
  bunch1 = bunch2
  slice1 = slice2
\end{example}
is overloaded by the routines \vn{beam_equal_beam}, \vn{bunch_equal_bunch} and
\vn{slice_equal_slice} to ensure that the pointers of \vn{beam1}, \vn{bunch1}
and \vn{slice1} do not point to the same memory locations as the pointers of 
\vn{beam2}, \vn{bunch2} and \vn{slice2}.

The \vn{init} variable (line 11) holds the information to initialize
the macroparticle distribution. The values of the components in the
\vn{init} variable are set using a namelist read (line 24). An example
of an input file that can be used to set \vn{init} is shown in
figure~\ref{f:beam_init}. Line 23 sets the initial reference energy of
the beam to be the reference energy at the start of the linac so this
does not have to be set in the input file.  The
\vn{init_macro_distribution} routine (line 27) initializes the
\vn{beam} variable. The \vn{beam} variable is a container that holds
the macroparticles that make up the beam. The \vn{track_macro_beam} routine
tracks these macroparticles through to the end of the linac. If it is
desired to track only through part of the linac then \vn{track_macro_beam}
has optional arguments that allow this. If it is desired to only track through 1
element (similarly to \vn{track1}) then the subroutine \vn{track1_beam} is also
available. Lines 30 through 41 print out
some horizontal parameters from the position at the end of the linac.

The subroutine \vn{calc_macro_bunch_params} will find the bunch
centroid, emittances, sigmas, twiss parameters and energy-Z
correlation.

\begin{figure}
\begin{listing}{1}
  &beam_init
    init%x%beta      = 4
    init%x%alpha     = 0.16
    init%x%norm_emit = 4.d-6  ! normalized emittance
    init%y%beta      = 16
    init%y%alpha     = 4
    init%y%norm_emit = 1.0d-8 ! normalized emittance
    init%center      = -5e-3, 1.0e-4, 1.0e-3, 0e-6, 0.0, 0.0
    init%sig_e       = 10e-3
    init%sig_z       = 10e-6
    init%sig_e_cut   = 3
    init%sig_z_cut   = 3
    init%n_bunch     = 1
    init%n_slice     = 2
    init%n_macro     = 1
    init%n_part      = 1e12
  /
\end{listing}
\caption{Example beam initialization file.}
\label{f:beam_init}
\end{figure}

\break

%----------------------------------------------------------------
\section{Reverse Tracking}
\label{s:reverse_track}
\index{Tracking!reverse}

There are two ways to do reverse tracking in which the particle goes
in the direction of decreasing \vn{s}. The first way is to use the
\vnr{track_many} routine. See the \vn{track_many} routine for more
details. The advantage of using \vn{track_many} is that it is
simple. The disadvantage is that it can slow things down some since
each element goes through a reversal process every time it is tracked
through. If a program is doing a lot of tracking the other option
is to form a reversed lattice with the elements in the reverse order
and track through that. The routine \vnr{lat_reverse} will do
this. One must be somewhat careful since the reversed lattice uses a
reversed coordinate system. The transformation between the reversed
and unreversed lattices is
\Begineq
  (x, p_x, y, p_y, z, p_z) -> (x, -p_x, y, -p_y, -z, p_z)
\Endeq
See the \vn{lat_reverse} routine for more details.

Generally tracking backwards is simply the reverse of tracking
forwards (time reversal symmetry). That is, if you start at some
place, track forward for some distance and then track back to the
starting place the ending orbit will be equal to the starting
orbit. However, it should always be kept in mind that radiation
damping or excitation breaks this symmetry.

%----------------------------------------------------------------
\section{Particle Distribution Tracking}
\label{s:part_track}
\index{Tracking!particle distributions}

Initializing a distribution of particles to conform to some initial set of
Twiss parameters and emittances can be done using the routine
\vnr{init_beam_distribution}. Tracking is then performed similarily to
macroparticle trakcing.

%----------------------------------------------------------------
\section{Spin Tracking}
\label{s:spin_track}
\index{Tracking!spin}

Spin tracking has been implemented for \vn{bmad_standard}, \vn{boris} and
\vn{adaptive_boris} tracking methods. To turn spin tracking on use the
\vn{bmad_com%spin_tracking_on} flag. Then, after properly initializing the spin
in the \vn{coord_struct}, calls to \vn{track1} will track both the
particle orbit and the spin.


%----------------------------------------------------------------
\chapter{Transfer Matrices}
\chapter{Twiss Parameters}
\chapter{Interface to FPP/PTC}
\chapter{CESR Centric Routines}
%----------------------------------------------------------------
\chapter{Routines Sorted by Functionality}

include dcslib and cesr\_utils routines

%----------------------------------------------------------------
%----------------------------------------------------------------
\part{Physics Notes}

%----------------------------------------------------------------
\chapter{Emittances and Synchrotron Radiation}



%----------------------------------------------------------------
%----------------------------------------------------------------
\begin{theindex}


\end{theindex}

\end{document}