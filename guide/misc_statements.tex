\chapter{Miscellaneous Statements}

This chapter deals with the statements not covered in the previous chapters.

%-----------------------------------------------------------------------------
\section{Parameter Statement}
\label{s:param}
\index{Parameter statement|textbf}


\index{Lattice!parameter statement}
\index{Lattice_type!parameter statement}
\index{Taylor_order!parameter statement}
\index{Beam_energy!parameter statement}
\index{Ran_seed!parameter statement}
\index{N_part!parameter statement}
The \vn{parameter} statement is used to set the \vn{lattice} name and
other variables.  The variables that can be set by \vn{parameter} is
\begin{example}
  parameter[lattice]      = <String>      ! Lattice name 
  parameter[lattice_type] = <Switch>      ! 
  parameter[taylor_order] = <Integer>     ! Default: 3
  parameter[beam_energy]  = <Real>        ! Total Energy. Default: 0
  parameter[n_part]       = <Real>        ! Number of particles in a bunch.
  parameter[ran_seed]     = <Integer>     ! Random number generator init.
\end{example}

\noindent
Examples
\begin{example}
  parameter[lattice]      = "L9A19C501.FD93S_4S_15KG"
  parameter[lattice_type] = circular_lattice
  parameter[taylor_order] = 5
  parameter[beam_energy]  = 5.6e9    ! eV
\end{example}

For more information on \vn{parameter[ran_seed]} see \sref{s:functions}.

The \vn{parameter[beam_energy]} is the reference energy at the start of the
lattice.  Each element in a lattice has an individual reference
\vn{beam_energy} which is a dependent parameter. 
The reference energy will only change between \vn{LCavity} or
\vn{Patch} elements. The starting reference energy must always be set.

\index{BeamBeam}
The \vn{parameter[n_part]} is the number of particle in a bunch.
it is used with \vn{BeamBeam} elements and is used to calculate the
change in energy through an \vn{Lcavity}. See~\sref{s:lcav} for more
details.

\index{Lattice statement}
The \vn{lattice} name is stored by \bmad for use by a program but it does
not otherwise effect any \bmad routines. 
Historically it is possible to set the lattice name using the syntax
\begin{example}
  lattice = <String>   ! DO NOT USE THIS SYNTAX
\end{example}
This syntax is obsolete since a typographical error cannot be caught.

\noindent
\index{Circular_lattice}
\index{Linear_lattice}
\index{LCavity!and lattice_type}
Valid \vn{lattice_type} switches are
\begin{example}
  circular_lattice  ! Default w/o LCavity element present.
  linear_lattice    ! Default if LCavity elements present.
\end{example}
a \vn{circular_lattice} is for a closed ring where one expects a
periodic solution for the Twiss parameters. A \vn{linear_lattice} is
not closed so that the initial Twiss parameters need to be given, not
computed. Although \vn{circular_lattice} is the nominal default, If
there are \vn{Lcavity} elements present, \vn{linear_lattice} will be used
as the default.

\index{Taylor_order statement}
\index{Taylor_order!parameter statement}
The Taylor order (\sref{s:taylor_phys}) is set by
\vn{parameter[taylor_order]} and is the maximum order for a Taylor map.
Historically it is possible to set the Taylor order using the syntax
\begin{example}
  taylor_order = <Integer>   ! DO NOT USE THIS SYNTAX
\end{example}
This syntax is obsolete since a typographical error is not easily caught.

%-----------------------------------------------------------------------------
\section{Bunch\_start Statement}
\label{s:bunch_start}
\index{Bunch_start statement|textbf}


\index{Bunch_start!x, p_x, y, p_y, z, p_z}
The \vn{bunch_start} statement is used to set the starting coordinates
for particle tracking
\begin{example}
  bunch_start[x]    = <Real> ! Horizontal position.
  bunch_start[p_x]  = <Real> ! Horizontal momentum.
  bunch_start[y]    = <Real> ! Vertical position.
  bunch_start[p_y]  = <Real> ! Vertical momentum.
  bunch_start[z]    = <Real> ! Longitudinal position.
  bunch_start[p_z]  = <Real> ! Longitudinal momentum (energy deviation).
\end{example}

\noindent
Examples
\begin{example}
  bunch_start[y] = 2 * bunch_start[x]
\end{example}

%-----------------------------------------------------------------------------
\section{Beam Statement}
\index{Beam statement|textbf}

\index{Energy!beam statement}
\index{Particle!beam statement}
\index{N_part!beam statement}
\index{MAD}
The \vn{Beam} statement is provided for compatibility with \mad. The syntax is
\begin{example}
  beam, energy = GeV, particle = <Switch>, n_part = <Real>
\end{example}
For example
\index{MAD}
\begin{example}
  beam, energy = 5.6  ! Note: Gev to be compatible with \mad
  beam, particle = electron, n_part = 1.6e10
\end{example}
Setting the reference energy using the \vn{energy} attribute is the same
as using \vn{parameter[beam_energy]}. Valid \vn{particle}
switches are
\index{positron}\index{electron}\index{proton}\index{antiproton}
\begin{example}
  positron  ! default
  electron
  proton
  antiproton
\end{example}

%-----------------------------------------------------------------------------
\section{Title Statement}
\index{Title statement|textbf}

The \vn{title} statement sets a title string which can be used by a program. 
For consistency with \mad there are two possible syntaxes
\begin{example}
  title, <String>
\end{example}
or the statement can be split into two lines
\begin{example}
  title
  <String>
\end{example}
For example
\begin{example}
  title
  "This is a title"
\end{example}

%--------------------------------------------------------------------------
\section{Call Statement}
\index{Call statement|textbf}

It is frequently convenient to separate the lattice definition into
several files.  Typically there might be a file (or files) that define
the layout of the lattice (something that doesn't change often) and a
file (or files) that define magnet strengths (something that changes
more often).  The \vn{call} is used to read in separated lattice
files. The syntax is
\begin{example}
  call, filename = <String>
\end{example}
Example:
\begin{example}
  call, filename = "../layout/my_layout.bmad"
\end{example}
\bmad will read the called file until a \vn{return} or \vn{end_file}
statement is encountered or the end of the file is reached.

The called file will be searched for in multiple directories.
The search path for files without a directory specification in their name is:
\begin{example}
	1) Current directory
	2) directory of the calling file
	3) BMAD_LAYOUT (system logical)
\end{example}
The first instance where a file is found is used.
The search path for files with a directory specification in their name
is relative to the above list. Thus, in the above example, the search
would be
\begin{example}
  1) ../layout/my_layout.bmad  (relative to the current directory)
  2) ../layout/my_layout.bmad  (relative to the calling file directory)
  3) \$BMAD_LAYOUT:../layout/my_layout.bmad 
\end{example}

%--------------------------------------------------------------------------
\section{Return and End\_File statements}
\index{Return statement}
\index{End_file statement}

\vn{Return} and \vn{end_file} have identical effect and tell \bmad to
ignore anything beyond the \vn{return} or \vn{end_file} statement in
the file.

%--------------------------------------------------------------------------
\section{Beginning Statement}
\label{s:beginning}
\index{Beginning statement|textbf}

\index{Beta_x!initialization}
\index{Alpha_x!initialization}
\index{Phi_x!initialization}
\index{Eta_x!initialization}
\index{Etap_x!initialization}
\index{Beta_y!initialization}
\index{Alpha_y!initialization}
\index{Phi_y!initialization}
\index{Eta_y!initialization}
\index{Etap_y!initialization}
\index{Cij!initialization}
\index{Beam_energy!initialization}
For non--circular lattices the \vn{beginning} statement can be used to
set the Twiss parameters and beam energy at the beginning of the ring
\begin{example}
  beginning[beta_x]  = <Real>  ! "a" mode beta
  beginning[alpha_x] = <Real>  ! "a" mode alpha
  beginning[phi_x]   = <Real>  ! "a" mode phase
  beginning[eta_x]   = <Real>  ! "a" mode dispersion
  beginning[etap_x]  = <Real>  ! "a" mode dispersion derivative.
  beginning[beta_y]  = <Real>  ! "b" mode beta
  beginning[alpha_y] = <Real>  ! "b" mode alpha
  beginning[phi_y]   = <Real>  ! "b" mode phase
  beginning[eta_y]   = <Real>  ! "b" mode dispersion
  beginning[etap_y]  = <Real>  ! "b" mode dispersion derivative.
  beginning[cij]     = <Real>  ! C coupling matrix. i, j = {``1'', or ``2''} 
  beginning[s]       = <Real>  ! Longitudinal starting position.
  beginning[beam_energy] = <Real>  ! Total energy in eV.
\end{example}
\index{Beam_energy!initialization}
The \vn{gamma_x}, \vn{gamma_y}, and \vn{gamma_c} (the coupling gamma
factor) will be kept consistent with the values set. If not set the
default values are all zero.  \vn{beginning[energy]} and
\vn{parameter[beam_energy]} are equivalent and one or the other may be
set but not both.

\index{X_position!initialization}
\index{Y_position!initialization}
\index{Z_position!initialization}
\index{Theta_position!initialization}
\index{Phi_position!initialization}
\index{Psi_position!initialization}
For any lattice the \vn{beginning} statement can be used to set the starting floor position 
(see~\ref{s:global}). The syntax is
\begin{example}
  beginning[x_position]     = <Real>  ! X position
  beginning[y_position]     = <Real>  ! Y position
  beginning[z_position]     = <Real>  ! Z position
  beginning[theta_position] = <Real>  ! Angle on floor
  beginning[phi_position]   = <Real>  ! Angle of attack
  beginning[psi_position]   = <Real>  ! Roll angle
\end{example}

%----------------------------------------------------------------------------
\section{Expand\_lattice Statement}
\label{s:lat_expand}
\index{Lattice!expansion|textbf}
\index{Expand_lattice statement}

At some point in parsing a lattice file, the ordered sequence of
elements that form a lattice must be constructed. This process is
called \vn{lattice expansion} since the element sequence can be built
up from sub--sequences. Normally lattice expansion happens
automatically at the end of parsing the lattice file but an explicit
\vn{expand_lattice} statement will cause immediate expansion. The
reason why this can be important is that there are restrictions, on
some types of operations which must come either before or after
lattice expansion. 
\begin{Itemize}
\item 
\index{Intrinsic functions!ran}
\index{Intrinsic functions!ran_gauss}
The \vn{ran} and \rn{ran_gauss} functions, when used with elements
that show up multiple times in a lattice, generally need to be used
after lattice expansion. See \sref{s:functions}.
\item 
Some dependent variables may be set as if they are independent
variables but only if done before lattice expansion. See
\sref{s:depend}.
\index{Multipass}
\item 
Setting \vn{dphi0} for an \vn{Lcavity} or \vn{RFcavity} multipass
slave may only be done after lattice expansion.
\end{Itemize}

%--------------------------------------------------------------------------
\section{Debugging Statements}
\index{Parser_debug statement}
\index{No_digested statement}
\index{Lattice files!parser debugging}

There are two statements, \vn{parser_debug} and \vn{no_digested},
which can help in debugging the \bmad lattice parser
itself.  That is, these statements are generally only used by programmers.

The \vn{no_digested} statement if present, will prevent \bmad from 
creating a digested file. 

The \vn{parser_debug} statement will cause information about the
lattice to be printed out at the terminal. It is recommended that this
statement be used with small test lattices since it can generate a lot
of output. The syntax is
\begin{example}
  parser_debug <switches>
\end{example}
Valid \vn{<switches>} are
\begin{example}
  beam_start          ! Print the beam_start information.
  ele <n1> <n2> ...   ! Print full info on selected elements.
  lattice             ! Print a list of lattice element information.
  lord                ! Print full information on all lord elements.
  seq                 ! Print sequence information.
  slave               ! Print full information on all regular elements.
  var                 ! Print variable information.
\end{example}
Here $<n1>$, $<n2>$, etc. are the index of the selected elements in
the lattice.  Example
\begin{example}
  parser\_debug var ring ele 34 78
\end{example}




