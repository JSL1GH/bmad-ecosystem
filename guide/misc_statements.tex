\chapter{Miscellaneous Statements}

This chapter deals with the statements not covered in the previous chapters.

%-----------------------------------------------------------------------------
\section{Parameter Statement}
\label{s:parameter}

The \vn{parameter} statement is used to set the \vn{lattice} name and other variables. 
The variables that can be set by \vn{parameter} is
\begin{example}
  parameter[lattice]      = String 
  parameter[lattice_type] = Switch      ! default: circular_lattice
  parameter[taylor_order] = Integer     ! default: 3
  parameter[beam_energy]  = Real        ! default: 0
\end{example}

\noindent
Examples
\begin{example}
  parameter[lattice]      = "L9A19C501.FD93S_4S_15KG"
  parameter[lattice_type] = circular_lattice
  parameter[taylor_order] = 5
  parameter[beam_energy]  = 5.6e9    ! eV
\end{example}

%-----------------------------------------------------------------------------
\section{Lattice Name and taylor\_order}
\label{s:beam}

The \vn{lattice} name is stored by \bmad\ for use by a program but it does
not otherwise effect any \bmad\ routines. 
Historically it is possible to set the lattice name using the syntax
\begin{example}
  lattice = String   ! DO NOT USE THIS SYNTAX
\end{example}
This syntax is obsolete since a typographical error is not easily caught.

\noindent
Valid \vn{lattice_type} switches are
\begin{example}
  circular_lattice  ! default
  linear_lattice
  linac_lattice
\end{example}
a \vn{circular_lattice} is for a closed ring where one expects a periodic
solution for the Twiss parameters. A \vn{linear_lattice} is not closed
so that the initial Twiss parameters need to be given, not computed. A
\vn{linac_lattice} is a linear lattice with \vn{Lcavity} elements.

The Taylor order (cf.~\ref{s:taylor_phys}) is the maximum order for a
Taylor map.  Historically it is possible to set the Taylor order using
the syntax
\begin{example}
  taylor_order = Integer   ! DO NOT USE THIS SYNTAX
\end{example}
This syntax is obsolete since a typographical error is not easily caught.

The \vn{beam_energy} is the reference energy.  Each element in a lattice 
has an individual reference energy. For a circular or linear lattice all 
the reference energies are the same. For a linac lattice
the reference energy changes between RF cavities.


%-----------------------------------------------------------------------------
\section{Beam Statement}

The \vn{Beam} statement is provided for compatibility with \mad. The syntax is
\begin{example}
  beam, energy = GeV, particle = Switch, n_part = Real
\end{example}
For example
\begin{example}
  beam, energy = 5.6  ! Note: Gev to be compatible with \mad
  beam, particle = electron, n_part = 1.6e10
\end{example}
Setting the reference energy using the \vn{energy} attribute is the same
as using \vn{parameter [beam_energy]}. Valid \vn{particle}
switches are
\begin{example}
  positron  ! default
  electron
\end{example}

%-----------------------------------------------------------------------------
\section{Title Statement}
The \vn{title} statement sets a title string which can be used by a program. 
For consistency there are two possible syntaxes
\begin{example}
  title, String
\end{example}
or
\begin{example}
  title
  String
\end{example}
For example
\begin{example}
  title
  "This is a title"
\end{example}

%-----------------------------------------------------------------------------------
\section{Call Statement}

It is typically convenient to separate the lattice definition into several files.
Typically there might be a file (or files) that define the layout of the lattice
(something that doesn't change often) and a file (or files) that define magnet strengths
(something that changes more often).
The \vn{call} is used to read in separated lattice files. The syntax is
\begin{example}
  call, filename = String
\end{example}
Example
\begin{example}
  call, filename = "../layout/my_layout.bmad"
\end{example}
\bmad\ will read the called file until a \vn{return} or \vn{end_file} statement is encountered
or the end of the file is reached.

%-----------------------------------------------------------------------------------
\section{Return and End\_File statements}

\vn{Return} and \vn{end_file} have identical effect and tell \bmad\ to ignore anything
beyond the \vn{return} or \vn{end_file} statement in the file.

%-----------------------------------------------------------------------------------
\section{Beginning Statement}

For non--circular lattices the \vn{beginning} statement can be used to set the Twiss parameters 
and beam energy at the beginning of the ring 
\begin{example}
  beginning[beta_x]  = Real  ! "a" mode beta
  beginning[alpha_x] = Real  ! "a" mode alpha
  beginning[phi_x]   = Real  ! "a" mode phase
  beginning[eta_x]   = Real  ! "a" mode dispersion
  beginning[etap_x]  = Real  ! "a" mode dispersion derivative.
  beginning[beta_y]  = Real  ! "b" mode beta
  beginning[alpha_y] = Real  ! "b" mode alpha
  beginning[phi_y]   = Real  ! "b" mode phase
  beginning[eta_y]   = Real  ! "b" mode dispersion
  beginning[etap_y]  = Real  ! "b" mode dispersion derivative.
  beginning[cij]     = Real  ! C coupling matrix. i, j = {``1'', or ``2''} 
  beginning[energy]  = Real  ! Total energy in eV.
\end{example}
The \vn{gamma_x}, \vn{gamma_y}, and \vn{gamma_c} (the coupling gamma factor) 
will be kept consistent with the values set. If not set the default values are all zero. 

For any lattice the \vn{beginning} statement can be used to set the starting floor position 
(see~\ref{s:global}). The syntax is
\begin{example}
  beginning[x_position]     = Real  ! X position
  beginning[y_position]     = Real  ! Y position
  beginning[z_position]     = Real  ! Z position
  beginning[theta_position] = Real  ! Angle on floor
  beginning[phi_position]   = Real  ! Angle of attack
\end{example}

%-----------------------------------------------------------------------------------
\section{Parser\_Debug Statement}

The \vn{parser_debug} statement is used to diagnose problems with the \bmad\ parser 
routine. That is, it is generally only used by programmers. 
It is recommended that this
statement be used with small test lattices since it can generate a lot of output.
The syntax is
\begin{example}
  parser\_debug Switches
\end{example}
Valid switches are
\begin{example}
  var     ! print variable information
  seq     ! print sequence information
  slave   ! print full information on all regular elements
  lord    ! print full information on all lord elements
  ring    ! print a list of regular element names, positions, and lengths
\end{example}
Example
\begin{example}
  parser\_debug var ring
\end{example}


%-----------------------------------------------------------------------------------
\section{No\_Digested Statement}

The \vn{no_digested} statement if present, will prevent \bmad\ from 
creating a digested file. This can be helpful for debugging purposes but
otherwise it is not generally used.


