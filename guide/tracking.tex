\chapter{Tracking}
\label{s:tm}

\bmad\ can do two types of tracking. One type uses a single particle 
and tracks its coordinates throughout the lattice. The other type
takes a beam distribution and tracks the centroids and sigmas of
``macro particles''.  

%----------------------------------------------------------------
\section{Track\_All Example}

The routines to track through a lattice are listed in
Section~\ref{r:track}. The \vn{track_all} routine serves as a good
example. \vn{track_all} tracks a particle through a lattice from
beginning to end. Its code, condensed slightly, is shown below.
\begin{example}
  subroutine track_all (ring, orbit_)
    use bmad_struct
    use bmad_interface
    implicit none
    type (ring_struct)  ring
    type (coord_struct), allocatable :: orbit_(:)
    integer n

  ! Init

    ring%param%lost = .false.
    if (size(orbit_) < ring%n_ele_max+1) &
                    call reallocate_coord (orbit_, ring%n_ele_max)

  ! Track through the elements and check for lost particles.

    do n = 1, ring%n_ele_use
      call track1 (orbit_(n-1), ring%ele_(n), ring%param, orbit_(n))
      if (ring%param%lost) then
        ring%param%ix_lost = n
        return
      endif
    enddo
  end subroutine
\end{example}
The \vn{reallocate_coord} call is done in case the number of elements
in the lattice has changed. \vn{orbit_(n)} holds the particles's
position at the exit end of the $n$\Th element. The call to
\vn{track1} tracks through one element from the exit end of the
$n-1$\St\ element to the exit end of the $n$\Th
particle. \vn{ring%param%lost} is a logical that signals the calling
routine whether a particle has been lost.  This happens when the
particle's position is larger then the aperture (See
Section~\ref{s:limit}.) or, because of a large orbit, \vn{track1}
cannot track through the element. An example of this happens in a bend
where, say, a particle with a large energy offset will have a
trajectory that doesn't intersect the exit face.  When a particle is
lost \vn{ring%param%ix_lost} is used to record in what element the
loss occured.

%----------------------------------------------------------------
\section{The coord\_struct}

For single particle tracking the starting point is the
\tn{coord_struct} whose definition is 
\begin{example}
  type coord_struct
    real(rp) vec(6)   ! (x, p_x, y, p_y, z, p_z)
  end type
\end{example}
shown
The \tn{coord_struct} defines the phase
space vector of the particle at a certain longitudinal location.

To get an orbit, that is, the particle position at every element in a
lattice, you will need an array of \tn{coord_struct}'s. Since the
number of elements in the lattice is not known in advance the array
must be declared to be allocatable.
\begin{verbatim}
  type (coord_struct), allocatable :: coords(:)
\end{verbatim}
You cannot, in general, use a pointer to an array since most \bmad\
subroutines expect an allocatable array and Fortran does not allow you
to mix and match. An example of how to do multiturn tracking is
\begin{example}
  type (ring_struct) ring             ! lattice to track through
  type (coord_struct), allocatable :: coord(:)
  ...
  call bmad_parser ('this_lattice', ring)
  ...
  call reallocate_coords (coord, ring%n_ele_max)
  coords(0)%vec = (/ 0.01, 0.2, 0.3, 0.4, 0.0, 0.0 /) ! initialization
  do i = 1, n_turns
    call track_all (ring, coord)
    coords(0) = coords(ring%n_ele_ring)
  end do
\end{example}
\rn{track_all} takes \vn{coord(0)} and tracks through the lattice from
the 1\St\ element through the $n$\Th\ where $n =$
\vn{ring%n_ele_ring}.  Each \vn{coord(i)} is associated with the
corresponding \vn{ring%ele_(i)} element and is the particle position
at the exit end of the i\Th\ element. \rn{reallocate_coords} simply
does an allocation. Its advantage over just allocating directly is
that it checks if the allocation is needed and if not then it does
nothing. Since \rn{track_all}, like many \bmad\ routines, will
reallocate if needed, only one call to \rn{reallocate_coords} is
needed initially even if there are repeated calls to \rn{track_all}.

If you are writing a routine where the \tn{coord_struct} array is
local (not passed as an argument to the routine) then you have to
decide how to cleanup the allocated \tn{coord_struct} memory at the
end of the routine. In general you have two choices: 1) Deallocate
the array. This is the cleanest solution but it can be slow since you
have to allocate afresh each time the routine is called. 2) Use the
save attribute so that the array stays around until the next time the
routine is called 
\begin{example}
  type (coord_struct), allocatable, save :: orb(:) 
\end{example}
Saving the \vn{coord_stuct} is faster but leaves memory tied up. 

%----------------------------------------------------------------
\section {Miscelaneous Notes}

Except for \vn{Linear} tracking, none of the \bmad\ tracking routines
make use of the \vn{ele%mat6} transfer matrix. The reverse, however,
is not true.  The transfer matrix routines (\vn{ring_make_mat6}, etc.)
will do tracking.

For each element the method of tracking may be set either via the
input lattice file (see section~\ref{s:tkm}) or directly in the
program by setting the \vn{%tracking_method} attribute of an element
\begin{verbatim}
  type (ele_struct) ele
  ...
  ele%tracking_method = boris$  ! for boris tracking
\end{verbatim}
To form the corresponding parameter to a given tracking method just
put ``\$'' after the name. For example, the \vn{bmad_standard}
tracking method is specified by the \vn{bmad_standard\$}
parameter.


\bmad\ simulates radiation damping and excitation by applying a kick
just before and after each element. To turn on radiation damping
and/or excitation use the \vn{setup_radiation_tracking} routine.

%----------------------------------------------------------------
\section{Macroparticle Tracking}



