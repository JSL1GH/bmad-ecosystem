\chapter{The Ele\_struct}
\section{overview}

This chapter describes the \tn{ele_struct} which is the structure that
holds all the information about an individual element: quadrupoles,
separators, wigglers, etc. Discusion on how one \tn{ele_struct} controls
another is deferred to the chapter on the \tn{ring_struct}. Also deferred
is a discussion of how to compute Twiss parameters, etc.

The definition of the \tn{ele_struct} is shown
in figure~\ref{f:ele_struct}. Substructure definitions are shown in
figure~\ref{f:subele_struct}. 

General rule: Components like the Twiss parameters, s--position, etc.\
always have values that are appropriate for the exit end of the element. To
get the Twiss parameters, etc.\ for the beginning of the element you
need to look at the preceding element in the \vn{ring%ele_(:)} array
(see the \tn{ring_struct} chapter).

\begin{figure}[tb]
\centering
\small
\begin{verbatim}
  type ele_struct
    character*16 name              ! name of element
    character*16 type              ! type name 
    character*16 alias             ! Another name
    character*16 attribute_name    ! Used by overlays
    type (twiss_struct)  x,y,z         ! Twiss parameters at end of element
    real(rp) value(n_attrib_maxx)    ! attribute values
    real(rp) gen0(6)                 ! constant part of the genfield map
    real(rp) vec0(6)                 ! 0th order transfer vector
    real(rp) mat6(6,6)               ! 1st order transfer matrix
    real(rp) c_mat(2,2)              ! 2x2 C coupling matrix
    real(rp) gamma_c                 ! gamma associated with C matrix
    real(rp) s                       ! longitudinal position at the end
    real(rp) x_position, y_position  ! Floor position of element
    real(rp) z_position              ! Elevation of element
    real(rp) theta_position          ! Floor orientation angle of element
    real(rp) phi_position            ! Angle of attack
    real(rp), pointer :: r(:) => null()  ! For general use. Not used  by BMAD.
    real(rp), pointer :: a(:) => null()              ! multipole
    real(rp), pointer :: b(:) => null()              ! multipoles
    real(rp), pointer :: const(:) => null()          ! Working constants.
    character(200), pointer :: descrip => null()     ! For general use
    type (genfield), pointer :: gen_field => null()  ! For symp_map$ $
    type (taylor_struct) :: taylor(6)                ! Taylor terms
    type (wig_term_struct), pointer :: wig_term(:) => null()   ! Wiggler Coefs
    integer key                    ! key value
    integer sub_key                ! For wigglers: map_type$, periodic_type$
    integer control_type           ! SUPER_SLAVE$, OVERLAY_LORD$, etc.
    integer ix_value               ! Pointer for attribute to control
    integer n_slave                ! Number of slaves
    integer ix1_slave              ! Start index for slave elements
    integer ix2_slave              ! Stop  index for slave elements
    integer n_lord                 ! Number of lords
    integer ic1_lord               ! Start index for lord elements
    integer ic2_lord               ! Stop  index for lord elements
    integer ix_pointer             ! For general use. Not used by BMAD.
    integer ixx                    ! Index for BMAD internal use
    integer iyy                    ! Index for BMAD internal use
    integer mat6_calc_method       ! bmad_standard$, taylor$, etc.
    integer tracking_method        ! bmad_standard$, taylor$, etc.
    integer num_steps              ! number of slices for DA_maps
    integer integration_order      ! For Etiennes' PTC: 2, 4, or 6.
    integer ptc_kind               ! For setting the ptc kind type.
    integer taylor_order           ! Order of the taylor series.
    logical symplectify            ! Symplectify mat6 matrices.
    logical mode_flip              ! Have the normal modes traded places?
    logical multipoles_on          ! For turning multipoles on/off
    logical exact_rad_int_calc     ! Exact radiation integral calculation?
    logical field_master           ! Calculate K from the B_field?
    logical is_on                  ! For turning element on/off.
    logical internal_logic         ! For BMAD internal use only.
    logical logic                  ! For general use. Not used by BMAD.
  end type
\end{verbatim}
\caption{Definition of the \tn{ele\_struct}.}
\label{f:ele_struct}
\end{figure}

\begin{figure}[tb]
\centering
\begin{verbatim}

  type twiss_struct
    real(rp) beta, alpha, gamma, phi, eta, etap
    real(rp) mobius_beta, mobius_eta   ! Mobius effective beta and eta
    real(rp) sigma
  end type

  type taylor_struct
    real (rp) ref
    type (taylor_term_struct), pointer :: term(:) => null()
  end type

  type taylor_term_struct
    real(rp) :: coef
    integer :: exp(6)
  end type

  type wig_term_struct
    real(rp) coef
    real(rp) kx, ky, kz
    real(rp) phi_z
    integer type      ! hyper_y$, hyper_xy$, or hyper_x$     $
  end type

\end{verbatim}
\caption{Substructures of the \tn{ele\_struct}.}
\label{f:subele_struct}
\end{figure}

%--------------------------------------------------------------------------
\section{Names}

The \vn{%name}, \vn{%type}, \vn{%alias}, and \vn{%descrip} strings
come directly from the input lattice file. For example an input file
with:
\begin{verbatim}
  q00w: quadrupole, type = "this", alias = "that", descrip = "a long string"
\end{verbatim}
gives an element with
\begin{verbatim}
  ele%name    = 'Q00W'
  ele%type    = 'this'
  ele%alias   = 'that'
  ele%descrip = 'a long string'
\end{verbatim}
Notice that the element name is mapped to upper case. Another
important point is that \vn{%descrip} is a pointer that does not get
allocated unless the element has a description so if you want to look
at an element's \vn{%descrip} you have to verify that it is associated
first.

%--------------------------------------------------------------------------
\section{Twiss Parameters}

\vn{x}, \vn{y}, \vn{z} components hold the Twiss parameters for the 
$a$, $b$ and $z$ modes respectively. [Yes it is known that the
labeling is misleading. Unfortunately it is a bit entrenched now.]
The Twiss parameters are for the exit end of the element.

%--------------------------------------------------------------------------
\section{Transfer Maps}

The first order transfer map through a element is stored in \vn{vec0}
and \vn{mat6}
\begin{verbatim}
  orbit_out = vec0 + mat6 * orbit_in
\end{verbatim}
See the chapter on tracking for more details.

%--------------------------------------------------------------------------
\section{Taylor Maps}

\vn{taylor_order} is the order of the Taylor Map. The map itself is stored 
in \vn{taylor(1:6)}. Each \vn{taylor(i)} is a Taylor series. \vn{taylor(1)}

%--------------------------------------------------------------------------
\section {multipoles}

%--------------------------------------------------------------------------
\section{General Use Components}

%--------------------------------------------------------------------------
\section{Floor Position}

%--------------------------------------------------------------------------
\section{Initializing}

Generally most \tn{ele_struct} variables are stored within a
\tn{ring_struct} so you don't have to worry about
allocation/deallocation issues directly. In case you do have an local
\tn{ele_struct} variable within a subroutine then you either have do
deallocate the pointers within it with a call to
\rn{deallocate_ele_pointers} or you use the save attribute.
\begin{verbatim}
  type (ele_struct), save :: ele     ! Either this or
  call deallocate_ele_pointers (ele) ! Do this at the end.
\end{verbatim}


%--------------------------------------------------------------------------
\section{Dependent and Independent Variables}

Some attrributes of an element are designated as "dependent variables"
which are dependent upon other independent variables. The dependent
and independent variables are: \hfil\break
\begin{center}
\begin{tabular}{|l|l|l|} \hline
           & {\em Dependent Variables}  & {\em Independent Variables}\\ \hline
  Rbend    & Rho, Angle, L\_Cord    & G, L                         \\ \hline
  Sbend    & Rho, Angle, L\_Cord    & G, L                         \\ \hline
  RFCavity & RF\_Wavelength         & Harmon                       \\ \hline
  BeamBeam & BBI\_Const             & Charge, Sig\_x, Sig\_y       \\ \hline
  Wiggler  & K1, Rho                & B\_max                       \\ \hline
\end{tabular}
\end{center}

When \rn{attribute_bookkeeper} routine is called (this is called by,
for example, \rn{make_mat6}) the values of the dependent variables
will be set based upon the values of the independent variables. Thus
trying to vary the strength of a bend by varying, say, the Rho
attribute is an exersize in futility. Also remember that routines are
allowed to assume that the dependent variables are consistant with the
independent variables. Thus you need to call either

\vn{b_field_master} ...

------------------------------------------------------------------------


  How to locate attributes

  Allocation/deallocation

%--------------------------------------------------------------------------
\section{Element Control}

    * Overlays
    * Superimpose
    * Groups


\endinput

\item[\%attribute\_name] This is used by overlays. See below.
\item[\%value(:)] this array holds the attribute values for the element. 
For example, the value of the k1 attribute for a quadrupole element is
stored in \vn{%value(k1)} where \vn{k1$} is an integer parameter. In
general to get the correct index for the \vn{%value(:)} for a given
attribute just use a "\$" as a suffix.
\item[\%gen0] Constant part of a \genfield.


