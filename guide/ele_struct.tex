\chapter{The Ele\_struct}
\label{c:ele_struct}

%-----------------------------------------------------------------------------
\section{overview}

This chapter describes the \tn{ele_struct} which is the structure that
holds all the information about an individual element: quadrupoles,
separators, wigglers, etc. This structure is somewhat complicated however,
a lot of the complexity is generally hidden, in practice, by the \bmad\ 
bookkeeping routines.

Part of the substructure of the \tn{ele_struct} is shown
in figure~\ref{f:ele_struct} (use \vn{getf} to see the entire structure 
definition). Some of the components of the \vn{ele_struct} like \vn{%name}, 
\vn{%tracking_method}, etc.\ have an obvious correspondance with 
attributes set in the lattice file and will not be discussed further.

\begin{figure}[tb]
\centering
\small
\begin{verbatim}
  type ele_struct
    character(16) name                ! name of element
    character(16) type                ! type name
    character(16) alias               ! Another name
    type (twiss_struct)  x, y, z      ! Twiss parameters at end of element
    type (floor_position_struct) position
    real(rp) value(n_attrib_maxx)     ! attribute values
    real(rp) vec0(6)                  ! 0th order transport vector
    real(rp) mat6(6,6)                ! 1st order transport matrix
    real(rp) c_mat(2,2)               ! 2x2 C coupling matrix
    real(rp) gamma_c                  ! gamma associated with C matrix
    real(rp) s                        ! longitudinal position at the end
    type (taylor_struct) :: taylor(6) ! Taylor terms
    type (wake_struct) wake           ! Wakefields
    integer key                       ! key value
    integer sub_key                   ! For wigglers: map_type$, periodic_type$
    integer control_type              ! SUPER_SLAVE$, OVERLAY_LORD$, etc.
    integer mat6_calc_method          ! bmad_standard$, taylor$, etc.
    integer tracking_method           ! bmad_standard$, taylor$, etc.
    integer field_calc                ! Used with integrators (Runge-Kutta et. al)
    integer num_steps                 ! number of slices for DA_maps
    integer integration_ord           ! For Etiennes' PTC: 2, 4, or 6.
    logical symplectify               ! Symplectify mat6 matrices.
    logical exact_rad_int_calc        ! Exact radiation integral calculation?
    logical field_master              ! Calculate strength from the field value?
    logical is_on                     ! For turning element on/off.
  end type
\end{verbatim}
\caption{The \tn{ele\_struct}. Only part of the substructure is shown.}
\label{f:ele_struct}
\end{figure}

%--------------------------------------------------------------------------
\section{Twiss Parameters, etc.}

There are a class of components of the \vn{ele_struct} whose values 
vary along the length of the element. In such a case the value of the 
component will be the value at the exit edge of the element. The 
components are:
\begin{example}
  type (twiss_struct)  x, y, z          ! Twiss parameters
  type (floor_position_struct) position ! Floor position
  real(rp) c_mat(2,2)                   ! Coupling c matrix
  real(rp) gamma_c                      ! Coupling parameter
\end{example}
To get the Twiss parameters, floor position, etc.\ for the beginning
of the element you need to look at the preceding element in the
\vn{ring%ele_(:)} array. To get the parameters at a position within an
element you can use the routines
\vn{twiss_and_track_at_s} if the element of part of a \vn{ring_struct} or
\vn{twiss_and_track_partial} for an individual element.

The \vn{%x}, \vn{%y}, and \vn{%z} components are themselves
\vn{twiss_struct} structures that hold the Twiss parameters for the
$a$, $b$ and $z$ modes respectively. Yes it is known that the labeling
is misleading. Unfortunately it is a bit entrenched now. The definition of
the \vn{twiss_struct} structure is
\begin{example}
  type twiss_struct
    real(rp) beta, alpha, gamma, phi, eta, etap
    real(rp) eta_lab, etap_lab   ! dispersion along the x or y axis
    real(rp) sigma
  end type 
\end{example} 
The $a$ mode is the ``nearly horizontal'' mode and the $b$ mode is the
``nearly vertical'' mode. Remember: The Twiss parameters are
associated with the normal modes. With coupling there is no Twiss
parameter associated soley with the horizontal axis.  \vn{eta} and
\vn{etap} are also the normal mode dispersion and dispersion derivative.
The true horizontal and vertical dipersions are given by \vn{eta_lab}
and \vn{etap_lab}.

%--------------------------------------------------------------------------
\section{Attribute Values: Independent and otherwise}

Most of the real valued attributes of an element are held in the 
\vn{%value(:)} array. For example, 
the value of the \vn{k1} attribute for a quadrupole element is
stored in \vn{%value(k1\$)} where \vn{k1\$} is an integer parameter. In
general to get the correct index in \vn{%value(:)} for a given
attribute just add a "\$" as a suffix. 

\vn{attribute_bookkeeper} is the
routine that makes sure the dependent variables of an element are keep
up--to--date. \vn{attribute_bookkeeper} is called behind the sceans when
\vn{make_mat6} is called to make transfer matrices. If you are designing
a program that will let a user decide what attribute to vary use the
routine \vn{pointer_to_attribute} which will return an error flag if the
attribute to be varied should not be. For example, 
trying to vary the strength of a bend by varying the \vn{rho}
attribute is an exercise in futility. Remember that in general
routines are allowed to assume that the dependent variables are consistent
with the independent variables. Thus if you do not remake that transfer matrices
after changing attribute values you need to call \vn{attribute_bookkeeper}.
\vn{getf attribute_bookkeeper} will give you a list of dependent variables.

\vn{%b_field_master} is the logical within appropriate elements that sets 
whether it is the strength or field that is the 
independent variable. The following table gives the independent variable
as a function of \vn{%b_field_master} setting
\begin{example}
                 %b_field_master setting
  Element         True           False
  ----------      ----           ----------
  SBend           g              b_field
  Solenoid        ks             b_field
  Quadrupole      k1             b_gradient
  Sextupole       k2             b_gradient
  Octupole        k3             b_gradient
\end{example}

%--------------------------------------------------------------------------
\section{Transfer Maps}

The first order transfer map through a element is stored in \vn{vec0}
and \vn{mat6}. Thus with \vn{Linear} tracking the appropriate formula is
\begin{example}
  orbit_out = %vec0 + %mat6 * orbit_in
\end{example}
The \bmad\ routines that compute \vn{%mat6} (for example \vn{ring_make_mat6})
take a reference orbit as an argument and the resulting \vn{%mat6} matrix
is the Jacobian about the reference orbit.

%--------------------------------------------------------------------------
\section{Taylor Maps}

\vn{taylor_order} is the order of the Taylor Map (see section 
\ref{s:taylor_phys}. The map itself is stored 
in \vn{%taylor(1:6)}. Each \vn{%taylor(i)} is a \vn{taylor_struct} structure
that defines a Taylor series. The structure is defined as
\begin{example}
  type taylor_struct
    real (rp) ref
    type (taylor_term_struct), pointer :: term(:) => null()
  end type
\end{example}
Each Taylor series is an array of \vn{taylor_term_struct} term defined as
\begin{example}
  type taylor_term_struct
    real(rp) :: coef
    integer :: exp(6)
  end type
\end{example}

To see if there is a Taylor map associated with an element you check the
association status of \vn{%taylor(1)%term}.
As an example the following finds the order of a Taylor map.
\begin{example}
  type (ele_struct) ele
  ...
  if (associated(ele%taylor(1)%term) then  ! Taylor Map exists
    taylor_order = 0
    do i = 1, 6
      do j = 1, size(ele%taylor(i)%term)
        taylor_order = max(taylor_order, sum(ele%taylor(i)%term(j)%exp)
      enddo
    enddo
  else  ! Taylor Map does not exist
    taylor_order = -1  ! flag non-existance
  endif
\end{example}

The Taylor map is made up around some reference phase space point
corresponding to the coordinates at the enertence of the
element. 
This reference is saved \vn{%taylor(1:6)%ref}. 
Once a Taylor map is made the reference point is not needed
in subsequent calculations. However, The Taylor map it self will
depend upon what reference point is choisen.


%--------------------------------------------------------------------------
\section {multipoles}

If an element has multipoles that are stored in 
\begin{example}
    real(rp), pointer :: a(:) => null()        ! skew component
    real(rp), pointer :: b(:) => null()        ! normal component
\end{example}
See setion~\ref{s:fields}. When a multipole is called for \vn{%a} and \vn{%b}
are always allocated to \vn{%a(0:20)} and \vn{%b(0:20)}.
If the element is a \vn{multipole} then \vn{%a(n)}
is taken to be the integrated multipole strength \vn{KnL}, 
and \vn{%b(n)} is taken to be the tilt \vn{Tn}. Routines for 
manipulating multipoles can be found in Section~\ref{r:multi}. 

%--------------------------------------------------------------------------
\section{General Use Components}

There are three components of an \vn{ele_struct} are garuenteed never be used by
any \bmad\ routine and so are available for use by someone writing a
program. These components are
\begin{example}
   real(rp), pointer :: r(:) => null()  ! For general use. Not used by \bmad. 
   integer ix_pointer                   ! For general use. Not used by \bmad.
   logical logic                        ! For general use. Not used by \bmad.
\end{example}

%--------------------------------------------------------------------------
\section{Initialization and Pointers}

Generally most \tn{ele_struct} elements are part of a
\tn{ring_struct} variable so you generally don't have to worry about
allocation/deallocation issues directly. In case you do have a local
\tn{ele_struct} variable within a subroutine then you either have do
deallocate the pointers within it with a call to
\rn{deallocate_ele_pointers} or you use the save attribute.
\begin{example}
  type (ele_struct), save :: ele     ! Either this or ...
  ...
  call deallocate_ele_pointers (ele) ! Do this at the end.
\end{example}

The equal sign in the assignment
\begin{example}
  ele1 = ele2
\end{example}
is overloaded by the routine \vn{ele_equal_ele} to automatically
make sure that the pointers of \vn{ele1} do not point to the same
memory locations as the pointers of \vn{ele2}.

%--------------------------------------------------------------------------
\section{Dependent and Independent Variables}

Some attributes of an element are designated as "dependent variables"
which are dependent upon other independent variables. The dependent
and independent variables are: \hfil\break
\begin{table}[h]
\centering {
\begin{tabular}{|l|l|l|} \hline
           & {\em Dependent Variables}  & {\em Independent Variables}\\ \hline
  Rbend    & Rho, Angle, L\_Cord    & G, L                         \\ \hline
  Sbend    & Rho, Angle, L\_Cord    & G, L                         \\ \hline
  RFCavity & RF\_Wavelength         & Harmon                       \\ \hline
  BeamBeam & BBI\_Const             & Charge, Sig\_x, Sig\_y       \\ \hline
  Wiggler  & K1, Rho                & B\_max                       \\ \hline
\end{tabular}
}
\end{table}

When \rn{attribute_bookkeeper} routine is called (this is called by,
for example, \rn{make_mat6}) the values of the dependent variables
will be set based upon the values of the independent variables. Thus
trying to vary the strength of a bend by varying, say, the \vn{rho}
attribute is an exercise in futility. Also remember that routines are
allowed to assume that the dependent variables are consistent with the
independent variables.
