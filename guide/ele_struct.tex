\chapter{The Ele\_struct}
\section{overview}

This chapter describes the \tn{ele_struct} which is the structure that
holds all the information about an individual element: quadrupoles,
separators, wigglers, etc. Discussion on how one \tn{ele_struct} controls
another is deferred to the chapter on the \tn{ring_struct}. Also deferred
is a discussion of how to compute global parameters such as the 
Twiss parameters, etc.

Part of the substructure of the \tn{ele_struct} is shown
in figure~\ref{f:ele_struct} (use \vn{getf} to see the entire structure 
definition). Some of the components of the \vn{ele_struct} like \vn{%name}, 
\vn{%tracking_method}, etc.\ have an obvious correspondance with 
attributes set in the lattice file and will not be discussed further.

\begin{figure}[tb]
\centering
\small
\begin{verbatim}
  type ele_struct
    character(16) name                ! name of element
    character(16) type                ! type name
    character(16) alias               ! Another name
    type (twiss_struct)  x,y,z        ! Twiss parameters at end of element
    type (floor_position_struct) position
    real(rp) value(n_attrib_maxx)     ! attribute values
    real(rp) vec0(6)                  ! 0th order transport vector
    real(rp) mat6(6,6)                ! 1st order transport matrix
    real(rp) c_mat(2,2)               ! 2x2 C coupling matrix
    real(rp) gamma_c                  ! gamma associated with C matrix
    real(rp) s                        ! longitudinal position at the end
    type (taylor_struct) :: taylor(6) ! Taylor terms
    type (wake_struct) wake           ! Wakefields
    integer key                       ! key value
    integer sub_key                   ! For wigglers: map_type$, periodic_type$
    integer control_type              ! SUPER_SLAVE$, OVERLAY_LORD$, etc.
    integer mat6_calc_method          ! bmad_standard$, taylor$, etc.
    integer tracking_method           ! bmad_standard$, taylor$, etc.
    integer field_calc                ! Used with integrators (Runge-Kutta et. al)
    integer num_steps                 ! number of slices for DA_maps
    integer integration_ord           ! For Etiennes' PTC: 2, 4, or 6.
    logical symplectify               ! Symplectify mat6 matrices.
    logical exact_rad_int_calc        ! Exact radiation integral calculation?
    logical field_master              ! Calculate strength from the field value?
    logical is_on                     ! For turning element on/off.
  end type
\end{verbatim}
\caption{The \tn{ele\_struct}. Only part of the substructure is shown.}
\label{f:ele_struct}
\end{figure}

%--------------------------------------------------------------------------
\section{Twiss Parameters, etc.}

There are a class of components of the \vn{ele_struct} whose values 
vary along the length of the element. In such a case the value of the 
component will be the value at the exit edge of the element. The 
components are:
\begin{example}
  %x, %y, %z      ! Twiss parameters
  %position       ! Floor position
  %c_mat(2,2)     ! Coupling c matrix
  %gamma_c        ! Coupling parameter
\end{example}
To get the Twiss parameters, etc.\ for the beginning of the element you
need to look at the preceding element in the \vn{ring%ele_(:)} array. To
get the parameters at a position within an element you can use the routines
\vn{

\vn{%x}, \vn{%y}, and \vn{%z} hold the Twiss parameters for the 
$a$, $b$ and $z$ modes respectively. [Yes it is known that the
labeling is misleading. Unfortunately it is a bit entrenched now.]
The $a$ mode is the ``nearly horizontal'' mode and the $b$ mode is the 
``nearly vertical'' mode. Remember: The Twiss parameters are associated
with the normal modes. With coupling there is no Twiss parameter associated
soley with the horizontal axis.

%--------------------------------------------------------------------------
\section{Transfer Maps}

The first order transfer map through a element is stored in \vn{vec0}
and \vn{mat6}. Thus with \vn{Linear} tracking the appropriate formula is
\begin{example}
  orbit_out = %vec0 + %mat6 * orbit_in
\end{example}
The \bmad\ routines that compute \vn{%mat6} (for example \vn{ring_make_mat6})
take a reference orbit as an argument and the resulting \vn{%mat6} matrix
is the Jacobian about the reference orbit.

%--------------------------------------------------------------------------
\section{Taylor Maps}

\vn{taylor_order} is the order of the Taylor Map. The map itself is stored 
in \vn{%taylor(1:6)}. Each \vn{%taylor(i)} is a Taylor series. 

%--------------------------------------------------------------------------
\section {multipoles}

%--------------------------------------------------------------------------
\section{General Use Components}

%--------------------------------------------------------------------------
\section{Floor Position}

%--------------------------------------------------------------------------
\section{Initializing}

Generally most \tn{ele_struct} components are stored within a
\tn{ring_struct} so you don't have to worry about
allocation/deallocation issues directly. In case you do have an local
\tn{ele_struct} variable within a subroutine then you either have do
deallocate the pointers within it with a call to
\rn{deallocate_ele_pointers} or you use the save attribute.
\begin{example}
  type (ele_struct), save :: ele     ! Either this or
  call deallocate_ele_pointers (ele) ! Do this at the end.
\end{example}


%--------------------------------------------------------------------------
\section{Dependent and Independent Variables}

Some attributes of an element are designated as "dependent variables"
which are dependent upon other independent variables. The dependent
and independent variables are: \hfil\break
\begin{table}[h]
\centering {
\begin{tabular}{|l|l|l|} \hline
           & {\em Dependent Variables}  & {\em Independent Variables}\\ \hline
  Rbend    & Rho, Angle, L\_Cord    & G, L                         \\ \hline
  Sbend    & Rho, Angle, L\_Cord    & G, L                         \\ \hline
  RFCavity & RF\_Wavelength         & Harmon                       \\ \hline
  BeamBeam & BBI\_Const             & Charge, Sig\_x, Sig\_y       \\ \hline
  Wiggler  & K1, Rho                & B\_max                       \\ \hline
\end{tabular}
}
\end{table}

When \rn{attribute_bookkeeper} routine is called (this is called by,
for example, \rn{make_mat6}) the values of the dependent variables
will be set based upon the values of the independent variables. Thus
trying to vary the strength of a bend by varying, say, the Rho
attribute is an exercise in futility. Also remember that routines are
allowed to assume that the dependent variables are consistent with the
independent variables. Thus you need to call either

\vn{b_field_master} ...

------------------------------------------------------------------------


  How to locate attributes

  Allocation/deallocation

%--------------------------------------------------------------------------
\section{Element Control}

    * Overlays
    * Superimpose
    * Groups


\endinput

\item[\%attribute\_name] This is used by overlays. See below.
\item[\%value(:)] this array holds the attribute values for the element. 
For example, the value of the k1 attribute for a quadrupole element is
stored in \vn{%value(k1)} where \vn{k1$} is an integer parameter. In
general to get the correct index for the \vn{%value(:)} for a given
attribute just use a "\$" as a suffix.
\item[\%gen0] Constant part of a \genfield.


