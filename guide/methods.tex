\chapter{Tracking and Transfer Matrix Calculation Methods}

Typically there are several ways to do tracking and transfer matrix
calculations for a given element type within \bmad. What method is used
is selected on an element--by--element basis using
an elements \vn{tracking_method} and \vn{mat6_calc_method} attributes 
(mat6 refers to the size of the 6 by 6 transfer matricies). By supplying
the appropriate routines a programmer can extend \bmad\ to do customized 
tracking.

%----------------------------------------------------------------------------
\section{Tracking\_method Switches}
\label{s:tkm}

Adaptive step size control used with the \vn{Adaptive_Boris} and
\vn{Runge_Kutta} integrators means that instead of taking fixed step sizes
the integrator chooses the proper step size so that the error in the
tracking is below the maximum alowable error set by \vn{rel_tol} and
\vn{abs_tol} tolerances. The advantage of step size control is that
the integrator is smarter about using a smaller step size when needed
but making larger stepswhen it can. The disadvantage is that a step is
more computationally intensive since the error in a step is estimated by
repeating a step using two mini steps.

\begin{description}
\item[Adaptive\_Boris]
Second order Boris integration\cite{b:boris} with adaptive step size control.
This should be nearly symplectic but slow.

\item[Boris]
Second order Boris Integration. Like \vn{Runge_Kutta} \vn{Boris} does
tracking by integrating the equation of motion. The difference is that
Boris integration is symplectic.

\item[Bmad\_Standard]
Uses formulas for tracking. The emphesis here is on speed and not
symplecticity. Appropriate when you are interested in single turn
stuff. May be appropriate for long term tracking depending upon how
many turns are tracked and what kind of elements are involved. 

\item[Custom]
This method will call a routine \vn{track1_custom} which must be
supplied by the programmer implementing the custom tracking. The
default \vn{track1_custom} supplied with the \bmad\ release will print
an error message if it is called which indicates a program linking
problem.

\item[Linear]
Linear just uses the 0th order vector with the 1st order 6x6 transfer
matrix for an element. Very simple.  Depending upon how the transfer
matrix was generated this might or might not be symplectic.

\item[Runge\_Kutta]
This uses a 4th order Runge Kutta integration algorithm with adaptive
step size control.  This is essentially ODEINT adopted from Numerical
Recipes\cite{b:nr}. This may be slow but it should be accurate. This
method is non-symplectic.

\item[Symp\_Lie\_Bmad]
Symplectic tracking using a Hamiltonian with Lie operation techniques.
This is similar to \vn{Symp_Lie_PTC} (see below) except this uses a
\bmad\ routine.  The difference between this and \vn{Symp_Lie_PTC} is
\vn{Symp_Lie_Bmad} is about a factor of 10 faster but \vn{Symp_lie_Bmad} is
currently only implemented for Wigglers.

\item[Symp\_Lie\_PTC]
Symplectic tracking using a Hamiltonian with Lie operator techniques.
This uses Etienne's PTC software for the calculation. This method is
symplectic but can be slow.

\item[Symp\_Map]
This uses an implicit (partially inverted) Taylor map. The calculation
uses Etienne's PTC software.  Since the map is implicit, a Newton
search method must be used. This will slow things down from the Taylor
method but this is guaranteed symplectic.

\item[Taylor]
This uses a Taylor map generated from Etienne's PTC
package. Generating the map may take time but once you have it it
should be very fast. One possible problem with using a Taylor map is
that you have to worry about the accuracy if you do tracking at points
that are far from the expansion point about which the map was
made. This method is non-symplectic away from the expansion point. 

\item[Wiedemann]
This is Wiedemann's hard edge model of a wiggler \cite{wiedemann}.

\end{description}

\vfill \break

\begin{table}[th]
\centering {
\begin{tabular}{|l|c|c|c|c|c|c|c|c|c|c|c|} \hline
\rule{0pt}{80pt} &
\begin{sideways}\vn{Adaptive_Boris}\end{sideways} &
\begin{sideways}\vn{Bmad_Standard}\end{sideways} &
\begin{sideways}\vn{Boris}\end{sideways} &
\begin{sideways}\vn{Custom}\end{sideways} &
\begin{sideways}\vn{Linear}\end{sideways} &
\begin{sideways}\vn{Runge_Kutta}\end{sideways} &
\begin{sideways}\vn{Symp_Lie_Bmad}\end{sideways} &
\begin{sideways}\vn{Symp_Lie_PTC}\end{sideways} &
\begin{sideways}\vn{Symp_Map}\end{sideways} &
\begin{sideways}\vn{Taylor}\end{sideways} &
\begin{sideways}\vn{Wiedemann}\end{sideways}
\\ \hline
%                               AB  BS   B   C   L  RK  SLB SLP SM   T   W
  \vn{ab_multipole}            &   & D &   & X & X &   &   & X & X & X &   \\ \hline 
  \vn{beambeam}                &   & D &   & X & X &   &   &   &   &   &   \\ \hline 
  \vn{custom}                  & X &   & X & D & X & X &   &   &   &   &   \\ \hline 
  \vn{drift}                   & X & D & X & X & X & X &   & X & X & X &   \\ \hline 
  \vn{ecollimator}             & X & D & X & X & X & X &   & X & X & X &   \\ \hline 
  \vn{elseparator}             & X & D & X & X & X & X &   & X & X & X &   \\ \hline 
  \vn{hkicker}                 & X & D & X & X & X & X &   & X & X & X &   \\ \hline 
  \vn{instrument}              & X & D & X & X & X & X &   & X & X & X &   \\ \hline 
  \vn{kicker}                  & X & D & X & X & X & X &   & X & X & X &   \\ \hline 
  \vn{lcavity}                 &   & D &   & X & X &   &   &   &   &   &   \\ \hline 
  \vn{marker}                  &   & D &   & X & X &   &   & X & X & X &   \\ \hline 
  \vn{monitor}                 & X & D & X & X & X & X &   & X & X & X &   \\ \hline 
  \vn{multipole}               &   & D &   & X & X &   &   & X & X & X &   \\ \hline 
  \vn{octupole}                & X & D & X & X & X & X &   & X & X & X &   \\ \hline
  \vn{patch}                   &   & D &   & X & X &   &   &   &   &   &   \\ \hline
  \vn{quadrupole}              & X & D & X & X & X & X &   & X & X & X &   \\ \hline
  \vn{rbend}                   & X & D & X & X & X & X &   & X & X & X &   \\ \hline
  \vn{rcollimator}             & X & D & X & X & X & X &   & X & X & X &   \\ \hline
  \vn{rfcavity}                &   & D &   & X & X &   &   & X & X & X &   \\ \hline
  \vn{sbend}                   & X & D & X & X & X & X &   & X & X & X &   \\ \hline
  \vn{sextupole}               & X & D & X & X & X & X &   & X & X & X &   \\ \hline
  \vn{solenoid}                & X & D & X & X & X & X &   & X & X & X &   \\ \hline
  \vn{sol_quad}                & X & D & X & X & X & X &   & X & X & X &   \\ \hline
  \vn{taylor}                  &   & D &   & X & X &   &   &   &   &   &   \\ \hline
  \vn{vkicker}                 & X & D & X & X & X & X &   & X & X & X &   \\ \hline
  \vn{wiggler} (periodic type) &   & D &   & X & X &   &   &   &   &   & X \\ \hline
  \vn{wiggler} (map type)      & X & D & X & X & X & X & X & X & X & X &   \\ \hline
\end{tabular}
}
\caption{Table of available \vn{tracking\_method} switches 
for a given element type. D denotes the default method. X denotes an
available method.}
\label{t:track_methods}
\end{table}

\vfill \break

%----------------------------------------------------------------------------
\section{mat6\_calc\_method Switches}
\label{s:xfer}

For methods that do not necessarily produce a symplectic matrix the
\vn{symplectify} attribute of an element can be set to \vn{True} to
solve the problem. See section~\ref{s:symp_method}. 

Symplectic integration is like ordinary integration of a function f(x)
but what is integrated here is the Hamiltonian H(y) where y here could
be a 6-dimensional vector (for tracking) or be a taylor series (for
the mat6 calculation). The order at which a Taylor series is truncated
at is set by \vn{taylor_order} (this is a global variable). Like
ordinary integration there are various formulas that one can use to do
symplectic integration. In \bmad\ (or more precisely Etienne's PTC)
you can use one of 3 methods. This is set by \vn{integration_ord}. 
\vn{integration_ord} = n (n = 2, 4, or 6)
means that the error scales as $dz^n$ where $dz$ is the integration step
size. The step size dz is set by the length of the element and the
value of \vn{num_steps}. Remember, as in ordinary integration, higher
order does not necessarily imply higher accuracy.

\begin{description}

\item[\vn{Bmad\_Standard}]
Uses formulas for the calculation. The emphesis here is on speed and not
symplecticity. Appropriate when you are interested in single turn
stuff. May be appropriate for long term tracking depending upon how
many turns are tracked and what kind of elements are involved. 

\item[\vn{Custom}]
This method will call a routine \vn{make_mat6_custom} which must be
supplied by the programmer implementing the custom transfer matrix
calculation. The default \vn{make_mat6_custom} supplied with the \bmad\
release will print an error message if it is called which indicates a
program linking problem.

\item[\vn{Symp\_Lie\_Bmad}]
A Symplectic calculation using a Hamiltonian with Lie operator techniques.
This is similar to \vn{Symp_Lie_PTC} (see below) except this uses a
\bmad\ routine.  The difference between this and \vn{Symp_Lie_PTC} is
\vn{Symp_Lie_Bmad} is about a factor of 10 faster but \vn{Symp_lie_Bmad} is
currently only implemented for Wigglers.

\item[\vn{Symp\_Lie\_PTC}]
Symplectic integration using a Hamiltonian and Lie operators.
This uses Etienne's PTC software for the calculation.
This method is symplectic but can be slow.

\item[\vn{Taylor}]
This uses a Taylor map generated from Etienne's PTC
package. Generating the map may take time but once you have it it
should be very fast. One possible problem with using a Taylor map is
that you have to worry about the accuracy if you do a calculation at points
that are far from the expansion point about which the map was
made. This method is non-symplectic away from the expansion point. 

\item[\vn{Tracking}]
This uses the tracking method set by \vn{tracking_method} to track 6
particles around the central orbit. This method is susceptible to inaccuracies
caused by nonlinearities. Furthermore this method
is almost surely slow. While non--symplectic, the advantage of this method
is that it is directly related to any tracking results.

\end{description}

\vfill \break

\begin{table}[th]
\centering {
\begin{tabular}{|l|c|c|c|c|c|} \hline
\rule{0pt}{80pt} &
\begin{sideways}\vn{Bmad_Standard}\end{sideways} &
\begin{sideways}\vn{Custom}\end{sideways} &
\begin{sideways}\vn{Symp_Lie_Bmad}\end{sideways} &
\begin{sideways}\vn{Symp_Lie_PTC}\end{sideways} &
\begin{sideways}\vn{Taylor}\end{sideways}
\\ \hline
%                               BS   C  SLB SLP  T 
  \vn{ab_multipole}            & D & X &   & X & X \\ \hline 
  \vn{beambeam}                & D & X &   &   &   \\ \hline 
  \vn{custom}                  &   & D &   &   &   \\ \hline 
  \vn{drift}                   & D & X &   & X & X \\ \hline 
  \vn{ecollimator}             & D & X &   & X & X \\ \hline 
  \vn{elseparator}             & D & X &   & X & X \\ \hline 
  \vn{hkicker}                 & D & X &   & X & X \\ \hline 
  \vn{instrument}              & D & X &   & X & X \\ \hline 
  \vn{kicker}                  & D & X &   & X & X \\ \hline 
  \vn{lcavity}                 & D & X &   &   &   \\ \hline 
  \vn{marker}                  & D & X &   & X & X \\ \hline 
  \vn{monitor}                 & D & X &   & X & X \\ \hline 
  \vn{multipole}               & D & X &   & X & X \\ \hline 
  \vn{octupole}                & D & X &   & X & X \\ \hline
  \vn{patch}                   & D & X &   &   &   \\ \hline
  \vn{quadrupole}              & D & X &   & X & X \\ \hline
  \vn{rbend}                   & D & X &   & X & X \\ \hline
  \vn{rcollimator}             & D & X &   & X & X \\ \hline
  \vn{rfcavity}                & D & X &   & X & X \\ \hline
  \vn{sbend}                   & D & X &   & X & X \\ \hline
  \vn{sextupole}               & D & X &   & X & X \\ \hline
  \vn{solenoid}                & D & X &   & X & X \\ \hline
  \vn{sol_quad}                & D & X &   & X & X \\ \hline
  \vn{taylor}                  & D & X &   &   &   \\ \hline
  \vn{vkicker}                 & D & X &   & X & X \\ \hline
  \vn{wiggler} (periodic type) & D & X &   &   &   \\ \hline
  \vn{wiggler} (map type)      & D & X & X & X & X \\ \hline
\end{tabular}
}
\caption{Table of available \vn{mat6\_calc\_method} switches for a given element type. 
D denotes the default method. X denotes an available method.}
\label{t:mat6_methods}
\end{table}

\vfill \break
