\chapter{The Bmad Distribution}
\label{c:distribution}

\section{Libraries in the CESR Distribution}
\label{s:libs}

When installing \bmad\ on a computer what one gets is not only the
\bmad\ subroutine library but additionally subsidiary libraries upon
which subroutines in the \bmad\ library depend. There are 5 other
libraries that are used: \bn{cesr\_utils}, \bn{forest},
\bn{numerical\_recipes}, \bn{pgplot}, and \bn{dcslib}.
\begin{description}
\item[cesr\_utils] This is a small low level library that primarily defines 
the precision that \bmad\ works at (see below) and defines the physical
and mathematical constants (\vn{pi}, \vn{c_light}, etc.) that \bmad\ knows
about.
\item[dcslib] This library defines a set of miscellaneous helper routines. 
Routines include spline fitting, Gaussian random number generation,
etc. The library name comes from its creator.
\item[forest] This is the FPP/PTC 
(Fully Polymorphic Package / Polymorphic Tracking Code) library of
Etienne Forest that handles Taylor maps to any arbitrary order (this
is also known as Truncated Power Series Algebra (TPSA)). See
Chapter~\ref{c:etienne} for more details.  FPP/PTC is a very general
package and \bmad\ only makes use of a small part of its features.
For more information see the FPP/PTC web site at
\begin{example} 
    <http://bc1.lbl.gov/CBP_pages/educational/TPSA_DA/Introduction.html>
\end{example}

An important point to keep in mind is that Etienne defines his phase
space coordinates as $(x, p_x, y, p_y, p_z, ct = -z)$. \bmad uses
$(x, p_x, y, p_y, z, p_z)$. There are
\bmad\ routines for converting back and forth between the two 
representations and if you never call PTC directly then you don't
have to worry about the difference.

\item[recipes] Numerical Recipes is a set of subroutines for doing 
scientific computing including Runge--Kutta integration, FFT's,
interpolation and extrapolation, etc., etc. The write up for this
library is the book ``Numerical Recipes, The Art of Scientific
Computing''\cite{b:nr}. For \bmad\ this library has been modified to handle
both single and double precision reals.
\item[pgplot] The PGPLOT Graphics Subroutine Library is a Fortran or 
C-callable, device-independent graphics package for making simple
scientific graphs.  One disadvantage of PGPLOT is that it is not the
most friendly software for the programmer. To remedy this, there is a
set of Fortran90 wrapper subroutines called \bn{quick\_plot}. The
\bn{quick\_plot} suite is part of the dcslib library. Documentation
including a user's manual
may be obtained from the PGPLOT web site at
\begin{verbatim}
    <http://www.astro.caltech.edu/~tjp/pgplot>.
\end{verbatim}

\end{description}

\section{Precision}

\bmad\ comes in two flavors: One where the real numbers are single
precision and a version with double precision reals. Which version you
are working with is controlled by the parameter \vn{rp}\ (Real Precision)
which is defined in \bn{cesr\_utils}. [Note: For compatibility with older
programs the parameter \vn{rdef} is defined to be equal to \vn{rp}.]  On most
machines single precision has \vn{rp}\ = 4 and double precision has \vn{rp}\ =
8. Normally the double precision version is used since round-off
errors can be significant in some calculations. Long--term tracking is
an example where the single precision version is not adequate. 

To define your variables with the correct precision use the syntax
{\tt real(rp)}. For example:
\begin{example}
    real(rp) var1, var2, var3
\end{example}
When you want to define a literal constant, for example to pass an
argument to a subroutine, add the suffix \vn{_rp} to the end of the
constant. For example: \vn{2.0_rp} is equivalent to \vn{2.0D0} if
\vn{rp} is defined to be double precision. Notice that this is not
equivalent to \vn{2_rp} which defines an integer (not a real) constant.


