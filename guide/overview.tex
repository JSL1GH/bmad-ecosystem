\section*{Overview}

\bmad\ (Otherwise known as "Baby MAD" or "Better MAD" or just plain "Be MAD!")
is a software subroutine library used for relativistic
charged--particle dynamics simulations in high energy accelerators and
storage rings. The \bmad\ subroutines, written in
Fortran90, have been developed to:
\begin{itemize}
\item Cut down on the time needed to develop programs,
\item Cut down on programming errors, and
\item Provide a standard input format for specifying lattices.
\end{itemize}

\bmad\ can be used to study single and multiparticle beam dynamics. Its 
features include the ability to track particles, calculate transfer matrices,
emittances, Twiss parameters, 
dispersion, coupling, etc. The elements that \bmad\ knows about include
quadrupoles, storage ring RF cavities, LINAC accelerating RF cavities, 
solendoids, dipole bends, etc. 

\bmad\ has been developed at Cornell University's Laboratory for Elementary
Particle Physics and has been in use since the 1990's. 
\vfill
\break
