\SECTION{Overview}

\bmad\ (Otherwise known as ``Baby MAD" or ``Better MAD" or just plain
``Be MAD!")  is a subroutine library used for relativistic
charged--particle dynamics simulations in high energy accelerators and
storage rings. \bmad\ has been developed at Cornell University's
Laboratory for Elementary Particle Physics and has been in use since
the 1990's. 

Prior to \bmad, programs were written almost from scratch to perform
calculations that were beyond the capability of existing, generally
available software. This practice was inefficient, leading to much
duplication of effort.  Since it was time consuming to write
simulations, needed calculations where not being done.

As a response, the \bmad\ subroutines, written in
Fortran90, were developed to:
\begin{Itemize}
\item Cut down on the time needed to develop programs.
\item Minimize computation times.
\item Cut down on programming errors, 
\item Provide a simple mechonism for lattice function calculations
from within programs (control system or otherwise).
\item Provide a flexible yet powerful lattice input format.
\item Standardize sharing of lattice information between 
programs.
\end{Itemize}

\bmad\ can be used to study single and multi--particle beam dynamics.
It has the ability to track both particles and macroparticles with
wakefields, and radiation excitation and damping. \bmad\ has routines
for calculating transfer matrices, emittances, Twiss parameters,
dispersion, coupling, etc. The elements that \bmad\ knows about
include quadrupoles, RF cavities (both storage ring and LINAC
accelerating types), solenoids, dipole bends, etc. In addition,
elements can be defined to control the attributes of other
elements. This can be used, for example, to simulate the action of
control room ``knobs'' where a control room knob may vary the
strengths of multiple magnets in the accelerator. Other possibilities
include the simulation of I--beams that determine the position of
multiple element.

\vfill
\break
