\chapter{Syntax}

\section{Lattice File Format}

Input for \bmad\ is free format. A \bmad\ lattice input file consists
of a sequence of statements. Normally a statement occupies a single
line in the file. Several statements may be placed on the same line by
inserting a semicolon (;) between them. A long statement can occupy
multiple lines by putting an ampersand (\&) at the end of each line of
the statement except for the last line of the statement. An
exclamation mark (!) denotes a comment and the exclamation mark and
everything after the exclamation mark on the line are ignored.
\begin{example}
  ! This is a comment
  q0: quad, l = 0.6, &     ! Continued to next line
  k1 = 1.4, tilt = pi/3
\end{example}

\section{Arithmetic Expressions}

Arithmetic expressions can be used in a place where a real value is required.
The standard operators are defined:
\begin{ventry}{10em}
\item[+] Addition
\item[-] Subtraction
\item[*] Multiplication
\item[/] Division
\item[$\wedge$] Exponentiation
\end{ventry}
The following functions are recognized by \bmad:
\begin{ventry}{acos(x)}
\item[sqrt(x)] Square Root 
\item[log(x)]  Logarithm
\item[exp(x)]  Exponential
\item[sin(x)]  Sine
\item[cos(x)]  Cosine
\item[tan(x)]  Tangent
\item[asin(x)] Arc sine
\item[acos(x)] Arc cosine
\item[atan(x)] Arc Tangant
\item[abs(x)]  Absolute Value
\end{ventry}


\section{Constants}

Literal constants can be entered with or without a decimal point. An
exponent is marked with the letter E. Example
\begin{example}
  1, 10.35, 5E3, 314.159E-2
\end{example}
Symbolic constants can be defined using the syntax
\begin{example}
  parameter_name = expression
\end{example}
alternatively, to be compatable with MAD, using ``:='' instead of ``='' is accepted
\begin{example}
  parameter_name := expression
\end{example}
is also accepted. Example:
\begin{example}
  my_const = sqrt(10.3) * pi^3
  abc      = my_const * 23
\end{example}
Unlike MAD, \bmad\ uses immediate substitution so that all constants
in an expression must have been previously defined. For example the
following is not valid:
\begin{example}
  abc      = my_const * 23   ! No: my_const not yet defined.
  my_const = sqrt(10.3) * pi^3
\end{example}
here the value of \vn{my_const} is not known when the line ``\vn{abc}
= $\ldots$'' is parsed and so \vn{abc} can not be calculated. Once
defined, symbolic constants cannot be redefined. For example:
\begin{example}
  my_const = 1
  my_const = 2  ! No: my_const cannot be redefined.
\end{example}


\section{Element attributes}

Element attributes can be set or used in an algerbraic expression using the syntax
\begin{example}
  element-name[attribute-name]
\end{example}
For example:
\begin{example}
  b01w: sbend, l = 6.0, rho = 89.0   ! Define a bend element.
  bo1w[roll] = 6.5                   ! Set an attribute value.
  b01w[l] = 6.5                      ! Change an attribute value.
  b01w[l] = b01w[rho] / 12           ! OK to reset an attribute value.
  my_const = b01[rho] / b01[l]       ! Use of attribute values in an expression
\end{example}

\section{Lattice Elements}

The syntax for defining a lattice element roughly follows MAD:
\begin{example}
  label: keyword [, attributes]
\end{example}
\tn{Overlay} and \tn{Group} elements have a slightly different syntax
\begin{example}
  label: keyword = \{ list \}, master-attribute [= value] [, attributes]
\end{example}
For example:
\begin{example}
  q01w: quad, type = "A String", l = 0.6, tilt = pi/2
  h10e: overlay = \{b08e, b10e\}, hkick
\end{example}
