\chapter{Syntax}

\section{Lattice File Format}

Input for \bmad\ is free format. A \bmad\ lattice input file consists
of a sequence of statements. Normally a statement occupies a single
line in the file. Several statements may be placed on the same line by
inserting a semicolon (;) between them. A long statement can occupy
multiple lines by putting an ampersand (\&) at the end of each line of
the statement except for the last line of the statement. An
exclamation mark (!) denotes a comment and the exclamation mark and
everything else after the mark on the line are ignored.

\section{Arithmetic Expressions}

Arithmetic expressions can be used in a place where a real value is required.
The standard operators are defined:
\begin{description}
\item[+] Addition
\item[-] Subtraction
\item[*] Multiplication
\item[/] Division
\item[\^] Exponentiation
\end{description}

The following functions are recognized by \bmad:
\begin{description}
\item[sqrt(x)] Square root 
\item[log(x)]  Logarithm
\item[exp(x)]  Exponential
\item[sin(x)]  Sine
\item[cos(x)]  Cosine
\item[tan(x)]  Tangent
\item[asin(x)] Arc sine
\item[acos(x)] Arc cosine
\item[atan(x)] Arc Tangant
\item[abs(x)]  Absolute value
\end{description}


\section{Constants}

Literal constants can be entered with or without a decimal point. An
exponent is marked with the letter E. Example
\begin{example}
  1, 10.35, 5E3, 314.159E-2
\end{example}
Symbolic constants can be defined using the syntax
\begin{example}
  parameter_name = expression
\end{example}
and to be compatable with MAD the syntax
\begin{example}
  parameter_name := expression
\end{example}
is also accepted. Example:
\begin{example}
  my_const = sqrt(10.3) * pi^3
  abc      = my_const * 23
\end{example}
Unlike MAD, \bmad\ uses immediate substitution so that all constants
in an expression must have been previously defined. For example the
following is not valid:
\begin{example}
  abc      = my_const * 23   ! No: my_const not yet defined.
  my_const = sqrt(10.3) * pi^3
\end{example}
here the value of \vn{my_const} is not known when the line ``\vn{abc}
= $\ldots$'' is parsed and so \vn{abc} can not be calculated. Once
defined, symbolic constants cannot be redefined. For example:
\begin{example}
  my_const = 1
  my_const = 2  ! No: my_const cannot be redefined.
\end{example}


\section{Lattice Elements}

The syntax for defining a lattice element follows MAD:
\begin{example}
\end{example}
