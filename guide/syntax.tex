\chapter{Lattice File Syntax}

%---------------------------------------------------------------------------
\section{Lattice Files and Digested Files}

The \bmad\ software library comes with a routine to read in (parse) a
lattice definition file. The syntax that this file must conform to is
modeled after the lattice input format of the \mad\ program.
Essentially, a \bmad\ input file is like a \mad\ input file except that
a \bmad\ file has no ``action'' commands (commands to calculate the
Twiss parameters, etc.). Interacting with the user to determine what
actions a program should take is left to the program and is not part of
\bmad\ (although \bmad\, of course, provides the routines to do many
calculations). A program, of course, is not required to use the \bmad\
parser routine but if it does the following chapters describes how to
construct a valid input file.

Normally the \bmad\ parser routine will create what is called a
``digested file'' after it has parsed a lattice file so that when a
program is run and the same lattice file is to be read in again, to save
time, the digested file can be used to load in the lattice information.
This digested file is in binary format and is not human readable. The
digested file will contain the transfer maps for all the elements. 
Using a digested file can save considerable time if some of the
elements in the lattice need to have Taylor maps computed.
(this occurs typically with map--type wigglers).

\bmad\ creates the digested file in the same area as the lattice file.
If \bmad\ is not able to create a digested file (typically because it
does not have write permission in the directory), an error message will
be generated but otherwise program operation will be normal.

Digested files can also be used for easy transport of lattices between
programs or between sessions of a program. For example, using one
program you might read in a lattice, make some adjustments (say to model
shifts in magnet positions) and then write out a digested version of the
lattice. This adjusted lattice can now be read in by another program to
do some custom analysis.

%---------------------------------------------------------------------------
\section{Lattice File Format}

Input for \bmad\ is free format. A \bmad\ lattice input file consists
of a sequence of statements. Normally a statement occupies a single
line in the file. Several statements may be placed on the same line by
inserting a semicolon (;) between them. A long statement can occupy
multiple lines by putting an ampersand (\&) at the end of each line of
a statement except for the last line of the statement. An
exclamation mark (!) denotes a comment and the exclamation mark and
everything after the exclamation mark on the line are ignored. Example:
\begin{example}
  ! This is a comment
  q0: quad, l = 0.6, &     ! Continued to the next line
        k1 = 1.4, tilt = pi/3
\end{example}

%---------------------------------------------------------------------------
\section{Variable Types}

There are four types of variables in \bmad: Reals, Integers, Logicals, and 
Strings. Reals are discussed below in more detail. Integers are just reals
that are evaluated to the nearest integer. Acceptable logical values are
\begin{example}
  .true.  .false.
   true    false
   t       f
\end{example}

\vskip0.1in
String literals can be quoted using double quotes (") or, if there are no
blanks or comma within a string, the quotes can be omitted. For example:
\begin{example}
  Q00W: Quad, type = "My Type", alias = Who_knows, &
                                  descrip = "Only the shadow knows"
\end{example}

%---------------------------------------------------------------------------
\section{Arithmetic Expressions}

Arithmetic expressions can be used in a place where a real value is required.
The standard operators are defined:
\begin{ventry}{10em}
\item[$a + b$] Addition
\item[$a - b$] Subtraction
\item[$a \, \ast \, b$] Multiplication
\item[$a \; / \; b$] Division
\item[$a \, \land \, b$] exponentiation
\end{ventry}
The following functions are recognized by \bmad:
\begin{ventry}{acos(x)}
\item[sqrt(x)] Square Root 
\item[log(x)]  Logarithm
\item[exp(x)]  Exponential
\item[sin(x)]  Sine
\item[cos(x)]  Cosine
\item[tan(x)]  Tangent
\item[asin(x)] Arc sine
\item[acos(x)] Arc cosine
\item[atan(x)] Arc Tangent
\item[abs(x)]  Absolute Value
\end{ventry}


%---------------------------------------------------------------------------
\section{Constants}

Literal constants can be entered with or without a decimal point. An
exponent is marked with the letter E. Example
\begin{example}
  1, 10.35, 5E3, 314.159E-2
\end{example}
Symbolic constants can be defined using the syntax
\begin{example}
  parameter_name = expression
\end{example}
Alternatively, to be compatible with \mad, using ``:='' instead of ``='' is accepted
\begin{example}
  parameter_name := expression
\end{example}
Examples:
\begin{example}
  my_const = sqrt(10.3) * pi^3
  abc      = my_const * 23
\end{example}
Unlike \mad, \bmad\ uses immediate substitution so that all constants
in an expression must have been previously defined. For example, the
following is not valid:
\begin{example}
  abc      = my_const * 23      ! No: my_const needs to be defined first.
  my_const = sqrt(10.3) * pi^3
\end{example}
here the value of \vn{my_const} is not known when the line ``\vn{abc}
= $\ldots$'' is parsed. Once
defined, symbolic constants cannot be redefined. For example:
\begin{example}
  my_const = 1
  my_const = 2  ! No: my_const cannot be redefined.
\end{example}


%---------------------------------------------------------------------------
\section{Element attributes}

Element attributes can be set or used in an algebraic expression
using the syntax
\begin{example}
  element-name[attribute-name]
\end{example}
For example:
\begin{example}
  b01w: sbend, l = 6.0, rho = 89.0   ! Define a bend element.
  bo1w[roll] = 6.5                   ! Set an attribute value.
  b01w[l] = 6.5                      ! Change an attribute value.
  b01w[l] = b01w[rho] / 12           ! OK to reset an attribute value.
  my_const = b01[rho] / b01[l]       ! Use of attribute values in an expression
\end{example}

%---------------------------------------------------------------------------
\section{Lattice Elements}

The syntax for defining a lattice element roughly follows \mad:
\begin{example}
  label: keyword [, attributes]
\end{example}
\tn{Overlay} and \tn{Group} elements have a slightly different syntax
\begin{example}
  label: keyword = \{ list \}, master-attribute [= value] [, attributes]
\end{example}
For example:
\begin{example}
  q01w: quad, type = "A String", l = 0.6, tilt = pi/2
  h10e: overlay = \{ b08e, b10e \}, hkick
\end{example}
