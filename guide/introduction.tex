\section*{Introduction}

The strength of \bmad\ is that as a subroutine library it provides a
flexible framework from which sophisticated simulation programs may
easily be developed.  The weakness of \bmad\ comes from its strength:
\bmad\ cannot be used straight out of the box. Someone must put the
pieces together into a program. As a consequence this manual serves
two masters: The programmer who wants to develop applications and
needs to know about the inner working of \bmad\ and the user who simply 
needs to know about
the \bmad\ standard input format and about the physics behind the various
calculations that \bmad\ can perform.

To this end this manual is divided into three parts. The first two
parts are for both the user and programmer while the third part is
meant just for programmers. Part~I gives the conventions used by
\bmad\ --- Coordinate systems, magnetic field expansions, etc. ---
along with some of the physics behind the calculations. Of necessity,
the physics writeup is brief and the reader is assumed to be familiar
with high energy accelerator physics formalism. Part~II discusses the
\bmad\ lattice input standard.  The \bmad\ lattice input standard was
developed using the \mad\ lattice input standard as a starting
point. \mad\ (Methodical Accelerator Design) is a widely used
stand--alone program developed at CERN by Christoph Iselin for
charged--particle optics calculations. Since it can be convenient
to do simulations with both \mad\ and \bmad, differences and
similarities between the two input formats are noted. 
Finally, Part~III gives the nitty--gritty details of the \bmad\
subroutines and the structures upon which they are based.

Errors and omissions are a fact of life for any reference work and
comments from you, dear reader, are therefore most welcome. Please
send any missives (or chocolates, or any other kind of sustanance) to:
\begin{example}
  David Sagan <dcs16@cornell.edu>
\end{example}
