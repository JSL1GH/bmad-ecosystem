\chapter{The Ring\_struct}
\section{overview}

\ringstruct is the structure that holds all the information about a lattice.
The definition of \ringstruct\ is shown in figure~\ref{f:ring_struct}.
Substructure definitions are shown in figure~\ref{f:subring_struct} (except for the
\elestruct\ definition which was defined in the previous section). Despite its
name, \bmad\ makes no assumption about whether an \ringstruct\ is circular as
with a storage ring or open as with a LINAC.

\begin{figure}[tb]
\centering
\begin{verbatim}
type ring_struct
  type (mode_info_struct)  x, y, z  ! tunes, etc.
  character*16 name            ! Name in USE statement
  character*40 lattice         ! Lattice
  character*80 input_file_name ! Lattice input file name
  character*80 title           ! general title
  type (param_struct) param    ! parameters
  integer version              ! Version number
  integer n_ele_ring           ! number of physical ring elements
  integer n_ele_symm           ! symmetry point for rings w/symmetry
  integer n_ele_use            ! number of elements used
  integer n_ele_max            ! Index of last element used
  integer n_ele_maxx           ! Index of last element allocated
  integer n_control_array      ! last index used in CONTROL_ array
  integer n_ic_array           ! last index used in IC_ array
  integer input_taylor_order   ! As set in the input file
  integer ic_(n_control_maxx)  ! index to %control_(:)
  type (ele_struct), pointer :: ele_(:)    ! Array of ring elements
  type (ele_struct)  ele_init              ! For use by any program
  type (control_struct)  control_(n_control_maxx)  ! control list
end type
\end{verbatim}
\caption{Definition of the \ringstruct.}
\label{f:ring_struct}
\end{figure}

\begin{figure}[tb]
\centering
\begin{verbatim}
type control_struct
  real(rp) coef                  ! control coefficient
  integer ix_lord                ! index to lord element
  integer ix_slave               ! index to slave element
  integer ix_attrib              ! index of attribute controlled
end type

type param_struct
  real(rp) energy          ! USE BEAM_ENERGY INSTEAD ! energy in GeV.
  real(rp) beam_energy     ! beam energy in eV
  real(rp) n_part          ! Number of particles in a bunch
  real(rp) total_length    ! total_length of ring
  real(rp) growth_rate     ! growth rate/turn if not stable
  real(rp) t1_mat6(6,6)    ! Full 1-turn 6x6 matrix
  real(rp) t1_mat4(4,4)    ! Transverse 1-turn 4x4 matrix (RF off).
  integer particle         ! +1 = positrons, -1 = electrons
  integer symmetry         ! symmetry of the ring (e/w symm, etc.)
  integer ix_lost          ! If lost at what element?
  integer lattice_type     ! LINAC_lattice$, circular_lattice$, etc.
  integer ixx              ! Integer for general use
  logical stable           ! is closed ring stable?
  logical lost             ! for use in tracking
  logical aperture_limit_on   ! use apertures in tracking?
end type

type mode_info_struct
  real(rp) tune          ! fractional tune in radians: 0 < tune < 2pi
  real(rp) emit          ! Emittance
  real(rp) chrom         ! Chromaticity
end type
\end{verbatim}
\caption{Definition of the \ringstruct\ substructures (except the \elestruct).}
\label{f:sub_ring_struct}
\end{figure}

The array \ele{0:} holds the elements of the lattice. The
array is divided up into two parts: A ``physical'' part and a
``control'' part (Cf.~(ref in lang section). The physical part of this
array holds the elements that are tracked through. The control part
holds the superimpose, overlay and group elements. The extent of these
parts is given in table~\ref{tab:part_extent}. 
The \ele{:} array always has a lower bound of 0. \ele{0} is essentially
a marker element. \ele{0}\%mat6 is always the unit matrix.

\begin{table}[tb]
\begin{center}
\begin{tabular}{|l|l|l|}
\hline
              & \multicolumn{2}{c|} {\em index n}      \\ \hline
{\em section} & {\em min}         & {\em max}          \\ \hline
physical      & 0                 & \nelering          \\ \hline
control       & \nelering+1       & \nelemax           \\ \hline
\end{tabular} 
\caption{Extent of the physical and control parts 
of the array \ele{n}.}
\end{center}
\label{tab:part_extent}
\end{table}

Other indexes within the \ringstruct\ are
\begin{description}
\item[\nelemaxx] The index of the upper bound of the \ele{:} array.
The lower bound is always 0 so the total size of the \ele{:} array is
\nelemaxx+1.
\item[\nelesymm] Back in the days when the CESR ring was E/W 
symmetric \nelesymm\ designated the symmetry point (roughly at \nelering/2). 
Currently this is seldem used.
\item[\neleuse] This designates what part of the physical ring to track 
through. This is the same as \nelering except if there is a symmetry point and \nelesymm\ is non--zero. In this case \neleuse\ = \nelesymm.
\end{description}

Local \ringstruct variables must have the save attribute or the
pointers within must be appropriately deallocated before leaving the
routine.

\section{Pointers}

Since the \ringstruct has pointers within it (plural since the \elestruct\
has pointers within it) there is an extra burden on the programmer to
make sure that allocation and deallocation is done properly. To this
end the equal sign has been overloaded by the routine \sn{ring\_equal\_ring}
so that when one writes
\begin{verbatim}
    ring1 = ring2
\end{verbatim}
the pointers will be handled properly. The result will be that ring1
will hold the same information as ring2 but the pointers in ring1 will
point to different locations in physical memory so that changes to one
ring will not affect the other.

Initial allocation of the pointers in \ringstruct\ is generally
handeled by \bmadparser\ and \sn{ring\_equal\_ring} and so typically not a
concern of most programmers. Deallocation problems can arise when a
subroutine has a local \ringstruct\ variable. 



Generally allocation is handeled

How to allocate \coordstruct\ arrays. see the tracking section
