\chapter{Beam Lines, Replacement Lists, and Superposition}

To \bmad\ a ``lattice'' is the sequence of physical elements that is to be studied.
The lattice is constructed using what are know as Beam Lines and Replacement Lists.
Beam Lines are further subdivided into Lines with and
without replacement arguments. This
esentially corresponds to the \mad\ definition of Lines and Lists. There can be multiple Beam
Lines and Replacement Lists defined in a lattice file and Lines and Lists can be
nested inside other Lines and Lists. The particular Line that defines
the lattice to be analyzed by \bmad\ is selected by the \vn{use} statement. For example
\begin{example}
  use, my_line
\end{example}
would pick the Line \vn{my_line} for analysis. Note that the names of the elements in the
lattice are always converted to upper case.

%-----------------------------------------------------------------------------
\section{Beam Lines without Arguments}
A Beam Line without arguments has the format
\begin{example}
  label: LINE = (member1, member2, ...)
\end{example}
where \vn{member1}, \vn{member2}, etc. are either elements, other Beam Lines or 
Replacement Lists, or sublines enclosed in parentheses.
Example:
\begin{example}
  line1: line = (a, b, c)
  line2: line = (d, line1, e)
  use, line2
\end{example}
This example shows how an Line member can refer to another Beam Line.
This is helpful if the same sequence of elements appears repeatedly in the lattice.
When \vn{line2} is expanded to form the lattice the definition of \vn{line1}
will be inserted 
in to produce the folling lattice for analysis
\begin{example}
  (d, a, b, c, e)
\end{example}

A member that is a Line or List can be reflected (elements taken in reverse order) if
a negative sign is put in front of it. For example:
\begin{example}
  line1: line = (a, b, c)
  line2: line = (d, -line1, e)
\end{example}
\vn{line2} when expanded gives
\begin{example}
  (d, c, b, a, e)
\end{example}
Reflecting a subline will also reflect any sublines of the subline. For example:
\begin{example}
  line0: line = (y, z)
  line1: line = (line0, b, c)
  line2: line = (d, -line1, e)
\end{example}
\vn{line2} when expanded gives
\begin{example}
  (d, c, b, z, y, e)
\end{example}

A repetition count, which is an integer followed by an asterisk, means that the member is
repeated. For example
\begin{example}
  line1: line = (a, b, c)
  line2: line = (d, 2*line1, e)
\end{example}
\vn{line2} when expanded gives
\begin{example}
  (d, a, b, c, a, b, c, e)
\end{example}
Repetition count can be combined with reflection. For example
\begin{example}
  line1: line = (a, b, c)
  line2: line = (d, -2*line1, e)
\end{example}
\vn{line2} when expanded gives
\begin{example}
  (d, c, b, a, c, b, a, e)
\end{example}
Instead of the name of a Line, subline members can also be given as an explicit list using parentheses. For example, the previous example could be rewritten as
\begin{example}
  line2: line = (d, -2*(a, b, c), e)
\end{example}

%-----------------------------------------------------------------------------
\section{Beam Lines with Replaceable Arguments}

Beam lines can have an argument list using the following syntax
\begin{example}
  label: LINE(dummy_arg1, dummy_arg2, ...) = (member1, member2, ...)
\end{example}
The dummy arguments are replaced by the actual arguments when the Line is used
elsewhere. For example:
\begin{example}
  line1(DA1, DA2): line = (a, DA2, b, DA1)
  line2: line = (h, line1(y, z), g)
\end{example}
When \vn{line2} is expanded the actual arguments of \vn{line1}, in this case \vn(y, z),
replaces the dummy arguments \vn{(DA1, DA2)} to give for \vn{line2}
\begin{example}
  (h, a, z, b, y, g)
\end{example} 
Unlike \mad\, Beam Line actual arguments can only be elements or Beam Lines. 
Thus the following is not allowed
\begin{example}
  line2: line = (h, line1(2*y, z), g)   ! NO: 2*y NOT allowed as an argument.
\end{example}

%-----------------------------------------------------------------------------
\section{Replacement Lists}

When a lattice is expanded, all the lattice members that correspond to a name of a Replacement List 
are replaced sucessively, by the members
in the Replacement List. The general syntax is
\begin{example}
  label: LIST = (member1, member2, ...)
\end{example}
For example:
\begin{example}
  list1: list = (a, b, c)
  line1: line = (z1, list1, z2, list1, z3, list1, z4, list1)
  use, line1
\end{example}
When the lattice is expanded the first instance of \vn{list1} in \vn{line1} is replaced by \vn{a} 
(which is the first element of \vn{list1}), the second instance of \vn{list1} is replaced by \vn{b},
etc. If there are more instances of \vn{list1} in the lattice then members of \vn{list1},
the replacement starts at the beginning of \vn{list1} after the last member of \vn{list1} is used.
In this case the lattice would be
\begin{example}
  (z1, a, z2, b, z3, c, z4, a)
\end{example}
Unlike \mad\, members of a replacement list can only be simple elements without reflaction or repetition count and not other Lines or Lists. For example the following is not allowed:
\begin{example}
  list1: list = (2*a, b)  ! NO: No repetition count allowed.
\end{example}

%-----------------------------------------------------------------------------
\section{Superposition}

In practice the field at a particular point in the lattice may be due to more than one physical
element. A common example is a quadrupole inside a solenoid. \bmad\ has a mechanism to handle some
of these cases using what is called ``superposition''. A simple example shows how this works
\begin{example}
  q: quad, l = 10
  d: drift, l = 4
  s: solenoid, l = 6, superimpose, ref = q, ref_end
  lat: line = (q, d)
  use, lat
\end{example}
The \vn{superimpose} attribute of \vn{s} superimposes the \vn{s} solenoid over the 
lattice \vn{(q, d)}. The placement of \vn{s} is such that the center of \vn{s} coinsides
with the end of \vn{q} (more on how superimposed elments get placed later). Now if one looks
at a list of lattice elements one would find:
\begin{example}
        Element   Key         Length
  1)    q\\        Quadrupole  7
  2)    q\\s       Sol_quad    3
  3)    s\\1       Solenoid    3
  4)    d\\        Drift       1
\end{example}
What \bmad\ has done is to split the original elements \vn{(q, d)} at the edges of \vn{s}.
The first element \vn{q\\} is the part of \vn{q} that is outside of \vn{s}. Since this is only
part of \vn{q} \bmad\ has put a \vn{\\} in the name so that there will be no confusion (\vn{\\}
has no special meaning other than the fact that \bmad\ uses it for mangleing names). The next
element, \vn{q\\s}, is the part of \vn{q} that is inside \vn{s}. \vn{q\\s} is a combination
solenoid/quadrupole element as one could expect. \vn{s\\1} is the part of \vn{s}
that is outside \vn{q} so this element is just a solenoid. Finally, \vn{d\\} is the rest of the 
drift outside \vn{s}. 

With the lattice broken up like this \bmad\ has constructed something that can be
easily analyzed. However, the original elements \vn{q} and \vn{s} still exist within the lattice
because of the following: The lattice is actually broken up into two parts. The first part, 
are the elements that \bmad\ uses for any analysis (tracking, Twiss parameter calculations, etc.).
In the example this first part is the 4 elements listed above. The second part is all those elements
who have been cut up via superposition (along with group and overlay elements 
discussed in chapter \ref{c:groups_and_overlays}). What is the purpose of this second list? \bmad\
keeps a record of how superpositions were formed. Thus from a programming standpoint \bmad\ makes
it simple if one, say, wants to change the \vn{k1} strength of \vn{q} to make the appropriate changes
to the \vn{k1} strength of \vn{q\\} and \vn{q\\s}.

It does not matter which element is superimposed. Thus, in the above example, \vn{s} could have 
been put in the the Beam Line (with a drift before it) and \vn{q} could then have been superimposed on 
top and the result would have been the same (except that the split elements could have a different
name). However,
superpositions are restriced in that \bmad\ may not have an appropriate element for the
superimposed fields. For example, a solenoid can not be superimposed over an octupole. 
If a zero length element, such as a marker, is superimposed with some other element (or vice versa)
the element is just split in two. For example:
\begin{example}
  q: quad, l = 10
  m: marker, superimpose, offset = 6
  lat: line = (q)
  use, lat
\end{example}
The resulting lattice (first part) would be
\begin{example}
        Element   Key           Length
  1)    q\\1       Quadrupole    6
  2)    m         Marker        0
  3)    q\\2       Quadrupole    4
\end{example}
and the second part of the lattice would have the \vn{q} element.
 
The placement of a superimposed element is determined by three factors: A point on the 
superimposed element, a reference point in the lattice line, and an offset between the
points. The attributes that determine these three quantities are:
\begin{example}
  ref = <element name in lattice>
  offset = <length>      (default = 0)
  ref_beginning
  ref_center             (default)
  ref_end
  ele_beginning
  ele_center             (default)
  ele_end
\end{example}
\vn{ele_beginning} \vn{ele_center} \vn{ele_end} makes the point on the superimposed element 
the beginning (``left'') edge, the center, or the end (``right'') edge respectively.
If neither of these attributes are given the default is to use the element center.


