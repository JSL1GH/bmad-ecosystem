\chapter{Beam Lines and Lists}

An input lattice to \bmad\ is a sequence of physical elements which
will be called a beam line. Beam lines are defined by using two
constructs called lines and lists. This is very similar to \mad lines
and lists. There can be mulitiple beam lines and lists defined in a
lattice file and lines and lists can be nested inside other beam
lines. The particular beam line that defines the lattice to be used by
\bmad\ is selected by the \vn{use} statement. For example
\begin{example}
  use, my_line
\end{example}
would pick the line \vn{my_line}.

\section{Beam Lines}
A beam line without arguments (beam lines with arguments will be
discussed below) has the format
\begin{example}
  label: LINE = (member1, member2, ...)
\end{example}
where \vn{member1}, etc. are either elements, or other beam lines or lists.
Example:
\begin{example}
  this: line = (a, b, c, a)
  use, this
\end{example}
The \vn{use} command defines the lattice to be the sequence \vn{a, b, c, a}.

Lines can be nested within lines:
\begin{example}
  a: line = (a1, a2)
  b: line = (b1, a)
  c: line = (b, c1, a)
  use, c
\end{example}
This results in the lattice being
\begin{example}
  c = (b1, a1, a2, c1, a1, a2)
\end{example}

For lines
